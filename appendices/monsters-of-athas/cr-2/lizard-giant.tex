\subsubsection{Lizard, Giant}
\begin{MonsterStats}
{Large Animal}
{3d8+15 (28 hp)}
{+7}
{12 m (8 squares), burrow 6 m}
{16 (-1 size, +3 Dex, +4 natural), touch 12, flat-footed 13}
{+2/+10}
{Bite +5 melee (1d8+6)}
{Bite +5 melee (1d8+6)}
{3 m/1.5 m}
{Improved grab, swallow whole}
{Low-light vision}
{Fort +8, Ref +6, Will +1}
{Str 18, Dex 17, Con 20, Int 1, Wis 10, Cha 9}
{\skill{Climb} +5, \skill{Hide} +4*, \skill{Listen} +4, \skill{Move Silently} +4, \skill{Spot} +3}
{\feat{Alertness}, \feat{Improved Initiative}}
{Deserts}
{Solitary, or herd (2--12)}
{2}
{None}
{Always neutral}
{4--6 HD (Large); 7--9 HD (Huge)}
{---}
\end{MonsterStats}

\textbf{Improved Grab (Ex):} If the giant lizard hits with its bite it can initiate a grapple check as a free action without provoking an attack of opportunity. If it wins the grapple check it establishes a hold and can attempt to swallow whole the next round.

\textbf{Swallow Whole (Ex):} The giant lizard can try to swallow a grabbed opponent of Medium or smaller size by making a successful grapple check. Once inside, the opponent takes 1d6+4 points of crushing damage plus 1d8+4 points of acid damage per round from the giant lizard's digestive juices. A swallowed creature can cut its way out by dealing 15 points of damage to the giant lizard's digestive tract (AC 12). Once the creature exits, muscular action closes the hole; another swallowed opponent must cut its own way out. The giant lizard's gullet can hold 2 Medium, 4 Small, or 8 Tiny or smaller creatures.

\textbf{Skills:} *Giant lizards gain a +8 bonus to \skill{Hide} checks when attempting to hide in the sand.
