\subsubsection{Kluzd}
\begin{MonsterStats}
{Large Animal}
{4d8+4 (22 hp)}
{+7}
{9 m (6 squares), burrow 9 m}
{15 (-1 size, +3 Dex, +3 natural), touch 12, flat-footed 12}
{+3/+16*}
{Bite +5 melee (1d8+1)}
{Bite +5 melee (1d8+1)}
{3 m/1.5 m}
{Constrict 1d8+1, improved grab, swallow whole}
{Low-light vision}
{Fort +5, Ref +7, Will +2}
{Str 12, Dex 17, Con 12, Int 1, Wis 12, Cha 2}
{\skill{Hide} +4, \skill{Intimidate} +4, \skill{Spot} +3}
{\feat{Improved Initiative}, \feat{Weapon Finesse}}
{Silt}
{Solitary or nest (3--10)}
{2}
{None}
{Always neutral}
{5--9 HD (Large), 10--14 HD (Huge)}
{---}
\end{MonsterStats}

\textbf{Constrict (Ex):} On a successful grapple check, a kluzd deals 1d8+1 points of damage.

\textbf{Improved Grab (Ex):} To use this ability, a kluzd must hit with its bite attack. It can then attempt to start a grapple as a free action without provoking an attack of opportunity. If it wins the grapple check, it establishes a hold and can attempt to swallow the foe the following round. *A kluzd has a +8 racial bonus on grapple checks.

\textbf{Swallow Whole (Ex):} A kluzd can try to swallow a grabbed opponent of a smaller size than itself by making a successful grapple check.

Once inside, the opponent takes 1d8+1 points of crushing damage plus 4 points of acid damage per round from the kluzd's stomach. A swallowed creature can cut its way out by using a light slashing or piercing weapon to deal 10 points of damage to the stomach (AC 11).

Once the creature exits, muscular action closes the hole; another swallowed opponent must cut its own way out. A Large kluzd's interior can hold 1 Medium, 2 Small, 8 Tiny, or 16 Diminutive or smaller opponents.

\textbf{Skills:} Kluzds have a +8 racial bonus on \skill{Intimidate} checks.
