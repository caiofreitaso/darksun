\subsubsection{Cha'thrang}
\begin{MonsterStats}
{Medium Animal}
{6d8+30 (57 hp)}
{+1}
{12 m (8 squares)}
{21 (+1 Dex, +10 natural), touch 11, flat-footed 20}
{+4/+13}
{Bite +9 melee (1d8+5) or tethered spike +5 ranged (1d6+2 plus poison)}
{Bite +9 melee (1d8+5) and 2 claws +4 melee (1d6+2); or tethered spike +5 ranged (1d6+2 plus poison)}
{1.5 m/1.5 m}
{Drag, poison, tethered spikes}
{Low-light vision}
{Fort +10, Ref +6, Will +4}
{Str 21, Dex 13, Con 20, Int 2, Wis 14, Cha 7}
{\skill{Hide} +6, \skill{Listen} +4, \skill{Spot} +4}
{\feat{Alertness}, \feat{Endurance}}
{Rocky badlands, sandy wastes, and stony barrens}
{Trine (3)}
{3}
{None}
{Always neutral}
{7--12 HD (Large)}
{---}
\end{MonsterStats}

\MonsterStatsDescription{
    Cha'thrangs are large, shelled predators, similar to giant tortoises, that hunt flying creatures. Due to the numerous reed shaped protrusions on their shell and their dun coloration, cha'thrangs are often mistaken for patches of dead plant growth. The protrusions on the cha'thrang's back are actually hollow appendages that allow the creature to shoot long barbed tethered darts at any creature passing overhead within range. The creature's shell is created by an alkaline lime secreted from its back that further holds the shell in place. This same lime also creates a thin, sinewy fiber that tethers the creatures
    darts and coats them in an alkaloid toxin.

    Cha'thrangs travel in groups of three called ``trines'', usually composed of two females and one male, but will adopt other cha'thrang that they meet, later breaking off into further trines. The creatures have problems mating because of their shell structures and often die in the process. Females lay annual clutches of 1--6 eggs, most of which are devoured by predators. Adult cha'thrang themselves can live for hundreds of years but often succumb to predators before this time. Cha'thrang meat can be eaten if special preparations are taken to remove the lime under the shell. Its tethers can also be braided together to form rope.
}

A cha'thrang preys almost exclusively on flying creatures, lying motionless for hours on end until a suitable target passes overhead. The cha'thrang then expels tethered darts from the hollow tubes studding its shell with a sudden burst of air. While many of these darts are fired at once, only one has a chance of hitting the target. Once it has hit a target, the cha'thrang digs in and attempts to control the tether. When the flying creature tires and lands, the cha'thrang retracts the tether as it crawls towards its prey, where it uses its melee attacks to finish the kill. Creatures hit by the darts are also subjected to the cha'thrang's lime toxin.

\textbf{Drag (Ex):} If a cha'thrang hits with a tethered spike attack, the strand latches onto the opponent's body unless the opponent succeeds at a Reflex save (DC equals 10 + damage dealt). The cha'thrang drags the stuck opponent 3 meters closer each subsequent round (provoking no attack of opportunity) unless that creature breaks free, which requires a DC 22 \skill{Escape Artist} check or a DC 18 Strength check. The check DCs are Strength-based, and the \skill{Escape Artist} DC includes a +4 racial bonus. A cha'thrang can draw in a creature within 3 meters of itself and attack in the same round. A strand has hardness 5 and 10 hit points, and can be attacked by making a successful sunder attempt. However, attacking a cha'thrang's strand does not provoke an attack of opportunity. If the strand is currently attached to a target, the cha'thrang takes a $-4$ penalty on its opposed attack roll to resist the sunder attempt. Severing a strand deals no damage to a cha'thrang.

\textbf{Poison (Ex):} Injury, Fortitude DC 18, initial damage 1 Str, secondary damage 2d6 Str. The save DC is Constitution-based.

\textbf{Tethered Spikes (Ex):} A cha'thrang can fire tethered spikes up to four times per day. It can fire spikes only at flying creatures up to 150 feet away (no range increment).

\textbf{Skills:} *Cha'thrang receive a +8 bonus on \skill{Hide} checks in natural surroundings.
