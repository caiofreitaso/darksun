\subsubsection{Lirr}
\begin{MonsterStats}
{Medium Animal}
{5d8+10 (32 hp)}
{+1}
{12 m (8 squares)}
{14 (+1 Dex, +3 natural), touch 11, flat-footed 13}
{+3/+4}
{Claw +4 melee (1d4+1)}
{2 claws +4 melee (1d4+1) and bite $-1$ melee (1d10)}
{1.5 m/1.5 m}
{Improved grab, rake 1d4, stun}
{Low-light vision}
{Fort +6, Ref +5, Will +2}
{Str 13, Dex 12, Con 14, Int 1, Wis 12, Cha 11}
{\skill{Hide} +6, \skill{Jump} +9, \skill{Listen} +3, \skill{Spot} +6}
{\feat{Alertness}, \feat{Combat Reflexes}}
{Deserts}
{Pack (2--12)}
{3}
{None}
{Always neutral}
{6--10 HD (Medium)}
{---}
\end{MonsterStats}

\MonsterStatsDescription{
    Lirrs are reptilian predators that hunt in packs. They are fast and fierce and possess a powerful roar that has the ability to stun their prey. The colored membrane around the lirr's neck can be inflated and flushed with blood to communicate with others of the species in a number of ways.

    Lirr packs are exceptionally quarrelsome, and particularly intelligent quarry might be able to escape one pack by leading pursuers into the lair of another. Lirrs pair off to mate, but such pairings are loose and will separate if resources becomes scarce. A female lirr produces 2--8 eggs every two years, which hatch in three months, with the young maturing in nine months. Though only the birth mother is concerned for her eggs, any female will protect the pack's young once they have hatched.

    Some lirr packs seem to prefer the rockier terrain of the mountain ranges, finding comfort in the cooler cave temperatures. Mountain lirrs are identical to their desert dwelling cousins, save for the fact that they lack the lirr's characteristically bright colors on their ringed membrane.

    A lirr typically weighs 150 kilograms and is 1.8 meter long from tip to tail.
}

Lirrs usually start combat by attempting to stun their opponents, then grabbing and raking them with their powerful claws.

\textbf{Improved Grab (Ex):} To use this ability, a lirr must hit with its bite attack. It can then attempt to start a grapple as a free action without provoking an attack of opportunity. If it wins the grapple check, it establishes a hold and can rake.

\textbf{Rake (Ex):} Attack bonus +4 melee, damage 1d4.

\textbf{Stun (Ex):} As a standard action, a lirr can emit a powerful roar capable of stunning creatures within a 12-meter cone. Creatures in the cone must make a Fortitude save (DC 14) or be stunned for 1d4 rounds. The save DC is Constitution-based.

\textbf{Skills:} The lirr has a +4 racial bonus to \skill{Jump} checks.
