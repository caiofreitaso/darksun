\subsubsection{Inix}
\begin{MonsterStats}
{Large Animal}
{6d8+18 (45 hp)}
{+2}
{12 m (8 squares)}
{17 ($-1$ size, +2 Dex, +6 natural), touch 11, flat-footed 15}
{+4/+12}
{Bite +7 melee (1d8+6) or tail slap +7 melee (1d4+6)}
{Bite +7 melee (1d8+6) or tail slap +7 melee (1d4+6)}
{3 m/1.5 m (3 m reach with tail)}
{Improved grab, swallow whole}
{Improved carrying capacity, low-light vision}
{Fort +8, Ref +7, Will +3}
{Str 19, Dex 15, Con 16, Int 2, Wis 12, Cha 6}
{\skill{Listen} +8, \skill{Spot} +7}
{\feat{Alertness}, \feat{Combat Reflexes}, \feat{Dodge}}
{Deserts}
{Solitary or pair}
{3}
{None}
{Always neutral}
{7--12 HD (Huge)}
{---}
\end{MonsterStats}

\textbf{Improved Grab (Ex):} To use this ability, an inix must hit with its bite attack. It can then attempt to start a grapple as a free action without provoking an attack of opportunity. If it wins the grapple check, it establishes a hold and can attempt to swallow the foe the following round.

\textbf{Swallow Whole (Ex):} An inix can try to swallow a grabbed opponent two or more sizes smaller than itself by making a successful grapple check. Once inside, the opponent takes 1d8+6 points of crushing damage plus 4 points of acid damage per round from the inix's stomach. A swallowed creature can cut its way out by using a light slashing or piercing weapon to deal 15 points of damage to the stomach (AC 13). Once the creature exits, muscular action closes the hole; another swallowed opponent must cut its own way out. A Large inix's interior can hold 1 Small, 2 Tiny, 4 Diminutive, or 8 Fine opponents.

\textbf{Improved Carrying Capacity:} An inix's carrying capacity is double the normal for a creature of its Strength. A light load for an inix is up to 348 kilograms; a medium load is up to 702 kilograms; a heavy load is up to 1,050 kilograms. An inix can drag 5,250 kilograms.
