\subsubsection{Baazrag}
\begin{MonsterStats}
{Small Animal}
{1d8 (4 hp)}
{+2}
{12 m (8 squares)}
{16 (+1 size, +2 Dex, +3 natural), touch 13, flat-footed 14}
{+0/+2*}
{Bite +3 melee (1d4--2 plus poison*)}
{Bite +3 melee (1d4--2 plus poison*)}
{1.5 m/1.5 m}
{Improved grab, poison}
{Low-light vision, scent}
{Fort +2, Ref +4, Will +1}
{Str 6, Dex 14, Con 10, Int 1, Wis 12, Cha 4}
{\skill{Hide} +5, \skill{Listen} +3, \skill{Spot} +6}
{\feat{Alertness}, \feat{Weapon Finesse}\textsuperscript{B}}
{Stony Barrens}
{Pack (4--40)}
{\onehalf}
{None}
{Always neutral}
{2--3 HD (Small); 4--5 HD (Medium)}
{---}
\end{MonsterStats}

\MonsterStatsDescription{
    This timid omnivore dwells in the rocky wastes of the Tablelands, scurrying here and there in search of food, always hoping to avoid the attention of a larger predator. The Athasian nobility also domesticates them as vermin hunters or beasts of burden. The baazrag has a poisonous bite that makes it a valuable and economical alternative to larger, more fearsome guardians.

    Baazrag flesh is also edible and up to 4 pints of water can be harvested from the creature's water-sac (located beneath its back armoring). The water is contaminated with the same poison present in the baazrag's bite and must be purified or neutralized before it can be safely ingested.

    Baazrag females usually bear litters of 2d3 young but occasionally a solitary offspring is born. This solitary child is invariably a greater baazrag, a voracious psionic mutation that will attempt to devour all that moves. In the wild, baazrag packs will move away once a greater baazrag has been born.

    Baazrags average 60 centimeters in length and weigh 12.5 kilograms. The young are red-brown, green, yellow or orange, but these colors uniformly fade to a sandy-grey with age.
}

The baazrag is extremely timid and will flee to avoid combat, only attacking if cornered. When the baazrag does bite, however, it will attempt to grapple. If successful, the baazrag will gnaw at the wound, its poisonous saliva flooding the target with toxins.

If initial attacks of this kind do not deter opponents, the baazrag overcomes its natural timidity and is possessed by a swarming instinct. This instinct unites the entire pack, which subsequently attacks all foes, only fleeing when 80\% or more of the baazrags have been slain.

\textbf{Improved Grab (Ex):} If a baazrag hits with its bite it may initiate a grapple check as a free action without provoking an attack of opportunity. If it succeeds at the grapple check it may gnaw at the wound, injecting its poison. *Baazrag receive a +8 racial bonus on grapple checks.

\textbf{Poison (Ex):} Injury, Fortitude DC 10, initial damage 1d6 Con, secondary damage 1d6 Con. The save DC is Constitution-based. *Note that a baazrag can only use its poison if it successfully grapples its target.
