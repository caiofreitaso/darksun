\subsubsection{Jankx}
\begin{MonsterStats}
{Tiny Animal}
{1d8+1 (5 hp)}
{+8}
{6 m (4 squares), burrow 9 m}
{16 (+2 size, +4 Dex), touch 16, flat-footed 12}
{+0/$-10$}
{Claw +6 melee (1d2$-2$ plus poison)}
{2 claws +6 melee (1d2$-2$ plus poison)}
{0.75 m/0 m}
{Poison}
{Low-light vision}
{Fort +3, Ref +6, Will +1}
{Str 6, Dex 19, Con 13, Int 1, Wis 12, Cha 4}
{\skill{Hide} +22, \skill{Listen} +8, \skill{Move Silently} +16}
{\feat{Improved Initiative}, \feat{Weapon Finesse}\textsuperscript{B}}
{Sandy wastes and stony barrens}
{Community (1-1000)}
{\onehalf}
{None}
{Always neutral}
{2 HD (Tiny); 3 HD (Small)}
{---}
\end{MonsterStats}

\MonsterStatsDescription{
    These furry mammals live in burrows in the desert. The common people of Athas think these prized creatures are too dangerous to bother with, but those that are killed make good furs and are a good source of food.

    Jankx communicate via ultrasonic squeaks and barks which are inaudible to most humanoid ears.
}

Jankx are not very combative, but they do have a poison that serves as a defense mechanism. It has a withering effect upon flesh, inflicting tremendous pain for such a small creature, and is quite capable of crippling a grown man in moments. Jankx have spurs and poison sacs located on the underside of each limb near the paw.

\textbf{Poison (Ex):} Injury, Fortitude DC 11, initial damage 1d6 Str, secondary damage 2d6 Dex. The save DC is Constitution-based.

\textbf{Skills:} Jankx receive a +10 racial bonus to \skill{Hide} and \skill{Move Silently} checks, and a +5 racial bonus to \skill{Listen} checks.
