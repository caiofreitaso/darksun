\subsubsection{Rasclinn}
\begin{MonsterStats}
{Small Animal}
{1d8 (4 hp)}
{+3}
{60 ft. (12 squares)}
{18 (+1 size, +3 Dex, +4 natural), touch 14, flat-footed 15}
{+0/-6}
{Bite +4 melee (1d4-2)}
{Bite +4 melee (1d4-2) and 2 claws $-1$ melee (1d4-2)}
{5 ft./5 ft.}
{Rage}
{Low-light vision, immunity to poison}
{Fort +2, Ref +5, Will +1}
{Str 6, Dex 17, Con 11, Int 1, Wis 12, Cha 14}
{\skill{Hide} +15, \skill{Listen} +11, \skill{Spot} +7, \skill{Survival} +9}
{\feat{Alertness}, \feat{Weapon Finesse}\textsuperscript{B}}
{Rocky badlands}
{Gang (1--12)}
{\onehalf}
{None}
{Always neutral}
{2 HD (Small); 3 HD (Medium)}
{---}
\end{MonsterStats}

\MonsterStatsDescription{
    Rasclinn are small dog-like creatures that feed on any vegetation by extracting the trace metals from the plants, which gives them a somewhat metallic hide. They are hunted due to this hide, but are very tough to kill.

    Rasclinn are small, standing about 1 meter at the shoulder, and weighing only 25 kilograms. They have a silver tint to their hide. They have no language of their own, but instead communicate by barks and yelps. These yelps and barks can mean a multitude of things.

    Very few creatures hunt rasclinn since their metallic hide makes them unpalatable.
}

Rasclinn attack with their bite when cornered or defending young. Rasclinn usually avoid combat at all costs and attempt to hide in patches of spider and sand cacti when available.

\textbf{Rage (Ex):} A rasclinn that takes damage in combat flies into a berserk rage on its next turn, clawing and biting madly until either it or its opponent is dead. It gains +4 to Strength, +4 to Constitution, and $-2$ to Armor Class. The creature cannot end its rage voluntarily.

\textbf{Skills:} Rasclinn receive a +8 racial bonus to all \skill{Hide}, \skill{Listen} and \skill{Survival} checks.
