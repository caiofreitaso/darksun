\subsubsection{Erdland}
\begin{MonsterStats}
{Large Animal}
{3d8+9 (22 hp)}
{+5}
{9 m (6 squares)}
{13 (-1 size, +1 Dex, +3 natural), touch 10, flat-footed 12}
{+2/+10}
{Bite +5 melee (1d8+6)}
{Bite +5 melee (1d8+6)}
{3 m/3 m}
{---}
{Low-light vision}
{Fort +6, Ref +4, Will +2}
{Str 19, Dex 13, Con 16, Int 1, Wis 13, Cha 5}
{\skill{Jump} +6, \skill{Listen} +4, \skill{Spot} +4, \skill{Survival} +3}
{\feat{Alertness}, \feat{Improved Initiative}}
{Plains}
{Herd (10--30)}
{1}
{None}
{Always neutral}
{4--6 HD (Large)}
{---}
\end{MonsterStats}

\MonsterStatsDescription{
%    Erdlands are a large variant of erdlus. They are generally used as mounts or to pull caravans.

    Erdlands are flightless, featherless birds that are covered with scales. They weigh around 2 tonnes and can stand up to 4 meters tall. Erdlands are used more for their endurance than speed, since they are not capable of fast speeds.

    Erdlands don't provide much in usable material. They do provide the savage halflings that inhabit the jungles of Athas with a major meat source. Erdlands can provide up to 350 kilograms of meat.

    Erdlands live in low-lying vegetation areas, and are omnivorous, eating both animals and vegetables. Erdlands eat esperweed as a delicacy, hence why some may be psionic, although if an erdland does eat some esperweed, it only has its psionic powers for about 10 minutes.

    Erdlands lay eggs, which are about 1 meter in diameter. Their eggs are less tasty than their smaller cousins, but can provide a meal for up to three Medium creatures. Erdlands will attack viciously to protect their young, which they lay in egg wells, small holes dug underground.
}

When attacked, erdland attack with their beak.