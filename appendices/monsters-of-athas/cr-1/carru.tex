\subsubsection{Carru}
\begin{MonsterStats}
{Large Animal}
{3d8+9 (22 hp)}
{+0}
{12 m (8 squares)}
{12 ($-1$ size, +3 natural), touch 9, flat-footed 12}
{+2/+12}
{Slam +7 melee (1d6+6)}
{Slam +7 melee (1d6+6) and gore +2 melee (1d8+3)}
{3 m/1.5 m}
{Trample 1d6+9}
{Low-light vision, scent}
{Fort +6, Ref +3, Will +2}
{Str 22, Dex 10, Con 17, Int 2, Wis 12, Cha 3}
{\skill{Spot} +4, \skill{Survival} +4}
{\feat{Alertness}, \feat{Endurance}}
{Any (Tablelands)}
{Domesticated or herd (5--50 plus 1--5 bull carrus)}
{1}
{None}
{Always neutral}
{4--5 HD (Large)}
{---}
\end{MonsterStats}

\MonsterStatsDescription{
    With many products and resources being derived from it, the carru is a staple of Athasian life. The females produce a nourishing, creamy milk (as much as three gallons per day), and both sexes are slaughtered for their meat. A male can yield as much as 125 kg of meat. This is less for females (only 100 kg), but they are seldom killed for their meat, as they are potentially much more valuable as milk producers.

    Carru make excellent beasts of burden, dragging plows and turning water and grain mills. Carru hide is soft and supple and holds a dye well, so it is used in the making of clothing, furniture, tents and the like. The thicker hide of the skull is a component in many leather and hide armors and is also highly suited as a shield covering.

    The carru's hump, though used for water storage, does not inflate or deflate like a camel's. The average hump holds 1d6+2 pints of water at any given time and can be used to make a waterskin of similar capacity. The leather will rot in contact with alcohol, however, so carru hide is unsuitable for the fashioning of wineskins.

    The adult carru is 3 meters long and weighs as much as 200 kilograms. Their soft hide is furred and colored in varying shades of dun grey or brown.
}

Carru are not aggressive creatures on the whole, although the males can be quite hostile when the herd is threatened. Their standard tactic is to charge, then gore an opponent. If they are able to, male carru will use their horns to grapple and then toss targets to the ground, where they are trampled by other carru. So often used is this tactic that most carru herders agree that it is an instinct of the species.

\textbf{Trample (Ex):} Reflex half DC 17. The save DC is Strength-based.
