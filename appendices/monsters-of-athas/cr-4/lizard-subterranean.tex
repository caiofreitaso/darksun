\subsubsection{Lizard, Subterranean}
\begin{MonsterStats}
{Large Animal}
{6d8+36 (63 hp)}
{+8}
{9 m (6 squares), climb 9 m}
{17 ($-1$ size, +4 Dex, +4 natural), touch 13, flat-footed 13}
{+4/+13}
{Bite +8 melee (1d8+7)}
{Bite +8 melee (1d8+7)}
{3 m/1.5 m}
{Improved grab, swallow whole}
{Low-light vision}
{Fort +11, Ref +11, Will +3}
{Str 20, Dex 18, Con 22, Int 2, Wis 12, Cha 11}
{\skill{Climb} +15, \skill{Hide} +9, \skill{Listen} +6, \skill{Move Silently} +13, \skill{Spot} +5}
{\feat{Alertness}, \feat{Improved Initiative}, \feat{Lightning Reflexes}}
{Underground}
{Solitary}
{4}
{None}
{Always neutral}
{7--12 HD (Large); 13--18 HD (Huge)}
{---}
\end{MonsterStats}

\textbf{Improved Grab (Ex):} If the subterranean lizard hits with its bite it can initiate a grapple check as a free action without provoking an attack of opportunity. If it wins the grapple check it establishes a hold and can attempt to swallow whole the next round.

\textbf{Swallow Whole (Ex):} The subterranean lizard can try to swallow a grabbed opponent of Medium or smaller size by making a successful grapple check. Once inside, the opponent takes 1d6+5 points of crushing damage plus 1d8+5 points of acid damage per round from the subterranean lizard's digestive juices. A swallowed creature can cut its way out by dealing 10 points of damage to the subterranean lizard's digestive tract (AC 12). Once the creature exits, muscular action closes the hole; another swallowed opponent must cut its own way out. The subterranean lizard's gullet can hold 2 Medium, 4 Small, or 8 Tiny or smaller creatures.

\textbf{Skills:} Subterranean lizards have a +8 racial bonus on \skill{Climb}, \skill{Hide}, and \skill{Move Silently} checks. They can always choose to take 10 on Climb checks, even if rushed or threatened.
