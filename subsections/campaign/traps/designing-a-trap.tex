\subsection{Designing A Trap}
\textbf{Mechanical Traps:} Simply select the elements you want the trap to have and add up the adjustments to the trap's Challenge Rating that those elements require (see \tabref{CR Modifiers for Mechanical Traps}) to arrive at the trap's final CR. From the CR you can derive the DC of the \skill{Craft} (trapmaking) checks a character must make to construct the trap.


\textbf{Magic Traps:} As with mechanical traps, you don't have to do anything other than decide what elements you want and then determine the CR of the resulting trap (see \tabref{CR Modifiers for Magic Traps}). If a player character wants to design and construct a magic trap, he must have the \feat{Craft Wondrous Item} feat. In addition, he must be able to cast the spell or spells that the trap requires---or, failing that, he must be able to hire an NPC to cast the spells for him.

\Table{CR Modifiers for Mechanical Traps}{Xl}{
  \tableheader Feature
& \tableheader CR Modifier\\
\cmidrule[0pt]{1-2}
\multicolumn{2}{l}{\TableSubheader{\skill{Search} DC}}\\
15 or lower            & $-1$ \\
25--29                 & +1 \\
30 or higher           & +2 \\

\multicolumn{2}{l}{\TableSubheader{\skill{Disable Device} DC}}\\
15 or lower            & $-1$ \\
25--29                 & +1 \\
30 or higher           & +2 \\

\multicolumn{2}{l}{\TableSubheader{Reflex Save DC (Pit or Other Save-Dependent Trap)}}\\
15 or lower            & $-1$ \\
16--24                 & +0 \\
25--29                 & +1 \\
30 or higher           & +2 \\

\cmidrule[0pt]{1-2}
\multicolumn{2}{l}{\TableSubheader{Attack Bonus (Melee or Ranged Attack Trap)}}\\
+0 or lower            & -2 \\
+1 to +5               & $-1$ \\
+6 to +14              & +0 \\
+15 to +19             & +1 \\
+20 to +24             & +2 \\
+25 to +34             & +3 \\
+35 to +45             & +4 \\
+50 or higher          & +5 \\

\cmidrule[0pt]{1-2}
\multicolumn{2}{l}{\TableSubheader{Damage/Effect}}\\
Average damage         & +1/7 points\footnotemark[1] \\

% \cmidrule[0pt]{1-2}
\multicolumn{2}{l}{\TableSubheader{Miscellaneous Features}}\\
Alchemical device      & Level of spell mimicked \\
Liquid                 & +5 \\
Multiple target        & +1 (or 0 if never miss) \\
Multiple melee attack  & +2 \\
Onset delay 1 round    & +3 \\
Onset delay 2 rounds   & +2 \\
Onset delay 3 rounds   & +1 \\
Onset delay 4+ rounds  & $-1$ \\
Poison                 & See \tabref{CR Modifiers for Poisons} \\
Pit spikes             & +1 \\
Touch attack           & +1 \\
\TableNote{2}{1 Rounded to the nearest multiple of 7 (round up for an average that lies exactly between two numbers).}
}

\Table{CR Modifiers for Poisons}{XlXl}{
  \tableheader Poison
& \tableheader CR
& \tableheader Poison
& \tableheader CR\\

Antloid, soldier (contact)     & +3 & Mastyrial, black               & +4 \\
Antloid, soldier (injury)      & +3 & Mastyrial, desert              & +4 \\
Assassin bug                   & +1 & Medium spider venom            & +2 \\
Beetle, dragon\footnotemark[1] & +4 & Mulworm (contact)              & +2 \\
Black adder venom              & +2 & Mulworm (injury)               & +2 \\
Black lotus extract            & +8 & Nitharit                       & +3 \\
Bloodgrass (jungle)            & +3 & Pulp bee                       & +2 \\
Bloodgrass (plains)            & +1 & Purple worm poison             & +6 \\
Bloodroot                      & +1 & Random displays                & +2 \\
Blossomkiller                  & +2 & S'thag zagath                  & +3 \\
Blue whinnis                   & +1 & Sassone leaf residue           & +2 \\
Boneclaw, lesser               & +2 & Scarlet warden                 & +4 \\
Brain seed powder              & +3 & Scorpion, barbed               & +3 \\
Burnt othur fumes              & +5 & Scorpion, gold                 & +2 \\
Cha'thrang                     & +4 & Shadow essence                 & +4 \\
Cistern fiend                  & +6 & Silk wyrm                      & +2 \\
Deathblade                     & +5 & Silt serpent                   & +2 \\
Dragon bile                    & +6 & Silt serpent, giant            & +3 \\
Drik, high                     & +4 & Silt serpent, giant (immature) & +4 \\
Dune freak                     & +2 & Single-mindedness              & +2 \\
Dust glider                    & +2 & Small centipede poison         & +1 \\
Fordorran                      & +6 & Spider, crystal                & +3 \\
Gaj poison gas                 & +2 & Spider, dark defiler           & +1 \\
Giant wasp poison              & +4 & Spider, dark psion             & +2 \\
Greenblood oil                 & +1 & Spider, dark queen             & +5 \\
Hej-kin                        & +1 & Spider, dark warrior           & +2 \\
Insanity mist                  & +3 & T'chowb ichor                  & +2 \\
Jankx                          & +3 & Terinav root                   & +4 \\
Kank, soldier                  & +2 & Ungol dust                     & +2 \\
Large spider venom             & +4 & Wyvern poison                  & +5 \\
Malyss root paste              & +3 & Zik-trin'ta                    & +3 \\

\TableNote{4}{1 Only affects dray and creatures with the Dragon type.}\\
}


\Table{CR Modifiers for Magic Traps}{lX}{
  \tableheader Feature
& \tableheader CR Modifier\\
Highest-level spell & + Spell level OR\\
\cmidrule[0pt]{1-2}
                    & +1 per 7 points of average damage per round\footnotemark[1]\\
\TableNote{2}{1 Rounded to the nearest multiple of 7 (round up for an average that lies exactly between two numbers).}
}

\subsubsection{Challenge Rating Of A Trap}
To calculate the Challenge Rating of a trap, add all the CR modifiers (see the tables below) to the base CR for the trap type.

\textbf{Mechanical Trap:} The base CR for a mechanical trap is 0. If your final CR is 0 or lower, add features until you get a CR of 1 or higher.

\textbf{Magic Trap:} For a spell trap or magic device trap, the base CR is 1. The highest-level spell used modifies the CR (see Table: CR Modifiers for Magic Traps).

\textbf{Average Damage:} If a trap (either mechanical or magic) does hit point damage, calculate the average damage for a successful hit and round that value to the nearest multiple of 7. Use this value to adjust the Challenge Rating of the trap, as indicated on the tables below. Damage from poisons and pit spikes does not count toward this value, but damage from a high strength rating and extra damage from multiple attacks does.

For a magic trap, only one modifier applies to the CR---either the level of the highest-level spell used in the trap, or the average damage figure, whichever is larger.

\textbf{Multiple Traps:} If a trap is really two or more connected traps that affect approximately the same area, determine the CR of each one separately.

\textit{Multiple Dependent Traps:} If one trap depends on the success of the other (that is, you can avoid the second trap altogether by not falling victim to the first), they must be treated as separate traps.

\textit{Multiple Independent Traps:} If two or more traps act independently (that is, none depends on the success of another to activate), use their CRs to determine their combined Encounter Level as though they were monsters. The resulting Encounter Level is the CR for the combined traps.

\subsubsection{Mechanical Trap Cost}
The base cost of a mechanical trap is 1,000 cp. Apply all the modifiers from \tabref{Cost Modifiers for Mechanical Traps} for the various features you've added to the trap to get the modified base cost.

The final cost is equal to (modified base cost $\times$ Challenge Rating) + extra costs. The minimum cost for a mechanical trap is (CR $\times$ 100) cp.

After you've multiplied the modified base cost by the Challenge Rating, add the price of any alchemical items or poison you incorporated into the trap. If the trap uses one of these elements and has an automatic reset, multiply the poison or alchemical item cost by 20 to provide an adequate supply of doses.

\textbf{Multiple Traps:} If a trap is really two or more connected traps, determine the final cost of each separately, then add those values together. This holds for both multiple dependent and multiple independent traps (see the previous section).

\Table{Cost Modifiers for Mechanical Traps}{p{25mm}X}{
  \tableheader Feature
& \tableheader Cost Modifier\\

\cmidrule[0pt]{1-2}
\multicolumn{2}{l}{\TableSubheader{Trigger Type}}\\
Location                                  & --- \\
Proximity                                 & +1,000 cp\\
Touch                                     & --- \\
Touch (attached)                          & $-100$ cp\\
Timed                                     & +1,000 cp\\

\multicolumn{2}{l}{\TableSubheader{Reset Type}}\\
No reset                                  & $-500$ cp\\
Repair                                    & $-200$ cp\\
Manual                                    & --- \\
Automatic                                 & +500 cp (or 0 if trap has timed trigger)\\

\cmidrule[0pt]{1-2}
\multicolumn{2}{l}{\TableSubheader{Bypass Type}}\\
Lock                                      & +100 cp (\skill{Open Lock} DC 30)\\
Hidden switch                             & +200 cp (\skill{Search} DC 25)\\
Hidden lock                               & +300 cp (\skill{Open Lock} DC 30, \skill{Search} DC 25)\\

% \cmidrule[0pt]{1-2}
\multicolumn{2}{l}{\TableSubheader{\skill{Search} DC}}\\
19 or lower                               & $-100$ cp $\times$ (20 $-$ DC)\\
20                                        & --- \\
21 or higher                              & +200 cp $\times$ (DC $-$ 20)\\

% \cmidrule[0pt]{1-2}
\multicolumn{2}{l}{\TableSubheader{\skill{Disable Device} DC}}\\
19 or lower                               & $-100$ cp $\times$ (20 $-$ DC)\\
20                                        & --- \\
21 or higher                              & +200 cp $\times$ (DC $-$ 20)\\

% \cmidrule[0pt]{1-2}
\multicolumn{2}{l}{\TableSubheader{Reflex Save DC (Pit or Other Save-Dependent Trap)}}\\
19 or lower                               & $-100$ cp $\times$ (20 $-$ DC)\\
20                                        & --- \\
21 or higher                              & +300 cp $\times$ (DC $-$ 20)\\

% \cmidrule[0pt]{1-2}
\multicolumn{2}{l}{\TableSubheader{Attack Bonus (Melee or Ranged Attack Trap)}}\\
+9 or lower                               & $-100$ cp $\times$ (10 $-$ bonus)\\
+10                                       & --- \\
+11 or higher                             & +200 cp $\times$ (bonus $-$ 10)\\

\multicolumn{2}{l}{\TableSubheader{Damage Bonus}}\\
High strength rating (ranged attack trap) & +100 cp $\times$ bonus (max +4)\\
High Strength bonus (melee attack trap)   & +100 cp $\times$ bonus (max +8)\\

\cmidrule[0pt]{1-2}
\multicolumn{2}{l}{\TableSubheader{Miscellaneous Features}}\\
Never miss                                & +1,000 cp\\
Poison                                    & Cost of poison\footnotemark[1]\\
Alchemical item                           & Cost of item\footnotemark[1]\\


\TableNote{2}{1 Multiply cost by 20 if trap features automatic reset.}\\
}

\subsubsection{Magic Device Trap Cost}
Building a magic device trap involves the expenditure of experience points as well as gold pieces, and requires the services of a spellcaster. \tabref{Cost Modifiers for Magic Device Traps} summarizes the cost information for magic device traps. If the trap uses more than one spell (for instance, a sound or visual trigger spell in addition to the main spell effect), the builder must pay for them all (except alarm, which is free unless it must be cast by an NPC; see below).

The costs derived from \tabref{Cost Modifiers for Magic Device Traps} assume that the builder is casting the necessary spells himself (or perhaps some other PC is providing the spells for free). If an NPC spellcaster must be hired to cast them those costs must be factored in as well.

A magic device trap takes one day to construct per 500 cp of its cost.

\Table{Cost Modifiers for Magic Device Traps}{lX}{
  \tableheader Feature
& \tableheader Cost Modifier\\

\spell{Alarm} spell used in trigger & ---  \\

% \cmidrule[0pt]{1-2}
\multicolumn{2}{l}{\TableSubheader{One-Shot Trap}}\\

Each spell used in trap & +50 cp $\times$ caster level $\times$ spell level, +4 XP $\times$ caster level $\times$ spell level\\
Material components     & + Cost of all material components\\
XP components           & + Total of XP components $\times$ 5 cp\\

% \cmidrule[0pt]{1-2}
\multicolumn{2}{l}{\TableSubheader{Automatic Reset Trap}}\\
Each spell used in trap & +500 cp $\times$ caster level $\times$ spell level, +40 XP $\times$ caster level $\times$ spell level\\
Material components     & + Cost of all material components $\times$ 100 cp\\
XP components           & + Total of XP components $\times$ 500 cp\\
}

\subsubsection{Spell Trap Cost}
A spell trap has a cost only if the builder must hire an NPC spellcaster to cast it.

\subsubsection{Craft DCs For Mechanical Traps}
Once you know the Challenge Rating of a trap determine the \skill{Craft} (trapmaking) DC by referring to the table and the modifiers given below.

\Table{}{XX}{
  \tableheader Trap CR
& \tableheader Base \skill{Craft} (Trapmaking) DC \\
 1--3  & 20 \\
 4--6  & 25 \\
 7--10 & 30 \\
11--15 & 35 \\
16--20 & 40 \\
}
\Table{}{XX}{
  \tableheader Additional Components
& \tableheader Modifier to \skill{Craft} (Trapmaking) DC \\
Proximity trigger & +5 \\
Automatic reset   & +5 \\
}

\textbf{Making the Checks:} To determine how much progress a character makes on building a trap each week, that character makes a \skill{Craft} (trapmaking) check. See the \skill{Craft} skill description for details on \skill{Craft} checks and the circumstances that can affect them.

