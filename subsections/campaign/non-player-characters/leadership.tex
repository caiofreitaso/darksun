\subsection{Leadership}
The \feat{Leadership} feat brings two types of NPC allies: cohorts and followers. Cohorts are loyal allies who go with the PC to any adventure to face the challenges side-by-side. Followers are low-level allies who are happy to be of any help, but they are too fragile to face any challenges head-on.

\subsubsection{Cohorts}
Cohorts serve as companions for any adventure a particular character may face. Although cohorts usually follow one character, there can be instances of cohorts following a group. Whether they follow a group or a character, they always work out a deal agreeable to both parties so that they work for the character or group.

Cohorts can serve as many things: a sidekick, a healer, a scout, etc. But no matter what they do, they are always part of the group. They have a subservient role in the group and they do as they're told, but they voice their opinion on any matter. Their servitude should not be taken for granted, and mistreated cohorts become disloyal and may leave or even seek revenge.

There are no limitations on class, race, or gender of a character's cohorts. But a cohort's alignment should not be opposed to the leader's alignment on either the law-vs.-chaos or good-vs.-evil axis. Even a different alignment imposes a penalty on the Leadership score.

\textbf{Experience Points:} Cohorts depend on their leaders to earn experience, and do not earn in the same rate as player characters. Cohorts earn XP as follows:

\begin{itemize*}
\item The cohort does not count as a party member when determining the party's XP.
\item Divide the cohort's level by the level of the PC with whom they are associated (the character with the \feat{Leadership} feat who attracted the cohort).
\item Multiply this result by the total XP awarded to the PC and add that number of experience points to the cohort's total.
\end{itemize*}

If a cohort gains enough XP to bring it to a level one lower than the associated PC's character level, the cohort does not gain the new level---its new XP total is 1 less than the amount needed attain the next level.

\textbf{Treasure:} The most common deal for cohorts is for them to get half share of any treasure the party gains. For instance, if a party of four PCs and one cohort earns 1,000 cp, divide the gold pieces by 9 (two for each PC and one for the cohort). The cohort is awarded 112 cp, while the PCs get each 222 cp.

In Athas, it is \emph{very} rare for any one to seek no pay to work, especially working a dangerous job as the PCs are meant to do.

\subsubsection{Followers}
Followers are the allies used for menial tasks. They are not meant to be effective in combat, since they are generally five or more levels behind their leader. They are usually used as messengers, errand-runners, or guards.

Like cohorts, there are no limitations on class, race or gender for followers. Unlike cohorts, followers do not voice opinions on any matter. They are merely a character's minions, they are not part of the group.

Followers can't earn experience, or gain levels. Whenever the character with the \feat{Leadership} feat gains a new level, they may gain new followers of higher level, but the existing ones do not gain levels.

Followers do not demand a share of treasure, but all their needs and equipment should be provided by their leader.

\subsubsection{Replacing Cohorts and Followers}
If a leader loses a cohort or followers, they can generally replace them, according to their current \feat{Leadership} score. It takes time (1d4 months) to recruit replacements. If the leader is to blame for the deaths of the cohort or followers, it takes extra time to replace them, up to a full year (1d4$\times$3 months). Note that the leader also picks up a reputation of failure, which decreases their \feat{Leadership} score.
