\subsection{Hirelings}
Hirelings and servants can be paid on a regular basis to take care of things back home while adventurers are out and about. Characters can also use hirelings to carry torches, tote their treasure, and fight for them. Hirelings differ from cohorts in that they have no investment in what's going on. They just do their jobs.

Even though diligent adventurers spend most of their time in dungeons and other dangerous environs, there's still other work to be done. In most cases, it's easier to simply delegate menial tasks and day-to-day chores to paid servants, partners, and allies. These NPCs are collectively known as hirelings.

Unlike cohorts, hirelings do not make decisions. They do as they're told (at least in theory). Thus, even if they go on an adventure with the PCs, they gain no experience and do not affect any calculations involving the party level. Like cohorts, hirelings must be treated fairly well, or they will leave and might even turn against their former employers. Some hirelings might require hazard pay (perhaps as high as double normal pay) if placed in particularly dangerous situations. In addition to demanding hazard pay, hirelings placed in great danger might be unfriendly, but characters potentially can influence them to a better attitude and perhaps even talk them out of
hazard pay.

The wages of hirelings do not include materials, tools, or weapons the hirelings might need to do their job.

\Table{Prices for Hireling Services}{Xlll}{
  \tableheader Hireling
& \tableheader Per Day
& \tableheader Per Week
& \tableheader Per Month \\

\multicolumn{4}{l}{\TableSubheader{Military}}\\
Archer              & 1 bit  & 1 cp   & 4 cp   \\
Cavalry, heavy      & 3 bits & 3 cp   & 1 sp   \\
Cavalry, light      & 1 bit  & 1 cp   & 4 cp   \\
Crossbowman         & 5 bd   & 5 bits & 2 cp   \\
Engineer            & 5 cp   & 4 sp   & 15 sp  \\
Infantry, heavy     & 5 bd   & 5 bits & 2 cp   \\
Infantry, light     & 2 bd   & 2 bit  & 1 cp   \\
Infantry, irregular & 1 bd   & 1 bit  & 5 bits \\

\multicolumn{4}{l}{\TableSubheader{Civilian}}\\
Draqoman                        & $\star$ cp & $\star$ sp & $\star\times4$ sp\\
Craftsman\footnotemark[1]       & 1 bit  & 1 cp   & 4 cp \\
Professional\footnotemark[2]    & 5 bd   & 5 bits & 2 cp \\
Specialist\footnotemark[3]      & 3 bits & 3 cp   & 1 sp \\
Spellcaster                     & 5 bits & 5 cp   & 2 sp \\
Unskilled labor\footnotemark[4] & 2 bd   & 2 bits & 1 cp \\

\TableNote{4}{$\star$ Draqoman's level.}\\
\TableNote{4}{1 Any \skill{Craft} skill available.}\\
\TableNote{4}{2 Any \skill{Profession} skill available.}\\
\TableNote{4}{3 Any skill, other than Craft and Profession.}\\
\TableNote{4}{4 Jobs that do not require associated skills.}\\
}

The types of hirelings characters might employ are described below.

\textbf{Archer:} A 2nd-level fighter specialized in longbows, weilding a composite longbow (Str +2), and wearing studded leather and a buckler.

\textbf{Cavalry, Heavy:} A 2nd-level fighter wearing scale mail, and wielding a lance two-handed. They mount a heavy crodlu wearing a scale mail, as well.

\textbf{Cavalry, Light:} A 1st-level fighter wearing a studded leather and a wooden shield, and wielding a lance. They mount a light crodlu with no armor.

\textbf{Craftsman:} Usually a 2nd-level settler, but they can be of any class. This pays for any \skill{Craft} check of DC 20 or lower. Higher DC checks double the price for each 5 points of difference. 

\textbf{Crossbowman:} A 1st-level fighter specialized in crossbows, weilding a composite heavy crossbow (Str +2), and wearing scale mail and a tower shield.

\textbf{Draqoman:} A bard of 6th or higher level with at least one contact (see Alternative Class Features). Their wages depends on the bard's level and are the default price for skill contacts. If the draqoman has different types of contact, then they might double the price.

\textbf{Engineer:} A 5th-level fighter with 8 ranks of \skill{Knowledge} (warcraft) and a martial prowess that depend on this skill. They are equipped with \emph{breastplate +1} and a \emph{tower shield +1}. Although a military engineer has combat capabilities, their function is to oversee construction of siege engines and of static defenses.

\textbf{Infantry, Heavy:} A 2nd-level fighter, wearing a half-plate and tower shield, wielding a one-handed weapon. 

\textbf{Infantry, Light:} A 1st-level fighter, specialized either in spear and shields, or longswords. Light infantry usually wears studded leather, but caravan's infantry wear caravan armor.

\textbf{Infantry, Irregular:} A 2nd-level barbarian or gladiator. Their specialization is atypical, and they typically use gladiator armor or shell armor.

\textbf{Professional:} Usually a 2nd-level settler, but they can be of any class. This pays for any \skill{Profession} check of DC 20 or lower. Higher DC checks double the price for each 5 points of difference.

\textbf{Specialist:} A 2nd-level NPC with \feat{Skill Focus} on the corresponding skill speciality, who can deliver a skill check of DC 20. Scouts are usually hired for \skill{Knowledge} (geography), \skill{Spot}, or \skill{Survival}. Thieves for \skill{Disable Device}, \skill{Forgery}, \skill{Open Lock}, \skill{Search}, or \skill{Sleight of Hand}. Bards for \skill{Appraise}, \skill{Diplomacy}, or \skill{Perform}. Wizards for \skill{Decipher Script}, any \skill{Knowledge}, or \skill{Spellcraft}. These are guidelines, but the wages are the same for any class.

\textbf{Spellcaster:} A 3rd-level cleric or wizard. These wages do not pay for expensive material components, which must be given by the employer.

\textbf{Unskilled Labor:} Any job that a freeman might do that does not require a skill check.