\subsection{Contacts}
\label{sec:contacts}
This system codifies a phenomenon commonplace in most long-term campaigns: the friendly bartender, gruff weaponsmith, or absentminded sage who points the PCs in the right direction, passes along important clues, or offers unusual skills and knowledge.

Some alternate class features and prestige classes grant a PC with a reliable contact to call when the time comes. The player defines his contact at the time he gains this ability: where she lives, her race, personality, and so on. He is encouraged to roll the background of his contact (see \chapref{Character Description}).

\subsubsection{NPCs and Contacts}
While all defined contacts are friendly NPCs, that doesn't mean that all friendly NPCs are defined contacts. The contact mechanic is intended to supplement, not replace, other social interactions with noncombatant NPCs. It gives a player a chance to insert a minor character into the ongoing drama.

Defined contacts should be among the campaign's most stable characters. Unless the characters are completely obtuse or have remarkable misfortune, the minor characters they define as contacts aren't going anywhere. They're generally available wherever they happen to live, and they usually have the time and inclination to help their friend the PC. Major NPC characters---those defined entirely by the GM---are off limits as contacts. A player can't just say, ``I want to define the sorcerer-queen as a contact.''

A contact won't risk life or livelihood on the PC's say-so, but a contact makes some sacrifices for a friend. For example, a contact will burn the midnight oil translating an ancient text or sneak the key to the pantry out of the castle (as long as it's back by morning).

\subsubsection{Types of Contacts}
While your contact is friendly and tries to be helpful in small ways, you can ask for a more serious favor once per week per contact. Contacts have half as many character levels as you do. Contacts always have settler levels. They come in three varieties, one of which you must choose: influence contacts, skill contacts, and trade contacts.

\textbf{Influence Contacts:} An influence contact is a noble associated with one of the big organizations of Athas (see \chapref{Organizations}). While a player can't define the leader of an organization as his character's contact, he can define one of the nobles associated with its logistic as a contact. The noble doesn't have a broad store of information, and she doesn't have any skills the PCs might need. But she might be able to put in a good word with the organization, and she can certainly make introductions between the PC and the rest of the organization's logistic staff. The purpose of an influence contact is to enable and smooth talks with more important, but less friendly, NPCs.

These contacts are not skillful nor have goods to trade, but they can pull some strings to help you in the following ways: 

\begin{itemize*}
\item Grant an appointment or meeting with an NPC (templar, noble, gladiatorial slave, chieftain, etc. At DM's discretion).
\item Access to a place to stay hidden for three days.
\item Avoid templar inspection.
\end{itemize*}

These are not the only ways an influence contact may help, but they give a direction for the DM to come up with other favors this contact can make.

\textbf{Skill Contacts:} A skill contact is a craftsman focused on a single skill. Some skills---especially categories of \skill{Craft}, \skill{Profession}, and \skill{Knowledge}---are rarely possessed by PCs. Skill contacts have those skills in abundance, so they're useful when characters need a smith to repair a lance or an honest broker to appraise a giant emerald. Some just have an uncanny sense of what's going on in their neighborhood or town, such as the grumpy bartender or the talkative fruit merchant. These are specialized in \skill{Diplomacy}, \skill{Gather Information}, or \skill{Sense Motive}.

This contact has maximum ranks in the skill he is best at, and his highest ability score is in the key ability for the skill in question. A skill contact always has the \feat{Skill Focus} feat related to his field of specialty. When called for a favor this contact can deliver the effect of a skill check of DC equal to 18 + \onehalf your character level.

\textbf{Trade Contacts:} A trade contact is a craftsman who has access to a particular type of service or product. A guy who knows where to buy from the black market, or who has mercenary contacts. Whenever a barter may happen in his neighborhood, he knows. When asked for a favor, a trader contact can give you one of the following options:

\begin{itemize*}
\item +5 circumstance bonus on \skill{Diplomacy} check to bargain a single transaction.
\item Access to purchase and sell black market goods.
\item Access to hire mercenary of trader's desired race and class (see Hirelings).
\item Access to purchasing spellcasting services from a particular class.
\end{itemize*}