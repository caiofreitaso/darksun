\subsection{Environmental Hazards}
Some hazards are intrinsic to the environment it is in. For example, being on a mountainous environment has the potential of an avalanche happening. Caves, forests, and mountains all have hazards associated with them.

\subsubsection{Avalanches (CR 7)}
The combination of high peaks and heavy snowfalls means that avalanches are a deadly peril in many mountainous areas. While avalanches of snow and ice are common, it's also possible to have an avalanche of rock and soil.

An avalanche can be spotted from as far away as 1d10 $\times$ 150 meters downslope by a character who makes a DC 20 \skill{Spot} check, treating the avalanche as a Colossal creature. If all characters fail their \skill{Spot} checks to determine the encounter distance, the avalanche moves closer to them, and they automatically become aware of it when it closes to half the original distance. It's possible to hear an avalanche coming even if you can't see it. Under optimum conditions (no other loud noises occurring), a character who makes a DC 15 \skill{Listen} check can hear the avalanche or landslide when it is 1d6 $\times$ 150 meters away. This check might have a DC of 20, 25, or higher in conditions where hearing is difficult (such as in the middle of a thunderstorm).

A landslide or avalanche consists of two distinct areas: the bury zone (in the direct path of the falling debris) and the slide zone (the area the debris spreads out to encompass). Characters in the bury zone always take damage from the avalanche; characters in the slide zone may be able to get out of the way. Characters in the bury zone take 8d6 points of damage, or half that amount if they make a DC 15 Reflex save. They are subsequently buried (see below). Characters in the slide zone take 3d6 points of damage, or no damage if they make a DC 15 Reflex save. Those who fail their saves are buried.

Buried characters take 1d6 points of nonlethal damage per minute. If a buried character falls unconscious, he or she must make a DC 15 Constitution check or take 1d6 points of lethal damage each minute thereafter until freed or dead.

The typical avalanche has a width of 1d6 $\times$ 30 meters, from one edge of the slide zone to the opposite edge. The bury zone in the center of the avalanche is half as wide as the avalanche's full width.

To determine the precise location of characters in the path of an avalanche, roll 1d6 $\times$ 6; the result is the number of meters from the center of the path taken by the bury zone to the center of the party's location. Avalanches of snow and ice advance at a speed of 150 meters per round, and rock avalanches travel at a speed of 75 meters per round.

\subsubsection{Cave-Ins And Collapses (CR 8)}
Cave-ins and collapsing tunnels are extremely dangerous. Not only do dungeon explorers face the danger of being crushed by tons of falling rock, even if they survive they may be buried beneath a pile of rubble or cut off from the only known exit. A cave-in buries anyone in the middle of the collapsing area, and then sliding debris damages anyone in the periphery of the collapse. A typical corridor subject to a cave-in might have a bury zone with a 4.5-meter radius and a 3-meter-radius slide zone extending beyond the bury zone. A weakened ceiling can be spotted with a DC 20 \skill{Knowledge} (architecture and engineering) or DC 20 \skill{Craft} (stonemasonry) check. Remember that \skill{Craft} checks can be made untrained as Intelligence checks. A dwarf can make such a check if he simply passes within 3 meters of a weakened ceiling.

A weakened ceiling may collapse when subjected to a major impact or concussion. A character can cause a cave-in by destroying half the pillars holding the ceiling up.

Characters in the bury zone of a cave-in take 8d6 points of damage, or half that amount if they make a DC 15 Reflex save. They are subsequently buried. Characters in the slide zone take 3d6 points of damage, or no damage at all if they make a DC 15 Reflex save. Characters in the slide zone who fail their saves are buried.

Characters take 1d6 points of nonlethal damage per minute while buried. If such a character falls unconscious, he must make a DC 15 Constitution check. If it fails, he takes 1d6 points of lethal damage each minute thereafter until freed or dead.

Characters who aren't buried can dig out their friends. In 1 minute, using only her hands, a character can clear rocks and debris equal to five times her heavy load limit. The amount of loose stone that fills a 1.5-meter-by-1.5-meter area weighs one tonne (1,000 kg). Armed with an appropriate tool, such as a pick, crowbar, or shovel, a digger can clear loose stone twice as quickly as by hand. You may allow a buried character to free himself with a DC 25 Strength check.

\subsubsection{Flash Floods (CR 4)}
A flash flood can suddenly raise the water level of an area, filling a dry gulch to the top of its walls. A flood raises the water level by 1d10+3 meters within a matter of minutes. Water washes through affected squares, traveling at a speed of 18 meters or more, unless impeded by slopes or solid barriers.

Character caught in a flash flood must make DC 20 \skill{Swim} checks every round to avoid going under, and an additional DC 20 \skill{Swim} check is required each round to keep the head above water. Characters who stay below the surface might drown. If a character gets a check result of 5 or more over the minimum necessary to avoid going under, he arrests his motion by catching a rock, tree limb, or bottom snag---he is no longer being carried along by the flow of the water. Escaping the rapids by reaching the bank requires three DC 20 \skill{Swim} checks in a row. 

Characters arrested by a rock, limb, or snag can't escape under their own power unless they strike out into the water and attempt to swim their way clear. Other characters can rescue them with a branch, spear haft, rope, or similar tool that enables the rescuer to reach the victim with one end of it. Then the rescuer must make a DC 15 Strength check to successfully pull the victim, and the victim must make a DC 10 Strength check to hold onto the branch, pole, or rope. If the victim fails to hold on, he must make a DC 15 \skill{Swim} check immediately to stay above the surface. If both checks succeed, the victim is pulled 1.5 meter closer to safety.

Along with the hazards of fast-flowing water, the flow uproots trees and rolls enormous boulders with deadly impact. Characters struck by a wall of water during a flash flood must make a successful DC 15 Reflex save or take 3d6 points of bludgeoning damage. A flash flood passes through an area in 3d4 hours.

\subsubsection{Sandstorm (CR 3)}
Sandstorms impose a $-4$ penalty on Dexterity-based skill checks, as well as \skill{Search} checks, \skill{Spot} checks, and any other checks that rely on vision. Winds automatically extinguish unprotected flames and have a 75\% chance of blowing out protected flames, such as those of lanterns. Ranged weapon attacks are impossible, and even siege weapons have a $-4$ penalty on attack rolls. \skill{Listen} checks are at a $-8$ penalty due to the howling of the wind.

Moreover, sandstorms deal 1d3 points of nonlethal damage each round to anyone caught out in the open without shelter and pose a suffocation hazard. A sandstorm leaves 2d8 $\times$ 10 centimeters of fine sand in its wake. One in ten sandstorms become siroccos.

\textit{Gray Death:} Every round under a sandstorm (or sirocco) in the Sea of Silt, a character must save against the gray death disease. Expertly worn cloth gives +4 on the save. Inhaled---Fortitude DC 15, incubation period 1 day, damage 1d3 Str and 1d3 Dex. When damaged, character must succeed on another saving throw or become permanently fatigued.

\textit{Sirocco (CR 5):} Siroccos occur when wind blows at hurricane-force. All flames are extinguished. Ranged attacks are impossible (except with siege weapons, which have a $-8$ penalty on attack rolls). \skill{Listen} checks are impossible: All characters can hear is the roaring of the wind. Hurricane-force winds often fell trees.

A character in a sirocco take a $-6$ penalty on Dexterity-based skill checks, and \skill{Search} checks. Moreover, siroccos deal 1d3 points of lethal damage each round to anyone caught out in the open without shelter and pose a suffocation hazard. A sirocco leaves 2d4$-1$ meters of fine sand in its wake.

\textit{Suffocation:} Exposed characters might begin to choke if their noses and mouths are not covered. A sufficiently large cloth expertly worn (\skill{Survival} DC 15) negates the effects of suffocation from dust and sand. An inexpertly worn cloth across the nose and mouth protects a character from the potential of suffocation for a number of rounds equal to 10 $\times$ her Constitution score. An unprotected character faces potential suffocation after a number rounds equal to twice her Constitution score. Once the grace period ends, the character must make a successful Constitution check (DC 10, +1 per previous check) each round or begin suffocating on the encroaching sand. In the first round after suffocation begins, the character falls unconscious (0 hp). In the following round, she drops to $-1$ hit points and is dying. In the third round, she suffocates to death.

\subsubsection{Tornado (CR 10)}
All flames are extinguished. All ranged attacks are impossible (even with siege weapons), as are \skill{Listen} checks. Instead of the normal wind effects, Large or smaller characters in close proximity to a tornado who fail their Fortitude saves (DC 30) are sucked toward the tornado. Those who come in contact with the actual funnel cloud (5d12 $\times$ 9 meters of radius) are picked up and whirled around for 1d10 rounds, taking 6d6 points of damage per round, before being violently expelled (falling damage may apply). While a tornado's rotational speed can be as great as 500 km/h, the funnel itself moves forward at an average of 45 km/h (roughly 75 meters per round). A tornado uproots trees, destroys buildings, and causes other similar forms of major destruction.

Tornadoes are accompanied by lightning that can pose a hazard to characters without proper shelter (especially those in metal armor). As a rule of thumb, assume one bolt per minute for a 1-hour period at the center of the storm. Each bolt causes electricity damage equal to 1d10 eight-sided dice.

\subsubsection{Wildfires (CR 6)}
Most campfire sparks ignite nothing, but if conditions are dry, winds are strong, or the forest floor is dried out and flammable, a wildfire can result. Lightning strikes often set trees afire and start wildfires in this way. Whatever the cause of the fire, travelers can get caught in the conflagration.

A wildfire can be spotted from as far away as 2d6 $\times$ 30 meters by a character who makes a \skill{Spot} check, treating the fire as a Colossal creature (reducing the DC by 16). If all characters fail their \skill{Spot} checks, the fire moves closer to them. They automatically see it when it closes to half the original distance.

Characters who are blinded or otherwise unable to make \skill{Spot} checks can feel the heat of the fire (and thus automatically ``spot'' it) when it is 30 meters away.

The leading edge of a fire (the downwind side) can advance faster than a human can run (assume 36 meters per round for winds of moderate strength). Once a particular portion of the forest is ablaze, it remains so for 2d4 $\times$ 10 minutes before dying to a smoking smolder. Characters overtaken by a wildfire may find the leading edge of the fire advancing away from them faster than they can keep up, trapping them deeper and deeper in its grasp.

Within the bounds of a wildfire, a character faces three dangers: heat damage, catching on fire, and smoke inhalation.

\textbf{Heat Damage:} Getting caught within a wildfire is even worse than being exposed to extreme heat (see Heat Dangers). Breathing the air causes a character to take 1d6 points of damage per round (no save). In addition, a character must make a Fortitude save every 5 rounds (DC 15, +1 per previous check) or take 1d4 points of nonlethal damage. A character who holds his breath can avoid the lethal damage, but not the nonlethal damage. Those wearing heavy clothing or any sort of armor take a $-4$ penalty on their saving throws. In addition, those wearing metal armor or coming into contact with very hot metal are affected as if by a heat metal spell.

\textbf{Catching on Fire:} Characters engulfed in a wildfire are at risk of catching on fire when the leading edge of the fire overtakes them, and are then at risk once per minute thereafter.

\textbf{Smoke Inhalation:} Wildfires naturally produce a great deal of smoke. A character who breathes heavy smoke must make a Fortitude save each round (DC 15, +1 per previous check) or spend that round choking and coughing. A character who chokes for 2 consecutive rounds takes 1d6 points of nonlethal damage. Also, smoke obscures vision, providing concealment to characters within it.