\subsection{Slimes, Molds, And Fungi}
In a dungeon's damp, dark recesses, molds and fungi thrive. While some plants and fungi are monsters and other slime, mold, and fungus is just normal, innocuous stuff, a few varieties are dangerous dungeon encounters. For purposes of spells and other special effects, all slimes, molds, and fungi are treated as plants.

A form of glistening organic sludge coats almost anything that remains in the damp and dark for too long. This kind of slime, though it might be repulsive, is not dangerous.

Molds and fungi flourish in dark, cool, damp places. While some are as inoffensive as the normal dungeon slime, others are quite dangerous. Mushrooms, puffballs, yeasts, mildew, and other sorts of bulbous, fibrous, or flat patches of fungi can be found throughout most dungeons. They are usually inoffensive, and some are even edible (though most are unappealing or odd-tasting).

\subsubsection{Green Slime (CR 4)}
This dungeon peril is a dangerous variety of normal slime. Green slime devours flesh and organic materials on contact and is even capable of dissolving metal. Bright green, wet, and sticky, it clings to walls, floors, and ceilings in patches, reproducing as it consumes organic matter. It drops from walls and ceilings when it detects movement (and possible food) below.

A single 1.5-meter square of green slime deals 1d6 points of Constitution damage per round while it devours flesh. On the first round of contact, the slime can be scraped off a creature (most likely destroying the scraping device), but after that it must be frozen, burned, or cut away (dealing damage to the victim as well). Anything that deals cold or fire damage, sunlight, or a remove disease spell destroys a patch of green slime. Against wood or metal, green slime deals 2d6 points of damage per round, ignoring metal's hardness but not that of wood. It does not harm stone.

\subsubsection{Yellow Mold (CR 6)}
If disturbed, a 1.5-meter square of this mold bursts forth with a cloud of poisonous spores. All within 3 meters of the mold must make a DC 15 Fortitude save or take 1d6 points of Constitution damage. Another DC 15 Fortitude save is required 1 minute later---even by those who succeeded on the first save---to avoid taking 2d6 points of Constitution damage. Fire destroys yellow mold, and sunlight renders it dormant.

\subsubsection{Brown Mold (CR 2)}
Brown mold feeds on warmth, drawing heat from anything around it. It normally comes in patches 1.5 meter in diameter, and the temperature is always cold in a 9-meter radius around it. Living creatures within 1.5 meter of it take 3d6 points of nonlethal cold damage. Fire brought within 1.5 meter of brown mold causes it to instantly double in size. Cold damage, such as from a cone of cold, instantly destroys it.

\subsubsection{Phosphorescent Fungus (No CR)}
This strange underground fungus grows in clumps that look almost like stunted shrubbery. It gives off a soft violet glow that illuminates underground caverns and passages as well as a candle does. Rare patches of fungus illuminate as well as a torch does.
