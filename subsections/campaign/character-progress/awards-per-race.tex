\subsection{Awards per Race}
These awards are for roleplaying some of the stereotypical aspects of athasian character races. Players should remember that races are more than just stats on their character sheets, they have culture and history which are what these stereotypes try to enforce.

The judgment of good roleplaying ultimately lies with the DM, and they must be familiar with the nuances of the character races. The communication between the DM and the players should be clear so that a good roleplaying experience can arise, and the nature of {\tableheader Dark Sun} can be emphasized.

To determine the XP award for a particular action, check the correspondent table for the character's race and multiply the award by the character's effective character level (ECL). Remember that this will quicken the rate of progress of the characters.

\subsubsection{Dwarf}
Dwarves live by their foci. A focus must take at least a week to complete. If a focus takes a least a year to complete, it becomes a major focus.

Focus can be changed in very rare circumstances. These circumstances must be agreed between the player and the DM.

\subsubsection{Elf}
Roleplaying an elf is centered around trust. Elves are self-reliant and do not want to gain friendship with every character they meet. They test redeemable outsiders (in the elvish perspective) to see if they are trustworthy.

\textit{Examples of subtle tests of trust:}
\begin{itemize*}
	\item entrust with confidential information,
	\item leave a valuable item easy for taking to see the outsider takes it,
	\item ask to deliver a message or item.
\end{itemize*}

\textit{Examples of life-threatening tests of trust:}
\begin{itemize*}
	\item let themselves get captured to see if there is a rescue attempt,
	\item fake unconsciousness after a battle to see what type of care is provided,
	\item cut supplies to see if they get a fair share.
\end{itemize*}

\subsubsection{Half-Elf}
Every half-elf seek acceptance among humans and elves, t hough they deny it as much as possible. Observing simple customs for the first time should award bonus experience points. These can be drinking the local ale with the elven chieftain or participating in a human wedding ritual.

If a local custom takes form of a competition, the half-elf gains bonus experience points if they perform better than any \emph{one} of the humans or elves also participating. If they perform better than \emph{all} the humans or elves, they get double the experience award.


\subsubsection{Half-Giant}
A half-giant seeks guidance and purpose in others' lifestyles. Player characters should seek to imitate the most charismatic member of the party in their racial and class customs. Whenever they do, they should be rewarded for it.

Half-giant can also look elsewhere for inspiration, imitating non-player characters and may even switch sides in an adventure. Whenever a player goes this far, they should get bonus experience points.

Whenever half-giants shift their alignment based on the events in the campaign, the DM should give them bonus experience points. This is only for appropriate shifts that are followed by a meaningful roleplay.

\subsubsection{Halfling}
Halflings come from isolated tribes and, similar to half-elves, they want to experiment other races' customs. Unlike half-elves, their drive is curiosity, instead of trying to fit in. Whenever halflings try a  custom for the first time, no matter how trivial, they gain an experience bonus.

Their sense of belonging makes them honor bound to aid another halfling in need. This should only be rewarded when there is danger of injury or loss of life to the aiding halfling.

\subsubsection{Mul}
As a race bred exclusively for slavery, muls lack a culture similar to the other races. What they all share is the culture of labor. Whenever they exert themselves, they should be awarded bonus experience. This should only be rewarded if the exertion is meaningful to the adventure.

\subsubsection{Thri-Kreen}
Thri-kreen come from an empire of hunter-gatherers. Whenever they take back a slain creature for food, it should warrant experience bonus.

\Table{Experience Awards Per Race}{X p{3cm}}{
  \tableheader Action
& \tableheader XP Award \\

\TableSubheader{Dwarves} & \\
~ Pursue present focus & 50 XP $\times$ days pursuing \\
~ Ignore present focus & $-100$ XP $\times$ days ignoring \\
~ Complete major focus & 1,000 XP \\

\TableSubheader{Elves} & \\
~ Subtle test of trust               & 25 XP \\
~ Life-threatening test of trust     & 200 XP \\
~ Refuse animal or magical transport & 50 XP \\
~ Continuous run                     & 10 XP $\times$ distance in km \\

\TableSubheader{Half-Elves} & \\
~ Observe human or elven custom   & 25 XP \\
~ Better a human or elf in custom & 250 XP \\

\TableSubheader{Half-Giants} & \\
~ Imitate charismatic friend    & 25 XP $\times$ days imitating \\
~ Shift alignment per influence & 50 XP \\


\TableSubheader{Halflings} & \\
~ Refuse money                   & 25 XP \\
~ Practice another race's custom & 50 XP \\
~ Eat slain foe                  & 50 XP \\
~ Aid another halfling           & 100 XP \\

\TableSubheader{Muls} & \\
~ Heavy exertion & 10 XP $\times$ day of work \\

\TableSubheader{Thri-Kreen} & \\
~ Defeat creature for food & 50 XP \\
~ Paralyze creature        & 100 XP \\
}
