\subsection{Tiered Powers}
There is a natural imbalance towards magical (and psionic) powers. So much so that the whole system is balanced around wonderous items, and it is expected to have spellcasters in an adventuring group. But at higher levels, this imbalance in power can become frustrating for both DMs and players, as the powers become increasingly more complex and game-changing.

In \tabref{Tiered Powers}, you can see the level adjustment based on the maximum spell level or power level available for a manifester or spellcaster. This means that a 17th-level human wizard should be equivalent to a 19th-level character---she has access to 9th-level magic and thus has a +2 level adjustment. An elf ranger of the same level (17th) should be equivalent to a 18th-level character---4th-level psionic powers grant +1 level adjustment.

\Table{Tiered Powers}{lXc}{
\tableheader Tier & \tableheader Maximum Spell/Power Level Available & \tableheader Total Level Adjustment \\
I   & 1st--3rd & +0 \\
II  & 4th--6th & +1 \\
III & 7th--9th & +2 \\
}

In an already running adventure, a character that will change their tier should only advance another level after gaining enough experience for two whole levels. For instance, Jozan is a 6th-level human cleric. In order for him to get his 7th level (and 4th-level spells), he must gain enough XP to advance to an effective 8th character level. Once the character makes a tier breakthrough, he advances normally with his level adjustment and new effective character level until another breakthrough must be made. So from his 7th level to 12th level, Jozan advances with a +1 level adjustment. After he becomes a 12th-level cleric (13 ECL), he should again acquire enough XP to advance two character levels in order to gain access to 7th-level spells---this time to an effective 15th character level. After that, he advances normally again with his +2 level adjustment.
