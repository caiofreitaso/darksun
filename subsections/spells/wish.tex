\Spell{Wish}{wish}
{Universal}
{
	\textbf{Level:}
	Wiz 9\\
	\textbf{Components:}
	V, XP\\
	\textbf{Casting Time:}
	1 standard action\\
	\textbf{Range:}
	See text\\
	\textbf{Target, Effect, or Area:}
	See text\\
	\textbf{Duration:}
	See text\\
	\textbf{Saving Throw:}
	See text\\
	\textbf{Spell Resistance:}
	Yes\\
}
{
	Wish is the mightiest spell a wizard or sorcerer can cast. By simply speaking aloud, you can alter reality to better suit you.

	Even wish, however, has its limits.

	A wish can produce any one of the following effects.


Duplicate any wizard or sorcerer spell of 8th level or lower, provided the spell is not of a school prohibited to you.
Duplicate any other spell of 6th level or lower, provided the spell is not of a school prohibited to you.
Duplicate any wizard or sorcerer spell of 7th level or lower even if it's of a prohibited school.
Duplicate any other spell of 5th level or lower even if it's of a prohibited school.
Undo the harmful effects of many other spells, such as geas/quest or insanity.
Create a nonmagical item of up to 25,000 gp in value.
Create a magic item, or add to the powers of an existing magic item.
	\textbf{Grant a creature a +1 inherent bonus to an ability score. Two to five wish spells cast in immediate succession can grant a creature a +2 to +5 inherent bonus to an ability score (two wishes for a +2 inherent bonus, three for a +3 inherent bonus, and so on). Inherent bonuses are instantaneous, so they cannot be dispelled. Note: An inherent bonus may not exceed +5 for a single ability score, and inherent bonuses to a particular ability score do not stack, so only the best one applies.}
Remove injuries and afflictions. A single wish can aid one creature per caster level, and all subjects are cured of the same kind of affliction. For example, you could heal all the damage you and your companions have taken, or remove all poison effects from everyone in the party, but not do both with the same wish. A wish can never restore the experience point loss from casting a spell or the level or Constitution loss from being raised from the dead.
Revive the dead. A wish can bring a dead creature back to life by duplicating a resurrection spell. A wish can revive a dead creature whose body has been destroyed, but the task takes two wishes, one to recreate the body and another to infuse the body with life again. A wish cannot prevent a character who was brought back to life from losing an experience level.
Transport travelers. A wish can lift one creature per caster level from anywhere on any plane and place those creatures anywhere else on any plane regardless of local conditions. An unwilling target gets a Will save to negate the effect, and spell resistance (if any) applies.
Undo misfortune. A wish can undo a single recent event. The wish forces a reroll of any roll made within the last round (including your last turn). Reality reshapes itself to accommodate the new result. For example, a wish could undo an opponent's successful save, a foe's successful critical hit (either the attack roll or the critical roll), a friend's failed save, and so on. The reroll, however, may be as bad as or worse than the original roll. An unwilling target gets a Will save to negate the effect, and spell resistance (if any) applies.

	You may try to use a wish to produce greater effects than these, but doing so is dangerous. (The wish may pervert your intent into a literal but undesirable fulfillment or only a partial fulfillment.)

	Duplicated spells allow saves and spell resistance as normal (but save DCs are for 9th-level spells).

	\textit{Material Component}:
	When a wish duplicates a spell with a material component that costs more than 10,000 gp, you must provide that component.

	\textit{XP Cost}:
	The minimum XP cost for casting wish is 5,000 XP. When a wish duplicates a spell that has an XP cost, you must pay 5,000 XP or that cost, whichever is more. When a wish creates or improves a magic item, you must pay twice the normal XP cost for crafting or improving the item, plus an additional 5,000 XP.

}
