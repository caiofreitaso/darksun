\Spell{Control Tides}{control tides}
{Transmutation [Water or Earth]}
{
	\textbf{Level:}
	Clr 4, Drd 4, Wiz 6, Tmp 6\\
	\textbf{Components:}
	V, S, M/DF\\
	\textbf{Casting Time:}
	1 standard action\\
	\textbf{Range:}
	Long (120 m + 12 m/level)\\
	\textbf{Area:}
	Water or silt in a volume of 3 m/level $\times$ 3 m/level $\times$ 0.5 m/level (S)\\
	\textbf{Duration:}
	10 minutes/level (D)\\
	\textbf{Saving Throw:}
	None; see text\\
	\textbf{Spell Resistance:}
	No\\
}
{
	Depending on the version you choose, the \emph{control tides} spell raises or lowers the level of water or silt.

	\textit{Lower Tide:} This causes silt (or water or similar liquid) to sink away to a minimum depth of 2 centimeters. The depth can be lowered by up to 50 centimeters per caster level. The surface is lowered within a squarish depression whose sides are up to 3 meters long per caster level. In extremely large and deep bodies of silt, such as deep in the Sea of Silt, the spell creates a whirlpool that sweeps ships and similar craft downward, putting them at risk and rendering them unable to leave by normal movement for the duration of the spell. When cast on water or silt elementals and other water- or silt-based creatures, this spell acts as a \spell{slow} spell (Will negates). The spell has no effect on other creatures.

	\textit{Raise Tide:} This causes silt (or water or similar liquid) to rise in height, just as the lower tide version causes it to lower. Silt skimmers raised in this way slide down the sides of the hump that the spell creates. If the area affected by the spell includes riverbanks, a beach, other land near the raised water or silt, the water or silt can spill over onto dry land. With either version, you may reduce one horizontal dimension by half and double the other horizontal dimension.

	\textit{Arcane Material Component:} A pinch of dust.
}
