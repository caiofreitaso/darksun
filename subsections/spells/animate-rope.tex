\Spell{Animate Rope}{animate rope}
{Transmutation}
{
	\textbf{Level:}
	Wiz 1\\
	\textbf{Components:}
	V, S\\
	\textbf{Casting Time:}
	1 standard action\\
	\textbf{Range:}
	Medium (30 m + 3 m/level)\\
	\textbf{Target:}
	One ropelike object, length up to 15 m + 1.5 m/level; see text\\
	\textbf{Duration:}
	1 round/level\\
	\textbf{Saving Throw:}
	None\\
	\textbf{Spell Resistance:}
	No\\
}
{
	You can animate a nonliving ropelike object. The maximum length assumes a rope with a 2-centimeter diameter.

	Reduce the maximum length by 50\% for every additional inch of thickness, and increase it by 50\% for each reduction of the rope's diameter by half.

	The possible commands are ``coil'' (form a neat, coiled stack), ``coil and knot,'' ``loop,'' ``loop and knot,'' ``tie and knot,'' and the opposites of all of the above (``uncoil,'' and so forth). You can give one command each round as a move action, as if directing an active spell.

	The rope can enwrap only a creature or an object within 30 centimeters of it---it does not snake outward---so it must be thrown near the intended target. Doing so requires a successful ranged touch attack roll (range increment 3 meters). A typical 2-centimeter-diameter hempen rope has 2 hit points, AC 10, and requires a DC 23 Strength check to burst it. The rope does not deal damage, but it can be used as a trip line or to cause a single opponent that fails a Reflex saving throw to become entangled. A creature capable of spellcasting that is bound by this spell must make a DC 15 \skill{Concentration} check to cast a spell. An entangled creature can slip free with a DC 20 \skill{Escape Artist} check.

	The rope itself and any knots tied in it are not magical.

	This spell grants a +2 bonus on any \skill{Use Rope} checks you make when using the transmuted rope.

	The spell cannot animate objects carried or worn by a creature.

}
