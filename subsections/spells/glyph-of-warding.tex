\Spell{Glyph of Warding}{glyph of warding}
{Abjuration}
{
	\textbf{Level:}
	Clr 3\\
	\textbf{Components:}
	V, S, M\\
	\textbf{Casting Time:}
	10 minutes\\
	\textbf{Range:}
	Touch\\
	\textbf{Target or Area:}
	Object touched or up to 5 sq. ft./level\\
	\textbf{Duration:}
	Permanent until discharged (D)\\
	\textbf{Saving Throw:}
	See text\\
	\textbf{Spell Resistance:}
	No (object) and Yes; see text\\
}
{
	This powerful inscription harms those who enter, pass, or open the warded area or object. A glyph of warding can guard a bridge or passage, ward a portal, trap a chest or box, and so on.

	You set the conditions of the ward. Typically, any creature entering the warded area or opening the warded object without speaking a password (which you set when casting the spell) is subject to the magic it stores. Alternatively or in addition to a password trigger, glyphs can be set according to physical characteristics (such as height or weight) or creature type, subtype, or kind. Glyphs can also be set with respect to good, evil, law, or chaos, or to pass those of your religion. They cannot be set according to class, Hit Dice, or level. Glyphs respond to invisible creatures normally but are not triggered by those who travel past them ethereally. Multiple glyphs cannot be cast on the same area. However, if a cabinet has three drawers, each can be separately warded.

	When casting the spell, you weave a tracery of faintly glowing lines around the warding sigil. A glyph can be placed to conform to any shape up to the limitations of your total square footage. When the spell is completed, the glyph and tracery become nearly invisible.

	Glyphs cannot be affected or bypassed by such means as physical or magical probing, though they can be dispelled. Mislead, polymorph, and nondetection (and similar magical effects) can fool a glyph, though nonmagical disguises and the like can't. Read magic allows you to identify a glyph of warding with a DC 13 \skill{Spellcraft} check. Identifying the glyph does not discharge it and allows you to know the basic nature of the glyph (version, type of damage caused, what spell is stored).

	\textit{Note:} Magic traps such as glyph of warding are hard to detect and disable. A rogue (only) can use the Search skill to find the glyph and Disable Device to thwart it. The DC in each case is 25 + spell level, or 28 for glyph of warding.

	Depending on the version selected, a glyph either blasts the intruder or activates a spell.

	\textit{Blast Glyph:}
	A blast glyph deals 1d8 points of damage per two caster levels (maximum 5d8) to the intruder and to all within 1.5 meter of him or her. This damage is acid, cold, fire, electricity, or sonic (caster's choice, made at time of casting). Each creature affected can attempt a Reflex save to take half damage. Spell resistance applies against this effect.

	\textit{Spell Glyph:}
	You can store any harmful spell of 3rd level or lower that you know. All level-dependent features of the spell are based on your caster level at the time of casting the glyph. If the spell has a target, it targets the intruder. If the spell has an area or an amorphous effect the area or effect is centered on the intruder. If the spell summons creatures, they appear as close as possible to the intruder and attack. Saving throws and spell resistance operate as normal, except that the DC is based on the level of the spell stored in the glyph.

	\textit{Material Component:}
	You trace the glyph with incense, which must first be sprinkled with powdered diamond worth at least 200 cp.

}
