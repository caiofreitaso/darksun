\Spell{Create Element}{create element}
{Conjuration (Creation) [Ritual, see text]}
{
	\textbf{Level:}
	Clr 0\\
	\textbf{Components:}
	V, S\\
	\textbf{Casting Time:}
	1 standard action\\
	\textbf{Range:}
	Close (7.5 m + 1.5 m/2 levels)\\
	\textbf{Effect:}
	Element created (see below)\\
	\textbf{Duration:}
	Instantaneous\\
	\textbf{Saving Throw:}
	None; see text\\
	\textbf{Spell Resistance:}
	No\\
}
{
	This spell creates a small amount of the caster's patron element. Specifics for each element follow:

	\textit{Air:} An air cleric can conjure a lungful of pure air. This air can be breathed by any one character within range. If that character is holding his breath or suffocating, he is no longer suffocating and if he must continue to hold his breath, he does so as if he had taken a deep breath of air. The pure air also invigorates a creature if it is not drowning or suffocating. The creature receives a +4 bonus on any check made for prolonged physical activity (as the \feat{Endurance} feat), provided the check is made within one minute.

	\textit{Earth:} An earth cleric can conjure a small amount of elemental earth. This earth can weigh no more than 0.5 kg/level, but can be either loose earth or unworked stone. This conjured earth appears level to the ground.

	\textit{Fire:} A fire cleric can conjure a torch-sized flame in the palm of his hand. (This deals no damage to the cleric and has no danger of setting him or his equipment on fire.) This flame provides light as a torch and lasts 1 round/level. The torch can be used to light flammable objects aflame or as a weapon that deals 1 point of fire damage (like a normal torch). If the flame is used in this way, it dissipates after hitting an object or creature.

	\textit{Magma:} A magma cleric can summon a small amount of red-hot magma. The magma appears anywhere in range at ground level, provided that space is not occupied by a creature. The magma is a 1.5-m-radius circle. It deals 1d6 points of fire damage to any creature or object touching it. It cools after one round into a block of stone that weighs 5 kg.

	\textit{Rain:} A rain cleric can conjure a small rainstorm. This storm rains 4 liters of water/level over a 1.5-m-radius circle. It puts out any torch-sized or smaller fires in the area.

	\textit{Silt:} A silt cleric can conjure a cloud of silt that surrounds the head of a creature within range. This silt obscures vision, inflicting a $-1$ penalty on attack rolls made for 1 round if the target fails a Fortitude save. Sightless creatures are not affected by this cloud.

	\textit{Sun:} A sun cleric can conjure bright light. This functions as the \spell{daylight} spell but lasts only 1 round.

	\textit{Water:} A water cleric can conjure up to 9 liter of water/level. This spell generates wholesome, drinkable water, just like clean rain water. Water can be created in an area as small as will actually contain the liquid, or in an area three times as large---possibly creating a downpour or filling many small receptacles.

	\textbf{Note:} Conjuration spells can't create substances or objects within a creature. Water weighs about 1 kilogram per liter.
}
