\Spell{Globe of Invulnerability, Lesser}{lesser globe of invulnerability}
{Abjuration}
{
	\textbf{Level:}
	Wiz 4\\
	\textbf{Components:}
	V, S, M\\
	\textbf{Casting Time:}
	1 standard action\\
	\textbf{Range:}
	3 m\\
	\textbf{Area:}
	3-m-radius spherical emanation, centered on you\\
	\textbf{Duration:}
	1 round/level (D)\\
	\textbf{Saving Throw:}
	None\\
	\textbf{Spell Resistance:}
	No\\
}
{
	An immobile, faintly shimmering magical sphere surrounds you and excludes all spell effects of 3rd level or lower. The area or effect of any such spells does not include the area of the \emph{lesser globe of invulnerability}. Such spells fail to affect any target located within the \emph{globe}. Excluded effects include spell-like abilities and spells or spell-like effects from items. However, any type of spell can be cast through or out of the magical \emph{globe}. Spells of 4th level and higher are not affected by the \emph{globe}, nor are spells already in effect when the \emph{globe} is cast. The \emph{globe} can be brought down by a targeted \spell{dispel magic} spell, but not by an area \spell{dispel magic}. You can leave and return to the \emph{globe} without penalty.

	Note that spell effects are not disrupted unless their effects enter the \emph{globe}, and even then they are merely suppressed, not dispelled.

	If a given spell has more than one level depending on which character class is casting it, use the level appropriate to the caster to determine whether \emph{lesser globe of invulnerability} stops it.

	\textit{Material Component:}
	A glass or crystal bead that shatters at the expiration of the spell.

}
