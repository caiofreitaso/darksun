\Spell{Prying Eyes}{prying eyes}
{Divination}
{
	\textbf{Level:}
	Wiz 5\\
	\textbf{Components:}
	V, S, M\\
	\textbf{Casting Time:}
	1 minute\\
	\textbf{Range:}
	One mile\\
	\textbf{Effect:}
	Ten or more levitating eyes\\
	\textbf{Duration:}
	1 hour/level; see text (D)\\
	\textbf{Saving Throw:}
	None\\
	\textbf{Spell Resistance:}
	No\\
}
{
	You create a number of semitangible, visible magical orbs (called ``eyes'') equal to 1d4 + your caster level. These eyes move out, scout around, and return as you direct them when casting the spell. Each eye can see 36 meters (normal vision only) in all directions.

	While the individual eyes are quite fragile, they're small and difficult to spot. Each eye is a Fine construct, about the size of a small apple, that has 1 hit point, AC 18 (+8 bonus for its size), flies at a speed of 9 meters with perfect maneuverability, and has a +16 Hide modifier. It has a Spot modifier equal to your caster level (maximum +15) and is subject to illusions, darkness, fog, and any other factors that would affect your ability to receive visual information about your surroundings. An eye traveling through darkness must find its way by touch.

	When you create the eyes, you specify instructions you want them to follow in a command of no more than twenty-five words. Any knowledge you possess is known by the eyes as well.

	In order to report their findings, the eyes must return to your hand. Each replays in your mind all it has seen during its existence. It takes an eye 1 round to replay 1 hour of recorded images. After relaying its findings, an eye disappears.

	If an eye ever gets more than 1 mile away from you, it instantly ceases to exist. However, your link with the eye is such that you won't know if the eye was destroyed because it wandered out of range or because of some other event.

	The eyes exist for up to 1 hour per caster level or until they return to you. Dispel magic can destroy eyes. Roll separately for each eye caught in an area dispel. Of course, if an eye is sent into darkness, it could hit a wall or similar obstacle and destroy itself.

	\textit{Material Component}:
	A handful of crystal marbles.

}
