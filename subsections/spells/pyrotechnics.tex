\Spell{Pyrotechnics}{pyrotechnics}
{Transmutation}
{
	\textbf{Level:}
	Wiz 2\\
	\textbf{Components:}
	V, S, M\\
	\textbf{Casting Time:}
	1 standard action\\
	\textbf{Range:}
	Long (120 m + 12 m/level)\\
	\textbf{Target:}
	One fire source, up to a 6-meter cube\\
	\textbf{Duration:}
	1d4+1 rounds, or 1d4+1 rounds after creatures leave the smoke cloud; see text\\
	\textbf{Saving Throw:}
	Will negates or Fortitude negates; see text\\
	\textbf{Spell Resistance:}
	Yes or No; see text\\
}
{
	Pyrotechnics turns a fire into either a burst of blinding fireworks or a thick cloud of choking smoke, depending on the version you choose.

	\textit{Fireworks:} The fireworks are a flashing, fiery, momentary burst of glowing, colored aerial lights. This effect causes creatures within 36 meters of the fire source to become blinded for 1d4+1 rounds (Will negates). These creatures must have line of sight to the fire to be affected. Spell resistance can prevent blindness.

	\textit{Smoke Cloud:} A writhing stream of smoke billows out from the source, forming a choking cloud. The cloud spreads 6 meters in all directions and lasts for 1 round per caster level. All sight, even darkvision, is ineffective in or through the cloud. All within the cloud take $-4$ penalties to Strength and Dexterity (Fortitude negates). These effects last for 1d4+1 rounds after the cloud dissipates or after the creature leaves the area of the cloud. Spell resistance does not apply.

	\textit{Material Component:} The spell uses one fire source, which is immediately extinguished. A fire so large that it exceeds a 6-meter cube is only partly extinguished. Magical fires are not extinguished, although a fire-based creature used as a source takes 1 point of damage per caster level.
}
