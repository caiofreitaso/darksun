\Spell{Giant Vermin}{giant vermin}
{Transmutation}
{
	\textbf{Level:}
	Clr 4, Cycle 4, Drd 4\\
	\textbf{Components:}
	V, S, DF\\
	\textbf{Casting Time:}
	1 standard action\\
	\textbf{Range:}
	Close (7.5 m + 1.5 m/2 levels)\\
	\textbf{Targets:}
	Up to three vermin, no two of which can be more than 9 m apart\\
	\textbf{Duration:}
	1 min./level\\
	\textbf{Saving Throw:}
	None\\
	\textbf{Spell Resistance:}
	Yes\\
}
{
\Table{}{XX}{
\tableheader Caster Level & \tableheader Vermin Size\\
	9th or lower & Medium\\
	10th--13th & Large\\
	14th--17th & Huge\\
	18th--19th & Gargantuan\\
	20th or higher & Colossal\\
}

	You turn three normal-sized centipedes, two normal-sized spiders, or a single normal-sized scorpion into larger forms. Only one type of vermin can be transmuted (so a single casting cannot affect both a centipede and a spider), and all must be grown to the same size. The size to which the vermin can be grown depends on your level; see the table.

	Any giant vermin created by this spell do not attempt to harm you, but your control of such creatures is limited to simple commands (``Attack,'' ``Defend,'' ``Stop,'' and so forth). Orders to attack a certain creature when it appears or guard against a particular occurrence are too complex for the vermin to understand. Unless commanded to do otherwise, the giant vermin attack whoever or whatever is near them.

}
