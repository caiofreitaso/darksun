\Spell{Wall of Ice}{wall of ice}
{Evocation [Cold]}
{
	\textbf{Level:}
	Wiz 4\\
	\textbf{Components:}
	V, S, M\\
	\textbf{Casting Time:}
	1 standard action\\
	\textbf{Range:}
	Medium (30 m + 3 m/level)\\
	\textbf{Effect:}
	Anchored plane of ice, up to one 3-m square/level, or hemisphere of ice with a radius of up to 1 m + 0.3 m/level\\
	\textbf{Duration:}
	1 min./level\\
	\textbf{Saving Throw:}
	Reflex negates; see text\\
	\textbf{Spell Resistance:}
	Yes\\
}
{
	This spell creates an anchored plane of ice or a hemisphere of ice, depending on the version selected. A wall of ice cannot form in an area occupied by physical objects or creatures. Its surface must be smooth and unbroken when created. Any creature adjacent to the wall when it is created may attempt a Reflex save to disrupt the wall as it is being formed. A successful save indicates that the spell automatically fails. Fire can melt a wall of ice, and it deals full damage to the wall (instead of the normal half damage taken by objects). Suddenly melting a wall of ice creates a great cloud of steamy fog that lasts for 10 minutes.

	\textit{Ice Plane:}
	A sheet of strong, hard ice appears. The wall is 1 inch thick per caster level. It covers up to a 3-meter-square area per caster level (so a 10th-level wizard can create a wall of ice 30 meters long and 3 meters high, a wall 15 meters long and 6 meters high, or some other combination of length and height that does not exceed 90 square meters). The plane can be oriented in any fashion as long as it is anchored. A vertical wall need only be anchored on the floor, while a horizontal or slanting wall must be anchored on two opposite sides.

	Each 3-meter square of wall has 3 hit points per inch of thickness. Creatures can hit the wall automatically. A section of wall whose hit points drop to 0 is breached. If a creature tries to break through the wall with a single attack, the DC for the Strength check is 15 + caster level.

	Even when the ice has been broken through, a sheet of frigid air remains. Any creature stepping through it (including the one who broke through the wall) takes 1d6 points of cold damage +1 point per caster level (no save).

	\textit{Hemisphere:}
	The wall takes the form of a hemisphere whose maximum radius is 1 meter + 30 centimeters per caster level. The hemisphere is as hard to break through as the ice plane form, but it does not deal damage to those who go through a breach.

	\textit{Material Component:}
	A small piece of quartz or similar rock crystal.

}
