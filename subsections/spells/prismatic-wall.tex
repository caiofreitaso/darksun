\Spell{Prismatic Wall}{prismatic wall}
{Abjuration}
{
	\textbf{Level:}
	Wiz 8\\
	\textbf{Components:}
	V, S\\
	\textbf{Casting Time:}
	1 standard action\\
	\textbf{Range:}
	Close (7.5 m + 1.5 m/2 levels)\\
	\textbf{Effect:}
	Wall 4 ft./level wide, 2 ft./level high\\
	\textbf{Duration:}
	10 min./level (D)\\
	\textbf{Saving Throw:}
	See text\\
	\textbf{Spell Resistance:}
	See text\\
}
{
	Prismatic wall creates a vertical, opaque wall---a shimmering, multicolored plane of light that protects you from all forms of attack. The wall flashes with seven colors, each of which has a distinct power and purpose. The wall is immobile, and you can pass through and remain near the wall without harm. However, any other creature with less than 8 HD that is within 6 meters of the wall is blinded for 2d4 rounds by the colors if it looks at the wall.

	The wall's maximum proportions are 1.2 meter wide per caster level and 60 centimeters high per caster level. A prismatic wall spell cast to materialize in a space occupied by a creature is disrupted, and the spell is wasted.

	Each color in the wall has a special effect. The accompanying table shows the seven colors of the wall, the order in which they appear, their effects on creatures trying to attack you or pass through the wall, and the magic needed to negate each color.

	The wall can be destroyed, color by color, in consecutive order, by various magical effects; however, the first color must be brought down before the second can be affected, and so on. A rod of cancellation or a mage's disjunction spell destroys a prismatic wall, but an antimagic field fails to penetrate it. Dispel magic and greater dispel magic cannot dispel the wall or anything beyond it. Spell resistance is effective against a prismatic wall, but the caster level check must be repeated for each color present.

	\emph{Prismatic wall} can be made permanent with a \spell{permanency} spell.
}
