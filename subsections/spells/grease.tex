\Spell{Grease}{grease}
{Conjuration (Creation)}
{
	\textbf{Level:}
	Wiz 1\\
	\textbf{Components:}
	V, S, M\\
	\textbf{Casting Time:}
	1 standard action\\
	\textbf{Range:}
	Close (7.5 m + 1.5 m/2 levels)\\
	\textbf{Target or Area:}
	One object or a 3-m square\\
	\textbf{Duration:}
	1 round/level (D)\\
	\textbf{Saving Throw:}
	See text\\
	\textbf{Spell Resistance:}
	No\\
}
{
	A \emph{grease} spell covers a solid surface with a layer of slippery grease. Any creature in the area when the spell is cast must make a successful Reflex save or fall. This save is repeated on your turn each round that the creature remains within the area. A creature can walk within or through the area of \emph{grease} at half normal speed with a DC 10 \skill{Balance} check. Failure means it can't move that round (and must then make a Reflex save or fall), while failure by 5 or more means it falls (see the \skill{Balance} skill for details).

	The spell can also be used to create a greasy coating on an item. Material objects not in use are always affected by this spell, while an object wielded or employed by a creature receives a Reflex saving throw to avoid the effect. If the initial saving throw fails, the creature immediately drops the item. A saving throw must be made in each round that the creature attempts to pick up or use the greased item. A creature wearing greased armor or clothing gains a +10 circumstance bonus on \skill{Escape Artist} checks and on grapple checks made to resist or escape a grapple or to escape a pin.

	\textit{Material Component:}
	A bit of pork rind or butter.

}
