\Spell{Curse of the Black Sands}{curse of the black sands}
{Transmutation}
{
	\textbf{Level:}
	Cycle 3, Drd 3, Drought 3, Silt 2\\
	\textbf{Components:}
	V, S\\
	\textbf{Casting Time:}
	1 standard action\\
	\textbf{Range:}
	Close (25 ft. + 5 ft./2 levels)\\
	\textbf{Target:}
	One creature\\
	\textbf{Duration:}
	1 day/level\\
	\textbf{Saving Throw:}
	Will negates\\
	\textbf{Spell Resistance:}
	Yes\\
}
{
	When \emph{curse of the black sands} is cast, the target leaves black, oily footprints in the earth or on silt. The prints are easily tracked and cannot be erased or destroyed until the spell expires. Any \skill{Survival} checks to track the target have the DC reduced by 10. These tracks can be covered, but not by earth. A giant leaf, for example, could hide a few footprints, but this would be a temporary fix at best. The target does not leave these tracks if he flies, \spellref{teleport}{teleports}, or climbs on a non-earthen surface (such as climbing through trees).

	In areas of silt, a black streak resembling a slow current will follow the cursed character whenever he is in contact with the surface. This includes wading or walking on the top of the silt by spell or magical devices, but not the use of a siltskimmer or floater. Should you cast several of these spells on multiple targets, you will be able to tell the various trails apart, even if they should cross or overlap.

	The trail disappears when the spell expires.
}
