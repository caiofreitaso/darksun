% \subsection{Environmental Hazards}

% \subsubsection{Sun Dangers}
% \textbf{Glare:} Salt flats reflect the sun's glare into the eyes even when not looked at directly. Sun glare is doubly dangerous when the sun is low on the horizon and thus difficult to avoid looking at.

% Characters traveling in such conditions must cover their eyes with a specialized eye covering, or improvise a eye protection (\skill{Survival} DC 15). Those whose eyes are unprotected in such conditions are automatically dazzled. Such characters take a $-1$ penalty on attack rolls, \skill{Search} checks, and \skill{Spot} checks. Characters who take the precaution of covering or shielding their eyes automatically eliminate the risk of being dazzled by sun glare and take no penalties.

% Glare-induced blindness lasts as long as characters remain in an area of sun glare and for 1d4 hours thereafter, or for 1 hour thereafter if the character enters a shadowed or enclosed area. The dazzling effect of sun glare can be negated by a \spell{remove blindness} spell, but an unprotected character still in an area of sun glare immediately becomes dazzled again when the spell's duration expires.

% \textbf{Mirages:} Air above salt flats and sandy wastes can distort features and produce optical illusions. Mirages can disorient travelers in the waste by obscuring landmarks or making distances seem shorter than they actually are.

% One can reduce the effect of a mirage by getting to higher elevation, which minimizes the amount of refraction. Of course, this requires not only a place to climb (or a \spell{fly} spell) but also the ability to recognize what you are looking at. An observer can make a DC 12 Will save to disbelieve the apparent image. A character who suspects a mirage gets a +4 circumstance bonus on this save. Once the existence of a mirage is revealed, disbelief is automatic.
% \subsubsection{Water Dangers}

\subsection{Conditional Hazards}
\subsubsection{Darkness}
Darkvision allows many characters and monsters to see perfectly well without any light at all, but characters with normal vision (or low-light vision, for that matter) can be rendered completely blind by putting out the lights. Torches or lanterns can be blown out by sudden gusts of subterranean wind, magical light sources can be dispelled or countered, or magical traps might create fields of impenetrable darkness.

In many cases, some characters or monsters might be able to see, while others are blinded. For purposes of the following points, a blinded creature is one who simply can't see through the surrounding darkness.
\begin{itemize*}
\item Creatures blinded by darkness lose the ability to deal extra damage due to precision (for example, a sneak attack).
\item Blinded creatures are hampered in their movement, and pay 2 squares of movement per square moved into (double normal cost). Blinded creatures can't run or charge.
\item All opponents have total concealment from a blinded creature, so the blinded creature has a 50\% miss chance in combat. A blinded creature must first pinpoint the location of an opponent in order to attack the right square; if the blinded creature launches an attack without pinpointing its foe, it attacks a random square within its reach. For ranged attacks or spells against a foe whose location is not pinpointed, roll to determine which adjacent square the blinded creature is facing; its attack is directed at the closest target that lies in that direction.
\item A blinded creature loses its Dexterity adjustment to AC and takes a $-2$ penalty to AC.
\item A blinded creature takes a $-4$ penalty on Search checks and most Strength- and Dexterity-based skill checks, including any with an armor check penalty. A creature blinded by darkness automatically fails any skill check relying on vision.
\item Creatures blinded by darkness cannot use gaze attacks and are immune to gaze attacks.
\end{itemize*}

A creature blinded by darkness can make a \skill{Listen} check as a free action each round in order to locate foes (DC equal to opponents' \skill{Move Silently} checks). A successful check lets a blinded character hear an unseen creature ``over there somewhere.'' It's almost impossible to pinpoint the location of an unseen creature. A \skill{Listen} check that beats the DC by 20 reveals the unseen creature's square (but the unseen creature still has total concealment from the blinded creature).
\begin{itemize*}
\item A blinded creature can grope about to find unseen creatures. A character can make a touch attack with his hands or a weapon into two adjacent squares using a standard action. If an unseen target is in the designated square, there is a 50\% miss chance on the touch attack. If successful, the groping character deals no damage but has pinpointed the unseen creature's current location. (If the unseen creature moves, its location is once again unknown.)
\item If a blinded creature is struck by an unseen foe, the blinded character pinpoints the location of the creature that struck him (until the unseen creature moves, of course). The only exception is if the unseen creature has a reach greater than 1.5 meter (in which case the blinded character knows the location of the unseen opponent, but has not pinpointed him) or uses a ranged attack (in which case, the blinded character knows the general direction of the foe, but not his location).
\item A creature with the scent ability automatically pinpoints unseen creatures within 1.5 meter of its location.
\end{itemize*}

\subsubsection{Dehydration}
\subsubsection{Falling}
\textbf{Falling Damage:} The basic rule is simple: 1d6 points of damage per 3 meters fallen, to a maximum of 20d6.

If a character deliberately jumps instead of merely slipping or falling, the damage is the same but the first 1d6 is nonlethal damage. A DC 15 \skill{Jump} check or DC 15 \skill{Tumble} check allows the character to avoid any damage from the first 3 meters fallen and converts any damage from the second 3 meters to nonlethal damage. Thus, a character who slips from a ledge 30 feet up takes 3d6 damage. If the same character deliberately jumped, he takes 1d6 points of nonlethal damage and 2d6 points of lethal damage. And if the character leaps down with a successful \skill{Jump} or \skill{Tumble} check, he takes only 1d6 points of nonlethal damage and 1d6 points of lethal damage from the plunge.

Falls onto yielding surfaces (soft ground, mud) also convert the first 1d6 of damage to nonlethal damage. This reduction is cumulative with reduced damage due to deliberate jumps and the \skill{Jump} skill.

\textbf{Falling into Water:} Falls into water are handled somewhat differently. If the water is at least 3 meters deep, the first 6 meters of falling do no damage. The next 6 meters do nonlethal damage (1d3 per 3-meter increment). Beyond that, falling damage is lethal damage (1d6 per additional 3-meter increment).

Characters who deliberately dive into water take no damage on a successful DC 15 Swim check or DC 15 \skill{Tumble} check, so long as the water is at least 3 meters deep for every 30 feet fallen. However, the DC of the check increases by 5 for every 50 feet of the dive.
\subsubsection{Sleep Deprivation}
\subsubsection{Starvation}
\subsubsection{Suffocation}
