\subsection{Diseases}
When a character is injured by a contaminated attack, touches an item smeared with diseased matter, or consumes disease-tainted food or drink, he must make an immediate Fortitude saving throw. If he succeeds, the disease has no effect---his immune system fought off the infection. If he fails, he takes damage after an incubation period. Once per day afterward, he must make a successful Fortitude saving throw to avoid repeated damage. Two successful saving throws in a row indicate that he has fought off the disease and recovers, taking no more damage.

These Fortitude saving throws can be rolled secretly so that the player doesn't know whether the disease has taken hold.

\subsubsection{Disease Descriptions}
Diseases have various symptoms and are spread through a number of vectors. The characteristics of several typical diseases are summarized on Table: Diseases and defined below.

\textit{Disease:} Diseases whose names are printed in italic in the table are supernatural in nature. The others are extraordinary.

\textit{Infection:} The disease's method of delivery---ingested, inhaled, via injury, or contact. Keep in mind that some injury diseases may be transmitted by as small an injury as a flea bite and that most inhaled diseases can also be ingested (and vice versa).

\textit{DC:} The Difficulty Class for the Fortitude saving throws to prevent infection, or (if the character has been infected), to prevent each instance of repeated damage, and to recover from the disease.

\textit{Incubation Period:} The time before damage begins.

\textit{Damage:} The ability damage the character takes after incubation and each day afterward.

\Table{Diseases}{llcll}{
  \tableheader Disease
& \tableheader Infection
& \tableheader DC
& \tableheader Incubation
& \tableheader Damage \\
Cerebral parasite & Contact  & 15 & 1d4 days & 1d2 Int\footnotemark[1] \\
Chitin rot        & Injury   & 16 & 1d3 days & 1d3 natural armor \\
Dehydrating fever & Ingested & 14 & 1d6 days & 1d4 Str\footnotemark[2] \\
Filth fever       & Injury   & 12 & 1d3 days & 1d3 Dex, 1d3 Con \\
Gray death        & Inhaled  & 15 &   1 day  & 1d3 Str, 1d3 Dex\footnotemark[3] \\
Red ache          & Injury   & 15 & 1d3 days & 1d6 Str \\
Wheezing death    & Injury   & 14 & 1d2 days & 1d6 Con \\
\TableNote{5}{1 Psionic creatures with cerebral parasites are limited to using each of their known powers only once per day (instead of freely manifesting them).}\\
\TableNote{5}{2 When damaged, character becomes dehydrated. Every full day without water after the damage, the creature takes 1 point of Constitution damage.}\\
\TableNote{5}{3 When damaged, character must succeed on another saving throw or become permanently fatigued.}\\
}

\textbf{Types of Diseases:} Typical diseases include the following:

\textit{Cerebral Parasites:} Cerebral parasites are tiny organisms, undetectable to normal sight. An afflicted character may not even know he carries the parasites---until he discovers he has fewer power points for the day than expected. 

\textit{Chitin Rot:} Sap from the Forest Ridge's trees spread it. Only affects thri-kreen and other creatures with exoskeleton.

\textit{Dehydrating Fever:} Spread in tainted water.

\textit{Filth Fever:} Dire rats and otyughs spread it. Those injured while in filthy surroundings might also catch it.

\textit{Gray Death:} Caused by long exposure to duststorms at Sea of Silt.

\textit{Red Ache:} Skin turns red, bloated, and warm to the touch.

% Ear Mites
% Eye Mites
% Desert Lice
% Sand Fleas
% Mind Worms

% Blinding sickness	Ingested	16	1d3 days	1d4 Str1
% Cackle fever	Inhaled	16	1 day	1d6 Wis
% Demon fever	Injury	18	1 day	1d6 Con2
% Devil chills3	Injury	14	1d4 days	1d4 Str
% Filth fever	Injury	12	1d3 days	1d3 Dex, 1d3 Con
% Mindfire	Inhaled	12	1 day	1d4 Int
% Mummy rot4	Contact	20	1 day	1d6 Con
% Red ache	Injury	15	1d3 days	1d6 Str
% Shakes	Contact	13	1 day	1d8 Dex
% Slimy doom	Contact	14	1 day	1d4 Con2

\subsubsection{Healing A Disease}
Use of the \skill{Heal} skill can help a diseased character. Every time a diseased character makes a saving throw against disease effects, the healer makes a check. The diseased character can use the healer's result in place of his saving throw if the \skill{Heal} check result is higher. The diseased character must be in the healer's care and must have spent the previous 8 hours resting.

Characters recover points lost to ability score damage at a rate of 1 per day per ability damaged, and this rule applies even while a disease is in progress. That means that a character with a minor disease might be able to withstand it without accumulating any damage.