\subsection{Temperature Dangers}
For game purposes, air temperature falls into one of the ten temperature bands describe on \tabref{Temperature Bands}.

\Table{Temperature Bands}{lllX}{
& & \multicolumn{2}{c}{\tableheader Fortitude Saves}\\
\cmidrule[.5pt]{3-4}
  \tableheader Name
& \tableheader Temperature Range
& \tableheader Every...
& \tableheader Damage \\
Unearthly heat & 83 °C or higher      & 1 round    & 1d6 lethal\footnotemark[1] + 1d4 nonlethal\footnotemark[1] \\
Extreme heat   & 61 °C to 82 °C       & 1 minute   & 1d6 lethal\footnotemark[1] + 1d4 nonlethal \\
Severe heat    & 43 °C to 60 °C       & 10 minutes & 1d4 nonlethal \\
Very hot       & 33 °C to 42 °C       & 1 hour     & 1d4 nonlethal \\
Moderate       & 16 °C to 32 °C       &            &  \\
Cool           & 5 °C to 15 °C        &            &  \\
Very cold      & $-17$ °C to 4 °C     & 1 hour     & 1d4 nonlethal \\
Severe cold    & $-28$ °C to $-18$ °C & 10 minutes & 1d4 nonlethal \\
Extreme cold   & $-45$ °C to $-29$ °C & 1 minute   & 1d6 lethal\footnotemark[1] + 1d4 nonlethal \\
Unearthly cold & $-46$ °C or lower    & 1 round    & 1d6 lethal\footnotemark[1] + 1d4 nonlethal\footnotemark[1] \\
\TableNote{4}{1 No save.}
}

Temperatures outside cool and moderate bands can be hazardous to unprepared characters. Characters who take any nonlethal damage from the temperature become fatigued. Damage from hot temperatures is known as heatstroke, while damage from cold temperatures is known as hypotermia.

A character with the \skill{Survival} skill can receive a bonus on saving throws against heat or cold, and can apply this bonus to other characters as well.

Unprotected characters in any hazardous temperature band must make successful Fortitude checks (DC 15, +1 per previous check) or take damage (see \tabref{Temperature Bands}).

\subsubsection{Cold Temperatures}
\textbf{Very cold:} Unprotected characters must make a Fortitude save each hour (DC 15, +1 per previous check) or take 1d6 points of nonlethal damage. 

\textbf{Severe cold:} Unprotected characters must make a Fortitude save once every 10 minutes (DC 15, +1 per previous check), taking 1d6 points of nonlethal damage on each failed save. Partially protected characters take damage once per hour instead of once per 10 minutes.

\textbf{Extreme cold:} Unprotected characters take 1d6 points of lethal damage per 10 minutes (no save). In addition, a character must make a Fortitude save (DC 15, +1 per previous check) or take 1d4 points of nonlethal damage. Those wearing metal armor or coming into contact with very cold metal are affected as if by a \spell{chill metal} spell. Partially protected characters take damage once per hour instead of once per 10 minutes.

\textbf{Unearthly cold:} Unprotected characters take 1d6 points of lethal damage and 1d4 points of nonlethal damage per round (no save). Partially protected characters take damage once per 10 minutes instead of once per round.

\Figure{t}{images/exploring-1.png}

\BigTableBottom{Random Weather}{lL*{6}{l}}{
  \tableheader d\%
& \tableheader Weather
& \tableheader Ringing Mountains
& \tableheader Tablelands
& \tableheader Hinterlands
& \tableheader Forest Ridge
& \tableheader Sea of Silt
& \tableheader Valley of Dust and Fire \\
01--10 & Calm
  & Calm winds
  & Calm winds
  & Calm winds
  & Calm winds
  & Calm winds
  & Moderate winds \\
11--70 & Normal
  & Calm winds
  & Moderate winds
  & Calm winds
  & Moderate winds
  & Moderate winds
  & Strong winds \\
71--80 & Abnormal
  & Moderate winds
  & Strong winds
  & Moderate winds
  & Moderate winds
  & Strong winds
  & Severe winds \\
% 81--86 & Precipitation
%   & Rain/snow
%   & Strong winds
%   & Rain
%   & Rain
%   & Severe winds
%   & Duststorm \\
% 87--89 & Abnormal precipitation
%   & Fog
%   & Severe winds
%   & Downpour
%   & Downpour
%   & Duststorm
%   & Sandstorm \\
% 90 & Rare precipitation
%   & Downpour/sleet
%   & Downpour
%   & Fog
%   & Fog
%   & Sandstorm
%   & Sirocco \\
81--90 & Inclement
 & Precipitation
 & Severe winds
 & Precipitation
 & Downpour
 & Duststorm
 & Sandstorm \\
91--99 & Storm
  & Thunderstorm
  & Duststorm
  & Thunderstorm
  & Thunderstorm
  & Sandstorm
  & Sirocco \\
100 & Powerful storm
  & Windstorm
  & Sandstorm
  & Tornado
  & Windstorm
  & Sirocco
  & Tornado \\
}

\subsubsection{Hot Temperatures}
Characters with heavy clothing or any kind of armor take $-4$ penalty on their saves against hot temperatures.

\textbf{Very hot:} Unprotected characters must make successful Fortitude saving throws each hour
(DC 15, +1 for each previous check) or take 1d4 points of nonlethal damage.

\textbf{Severe heat:} Unprotected characters must make successful Fortitude saving throws once every 10 minutes (DC 15, +1 for each previous check) or take 1d4 points of nonlethal damage. Partially protected characters take damage once per hour instead of once per 10 minutes.

\textbf{Extreme heat:} unprotected characters take 1d6 points of lethal damage per 10 minutes (no save). In addition, unprotected characters must make successful Fortitude saving throws (DC 15, +1 per previous check) every 10 minutes or take 1d4 points of nonlethal damage. In addition, those wearing metal armor or coming into contact with very hot metal are affected as if by a \spell{heat metal} spell (which lasts as long as the character remains in the area of extreme heat). Partially protected characters take damage once per hour instead of once per 10 minutes.

\textbf{Unearthly heat:} Unprotected characters take 1d6 points of lethal damage and 1d4 points of nonlethal damage per round (no save). In addition, those wearing metal armor or coming into contact with very hot metal are affected as if by a \spell{heat metal} spell (which lasts as long as the character remains in the area of unearthly heat). Partially protected characters take damage once per hour instead of once per 10 minutes.

\subsubsection{Temperature Damage}
\textbf{Heatstroke:} A character who takes any nonlethal damage from heat from the environment suffer from heatstroke and becomes fatigued. At lower temperatures, this damage starts off as nonlethal while the character is still conscious, but it becomes lethal for those already rendered unconscious by heatstroke (with no saving throw allowed).

\textit{Treating heatstroke:} Nonlethal damage from heatstroke (including the accompanying fatigue) cannot be recovered until a character gets cooled off---by reaching shade, surviving until nightfall, getting doused in water, being targeted by \spell{endure elements}, or the equivalent. Once the character is cooled or reaches a cooler environment (moderate or colder), the character responds normally to healing that removes nonlethal damage. When the character recovers the nonlethal damage taken from heatstroke, the fatigue penalties also end.

\textbf{Hypothermia:} A character who takes any nonlethal damage from cold or exposure is beset by mild hypothermia and therefore treated as fatigued. Immersion in chilled waters calls for an immediate check to resist the effects of cold or exposure and increases the DC of all Fortitude saves to avoid taking damage from cold or resisting cold-based spells and effects by 10 until the character and his clothes become dry.

Once a character succumbs to mild hypothermia, he becomes susceptible to moderate and severe levels of hypothermia. Any character with mild hypothermia who fails a Fortitude save to avoid the effects of cold or exposure is beset by moderate hypothermia and is treated as exhausted. Any character with moderate hypothermia who fails a Fortitude save to avoid the effects of cold or exposure is beset by severe hypothermia and is treated as disabled.

\textit{Treating hypothermia:} A successful DC 15 \skill{Heal} check can lower the level of hypothermia of the victim by one level (severe to moderate, moderate to mild, mild to none). The DC is modified by the snow cave, den, or similar shelter, can gain a bonus to their conditions listed in the table below.

\Table{}{XC}{
  \tableheader Condition
& \tableheader \skill{Heal} DC Modifier\\
Heat from fire   & $-5$ \\
Body contact     & $-1$ \\
Wet clothing     & +2 \\
Cold environment & +3 \\
}
