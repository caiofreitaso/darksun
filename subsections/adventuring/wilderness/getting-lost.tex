\subsection{Getting Lost}
There are many ways to get lost in the wilderness. Following an obvious road, trail, or feature such as a stream or shoreline prevents any possibility of becoming lost, but travelers striking off cross-country may become disoriented---especially in conditions of poor visibility or in difficult terrain.

\Table{Survival DCs to Avoid Getting Lost}{XC}{
\tableheader Circumstance & \tableheader Survival DC\\
Moor or hill, map & 6\\
Mountain, map & 8\\
Moor or hill, no map & 10\\
Poor visibility & 12\\
Mountain, no map & 12\\
Forest & 15\\
}

\textbf{Poor Visibility:} Any time characters cannot see at least 18 meters in the prevailing conditions of visibility, they may become lost. Characters traveling through fog, snow, or a downpour might easily lose the ability to see any landmarks not in their immediate vicinity. Similarly, characters traveling at night may be at risk, too, depending on the quality of their light sources, the amount of moonlight, and whether they have darkvision or low-light vision.

\textbf{Difficult Terrain:} Any character in forest, moor, hill, or mountain terrain may become lost if he or she moves away from a trail, road, stream, or other obvious path or track. Forests are especially dangerous because they obscure far-off landmarks and make it hard to see the sun or stars.

\textit{Chance to Get Lost:} If conditions exist that make getting lost a possibility, the character leading the way must succeed on a \skill{Survival} check or become lost. The difficulty of this check varies based on the terrain, the visibility conditions, and whether or not the character has a map of the area being traveled through. Refer to the table below and use the highest DC that applies.

A character with at least 5 ranks in \skill{Knowledge} (geography) or \skill{Knowledge} (local) pertaining to the area being traveled through gains a +2 bonus on this check.

Check once per hour (or portion of an hour) spent in local or overland movement to see if travelers have become lost. In the case of a party moving together, only the character leading the way makes the check.

\textit{Effects of Being Lost:} If a party becomes lost, it is no longer certain of moving in the direction it intended to travel. Randomly determine the direction in which the party actually travels during each hour of local or overland movement. The characters' movement continues to be random until they blunder into a landmark they can't miss, or until they recognize that they are lost and make an effort to regain their bearings.

\textit{Recognizing that You're Lost:} Once per hour of random travel, each character in the party may attempt a \skill{Survival} check (DC 20, $-1$ per hour of random travel) to recognize that they are no longer certain of their direction of travel. Some circumstances may make it obvious that the characters are lost.

\textit{Setting a New Course:} A lost party is also uncertain of determining in which direction it should travel in order to reach a desired objective. Determining the correct direction of travel once a party has become lost requires a \skill{Survival} check (DC 15, +2 per hour of random travel). If a character fails this check, he chooses a random direction as the ``correct'' direction for resuming travel.

Once the characters are traveling along their new course, correct or incorrect, they may get lost again. If the conditions still make it possible for travelers to become lost, check once per hour of travel as described in Chance to Get Lost, above, to see if the party maintains its new course or begins to move at random again.

\textit{Conflicting Directions:} It's possible that several characters may attempt to determine the right direction to proceed after becoming lost. Make a \skill{Survival} check for each character in secret, then tell the players whose characters succeeded the correct direction in which to travel, and tell the players whose characters failed a random direction they think is right.

\textbf{Regaining Your Bearings:} There are several ways to become un-lost. First, if the characters successfully set a new course and follow it to the destination they're trying to reach, they're not lost anymore. Second, the characters through random movement might run into an unmistakable landmark. Third, if conditions suddenly improve---the fog lifts or the sun comes up---lost characters may attempt to set a new course, as described above, with a +4 bonus on the \skill{Survival} check. Finally, magic may make their course clear.
