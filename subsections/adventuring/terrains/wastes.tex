\subsection{Wastes}
Wastes are terrains that lack vegetation because of a combination of poor soil and lack of rain. In these environments, water is increasingly more important for any lifeform as the heat is .

\subsubsection{Barren Waste}
Barren 

Huge dunes function like hills, and are designed like such while having the other features of a sandy waste.

\Table{Barren Waste Features}{lCCC}{
& \multicolumn{3}{c}{\tableheader Barren Waste Category}\\
\cmidrule[0.5pt]{2-4}

  \tableheader Terrain Feature
& \tableheader Boulder Field
& \tableheader Sandy Wastes
& \tableheader Stony Barrens \\

Deep sand          &      & 10\% &      \\
Shallow sand       &      & 20\% &  5\% \\
Shallow sand crust &      & 10\% &      \\
Sand dunes         &      & 20\% &      \\
Light rubble       & 20\% & 10\% & 50\% \\
Dense rubble       & 50\% &      & 20\% \\
}

\textbf{Stealth and Detection in Barren Waste:} In most cases, the maximum distance in barren waste terrain at which a Spot check for detecting the nearby presence of others can succeed is 6d6 $\times$ 6 meters. Beyond this distance, elevation changes and heat distortion make visual spotting impossible. Where sand dunes are present, the spotting distance is halved.

Barren waste imposes neither a bonus nor a penalty on \skill{Listen} or \skill{Spot} checks. The DC of \skill{Move Silently} checks increases by 2 in gravel, however.


\subsubsection{Evaporated Sea}


\Table{Evaporated Sea Features}{lCCC}{
& \multicolumn{3}{c}{\tableheader Evaporated Sea Category}\\
\cmidrule[0.5pt]{2-4}

  \tableheader Terrain Feature
& \tableheader Silt Sea
& \tableheader Dry Sea
& \tableheader Salt Flat \\

Deep sand            &      & 10\% &      \\
Shallow sand         &      & 20\% &  5\% \\
Deep sand crust      &      & 10\% &      \\
Shallow sand crust   &      & 10\% &      \\
Sand dunes           &      & 20\% &      \\
Light rubble         & 20\% & 10\% & 50\% \\
Shallow bog          & 50\% &      & 20\% \\
}

\subsubsection{Obsidian Plain}
