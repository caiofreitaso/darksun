\subsection{Wastes}
Wastes are terrains that lack vegetation because of a combination of poor soil and lack of rain. In these environments, water is increasingly more important for any lifeform.

\subsubsection{Barren Waste}
Barren wastes are the dominant terrain on the Tablelands, created by many types of weathering. Boulder fields were created when Athas suddenly warmed and the rock underground fractured as the water evaporated. Sandy wastes of today were created just before the Sea of Silt, when the sea evaporated and the sediments got caught by the wind. Stony barrens are made of hard orange-red bedrock that fracture into large sheets of rock by the intense temperature difference of the Athasian day.

Huge dunes function like hills, and are designed like such while having the other features of a sandy waste.

\Table{Barren Waste Features}{lCCC}{
& \multicolumn{3}{c}{\tableheader Barren Waste Category}\\
\cmidrule[0.5pt]{2-4}

  \tableheader Terrain Feature
& \tableheader Boulder Field
& \tableheader Sandy Wastes
& \tableheader Stony Barrens \\

Deep sand          &      & 10\% &      \\
Shallow sand       &      & 20\% &  5\% \\
Shallow sand crust &      & 10\% &      \\
Sand dunes         &      & 20\% &      \\
Light rubble       & 20\% & 10\% & 50\% \\
Dense rubble       & 50\% &      & 20\% \\
}

\textbf{Stealth and Detection in Barren Waste:} In most cases, the maximum distance in barren waste terrain at which a \skill{Spot} check for detecting the nearby presence of others can succeed is 6d6 $\times$ 6 meters. Beyond this distance, elevation changes and heat distortion make visual spotting impossible. Where sand dunes are present, the spotting distance is halved.

Barren waste imposes neither a bonus nor a penalty on \skill{Listen} or \skill{Spot} checks. The DC of \skill{Move Silently} checks increases by 2 in gravel, however.


\subsubsection{Evaporated Sea}
When Athas dried, its seas and lakes were gone, and different types of terrain appeared on its surface. The silt seas and dry seas are the remains of previous bodies of water. Silt seas retained some of the moisture, while dry seas did not. The Sea of Silt, despite the name, is a dry sea with pockets of silt seas within its vastness. Salt flats happen when the water could not drain into the ground and evaporated, leaving all the minerals behind.

\Table{Evaporated Sea Features}{lCCC}{
& \multicolumn{3}{c}{\tableheader Evaporated Sea Category}\\
\cmidrule[0.5pt]{2-4}

  \tableheader Terrain Feature
& \tableheader Silt Sea
& \tableheader Dry Sea
& \tableheader Salt Flat \\

Deep sand            &      & 10\% &      \\
Shallow sand         &      & 20\% &  5\% \\
Deep sand crust      &      & 10\% &      \\
Shallow sand crust   &      & 10\% &      \\
Sand dunes           &      & 20\% &      \\
Light rubble         & 20\% & 10\% & 50\% \\
}

\textbf{Stealth and Detection in Evaporated Seas:} In evaporated seas, the maximum distance at which a \skill{Spot} check for detecting the nearby presence of others can succeed is 2d12$\times$3 meters. In dry seas, this distance is 2d8$\times$3 meters.

Hiding in salt flats is virtually impossible (because of the flat terrain), and silt seas (with the comparative lack of vegetation) aren’t much better. Dry seas, however, provide many more opportunities, if only in ridges and peaks. Evaporated seas have no particular effect on \skill{Listen} or \skill{Move Silently} checks.

\subsubsection{Obsidian Plain}
The side-effect of an epic spell gone bad at the time of the Cleasing Wars, the Obsidian Plain cover a vast area south of the Tablelands. Years after the Wars, the ground beneath has shifted ever so slowly, but enough to crack the surface in many places. What was once a single piece of glass is now a mixture of smooth plains and shard fields. The obsidian glass is fragile and its shards are so sharp, they can slice through the toughest boots. 

\Table{Obsidian Plain Features}{lCC}{
& \multicolumn{2}{c}{\tableheader Obsidian Plain Category}\\
\cmidrule[0.5pt]{2-3}

  \tableheader Terrain Feature
& \tableheader Solid
& \tableheader Shattered \\

Crevasse             & 10\% & 30\% \\
Light rubble         &  5\% & 20\% \\
Razor glass          &      &  5\% \\
}

\textbf{Stealth and Detection in Glass Seas:} In a solid obsidian, the maximum distance at which a Spot check for detecting the nearby presence of others can succeed is 2d10$\times$3 meters. In a shattered obsidian, this distance is reduced to 2d6$\times$3 meters.

Hiding places are rare in solid obsidian, though somewhat more common in shattered obsidian terrain. The occasional patch of dense rubble or razor glass affords a few opportunities for those within to make \skill{Hide} checks. Obsidian Plain has no effect on \skill{Listen} or \skill{Move Silently} checks.

\subsubsection{Wastes Features}

\textbf{Crevasse:} Tectonic shifts and air pockets in the glass create crevasses. They function much like pits or chasms in a dungeon setting. A typical crevasse is 1d4$\times$3 meters deep, 4d12$\times$3 meters long, and 1d8$\times$1.5 meters wide.

A thin layer of solid-looking glass can hide the existence of a dangerous crevasse underneath (25\% chance). This glass sheet is too weak to support any creature larger than Tiny. A character approaching a hidden crevasse at a normal pace is entitled to a DC 10 \skill{Survival} check to spot the danger before stepping in, but charging or running characters don't have a chance to detect the crevasse before falling in. A character falling into a crevasse can attempt a DC 20 Reflex save to catch himself on the edge, in which case he falls prone in a square at the edge of the crevasse. Many crevasses in the Obsidian Plains have a large quantity of broken glass lying on the bottom, dealing an extra 1d6 points of slashing damage to those who fall in.

Because glass conducts and, in some cases, intensifies light, Obsidian Plain crevasses can build up a great deal of heat during the day. The temperature increases by 3 °C every hour that the sun shines on the glass until midday; after midday, the temperature decreases by 3 °C every 2 hours.

Obsidian Plain crevasses can be climbed (up or down) with a DC 22 \skill{Climb} check.

\textbf{Razor Glass:} Shards of broken glass poke up from the ground, slashing any creatures that come into contact with it. Razor glass deals 1d6 points of slashing damage to those who pass through it, but it is fairly easy to identify (DC 10 \skill{Survival} check).


\textbf{Sand:} Coarse, rough, irritating. It gets everywhere in the Tablelands. Most of the times, sand accumulates as a shallow sediment over the bedrock. However, on sandy wastes and dry seas, layers of sand can over three times as deep.

\textit{Deep Sand:} These areas feature a layer of loose sand up to 1 meter deep. It costs Medium or larger creatures 3 squares of movement to move into a square with deep sand. It costs Small or smaller creatures 4 squares of movement to move into a square with deep sand. Tumbling is impossible in deep sand.

\textit{Shallow Sand:} Shallow sand is common in desert areas. Areas of this terrain feature a layer of loose sand about 30 centimeters deep. It costs 2 squares of movement to move into a square with shallow sand, and the DC of Tumble checks in such a square increases by 2.

\textit{Sand Crust:} A sand crust appears as normal solid ground, but it actually conceals a layer of sand. If a creature weighing more than 50 kilograms (including weight of equipment carried) enters a square covered with sand crust, it breaks through to the sand below (deep or shallow). The creature treats the square as the corresponding sand, and it must deal with its effects on movement. These creatures leave a trail of crushed sand crust in their wake, turning the sand crust they pass through into the corresponding sand. Creatures weighing 50 kilograms or less treat sand crust as normal terrain.

\textit{Sand Dunes:} Created by the action of wind on sand, sand dunes function as hills that move. If the wind is strong and consistent, a sand dune can move as much as 30 meters in a year's time. A sand dune can cover hundreds of squares and might reach a height of 300 meters. It slopes gently on the side pointing in the direction of the prevailing wind, but can be extremely steep on the leeward side. Where the wind blows from several different directions, depending on the season, sand dunes take the shape of ``stars'' with three or more points---but where the wind blows steadily in one direction, sand dunes form row upon row of dusty ridges.
