\subsection{Vegetation}
Vegetation can be divided into three major types based on the predominant type of plants: grassland, scrubland, and forest.

Each table describes in general terms how likely it is that a given square has a terrain element in it. It is not necessary to roll for each square; rather, these percentages are presented as a guide for drawing maps. Undergrowth and trees aren't mutually exclusive; it's common for a 1.5-meter square to have both a tree and undergrowth.


\subsubsection{Grasslands}
Grasslands are divided into three categories: pastures filled with low grass and sparse tall grass, prairies filled with tall grass, and savannas that act as transition zones between forests and either other types of grassland or wastes.

\Table{Grassland Features}{lCCC}{
& \multicolumn{3}{c}{\tableheader Grassland Category}\\
\cmidrule[0.5pt]{2-4}

  \tableheader Terrain Feature
& \tableheader Pasture
& \tableheader Prairie
& \tableheader Savannah \\

Typical trees     &  5\% &  5\% & 20\% \\
Massive trees     &      &      &  5\% \\
Light undergrowth & 20\% & 60\% & 45\% \\
Heavy undergrowth &  5\% &  5\% &  5\% \\
}

\textbf{Stealth and Detection in Grasslands:} In pastures or prairies, the maximum distance at which a \skill{Spot} check for detecting the nearby presence of others can succeed is 6d6 $\times$ 12 meters, although the specifics of your map may restrict line of sight. In savannas, this distance is 6d6 $\times$ 6 meters. Grasslands provides no bonuses or penalties on \skill{Listen} and \skill{Spot} checks. Cover and concealment are not uncommon, so a good place of refuge is often nearby, if not right at hand.

\Figure{t}{images/desert-1.png}

\subsubsection{Forests}
Forests can be divided into three categories: sparse, mixed, and dense. An immense forest could have all three categories within its borders, with more sparse terrain at the outer edge of the forest and dense forest at its heart.

\Table{Forest Features}{lCCC}{
& \multicolumn{3}{c}{\tableheader Forest Category}\\
\cmidrule[0.5pt]{2-4}

  \tableheader Terrain Feature
& \tableheader Sparse
& \tableheader Mixed
& \tableheader Dense \\

Typical trees     & 50\% & 70\% & 80\%\\
Massive trees     &      & 10\% & 20\%\\
Light undergrowth & 50\% & 70\% & 50\%\\
Heavy undergrowth &      & 20\% & 50\%\\
}

\textbf{Forest Canopy:} It's common for halflings and other forest dwellers to live on raised platforms far above the surface floor. These wooden platforms generally have rope bridges between them. To get to the treehouses, characters generally ascend the trees' branches (\skill{Climb} DC 15), use rope ladders (\skill{Climb} DC 0), or take pulley elevators (which can be made to rise 30 cm $\times$ Strength check, made each round as a full-round action). Creatures on platforms or branches in a forest canopy are considered to have cover when fighting creatures on the ground, and in mixed or dense forests they have concealment as well.

\textbf{Stealth and Detection in Forests:} In a sparse forest, the maximum distance at which a Spot check for detecting the nearby presence of others can succeed is 3d6 $\times$ 3 meters. In a mixed forest, this distance is 2d8 $\times$ 3 meters, and in a dense forest it is 2d6 $\times$ 3 meters.

Because any square with undergrowth provides concealment, it's usually easy for a creature to use the \skill{Hide} skill in the forest. Logs and massive trees provide cover, which also makes hiding possible.

The background noise in the forest makes \skill{Listen} checks more difficult, increasing the DC of the check by 2 per 3 meters, not 1 (but note that \skill{Move Silently} is also more difficult in undergrowth).


\subsubsection{Scrublands}
Scrublands can be divided into three categories: sparse, mixed, and dense. Unlike forests, scrublands never get denser at their hearts. A big scrubland is either a mosaic of different densities or a single density region.

\Table{Scrubland Features}{lCCC}{
& \multicolumn{3}{c}{\tableheader Scrubland Category}\\
\cmidrule[0.5pt]{2-4}

  \tableheader Terrain Feature
& \tableheader Sparse
& \tableheader Mixed
& \tableheader Dense \\

Typical trees     &  5\% &  5\% &  5\% \\
Light undergrowth & 10\% & 15\% & 20\% \\
Heavy undergrowth & 40\% & 60\% & 80\% \\
}

\textbf{Stealth and Detection in Scrublands:} In a sparse scrubland, the maximum distance at which a Spot check for detecting the nearby presence of others can succeed is 6d6 $\times$ 6 meters. In a mixed scrubland, this distance is 4d6 $\times$ 6 meters, and in a dense scrubland it is 2d6 $\times$ 6 meters.

Because any square with undergrowth provides concealment, it's usually easy for a creature to use the \skill{Hide} skill in the scublands.


\subsubsection{Vegetation Features}
\textbf{Trees:} The most important terrain element in a forest is the trees, obviously. A creature standing in the same square as a tree gains a +2 bonus to Armor Class and a +1 bonus on Reflex saves (these bonuses don't stack with cover bonuses from other sources). The presence of a tree doesn't otherwise affect a creature's fighting space, because it's assumed that the creature is using the tree to its advantage when it can. The trunk of a typical tree has AC 4, hardness 5, and 150 hp. A DC 15 \skill{Climb} check is sufficient to climb a tree.

\textit{Massive Trees:} Forests and savannas have massive trees as well. These trees take up an entire square and provide cover to anyone behind them. They have AC 3, hardness 5, and 600 hp. Like their smaller counterparts, it takes a DC 15 \skill{Climb} check to climb them.


\textbf{Undergrowth:} The tall grasses of the grasslands, small bushes of the scrublands, and vines and roots of forests function like light undergrowth. Particularly thick bushes or dense agglomeration of vines form patches of heavy undergrowth.

Squares with undergrowth are often clustered together. % Undergrowth and trees aren't mutually exclusive; it's common for a 1.5-meter square to have both a tree and undergrowth.

\textit{Light Undergrowth:} A space covered with light undergrowth costs 2 squares of movement to move into, and it provides concealment. Undergrowth increases the DC of \skill{Tumble} and \skill{Move Silently} checks by 2 because the leaves and branches get in the way.

\textit{Heavy Undergrowth:} Heavy undergrowth costs 4 squares of movement to move into, and it provides concealment with a 30\% miss chance (instead of the usual 20\%). It increases the DC of \skill{Tumble} and \skill{Move Silently} checks by 5. Heavy undergrowth is easy to hide in, granting a +5 circumstance bonus on \skill{Hide} checks. Running and charging are impossible.

\textbf{Other Elements:} Fallen logs generally stand about 1 meter high and provide cover just as low walls do. They cost 1.5 meter of movement to cross. Streams are generally 1.5 to 3 meters wide and no more than 1.5 meter deep. Pathways wind through most vegetation, allowing normal movement and providing neither cover nor concealment. These paths are less common in dense forests, but even unexplored forests will have occasional game trails.

\subsubsection{Wildfires (CR 6)}
Most campfire sparks ignite nothing, but if conditions are dry, winds are strong, or the vegetation is dried out and flammable, a wildfire can result. Lightning strikes often set plants afire and start wildfires in this way. Whatever the cause of the fire, travelers can get caught in the conflagration.

A wildfire can be spotted from as far away as 2d6 $\times$ 30 meters by a character who makes a \skill{Spot} check, treating the fire as a Colossal creature (reducing the DC by 16). If all characters fail their \skill{Spot} checks, the fire moves closer to them. They automatically see it when it closes to half the original distance.

Characters who are blinded or otherwise unable to make \skill{Spot} checks can feel the heat of the fire (and thus automatically ``spot'' it) when it is 30 meters away.

The leading edge of a fire (the downwind side) can advance faster than a human can run (assume 36 meters per round for winds of moderate strength). Once a particular portion of the forest is ablaze, it remains so for 2d4 $\times$ 10 minutes before dying to a smoking smolder. Characters overtaken by a wildfire may find the leading edge of the fire advancing away from them faster than they can keep up, trapping them deeper and deeper in its grasp.

Within the bounds of a wildfire, a character faces three dangers: heat damage, catching on fire, and smoke inhalation.

\textbf{Heat Damage:} Getting caught within a wildfire is even worse than being exposed to extreme heat (see Temperature Dangers). Breathing the air causes a character to take 1d6 points of damage per round (no save). In addition, a character must make a Fortitude save every 5 rounds (DC 15, +1 per previous check) or take 1d4 points of nonlethal damage. A character who holds his breath can avoid the lethal damage, but not the nonlethal damage. Those wearing heavy clothing or any sort of armor take a $-4$ penalty on their saving throws. In addition, those wearing metal armor or coming into contact with very hot metal are affected as if by a heat metal spell.

\textbf{Catching on Fire:} Characters engulfed in a wildfire are at risk of catching on fire when the leading edge of the fire overtakes them, and are then at risk once per minute thereafter.

\textbf{Smoke Inhalation:} Wildfires naturally produce a great deal of smoke. A character who breathes heavy smoke must make a Fortitude save each round (DC 15, +1 per previous check) or spend that round choking and coughing. A character who chokes for 2 consecutive rounds takes 1d6 points of nonlethal damage. Also, smoke obscures vision, providing concealment to characters within it.
