\subsection{Water}

\subsubsection{Marshes}
Two categories of marsh exist: relatively dry moors and watery swamps. Both are often bordered by lakes, which effectively are a third category of terrain found in marshes.

The table below describes terrain features found in marshes.

\Table{Marsh Features}{lCC}{
& \multicolumn{2}{c}{\tableheader Marsh Category}\\
\cmidrule[0.5pt]{2-3}

  \tableheader Terrain Feature
& \tableheader Moor
& \tableheader Swamp \\

Shallow bog & 20\% & 40\% \\
Deep bog    &  5\% & 20\% \\
}

\textbf{Stealth and Detection in a Marsh:} In a moor, the maximum distance at which a \skill{Spot} check for detecting the nearby presence of others can succeed is 6d6$\times$3 meters. In a swamp, this distance is 2d8$\times$3 meters.

Vegetation and deep bogs provide plentiful concealment, so it's easy to hide in a marsh.

A marsh imposes no penalties on \skill{Listen} checks, and using the \skill{Move Silently} skill is more difficult in both undergrowth and bogs.

\subsubsection{Underwater}
Underwater is the least hospitable to most PCs, because they can't breathe there. Underwater doesn't offer the variety that land terrain does. The ocean floor holds many marvels, including undersea analogues of any of the terrain elements described earlier in this section. But if characters find themselves in the water because they were bull rushed off the deck of a pirate ship, the tall kelp beds hundreds of feet below them don't matter. Accordingly, these rules simply divide underwater environments into two categories: flowing water (such as streams and rivers) and still water (such as lakes and oceans).

\textbf{Flowing Water:} Large, placid rivers move at only a few kilometers per hour, so they function as still water for most purposes. But some rivers and streams are swifter; anything floating in them moves downstream at a speed of 3 to 12 meters per round. The fastest rapids send swimmers bobbing downstream at 18 to 27 meters per round. Fast rivers are always at least rough water (\skill{Swim} DC 15), and whitewater rapids are stormy water (\skill{Swim} DC 20). If a character is in moving water, move her downstream the indicated distance at the end of her turn. A character trying to maintain her position relative to the riverbank can spend some or all of her turn swimming upstream.

\textit{Swept Away:} Characters swept away by a river moving 18 meters per round or faster must make DC 20 \skill{Swim} checks every round to avoid going under. If a character gets a check result of 5 or more over the minimum necessary, he arrests his motion by catching a rock, tree limb, or bottom snag---he is no longer being carried along by the flow of the water. Escaping the rapids by reaching the bank requires three DC 20 \skill{Swim} checks in a row. Characters arrested by a rock, limb, or snag can't escape under their own power unless they strike out into the water and attempt to swim their way clear. Other characters can rescue them as if they were trapped in quicksand.

\textbf{Still Water:} Lakes and oceans simply require a swim speed or successful \skill{Swim} checks to move through (DC 10 in calm water, DC 15 in rough water, DC 20 in stormy water). Characters need a way to breathe if they're underwater; failing that, they risk drowning. When underwater, characters can move in any direction as if they were flying with perfect maneuverability.

\textbf{Stealth and Detection Underwater:} How far you can see underwater depends on the water's clarity. As a guideline, creatures can see 4d8$\times$3 meters if the water is clear, and 1d8$\times$3 meters if it's murky. Moving water is always murky, unless it's in a particularly large, slow-moving river.

It's hard to find cover or concealment to hide underwater (except along the seafloor). \skill{Listen} and \skill{Move Silently} checks function normally underwater.

\textit{Invisibility:} An invisible creature displaces water and leaves a visible, body-shaped ``bubble'' where the water was displaced. The creature still has concealment (20\% miss chance), but not total concealment (50\% miss chance).

\subsubsection{Water Features}
\textbf{Bogs:} If a square is part of a shallow bog, it has deep mud or standing water of about 1 foot in depth. It costs 2 squares of movement to move into a square with a shallow bog, and the DC of Tumble checks in such a square increases by 2.

A square that is part of a deep bog has roughly 4 feet of standing water. It costs Medium or larger creatures 4 squares of movement to move into a square with a deep bog, or characters can swim if they wish. Small or smaller creatures must swim to move through a deep bog. Tumbling is impossible in a deep bog.

The water in a deep bog provides cover for Medium or larger creatures. Smaller creatures gain improved cover (+8 bonus to AC, +4 bonus on Reflex saves). Medium or larger creatures can crouch as a move action to gain this improved cover. Creatures with this improved cover take a $-10$ penalty on attacks against creatures that aren't underwater.

Deep bog squares are usually clustered together and surrounded by an irregular ring of shallow bog squares.

Both shallow and deep bogs increase the DC of \skill{Move Silently} checks by 2.

\textbf{Quicksand:} Patches of quicksand present a deceptively solid appearance (appearing as undergrowth or open land) that may trap careless characters. A character approaching a patch of quicksand at a normal pace is entitled to a DC 8 Survival check to spot the danger before stepping in, but charging or running characters don't have a chance to detect a hidden bog before blundering in. A typical patch of quicksand is 20 feet in diameter; the momentum of a charging or running character carries him or her 1d2$\times$1.5 meter into the quicksand.

\textit{Effects of Quicksand:} Characters in quicksand must make a DC 10 Swim check every round to simply tread water in place, or a DC 15 Swim check to move 1.5 meter in whatever direction is desired. If a trapped character fails this check by 5 or more, he sinks below the surface and begins to drown whenever he can no longer hold his breath (see the Swim skill description).

Characters below the surface of a bog may swim back to the surface with a successful Swim check (DC 15, +1 per consecutive round of being under the surface).

\textit{Rescue:} Pulling out a character trapped in quicksand can be difficult. A rescuer needs a branch, spear haft, rope, or similar tool that enables him to reach the victim with one end of it. Then he must make a DC 15 Strength check to successfully pull the victim, and the victim must make a DC 10 Strength check to hold onto the branch, pole, or rope. If the victim fails to hold on, he must make a DC 15 Swim check immediately to stay above the surface. If both checks succeed, the victim is pulled 1.5 meter closer to safety.

\textbf{Hedgerows:} Common in moors, hedgerows are tangles of stones, soil, and thorny bushes. Narrow hedgerows function as low walls, and it takes 15 feet of movement to cross them. Wide hedgerows are more than 1.5 meter tall and take up entire squares. They provide total cover, just as a wall does. It takes 4 squares of movement to move through a square with a wide hedgerow; creatures that succeed on a DC 10 Climb check need only 2 squares of movement to move through the square.

\textbf{Other Elements:} Some marshes, particularly swamps, have trees just as forests do, usually clustered in small stands. Paths lead across many marshes, winding to avoid bog areas. As in forests, paths allow normal movement and don't provide the concealment that undergrowth does.
