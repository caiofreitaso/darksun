\subsection{Tactical Movement}
The characteristics of moving in a combat situation: distance to be covered, the possible impediments, and special rules are described below.


\subsubsection{How Far Can Your Character Move?}
Your speed is determined by your race and your armor (see \tabref{Tactical Speed}). Your speed while unarmored is your base land speed.

\textbf{Encumbrance:} A character encumbered by carrying a large amount of gear, treasure, or fallen comrades may move slower than normal.

\textbf{Hampered Movement:} Difficult terrain, obstacles, or poor visibility can hamper movement.

\textbf{Movement in Combat:} Generally, you can move your speed in a round and still do something (take a move action and a standard action).

If you do nothing but move (that is, if you use both of your actions in a round to move your speed), you can move double your speed.

If you spend the entire round running, you can move quadruple your speed. If you do something that requires a full round you can only take a 1.5-meter step.

\textbf{Bonuses to Speed:} A barbarian has a +3 meters bonus to his speed (unless he's wearing heavy armor). In addition, many spells and magic items can affect a character's speed. Always apply any modifiers to a character's speed before adjusting the character's speed based on armor or encumbrance, and remember that multiple bonuses of the same type to a character's speed don't stack.

\Table{Tactical Speed}{y{33mm}CC}{
  \tableheader Race
& \tableheader No Armor or Light Armor
& \tableheader Medium or Heavy Armor \\
Half-giant, thri-kreen    & 12m(8 squares) & 9m(6 squares) \\
Elf, Human, half-elf, mul & 9m(6 squares)  & 6m(4 squares) \\
Dwarf                     & 6m(4 squares)  & 6m(4 squares) \\
Halfling                  & 6m(4 squares)  & 4.5m(3 squares) \\
}

\subsubsection{Measuring Distance}
\textbf{Diagonals:} When measuring distance, the first diagonal counts as 1 square, the second counts as 2 squares, the third counts as 1, the fourth as 2, and so on.

You can't move diagonally past a corner (even by taking a 1.5-meter step). You can move diagonally past a creature, even an opponent.

You can also move diagonally past other impassable obstacles, such as pits.

\textbf{Closest Creature:} When it's important to determine the closest square or creature to a location, if two squares or creatures are equally close, randomly determine which one counts as closest by rolling a die.

\subsubsection{Moving through a Square}
\textbf{Friend:} You can move through a square occupied by a friendly character, unless you are charging. When you move through a square occupied by a friendly character, that character doesn't provide you with cover.

\textbf{Opponent:} You can't move through a square occupied by an opponent, unless the opponent is helpless. You can move through a square occupied by a helpless opponent without penalty. (Some creatures, particularly very large ones, may present an obstacle even when helpless. In such cases, each square you move through counts as 2 squares.)

\textbf{Ending Your Movement:} You can't end your movement in the same square as another creature unless it is helpless.

\textbf{Overrun:} During your movement you can attempt to move through a square occupied by an opponent.

\textbf{Tumbling:} A trained character can attempt to tumble through a square occupied by an opponent (see the Tumble skill).

\textbf{Very Small Creature:} A Fine, Diminutive, or Tiny creature can move into or through an occupied square. The creature provokes attacks of opportunity when doing so.

\textbf{Square Occupied by Creature Three Sizes Larger or Smaller:} Any creature can move through a square occupied by a creature three size categories larger than it is.

A big creature can move through a square occupied by a creature three size categories smaller than it is.

\textbf{Designated Exceptions:} Some creatures break the above rules. A creature that completely fills the squares it occupies cannot be moved past, even with the Tumble skill or similar special abilities.

\BigTablePair{Creature Size and Scale}{l*{3}{Z{12mm}}XX*{3}{Z{14mm}}}{
\tableheader Size Category &
\tableheader Size\footnotemark[1] Modifier &
\tableheader Grapple\footnotemark[2] Modifier &
\tableheader Hide\footnotemark[3] Modifier &
\tableheader Height or Length\footnotemark[4] &
\tableheader Weight\footnotemark[5] &
\tableheader Space\footnotemark[6] &
\tableheader Natural Reach\footnotemark[6] (Tall) &
\tableheader Natural Reach\footnotemark[6] (Long)\\

Fine       & +8   & $-16$ & +16   & 15 cm or less  & 60 g or less        & 15 cm & 0 m   & 0 m   \\
Diminutive & +4   & $-12$ & +12   & 15 cm--30 cm   & 60 g--0.5 kg        & 30 cm & 0 m   & 0 m   \\
Tiny       & +2   & $-8$  & +8    & 30 cm--60 cm   & 0.5 kg--4 kg        & 75 cm & 0 m   & 0 m   \\
Small      & +1   & $-4$  & +4    & 60 cm--1.2 m   & 4 kg--30 kg         & 1.5 m & 1.5 m & 1.5 m \\
Medium     & +0   & +0    & +0    & 1.2 m--2.5 m   & 30 kg--250 kg       & 1.5 m & 1.5 m & 1.5 m \\
Large      & $-1$ & +4    & $-4$  & 2.5 m--5 m   & 250 kg--1 tonne     & 3 m   & 3 m   & 1.5 m \\
Huge       & $-2$ & +8    & $-8$  & 5 m--10 m   & 1 tonne--8 tonnes   & 4.5 m & 4.5 m & 3 m   \\
Gargantuan & $-4$ & +12   & $-12$ & 10 m--20 m  & 8 tonnes--60 tonnes & 6 m   & 6 m   & 4.5 m \\
Colossal   & $-8$ & +16   & $-16$ & 20 m or more & 60 tonnes or more   & 9 m   & 9 m   & 6 m   \\

\BigTableNote{9}{1 A creature's size modifier is applied to it's attack bonus and Armor Class.}\\
\BigTableNote{9}{2 See the Grapple special attack.}\\
\BigTableNote{9}{3 See the \skill{Hide} skill.}\\
\BigTableNote{9}{4 Biped's height, quadruped's body length (nose to base of tail)}\\
\BigTableNote{9}{5 Assumes that the creature is roughly as dense as a regular animal. A creature made of stone will weigh considerably more. A gaseous creature will weigh much less.}\\
\BigTableNote{9}{6 These values are typical for creatures of the indicated size. Some exceptions exist.}\\
}
\subsubsection{Terrain and Obstacles}
\textbf{Difficult Terrain:} Difficult terrain hampers movement. Each square of difficult terrain counts as 2 squares of movement. (Each diagonal move into a difficult terrain square counts as 3 squares.) You can't run or charge across difficult terrain.

If you occupy squares with different kinds of terrain, you can move only as fast as the most difficult terrain you occupy will allow.

Flying and incorporeal creatures are not hampered by difficult terrain.

\textbf{Obstacles:} Like difficult terrain, obstacles can hamper movement. If an obstacle hampers movement but doesn't completely block it each obstructed square or obstacle between squares counts as 2 squares of movement. You must pay this cost to cross the barrier, in addition to the cost to move into the square on the other side. If you don't have sufficient movement to cross the barrier and move into the square on the other side, you can't cross the barrier. Some obstacles may also require a skill check to cross.

On the other hand, some obstacles block movement entirely. A character can't move through a blocking obstacle.

Flying and incorporeal creatures can avoid most obstacles.

\textbf{Squeezing:} In some cases, you may have to squeeze into or through an area that isn't as wide as the space you take up. You can squeeze through or into a space that is at least half as wide as your normal space. Each move into or through a narrow space counts as if it were 2 squares, and while squeezed in a narrow space you take a $-4$ penalty on attack rolls and a $-4$ penalty to AC.

When a Large creature (which normally takes up four squares) squeezes into a space that's one square wide, the creature's miniature figure occupies two squares, centered on the line between the two squares. For a bigger creature, center the creature likewise in the area it squeezes into.

A creature can squeeze past an opponent while moving but it can't end its movement in an occupied square.

To squeeze through or into a space less than half your space's width, you must use the \skill{Escape Artist} skill. You can't attack while using \skill{Escape Artist} to squeeze through or into a narrow space, you take a $-4$ penalty to AC, and you lose any Dexterity bonus to AC.

\subsubsection{Special Movement Rules}
These rules cover special movement situations.

\textbf{Accidentally Ending Movement in an Illegal Space:} Sometimes a character ends its movement while moving through a space where it's not allowed to stop. When that happens, put your miniature in the last legal position you occupied, or the closest legal position, if there's a legal position that's closer.

\textbf{Double Movement Cost:} When your movement is hampered in some way, your movement usually costs double. For example, each square of movement through difficult terrain counts as 2 squares, and each diagonal move through such terrain counts as 3 squares (just as two diagonal moves normally do).

If movement cost is doubled twice, then each square counts as 4 squares (or as 6 squares if moving diagonally). If movement cost is doubled three times, then each square counts as 8 squares (12 if diagonal) and so on. This is an exception to the general rule that two doublings are equivalent to a tripling.

\textbf{Minimum Movement:} Despite penalties to movement, you can take a full-round action to move 1.5 meter (1 square) in any direction, even diagonally. (This rule doesn't allow you to move through impassable terrain or to move when all movement is prohibited.) Such movement provokes attacks of opportunity as normal (despite the distance covered, this move isn't a 1.5-meter step).