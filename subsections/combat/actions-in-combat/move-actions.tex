\subsection{Move Actions}

\Table{Move Actions}{LZ{14mm}}{
\tableheader Action & \tableheader Attack of Opportunity\footnotemark[1] \\
Move & Yes \\
Control a frightened mount & Yes \\
Direct or redirect an active spell & No \\
Draw a weapon\footnotemark[2] & No \\
Load a hand crossbow or light crossbow & Yes \\
Open or close a door & No \\
Mount a horse or dismount & No \\
Move a heavy object & Yes \\
Pick up an item & Yes \\
Sheathe a weapon & Yes \\
Stand up from prone & Yes \\
Ready or loose a shield\footnotemark[2] & No \\
Retrieve a stored item & Yes \\
\TableNote{2}{1 Regardless of the action, if you move out of a threatened square, you usually provoke an attack of opportunity. This column indicates whether the action itself, not moving, provokes an attack of opportunity.}\\
\TableNote{2}{2 If you have a base attack bonus of +1 or higher, you can combine one of these actions with a regular move. If you have the \feat{Two-Weapon Fighting} feat, you can draw two light or one-handed weapons in the time it would normally take you to draw one.}\\
}

With the exception of specific movement-related skills, most move actions don't require a check.

\subsubsection{Move}
The simplest move action is moving your speed. If you take this kind of move action during your turn, you can't also take a 1.5-meter step.

Many nonstandard modes of movement are covered under this category, including climbing (up to one-quarter of your speed) and swimming (up to one-quarter of your speed).

\textbf{Accelerated Climbing:} You can climb one-half your speed as a move action by accepting a $-5$ penalty on your \skill{Climb} check.

\textbf{Crawling:} You can crawl 5 feet as a move action. Crawling incurs attacks of opportunity from any attackers who threaten you at any point of your crawl.

\subsubsection{Draw or Sheathe a Weapon}
Drawing a weapon so that you can use it in combat, or putting it away so that you have a free hand, requires a move action. This action also applies to weapon-like objects carried in easy reach, such as wands. If your weapon or weapon-like object is stored in a pack or otherwise out of easy reach, treat this action as retrieving a stored item.

If you have a base attack bonus of +1 or higher, you may draw a weapon as a free action combined with a regular move. If you have the \feat{Two-Weapon Fighting} feat, you can draw two light or one-handed weapons in the time it would normally take you to draw one.

Drawing ammunition for use with a ranged weapon (such as arrows, bolts, sling bullets, or shuriken) is a free action.

\subsubsection{Ready or Loose a Shield}
Strapping a shield to your arm to gain its shield bonus to your AC, or unstrapping and dropping a shield so you can use your shield hand for another purpose, requires a move action. If you have a base attack bonus of +1 or higher, you can ready or loose a shield as a free action combined with a regular move.

Dropping a carried (but not worn) shield is a free action.

\subsubsection{Manipulate an Item}
In most cases, moving or manipulating an item is a move action.

This includes retrieving or putting away a stored item, picking up an item, moving a heavy object, and opening a door. Examples of this kind of action, along with whether they incur an attack of opportunity, are given in \tabref{Move Actions}.

\subsubsection{Direct or Redirect a Spell}
Some spells allow you to redirect the effect to new targets or areas after you cast the spell. Redirecting a spell requires a move action and does not provoke attacks of opportunity or require concentration.

\subsubsection{Stand Up}
Standing up from a prone position requires a move action and provokes attacks of opportunity.

\subsubsection{Mount/Dismount a Steed}
Mounting or dismounting from a steed requires a move action.

\textbf{Fast Mount or Dismount:} You can mount or dismount as a free action with a DC 20 \skill{Ride} check (your armor check penalty, if any, applies to this check). If you fail the check, mounting or dismounting is a move action instead. (You can't attempt a fast mount or fast dismount unless you can perform the mount or dismount as a move action in the current round.)

