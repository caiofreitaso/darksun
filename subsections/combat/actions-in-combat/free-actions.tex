\Table{Free Actions}{LZ{14mm}}{
\tableheader Action & \tableheader Attack of Opportunity\footnotemark[1] \\
Cease concentration on a spell & No \\
Drop an item & No \\
Drop to the floor & No \\
Prepare spell components to cast a spell\footnotemark[2] & No \\
Speak & No \\
\TableNote{2}{1 Regardless of the action, if you move out of a threatened square, you usually provoke an attack of opportunity. This column indicates whether the action itself, not moving, provokes an attack of opportunity.}\\
\TableNote{2}{2 Unless the component is an extremely large or awkward item.}\\
}

\subsection{Free Actions}
Free actions don't take any time at all, though there may be limits to the number of free actions you can perform in a turn. Free actions rarely incur attacks of opportunity. Some common free actions are described below.

\subsubsection{Drop an Item}
Dropping an item in your space or into an adjacent square is a free action.

\subsubsection{Drop Prone}
Dropping to a prone position in your space is a free action.

\subsubsection{Speak}
In general, speaking is a free action that you can perform even when it isn't your turn. Speaking more than few sentences is generally beyond the limit of a free action.

\subsubsection{Cease Concentration on Spell}
You can stop concentrating on an active spell as a free action.

