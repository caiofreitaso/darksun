\Table{Miscellaneous Actions}{LZ{14mm}}{
\tableheader No Action & \tableheader Attack of Opportunity\footnotemark[1] \\
Delay & No \\
1.5-meter step & No \\
\cmidrule[0pt]{1-2}
\tableheader Action Type Varies & \tableheader Attack of Opportunity\footnotemark[1] \\
Disarm\footnotemark[2] & Yes \\
Grapple\footnotemark[2] & Yes \\
Trip an opponent\footnotemark[2] & Yes \\
Use feat\footnotemark[3] & Varies \\

\TableNote{2}{1 Regardless of the action, if you move out of a threatened square, you usually provoke an attack of opportunity. This column indicates whether the action itself, not moving, provokes an attack of opportunity.}\\
\TableNote{2}{2 These attack forms substitute for a melee attack, not an action. As melee attacks, they can be used once in an attack or charge action, one or more times in a full attack action, or even as an attack of opportunity.}\\
\TableNote{2}{3 The description of a feat defines its effect.}\\
}

\subsection{Miscellaneous Actions}
\subsubsection{Take 1.5-Meter Step}
You can move 1.5 meter in any round when you don't perform any other kind of movement. Taking this 1.5-meter step never provokes an attack of opportunity. You can't take more than one 1.5-meter step in a round, and you can't take a 1.5-meter step in the same round when you move any distance.

You can take a 1.5-meter step before, during, or after your other actions in the round.

You can only take a 1.5-meter step if your movement isn't hampered by difficult terrain or darkness. Any creature with a speed of 1.5 meter or less can't take a 1.5-meter step, since moving even 1.5 meter requires a move action for such a slow creature.

You may not take a 1.5-meter step using a form of movement for which you do not have a listed speed.

\subsubsection{Use Feat}
Certain feats let you take special actions in combat. Other feats do not require actions themselves, but they give you a bonus when attempting something you can already do. Some feats are not meant to be used within the framework of combat. The individual feat descriptions tell you what you need to know about them.

\subsubsection{Use Skill}
Most skill uses are standard actions, but some might be move actions, full-round actions, free actions, or something else entirely.

The individual skill descriptions tell you what sorts of actions are required to perform skills.