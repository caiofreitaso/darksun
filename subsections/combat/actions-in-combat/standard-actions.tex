\subsection{Standard Actions}

\Table{Standard Actions}{LZ{14mm}}{
\tableheader Action & \tableheader Attack of Opportunity\footnotemark[1]\\
Attack (melee) & No\\
Attack (unarmed) & Yes\\
Attack (ranged) & Yes\\
Activate a magic item & \\
~ Command word & No \\
~ Manipulation & Yes \\
~ ~ Apply oil & Yes \\
~ ~ Drink potion & Yes \\
~ Spell completion (such as scroll) & Yes \\
~ Spell trigger (such as a wand or staff) & No \\
~ Use-activated & No \\
Activate a psionic tattoo & Yes\\
Aid another & Maybe\footnotemark[2]\\
Bull rush & Yes\\
Cast a spell (1 standard action casting time) & Yes\\
Concentrate to maintain an active spell & No\\
Concentrate to maintain an active power & No\\
Demoralize opponent (see \skill{Intimidate} skill) & No\\
Dismiss a spell & No\\
Dismiss a power & No\\
Draw a hidden weapon (see \skill{Sleight of Hand} skill) & No\\
% Drink a potion or apply an oil & Yes\\
Escape a grapple or pin & No\\
Feint & No\\
Light a torch with a tindertwig & Yes\\
Lower spell resistance & No\\
Make a dying friend stable (see \skill{Heal} skill) & Yes\\
Manifest a power (1 standard action manifesting time) & Yes\\
Overrun & No\\
Read a scroll & Yes\\
Ready (triggers a standard action) & No\\
Sunder a weapon (attack) & Yes\\
Sunder an object (attack) & Maybe\footnotemark[3]\\
Suppress psionic tattoo & Yes\\
Total defense & No\\
Turn or rebuke undead & No\\
Use extraordinary ability & No\\
Use skill that takes 1 standard action & Usually\\
Use spell-like ability & Yes\\
Use psi-like ability & Yes\\
Use supernatural ability & No\\

\TableNote{2}{1 Regardless of the action, if you move out of a threatened square, you usually provoke an attack of opportunity. This column indicates whether the action itself, not moving, provokes an attack of opportunity.}\\
\TableNote{2}{2 If you aid someone performing an action that would normally provoke an attack of opportunity, then the act of aiding another provokes an attack of opportunity as well.}\\
\TableNote{2}{3 If the object is being held, carried, or worn by a creature, yes. If not, no.}\\
}

\subsubsection{Attack}
Making an attack is a standard action.

\textbf{Melee Attacks:} With a normal melee weapon, you can strike any opponent within 1.5 meter. (Opponents within 1.5 meter are considered adjacent to you.) Some melee weapons have reach, as indicated in their descriptions (see \chapref{Equipment}). With a typical reach weapon, you can strike opponents 3 meters away, but you can't strike adjacent foes (those within 1.5 meter).

\textbf{Unarmed Attacks:} Striking for damage with punches, kicks, and head butts is much like attacking with a melee weapon, except for the following:

\textbf{Attacks of Opportunity:} Attacking unarmed provokes an attack of opportunity from the character you attack, provided she is armed. The attack of opportunity comes before your attack. An unarmed attack does not provoke attacks of opportunity from other foes nor does it provoke an attack of opportunity from an unarmed foe.

An unarmed character can't take attacks of opportunity (but see ``Armed'' Unarmed Attacks, below).

\textit{``Armed'' Unarmed Attacks:} Sometimes a character's or creature's unarmed attack counts as an armed attack. A character with the \feat{Improved Unarmed Strike} feat, a spellcaster delivering a touch attack spell, and a creature with natural physical weapons all count as being armed.

Note that being armed counts for both offense and defense (the character can make attacks of opportunity)

\textit{Unarmed Strike Damage:} An unarmed strike from a Medium character deals 1d3 points of damage (plus your Strength modifier, as normal). A Small character's unarmed strike deals 1d2 points of damage, while a Large character's unarmed strike deals 1d4 points of damage. All damage from unarmed strikes is nonlethal damage. Unarmed strikes count as light weapons (for purposes of two-weapon attack penalties and so on).

\textit{Dealing Lethal Damage:} You can specify that your unarmed strike will deal lethal damage before you make your attack roll, but you take a $-4$ penalty on your attack roll. If you have the \feat{Improved Unarmed Strike} feat, you can deal lethal damage with an unarmed strike without taking a penalty on the attack roll.

\textbf{Ranged Attacks:} With a ranged weapon, you can shoot or throw at any target that is within the weapon's maximum range and in line of sight. The maximum range for a thrown weapon is five range increments. For projectile weapons, it is ten range increments. Some ranged weapons have shorter maximum ranges, as specified in their descriptions.

\textbf{Attack Rolls:} An attack roll represents your attempts to strike your opponent.

Your attack roll is 1d20 + your attack bonus with the weapon you're using. If the result is at least as high as the target's AC, you hit and deal damage.

\textbf{Automatic Misses and Hits:} A natural 1 (the d20 comes up 1) on the attack roll is always a miss. A natural 20 (the d20 comes up 20) is always a hit. A natural 20 is also a threat---a possible critical hit.

\textbf{Damage Rolls:} If the attack roll result equals or exceeds the target's AC, the attack hits and you deal damage. Roll the appropriate damage for your weapon. Damage is deducted from the target's current hit points.

\textbf{Multiple Attacks:} A character who can make more than one attack per round must use the full attack action in order to get more than one attack.

\textbf{Shooting or Throwing into a Melee:} If you shoot or throw a ranged weapon at a target engaged in melee with a friendly character, you take a $-4$ penalty on your attack roll. Two characters are engaged in melee if they are enemies of each other and either threatens the other. (An unconscious or otherwise immobilized character is not considered engaged unless he is actually being attacked.)

If your target (or the part of your target you're aiming at, if it's a big target) is at least 3 meters away from the nearest friendly character, you can avoid the $-4$ penalty, even if the creature you're aiming at is engaged in melee with a friendly character.

\textit{Precise Shot:} If you have the \feat{Precise Shot} feat you don't take this penalty.

\textbf{Fighting Defensively as a Standard Action:} You can choose to fight defensively when attacking. If you do so, you take a $-4$ penalty on all attacks in a round to gain a +2 dodge bonus to AC for the same round. See also: Fighting Defensively as a Full-Round Action.

\textbf{Critical Hits:} When you make an attack roll and get a natural 20 (the d20 shows 20), you hit regardless of your target's Armor Class, and you have scored a threat. The hit might be a critical hit (or ``crit''). To find out if it's a critical hit, you immediately make a critical roll---another attack roll with all the same modifiers as the attack roll you just made. If the critical roll also results in a hit against the target's AC, your original hit is a critical hit. (The critical roll just needs to hit to give you a crit. It doesn't need to come up 20 again.) If the critical roll is a miss, then your hit is just a regular hit.

A critical hit means that you roll your damage more than once, with all your usual bonuses, and add the rolls together. Unless otherwise specified, the threat range for a critical hit on an attack roll is 20, and the multiplier is $\times$2.

\textit{Exception:} Extra damage dice over and above a weapon's normal damage is not multiplied when you score a critical hit.

\textit{Increased Threat Range:} Sometimes your threat range is greater than 20. That is, you can score a threat on a lower number. In such cases, a roll of lower than 20 is not an automatic hit. Any attack roll that doesn't result in a hit is not a threat.

\textit{Increased Critical Multiplier:} Some weapons deal better than double damage on a critical hit.

\textit{Spells and Critical Hits:} A spell that requires an attack roll can score a critical hit. A spell attack that requires no attack roll cannot score a critical hit.
\subsubsection{Cast a Spell}
Most spells require 1 standard action to cast. You can cast such a spell either before or after you take a move action.

\textit{Note:} You retain your Dexterity bonus to AC while casting.

\textbf{Spell Components:} To cast a spell with a verbal (V) component, your character must speak in a firm voice. If you're gagged or in the area of a silence spell, you can't cast such a spell. A spellcaster who has been deafened has a 20\% chance to spoil any spell he tries to cast if that spell has a verbal component.

To cast a spell with a somatic (S) component, you must gesture freely with at least one hand. You can't cast a spell of this type while bound, grappling, or with both your hands full or occupied.

To cast a spell with a material (M), focus (F), or divine focus (DF) component, you have to have the proper materials, as described by the spell. Unless these materials are elaborate preparing these materials is a free action. For material components and focuses whose costs are not listed, you can assume that you have them if you have your spell component pouch.

Some spells have an experience point (XP) component and entail an experience point cost to you. No spell or power can restore the lost XP. You cannot spend so much XP that you lose a level, so you cannot cast the spell unless you have enough XP to spare. However, you may, on gaining enough XP to achieve a new level, immediately spend the XP on casting the spell rather than keeping it to advance a level. The XP are expended when you cast the spell, whether or not the casting succeeds.

\textbf{Concentration:} You must concentrate to cast a spell. If you can't concentrate you can't cast a spell. If you start casting a spell but something interferes with your concentration you must make a \skill{Concentration} check or lose the spell. The check's DC depends on what is threatening your concentration (see the \skill{Concentration} skill). If you fail, the spell fizzles with no effect. If you prepare spells, it is lost from preparation. If you cast at will, it counts against your daily limit of spells even though you did not cast it successfully.

\textbf{Concentrating to Maintain a Spell:} Some spells require continued concentration to keep them going. Concentrating to maintain a spell is a standard action that doesn't provoke an attack of opportunity. Anything that could break your concentration when casting a spell can keep you from concentrating to maintain a spell. If your concentration breaks, the spell ends.

\textbf{Casting Time:} Most spells have a casting time of 1 standard action. A spell cast in this manner immediately takes effect.

\textbf{Attacks of Opportunity:} Generally, if you cast a spell, you provoke attacks of opportunity from threatening enemies. If you take damage from an attack of opportunity, you must make a \skill{Concentration} check (DC 10 + points of damage taken + spell level) or lose the spell. Spells that require only a swift action or immediate action to cast don't provoke attacks of opportunity.

\textbf{Casting on the Defensive:} Casting a spell while on the defensive does not provoke an attack of opportunity. It does, however, require a \skill{Concentration} check (DC 15 + spell level) to pull off. Failure means that you lose the spell.

\textbf{Touch Spells in Combat:} Many spells have a range of touch. To use these spells, you cast the spell and then touch the subject, either in the same round or any time later. In the same round that you cast the spell, you may also touch (or attempt to touch) the target. You may take your move before casting the spell, after touching the target, or between casting the spell and touching the target. You can automatically touch one friend or use the spell on yourself, but to touch an opponent, you must succeed on an attack roll.

\textit{Touch Attacks:} Touching an opponent with a touch spell is considered to be an armed attack and therefore does not provoke attacks of opportunity. However, the act of casting a spell does provoke an attack of opportunity. Touch attacks come in two types: melee touch attacks and ranged touch attacks. You can score critical hits with either type of attack. Your opponent's AC against a touch attack does not include any armor bonus, shield bonus, or natural armor bonus. His size modifier, Dexterity modifier, and deflection bonus (if any) all apply normally.

\textit{Holding the Charge:} If you don't discharge the spell in the round when you cast the spell, you can hold the discharge of the spell (hold the charge) indefinitely. You can continue to make touch attacks round after round. You can touch one friend as a standard action or up to six friends as a full-round action. If you touch anything or anyone while holding a charge, even unintentionally, the spell discharges. If you cast another spell, the touch spell dissipates. Alternatively, you may make a normal unarmed attack (or an attack with a natural weapon) while holding a charge. In this case, you aren't considered armed and you provoke attacks of opportunity as normal for the attack. (If your unarmed attack or natural weapon attack doesn't provoke attacks of opportunity, neither does this attack.) If the attack hits, you deal normal damage for your unarmed attack or natural weapon and the spell discharges. If the attack misses, you are still holding the charge.

\textbf{Dismiss a Spell:} Dismissing an active spell is a standard action that doesn't provoke attacks of opportunity.
\subsubsection{Manifest a Power}
Except when noted here, manifesting a power follows the same rules as casting a spell, such as provoking attacks of opportunity, concentrating, manifesting on the defensive, making touch attacks, and holding the charge of a power.

\textbf{Power Cost:} To manifest a power, you must pay power points, which count against your daily total. You can manifest the same power multiple times if you have points left to pay for it.

Some powers allow you to spend more than their base cost to achieve an improved effect, or augment the power. The maximum number of points you can spend on a power (for any reason) is equal to your manifester level.

On the same line that the power point cost of a power is indicated, the power's experience point cost, if any, is noted. Particularly powerful effects entail an experience point cost to you. No spell or power can restore XP lost in this manner. You cannot spend so much XP that you lose a level, so you cannot manifest a power with an XP cost unless you have enough XP to spare. However, you can, on gaining enough XP to attain a new level, use those XP for manifesting a power rather than keeping them and advancing a level. The XP are expended when you manifest the power, whether or not the manifestation succeeds.

\subsubsection{Activate Magic Item}
Many magic items don't need to be activated. However, certain magic items need to be activated, especially potions, scrolls, wands, rods, and staffs. Activating a magic item is a standard action (unless the item description indicates otherwise).

\textbf{Spell Completion Items:} Activating a spell completion item is the equivalent of casting a spell. It requires concentration and provokes attacks of opportunity. You lose the spell if your concentration is broken, and you can attempt to activate the item while on the defensive, as with casting a spell.

\textbf{Spell Trigger, Command Word, or Use-Activated Items:} Activating any of these kinds of items does not require concentration and does not provoke attacks of opportunity.

\subsubsection{Suppressing Psionic Tattoo}
You can suppress and reactivate a psionic tattoo as a standard action. Suppressing or activating psionic tattoos provoke attacks of opportunity. The character must use one standard action for each tattoo.

\subsubsection{Use Special Ability}
Using a special ability is usually a standard action, but whether it is a standard action, a full-round action, or not an action at all is defined by the ability.

\textbf{Spell-Like Abilities:} Using a spell-like ability works like casting a spell in that it requires concentration and provokes attacks of opportunity. Spell-like abilities can be disrupted. If your concentration is broken, the attempt to use the ability fails, but the attempt counts as if you had used the ability. The casting time of a spell-like ability is 1 standard action, unless the ability description notes otherwise.

\textit{Using a Spell-Like Ability on the Defensive:} You may attempt to use a spell-like ability on the defensive, just as with casting a spell. If the \skill{Concentration} check (DC 15 + spell level) fails, you can't use the ability, but the attempt counts as if you had used the ability.

\textbf{Supernatural Abilities:} Using a supernatural ability is usually a standard action (unless defined otherwise by the ability's description). Its use cannot be disrupted, does not require concentration, and does not provoke attacks of opportunity.

\textbf{Extraordinary Abilities:} Using an extraordinary ability is usually not an action because most extraordinary abilities automatically happen in a reactive fashion. Those extraordinary abilities that are actions are usually standard actions that cannot be disrupted, do not require concentration, and do not provoke attacks of opportunity.

\subsubsection{Total Defense}
You can defend yourself as a standard action. You get a +4 dodge bonus to your AC for 1 round. Your AC improves at the start of this action. You can't combine total defense with fighting defensively or with the benefit of the \feat{Combat Expertise} feat (since both of those require you to declare an attack or full attack). You can't make attacks of opportunity while using total defense.

\subsubsection{Start/Complete Full-Round Action}
The ``start full-round action'' standard action lets you start undertaking a full-round action, which you can complete in the following round by using another standard action. You can't use this action to start or complete a full attack, charge, run, or withdraw.