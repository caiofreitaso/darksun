\subsubsection{Attack}
Making an attack is a standard action.

\textbf{Melee Attacks:} With a normal melee weapon, you can strike any opponent within 1.5 meter. (Opponents within 1.5 meter are considered adjacent to you.) Some melee weapons have reach, as indicated in their descriptions (see \chapref{Equipment}). With a typical reach weapon, you can strike opponents 3 meters away, but you can't strike adjacent foes (those within 1.5 meter).

\textbf{Unarmed Attacks:} Striking for damage with punches, kicks, and head butts is much like attacking with a melee weapon, except for the following:

\textbf{Attacks of Opportunity:} Attacking unarmed provokes an attack of opportunity from the character you attack, provided she is armed. The attack of opportunity comes before your attack. An unarmed attack does not provoke attacks of opportunity from other foes nor does it provoke an attack of opportunity from an unarmed foe.

An unarmed character can't take attacks of opportunity (but see ``Armed'' Unarmed Attacks, below).

\textit{``Armed'' Unarmed Attacks:} Sometimes a character's or creature's unarmed attack counts as an armed attack. A character with the \feat{Improved Unarmed Strike} feat, a spellcaster delivering a touch attack spell, and a creature with natural physical weapons all count as being armed.

Note that being armed counts for both offense and defense (the character can make attacks of opportunity)

\textit{Unarmed Strike Damage:} An unarmed strike from a Medium character deals 1d3 points of damage (plus your Strength modifier, as normal). A Small character's unarmed strike deals 1d2 points of damage, while a Large character's unarmed strike deals 1d4 points of damage. All damage from unarmed strikes is nonlethal damage. Unarmed strikes count as light weapons (for purposes of two-weapon attack penalties and so on).

\textit{Dealing Lethal Damage:} You can specify that your unarmed strike will deal lethal damage before you make your attack roll, but you take a $-4$ penalty on your attack roll. If you have the \feat{Improved Unarmed Strike} feat, you can deal lethal damage with an unarmed strike without taking a penalty on the attack roll.

\textbf{Ranged Attacks:} With a ranged weapon, you can shoot or throw at any target that is within the weapon's maximum range and in line of sight. The maximum range for a thrown weapon is five range increments. For projectile weapons, it is ten range increments. Some ranged weapons have shorter maximum ranges, as specified in their descriptions.

\textbf{Attack Rolls:} An attack roll represents your attempts to strike your opponent.

Your attack roll is 1d20 + your attack bonus with the weapon you're using. If the result is at least as high as the target's AC, you hit and deal damage.

\textbf{Automatic Misses and Hits:} A natural 1 (the d20 comes up 1) on the attack roll is always a miss. A natural 20 (the d20 comes up 20) is always a hit. A natural 20 is also a threat---a possible critical hit.

\textbf{Damage Rolls:} If the attack roll result equals or exceeds the target's AC, the attack hits and you deal damage. Roll the appropriate damage for your weapon. Damage is deducted from the target's current hit points.

\textbf{Multiple Attacks:} A character who can make more than one attack per round must use the full attack action in order to get more than one attack.

\textbf{Shooting or Throwing into a Melee:} If you shoot or throw a ranged weapon at a target engaged in melee with a friendly character, you take a $-4$ penalty on your attack roll. Two characters are engaged in melee if they are enemies of each other and either threatens the other. (An unconscious or otherwise immobilized character is not considered engaged unless he is actually being attacked.)

If your target (or the part of your target you're aiming at, if it's a big target) is at least 3 meters away from the nearest friendly character, you can avoid the $-4$ penalty, even if the creature you're aiming at is engaged in melee with a friendly character.

\textit{Precise Shot:} If you have the \feat{Precise Shot} feat you don't take this penalty.

\textbf{Fighting Defensively as a Standard Action:} You can choose to fight defensively when attacking. If you do so, you take a $-4$ penalty on all attacks in a round to gain a +2 dodge bonus to AC for the same round. See also: Fighting Defensively as a Full-Round Action.

\textbf{Critical Hits:} When you make an attack roll and get a natural 20 (the d20 shows 20), you hit regardless of your target's Armor Class, and you have scored a threat. The hit might be a critical hit (or ``crit''). To find out if it's a critical hit, you immediately make a critical roll---another attack roll with all the same modifiers as the attack roll you just made. If the critical roll also results in a hit against the target's AC, your original hit is a critical hit. (The critical roll just needs to hit to give you a crit. It doesn't need to come up 20 again.) If the critical roll is a miss, then your hit is just a regular hit.

A critical hit means that you roll your damage more than once, with all your usual bonuses, and add the rolls together. Unless otherwise specified, the threat range for a critical hit on an attack roll is 20, and the multiplier is $\times$2.

\textit{Exception:} Extra damage dice over and above a weapon's normal damage is not multiplied when you score a critical hit.

\textit{Increased Threat Range:} Sometimes your threat range is greater than 20. That is, you can score a threat on a lower number. In such cases, a roll of lower than 20 is not an automatic hit. Any attack roll that doesn't result in a hit is not a threat.

\textit{Increased Critical Multiplier:} Some weapons deal better than double damage on a critical hit.

\textit{Spells and Critical Hits:} A spell that requires an attack roll can score a critical hit. A spell attack that requires no attack roll cannot score a critical hit.