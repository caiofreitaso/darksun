\subsubsection{Manifest a Power}
Except when noted here, manifesting a power follows the same rules as casting a spell, such as provoking attacks of opportunity, concentrating, manifesting on the defensive, making touch attacks, and holding the charge of a power.

\textbf{Power Cost:} To manifest a power, you must pay power points, which count against your daily total. You can manifest the same power multiple times if you have points left to pay for it.

% Some powers allow you to spend more than their base cost to achieve an improved effect, or augment the power. The maximum number of points you can spend on a power (for any reason) is equal to your manifester level.

On the same line that the power point cost of a power is indicated, the power's experience point cost, if any, is noted. Particularly powerful effects entail an experience point cost to you. No spell or power can restore XP lost in this manner. You cannot spend so much XP that you lose a level, so you cannot manifest a power with an XP cost unless you have enough XP to spare. However, you can, on gaining enough XP to attain a new level, use those XP for manifesting a power rather than keeping them and advancing a level. The XP are expended when you manifest the power, whether or not the manifestation succeeds.

\textbf{Maintenance Cost:} Some powers are instantaneous, while others may last longer. These powers may last rounds, minutes, or hours. When first manifested, the power lasts for the one time step (round, minute, or hour). If you want to make the power last longer, you must pay power points after each time step in the maintenance cost line.

Paying maintenance cost does not require an action, but it must be done at the beginning of your turn---before any action is taken.
