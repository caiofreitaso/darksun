\subsubsection{Cast a Spell}
Most spells require 1 standard action to cast. You can cast such a spell either before or after you take a move action.

\textit{Note:} You retain your Dexterity bonus to AC while casting.

\textbf{Spell Components:} To cast a spell with a verbal (V) component, your character must speak in a firm voice. If you're gagged or in the area of a silence spell, you can't cast such a spell. A spellcaster who has been deafened has a 20\% chance to spoil any spell he tries to cast if that spell has a verbal component.

To cast a spell with a somatic (S) component, you must gesture freely with at least one hand. You can't cast a spell of this type while bound, grappling, or with both your hands full or occupied.

To cast a spell with a material (M), focus (F), or divine focus (DF) component, you have to have the proper materials, as described by the spell. Unless these materials are elaborate preparing these materials is a free action. For material components and focuses whose costs are not listed, you can assume that you have them if you have your spell component pouch.

Some spells have an experience point (XP) component and entail an experience point cost to you. No spell or power can restore the lost XP. You cannot spend so much XP that you lose a level, so you cannot cast the spell unless you have enough XP to spare. However, you may, on gaining enough XP to achieve a new level, immediately spend the XP on casting the spell rather than keeping it to advance a level. The XP are expended when you cast the spell, whether or not the casting succeeds.

\textbf{Concentration:} You must concentrate to cast a spell. If you can't concentrate you can't cast a spell. If you start casting a spell but something interferes with your concentration you must make a \skill{Concentration} check or lose the spell. The check's DC depends on what is threatening your concentration (see the \skill{Concentration} skill). If you fail, the spell fizzles with no effect. If you prepare spells, it is lost from preparation. If you cast at will, it counts against your daily limit of spells even though you did not cast it successfully.

\textbf{Concentrating to Maintain a Spell:} Some spells require continued concentration to keep them going. Concentrating to maintain a spell is a standard action that doesn't provoke an attack of opportunity. Anything that could break your concentration when casting a spell can keep you from concentrating to maintain a spell. If your concentration breaks, the spell ends.

\textbf{Casting Time:} Most spells have a casting time of 1 standard action. A spell cast in this manner immediately takes effect.

\textbf{Attacks of Opportunity:} Generally, if you cast a spell, you provoke attacks of opportunity from threatening enemies. If you take damage from an attack of opportunity, you must make a \skill{Concentration} check (DC 10 + points of damage taken + spell level) or lose the spell. Spells that require only a swift action or immediate action to cast don't provoke attacks of opportunity.

\textbf{Casting on the Defensive:} Casting a spell while on the defensive does not provoke an attack of opportunity. It does, however, require a \skill{Concentration} check (DC 15 + spell level) to pull off. Failure means that you lose the spell.

\textbf{Touch Spells in Combat:} Many spells have a range of touch. To use these spells, you cast the spell and then touch the subject, either in the same round or any time later. In the same round that you cast the spell, you may also touch (or attempt to touch) the target. You may take your move before casting the spell, after touching the target, or between casting the spell and touching the target. You can automatically touch one friend or use the spell on yourself, but to touch an opponent, you must succeed on an attack roll.

\textit{Touch Attacks:} Touching an opponent with a touch spell is considered to be an armed attack and therefore does not provoke attacks of opportunity. However, the act of casting a spell does provoke an attack of opportunity. Touch attacks come in two types: melee touch attacks and ranged touch attacks. You can score critical hits with either type of attack. Your opponent's AC against a touch attack does not include any armor bonus, shield bonus, or natural armor bonus. His size modifier, Dexterity modifier, and deflection bonus (if any) all apply normally.

\textit{Holding the Charge:} If you don't discharge the spell in the round when you cast the spell, you can hold the discharge of the spell (hold the charge) indefinitely. You can continue to make touch attacks round after round. You can touch one friend as a standard action or up to six friends as a full-round action. If you touch anything or anyone while holding a charge, even unintentionally, the spell discharges. If you cast another spell, the touch spell dissipates. Alternatively, you may make a normal unarmed attack (or an attack with a natural weapon) while holding a charge. In this case, you aren't considered armed and you provoke attacks of opportunity as normal for the attack. (If your unarmed attack or natural weapon attack doesn't provoke attacks of opportunity, neither does this attack.) If the attack hits, you deal normal damage for your unarmed attack or natural weapon and the spell discharges. If the attack misses, you are still holding the charge.

\textbf{Dismiss a Spell:} Dismissing an active spell is a standard action that doesn't provoke attacks of opportunity.