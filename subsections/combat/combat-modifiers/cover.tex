\subsection{Cover}
To determine whether your target has cover from your ranged attack, choose a corner of your square. If any line from this corner to any corner of the target's square passes through a square or border that blocks line of effect or provides cover, or through a square occupied by a creature, the target has cover (+4 to AC).

When making a melee attack against an adjacent target, your target has cover if any line from your square to the target's square goes through a wall (including a low wall). When making a melee attack against a target that isn't adjacent to you (such as with a reach weapon), use the rules for determining cover from ranged attacks.

\textbf{Low Obstacles and Cover:} A low obstacle (such as a wall no higher than half your height) provides cover, but only to creatures within 9 meters (6 squares) of it. The attacker can ignore the cover if he's closer to the obstacle than his target.

\textbf{Cover and Attacks of Opportunity:} You can't execute an attack of opportunity against an opponent with cover relative to you.

\textbf{Cover and Reflex Saves:} Cover grants you a +2 bonus on Reflex saves against attacks that originate or burst out from a point on the other side of the cover from you. Note that spread effects can extend around corners and thus negate this cover bonus.

\textbf{Cover and Hide Checks:} You can use cover to make a \skill{Hide} check. Without cover, you usually need concealment to make a \skill{Hide} check.

\textbf{Soft Cover:} Creatures, even your enemies, can provide you with cover against ranged attacks, giving you a +4 bonus to AC. However, such soft cover provides no bonus on Reflex saves, nor does soft cover allow you to make a \skill{Hide} check.

\textbf{Big Creatures and Cover:} Any creature with a space larger than 1.5 meter (1 square) determines cover against melee attacks slightly differently than smaller creatures do. Such a creature can choose any square that it occupies to determine if an opponent has cover against its melee attacks. Similarly, when making a melee attack against such a creature, you can pick any of the squares it occupies to determine if it has cover against you.

\textbf{Total Cover:} If you don't have line of effect to your target he is considered to have total cover from you. You can't make an attack against a target that has total cover.

\textbf{Varying Degrees of Cover:} In some cases, cover may provide a greater bonus to AC and Reflex saves. In such situations the normal cover bonuses to AC and Reflex saves can be doubled (to +8 and +4, respectively). A creature with this improved cover effectively gains improved evasion against any attack to which the Reflex save bonus applies. Furthermore, improved cover provides a +10 bonus on \skill{Hide} checks.