\subsection{Concealment}
To determine whether your target has concealment from your ranged attack, choose a corner of your square. If any line from this corner to any corner of the target's square passes through a square or border that provides concealment, the target has concealment.

When making a melee attack against an adjacent target, your target has concealment if his space is entirely within an effect that grants concealment. When making a melee attack against a target that isn't adjacent to you use the rules for determining concealment from ranged attacks.

In addition, some magical effects provide concealment against all attacks, regardless of whether any intervening concealment exists.

\textbf{Concealment Miss Chance:} Concealment gives the subject of a successful attack a 20\% chance that the attacker missed because of the concealment. If the attacker hits, the defender must make a miss chance percentile roll to avoid being struck. Multiple concealment conditions do not stack.

\textbf{Concealment and Hide Checks:} You can use concealment to make a \skill{Hide} check. Without concealment, you usually need cover to make a \skill{Hide} check.

\textbf{Total Concealment:} If you have line of effect to a target but not line of sight he is considered to have total concealment from you. You can't attack an opponent that has total concealment, though you can attack into a square that you think he occupies. A successful attack into a square occupied by an enemy with total concealment has a 50\% miss chance (instead of the normal 20\% miss chance for an opponent with concealment).

You can't execute an attack of opportunity against an opponent with total concealment, even if you know what square or squares the opponent occupies.

\textbf{Ignoring Concealment:} Concealment isn't always effective. A shadowy area or darkness doesn't provide any concealment against an opponent with darkvision. Characters with low-light vision can see clearly for a greater distance with the same light source than other characters. Although invisibility provides total concealment, sighted opponents may still make Spot checks to notice the location of an invisible character. An invisible character gains a +20 bonus on \skill{Hide} checks if moving, or a +40 bonus on \skill{Hide} checks when not moving (even though opponents can't see you, they might be able to figure out where you are from other visual clues).

\textbf{Varying Degrees of Concealment:} Certain situations may provide more or less than typical concealment, and modify the miss chance accordingly.