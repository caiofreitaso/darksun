\subsection{Disarm}
As a melee attack, you may attempt to disarm your opponent. If you do so with a weapon, you knock the opponent's weapon out of his hands and to the ground. If you attempt the disarm while unarmed, you end up with the weapon in your hand.

If you're attempting to disarm a melee weapon, follow the steps outlined here. If the item you are attempting to disarm isn't a melee weapon the defender may still oppose you with an attack roll, but takes a penalty and can't attempt to disarm you in return if your attempt fails.

\begin{enumerate*}
\item \textbf{Attack of Opportunity.} You provoke an attack of opportunity from the target you are trying to disarm. (If you have the \feat{Improved Disarm} feat, you don't incur an attack of opportunity for making a disarm attempt.) If the defender's attack of opportunity deals any damage, your disarm attempt fails.

\item \textbf{Opposed Rolls.} You and the defender make opposed attack rolls with your respective weapons. The wielder of a two-handed weapon on a disarm attempt gets advantage on this roll, and the wielder of a light weapon get disadvantage on the roll. (An unarmed strike is considered a light weapon, so you always get disadvantage when trying to disarm an opponent by using an unarmed strike.) If the combatants are of different sizes, the larger combatant gets a bonus on the attack roll of +4 per difference in size category. If the targeted item isn't a melee weapon, the defender gets disadvantage on the roll.

\item \textbf{Consequences.} If you beat the defender, the defender is disarmed. If you attempted the disarm action unarmed, you now have the weapon. If you were armed, the defender's weapon is on the ground in the defender's square.
\end{enumerate*}

If you fail on the disarm attempt, the defender may immediately react and attempt to disarm you with the same sort of opposed melee attack roll. His attempt does not provoke an attack of opportunity from you. If he fails his disarm attempt, you do not subsequently get a free disarm attempt against him.

\textit{Note:} A defender wearing spiked gauntlets can't be disarmed. A defender using a weapon attached to a locked gauntlet gets a +10 bonus to resist being disarmed.

\subsubsection{Grabbing Items}
You can use a disarm action to snatch an item worn by the target. If you want to have the item in your hand, the disarm must be made as an unarmed attack.

If the item is poorly secured or otherwise easy to snatch or cut away the attacker gets a +4 bonus. Unlike on a normal disarm attempt, failing the attempt doesn't allow the defender to attempt to disarm you. This otherwise functions identically to a disarm attempt, as noted above.

You can't snatch an item that is well secured unless you have pinned the wearer (see Grapple). Even then, the defender gains a +4 bonus on his roll to resist the attempt.