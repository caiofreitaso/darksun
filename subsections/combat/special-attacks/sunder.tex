\subsection{Sunder}
You can use a melee attack with a slashing or bludgeoning weapon to strike a weapon or shield that your opponent is holding. If you're attempting to sunder a weapon or shield, follow the steps outlined here. (Attacking held objects other than weapons or shields is covered below.)

\Table{Common Armor, Weapon, and Shield Hardness and Hit Points}{X cc}{
\tableheader Weapon or Shield & \tableheader Hardness & \tableheader HP\footnotemark[1]\\
\TableSubheader{Light blade} &&\\
~ wood & 5 & 1\\
~ bone & 6 & 1\\
~ stone & 8 & 1\\
~ metal & 10 & 2\\
\TableSubheader{One-handed blade} &&\\
~ wood & 5 & 2\\
~ bone & 6 & 2\\
~ stone & 8 & 3\\
~ metal & 10 & 5\\
\TableSubheader{Two-handed blade} &&\\
~ wood & 5 & 4\\
~ bone & 6 & 4\\
~ stone & 8 & 5\\
~ metal & 10 & 10\\
\TableSubheader{Light weapon} &&\\
~ wood-hafted & 5 & 2\\
~ bone-hafted & 6 & 2\\
~ stone-hafted & 8 & 3\\
~ metal-hafted & 10 & 10\\
\TableSubheader{One-handed weapon} &&\\
~ wood-hafted & 5 & 5\\
~ bone-hafted & 6 & 5\\
~ stone-hafted & 8 & 8\\
~ metal-hafted & 10 & 20\\
\TableSubheader{Two-handed weapon} &&\\
~ wood-hafted & 5 & 10\\
~ bone-hafted & 6 & 10\\
~ stone-hafted & 8 & 15\\
Projectile weapon & 5 & 5\\
Armor & $\star$ & armor bonus $\times5$\\
Buckler & 10 & 5\\
\TableSubheader{Light shield} &&\\
~ wooden & 5 & 7\\
~ steel & 10 & 10\\
\TableSubheader{Heavy shield} &&\\
~ wooden & 5 & 15\\
~ steel & 10 & 20\\
Tower shield & 5 & 20\\

\TableNote{3}{1 The hp value given is for Medium armor, weapons, and shields. Divide by 2 for each size category of the item smaller than Medium, or multiply it by 2 for each size category larger than Medium.}\\
\TableNote{3}{$\star$ Varies by material; see \tabref{Substance Hardness and Hit Points}.}\\
}

\begin{enumerate*}
\item \textbf{Attack of Opportunity.} You provoke an attack of opportunity from the target whose weapon or shield you are trying to sunder. (If you have the \feat{Improved Sunder} feat, you don't incur an attack of opportunity for making the attempt.)

\item \textbf{Opposed Rolls.} You and the defender make opposed attack rolls with your respective weapons. The wielder of a two-handed weapon on a sunder attempt gets advantage on this roll, and the wielder of a light weapon gets disadvantage on the roll. If the combatants are of different sizes, the larger combatant gets a bonus on the attack roll of +4 per difference in size category.

\item \textbf{Consequences.} If you beat the defender, roll damage and deal it to the weapon or shield. See \tabref{Common Armor, Weapon, and Shield Hardness and Hit Points} to determine how much damage you must deal to destroy the weapon or shield.
\end{enumerate*}

If you fail the sunder attempt, you don't deal any damage.

\textbf{Sundering a Carried or Worn Object:} You don't use an opposed attack roll to damage a carried or worn object. Instead, just make an attack roll against the object's AC. A carried or worn object's AC is equal to 10 + its size modifier + the Dexterity modifier of the carrying or wearing character. Attacking a carried or worn object provokes an attack of opportunity just as attacking a held object does. To attempt to snatch away an item worn by a defender rather than damage it, see Disarm. You can't sunder armor worn by another character.