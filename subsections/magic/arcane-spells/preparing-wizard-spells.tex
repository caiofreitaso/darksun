\subsection{Preparing Wizard Spells}
A wizard's level limits the number of spells she can prepare and cast. Her high Intelligence score might allow her to prepare a few extra spells. She can prepare the same spell more than once, but each preparation counts as one spell toward her daily limit. To prepare a spell the wizard must have an Intelligence score of at least 10 + the spell's level.

\textbf{Rest:} To prepare her daily spells, a wizard must first sleep for 8 hours. The wizard does not have to slumber for every minute of the time, but she must refrain from movement, combat, spellcasting, skill use, conversation, or any other fairly demanding physical or mental task during the rest period. If her rest is interrupted, each interruption adds 1 hour to the total amount of time she has to rest in order to clear her mind, and she must have at least 1 hour of uninterrupted rest immediately prior to preparing her spells. If the character does not need to sleep for some reason, she still must have 8 hours of restful calm before preparing any spells.

\textbf{Recent Casting Limit/Rest Interruptions:} If a wizard has cast spells recently, the drain on her resources reduces her capacity to prepare new spells. When she prepares spells for the coming day, all the spells she has cast within the last 8 hours count against her daily limit.

\textbf{Preparation Environment:} To prepare any spell, a wizard must have enough peace, quiet, and comfort to allow for proper concentration. The wizard's surroundings need not be luxurious, but they must be free from overt distractions. Exposure to inclement weather prevents the necessary concentration, as does any injury or failed saving throw the character might experience while studying. Wizards also must have access to their spellbooks to study from and sufficient light to read them by. There is one major exception: A wizard can prepare a read magic spell even without a spellbook.

\textbf{Spell Preparation Time:} After resting, a wizard must study her spellbook to prepare any spells that day. If she wants to prepare all her spells, the process takes 1 hour. Preparing some smaller portion of her daily capacity takes a proportionally smaller amount of time, but always at least 15 minutes, the minimum time required to achieve the proper mental state.

\textbf{Spell Selection and Preparation:} Until she prepares spells from her spellbook, the only spells a wizard has available to cast are the ones that she already had prepared from the previous day and has not yet used. During the study period, she chooses which spells to prepare. If a wizard already has spells prepared (from the previous day) that she has not cast, she can abandon some or all of them to make room for new spells.

When preparing spells for the day, a wizard can leave some of these spell slots open. Later during that day, she can repeat the preparation process as often as she likes, time and circumstances permitting. During these extra sessions of preparation, the wizard can fill these unused spell slots. She cannot, however, abandon a previously prepared spell to replace it with another one or fill a slot that is empty because she has cast a spell in the meantime. That sort of preparation requires a mind fresh from rest. Like the first session of the day, this preparation takes at least 15 minutes, and it takes longer if the wizard prepares more than one-quarter of her spells.

\textbf{Spell Slots:} The various character class tables show how many spells of each level a character can cast per day. These openings for daily spells are called spell slots. A spellcaster always has the option to fill a higher-level spell slot with a lower-level spell. A spellcaster who lacks a high enough ability score to cast spells that would otherwise be his or her due still gets the slots but must fill them with spells of lower level.

\textbf{Prepared Spell Retention:} Once a wizard prepares a spell, it remains in her mind as a nearly cast spell until she uses the prescribed components to complete and trigger it or until she abandons it. Certain other events, such as the effects of magic items or special attacks from monsters, can wipe a prepared spell from a character's mind.

\textbf{Death and Prepared Spell Retention:} If a spellcaster dies, all prepared spells stored in his or her mind are wiped away. Potent magic (such as \spell{raise dead}, \spell{resurrection}, or \spell{true resurrection}) can recover the lost energy when it recovers the character.