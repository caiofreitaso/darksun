\subsection{Arcane Magical Writings}
To record an arcane spell in written form, a character uses complex notation that describes the magical forces involved in the spell. The writer uses the same system no matter what her native language or culture. However, each character uses the system in her own way. Another person's magical writing remains incomprehensible to even the most powerful wizard until she takes time to study and decipher it.

To decipher an arcane magical writing (such as a single spell in written form in another's spellbook or on a scroll), a character must make a \skill{Spellcraft} check (DC 20 + the spell's level). If the skill check fails, the character cannot attempt to read that particular spell again until the next day. A read magic spell automatically deciphers a magical writing without a skill check. If the person who created the magical writing is on hand to help the reader, success is also automatic.

Once a character deciphers a particular magical writing, she does not need to decipher it again. Deciphering a magical writing allows the reader to identify the spell and gives some idea of its effects (as explained in the spell description). If the magical writing was a scroll and the reader can cast arcane spells, she can attempt to use the scroll.

\subsubsection{Wizard Spells and Borrowed Spellbooks}
A wizard can use a borrowed spellbook to prepare a spell she already knows and has recorded in her own spellbook, but preparation success is not assured. First, the wizard must decipher the writing in the book (see Arcane Magical Writings, above). Once a spell from another spellcaster's book is deciphered, the reader must make a \skill{Spellcraft} check (DC 15 + spell's level) to prepare the spell. If the check succeeds, the wizard can prepare the spell. She must repeat the check to prepare the spell again, no matter how many times she has prepared it before. If the check fails, she cannot try to prepare the spell from the same source again until the next day. (However, as explained above, she does not need to repeat a check to decipher the writing.)

\textbf{Adding Spells to a Wizard's Spellbook}: Wizards can add new spells to their spellbooks through several methods. If a wizard has chosen to specialize in a school of magic, she can learn spells only from schools whose spells she can cast.

\textbf{Spells Gained at a New Level}: Wizards perform a certain amount of spell research between adventures. Each time a character attains a new wizard level, she gains two spells of her choice to add to her spellbook. The two free spells must be of spell levels she can cast. If she has chosen to specialize in a school of magic, one of the two free spells must be from her specialty school.

\textbf{Spells Copied from Another's Spellbook or a Scroll}: A wizard can also add a spell to her book whenever she encounters one on a magic scroll or in another wizard's spellbook. No matter what the spell's source, the wizard must first decipher the magical writing (see Arcane Magical Writings, above). Next, she must spend a day studying the spell. At the end of the day, she must make a \skill{Spellcraft} check (DC 15 + spell's level). A wizard who has specialized in a school of spells gains a +2 bonus on the \skill{Spellcraft} check if the new spell is from her specialty school. She cannot, however, learn any spells from her prohibited schools. If the check succeeds, the wizard understands the spell and can copy it into her spellbook (see Writing a New Spell into a Spellbook, below). The process leaves a spellbook that was copied from unharmed, but a spell successfully copied from a magic scroll disappears from the parchment.

If the check fails, the wizard cannot understand or copy the spell. She cannot attempt to learn or copy that spell again until she gains another rank in \skill{Spellcraft}. A spell that was being copied from a scroll does not vanish from the scroll.

In most cases, wizards charge a fee for the privilege of copying spells from their spellbooks. This fee is usually equal to the spell's level $\times$ 50 cp.

\textbf{Independent Research}: A wizard also can research a spell independently, duplicating an existing spell or creating an entirely new one.

\subsubsection{Writing a New Spell into a Spellbook}
Once a wizard understands a new spell, she can record it into her spellbook.

\textbf{Time}: The process takes 24 hours, regardless of the spell's level.

\textbf{Space in the Spellbook}: A spell takes up one page of the spellbook per spell level. Even a 0-level spell (cantrip) takes one page. A spellbook has one hundred pages.

\textbf{Materials and Costs}: Materials for writing the spell cost 100 cp per page.

Note that a wizard does not have to pay these costs in time or gold for the spells she gains for free at each new level.

\subsubsection{Replacing and Copying Spellbooks}
A wizard can use the procedure for learning a spell to reconstruct a lost spellbook. If she already has a particular spell prepared, she can write it directly into a new book at a cost of 100 cp per page (as noted in Writing a New Spell into a Spellbook, above). The process wipes the prepared spell from her mind, just as casting it would. If she does not have the spell prepared, she can prepare it from a borrowed spellbook and then write it into a new book.

Duplicating an existing spellbook uses the same procedure as replacing it, but the task is much easier. The time requirement and cost per page are halved.

\subsubsection{Selling a Spellbook}
Captured spellbooks can be sold for a cp amount equal to one-half the cost of purchasing and inscribing the spells within (that is, one-half of 100 cp per page of spells). A spellbook entirely filled with spells (that is, with one hundred pages of spells inscribed in it) is worth 5,000 cp.

\subsubsection{Disguising a Spellbook}
Athasian wizards conceal their ``spellbooks'' from templars, rival wizards and others with ability to discern them for what they are. Spellbooks take many forms, including animal hides, stone and clay tablets, bone staves, knotted giant hair and necklaces of colored beads. Wizards use different, often personalized codes and systems for organizing their spells.

The \skill{Disguise} skill masks a spellbook's true nature. Someone inspecting the spellbook must win an opposed \skill{Spellcraft} vs. \skill{Disguise} check to identify it as such. Every time a new spell is added, a spellbook must be disguised anew. Unless in a hurry, a wizard normally takes 20 on this check.