\subsection{Saving Throw}
Usually a harmful spell allows a target to make a saving throw to avoid some or all of the effect. The Saving Throw entry in a spell description defines which type of saving throw the spell allows and describes how saving throws against the spell work.

\textbf{Negates}: The spell has no effect on a subject that makes a successful saving throw.

\textbf{Partial}: The spell causes an effect on its subject. A successful saving throw means that some lesser effect occurs.

\textbf{Half}: The spell deals damage, and a successful saving throw halves the damage taken (round down).

\textbf{None}: No saving throw is allowed.

\textbf{Disbelief}: A successful save lets the subject ignore the effect.

\textbf{(object)}: The spell can be cast on objects, which receive saving throws only if they are magical or if they are attended (held, worn, grasped, or the like) by a creature resisting the spell, in which case the object uses the creature's saving throw bonus unless its own bonus is greater. (This notation does not mean that a spell can be cast only on objects. Some spells of this sort can be cast on creatures or objects.) A magic item's saving throw bonuses are each equal to 2 + one-half the item's caster level.

\textbf{(harmless)}: The spell is usually beneficial, not harmful, but a targeted creature can attempt a saving throw if it desires.

\textbf{Saving Throw Difficulty Class}: A saving throw against your spell has a DC of 10 + the level of the spell + your bonus for the relevant ability (Intelligence for a wizard, Charisma for a sorcerer or bard, or Wisdom for a cleric, druid, paladin, or ranger). A spell's level can vary depending on your class. Always use the spell level applicable to your class.

\textbf{Succeeding on a Saving Throw}: A creature that successfully saves against a spell that has no obvious physical effects feels a hostile force or a tingle, but cannot deduce the exact nature of the attack. Likewise, if a creature's saving throw succeeds against a targeted spell you sense that the spell has failed. You do not sense when creatures succeed on saves against effect and area spells.

\textbf{Automatic Failures and Successes}: A natural 1 (the d20 comes up 1) on a saving throw is always a failure, and the spell may cause damage to exposed items (see Items Surviving after a Saving Throw, below). A natural 20 (the d20 comes up 20) is always a success.

\textbf{Voluntarily Giving up a Saving Throw}: A creature can voluntarily forego a saving throw and willingly accept a spell's result. Even a character with a special resistance to magic can suppress this quality.

\textbf{Items Surviving after a Saving Throw}: Unless the descriptive text for the spell specifies otherwise, all items carried or worn by a creature are assumed to survive a magical attack. If a creature rolls a natural 1 on its saving throw against the effect, however, an exposed item is harmed (if the attack can harm objects). Refer to \tabref{Items Affected by Magical Attacks}. Determine which four objects carried or worn by the creature are most likely to be affected and roll randomly among them. The randomly determined item must make a saving throw against the attack form and take whatever damage the attack deal.

If an item is not carried or worn and is not magical, it does not get a saving throw. It simply is dealt the appropriate damage.

\Table{Items Affected by Magical Attacks}{lX}{
\tableheader Order\footnotemark[1] & \tableheader Item\\
1st & Shield\\
2nd & Armor\\
3rd & Magic helmet, hat, or headband\\
4th & Item in hand (including weapon, wand, or the like)\\
5th & Magic cloak\\
6th & Stowed or sheathed weapon\\
7th & Magic bracers\\
8th & Magic clothing\\
9th & Magic jewelry (including rings)\\
10th & Anything else\\

\TableNote{2}{1 In order of most likely to least likely to be affected.}\\
}