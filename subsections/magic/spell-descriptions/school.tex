\subsection{School (Subschool)}
Beneath the spell name is a line giving the school of magic (and the subschool, if appropriate) that the spell belongs to.

Almost every spell belongs to one of eight schools of magic. A school of magic is a group of related spells that work in similar ways. A small number of spells (\spell{arcane mark}, \spell{limited wish}, \spell{permanency}, \spell{prestidigitation}, and \spell{wish}) are universal, belonging to no school.

\textbf{Abjuration:} Abjurations are protective spells. They create physical or magical barriers, negate magical or physical abilities, harm trespassers, or even banish the subject of the spell to another plane of existence.

If one abjuration spell is active within 3 meters of another for 24 hours or more, the magical fields interfere with each other and create barely visible energy fluctuations. The DC to find such spells with the Search skill drops by 4.

If an abjuration creates a barrier that keeps certain types of creatures at bay, that barrier cannot be used to push away those creatures. If you force the barrier against such a creature, you feel a discernible pressure against the barrier. If you continue to apply pressure, you end the spell.

\textbf{Conjuration:} Each conjuration spell belongs to one of five subschools. Conjurations bring manifestations of objects, creatures, or some form of energy to you (the summoning subschool), actually transport creatures from another plane of existence to your plane (calling), heal (healing), transport creatures or objects over great distances (teleportation), or create objects or effects on the spot (creation). Creatures you conjure usually, but not always, obey your commands.

A creature or object brought into being or transported to your location by a conjuration spell cannot appear inside another creature or object, nor can it appear floating in an empty space. It must arrive in an open location on a surface capable of supporting it.

The creature or object must appear within the spell's range, but it does not have to remain within the range.

\textit{Calling:} A calling spell transports a creature from another plane to the plane you are on. The spell grants the creature the one-time ability to return to its plane of origin, although the spell may limit the circumstances under which this is possible. Creatures who are called actually die when they are killed; they do not disappear and reform, as do those brought by a summoning spell (see below). The duration of a calling spell is instantaneous, which means that the called creature can't be dispelled.

\textit{Creation:} A creation spell manipulates matter to create an object or creature in the place the spellcaster designates (subject to the limits noted above). If the spell has a duration other than instantaneous, magic holds the creation together, and when the spell ends, the conjured creature or object vanishes without a trace. If the spell has an instantaneous duration, the created object or creature is merely assembled through magic. It lasts indefinitely and does not depend on magic for its existence.

\textit{Healing:} Certain divine conjurations heal creatures or even bring them back to life.

\textit{Summoning:} A summoning spell instantly brings a creature or object to a place you designate. When the spell ends or is dispelled, a summoned creature is instantly sent back to where it came from, but a summoned object is not sent back unless the spell description specifically indicates this. A summoned creature also goes away if it is killed or if its hit points drop to 0 or lower. It is not really dead. It takes 24 hours for the creature to reform, during which time it can't be summoned again.

When the spell that summoned a creature ends and the creature disappears, all the spells it has cast expire. A summoned creature cannot use any innate summoning abilities it may have, and it refuses to cast any spells that would cost it XP, or to use any spell-like abilities that would cost XP if they were spells.

\textit{Teleportation:} A teleportation spell transports one or more creatures or objects a great distance. The most powerful of these spells can cross planar boundaries. Unlike summoning spells, the transportation is (unless otherwise noted) one-way and not dispellable.

Teleportation is instantaneous travel through the Astral Plane. Anything that blocks astral travel also blocks teleportation.

\textbf{Divination:} Divination spells enable you to learn secrets long forgotten, to predict the future, to find hidden things, and to foil deceptive spells.

Many divination spells have cone-shaped areas. These move with you and extend in the direction you look. The cone defines the area that you can sweep each round. If you study the same area for multiple rounds, you can often gain additional information, as noted in the descriptive text for the spell.

\textit{Scrying:} A scrying spell creates an invisible magical sensor that sends you information. Unless noted otherwise, the sensor has the same powers of sensory acuity that you possess. This level of acuity includes any spells or effects that target you, but not spells or effects that emanate from you. However, the sensor is treated as a separate, independent sensory organ of yours, and thus it functions normally even if you have been blinded, deafened, or otherwise suffered sensory impairment.

Any creature with an Intelligence score of 12 or higher can notice the sensor by making a DC 20 Intelligence check. The sensor can be dispelled as if it were an active spell.

Lead sheeting or magical protection blocks a scrying spell, and you sense that the spell is so blocked.

\textbf{Enchantment:} Enchantment spells affect the minds of others, influencing or controlling their behavior.

All enchantments are mind-affecting spells. Two types of enchantment spells grant you influence over a subject creature.

\textit{Charm:} A charm spell changes how the subject views you, typically making it see you as a good friend.

\textit{Compulsion:} A compulsion spell forces the subject to act in some manner or changes the way her mind works. Some compulsion spells determine the subject's actions or the effects on the subject, some compulsion spells allow you to determine the subject's actions when you cast the spell, and others give you ongoing control over the subject.

\textbf{Evocation:} Evocation spells manipulate energy or tap an unseen source of power to produce a desired end. In effect, they create something out of nothing. Many of these spells produce spectacular effects, and evocation spells can deal large amounts of damage.

\textbf{Illusion:} Illusion spells deceive the senses or minds of others. They cause people to see things that are not there, not see things that are there, hear phantom noises, or remember things that never happened.

\textit{Figment:} A figment spell creates a false sensation. Those who perceive the figment perceive the same thing, not their own slightly different versions of the figment. (It is not a personalized mental impression.) Figments cannot make something seem to be something else. A figment that includes audible effects cannot duplicate intelligible speech unless the spell description specifically says it can. If intelligible speech is possible, it must be in a language you can speak. If you try to duplicate a language you cannot speak, the image produces gibberish. Likewise, you cannot make a visual copy of something unless you know what it looks like.

Because figments and glamers (see below) are unreal, they cannot produce real effects the way that other types of illusions can. They cannot cause damage to objects or creatures, support weight, provide nutrition, or provide protection from the elements. Consequently, these spells are useful for confounding or delaying foes, but useless for attacking them directly.

A figment's AC is equal to 10 + its size modifier.

\textit{Glamer:} A glamer spell changes a subject's sensory qualities, making it look, feel, taste, smell, or sound like something else, or even seem to disappear.

\textit{Pattern:} Like a figment, a pattern spell creates an image that others can see, but a pattern also affects the minds of those who see it or are caught in it. All patterns are mind-affecting spells.

\textit{Phantasm:} A phantasm spell creates a mental image that usually only the caster and the subject (or subjects) of the spell can perceive. This impression is totally in the minds of the subjects. It is a personalized mental impression. (It's all in their heads and not a fake picture or something that they actually see.) Third parties viewing or studying the scene don't notice the phantasm. All phantasms are mind-affecting spells.

\textit{Shadow:} A shadow spell creates something that is partially real from extradimensional energy. Such illusions can have real effects. Damage dealt by a shadow illusion is real.

\textit{Saving Throws and Illusions (Disbelief):} Creatures encountering an illusion usually do not receive saving throws to recognize it as illusory until they study it carefully or interact with it in some fashion.

A successful saving throw against an illusion reveals it to be false, but a figment or phantasm remains as a translucent outline.

A failed saving throw indicates that a character fails to notice something is amiss. A character faced with proof that an illusion isn't real needs no saving throw. If any viewer successfully disbelieves an illusion and communicates this fact to others, each such viewer gains a saving throw with a +4 bonus.

\textbf{Necromancy:} Necromancy spells manipulate the power of death, unlife, and the life force. Spells involving undead creatures make up a large part of this school.

\textbf{Transmutation:} Transmutation spells change the properties of some creature, thing, or condition.