\subsection{Concentration}
To cast a spell, you must concentrate. If something interrupts your concentration while you're casting, you must make a \skill{Concentration} check or lose the spell. The more distracting the interruption and the higher the level of the spell you are trying to cast, the higher the DC is. If you fail the check, you lose the spell just as if you had cast it to no effect.

\textbf{Injury:} If while trying to cast a spell you take damage, you must make a \skill{Concentration} check (DC 10 + points of damage taken + the level of the spell you're casting). If you fail the check, you lose the spell without effect. The interrupting event strikes during spellcasting if it comes between when you start and when you complete a spell (for a spell with a casting time of 1 full round or more) or if it comes in response to your casting the spell (such as an attack of opportunity provoked by the spell or a contingent attack, such as a readied action).

If you are taking continuous damage half the damage is considered to take place while you are casting a spell. You must make a \skill{Concentration} check (DC 10 + \onehalf the damage that the continuous source last dealt + the level of the spell you're casting). If the last damage dealt was the last damage that the effect could deal then the damage is over, and it does not distract you.

Repeated damage does not count as continuous damage.

\textbf{Spell:} If you are affected by a spell while attempting to cast a spell of your own, you must make a \skill{Concentration} check or lose the spell you are casting. If the spell affecting you deals damage, the DC is 10 + points of damage + the level of the spell you're casting.

If the spell interferes with you or distracts you in some other way, the DC is the spell's saving throw DC + the level of the spell you're casting. For a spell with no saving throw, it's the DC that the spell's saving throw would have if a save were allowed.

\textbf{Grappling or Pinned:} The only spells you can cast while grappling or pinned are those without somatic components and whose material components (if any) you have in hand. Even so, you must make a \skill{Concentration} check (DC 20 + the level of the spell you're casting) or lose the spell.

\textbf{Vigorous Motion:} If you are riding on a moving mount, taking a bouncy ride in a wagon, on a small boat in rough water, below-decks in a storm-tossed ship, or simply being jostled in a similar fashion, you must make a \skill{Concentration} check (DC 10 + the level of the spell you're casting) or lose the spell.

\textbf{Violent Motion:} If you are on a galloping horse, taking a very rough ride in a wagon, on a small boat in rapids or in a storm, on deck in a storm-tossed ship, or being tossed roughly about in a similar fashion, you must make a \skill{Concentration} check (DC 15 + the level of the spell you're casting) or lose the spell.

\textbf{Violent Weather:} You must make a \skill{Concentration} check if you try to cast a spell in violent weather. If you are in a high wind carrying blinding rain or sleet, the DC is 5 + the level of the spell you're casting. If you are in wind-driven hail, dust, or debris, the DC is 10 + the level of the spell you're casting. In either case, you lose the spell if you fail the \skill{Concentration} check. If the weather is caused by a spell, use the rules in the Spell subsection above.

\textbf{Casting Defensively:} If you want to cast a spell without provoking any attacks of opportunity, you must make a \skill{Concentration} check (DC 15 + the level of the spell you're casting) to succeed. You lose the spell if you fail.

\textbf{Entangled:} If you want to cast a spell while entangled in a net or by a tanglefoot bag or while you're affected by a spell with similar effects, you must make a DC 15 \skill{Concentration} check to cast the spell. You lose the spell if you fail.