\subsection{Sensory Effects}
The stealthy preserver crouches low behind the stone walls of the ruins, fumbling through his belt pouch for his material components, peering cautiously around for signs of the approaching gith marauders. Breathless, he draws out his precious may also have effects that appeal to the senses of components and begins his chant and hand motions. Verbal, somatic, and material components are all in play, but what's really happening? What are the sensory effects associated with casting a spell?

On Athas the casting of magical spells often draws unwanted attention. The sensory effects of casting and the ways a wizard might cover, expand, or mimic them are acutely important. These sensory effects relate directly to detection; the greater the effects during casting, the greater the chance the wizard is found out. Of course, when a wizard wishes to dramatically announce his spellcasting abilities, the greater the effects the better.

All spells have both a visual and aural effect during casting. Middle-level (4th- to 6th-level) spells may also have effects that appeal to the senses of touch and smell. High-level spells may have grand effects.

Sensory effects for casting are particular to each spellcaster. The effects themselves remain constant for each sensory category, no matter what the spell level.

\textbf{Visual effects:} Streaks of sparkling multicolored light emerge from the vanishing material components, follow the movements of the spell's somatics (if any), then settle on the subject of the spell. The sparkling lights slightly illuminate the caster and target of the spell for the spell's entire casting time. Brightness is determined by spell level; color varies by spell and by caster. Other possible effects include glowing rings of light, heatless flames, and so on. A visual effect cannot substitute for an existing spell such as \spell{light} or the various illusions.

\textbf{Aural effects:} Along with any verbal components, a shimmer like that of tiny, metal wind chimes accompanies the caster's words, rising and falling with the spell's somatics. The jingling emanates from the caster's location, rising from silence to its maximum volume and back to silence over the casting time. Volume is determined by spell level. Other possible effects include a roaring wind, thousands of slithering snakes, etc.

\textbf{Olfactory effects:} For spells with olfactory sensory effects, an odor unique to the caster or the spell permeates the air. The smell may be pleasant, such as flowers or perfumes, or quite unpleasant, such as rotting meat. Intensity of the smell is determined by spell level.

\textbf{Tactile effects:} Creatures near the caster feel something brush up against them. The nature of the sensation can be soft and pleasant, such as feathers, or abrasive, such as grit or jagged bone. Intensity of the sensation is determined by spell level.

\textbf{Grand effects:} Spells of especially high level may cause grand effects in casting. The ground may tremble, rocking tables and tipping over bottles. The weather may temporarily change, clouding over ominously, wind picking up or stopping, temperature growing abnormally hot or chill.

\Table{Sensory Effects of Casting}{lCCC}{
\tableheader Spell Level & \tableheader Visual/Aural & \tableheader Olfactory/Tacticle & \tableheader Grand \\
1st--3rd & Yes & No       & No       \\
4th--6th & Yes & Optional & No       \\
7th--9th & Yes & Yes      & Optional \\
}


\subsubsection{Detecting Spellcasting}
Whenever a spell is cast, there is a chance that any casual observer will notice it. Aural and visual effects require \skill{Listen} and \skill{Spot} checks, respectively. Olfactory and tacticle effects require Wisdom checks, and the DC increases by 1 for each 3 meters of distance between the spellcaster and the observer. Grand effects are automatically detected. DC for each check is 10 minus the spell level.

% \Table{Sensory Range}{lCCC}{
% \tableheader Spell Level & \tableheader Visual/Aural & \tableheader Olfactory/Tacticle & \tableheader Grand \\
% 1st--3rd & 18 m &      &      \\
% 4th--6th & 27 m & 18 m &      \\
% 7th--9th & 36 m & 27 m & 36 m \\	
% }
