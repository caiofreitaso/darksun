\subsection{Special Spell Effects}
Many special spell effects are handled according to the school of the spells in question Certain other special spell features are found across spell schools.

\textbf{Attacks:} Some spell descriptions refer to attacking. All offensive combat actions, even those that don't damage opponents are considered attacks. Attempts to turn or rebuke undead count as attacks. All spells that opponents resist with saving throws, that deal damage, or that otherwise harm or hamper subjects are attacks. Spells that summon monsters or other allies are not attacks because the spells themselves don't harm anyone.

\textbf{Bonus Types:} Usually, a bonus has a type that indicates how the spell grants the bonus. The important aspect of bonus types is that two bonuses of the same type don't generally stack. With the exception of dodge bonuses, most circumstance bonuses, and racial bonuses, only the better bonus works (see Combining Magical Effects, below). The same principle applies to penalties---a character taking two or more penalties of the same type applies only the worst one.

\textbf{Bringing Back the Dead:} Several spells have the power to restore slain characters to life.

When a living creature dies, its soul departs its body, leaves the Material Plane, travels through the Astral Plane, and goes to abide on the plane where the creature's deity resides. If the creature did not worship a deity, its soul departs to the plane corresponding to its alignment. Bringing someone back from the dead means retrieving his or her soul and returning it to his or her body.

\textit{Level Loss:} Any creature brought back to life usually loses one level of experience. The character's new XP total is midway between the minimum needed for his or her new (reduced) level and the minimum needed for the next one. If the character was 1st level at the time of death, he or she loses 2 points of Constitution instead of losing a level.

This level loss or Constitution loss cannot be repaired by any mortal means, even \spell{wish} or \spell{miracle}. A revived character can regain a lost level by earning XP through further adventuring. A revived character who was 1st level at the time of death can regain lost points of Constitution by improving his or her Constitution score when he or she attains a level that allows an ability score increase.

\textit{Preventing Revivification:} Enemies can take steps to make it more difficult for a character to be returned from the dead. Keeping the body prevents others from using \spell{raise dead} or \spell{resurrection} to restore the slain character to life. Casting \spell{trap the soul} prevents any sort of revivification unless the soul is first released.

\textit{Revivification against One's Will:} A soul cannot be returned to life if it does not wish to be. A soul knows the name, alignment, and patron deity (if any) of the character attempting to revive it and may refuse to return on that basis.

