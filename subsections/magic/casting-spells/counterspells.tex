\subsection{Counterspells}
It is possible to cast any spell as a counterspell. By doing so, you are using the spell's energy to disrupt the casting of the same spell by another character. Counterspelling works even if one spell is divine and the other arcane.

\textbf{How Counterspells Work}: To use a counterspell, you must select an opponent as the target of the counterspell. You do this by choosing the ready action. In doing so, you elect to wait to complete your action until your opponent tries to cast a spell. (You may still move your speed, since ready is a standard action.)

If the target of your counterspell tries to cast a spell, make a \skill{Spellcraft} check (DC 15 + the spell's level). This check is a free action. If the check succeeds, you correctly identify the opponent's spell and can attempt to counter it. If the check fails, you can't do either of these things.

To complete the action, you must then cast the correct spell. As a general rule, a spell can only counter itself. If you are able to cast the same spell and you have it prepared (if you prepare spells), you cast it, altering it slightly to create a counterspell effect. If the target is within range, both spells automatically negate each other with no other results.

\textbf{Counterspelling Metamagic Spells}: Metamagic feats are not taken into account when determining whether a spell can be countered

\textbf{Specific Exceptions}: Some spells specifically counter each other, especially when they have diametrically opposed effects.

\textbf{Dispel Magic as a Counterspell}: You can use dispel magic to counterspell another spellcaster, and you don't need to identify the spell he or she is casting. However, dispel magic doesn't always work as a counterspell.