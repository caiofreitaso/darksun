\Skill{Heal}{Wis}
\textbf{Check:} The DC and effect depend on the task you attempt.

\Table{Heal DCs}{X r{2.5cm}}{
\tableheader Task & \tableheader Heal DC\\
First aid & 15\\
Long-term care & 15\\
Treat wound from caltrop, \spell{spike growth}, or \spell{spike stones} & 15\\
Treat poison & Poison's save DC\\
Treat disease & Disease's save DC
}

\textit{First Aid:} You usually use first aid to save a dying character. If a character has negative hit points and is losing hit points (at the rate of 1 per round, 1 per hour, or 1 per day), you can make him or her stable. A stable character regains no hit points but stops losing them.

\textit{Long-Term Care:} Providing long-term care means treating a wounded person for a day or more. If your Heal check is successful, the patient recovers hit points or ability score points (lost to ability damage) at twice the normal rate: 2 hit points per level for a full 8 hours of rest in a day, or 4 hit points per level for each full day of complete rest; 2 ability score points for a full 8 hours of rest in a day, or 4 ability score points for each full day of complete rest.

You can tend as many as six patients at a time. You need a few items and supplies (bandages, salves, and so on) that are easy to come by in settled lands. Giving long-term care counts as light activity for the healer. You cannot give long-term care to yourself.

Treat Wound from Caltrop, Spike Growth, or Spike Stones
A creature wounded by stepping on a caltrop moves at one-half normal speed. A successful Heal check removes this movement penalty.

A creature wounded by a spike growth or spike stones spell must succeed on a Reflex save or take injuries that reduce his speed by one-third. Another character can remove this penalty by taking 10 minutes to dress the victim's injuries and succeeding on a Heal check against the spell's save DC.

\textit{Treat Poison:} To treat poison means to tend a single character who has been poisoned and who is going to take more damage from the poison (or suffer some other effect). Every time the poisoned character makes a saving throw against the poison, you make a Heal check. The poisoned character uses your check result or his or her saving throw, whichever is higher.

\textit{Treat Disease:} To treat a disease means to tend a single diseased character. Every time he or she makes a saving throw against disease effects, you make a Heal check. The diseased character uses your check result or his or her saving throw, whichever is higher.

\textbf{Action:} Providing first aid, treating a wound, or treating poison is a standard action. Treating a disease or tending a creature wounded by a spike growth or spike stones spell takes 10 minutes of work. Providing long-term care requires 8 hours of light activity.

\textbf{Try Again:} Varies. Generally speaking, you can't try a Heal check again without proof of the original check's failure. You can always retry a check to provide first aid, assuming the target of the previous attempt is still alive.

\textbf{Special:} A character with the Self-Sufficient feat gets a +2 bonus on Heal checks.

A healer's kit gives you a +2 circumstance bonus on Heal checks.