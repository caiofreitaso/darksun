\Skill{Search}{Int}
\textbf{Check:} You generally must be within 10 feet of the object or surface to be searched. The table below gives DCs for typical tasks involving the Search skill.

\textbf{Action:} It takes a full-round action to search a 5-foot-by-5-foot area or a volume of goods 5 feet on a side.

\textbf{Special:} An elf has a +2 racial bonus on Search checks, and a half-elf has a +1 racial bonus.

If you have the \feat{Investigator} feat, you get a +2 bonus on Search checks.

The spells \spell{explosive runes}, \spell{fire trap}, \spell{glyph of warding}, \spell{symbol}, and \spell{teleportation circle} create magic traps that a rogue can find by making a successful Search check and then can attempt to disarm by using \skill{Disable Device}. Identifying the location of a snare spell has a DC of 23. \spell{Spike growth} and \spell{spike stones} create magic traps that can be found using Search, but against which \skill{Disable Device} checks do not succeed. See the individual spell descriptions for details.

Active abjuration spells within 10 feet of each other for 24 hours or more create barely visible energy fluctuations. These fluctuations give you a +4 bonus on Search checks to locate such abjuration spells.

\textbf{Synergy:} If you have 5 or more ranks in Search, you get a +2 bonus on \skill{Survival} checks to find or follow tracks.

If you have 5 or more ranks in \skill{Knowledge} (architecture and engineering), you get a +2 bonus on Search checks to find secret doors or hidden compartments.

\textbf{Restriction:} While anyone can use Search to find a trap whose DC is 20 or lower, only a rogue can use Search to locate traps with higher DCs. (Exception: The spell \spell{find traps} temporarily enables a cleric to use the Search skill as if he were a rogue.)