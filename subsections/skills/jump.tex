\Skill{Jump}{str; armor check penalty}
\textbf{Check:} The DC and the distance you can cover vary according to the type of jump you are attempting (see below).

Your Jump check is modified by your speed. If your speed is 9 meters then no modifier based on speed applies to the check. If your speed is less than 9 meters, you take a $-6$ penalty for every 3 meters of speed less than 9 meters. If your speed is greater than 9 meters, you gain a +4 bonus for every 3 meters beyond 9 meters.

All Jump DCs given here assume that you get a running start, which requires that you move at least 6 meters in a straight line before attempting the jump. If you do not get a running start, the DC for the jump is doubled.

Distance moved by jumping is counted against your normal maximum movement in a round.

If you have ranks in Jump and you succeed on a Jump check, you land on your feet (when appropriate). If you attempt a Jump check untrained, you land prone unless you beat the DC by 5 or more.

\Table{Long Jump DCs}{C C}{
\tableheader Long Jump Distance & \tableheader Jump DC\\
1.5 meter & 5\\
3 meters & 10\\
4.5 meters & 15\\
6 meters & 20\\
7.5 meters & 25\\
9 meters & 30\\
}

\textit{Long Jump:} A long jump is a horizontal jump, made across a gap like a chasm or stream. At the midpoint of the jump, you attain a vertical height equal to one-quarter of the horizontal distance. The DC for the jump is equal to the distance jumped (in meters).

If your check succeeds, you land on your feet at the far end. If you fail the check by less than 5, you don't clear the distance, but you can make a DC 15 Reflex save to grab the far edge of the gap. You end your movement grasping the far edge. If that leaves you dangling over a chasm or gap, getting up requires a move action and a DC 15 Climb check.

\Table{High Jump DCs}{C C}{
\tableheader High Jump Distance & \tableheader Jump DC\\
0.3 meter & 4\\
0.6 meter & 8\\
0.9 meter & 12\\
1.2 meter & 16\\
1.5 meter & 20\\
1.8 meter & 24\\
2.1 meters & 28\\
2.4 meters & 32\\
}

\textit{High Jump:} A high jump is a vertical leap made to reach a ledge high above or to grasp something overhead. The DC is equal to 4 times the distance to be cleared.

If you jumped up to grab something, a successful check indicates that you reached the desired height. If you wish to pull yourself up, you can do so with a move action and a DC 15 Climb check. If you fail the Jump check, you do not reach the height, and you land on your feet in the same spot from which you jumped. As with a long jump, the DC is doubled if you do not get a running start of at least 6 meters.

\Table{Vertical Reach}{C C}{
\tableheader Creature Size & \tableheader Vertical Reach\\
Colossal & 38.4 m\\
Gargantuan & 19.2 m\\
Huge & 9.6 m\\
Large & 4.8 m\\
Medium & 2.4 m\\
Small & 1.2 m\\
Tiny & 60 cm\\
Diminutive & 30 cm\\
Fine & 15 cm\\
}

Obviously, the difficulty of reaching a given height varies according to the size of the character or creature. The maximum vertical reach (height the creature can reach without jumping) for an average creature of a given size is shown on the table below. (As a Medium creature, a typical human can reach 2.4 meters without jumping.)

Quadrupedal creatures don't have the same vertical reach as a bipedal creature; treat them as being one size category smaller.

\textit{Hop Up:} You can jump up onto an object as tall as your waist, such as a table or small boulder, with a DC 10 Jump check. Doing so counts as 3 meters of movement, so if your speed is 9 meters, you could move 6 meters, then hop up onto a counter. You do not need to get a running start to hop up, so the DC is not doubled if you do not get a running start.

\textit{Jumping Down:} If you intentionally jump from a height, you take less damage than you would if you just fell. The DC to jump down from a height is 15. You do not have to get a running start to jump down, so the DC is not doubled if you do not get a running start.

If you succeed on the check, you take falling damage as if you had dropped 3 fewer meters than you actually did.

\textbf{Action:} None. A Jump check is included in your movement, so it is part of a move action. If you run out of movement mid-jump, your next action (either on this turn or, if necessary, on your next turn) must be a move action to complete the jump.

\textbf{Special:} Effects that increase your movement also increase your jumping distance, since your check is modified by your speed.

If you have the \feat{Run} feat, you get a +4 bonus on Jump checks for any jumps made after a running start.

A halfling has a +2 racial bonus on Jump checks because halflings are agile and athletic.

A thri-kreen has a +20 racial bonus on Jump checks because thri-kreens are natural jumpers.

If you have the \feat{Acrobatic} feat, you get a +2 bonus on Jump checks.

\textbf{Synergy:} If you have 5 or more ranks in \skill{Tumble}, you get a +2 bonus on Jump checks.

If you have 5 or more ranks in Jump, you get a +2 bonus on \skill{Tumble} checks.
