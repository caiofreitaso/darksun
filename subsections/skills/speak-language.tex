\Skill{Speak Language}{None; Trained Only}
\Table{Common Languages and Their Alphabets}{l X l}{
\tableheader Language & \tableheader Typical Speakers & \tableheader Alphabet\\
Aquan & Water-based creatures & Elven\\
Auran & Air-based creatures, aarakocra & Saurian\\
Common & Humans, half-elves, half-giants & Common\\
Druidic & Druids (only) & Druidic\\
Dwarven & Dwarves, muls & Dwarven\\
Elven & Elves, half-elves & Elven\\
Entomic & Insectoid creatures, scrabs & Kreen\\
Giant & Giants & Dwarven\\
Gith & Gith & Elven\\
Halfling & Halflings & Halfling\\
Ignan & Fire-based creatures & Saurian\\
Kreen & Thri-kreen, tohr-kreen & Kreen\\
Rhul-thaun & Rhul-thaun & Rhulisti\\
Saurian & Jozhals, pterrans, ssurrans & Saurian\\
Sylvan & Druids, halflings & Halfling\\
Terran & Earth-based creatures, taris & Dwarven\\
Yuan-ti & Yuan-ti & Saurian
}

In addition to the languages on \tabref{Common Languages and Their Alphabets}, each city-state and Dynastic Merchant House has its own language, although merchant houses chiefly use it in cipher form. The following languages are considered dead and are restricted, at DM's discretion: Bodachi, Draxan, Giustenal, Kalidnese, Rhulisti, Saragarian, and Yaramukite.

\textbf{Action:} Not applicable.

\textbf{Try Again:} Not applicable. There are no Speak Language checks to fail.

The Speak Language skill doesn't work like other skills. Languages work as follows.

\begin{itemize*}
\item You start at 1st level knowing one or two languages (based on your race), plus an additional number of languages equal to your starting Intelligence bonus.
\item You can purchase Speak Language just like any other skill, but instead of buying a rank in it, you choose a new language that you can speak.
\item You don't make Speak Language checks. You either know a language or you don't.
\item Each language has an alphabet, though sometimes several spoken languages share a single alphabet.
\end{itemize*}
