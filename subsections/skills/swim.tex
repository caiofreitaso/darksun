\Skill{Swim}{str; armor check penalty}
\textbf{Check:} Make a Swim check once per round while you are in the water. Success means you may swim at up to one-half your speed (as a full-round action) or at one-quarter your speed (as a move action). If you fail by 4 or less, you make no progress through the water. If you fail by 5 or more, you go underwater.

If you are underwater, either because you failed a Swim check or because you are swimming underwater intentionally, you must hold your breath. You can hold your breath for a number of rounds equal to twice your Constitution score, but only if you do nothing other than take move actions or free actions. If you take a standard action or a full-round action (such as making an attack), the remainder of the duration for which you can hold your breath is reduced by 1 round. (Effectively, a character in combat can hold his or her breath only half as long as normal.) After that period of time, you must make a DC 10 Constitution check every round to continue holding your breath. Each round, the DC for that check increases by 1. If you fail the Constitution check, you begin to drown.

The DC for the Swim check depends on the water, as given on the table below.

\Table{Swim DCs}{lCC}{
  \tableheader Water Strength
& \tableheader Current Speed
& \tableheader Swim DC\\
Calm water    &    1.5 m/round & 10 \\
Rough water   &   3--9 m/round & 15\textsuperscript{1} \\
Stormy water  & 12--18 m/round & 20\textsuperscript{1, 2} \\
Violent water & 21--27 m/round & 25\textsuperscript{1, 2} \\
\TableNote{3}{1 If it is flowing water, you take damage every round and must make a Swim check every round to avoid going under.}\\
\TableNote{3}{2 You can't take 10 on a Swim check, even if you aren't otherwise being threatened or distracted.}\\
}

Fast-moving flowing water (such as rivers and streams) deals 1d3 points of nonlethal damage per round, or 1d6 points of lethal damage if flowing over rocks and cascades.

% You can't take 10 on a Swim check in stormy water, even if you aren't otherwise being threatened or distracted.

Each hour that you swim, you must make a DC 20 Swim check or take 1d6 points of nonlethal damage from fatigue.

\textbf{Action:} A successful Swim check allows you to swim one-quarter of your speed as a move action or one-half your speed as a full-round action.

\textbf{Special:} Swim checks are subject to double the normal armor check penalty and encumbrance penalty.

If you have the \feat{Athletic} feat, you get a +2 bonus on Swim checks.

If you have the \feat{Endurance} feat, you get a +4 bonus on Swim checks made to avoid taking nonlethal damage from fatigue.

A creature with a swim speed can move through water at its indicated speed without making Swim checks. It gains a +8 racial bonus on any Swim check to perform a special action or avoid a hazard. The creature always can choose to take 10 on a Swim check, even if distracted or endangered when swimming. Such a creature can use the run action while swimming, provided that it swims in a straight line.

\textbf{Restriction:} Large bodies of water are so uncommon on Athas, that swimming is not a class skill for any class other than for some clerics of elemental water.
