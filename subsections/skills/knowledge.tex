\Skill{Knowledge}{Int; Trained Only}
Like the \skill{Craft} and \skill{Profession} skills, Knowledge actually encompasses a number of unrelated skills. Knowledge represents a study of some body of lore, possibly an academic or even scientific discipline.

Below are listed typical fields of study.

\begin{itemize*}
\item Ancient History (Green Age, the Cleansing Wars, the Champions, Rajaat)
\item Arcana (ancient mysteries, magic traditions, arcane symbols, cryptic phrases, constructs, dragons, magical beasts)
\item Architecture and engineering (buildings, aqueducts, bridges, fortifications)
\item Dungeoneering (aberrations, caverns, oozes, spelunking)
\item Geography (lands, terrain, climate, people)
\item History (royalty, wars, colonies, migrations, founding of cities)
\item Local (legends, personalities, inhabitants, laws, customs, traditions, humanoids)
\item Nature (animals, fey, giants, monstrous humanoids, plants, seasons and cycles, weather, vermin)
\item Nobility and royalty (lineages, heraldry, family trees, mottoes, personalities)
\item Psionics (ancient mysteries, psionic traditions, psychic symbols, cryptic phrases, astral constructs, and psionic races)
\item Religion (gods and goddesses, mythic history, ecclesiastic tradition, holy symbols, undead)
\item The planes (the Inner Planes, the Outer Planes, the Astral Plane, the Ethereal Plane, outsiders, elementals, magic related to the planes)
\item Warcraft (construction of defenses, construction of siege weaponry, logistics, siege weapon operations, war beetle operations, teaching in the use of weapons and communication through signals and messengers)
\end{itemize*}

\textbf{Check:} Answering a question within your field of study has a DC of 10 (for really easy questions), 15 (for basic questions), or 20 to 30 (for really tough questions).

In many cases, you can use this skill to identify monsters and their special powers or vulnerabilities. In general, the DC of such a check equals 10 + the monster's HD. A successful check allows you to remember a bit of useful information about that monster.

For every 5 points by which your check result exceeds the DC, you recall another piece of useful information.

\textit{Identify tactics:} You can use Knowledge (warcraft) to identify tactics, armies, and battle formations. In general, the DC of such a check equals 10 + the army's Encounter Level. A successful check allows you to remember a bit of useful information about that particular army, or tactic or battle formation. For every 5 points by which your check result exceeds the DC, you recall another piece of useful information.

\textit{Coordinate Allies:} You can also use Knowledge (warcraft) to coordinate allies. Each creature to be affected must be able to see and hear you, and able to pay attention to you. To coordinate, make a Knowledge (warcraft) check with a DC equal to 15 + the number of allies affected. If the check succeeds, all affected allies gain a competence bonus on attack rolls or a dodge bonus to AC equal to your Charisma modifier. You choose which
of the two benefits to impart and must impart the same benefit to all affected allies. The benefits last for 1 round.

You cannot use this ability on yourself.

Coordinate allies does not provoke an attack of opportunity.

\textbf{Action:} Usually none. A Knowledge (warcraft) check made to coordinate allies is a full-round action. In most cases, making a Knowledge check doesn't take an action---you simply know the answer or you don't.

\textbf{Try Again:} No. The check represents what you know, and thinking about a topic a second time doesn't let you know something that you never learned in the first place.

\textbf{Special:} There are no easy or basic questions for Knowledge (ancient history). While most people have heard legends of a better time, to most they are simply disregarded as fanciful mythology. The sorcerer-kings have destroyed most written records of the history of Athas, and what little remains is plagued with half-truths and outright lies.

If you have the \feat{Field Officer} feat, you get a +2 bonus on Knowledge (warcraft) checks.

If you have the \feat{Autonomous} feat, you get a +2 bonus on Knowledge (psionics) checks.

\textbf{Synergy:} If you have 5 or more ranks in Knowledge (arcana), you get a +2 bonus on Spellcraft checks.

If you have 5 or more ranks in Knowledge (architecture and engineering), you get a +2 bonus on Search checks made to find secret doors or hidden compartments.

If you have 5 or more ranks in Knowledge (dungeoneering), you get a +2 bonus on Survival checks made while underground.

If you have 5 or more ranks in Knowledge (geography), you get a +2 bonus on Survival checks made to keep from getting lost or to avoid natural hazards.

If you have 5 or more ranks in Knowledge (history), you get a +2 bonus on bardic knowledge checks.

If you have 5 or more ranks in Knowledge (local), you get a +2 bonus on Gather Information checks.

If you have 5 or more ranks in Knowledge (nature), you get a +2 bonus on Survival checks made in aboveground natural environments (aquatic, desert, forest, hill, marsh, mountains, or plains).

If you have 5 or more ranks in Knowledge (nobility and royalty), you get a +2 bonus on Diplomacy checks.

If you have 5 or more ranks in Knowledge (psionics), you get a +2 bonus on Psicraft checks.

If you have 5 or more ranks in Knowledge (religion), you get a +2 bonus on turning checks against undead.

If you have 5 or more ranks in Knowledge (the planes), you get a +2 bonus on Survival checks made while on other planes.

If you have 5 or more ranks in Knowledge (warcraft), you get a +2 bonus on Diplomacy checks related to dealing with troops.

If you have 5 or more ranks in Autohypnosis, you get a +2 bonus on Knowledge (psionics) checks.

If you have 5 or more ranks in Survival, you get a +2 bonus on Knowledge (nature) checks.

\textbf{Untrained:} An untrained Knowledge check is simply an Intelligence check. Without actual training, you know only common knowledge (DC 10 or lower).


