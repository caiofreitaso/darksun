\Skill{Concentration}{Con}
\textbf{Check:} You must make a Concentration check whenever you might potentially be distracted (by taking damage, by harsh weather, and so on) while engaged in some action that requires your full attention. Such actions include casting a spell, concentrating on an active spell, directing a spell, using a spell-like ability, or using a skill that would provoke an attack of opportunity. In general, if an action wouldn't normally provoke an attack of opportunity, you need not make a Concentration check to avoid being distracted.

If the Concentration check succeeds, you may continue with the action as normal. If the check fails, the action automatically fails and is wasted. If you were in the process of casting a spell, the spell is lost. If you were concentrating on an active spell, the spell ends as if you had ceased concentrating on it. If you were directing a spell, the direction fails but the spell remains active. If you were using a spell-like ability, that use of the ability is lost. A skill use also fails, and in some cases a failed skill check may have other ramifications as well.

The table below summarizes various types of distractions that cause you to make a Concentration check. If the distraction occurs while you are trying to cast a spell, you must add the level of the spell you are trying to cast to the appropriate Concentration DC. If more than one type of distraction is present, make a check for each one; any failed Concentration check indicates that the task is not completed.

If you are trying to cast, concentrate on, or direct a spell when the distraction occurs, add the level of the spell to the indicated DC.

If the distracting spell or power allows no save, use the save DC it would have if it did allow a save.

\Table{Concentration DCs}{Y{3cm} >{\raggedright\arraybackslash}X}{
\tableheader Concentration DC & \tableheader Distraction\\
% 
% Such as during the casting of a spell with a casting time of 1 round or more, or the execution of an activity that takes more than a single full-round action (such as Disable Device). Also, damage stemming from an attack of opportunity or readied attack made in response to the spell being cast (for spells with a casting time of 1 standard action) or the action being taken (for activities requiring no more than a full-round action).
% Such as from acid arrow.
% 
10 + damage dealt & Damaged during the action.\\
10 + half of continuous damage last dealt & Taking continuous damage during the action, such as from \emph{acid arrow}.\\
Distracting spell's save DC & Distracted by nondamaging spell.\\
10 & Vigorous motion (on a moving mount, taking a bouncy wagon ride, in a small boat in rough water, belowdecks in a stormtossed ship).\\
15 & Violent motion (on a galloping horse, taking a very rough wagon ride, in a small boat in rapids, on the deck of a storm-tossed ship).\\
20 & Extraordinarily violent motion (earthquake).\\
15 & Entangled.\\
20 & Grappling or pinned. (You can cast only spells without somatic components for which you have any required material component in hand.)\\
5 & Weather is a high wind carrying blinding rain or sleet.\\
10 & Weather is wind-driven hail, dust, or debris.\\
Distracting spell's save DC & Weather caused by a spell, such as \emph{storm of vengeance}.\\
% \hline
Distracting power's save DC & Distracted by nondamaging power.\\
15 + power level & Attempting to manifest a power without its display.\\
20 & Gain psionic focus.\\
20 & Grappling or pinned. (You can manifest powers normally unless you fail your Concentration check.)\\
Distracting power's save DC & Weather caused by power
}

\textit{Gain Psionic Focus:} Merely holding a reservoir of psionic power points in mind gives psionic characters a special energy. Psionic characters can put that energy to work without actually paying a power point cost---they can become psionically focused as a special use of the Concentration skill.

If you have 1 or more power points available, you can meditate to attempt to become psionically focused. The DC to become psionically focused is 20. Meditating is a full-round action that provokes attacks of opportunity. When you are psionically focused, you can expend your focus on any single Concentration check you make thereafter. When you expend your focus in this manner, your Concentration check is treated as if you rolled a 15. It's like taking 10, except that the number you add to your Concentration modifier is 15. You can also expend your focus to gain the benefit of a psionic feat---many psionic feats are activated in this way.

Once you are psionically focused, you remain focused until you expend your focus, become unconscious, or go to sleep (or enter a meditative trance, in the case of elans), or until your power point reserve drops to 0.

\textbf{Action:} Usually none. In most cases, making a Concentration check doesn't require an action; it is either a free action (when attempted reactively) or part of another action (when attempted actively). Meditating to gain psionic focus is a full-round action.

\textbf{Try Again:} Yes, though a success doesn't cancel the effect of a previous failure, such as the loss of a spell you were casting or the disruption of a spell you were concentrating on.

\textbf{Special:} You can use Concentration to cast a spell, use a spell-like ability, or use a skill defensively, so as to avoid attacks of opportunity altogether. This doesn't apply to other actions that might provoke attacks of opportunity.

The DC of the check is 15 (plus the spell's level, if casting a spell or using a spell-like ability defensively). If the Concentration check succeeds, you may attempt the action normally without provoking any attacks of opportunity. A successful Concentration check still doesn't allow you to take 10 on another check if you are in a stressful situation; you must make the check normally. If the Concentration check fails, the related action also automatically fails (with any appropriate ramifications), and the action is wasted, just as if your concentration had been disrupted by a distraction.

A character with the Combat Casting feat gets a +4 bonus on Concentration checks made to cast a spell or use a spell-like ability while on the defensive or while grappling or pinned.

You can use Concentration to manifest a power or use a psi-like ability defensively, so as to avoid attacks of opportunity altogether. The DC of the check is 15 + the power's level. If the Concentration check succeeds, you can manifest normally without provoking any attacks of opportunity. If the Concentration check fails, the power also automatically fails and the power points are wasted, just as if your concentration had been disrupted by a distraction.

A character with the Combat Manifestation feat gets a +4 bonus on Concentration checks made to manifest a power or use a psi-like ability while on the defensive or while grappling or pinned.

\textbf{Synergy:} If you have 5 or more ranks in Concentration, you get a +2 bonus on \skill{Autohypnosis} checks.