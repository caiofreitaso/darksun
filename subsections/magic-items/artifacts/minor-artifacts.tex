\subsectionA{Minor Artifacts}
Minor artifacts are not necessarily unique items. Even so, they are magic items that no longer can be created, at least by common mortal means.

\subsubsection{Minor Artifact Descriptions}
Described below is a selection of the most well-known (not necessarily the most numerous) minor artifacts.

\textbf{Book of Infinite Spells:} This work bestows upon any character of any class the ability to use the spells within its pages. However, any character not already able to use spells gains one negative level for as long as the book is in her possession or while she uses its power. A book of infinite spells contains 1d8+22 pages. The nature of each page is determined by a dice roll: 01--50, arcane spell; 51--100, divine spell.

Determine the exact spell by using the tables for determining major scroll spells.

Once a page is turned, it can never be flipped back---paging through a book of infinite spells is a one-way trip. If the book is closed, it always opens again to the page it was on before the book was closed. When the last page is turned, the book vanishes.

Once per day the owner of the book can cast the spell to which the book is opened. If that spell happens to be one that is on the character's class spell list, she can cast it up to four times per day. The pages cannot be ripped out without destroying the book. Similarly, the spells cannot be cast as scroll spells, nor can they be copied into a spellbook---their magic is bound up permanently within the book itself.

The owner of the book need not have the book on her person in order to use its power. The book can be stored in a place of safety while the owner is adventuring and still allow its owner to cast spells by means of its power.

Each time a spell is cast, there is a chance that the energy connected with its use causes the page to magically turn despite all precautions. The owner knows this and may even benefit from the turning by gaining access to a new spell. The chance of a page turning depends on the spell the page contains and what sort of spellcaster the owner is.

\Table{}{LZ{17mm}}{
  \tableheader Condition
& \tableheader Chance of Page Turning \\
Caster employing a spell usable by own class and level     & 10\% \\
Caster employing a spell not usable by own class and level & 20\% \\
Nonspellcaster employing divine spell                      & 25\% \\
Nonspellcaster employing arcane spell                      & 30\% \\
}

Treat each spell use as if a scroll were being employed, for purposes of determining casting time, spell failure, and so on.

Strong (all schools); CL 18th; Weight 1.5 kg.

\textbf{Deck of Many Things:} A deck of many things (both beneficial and baneful) is usually found in a box or leather pouch. Each deck contains a number of cards or plaques made of ivory or vellum. Each is engraved with glyphs, characters, and sigils. As soon as one of these cards is drawn from the pack, its magic is bestowed upon the person who drew it, for better or worse.

The character with a deck of many things who wishes to draw a card must announce how many cards she will draw before she begins. Cards must be drawn within 1 hour of each other, and a character can never again draw from this deck any more cards than she has announced. If the character does not willingly draw her allotted number (or if she is somehow prevented from doing so), the cards flip out of the deck on their own.

Exception: If the jester is drawn, the possessor of the deck may elect to draw two additional cards.

Each time a card is taken from the deck, it is replaced (making it possible to draw the same card twice) unless the draw is the jester or the fool, in which case the card is discarded from the pack. A deck of many things contains 22 cards. To simulate the magic cards, you may want to use tarot cards, as indicated in the second column of the accompanying table. If no tarot deck is available, substitute ordinary playing cards instead, as indicated in the third column. The effects of each card, summarized on the table, are fully described below.

\BigTableBottom{Deck of Many Things}{LLLl}{
  \tableheader Plaque
& \tableheader Tarot Card
& \tableheader Playing Card
& \tableheader Summary of Effect \\

Balance   & XI. Justice         & Two of spades             & Change alignment instantly. \\
Comet     & Two of swords       & Two of diamonds           & Defeat the next monster you meet to gain one level. \\
Donjon    & Four of swords      & Ace of spades             & You are imprisoned. \\
Euryale   & Ten of swords       & Queen of spades           & $-1$ penalty on all saving throws henceforth. \\
The Fates & Three of cups       & Ace of hearts             & Avoid any situation you choose . . . once. \\
Flames    & XV. The Devil       & Queen of clubs            & Enmity between you and an outsider. \\
Fool      & 0. The Fool         & Joker (with trademark)    & Lose 10,000 experience points and you must draw again. \\
Gem       & Seven of cups       & Two of hearts             & Gain your choice of twenty-five pieces of jewelry or fifty gems. \\
Idiot     & Two of pentacles    & Two of clubs              & Lose Intelligence (permanent drain). You may draw again. \\
Jester    & XII. The Hanged Man & Joker (without trademark) & Gain 10,000 XP or two more draws from the deck. \\
Key       & V. The Hierophant   & Queen of hearts           & Gain a major magic weapon. \\
Knight    & Page of swords      & Jack of hearts            & Gain the service of a 4th-level fighter. \\
Moon      & XVIII. The Moon     & Queen of diamonds         & You are granted 1d4 \spellref{wish}{wishes}. \\
Rogue     & Five of swords      & Jack of spades            & One of your friends turns against you. \\
Ruin      & XVI. The Tower      & King of spades            & Immediately lose all wealth and real property. \\
Skull     & XIII. Death         & Jack of clubs             & Defeat dread wraith or be forever destroyed. \\
Star      & XVII. The Star      & Jack of diamonds          & Immediately gain a +2 inherent bonus to one ability score. \\
Sun       & XIX. The Sun        & King of diamonds          & Gain beneficial medium wondrous item and 50,000 XP. \\
Talons    & Queen of pentacles  & Ace of clubs              & All magic items you possess disappear permanently. \\
Throne    & Four of staves      & King of hearts            & Gain a +6 bonus on \skill{Diplomacy} checks plus a small keep. \\
Vizier    & IX. The Hermit      & Ace of diamonds           & Know the answer to your next dilemma. \\
The Void  & Eight of swords     & King of clubs             & Body functions, but soul is trapped elsewhere. \\
}


\textit{Balance:} The character must change to a radically different alignment. If the character fails to act according to the new alignment, she gains a negative level.

\textit{Comet:} The character must single-handedly defeat the next hostile monster or monsters encountered, or the benefit is lost. If successful, the character gains enough XP to attain the next experience level.

\textit{Donjon:} This card signifies imprisonment--- either by the imprisonment spell or by some powerful being. All gear and spells are stripped from the victim in any case. Draw no more cards.

\textit{Euryale:} The medusalike visage of this card brings a curse that only the fates card or a deity can remove. The -1 penalty on all saving throws is otherwise permanent.

\textit{Fates:} This card enables the character to avoid even an instantaneous occurrence if so desired, for the fabric of reality is unraveled and respun. Note that it does not enable something to happen. It can only stop something from happening or reverse a past occurrence. The reversal is only for the character who drew the card; other party members may have to endure the situation.

\textit{Flames:} Hot anger, jealousy, and envy are but a few of the possible motivational forces for the enmity. The enmity of the outsider can't be ended until one of the parties has been slain. Determine the outsider randomly, and assume that it attacks the character (or plagues her life in some way) within 1d20 days.

\textit{Fool:} The payment of XP and the redraw are mandatory. This card is always discarded when drawn, unlike all others except the jester.

\textit{Gem:} This card indicates wealth. The jewelry is all gold set with gems, each piece worth 2,000 gp, the gems 1,000 gp value each.

\textit{Idiot:} This card causes the drain of 1d4+1 points of Intelligence immediately. The additional draw is optional.

\textit{Jester:} This card is always discarded when drawn, unlike all others except the fool. The redraws are optional.

\textit{Key:} The magic weapon granted must be one usable by the character. It suddenly appears out of nowhere in the character's hand.

\textit{Knight:} The fighter appears out of nowhere and serves loyally until death. He or she is of the same race (or kind) and gender as the character.

\textit{Moon:} This card sometimes bears the image of a moonstone gem with the appropriate number of \spellref{wish}{wishes} shown as gleams therein; sometimes it depicts a moon with its phase indicating the number of \spellref{wish}{wishes} (full = four; gibbous = three; half = two; quarter = one). These \spellref{wish}{wishes} are the same as those granted by the 9th-level wizard spell and must be used within a number of minutes equal to the number received.

\textit{Rogue:} When this card is drawn, one of the character's NPC friends (preferably a cohort) is totally alienated and forever after hostile. If the character has no cohorts, the enmity of some powerful personage (or community, or religious order) can be substituted. The hatred is secret until the time is ripe for it to be revealed with devastating effect.

\textit{Ruin:} As implied by its name, when this card is drawn, all nonmagical possessions of the drawer are lost.

\textit{Skull:} A dread wraith appears. Treat this creature as an unturnable undead. The character must fight it alone---if others help, they get dread wraiths to fight as well. If the character is slain, she is slain forever and cannot be revived, even with a wish or a miracle.

\textit{Star:} The 2 points are added to any ability the character chooses. They cannot be divided among two abilities.

\textit{Sun:} Roll for a medium wondrous item until a useful item is indicated.

\textit{Talons:} When this card is drawn, every magic item owned or possessed by the character is instantly and irrevocably gone.

\textit{Throne:} The character becomes a true leader in people's eyes. The castle gained appears in any open area she wishes (but the decision where to place it must be made within 1 hour).

\textit{Vizier:} This card empowers the character drawing it with the one-time ability to call upon a source of wisdom to solve any single problem or answer fully any question upon her request. The query or request must be made within one year. Whether the information gained can be successfully acted upon is another question entirely.

\textit{The Void:} This black card spells instant disaster. The character's body continues to function, as though comatose, but her psyche is trapped in a prison somewhere---in an object on a far plane or planet, possibly in the possession of an outsider. A wish or a miracle does not bring the character back, instead merely revealing the plane of entrapment. Draw no more cards.

Strong (all schools); CL 20th.

\textbf{Hammer of Thunderbolts:} This \emph{+3 Large returning warhammer} deals 4d6 points of damage on any hit. Further, if the wielder wears a \emph{belt of giant strength} and \emph{gauntlets of ogre power} and he knows that the hammer is a \emph{hammer of thunderbolts} (not just a \emph{+3 warhammer}), the weapon can be used to full effect: It gains a total +5 enhancement bonus, allows all belt and gauntlet bonuses to stack (only when using this weapon), and strikes dead any giant upon whom it scores a hit (Fortitude DC 20 negates the death effect but not the damage).

When hurled, on a successful attack the hammer emits a great noise, like a clap of thunder, causing all creatures within 27 meters to be stunned for 1 round (Fortitude DC 15 negates). The hammer's range increment is 9 meters.

Strong evocation, necromancy, and transmutation; CL 20th; Weight 7.5 kg.

\textbf{Philosopher's Stone:} This rare substance appears to be an ordinary, sooty piece of blackish rock. If the stone is broken open (break DC 20), a cavity is revealed at the stone's heart. This cavity is lined with a magical type of quicksilver that enables any arcane spellcaster to transmute base metals (iron and lead) into silver and gold. A single \emph{philosopher's stone} can turn from up to 2,500 kilograms of iron into silver, or up to 500 kilograms of lead into gold. However, the quicksilver becomes unstable once the stone is opened and loses its potency within 24 hours, so all transmutations must take place within that period.

The quicksilver found in the center of the stone may also be put to another use. If mixed with any cure potion while the substance is still potent, it creates a special oil of life that acts as a true resurrection spell for any dead body it is sprinkled upon.

Strong transmutation; CL 20th; Weight 1.5 kg.

\textbf{Sphere of Annihilation:} A \emph{sphere of annihilation} is a globe of absolute blackness, a ball of nothingness 2 feet in diameter. The object is actually a hole in the continuity of the multiverse. Any matter that comes in contact with a sphere is instantly sucked into the void, gone, and utterly destroyed. Only the direct intervention of a deity can restore an annihilated character.

A \emph{sphere of annihilation} is static, resting in some spot as if it were a normal hole. It can be caused to move, however, by mental effort (think of this as a mundane form of telekinesis, too weak to move actual objects but a force to which the sphere, being weightless, is sensitive). A character's ability to gain control of a \emph{sphere of annihilation} (or to keep controlling one) is based on the result of a control check against DC 30 (a move action). A control check is 1d20 + character level + character Int modifier. If the check succeeds, the character can move the sphere (perhaps to bring it into contact with an enemy) as a free action.

Control of a sphere can be established from as far away as 12 meters (the character need not approach too closely). Once control is established, it must be maintained by continuing to make control checks (all DC 30) each round. For as long as a character maintains control (does not fail a check) in subsequent rounds, he can control the sphere from a distance of 12 meters + 3 meters per character level. The sphere's speed in a round is 3 meters +1.5 meter for every 5 points by which the character's control check result in that round exceeded 30.

If a control check fails, the sphere slides 3 meters in the direction of the character attempting to move it.

If two or more creatures vie for control of a \emph{sphere of annihilation}, the rolls are opposed. If none are successful, the sphere slips toward the one who rolled lowest.

Should a \spell{gate} spell be cast upon a \emph{sphere of annihilation}, there is a 50\% chance (01--50 on d\%) that the spell destroys it, a 35\% chance (51--85) that the spell does nothing, and a 15\% chance (86--100) that a gap is torn in the spatial fabric, catapulting everything within a 54-meter radius into another plane. If a \emph{rod of cancellation} touches a \emph{sphere of annihilation}, they negate each other in a tremendous explosion. Everything within a 18-meter radius takes 2d6$\times$10 points of damage. \spellref{dispel magic}{Dispel magic} and \spell{mage's disjunction} have no effect on a sphere.

See also: \emph{talisman of the sphere} (below).

Strong transmutation; CL 20th.

\textbf{Staff of the Magi:} A long wooden staff, shod in iron and inscribed with sigils and runes of all types, this potent artifact contains many spell powers and other functions. Some of its powers use charges, while others don't. The following powers do not use charges:
\begin{itemize*}
\item \spellref{detect magic}{Detect magic}
\item \spellref{enlarge person}{Enlarge person} (Fortitude DC 15 negates)
\item \spellref{hold portal}{Hold portal}
\item \spell{Light}
\item \spellref{mage armor}{Mage armor}
\item \spellref{mage hand}{Mage hand}
\end{itemize*}

The following powers drain 1 charge per usage:
\begin{itemize*}
\item \spellref{dispel magic}{Dispel magic}
\item \spell{Fireball} (10d6 damage, Reflex DC 17 half)
\item \spellref{ice storm}{Ice storm}
\item \spell{Invisibility}
\item \spell{Knock}
\item \spellref{lightning bolt}{Lightning bolt} (10d6 damage, Reflex DC 17 half)
\item \spell{Passwall}
\item \spell{Pyrotechnics} (Will or Fortitude DC 16 negates)
\item \spellref{wall of fire}{Wall of fire}
\item \spell{Web}
\end{itemize*}

These powers drain 2 charges per usage:
\begin{itemize*}
\item \spellref{summon monster IX}{Summon monster IX}
\item \spellref{plane shift}{Plane shift} (Will DC 21 negates)
\item \spell{Telekinesis} (200 kg maximum weight; Will DC 19 negates)
\end{itemize*}

A \emph{staff of the magi} gives the wielder spell resistance 23. If this is willingly lowered, however, the staff can also be used to absorb arcane spell energy directed at its wielder, as a \emph{rod of absorption} does. Unlike the rod, this staff converts spell levels into charges rather than retaining them as spell energy usable by a spellcaster. If the staff absorbs enough spell levels to exceed its limit of 50 charges, it explodes as if a retributive strike had been performed (see below). The wielder has no idea how many spell levels are cast at her, for the staff does not communicate this knowledge as a \emph{rod of absorption} does. (Thus, absorbing spells can be risky.)

\textit{Retributive Strike:} A \emph{staff of the magi} can be broken for a retributive strike. Such an act must be purposeful and declared by the wielder. All charges in the staff are released in a 9-meter spread. All within 3 meters of the broken staff take hit points of damage equal to 8 times the number of charges in the staff, those between 3.1 meters and 6 meters away take points equal to 6 times the number of charges, and those 6.1 meters to 9 meters distant take 4 times the number of charges. A DC 17 Reflex save reduces damage by half.

The character breaking the staff has a 50\% chance (01--50 on d\%) of traveling to another plane of existence, but if she does not (51--100), the explosive release of spell energy destroys her. Only specific items, including the \emph{staff of the magi} and the staff of power are capable of a retributive strike.

Strong (all schools); CL 20th; Weight 2.5 kg.

\textbf{Talisman of Pure Good:} A good (LG, NG, CG) divine spellcaster who possesses this item can cause a flaming crack to open at the feet of an evil (LE, NE, CE) divine spellcaster who is up to 30 meters away. The intended victim is swallowed up forever and sent hurtling to the center of the earth. The wielder of the talisman must be good, and if he is not exceptionally pure in thought and deed the evil character gains a DC 19 Reflex saving throw to leap away from the crack. Obviously, the target must be standing on solid ground for this item to function.

A talisman of pure good has 6 charges. If a neutral (LN, N, CN) divine spellcaster touches one of these stones, he takes 6d6 points of damage. If an evil divine spellcaster touches one, he takes 8d6 points of damage. All other characters are unaffected by the device.

Strong evocation [good]; CL 18th.

\textbf{Talisman of the Sphere:} This small adamantine loop and handle are useless to those unable to cast arcane spells.

Characters who cannot cast arcane spells take 5d6 points of damage merely from picking up and holding a talisman of this sort. However, when held by an arcane spellcaster who is concentrating on control of a \emph{sphere of annihilation}, a \emph{talisman of the sphere} doubles the character's modifier on his control check (doubling both his Intelligence bonus and his character level for this purpose).

If the wielder of a talisman establishes control, he need check for maintaining control only every other round thereafter. If control is not established, the sphere moves toward him. Note that while many spells and effects of cancellation have no effect upon a \emph{sphere of annihilation}, the talisman's power of control can be suppressed or canceled.

Strong transmutation; CL 16th; Weight 0.5 kg.

\textbf{Talisman of Reluctant Wishes:} A talisman of this sort appears the same as a stone of controlling earth elementals. Its powers are quite different, however, and dependent on the Charisma of the individual holding the talisman. Whenever a character touches a \emph{talisman of reluctant wishes}, he must make a DC 15 Charisma check.

If he fails, the device acts as a stone of weight. Discarding or destroying it results in 5d6 points of damage to the character and the disappearance of the talisman.

If he succeeds, the talisman remains with the character for 5d6 hours, or until a wish is made with it, whichever comes first. It then disappears.

If he rolls a natural 20, the character finds it impossible to be rid of the talisman for as many months as he has points of Charisma. In addition, the artifact grants him one wish for every 6 points of the character's Charisma. It also grows warm and throbs whenever its possessor comes within 6 meters of a mechanical or magic trap. (If the talisman is not held, its warning heat and pulses are of no avail.)

Regardless of which reaction results, a \emph{talisman of reluctant wishes} disappears when its time period expires, leaving behind a 10,000 gp diamond in its stead.

Strong conjuration; CL 20th; Weight 0.5 kg.

\textbf{Talisman of Ultimate Evil:} An evil (LE, NE, CE) divine spellcaster who possesses this item can cause a flaming crack to open at the feet of a good (LG, NG, CG) divine spellcaster who is up to 30 meters away. The intended victim is swallowed up forever and sent hurtling to the center of the earth. The wielder of the talisman must be evil, and if she is not exceptionally foul and perverse in the sights of her evil deity the good character gains a DC 19 Reflex save to leap away from the crack. Obviously, the target must be standing on solid ground for this item to function.

A talisman of ultimate evil has 6 charges. If a neutral (LN, N, CN) divine spellcaster touches one of these stones, she takes 6d6 points of damage. If a good divine spellcaster touches one, she takes 8d6 points of damage. All other characters are unaffected by the device.

Strong evocation [evil]; CL 18th.
