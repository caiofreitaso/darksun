\subsection{Item of Power Values}
Many factors must be considered when determining the price of new magic items. The easiest way to come up with a price is to match the new item to an item that is already priced that price as a guide. Otherwise, use the guidelines summarized on \tabref{Estimating Item of Power Values}.

\textbf{Multiple Similar Abilities:} For items with multiple similar abilities that don't take up space on a character's body use the following formula: Calculate the price of the single most costly ability, then add 75\% of the value of the next most costly ability, plus one-half the value of any other abilities.

\textbf{Multiple Different Abilities:} Abilities such as an attack roll bonus or saving throw bonus and a spell-like function are not similar, and their values are simply added together to determine the cost. For items that do take up a space on a character's body each additional power not only has no discount but instead has a 50% increase in price.

\textbf{0-Level Spells:} When multiplying spell levels to determine value, 0-level spells should be treated as \onehalf level.

\textbf{Other Considerations:} Once you have a final cost figure, reduce that number if either of the following conditions applies:

\textit{Item Requires Skill to Use:} Some items require a specific skill to get them to function. This factor should reduce the cost about 10\%.

\textit{Item Requires Specific Class or Alignment to Use:} Even more restrictive than requiring a skill, this limitation cuts the cost by 30\%.

Prices presented in the magic item descriptions (the gold piece value following the item's caster level) are the market value, which is generally twice what it costs the creator to make the item.

Since different classes get access to certain spells at different levels, the prices for two characters to make the same item might actually be different. An item is only worth two times what the caster of lowest possible level can make it for. Calculate the market price based on the lowest possible level caster, no matter who makes the item.

Not all items adhere to these formulas directly. The reasons for this are several. First and foremost, these few formulas aren't enough to truly gauge the exact differences between items. The price of a magic item may be modified based on its actual worth. The formulas only provide a starting point. The pricing of scrolls assumes that, whenever possible, a wizard or cleric created it. Potions and wands follow the formulas exactly. Staffs follow the formulas closely, and other items require at least some judgment calls.
