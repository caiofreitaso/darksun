\subsectionA{Living Magical Items}
Preservers of a bygone age created means to help their cause with magical trees that could store and channel powers. The most common and fragile of them is a \emph{potion tree}, which can be planted by a very skillful gardener. The other one is a \emph{tree of life}, which requires a powerful spellcaster to create and may live forever.

\subsubsection{Potion Trees}
In Athas, \emph{potion fruits} can be planted in order to grow more \emph{potion fruits}. Once planted, they need to be tended every day for 1d6 weeks (\skill{Profession} (gardener) DC 25, each day). After that time, the \emph{potion tree} grows 1d3 new \emph{potion fruits} among all nonmagical fruits. Those are all the \emph{potion fruits} tree will normally give.

If a \spell{permanency} spell is cast on it, the tree continues to grow \emph{potion fruits}. A \emph{permanent potion tree} can hold only one \emph{potion fruit}, but once it is picked a new one grows in 1d6 days.

\emph{Potion trees} are very sensitive to its environment. Any severe change in the weather, such as a drought or freeze, will ruin the tree and it will not bear any fruits. Any defiling that happens near the tree will kill it and its fruits.

\Figure{t}{images/tree-of-life.png}
\subsubsection{Trees of Life}
A \emph{tree of life} is a powerful magical tree, that bolsters the capabilities of elemental clerics and druids. Initially created during the Preserver Jihad, this type of living magical item was designed to withstand the destruction defilers were causing on vegetation. Since they are virtually eternal, most of those existing in the present predate the villages that have grown around them.

A \emph{tree of life} can endure any climate or terrain---they will flourish in the middle of the desert, or on a rocky mountain, through drought, lightning, and even earthquakes. Nothing in the natural world can affect a \emph{tree}, as when chopped down it will regrow back to full size after a month. They can sustain even the effects of the defiling radius.

This ability to sustain destruction is why defilers seek those trees. Sorcerer-monarchs have groves of \emph{trees of life}, where they and their royal defilers can exercise magic without decimating their cities.

A \emph{tree of life} is always created from a living sapling, no older than one year. After a week, it grows to its full size. It has all its abilities at time of creation.

\textbf{Granted Abilities:} Elemental clerics and druids in contact with a \emph{tree of life} gain the following spell-like abilities: 1/day---\spell{augury}, \spell{divination}, \spell{heal}, and \spell{scrying}.

\textbf{Vital Force:} A \emph{tree of life} consists of two parts, its physical form and its vital force. It can only be destroyed if both parts die. The physical form can be destroyed as a normal tree, as it is not magical. A \emph{tree of life}'s vital force has a total of 100 hit points, divided in 10 layers of 10 hit points each. It regenerates one layer per hour, and it can be affected in the following ways:
\begin{itemize*}
    \item Each negative level removes one layer;
    \item A necromancy death spell that can target a living tree removes three layers;
    \item A \emph{restoration} spell restores one layer;
    \item A \emph{raise dead} or \emph{resurrection} spell restore three layers;
    \item A \emph{greater restoration} or \emph{true resurrection} spell restore all layers;
    \item Only \spell{wish} or \spell{miracle} can outright slay a \emph{tree of life}.
\end{itemize*}

Defiling will damage its vital force as well. Whenever a wizard defiles within 90 meters of a \emph{tree of life}, it loses a layer for each spell level cast. This damage replaces the normal defiling radius, protecting surrounding vegetation.

Strong transmutation; CL 15th; \feat{Craft Wondrous Item}, creator must have 15 ranks in the \skill{Knowledge} (nature) skill; Price 60,000 cp.
