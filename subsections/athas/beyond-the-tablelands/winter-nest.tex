\SubCity{Winter Nest}
{650 (100\% aarakocra)}
{Ice, feathers}
{Auran, Kurnan}
{
\Figure{t}{images/aarakocra-1.png}
	The village of Winter Nest is located in the frozen peaks of the White Mountains. It is the home of a civilized tribe of aarakocra.

	The unusual buildings of Winter Nest are formed from a mixture of ice, stone, and shaped bricks. To new visitors the village looks like a cluster of towers, giving the appearance that the mountain peak has a crown. There are no roads in Winter Nest and very few connecting walkways between the buildings, as the aarakocra fly rather than walk. Doorways appear all along the face of the buildings, though most are clustered near the top of each tower. Landing platforms and resting perches decorate the outsides of most building. Each tower is topped with a large rounded structure. Most of these sphere-shaped constructs are communal areas, though the highest are the personal quarters of the leaders of Winter Nest.
}
{
	The aarakocra of Winter Nest called themselves ``silvaarak,'' which means ``people of the silver wing.'' They are perceptive, and have great confidence and pride in themselves. This translates into arrogance at times, because the silvaarak believe that their ability to fly makes them superior to all other races. Though they often express sympathy for people unable to fly, this more often comes across as condescending.

	The aarakocra have had a difficult time forming friendly relations with others over the years. Only in Kurn have they made dedicated friends. Traders from Winter Nest visit the city-state of Kurn a few times each year for trade. Other attempts to make contact with other communities have meet with failure. Either due to the hostility of the natives such as in Eldaarich and the Bandit States, or the silvaarak's condescending nature towards other races.
}
{
	Winter Nest is lead by Traaka (LG female aarakocra, air cleric 5/elementalist 2) a female aarakocra of many years. Traditionally, the aarakocra are isolationists, and Traaka supports this policy. The isolationist policy was adopted years ago after bad experiences with Eldaarich and later with the peoples of the Bandit States. The policy has kept the village safe over the years and most of the silvaarak want to see it continue.

	However, many of the younger generation of bird-people desire to explore the world beyond the White Mountains. They have been vocal in their wish to explore and make contact with other civilizations, believing they will not experience such bad receptions as those the aarakocra received in Eldaarich or the Bandit States. Pointing to Kurn, these young bloods believe there is opportunity for the silvaarak in positive relationships with outsiders.

	Traaka understands the young aarakocra's desires, but wishes to maintain the status quo for the protection of the village. She is trying to develop a middle path that would allow some exploration without making the location of the village well known to its enemies.
}
{
	\textbf{Air Clerics:} Winter Nest is ruled by clerics of Air and Ice drawn from the leading aarakocra families. The clerics meet in a large hall in Winter Nest to discuss community issues; when there is a particularly contentious debate, the priests adjourn to the very summit of a nearby mountain overlooking the village. There, perched on the ice and surrounded by the sky, the priests of the two faiths pray for guidance together.	
}
{}
{
	\textbf{Air Temple:} The Air Temple is the grandest structure in the village. The temple is built like a huge brazier, with four legs made of massive evergreen tree trunks dragged up from the foothills centuries ago. These tree boles, each more than 30 meters long, are set in the icy ground and canted to nearly join at the tops. There is a concave plate of ice, 6 meters in diameter, held up between the four posts with a hole 2.5 meters in diameter cut in its center. Priests of Air preach from the center of the bowl, while congregants gather on the rim of the bowl and on the perches placed at intervals along the legs.

	\textbf{Ice Temple:} Smaller only to the Air Temple, the Ice Temple (which is basically another word for water at such high altitudes most of the year) is built of large sheets of translucent white and blue ice, layered upon one another to create a five-sided pyramid more than 10 meters tall. The interior is sunken below ground level dug into the glacier so all the worshippers are surrounded by primordial ice throughout the services. Fresh plates of ice are added to the temple throughout the High Sun.
}
{
	\item Few in Winter Nest took much notice when a roc landed on a perch overlooking the village. Two days later the roc has been joined by a dozen more of his kind. The large birds rarely move from their perches, but their menacing presence is unnerving the aarakocra of Winter Nest.
	\item The wind patterns around Winter Nest have changed drastically. A dangerous downdraft has developed making any attempt at flying from the village fraught with peril. Town elders are puzzled by this sudden change, and have forbidden all but the strongest, most agile fliers from leaving Winter Nest. Air clerics are calling for a sacrifice to appease the air spirits, but the town elders want to understand what is going on before they decide, and the local druid who communes with the spirit of the land has disappeared.
	\item The aarakocra of Winter Nest tell tales of a wise old aviarag named Vocia that lives in a cave near the base of the White Mountains. The noble beast has not been seen for three years. Templars from Eldaarich, intent on plundering Vocia's lair, have been spotted approaching the cave. Traaka needs volunteers to warn Vocia. Unfortunately Vocia has passed away due to old age, leaving the PCs to defend her cave, as well as her remains, which the templars wish to plunder.
	\item A defiler has polymorphed himself into an aarakocra and infiltrated Winter Nest, seeking to gain some of the knowledge from the preservers of Winter Nest. His defiling is having an adverse affect on the ice sculpted portions of Winter Nest's buildings. If he is not unmasked soon, one or more buildings in the community may collapse.
	\item Some of the more adventuresome young aarakocra enjoy a deadly challenge. They know of a lair of an air drake on one of the other mountains in the White Mountain range. To show their bravery they occasionally sneak into the lair and come back with a scale or other souvenir. The act is not as dangerous as it sounds, since the aarakocra know the air drake's migration pattern and typically know when it is not in this particular lair. Some of these youths could challenge the PCs to try this stunt, but unfortunately for the PCs the air drake has returned to the lair earlier than expected in order to lay eggs.
	\item A heavily armed Tsalaxan caravan has arrived at the foot of the White Mountains from Draj. The Tsalaxans seem to be trying to reach Winter Nest but the steep mountain sloop prevents them from approaching from below. They have not given up and continue to search for some path up the mountain to Winter Nest. The aarakocra believe the Tsalaxans are raiders and wish to avoid them. The Winter Nesters know their village cannot be reached except through the air, and are not concerned that the Tsalaxans will be able to reach the village. However, Traaka wishes to determine the caravan master's true intentions in case the aarakocra are mistaken. PC allies of the aarakocra could infiltrate the caravan while not obviously tying directly back to the aarakocra.
}
