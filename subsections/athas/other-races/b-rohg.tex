\subsectionA{B'rohg}
B'rohg are a species of four-armed humanoids akin to giants. They have burnt orange skin, the result of having spent their lives on the hot deserts of Athas. They have sharp, angled features, a flat nose, and pointed ears located towards the backside of their skull. Some b'rohg are bald on top, but do have hair (which they will grow to waist-length) growing from the back of their heads. B'rohg have no facial hair. A b'rohg's garb is simple and well-suited to his primitive lifestyle.

Simpleminded, nomadic hunters and gatherers, b'rohg tend to keep to themselves, only attacking when they feel threatened. Bands have been known to turn to raiding in desperate situations. B'rohg are often captured and put to work in the arena as gladiators. Some are even tricked into service with promises of food and sweetmeats but this is the exception rather than the rule. Most b'rohg are not intelligent enough to remember their friends, let alone long-term promises.

\subsubsection{B'rohg Society}
Dominated by their strongest males, b'rohg are a throwback to simpler times. They live in small nomadic bands comprised of 1--4 family units called cliques, consisting of one male, one or two females, and generally fewer than four total offspring. The strongest b'rohg in a band will primarily be hunters, while the older, weaker members and the children are gatherers and water bearers.

Although they are treated like animals by their owners, enslaved b'rohg still possess enough cunning to be able to escape from time to time. These ``renegade'' b'rohg often have developed considerable skill with weaponry and combat techniques during their time in captivity and are viewed with concern as a result b'rohg communicate through a series of grunts and hand signals. Adult b'rohg stand about 5 meters high.

\Figure{t}{images/b-rohg-2.png}

B'rohg have no mastery of fire but do not fear it. Neither suspicious nor superstitious, b'rohg are reactionary when magic is used in their presence. Depending on previous experiences with spellcasters, the creatures may be awed or angered. When encountering magic for the first time, their reaction tends to be one of curiosity (until the spellcaster's intent---malevolent or benevolent---has been established).

B'rohg typically live to be 80 years of age but seldom do because of the harshness of their environment and a high mortality rate among their young. They also have no real understanding of death and will ignore objects or creatures that do not display signs of life (although they have learned that ``playing dead'' is a tactic sometimes used by their foes and will repeatedly strike felled enemies, just to be sure).

Their nomadic lifestyle calls for periods of movement followed by periods of rest. While on the move, the adults carry the few belongings they have and their children in simple sleds made from skins or leathers stretched across a triangle of wooden poles. Each adult drags a single sled across the ground to the next temporary settlement. Once in an area fresh for further hunting and gathering, the group settles down, forming small hovels out of their sleds and additional skins. Where possible, the hovels use existing rocks and crevices to serve as walls or additional rooms, respectively. The group always selects an easily defended position over any others.

While they are too stupid to learn more contemporary speech, others can learn the grunt and sign language of the b'rohg.

\subsubsection{B'rohg Racial Traits}
\begin{itemize*}
    \item +20 Strength, +4 Constitution, $-4$ Intelligence: B'rohg are stupid but very strong and agile.
    \item Giant: B'rohg are creatures with the giant type.
    \item Huge: $-2$ penalty to Armor Class, $-2$ penalty on attack rolls, $-8$ penalty on \skill{Hide} checks, +8 bonus on grapple checks, lifting and carrying limits quadruple those of Medium characters.
    \item B'rohg occupy a space of 4.5 meters and have a reach of 4.5 meters.
    \item B'rohg base land speed is 15 meters.

    \item Low-Light Vision: B'rohg can see twice as far as a human in moonlight and similar conditions of poor illumination, retaining the ability to distinguish color and detail.

    \item Racial Hit Dice: A b'rohg begins with 6 levels of giant, which provide 6d8 Hit Dice, a base attack bonus of +4, and base saving throw bonuses of Fort +5, Ref +2 and  Will +2.
    \item Racial Skills: A b'rohg's giant levels give it skill points equal to 9 $\times$ (2 + Int modifier). Its class skills are \skill{Climb}, \skill{Listen}, \skill{Spot}, and \skill{Survival}.
    \item Racial Feats: A b'rohg's giant levels give it 3 feats.
    \item Weapon Proficiency: A b'rohg is proficient with simple and martial weapons as well as its natural weaponry.

    \item Natural Armor: +5 natural armor bonus to AC.
    \item Natural Weapons: 4 slams (1d6).

    \item Rock Catching (Ex): Once per round, a b'rohg that would normally be hit by a rock can make a Reflex save to catch a rock (or similar projectile) as a free action. The DC is 15 for a Small rock, 20 for a Medium one, and 25 for a Large one (if the projectile provides a magical bonus on attack rolls, the DC increases by that amount.) The b'rohg must be ready for and aware of the attack in order to make a rock catching attempt.
    \item Rock Throwing (Ex): Adult b'rohg are accomplished rock throwers and receive a +1 racial bonus on attack rolls when throwing rocks. A b'rohg can hurl rocks weighing 30 to 40 kilograms each (Medium objects) up to five range increments. The size of the range increment is 42 meters.

    \item Automatic Languages: B'rohg.
    \item Favored Class: Scout.
    \item Level Adjustment: +4.
\end{itemize*}
