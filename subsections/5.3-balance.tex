\Skill{Balance}{Dex; Armor Check Penalty}
\textbf{Check:} You can walk on a precarious surface. A successful check lets you move at half your speed along the surface for 1 round. A failure by 4 or less means you can't move for 1 round. A failure by 5 or more means you fall. The difficulty varies with the surface, as follows:

\Table{Balance DCs}{X Z{1cm} X Z{1cm}}{
\tableheader Narrow Surface & \tableheader Balance DC & \tableheader Difficult Surface & \tableheader Balance DC\\
7-12 inches wide & 10 & Uneven flagstone & 10\\
2-6 inches wide & 15 & Hewn stone floor & 10\\
Less than 2 inches wide & 20 & Sloped or angled floor & 10
}

\Table{Narrow Surface Modifiers}{X R}{
\tableheader Surface & \tableheader DC Modifier \\
Lightly obstructed & +2\\
Severely obstructed & +5\\
Lightly slippery & +2\\
Severely slippery & +5\\
Sloped or angled & +2
}

For narrow surfaces, use the DC given on \tabref{Balance DCs} and add modifiers from \tabref{Narrow Surface Modifiers}, as appropriate. Those modifiers stack.

Make checks for difficult surfaces only if running or charging. Failure by 4 or less means the character can't run or charge, but may otherwise act normally.

\textit{Being Attacked while Balancing:} You are considered flat-footed while balancing, since you can't move to avoid a blow, and thus you lose your Dexterity bonus to AC (if any). If you have 5 or more ranks in Balance, you aren't considered flat-footed while balancing. If you take damage while balancing, you must make another Balance check against the same DC to remain standing.

\textit{Accelerated Movement:} You can try to walk across a precarious surface more quickly than normal. If you accept a $-5$ penalty, you can move your full speed as a move action. (Moving twice your speed in a round requires two Balance checks, one for each move action used.) You may also accept this penalty in order to charge across a precarious surface; charging requires one Balance check for each multiple of your speed (or fraction thereof) that you charge.

\textbf{Action:} None. A Balance check doesn't require an action; it is made as part of another action or as a reaction to a situation.

\textbf{Special:} If you have the Agile feat, you get a +2 bonus on Balance checks.

\textbf{Synergy:} If you have 5 or more ranks in \skill{Tumble}, you get a +2 bonus on Balance checks.