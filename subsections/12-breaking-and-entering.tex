\subsection{Breaking And Entering}
When attempting to break an object, you have two choices: smash it with a weapon or break it with sheer strength.

\subsubsection{Smashing an Object}
Smashing a weapon or shield with a slashing or bludgeoning weapon is accomplished by the sunder special attack. Smashing an object is a lot like sundering a weapon or shield, except that your attack roll is opposed by the object’s AC. Generally, you can smash an object only with a bludgeoning or slashing weapon.

\textbf{Armor Class:} Objects are easier to hit than creatures because they usually don’t move, but many are tough enough to shrug off some damage from each blow. An object’s Armor Class is equal to 10 + its size modifier + its Dexterity modifier. An inanimate object has not only a Dexterity of 0 (-5 penalty to AC), but also an additional -2 penalty to its AC. Furthermore, if you take a full-round action to line up a shot, you get an automatic hit with a melee weapon and a +5 bonus on attack rolls with a ranged weapon.

\textbf{Hardness:} Each object has hardness---a number that represents how well it resists damage. Whenever an object takes damage, subtract its hardness from the damage. Only damage in excess of its hardness is deducted from the object’s hit points (see \tabref{Common Armor, Weapon, and Shield Hardness and Hit Points}; \tabref{Substance Hardness and Hit Points}; and \tabref{Object Hardness and Hit Points}).

\Table{Common Armor, Weapon, and Shield Hardness and Hit Points}{l C C}{
\tableheader Weapon or Shield & \tableheader Hardness & \tableheader HP\\
\textit{Light blade} &&\\
~ bone & 6 & 1\\
~ metal & 10 & 2\\
~ stone & 8 & 1\\
~ wood & 5 & 1\\
\textit{One-handed blade} &&\\
~ bone & 6 & 2\\
~ metal & 10 & 5\\
~ stone & 8 & 3\\
~ wood & 5 & 2\\
\textit{Two-handed blade} &&\\
~ bone & 6 & 4\\
~ metal & 10 & 10\\
~ stone & 8 & 5\\
~ wood & 5 & 4\\
\textit{Light weapon} &&\\
~ bone-hafted & 6 & 2\\
~ metal-hafted & 10 & 10\\
~ stone-hafted & 8 & 3\\
~ wood-hafted & 5 & 2\\
\textit{One-handed weapon} &&\\
~ bone-hafted & 6 & 5\\
~ metal-hafted & 10 & 20\\
~ stone-hafted & 8 & 8\\
~ wood-hafted & 5 & 5\\
\textit{Two-handed weapon} &&\\
~ bone-hafted & 6 & 10\\
~ stone-hafted & 8 & 15\\
~ wood-hafted & 5 & 10\\
Projectile weapon & 5 & 5\\
Armor & special & armor bonus $\times5$\\
Buckler & 10 & 5\\
\textit{Light shield} &&\\
~ wooden & 5 & 7\\
~ steel & 10 & 10\\
\textit{Heavy shield} &&\\
~ wooden & 5 & 15\\
~ steel & 10 & 20\\
Tower shield & 5 & 20\\
}

Athasians use a variety of materials in constructing everyday items, many of which are uncommon in other campaign worlds. Doors made from giant pieces of chitin, ceilings made of mekillot rib truss with leather stretched in-between...

\Table{Substance Hardness and Hit Points}{l C R}{
\tableheader Substance & \tableheader Hardness & \tableheader HP\\
Paper or cloth & 0 & 2/inch of thickness\\
Rope & 0 & 2/inch of thickness\\
Rope, giant hair & 2 & 5/inch of thickness\\
Glass & 1 & 1/inch of thickness\\
Ice & 0 & 3/inch of thickness\\
Leather or hide & 2 & 5/inch of thickness\\
Chitin & 3 & 5/inch of thickness\\
Wood & 5 & 10/inch of thickness\\
Bone or nen & 6 & 10/inch of thickness\\
Rahn-rath & 7 & 12/inch of thickness\\
Dasl & 7 & 15/inch of thickness\\
Stone & 8 & 15/inch of thickness\\
Living crystal & 8 & 15/inch of thickness\\
Drake ivory & 10 & 30/inch of thickness\\
Iron or steel & 10 & 30/inch of thickness\\
Gray-forged steel & 15 & 30/inch of thickness\\
Dwarven steel & 20 & 40/inch of thickness\\
}

Table: Object Hardness and Hit Points
Object	Hardness	Hit Points	Break DC
Rope (1 inch diam.)	0	2	23
Simple wooden door	5	10	13
Small chest	5	1	17
Good wooden door	5	15	18
Treasure chest	5	15	23
Strong wooden door	5	20	23
Masonry wall (1 ft. thick)	8	90	35
Hewn stone (3 ft. thick)	8	540	50
Chain	10	5	26
Manacles	10	10	26
Masterwork manacles	10	10	28
Iron door (2 in. thick)	10	60	28

\textbf{Hit Points:} An object’s hit point total depends on what it is made of and how big it is (see \tabref{Common Armor, Weapon, and Shield Hardness and Hit Points}; \tabref{Substance Hardness and Hit Points}; and \tabref{Object Hardness and Hit Points}). When an object’s hit points reach 0, it’s ruined.

Very large objects have separate hit point totals for different sections.

\textit{Energy Attacks:} Acid and sonic attacks deal damage to most objects just as they do to creatures; roll damage and apply it normally after a successful hit. Electricity and fire attacks deal half damage to most objects; divide the damage dealt by 2 before applying the hardness. Cold attacks deal one-quarter damage to most objects; divide the damage dealt by 4 before applying the hardness.

\textit{Ranged Weapon Damage:} Objects take half damage from ranged weapons (unless the weapon is a siege engine or something similar). Divide the damage dealt by 2 before applying the object’s hardness.

\textit{Ineffective Weapons:} Certain weapons just can’t effectively deal damage to certain objects.

\textit{Immunities:} Objects are immune to nonlethal damage and to critical hits.

Even animated objects, which are otherwise considered creatures, have these immunities because they are constructs.

\textit{Magic Armor, Shields, and Weapons:} Each +1 of enhancement bonus adds 2 to the hardness of armor, a weapon, or a shield and +10 to the item’s hit points.

\textit{Vulnerability to Certain Attacks:} Certain attacks are especially successful against some objects. In such cases, attacks deal double their normal damage and may ignore the object’s hardness.

\textit{Damaged Objects:} A damaged object remains fully functional until the item’s hit points are reduced to 0, at which point it is destroyed.

Damaged (but not destroyed) objects can be repaired with the Craft skill.

\textbf{Saving Throws:} Nonmagical, unattended items never make saving throws. They are considered to have failed their saving throws, so they always are affected by spells. An item attended by a character (being grasped, touched, or worn) makes saving throws as the character (that is, using the character’s saving throw bonus).

Magic items always get saving throws. A magic item’s Fortitude, Reflex, and Will save bonuses are equal to 2 + one-half its caster level. An attended magic item either makes saving throws as its owner or uses its own saving throw bonus, whichever is better.

\textit{Animated Objects:} Animated objects count as creatures for purposes of determining their Armor Class (do not treat them as inanimate objects).

\subsubsection{Breaking Items}
When a character tries to break something with sudden force rather than by dealing damage, use a Strength check (rather than an attack roll and damage roll, as with the sunder special attack) to see whether he or she succeeds. The DC depends more on the construction of the item than on the material.

If an item has lost half or more of its hit points, the DC to break it drops by 2.

Larger and smaller creatures get size bonuses and size penalties on Strength checks to break open doors as follows: Fine -16, Diminutive -12, Tiny -8, Small -4, Large +4, Huge +8, Gargantuan +12, Colossal +16.

A crowbar or portable ram improves a character’s chance of breaking open a door.
