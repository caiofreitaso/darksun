\subsection{Divine Rank}
\label{Divine Rank}
Some epic prestige classes grant the character the divine rank. These characters act like demigods and can grant spells and perform a few deeds that are beyond mortal limits. They have anywhere from a few hundred to a few thousand devoted mortal worshipers and may receive veneration or respect from many more.

All characters with divine rank have all alignment subtypes that correspond with their alignment. Additionally, they have all of the following additional qualities.

\textbf{Hit Points:} Divine characters receive maximum hit points for each Hit Die.

% \textbf{Armor Class:} Divine characters have a deflection bonus to their AC equal to their Charisma bonus (if any).

\textbf{Saving Throws:} Divine characters do not automatically fail on a natural saving throw roll of 1.

\textbf{Immunities:} Divine characters have the following immunities.

\textit{Transmutation:} A divine character is immune to polymorphing, petrification, or any other attack that alters its form. Any shape-altering powers the divine character might have work normally on itself.

\textit{Energy Drain, Ability Drain, Ability Damage:} A divine character is not subject to energy drain, ability drain, or ability damage.

\textit{Mind-Affecting Effects:} A divine character is immune to mind-affecting effects (charms, compulsions, phantasms, patterns, and morale effects).

% \textit{Energy Immunity:} Divine characters are immune to electricity, cold, and acid.

Divine characters are immune to disease and poison, stunning, sleep, paralysis, and death effects, and disintegration.

\textbf{Immortality:} All divine characters are naturally immortal and cannot die from natural causes. Divine characters do not age, and they do not need to eat, sleep, or breathe. The only way for a divine character to die is through special circumstances, usually by being slain in magical or physical combat. Divine characters are not subject to death from massive damage.

% \textbf{Divine Aura:} The mere presence of a divine character can deeply affect mortals. All divine aura effects are mind-affecting, extraordinary abilities. Mortals can resist the aura's effects with successful Will saves; the DC is 10 + \onehalf Hit Dice + Charisma modifier. Any being who makes a successful saving throw against a divine character's aura power becomes immune to that divine character's aura power for one day. Divine aura is an emanation that extends around the divine character in a 15-meter radius. The divine character chooses the size of the radius and can change it as a free action. If the divine character chooses a radius of 0 meters, its aura power effectively becomes non-functional. When two or more divine characters' auras cover the same area, the auras coexist.

% The divine character can make its own worshipers, beings of its alignment, or both types of individuals immune to the effect as a free action. The immunity lasts one day or until the divine character dismisses it. Once affected by an aura power, creatures remain affected as long as they remain within the aura's radius. The divine character can choose from the following effects each round as a free action.

% \textit{Daze:} Affected beings just stare at the divine character in fascination. They can defend themselves normally but can take no actions.

% \textit{Fright:} Affected beings become shaken and suffer a $-2$ morale penalty on attack rolls, saves, and checks. The merest glance or gesture from the divine character makes them frightened, and they flee as quickly as they can, although they can choose the path of their flight.

% \textit{Resolve:} The divine character's allies receive a +4 morale bonus on attack rolls, saves, and checks, while the divine character's foes receive a $-4$ morale penalty on attack rolls, saves, and checks.

\textbf{Grant Spells:} A divine character automatically grants spells and domain powers to mortal divine spellcasters who pray to it. A divine character can withhold spells from any particular mortal as a free action; once a spell has been granted, it remains in the mortal's mind until expended.