\SpellcasterTable[11]{5mm}{The Epic Cleric}{
\SpellHeader{10} \\
21 & +15/+10/+5        & +12 & +7  & +12 & Bonus domain    & 6 & 5+1 & 5+1 & 5+1 & 5+1 & 5+1 & 4+1 & 4+1 & 3+1 & 3+1 & 1 \\
22 & +16/+11/+6/+1     & +13 & +7  & +13 & Spectral strike & 6 & 5+1 & 5+1 & 5+1 & 5+1 & 5+1 & 4+1 & 4+1 & 3+1 & 3+1 & 1 \\
23 & +17/+12/+7/+2     & +13 & +7  & +13 & Bonus domain    & 6 & 5+1 & 5+1 & 5+1 & 5+1 & 5+1 & 4+1 & 4+1 & 3+1 & 3+1 & 1 \\
24 & +18/+13/+8/+3     & +14 & +8  & +14 & Energy aura     & 6 & 5+1 & 5+1 & 5+1 & 5+1 & 5+1 & 4+1 & 4+1 & 3+1 & 3+1 & 1 \\
25 & +18/+13/+8/+3     & +14 & +8  & +14 & Bonus domain    & 6 & 5+1 & 5+1 & 5+1 & 5+1 & 5+1 & 4+1 & 4+1 & 3+1 & 3+1 & 2 \\
26 & +19/+14/+9/+4     & +15 & +8  & +15 &                 & 6 & 5+1 & 5+1 & 5+1 & 5+1 & 5+1 & 4+1 & 4+1 & 3+1 & 3+1 & 2 \\
27 & +20/+15/+10/+5    & +15 & +9  & +15 & Bonus domain    & 6 & 5+1 & 5+1 & 5+1 & 5+1 & 5+1 & 4+1 & 4+1 & 3+1 & 3+1 & 2 \\
28 & +21/+16/+11/+6/+1 & +16 & +9  & +16 &                 & 6 & 5+1 & 5+1 & 5+1 & 5+1 & 5+1 & 4+1 & 4+1 & 3+1 & 3+1 & 2 \\
29 & +21/+16/+11/+6/+1 & +16 & +9  & +16 &                 & 6 & 5+1 & 5+1 & 5+1 & 5+1 & 5+1 & 4+1 & 4+1 & 3+1 & 3+1 & 3 \\
30 & +22/+17/+12/+7/+2 & +17 & +10 & +17 &                 & 6 & 5+1 & 5+1 & 5+1 & 5+1 & 5+1 & 4+1 & 4+1 & 3+1 & 3+1 & 3 \\
}

\subsectionA{Epic Cleric}
The epic cleric is a master of his element and his mastery spreads to all adjacent paraelemental planes to his own elemental plane.

\textbf{Hit Die:} d8.

\textbf{Skill Points per Level:} 2 + Int modifier.

\textbf{Spells:} An epic cleric gain access to psionic enchantments: 10th-level spells that need profound psionic knowledge. In order to cast those psionic enchantments, the epic cleric must have Intelligence and Wisdom scores both equal to at least 20, and be capable of manifesting 9th-level powers. Casting a psionic enchantment requires spending 50 power points, besides the spell slot of 10th-level. The Difficulty Class for a saving throw against a psionic enchantment is 20 + the epic cleric's Wisdom modifier.

\textbf{Bonus Domain:} An epic cleric gains access to adjacent planes (see \tabref{Adjacent Paraelements}). If he is an elemental cleric, he selects an adjacent paraelement to gain access. If he is a paraelemental cleric, he selects an adjacent element.

At 21st level, the epic cleric gains access to a bonus domain from the selected adjacent plane. At 25th level, he gains access to another bonus domain from this first adjacent plane.

At 23rd level, the epic cleric gains access to a bonus domain from a second adjacent plane. For example, an epic Air cleric who selected Rain as his first adjacent plane must select Sun at 23rd level. At 27th level, he gains access to another bonus domain from the second adjacent plane.

\Table{Adjacent Paraelements}{lX||lX}{
  \tableheader Element
& \tableheader Adjacent Paraelements
& \tableheader Element
& \tableheader Adjacent Paraelements \\

Air   & Rain, Sun   & Fire  & Magma, Sun \\
Earth & Magma, Silt & Water & Rain, Silt \\
}

\textbf{Spectral Strike (Su):} At 22nd level, an epic cleric's attacks ignore any chance of not dealing damage against incorporeal creatures. His attacks deal damage normally against incorporeal creatures.

\textbf{Energy Aura (Su):} At 24th level, an epic cleric gains an 4.5-meter aura that automatically affects undead creatures. This doesn't cost a turning or rebuking attempt, and the epic cleric doesn't have to roll turning damage (it automatically affects all undead in a 4.5-meter burst). Just as with normal turning, he can't affect undead that have total cover relative to him.

An epic cleric who turns undead gains a positive energy aura, and every undead creature is affected as if he had turned it. The aura only turns undead with Hit Dice equal to or less than his effective cleric level minus 10 (and automatically destroys undead with Hit Dice equal to or less than his effective cleric level minus 20).

An epic cleric who rebukes undead gains a negative energy aura, and every undead creature is affected as if he had rebuked it. The aura only rebukes undead with Hit Dice equal to or less than his effective cleric level minus 10 (and automatically commands undead with Hit Dice equal to or less than his effective cleric level minus 20).
