\subsection{Divine Feats}
\GFeat[Epic]{Elemental Turning}
{Wis 25, Cha 25, ability to turn or rebuke undead, patron element.}
{
  You can turn or rebuke elementals opposed to your patron element as if they were undead. An elemental has effective turn resistance equal to half its spell resistance (round down). If you can turn undead, you turn (or destroy) all opposed elementals. If you can rebuke undead, you rebuke (or command) all opposed elementals.
}

\GFeat[Epic]{Planar Turning}
{Wis 25, Cha 25, ability to turn or rebuke undead.}
{
  You can turn or rebuke outsiders as if they were undead. An outsider has effective turn resistance equal to half its spell resistance (round down). If you can turn undead, you turn (or destroy) all evil outsiders and rebuke (or command) all nonevil outsiders. If you can rebuke undead, you rebuke (or command) all evil outsiders and turn (or destroy) all nonevil outsiders.
}

\GFeat[Divine, Epic]{Undead Mastery}
{Cha 21, ability to rebuke or command undead.}
{You may command up to ten times your level in HD of undead.}

\GFeat[Divine, Epic]{Zone Of Animation}
{Cha 25, Undead Mastery, ability to rebuke or command undead.}
{You can use a rebuke or command undead attempt to animate corpses within range of your rebuke or command attempt. You animate a total number of HD of undead equal to the number of undead that would be commanded by your result (though you can’t animate more undead than there are available corpses within range). You can’t animate more undead with any single attempt than the maximum number you can command (including any undead already under your command). These undead are automatically under your command, though your normal limit of commanded undead still applies. If the corpses are relatively fresh, the animated undead are zombies. Otherwise, they are skeletons.}
