\subsection{House Tsalaxa}
\textbf{Location:} Draj.

\textbf{Banner:} A pair of glaring, bestial yellow eyes on a black field.

\textbf{Trade Goods:} Hemp, grain, artwork, ornate weapons, and slaves.

\textbf{Troops:} 1,500 soldiers.

The leading merchant house of the city of Draj is infamous throughout the Tyr region. Tsalaxa engages in endless games of espionage and intrigue in order to secure the most valuable trading contracts. This house is well known for its ruthless business practices.

\subsubsection{Brief History}
Tsalaxa has a long and intricate history. After several centuries of existence, no one is certain how much of its origin is fact and how much is mere legend. The Tsalaxa family claims that the house was founded by a powerful mage over 500 years ago; the mage's only aides were loyal half-giants. After a long series of struggles the house finally triumphed, becoming a permanent fixture in the city of Draj.

While Tsalaxa officially adheres to the Merchants' Code and deals reasonably fairly with its customers, it is also widely known as an underhanded den of intriguers and schemers whose spies and assassins are second to none.

Recent conflicts with House Wavir, in which raiders hired by Tsalaxa have attempted to disrupt Wavir's most profitable trade routes, are only part of the story. Tsalaxa is also infamous for its long memory. ``Never cross the dragon or House Tsalaxa'' is a popular proverb in Draj.

\subsubsection{Tsalaxa's Assets}
Tsalaxa maintains outposts in all the major cities of the Tyr region. The exception is Balic, where Houses Wavir and Rees have managed to keep Tsalaxa out. It also administers several villages along the trade routes between Draj, Raam, and Urik.

\textbf{Caravans:} A typical Tsalaxan caravan is five to ten wagons preceded by a dozen or more crodlu-riders. The wagons are usually open, with 2,500 to 5,000-kg capacity. Armored caravans are rarely used. When a slave coffle is present, it drags along behind the main caravan, constantly patrolled by crodlu-riders and whip-wielding mul or human overseers.

\textbf{Facilities:} Tsalaxa's Draj headquarters are disarmingly innocuous---a simple walled villa with a couple of bored-looking guards lounging about. As might be expected, appearances are deceiving. The villa is protected by numerous traps, magical wards, and hidden snipers. The seemingly bored guards are actually elite senior agents. The structure is built atop solid rock, and most of its rooms are located underground.

\textit{Fort Ebon:} 100 employees, 250 slaves. Supply point between Draj and Raam. Ebon is a vital supply link for all of Tsalaxa's caravans.

\textit{Fort Kalvis:} 50 employees, 125 slaves. Supply and storage point and trading post in verdant belt between Gulg and Altaruk.

\textit{Rumish's Rock:} 20 employees, 20 slaves. Outpost and trading post 30 miles southeast of Lost Oasis. A small but important post, this is where Tsalaxa obtains much of its gold via trading with caravans coming east from Walis.

\textit{Ablath:} Trade village, 500 citizens. Located near an oasis 20 miles southwest of Silver Spring. Ablath is Tsalaxa's contact with the tribes of the Tablelands. It is a frequent stopover spot for caravans bound to and from Altaruk.

\subsubsection{House Tsalaxa in Athas}
Tsalaxa's attitude toward competition is that the fit survive, the rest die. Tsalaxa specializes in undercutting rivals by offering premium prices for its goods, losing money for a season or two until opposing houses are eliminated, then tripling or quadrupling prices to make up for the losses. By this time, of course, most of the competition has been eliminated, and the customer is at Tsalaxa's mercy.

Tsalaxa maintains cordial relations with the sorcerer-kings, realizing that their good will is necessary for continued success. House Stel, which often allows Tsalaxa to hire its mercenaries, is the only house with which Tsalaxa maintains a good relationship. Some have suggested that this is not surprising, in that Tsalaxa and Stel are the most unscrupulous and violent merchant houses.

Small houses are ruthlessly exterminated by agents wearing Tsalaxa colors and freely admitting their allegiance. Actions against larger houses are carried out more stealthily, by secret agents and hired mercenaries. When hirelings learn too much about Tsalaxa's operations and inner workings, they may be invited to join the household, but this is often simply to keep them on a short leash. On other occasions, hirelings simply disappear.

\subsubsection{Tsalaxa Lore}
Characters with ranks in \skill{Knowledge} (local [Urik]) or \skill{Gather Information} can research House Tsalaxa to learn more about it. When a character makes a skill check, read or paraphrase the following, including the information from lower DCs.

\textbf{DC 10:} House Tsalaxa is the major merchant house in Draj.

\textbf{DC 15:} Tsalaxa is mainly dedicated in hemp and grain, but also deals in artwork, ornate weapons, and slaves.

\textbf{DC 20:} Tsalaxa caravans are generally built
for speed---small, swift, with numerous riders and
few, if any, wagons. Slave caravans are much slower
and more ponderous than ordinary caravans; these
contain many more wagons and are heavily guard-
ed.

\textbf{DC 25:} Tsalaxa is especially talented at discovering illegal dealings, illicit assignations, or suspicious political views. Such information can be kept secret for a high price. Tsalaxa also avoids being caught in similar traps, so digging up their past might prove both difficult and dangerous.

\textbf{DC 30:} Blackmail, kidnapping, and even assassination have all entered Tsalaxa's arsenal, along with outright military harassment utilizing the services of mercenary raiding tribes.

A bardic knowledge check can reveal the same information as these knowledge checks, but in each case the DC is 5 higher than the given value.

\subsubsection{Agents}
Tsalaxa is always on the lookout for caravan guards and individuals with skills as spies or assassins. Unfortunately, once hirelings learn some of Tsalaxa's secrets, the house is reluctant to let them leave. Keeping such individuals around is accomplished through a combination of rewards and threats.

This house has few qualms about who it hires---defilers, assassins, raiding tribes, even gith, braxat, and anakore, individuals whom more sensible employers avoid like the dragon. Tsalaxa has a tendency to distrust traditional mages because of the supposed influence and threat of the Veiled Alliance.

\textbf{Wages:} 10--25\% higher than average.

\textbf{Additional Requirements:} \skill{Diplomacy} 5 ranks, \skill{Gather Information} 5 ranks, \skill{Spot} 5 ranks; or able to cast two divination spells; or able to manifest two clairsentience powers; or base attack bonus +2.

\textbf{Desired Roles:} Spies, spellcasters focused on divination spells or focused on combat spells, manifesters focused on clairsentience, or any potential raider.