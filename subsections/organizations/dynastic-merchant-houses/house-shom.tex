\subsectionA{House Shom}
\textbf{Location:} Nibenay.

\textbf{Banner:} Three white dragonflies on a red-and-black, diagonally divided field.

\textbf{Trade Goods:} Obsidian, rice, water, wood, as well as a host of minor products, such as weapons and works of art.

\textbf{Troops:} 2,000+ soldiers.

House Shom, the leading trade faction in the city of Nibenay, is an old and (in the eyes of many) corrupt house. Perhaps the decadent atmosphere of Nibenay has affected the house's rulers, making them strange and merciless. Perhaps centuries of sybaritic luxuries have dulled their minds and convinced them that any means of making a profit is acceptable.

\subsubsection{Brief History}
Shom may be the oldest trading house in the Tyr region, with a history that stretches back for over a thousand years. The house's origins are shrouded in mystery and legend, but Nibenay's historians are certain that until eight centuries ago, Shom was a tiny house with a few secure trade routes and little ambition. It was when a young merchant named Kys came to power after the unexplained deaths of both his grandfather and mother that House Shom began to grow in size and influence. Nowadays, as the house's fortunes wane and its rulers grow more isolated, some claim that House Shom's masters have become sorcerer-kings themselves, in all but name.

Kys forged alliances with thri-kreen tribes, and to this day, belgoi are sometimes employed by House Shom---an unpleasant surprise for those who attack Shom's caravans. It is not known how Shom maintains friendly relations with these hostile creatures.


\subsubsection{Shom's Assets}
House Shom's assets are truly impressive. Unfortunately, much of the house's wealth is tied up in massive mansions and extravagant art treasures. The actual volume of trade the house engages in has shrunk steadily over the past few decades.

\textbf{Caravans:} Today, perhaps 20 caravans fly Shom's white dragonflies; these numbers decline each year. A typical Shom caravan employs 20 to 50 kank- or crodlu-mounted outriders, (who usually stick close to the main body of the caravan, thus defeating their purpose), as many as 150 leather-armored infantry, up to ten half-giants, and up to 20 mul gladiators who serve as guards for any family members or senior agents with the caravan. Family members rarely accompany caravans, but when they do, it is in a private mekillot wagon with extensive facilities.

\textbf{Facilities:} Typical outposts feature rather spartan quarters for the mercenary guards and outrageously extravagant quarters for the family member or agent assigned to command.

\textit{Fort Melidor:} 60 employees, 150 slaves. Supply point located near Lost Oasis. Most of the time, the fort sits idle, waiting for family members to visit.

\textit{Fort Inix:} 75 employees, 200 slaves. Supply and storage point located at oasis 45 kilometers east of Nibenay. It is sometimes raided by slaves from Salt View or by various desert tribes, but in general it is a dull place.

\textit{Fort Sunset:} 25 employees. A tiny outpost in the shelter of the Ringing Mountains. It is quite impoverished, but it occasionally services a Shom caravan or fights off an attack by gith.

\textit{Cromlin:} Trade village, 300 citizens. Located on the shore of the Sea of Silt, 45 kilometers west of Giustenal. Cromlin serves as a trading city for nomadic tribes. It also maintains facilities for the repair and storage of silt skimmers, which many trading houses use to cut across the sea and reduce travel time between Nibenay and Raam.


\subsubsection{House Shom in Athas}
House Shom generally does not deign to acknowledge that any other merchant houses even exist; their attitude toward the mighty sorcerer-kings is only slightly better. Aggressive houses, particularly the militaristic House Stel, have engaged in an active campaign of raiding and disruption of Shom's routes between Nibenay and Raam.

Despite its diminished status, Shom maintains good relations with several tribes of thri-kreen, as well as numerous belgoi raiders. Those who attack Shom caravans or outposts occasionally find themselves confronted by unexpected hordes of mantis warriors or belgoi in disguise. 

Outsiders dealing with Shom usually have an easy time if they have enough cash for bribes. Shom agents can be persuaded to almost any course with enough ceramic pieces, although reform-minded agents respond to attempted bribes with disdain or even violence.

\subsubsection{Shom Lore}
Characters with ranks in \skill{Knowledge} (local [Nibenay]) or \skill{Gather Information} can research House Shom to learn more about it. When a character makes a skill check, read or paraphrase the following, including the information from lower DCs.

\textbf{DC 10:} House Shom is the major merchant house in Nibenay.

\textbf{DC 15:} Shom trades mainly in food supplies from the Crescent Forest, but also weapons and art. A typical Shom caravan is large and slow-moving, burdened by far more troops than it really needs, and numerous employees who do little more than take up cargo space.

\textbf{DC 20:} Shom has been shrinking over the past few decades, now they can maintain up to 20 caravans at the same time. Family members are always in the padded, silken confines of their vast mansions and isolated villas.

\textbf{DC 25:} House Shom has good relations with many thri-kreen tribes, even belgoi. But these relations are becoming brittle as the family is directing all their wealth to defend their house against enemies.

\textbf{DC 30:} A few of Shom's younger members have realized the dire straits their house is in, and they have begun to move to counteract this peril.

A bardic knowledge check can reveal the same information as these knowledge checks, but in each case the DC is 5 higher than the given value.

\subsubsection{Agents}
More than any other house, Shom depends on agents and hirelings for its survival. Since the members of the Shom family are loathe to even acknowledge the need for profit and mercantile practices, what remains of House Shom's once-vast influence is in the hands of the few agents who retain a shred of loyalty.

For their part, agents only rarely allow hirelings to rise through the ranks to the position of agent. For this reason, there is a great deal of intrigue and double-dealing among the employees of House Shom.

\textbf{Wages:} Up to double normal rates.

\textbf{Additional Requirements:} Base Fortitude save +3; must not have a power point reserve.

\textbf{Desired Roles:} Fighters, rangers, and gladiators are the most commonly employed. Clerics are also among the most desired roles. Shom's agents are paranoid with manifesters, distrusting anyone who might be able to discern their secrets. Shom has few qualms about hiring defilers, but they are hired only on a short-term basis.