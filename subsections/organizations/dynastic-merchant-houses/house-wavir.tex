\subsection{House Wavir}
\textbf{Location:} Balic.

\textbf{Banner:} A silver jozhal on a blue field.

\textbf{Trade Goods:} Grain, ceramics, and precious metals.

\textbf{Troops:} None.

Under the leadership of its current patriarch, Tabaros, House Wavir has risen far above its humble beginnings to become one of the most powerful merchant houses in the Tyr region. Wavir's business practices border on the ruthless, but they are always conducted in accordance with the Merchants' Code. All profits are reinvested in house operations. House Wavir's enormous wealth and influence have led many to speculate that Tabaros has a secret source of income, but so far no proof of this has been found.

\subsubsection{Brief History}
House Wavir's history is a success story that has few equals. The house began a little over two centuries ago when a freelance adventurer, Traxidos Wavir (now fondly remembered simply as ``The Wavir'') founded the company as a joint venture with several of his adventuring companions.

After a very difficult beginning, it became apparent that House Wavir would not simply disappear. Rival houses began to bow to the inevitable; they allowed Wavir to continue to do business, relatively unmolested. This period (often called ``the blooding'') is typical of new and ambitious merchant houses. Few upstart houses ever survive it. Wavir proved an exception in more ways than one, for Traxidos proceeded to grow wealthy and powerful, laying the foundations of the legendary house that was to survive him. He died peacefully in bed nearly six decades later, safe in the assurance that his descendants would rule over one of the greatest Athasian merchant houses.

Remembering how intrigue and assassins almost destroyed them, Traxidos's descendants remain ever vigilant. Their guards and agents are trained in the arts of counter espionage and detection. In addition, House Wavir seems to value competition as a means of maintaining healthy markets (and also for the challenge it represents), and it rarely harasses or attacks smaller trading houses. In fact, many claim that they have received loans and other assistance from House Wavir, particularly if the smaller house's activities serve to frustrate Wavir's rivals.

\subsubsection{Wavir's Assets}
Wavir is acutely aware of its origins as a two-caravan house. It understands that only good luck and vigilant business practices have brought it this far. Young members of the family are encouraged to work the caravan routes or help staff outposts; high-ranking family members expect little of the luxury and sybaritic privileges extended to other merchant house members.

Another of House Wavir's features is its deep and abiding hatred of slavery. Wavir was founded by a former slave, and future generations were raised to feel that slavery was an evil thing. Many of the house's employees are former slaves who wholeheartedly support this position. 

\textbf{Caravans:} At any given time, there are 40 to 50 caravans that fly the Wavir blue and silver, loaded with every imaginable commodity. Caravans are generally well guarded, with both elf mercenaries and crodlu- or kank-riders as scouts. Numerous lightly armed archers accompany the wagons and pack inix. Wavir uses armored caravan wagons on long journeys, or when the cargo being carried is extremely valuable.

\textbf{Facilities:} Wavir's headquarters in Balic rivals the palaces of many sorcerer-kings. Well-paid mercenaries patrol the outer walls, while trusted mages and psionicists appear from time to time to check for surreptitious or magical entry.

Other bases and outposts beyond Balic are organized along similar lines, under the command of a family member or senior agent. Magical, psionic, and military defenses are quite impressive, as Wavir's concern for security often borders on the paranoid. Wavir facilities are almost unique in that they utilize no slave labor. For this reason, Wavir's forts tend to have larger staffs, as all menial labor is performed by employees.

Along with former rival houses Rees and Tomblador, Wavir maintains the major trading village of Altaruk, southeast of Tyr.

\textit{Fort Glamis:} 150 employees. Supply point at junction of Balic/Ledopolus road. An important crossroads between Balic and the rest of the region.

\textit{Fort Thetis:} 75 employees. Supply point and trading post at southern end of mountains, 90 kilometers east of Walis. Wavir completely controls this gold route between Balic and Walis. This fortress is a frequent target of attacks by gith and human tribes, as well as by trading rivals, such as House Tsalaxa.

\textit{Outpost Ten:} 15 employees. Trading post located on western edge of boulder field, 70 miles southwest of Tyr. Wavir's outposts have numbers rather than names. This small fort lies on the very edge of the Forest Ridge. Here Wavir maintains tenuous trading connections with the halfling savages, who trade hardwood, gems, and exotic animals for gold, spice, and weapons.

\textit{Outpost 19:} 20 employees. Outpost at northeastern end of Mekillot Mountains. Here Wavir supplies caravans and trades with the former slaves of Salt View. Wavir often trades weapons and other vital items to the slaves at unprofitable rates, but continues to do so because of its hatred of slavery.

\subsubsection{House Wavir in Athas}

\subsubsection{Wavir Lore}
Characters with ranks in \skill{Knowledge} (local [Balic]) or \skill{Gather Information} can research House Wavir to learn more about it. When a character makes a skill check, read or paraphrase the following, including the information from lower DCs.

\textbf{DC 10:}

\textbf{DC 15:}

\textbf{DC 20:}

\textbf{DC 25:} Some of Wavir's rivals have gone so far as to accuse the house of fostering and supporting slave revolts in contravention of the Merchants' Code, which forbids direct intervention in partisan matters. Nothing has ever been proven.

\textbf{DC 30:}

A bardic knowledge check can reveal the same information as these knowledge checks, but in each case the DC is 5 higher than the given value.

\subsubsection{Agents}


\textbf{Wages:}

\textbf{Additional Requirements:}

\textbf{Desired Roles:}