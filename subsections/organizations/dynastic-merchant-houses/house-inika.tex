\subsection{House Inika}
\textbf{Location:} Gulg.

\textbf{Banner:} A plain gold circle on a black field.

\textbf{Trade Goods:} Small and valuable items such as Kola nuts, spices, gems, feathers.

\textbf{Troops:} 500 soldiers (5th level or higher).

Inika operates out of its headquarters in the city of Gulg. It is small compared to some of the other major houses, but this is by choice. Dealing in small, valuable cargoes, such as kola nuts, exotic feathers, spices, and gemstones, Inika sees the benefit in remaining small and efficient. As a result, the house rakes in profits far out of proportion to its size, gaining a reputation for being one of the shrewdest houses in the region.

\subsubsection{Brief History}
Over three centuries ago, in the year of the Desert's Fury, in the city of Gulg, Taro Inika, a trusted agent of House Riben, broke his merchant's oath and left his employers, starting his own house. Only the good will of Biria Riben, the house's matriarch who had once been Inika's lover, prevented the new house from being crushed within a year.

In Gulg, rivalries between trade houses are kept to a minimum to maintain the peace and tranquility of the city. This proved a rich environment for the new concern. Within a few years, the house assumed its current form: a streamlined business dealing in small but valuable cargoes.

The House Inika's major philosophies are that force is to be used only as a last resort, and that strategic withdrawal to a superior position is often the best route.

\subsubsection{Inika's Assets}
Gulg's relative tranquility and the regularity of its production provides House Inika with a regular income. Wise investment and conservative spending combine to keep Inika on a solid financial footing.

\textbf{Caravans:} Inika's caravans are small and fast. Elven scouts provide advance warning of ambushes or attacks, which are usually avoided rather than confronted. Cargo is carried almost exclusively on kanks, although inix are sometimes used for larger cargoes. Mekillots and wagons are almost never found in Inika caravans.

Average caravans include about a dozen elven scouts with up to 20 kanks and (rarely) four to six inix. Due to the caravans' small size, House Inika can afford to maintain many more than other merchant houses. At any one time, Inika can have 50 to 75 caravans carrying cargo across the region.

Inika caravans' tactics are, first and foremost, to avoid any enemy forces. Should this fail, they scatter to break up pursuit. The scouts are expected to seek out the dispersed kank riders and bring them back together after the attack has ended.

\textbf{Facilities:} House Inika maintains very few permanent installations, preferring to temporarily rent or lease space in villages, forts, or cities.

\textit{Fort Adros:} 75 employees, 150 slaves. Supply point along the gold route between Walis and Altaruk. Often a target of attacks by elves and gith.

\textit{Fort Harbeth:} 50 employees, 100 slaves. A major trading post for the slaves of Salt View and a place to purchase dwarven items from Ledopolus.

\textit{Fort Skonz:} 50 employees, 75 slaves. Supply point at junction of three roads between Tyr, Altaruk, and Silver Spring.

\textit{Shazlim:} Trade village, 500 citizens. Located along the southern edge of Dragon's Bowl between Raam and Silver Spring, Shazlin represents an important stopover and trade point in the area.

\subsubsection{House Inika in Athas}
House Inika stays out of trouble. This is not to say that Inika never engages in intrigue or double-dealing; it's just very difficult to catch Inika at it. Such conduct is usually directed only at those who have done wrong by Inika. House Inika does not get mad, but it does get even.

In dealing with other houses and the sorcerer-kings, Inika takes a nonconfrontational approach. If challenged for control of an important route or commodity, Inika resists strongly to persuade its opponents to commit time and resources to the conflict. Inika then withdraws suddenly, changes tactics, and ends up putting pressure on the opponents in a totally unexpected quarter.

\subsubsection{Inika Lore}
Characters with ranks in \skill{Knowledge} (local [Gulg]) or \skill{Gather Information} can research the House of the Mind to learn more about it. When a character makes a skill check, read or paraphrase the following, including the information from lower DCs.

\textbf{DC 10:} House Inika is the major merchant house in Gulg.

\textbf{DC 15:} Inika is dedicated in trading spices, gemstones and other luxury items.

\textbf{DC 20:} Caravans are small and fast, in order to minimize the losses from raids. Becoming an agent for Inika is safe and profitable, even being their slave is not (that) bad---just don't cross them in any way.

\textbf{DC 25:} Their tactic is to stay out of trouble from other houses.  If it comes to it, they hit where it hurts. 

\textbf{DC 30:} House Wavir in Balic is counted as an ally by House Inika. Most others at least have nothing against House Inika, though trade rivalries are common. House Tsalaxa harbors bad feelings toward Inika, but Inika's reputation prevents Tsalaxa from taking overt action.

A bardic knowledge check can reveal the same information as these knowledge checks, but in each case the DC is 5 higher than the given value.

\subsubsection{Agents}
% Inika is far from a generous employer, preferring to lavish its wealth on trusted agents. Hirelings can expect to work for average pay or less, and they should not anticipate long-term employment unless their performance is exemplary.
Should a hireling rise through the ranks to become a valued employee, however, House Inika may offer the ultimate honor---a permanent position as an agent. Particularly successful agents may be invited to join the family itself, but this is very rare.

Agents are well cared for, due in part to their small numbers. Inika is reluctant to share its prosperity with too many house members, so its agents are few but highly skilled. Inika cares for its agents and their families from the cradle to the grave, on the wise thought that such a major investment is not lightly set aside.

% Scouts, spies, kank riders, and lightly armed, fast-moving fighters are favored as hirelings. Mages and psionicists are very high priced and are employed only in dire need.
\textbf{Wages:} Hirelings receive average pay, at most. Agents are well cared for all their needs.

\textbf{Additional Requirements:} Base attack bonus +6, \skill{Diplomacy} 13 ranks.

\textbf{Desired Roles:} Scouts, spies, kank riders, and lightly armed, fast-moving fighters. Mages and psionicists only in dire need.