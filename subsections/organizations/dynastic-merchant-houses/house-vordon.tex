\subsection{House Vordon}
\textbf{Location:} Tyr.

\textbf{Banner:} A black diamond on a red-brown field.

\textbf{Trade Goods:} Iron, food, water, and kank nectar.

\textbf{Troops:} 2,000 soldiers officially. Over 1,000 more through dummy houses.

Once one of the most feared and respected of Athas's great houses, House Vordon has become something of a laughingstock. This transformation is mainly due to the demented excesses of Tyr's sorcerer-king Kalak. Kalak diverted valuable resources to the construction of a massive ziggurat and all but bankrupted his city.

\subsubsection{Brief History}
House Vordon originated in the ancient city of Kalidnay, strategically located astride the vital trade route between the gold mines of Walis and the hungry markets of Tyr and Gulg. Exactly what caused the fall of this great city is not known today, but the patriarch of House Vordon apparently had some warning of the catastrophe. He and his retainers managed to escape just before the disaster, bearing what wealth they could.

Relocated in Tyr, the house struggled in dire financial straits for years, slowly rebuilding its shattered empire. When iron was discovered in the mountains surrounding Tyr, all of the city's merchants prospered, including Vordon. With increased prosperity came respect. Vordon's offices and outposts sprang up throughout the region. Soon even the hardiest desert raider came to realize that Vordon's caravans and warriors were forces to be reckoned with. Vordon gobbled up numerous smaller houses, even displacing the venerable House Krosi of Urik as the chief carrier of iron and obsidian. By the time sorcerer-king Kalak ascended to the throne, House Vordon was the richest and most influential merchant dynasty in the region.

For a thousand years this situation prevailed and House Vordon grew powerful and arrogant. But in recent years, as Tyr's vast resources were diverted to constructing Kalak's ziggurat, House Vordon's fortunes declined. Trade goods were sold to obtain cash for materials, slaves were requisitioned and set to work on the monument, and the entire city suffered. Soon, Vordon had become an object of derision and ridicule by other houses, who chortled at Kalak's senility and Vordon's troubles.

\subsubsection{Vordon's Assets}
Financially strapped and on the verge of starvation, Tyr was forced to import vast quantities of food, water, and kank nectar. Vordon helped to provide the city with these items, but a chronic cash shortage within Tyr kept their profits low.

\textbf{Caravans:} House Vordon operates at least 20 caravans at a time, while its dummy houses control a dozen more. Their caravans are large and well guarded. Twenty or more armored crodlu riders provide cover, while as many as 50 bow-armed footmen accompany the five to ten armored wagons that make up the caravan proper.

\textbf{Facilities:} Outposts are maintained mostly along the Tyr-Altaruk, Tyr-Urik, and Tyr-Balic routes, which have traditionally been the house's most profitable. Other outposts have been abandoned, but some are kept operating even at a loss, or are leased out to other houses simply to bring in profits.

\textit{Fort Amber:} 75 employees, 100 slaves. Supply and storage point between Tyr and Altaruk. Also intended as a refuge for the household if the chaos in Tyr becomes too great, Amber is well stocked with supplies and secret cash reserves hidden in underground vaults.

\textit{Fort Thamo:} 50 employees, 125 slaves. Supply point and trading post between Grak's Pool and South Ledopolus. This fortress maintains an important link with the south, particularly Balic.

\textit{Mira's Halo:} 20 employees, 30 slaves. Outpost located in sandy wastes between Tyr and Urik. Named for an unusual rock formation nearby, this outpost is officially owned by House Qual, one of Thaxos's dummy trade houses.

\subsubsection{House Vordon in Athas}
Vordon was once held in enormous esteem by other trade houses---the object of both fear and respect. Other houses stayed scrupulously out of Vordon's way. The sorcerer-kings themselves disliked the house but quickly acknowledged its importance. Vordon earned this respect through a combination of deep business sense, ruthless efficiency, and sheer luck.

Today, much of the awe that other houses felt for Vordon has evaporated with Vordon's fortunes. No longer do other houses bow and scrape in Vordon's presence. As the larger houses aggressively subvert Vordon's influence along the major caravan routes, the smaller houses nibble away at less well-defended assets. Vordon scrambles to repair the damage.

\subsubsection{Vordon Lore}
Characters with ranks in \skill{Knowledge} (local [Tyr]) or \skill{Gather Information} can research House Vordon to learn more about it. When a character makes a skill check, read or paraphrase the following, including the information from lower DCs.

\textbf{DC 10:} House Vordon is the major merchant house in Tyr.

\textbf{DC 15:} Vordon trades iron from Tyr. Before the ziggurat, Vordon used to trade in slaves, artwork, and textiles, but these have virtually been eliminated. Now, they import food, water, and kank nectar.

\textbf{DC 20:} Although slaves have officially been freed within Tyr, Thaxos continues to keep slaves in his facilities outside the city; he has come up with numerous excuses for not freeing them.

\textbf{DC 25:} The patriarch Thaxos began diverting house resources into the creation of several dummy merchant houses secretly controlled by House Vordon.

\textbf{DC 30:} Thaxos secretly desired to take the throne for himself, elevating himself from simple merchant prince to powerful sorcerer-king. He also started to assemble a secret army, with which he intended to seize power ``in the name of the oppressed people of Tyr.''


A bardic knowledge check can reveal the same information as these knowledge checks, but in each case the DC is 5 higher than the given value.

\subsubsection{Agents}
To outsiders, House Vordon has seemed to be retrenching recently. It appears to be hiring very few, if any, new employees. In reality, large numbers of warriors and spellcasters are being hired, but Thaxos employs them through his dummy houses.

When Thaxos feels the time is right, those new employees who have distinguished themselves in the service of his dummy houses will be told of the coming conflict. These select hirelings will be offered valuable rewards in exchange for their continued loyalty. Thaxos intends these individuals, along with his most trusted agents, to serve as the backbone of the templars who will serve him when he is king.

\textbf{Wages:} Up to 50\% more than normal wages.

\textbf{Additional Requirements:} Base attack bonus +3; or \skill{Knowledge} (any two) 5 ranks; or \skill{Bluff} 5 ranks, \skill{Hide} 5 ranks, \skill{Spot} 5 ranks.

\textbf{Desired Roles:} Caravan guards, accountants, spies, and any magical support.