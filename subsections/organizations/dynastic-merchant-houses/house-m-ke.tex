\subsection{House M'ke}
\textbf{Location:} Raam.

\textbf{Banner:} A silver quill pen on a red field.

\textbf{Trade Goods:} Metals, food, weapons, and obsidian.

\textbf{Troops:} 600 soldiers.

As chief house of the troubled city of Raam, M'ke maintains a precarious balance between the reviled Great Vizier and the restive populace. The house's reputation has fallen somewhat in recent years due to the near-anarchy in Raam, but M'ke has fallen back on its vast cash reserves to see it through.

\subsubsection{Brief History}
House M'ke's origins are somewhat mysterious. Some claim that the house originated in an unknown area beyond the Tyr region, while others believe that M'ke was founded by dissident templars fleeing their sorcerer-king's wrath. Whatever its origin, M'ke became a force to be reckoned with.

M'ke's subsequent behavior has been an interesting combination of ruthlessness and caution. They treat more powerful rivals with intense respect, but they steal trade routes and even engage in open combat with weaker houses. Despite their occasional vicious streak, members of House M'ke are generally considered pleasant and intelligent individuals with few bad habits. This is only one of many bizarre contrasts in the confusing, contradictory House M'ke.

Trade by the house has shrunk of late, with M'ke withdrawing to secure citadels and outposts and adopting a fortress mentality to withstand the chaos in Raam. Truvo M'ke realizes that this means the future will hold intense struggles to regain old routes, but the house has been equal to the task in the past.

\subsubsection{M'ke's Assets}
M'ke's greatest asset is its hidden wealth, much of which is contained in the vaults beneath the family palace in Raam. Other caches of wealth are rumored to be scattered throughout the Tyr region as a hedge against future reverses.

\textbf{Caravans:} M'ke can maintain only ten to 20 caravans at any one time. A typical caravan consists of 20-30 crodlu-mounted scouts, six to ten medium-sized wagons and, on especially important caravans, up to four well-defended armored caravan wagons. Crossbow-armed troops ride in the caravans, while foot soldiers or slaves walk alongside to provide further protection.

\textbf{Facilities:} House M'ke's holdings have shrunk along with their caravan routes. The house maintains small offices, often with skeleton staffs, in most major cities of the Tyr region.

\textit{Fort Firstwatch:} 30 employees, 50 slaves. Storage and supply point between Raam and Draj. Frequent target of raids by trade rivals and elf nomads.

\textit{Fort Isus:} 50 employees, 100 slaves. Supply point and trading post between Raam and Nibenay. Popular spot for trading with nomadic tribes.

\textit{Fort Xalis:} 100 employees, 150 slaves. Major trading post, supply point, and military base near Black Waters, between Raam and Urik. Much of M'ke's reduced military manpower has been concentrated here, because the obsidian and slave routes between Raam and Urik are vital to Raam's survival and M'ke's continued prosperity.

\textit{Jalaka:} Trade village, 250 citizens. Despite difficult terrain and massive logistical problems, M'ke continues to maintain this village, located on the edge of the Forest Ridge approximately 60 kilometers north of Tyr.

\subsubsection{House M'ke in Athas}
Powerful rivals are treated with respect and even friendship, while weaker houses are treated with contempt. M'ke's mercenaries have few qualms about raiding the caravans of small houses or even assaulting their headquarters or warehouses in major cities. Perhaps the best way for a small house to deal with M'ke is to put up a bold front, for M'ke will pounce at the slightest sign of weakness.

M'ke also treats the sorcerer-kings with great respect, even going so far as to maintain strict neutrality while the rest of Raam's populace calls for the Great Vizier's overthrow. M'ke never engages in smuggling, and it is particularly friendly with the kings of Draj and Urik.

\subsubsection{M'ke Lore}
Characters with ranks in \skill{Knowledge} (local [Raam]) or \skill{Gather Information} can research House M'ke to learn more about it. When a character makes a skill check, read or paraphrase the following, including the information from lower DCs.

\textbf{DC 10:} House M'ke is the major merchant house in Raam.

\textbf{DC 15:} M'ke trades in food and weapon supplies.

\textbf{DC 20:} They can't maintain many caravans at the same time, 20 at most. Becoming an agent for M'ke is busy work---and dangerous. They pay much better than most, but it is dangerous to switch houses.

\textbf{DC 25:} M'ke is shrinking lately, but they still maintains a halfling village. The reasons for this persistence are the profits to be gained from the hardwood found in the forest and the unique trade goods from the halflings of the forest. Some also suggest that Jalaka harbors a major cache of M'ke's wealth, though no specifics of this alleged hoard are known.

\textbf{DC 30:} Houses Vordon and Tsalaxa have been applying pressure on M'ke of late, urging it to join the rebels seeking to overthrow the Great Vizier, but so far nothing has come of this.

A bardic knowledge check can reveal the same information as these knowledge checks, but in each case the DC is 5 higher than the given value.

\subsubsection{Agents}
Unfortunately, the mortality rate among M'ke employees, who are expected to die rather than betray or compromise their masters, is quite high. Employees who transfer loyalty (when offered higher wages, for example) are dealt with harshly. Hirelings must obtain permission from the top in order to change employers.

Work with M'ke is high-paying but hazardous, and new employees can expect constant action, including dealing with raids from outside tribes, nobles, and even rival merchant houses.

\textbf{Wages:} 25--50\% more higher wages than average. Special bonuses are paid for particularly resourceful or skilled acts in battle.

\textbf{Additional Requirements:} \skill{Bluff} 5 ranks, \skill{Diplomacy} 5 ranks; or ability to manifest 1st-level psionic powers.

\textbf{Desired Roles:} Bards, rogues, and psionicists are often employed; raamite templars have been known to offer services as guards, spies, or spellcasters.