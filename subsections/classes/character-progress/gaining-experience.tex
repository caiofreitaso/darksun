\subsection{Gaining Experience}
Award players for combat, skills, and role-playing. Combat can be inevitable and emerging victorious in a combat should earn experience points. Skills are the way of dealing with problems without violence. Like combat, awards are only for successful attempts.

There are also some guidelines to award players experience points for their roleplay. These are based on their race, so that they are incentivized to follow the racial culture.

\subsubsection{Awards per Race}
These awards are for roleplaying some of the stereotypical aspects of athasian character races. Players should remember that races are more than just stats on their character sheets, they have culture and history which are what these stereotypes try to enforce.

The judgment of good roleplaying ultimately lies with the DM, and they must be familiar with the nuances of the character races. The communication between the DM and the players should be clear so that a good roleplaying experience can arise, and the nature of \textbf{Dark Sun} can be emphasized.

To determine the XP award for a particular action, check the correspondent table for the character's race and multiply the award by the character's effective character level (ECL). Remember that this will quicken the rate of progress of the characters.

\textbf{Aarakocra:} Aarakocras are claustrophobic. Whenever not entering a closed space becomes a hard choice, aarakocras should be rewarded for choosing their phobia over the closed space. This should not be awarded when the stakes for not entering are low.

Aarakocras are a flying race. Whenever aarakocras enter buildings through windows because they were flying high, they should get experience bonus.

\XPTable{Aarakocras}{
Enter building through window & 25 XP \\
Refuse to enter closed spaces & 100 XP \\
}

\textbf{Dwarf:} Dwarves live by their foci. A focus must take at least a week to complete. If a focus takes a least a year to complete, it becomes a major focus.

Focus can be changed in very rare circumstances. These circumstances must be agreed between the player and the DM.

\XPTable{Dwarves}{
Pursue present focus & 50 XP $\times$ days pursuing \\
Ignore present focus & $-100$ XP $\times$ days ignoring \\
Complete major focus & 1,000 XP \\
}

\textbf{Elf:} Roleplaying an elf is centered around trust. Elves are self-reliant and do not want to gain friendship with every character they meet. They test redeemable outsiders (in the elvish perspective) to see if they are trustworthy.

\textit{Examples of subtle tests of trust:}
\begin{itemize*}
	\item entrust with confidential information,
	\item leave a valuable item easy for taking to see the outsider takes it,
	\item ask to deliver a message or item.
\end{itemize*}

\textit{Examples of life-threatening tests of trust:}
\begin{itemize*}
	\item let themselves get captured to see if there is a rescue attempt,
	\item fake unconsciousness after a battle to see what type of care is provided,
	\item cut supplies to see if they get a fair share.
\end{itemize*}

\XPTable{Elves}{
Subtle test of trust & 25 XP \\
Life-threatening test of trust & 200 XP \\
Refuse animal or magical transport & 50 XP \\
Continuous run & 10 XP $\times$ distance in km \\
}

\textbf{Half-Elf:} Every half-elf seek acceptance among humans and elves, t hough they deny it as much as possible. Observing simple customs for the first time should award bonus experience points. These can be drinking the local ale with the elven chieftain or participating in a human wedding ritual.

If a local custom takes form of a competition, the half-elf gains bonus experience points if they perform better than any \emph{one} of the humans or elves also participating. If they perform better than \emph{all} the humans or elves, they get double the experience award.

\XPTable{Half-Elves}{
Observe human or elven custom & 25 XP \\
Better a human or elf in custom & 250 XP \\
}

\textbf{Half-Giant:} A half-giant seeks guidance and purpose in others' lifestyles. Player characters should seek to imitate the most charismatic member of the party in their racial and class customs. Whenever they do, they should be rewarded for it.

Half-giant can also look elsewhere for inspiration, imitating non-player characters and may even switch sides in an adventure. Whenever a player goes this far, they should get bonus experience points.

Whenever half-giants shift their alignment based on the events in the campaign, the DM should give them bonus experience points. This is only for appropriate shifts that are followed by a meaningful roleplay.

\XPTable{Half-Giants}{
Imitate charismatic friend & 25 XP $\times$ days imitating \\
Shift alignment per influence & 50 XP \\
}

\textbf{Halfling:} Halflings come from isolated tribes and, similar to half-elves, they want to experiment other races' customs. Unlike half-elves, their drive is curiosity, instead of trying to fit in. Whenever halflings try a  custom for the first time, no matter how trivial, they gain an experience bonus.

Their sense of belonging makes them honor bound to aid another halfling in need. This should only be rewarded when there is danger of injury or loss of life to the aiding halfling.

\XPTable{Halflings}{
Refuse money & 25 XP \\
Practice another race's custom & 50 XP \\
Eat slain foe & 50 XP \\
Aid another halfling & 100 XP \\
}

\textbf{Mul:} As a race bred exclusively for slavery, muls lack a culture similar to the other races. What they all share is the culture of labor. Whenever they exert themselves, they should be awarded bonus experience. This should only be rewarded if the exertion is meaningful to the adventure.

\XPTable{Muls}{
Heavy exertion & 10 XP $\times$ day of work \\
}

\textbf{Pterran:} Pterran culture venerates Earth Mother, so whenever pterrans celebrate her they should be rewarded. This also means that defilers are natural foes to pterrans, since they desecrate Earth Mother.

As a vibrant culture, they are curious to witness other cultures---as long as they don't harm the Earth Mother.

\XPTable{Pterrans}{
Celebration for Earth Mother & 25 XP\\
Practice another race's custom & 50 XP \\
Defeat defiler & 50 XP $\times$ defiler's CR\\
}

\textbf{Thri-Kreen:} Thri-kreen come from an empire of hunter-gatherers. Whenever they take back a slain creature for food, it should warrant experience bonus.

\XPTable{Thri-Kreens}{
Defeat creature for food & 50 XP \\
Paralyze creature & 100 XP \\
}

% \subsubsection{Awards per Class}
% Multiclass characters must choose which class to consider when receiving awards---you can only gain award for a single class.

% \textbf{\class{Barbarian}:} Barbarians are survivalists, so beyond defeating living creatures, defeating traps and natural hazards on their might alone award them bonus experience points.

% \XPTable{Barbarians}{
% Use special attack & 5 XP\\
% Use rage & 10 XP\\
% Defeat a creature & 10 XP $\times$ creature's CR\\
% Defeat trap or natural hazard & 10 XP $\times$ trap's CR\\
% }

% \textbf{\class{Bard}:} Bards gain bonus experience points for successful use of their bardic abilities. However they also gain XP for using poison against a creature---to weaken or kill the victim.

% \XPTable{Bards}{
% Use bardic music & 10 XP\\
% Use bardic knowledge & 25 XP\\
% Use poison effectively & 5 XP $\times$ creature's CR\\
% Defeat a creature & 5 XP $\times$ creature's CR\\
% Obtain treasure & 5 XP $\times$ value in cp\\
% }

% \textbf{\class{Cleric}:} Using elements with finesse and flair to overcome an obstacle should reaward clerics bonus experience points.

% \XPTable{Clerics}{
% Use domain power & 10 XP\\
% Use element creatively & 50 XP\\
% Cast spell & 5 XP $\times$ spell level\\
% Cast Healing spell & 10 XP $\times$ spell level\\
% Turn/rebuke undead & 25 XP $\times$ undead's CR\\
% Destroy/command undead & 50 XP $\times$ undead's CR\\
% Create magic item & 200 XP\\
% }

% \textbf{\class{Druid}:}

% \XPTable{Druids}{
% Cast spell & 5 XP $\times$ spell level\\
% Case Healing spell & 10 XP $\times$ spell level\\
% Use wild empathy & 25 XP\\
% Use wild shape & 10 XP\\
% Defeat defiler & 50 XP $\times$ defiler's CR\\
% Create magic item & 200 XP\\
% }

% \textbf{\class{Fighter}:} The fighter's role in society is about mass warfare, so being a good soldier during these times will award her additional experience points. Fighters do not gain experience points for spending weeks in reserve, even if they're commanding followers.

% \XPTable{Fighters}{
% Use special attack & 5 XP\\
% Defeat a creature alone & 5 XP $\times$ creature's CR\\
% Defeat a creature with a group & 10 XP $\times$ creature's CR\\
% Follow commands in battle & 25 XP\\
% Command a battle & 50 XP\\
% Build a war machine & 100 XP\\
% }

% \textbf{\class{Gladiator}:} Gladiators desire to be in the spotlight of an arena, to be the victor of a duel. Therefore, they receive additional experience points for defeating creatures in an arena without outside aid. The glory must be their alone.

% \XPTable{Gladiators}{
% Use special attack & 5 XP\\
% Use gladiatorial perfomance & 10 XP\\
% Defeat a creature & 5 XP $\times$ creature's CR\\
% Defeat a creature alone in an arena & 10 XP $\times$ creature's CR\\
% }

% \textbf{\class{Psion}:}

% \XPTable{Psions}{
% Defeat a psionic creature & 5 XP $\times$ creature's CR\\
% Research new psionic knowledge & 50 XP\\
% Manifest a power & 5 XP $\times$ power level\\
% Manifest a power to avoid combat & 10 XP $\times$ power level\\
% Create psionic item & 200 XP\\
% }

% \textbf{\class{Psychic Warrior}:}

% \XPTable{Psychic Warriors}{
% Defeat a creature & 5 XP $\times$ creature's CR\\
% Defeat a psionic creature & 5 XP $\times$ creature's CR\\
% Manifest a power & 5 XP $\times$ power level\\
% }

% \textbf{\class{Ranger}:} Rangers track their foes and hunt favored enemies. Accomplishing these taks award them additional experience points.

% \XPTable{Ranger}{
% Cast spell & 5 XP $\times$ spell level\\
% Defeat a creature & 5 XP $\times$ creature's CR\\
% Defeat a creature in a favorite territory & 5 XP $\times$ creature's CR\\
% Defeat a favorite enemy & 10 XP $\times$ enemy's CR\\
% Track a creature & 10 XP\\
% Track a creature in a favorite territory & 10 XP\\
% Track a favorite enemy & 10 XP\\
% Use wild empathy & 25 XP\\
% }

% \textbf{\class{Rogue}:}

% \XPTable{Rogue}{
% Defeat a creature & 5 XP $\times$ creature's CR\\
% Disable trap & 25 XP $\times$ trap's CR\\
% Obtain treasure & 5 XP $\times$ value in cp\\
% Obtain treasure for patron & 5 XP $\times$ value in cp\\
% Use sneak attack & 5 XP\\
% }

% \textbf{\class{Templar}:}

% \XPTable{Templar}{
% Cast spell & 5 XP $\times$ spell level\\
% Use secular authority on slave & 5 XP\\
% Use secular authority on freeman & 10 XP\\
% Use secular authority on noble & 25 XP\\
% Use secular authority on templar & 50 XP\\
% Fulfill sorcerer-king's mission & 100 XP\\
% }

% \textbf{\class{Wilder}:}

% \XPTable{Wilders}{
% Defeat a psionic creature & 5 XP $\times$ creature's CR\\
% Create psionic item & 200 XP\\
% Surge without enervation & 10 XP $\times$ bonus level\\
% Manifest a power & 5 XP $\times$ power level\\
% Manifest a power to avoid combat & 10 XP $\times$ power level\\
% }

% \textbf{\class{Wizard}:} Preservers and defilers serve different roles in the athasian society. Preservers want to remain hidden from the eyes of the templars, so they gain additional experience points for keeping spellcasting secret. Defilers on the other hand are aligned with the \emph{status quo} and are awarded additional experience points for carrying out the business of their sorcerer-monarchs.

% \XPTable{Wizards}{
% Cast spell & 5 XP $\times$ spell level\\
% Cast spell for sorcerer-king (defiler) & 5 XP $\times$ spell level\\
% Find new spell to spellbook & 50 XP $\times$ spell level\\
% Keep spellcasting secret (preserver) & 25 XP\\
% Create magic item & 200 XP\\
% }

\subsubsection{Awards for Skill Check}
Skill checks represent nonviolent challenges in the game. To determine the XP award for a successful skill check, follow these steps:

\begin{enumerate*}
	\item Calculate the adjusted Difficulty Class (DC) of the check. It is equal to the skill check's DC minus any racial bonus the character has in that check.
	\item Use \tabref{Experience Awards from Skill Checks} to cross-reference the campaign's pace with the adjusted DC to find the XP award.
\end{enumerate*}
 
Only successful checks award XP. Checks with adjusted DC lower than 10 do not give XP. The award is not based on the check's result.

Opposed checks do not have Difficulty Class. For the purposes of experience awards, they are considered to have a DC equal to a roll of 12. For example, a barbarian with a Hide modifier of +4 would be awarded 50 XP if they pass a \skill{Hide} check DC 16---regardless of the result of the barbarian's \skill{Hide} check or any opposed \skill{Spot} check.

\Table{Experience Awards from Skill Checks}{L *5C}{
\rowcolor{white}
\multirow[c]{2}{1cm}{\tableheader Adjusted DC} & \multicolumn{5}{c}{\tableheader Experience Points Award}\\
\cmidrule[0.5pt]{2-6}
& \tableheader Rushed & \tableheader Fast & \tableheader Normal & \tableheader Slow & \tableheader Dragged \\

10--11 & 25 & 15 & 10 & 8 & 5 \\
12--14 & 60 & 30 & 25 & 20 & 15 \\
15--19 & 120 & 60 & 50 & 40 & 30 \\
20--24 & 250 & 120 & 100 & 80 & 60 \\
25--29 & 600 & 300 & 250 & 200 & 150 \\
30--39 & 1,200 & 600 & 500 & 400 & 300 \\
40--49 & 2,400 & 1,200 & 1,000 & 800 & 600 \\
50+ & 6,000 & 3,000 & 2,500 & 2,000 & 1,500 \\
}

\subsubsection{Awards for Combat}
When the players defeat the enemy in battle, they earn experience points. Only characters who take part in the battle should gain XP. Characters who died before the combat, or did not participate for any other reason, should not be awarded.

To determine the XP award for a combat, follow these steps.
\begin{enumerate*}
	\item Determine each character's effective character level (ECL).
	\item For each monster or trap defeated, determine its Challenge Rating (CR).
	\item Use \tabref{Experience Awards from Combats} to cross-reference the campaign's pace with the CR of each monster or trap to find the base XP award.
	\item Divide the base XP award by the number of characters in the party.
	\item Add up all the XP awards for all the monsters or traps the character helped defeat.
	\item Repeat the process for each character.
\end{enumerate*}

Creatures summoned or that otherwise are related to an enemy's ability (such as animal companion) do not award XP. These abilities are already taken into account in the enemy's CR.

\Table{Experience Awards from Combats}{X *{5}{C}} {
\rowcolor{white}
\multirow[l]{2}{1cm}{\tableheader Challenge Rating} & \multicolumn{5}{c}{\tableheader Experience Points Award (per monster)}\\
\cmidrule[0.5pt]{2-6}
& \tableheader Rushed & \tableheader Fast & \tableheader Normal & \tableheader Slow & \tableheader Dragged \\

1 & 1,200 & 600 & 500 & 400 & 300 \\
2 & 2,400 & 1,200 & 1,000 & 800 & 600 \\
3 & 3,600 & 1,800 & 1,500 & 1,200 & 900 \\
4 & 4,800 & 2,400 & 2,000 & 1,600 & 1,200 \\
5 & 6,000 & 3,000 & 2,500 & 2,000 & 1,500 \\
6 & 7,200 & 3,600 & 3,000 & 2,400 & 1,800 \\
7 & 8,400 & 4,200 & 3,500 & 2,800 & 2,100 \\
8 & 9,600 & 4,800 & 4,000 & 3,200 & 2,400 \\
9 & 10,800 & 5,400 & 4,500 & 3,600 & 2,700 \\
10 & 12,000 & 6,000 & 5,000 & 4,000 & 3,000 \\
11 & 13,200 & 6,600 & 5,500 & 4,400 & 3,300 \\
12 & 14,400 & 7,200 & 6,000 & 4,800 & 3,600 \\
13 & 15,600 & 7,800 & 6,500 & 5,200 & 3,900 \\
14 & 16,800 & 8,400 & 7,000 & 5,600 & 4,200 \\
15 & 18,000 & 9,000 & 7,500 & 6,000 & 4,500 \\
16 & 19,200 & 9,600 & 8,000 & 6,400 & 4,800 \\
17 & 20,400 & 10,200 & 8,500 & 6,800 & 5,100 \\
18 & 21,600 & 10,800 & 9,000 & 7,200 & 5,400 \\
19 & 22,800 & 11,400 & 9,500 & 7,600 & 5,700 \\
20 & 24,000 & 12,000 & 10,000 & 8,000 & 6,000 \\
}
