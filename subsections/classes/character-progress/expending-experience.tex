\subsection{Expending Experience}
Characters gain experience and that experience may be represented as a new level. But that is not the only way to mark character growth. Here we present some alternate rules for expending experience points. These options may be adequate for adventures that don't strive to get into high levels. Maybe you want for the characters to stay in the same level for longer than normal. With these options, they can get meaningful progression without receiving an actual level.

In \tabref{Partial Progression}, you can see the benefits and their costs. Some benefits are intentionally left out of the table, such as general feats or extra spell slots. Those benefits are too close to a new level, so it would be more meaningful to give an actual level.

\Table{Partial Progression}{XX}{
\tableheader Benefit & \tableheader XP Cost (per ECL)\\
1 skill point & 100 XP\\
1 skill feat & 200 XP\\
1 racial feat & 300 XP\\
1 regional feat & 300 XP\\
1 item creation feat & 500 XP\\
}

\textbf{Skill Point:} You gain one skill point, just as the skill points gained by acquiring a new level. You can't have more ranks than you are allowed for your level. Spending this skill point on a cross-class skill still gets your character \onehalf rank in that skill. For more information on skill points, see \chapref{Skills}.

\textbf{Feats:} You may purchase a new feat, granted you have its prerequisites. You can only purchase some types of feats: item creation feats, racial feats, regional feats, and skill feats. For more information on these types of feats, see \chapref{Feats}.
