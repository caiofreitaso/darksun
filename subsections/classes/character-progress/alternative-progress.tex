\subsection{Alternative Progress}
Here are presented two alternative ways to change the experience progress in your campaign: partial progression and tiered powers. Each variant rule tries to delay (or halt) level progress. You are not required to use any of those variant rules to run a {\tableheader Dark Sun} adventure, since these are designed to tackle different problems and different tastes.

\subsubsection{Partial Progression}
Characters gain experience and that experience may be represented as a new level. But that is not the only way to mark character growth. Here we present some alternate rules for expending experience points. These options may be adequate for adventures that don't strive to get into high levels. Maybe you want for the characters to stay in the same level for longer than normal. With these options, they can get meaningful progression without receiving an actual level.

In \tabref{Partial Progression}, you can see the benefits and their costs. Some benefits are intentionally left out of the table, such as general feats or extra spell slots. Those benefits are too close to a new level, so it would be more meaningful to give an actual level.

\Table{Partial Progression}{XX}{
\tableheader Benefit & \tableheader XP Cost (per ECL)\\
1 skill point        & 100 XP\\
1 skill feat         & 200 XP\\
1 racial feat        & 300 XP\\
1 regional feat      & 300 XP\\
1 item creation feat & 500 XP\\
}

\textbf{Skill Point:} You gain one skill point, just as the skill points gained by acquiring a new level. You can't have more ranks than you are allowed for your level. Spending this skill point on a cross-class skill still gets your character \onehalf rank in that skill. For more information on skill points, see \chapref{Skills}.

\textbf{Feats:} You may purchase a new feat, granted you have its prerequisites. You can only purchase some types of feats: item creation feats, racial feats, regional feats, and skill feats. For more information on these types of feats, see \chapref{Feats}.

\subsubsection{Tiered Powers}
There is a natural imbalance towards magical (and psionic) powers. So much so that the whole system is balanced around wonderous items, and it is expected to have spellcasters in an adventuring group. But at higher levels, this imbalance in power can become frustrating for both DMs and players, as the powers become increasingly more complex and game-changing.

In \tabref{Tiered Powers}, you can see the level adjustment based on the maximum spell level or power level available for a manifester or spellcaster. This means that a 17th-level human wizard should be equivalent to a 19th-level character---she has access to 9th-level magic and thus has a +2 level adjustment. An elf psychic warrior of the same level (17th) should be equivalent to a 18th-level character---6th-level psionic powers grant +1 level adjustment.

\Table{Tiered Powers}{lXc}{
\tableheader Tier & \tableheader Maximum Spell/Power Level Available & \tableheader Total Level Adjustment \\
I   & 1st--3rd & +0 \\
II  & 4th--6th & +1 \\
III & 7th--9th & +2 \\
}

In an already running adventure, a character that will change their tier should only advance another level after gaining enough experience for two whole levels. For instance, Jozan is a 6th-level human cleric. In order for him to get his 7th level (and 4th-level spells), he must gain enough XP to advance to an effective 8th character level. Once the character makes a tier breakthrough, he advances normally with his level adjustment and new effective character level until another breakthrough must be made. So from his 7th level to 12th level, Jozan advances with a +1 level adjustment. After he becomes a 12th-level cleric (13 ECL), he should again acquire enough XP to advance two character levels in order to gain access to 7th-level spells---this time to an effective 15th character level. After that, he advances normally again with his +2 level adjustment.