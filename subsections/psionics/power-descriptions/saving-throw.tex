\subsection{Saving Throw}
Usually a harmful power allows a target to make a saving throw to avoid some or all of the effect. The Saving Throw line in a power description defines which type of saving throw the power allows and describes how saving throws against the power work.

\textbf{Negates}: The power has no effect on a subject that makes a successful saving throw.

\textbf{Partial}: The power causes an effect on its subject, such as death. A successful saving throw means that some lesser effect occurs (such as being dealt damage rather than being killed).

\textbf{Half}: The power deals damage, and a successful saving throw halves the damage taken (round down).

\textbf{None}: No saving throw is allowed.

\textbf{(object)}: The power can be manifested on objects, which receive saving throws only if they are psionic or if they are attended (held, worn, grasped, or the like) by a creature resisting the power, in which case the object uses the creature's saving throw bonus unless its own bonus is greater. (This notation does not mean that a power can be manifested only on objects. Some powers of this sort can be manifested on creatures or objects.) A psionic item's saving throw bonuses are each equal to 2 + one-half the item's manifester level.

\textbf{(harmless)}: The power is usually beneficial, not harmful, but a targeted creature can attempt a saving throw if it desires.

\textbf{Saving Throw Difficulty Class}: A saving throw against your power has a DC 10 + the level of the power + your key ability modifier (Intelligence for a psion, Wisdom for a psychic warrior, or Charisma for a wilder). A power's level can vary depending on your class. Always use the power level applicable to your class.

\textbf{Succeeding on a Saving Throw}: A creature that successfully saves against a power that has no obvious physical effects feels a hostile force or a tingle, but cannot deduce the exact nature of the attack. Likewise, if a creature's saving throw succeeds against a targeted power you sense that the power has failed. You do not sense when creatures succeed on saves against effect and area powers.

\textbf{Failing a Saving Throw against Mind-Affecting Powers}: If you fail your save, you are unaware that you have been affected by a power.

\textbf{Automatic Failures and Successes}: A natural 1 (the d20 comes up 1) on a saving throw is always a failure, and the power may deal damage to exposed items (see Items Surviving after a Saving Throw, below). A natural 20 (the d20 comes up 20) is always a success.

\textbf{Voluntarily Giving up a Saving Throw}: A creature can voluntarily forego a saving throw and willingly accept a power's result. Even a character with a special resistance to psionics can suppress this quality.

\textbf{Items Surviving after a Saving Throw}: Unless the descriptive text for the power specifies otherwise, all items carried or worn by a creature are assumed to survive a psionic attack. If a creature rolls a natural 1 on its saving throw against the effect, however, an exposed item is harmed (if the attack can harm objects). Refer to Table: Items Affected by Psionic Attacks.

Determine which four objects carried or worn by the creature are most likely to be affected and roll randomly among them. The randomly determined item must make a saving throw against the attack form or take whatever damage the attack deals.

\Table{Items Affected by Psionic Attacks}{lX}{
\tableheader Order\footnotemark[1] & \tableheader Item\\
1st & Shield\\
2nd & Armor\\
3rd & Psionic or magic helmet, or psicrown\\
4th & Item in hand (including weapon, dorje, or the like)\\
5th & Psionic or magic cloak\\
6th & Stowed or sheathed weapon\\
7th & Psionic or magic bracers\\
8th & Psionic or magic clothing\\
9th & Psionic or magic jewelry (including rings)\\
10th & Anything else\\
\rowcolor{white}
\multicolumn{2}{l}{1 In order of most likely to least likely to be affected.}\\
}