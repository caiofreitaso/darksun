\subsection{Aiming A Power}
You must make some choice about whom the power is to affect or where the power's effect is to originate, depending on the type of power. The next entry in a power description defines the power's target (or targets), its effect, or its area, as appropriate.

\textbf{Target or Targets}: Some powers have a target or targets. You manifest these powers on creatures or objects, as defined by the power itself. You must be able to see or touch the target, and you must specifically choose that target. However, you do not have to select your target until you finish manifesting the power.

If you manifest a targeted power on the wrong type of target the power has no effect. If the target of a power is yourself (the power description has a line that reads ``Target: You''), you do not receive a saving throw and power resistance does not apply. The Saving Throw and Power Resistance lines are omitted from such powers.

Some powers can be manifested only on willing targets. Declaring yourself as a willing target is something that can be done at any time (even if you're flat-footed or it isn't your turn). Unconscious creatures are automatically considered willing, but a character who is conscious but immobile or helpless (such as one who is bound, cowering, grappling, paralyzed, pinned, or stunned) is not automatically willing. The Saving Throw and Power Resistance lines are usually omitted from such powers, since only willing subjects can be targeted.

\textbf{Effect}: Some powers, such as most metacreativity powers, create things rather than affect things that are already present. Unless otherwise noted in the power description, you must designate the location where these things are to appear, either by seeing it or defining it. Range determines how far away an effect can appear, but if the effect is mobile, it can move regardless of the power's range once created.

\textit{Ray}: Some effects are rays. You aim a ray as if using a ranged weapon, though typically you make a ranged touch attack rather than a normal ranged attack. As with a ranged weapon, you can fire into the dark or at an invisible creature and hope you hit something. You don't have to see the creature you're trying to hit, as you do with a targeted power. Intervening creatures and obstacles, however, can block your line of sight or provide cover for the creature you're aiming at.

If a ray power has a duration, it's the duration of the effect that the ray causes, not the length of time the ray itself persists.

If a ray power deals damage, you can score a critical hit just as if it were a weapon. A ray power threatens a critical hit on a natural roll of 20 and deals double damage on a successful critical hit.

\textit{Spread}: Some effects spread out from a point of origin (which may be a grid intersection, or may be the manifester) to a distance described in the power. The effect can extend around corners and into areas that you can't see. Figure distance by actual distance traveled, taking into account turns the effect may take. When determining distance for spread effects, count around walls, not through them. As with movement, do not trace diagonals across corners. You must designate the point of origin for such an effect (unless the effect is centered on you), but you need not have line of effect (see below) to all portions of the effect.

\textit{(S) Shapeable}: If an Effect line ends with ``(S)'' you can shape the power. A shaped effect can have no dimension smaller than 3 meters.

\textbf{Area}: Some powers affect an area. Sometimes a power description specifies a specially defined area, but usually an area falls into one of the categories defined below.

Regardless of the shape of the area, you select the point where the power originates, but otherwise you usually don't control which creatures or objects the power affects. The point of origin of a power that affects an area is always a grid intersection. When determining whether a given creature is within the area of a power, count out the distance from the point of origin in squares just as you do when moving a character or when determining the range for a ranged attack. The only difference is that instead of counting from the center of one square to the center of the next, you count from intersection to intersection.

You can count diagonally across a square, but every second diagonal counts as 2 squares of distance. If the far edge of a square is within the power's area, anything within that square is within the power's area. If the power's area touches only the near edge of a square, however, anything within that square is unaffected by the power.

\textit{Burst, Emanation, or Spread}: Most powers that affect an area function as a burst, an emanation, or a spread. In each case, you select the power's point of origin and measure its effect from that point. A burst power affects whatever it catches in its area, even including creatures that you can't see. It can't affect creatures with total cover from its point of origin (in other words, its effects don't extend around corners). The default shape for a burst effect is a sphere, but some burst powers are specifically described as cone-shaped.

A burst's area defines how far from the point of origin the power's effect extends.

An emanation power functions like a burst power, except that the effect continues to radiate from the point of origin for the duration of the power.

A spread power spreads out like a burst but can turn corners. You select the point of origin, and the power spreads out a given distance in all directions. Figure the area the power effect fills by taking into account any turns the effect takes.

\textit{Cone, Line, or Sphere}: Most powers that affect an area have a particular shape, such as a cone, line, or sphere. A cone-shaped power shoots away from you in a quarter-circle in the direction you designate. It starts from any corner of your square and widens out as it goes. Most cones are either bursts or emanations (see above), and thus won't go around corners.

A line-shaped power shoots away from you in a line in the direction you designate. It starts from any corner of your square and extends to the limit of its range or until it strikes a barrier that blocks line of effect. A line-shaped power affects all creatures in squares that the line passes through or touches.

A sphere-shaped power expands from its point of origin to fill a spherical area. Spheres may be bursts, emanations, or spreads.

\textit{Other}: A power can have a unique area, as defined in its description.

\textbf{Line of Effect}: A line of effect is a straight, unblocked path that indicates what a power can affect. A solid barrier cancels a line of effect, but it is not blocked by fog, darkness, and other factors that limit normal sight. You must have a clear line of effect to any target that you manifest a power on or to any space in which you wish to create an effect. You must have a clear line of effect to the point of origin of any power you manifest.

A burst, cone, or emanation power affects only an area, creatures, or objects to which it has line of effect from its origin (a spherical burst's center point, a cone-shaped burst's starting point, or an emanation's point of origin). An otherwise solid barrier with a hole of at least 30 centimeters square through it does not block a power's line of effect. Such an opening means that the 1.5-meter length of wall containing the hole is no longer considered a barrier for the purpose of determining a power's line of effect.