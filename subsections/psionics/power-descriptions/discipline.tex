\subsection{Discipline (Subdiscipline)}
Beneath the power name is a line giving the discipline (and the subdiscipline in parentheses, if appropriate) that the power belongs to.

Every power is associated with one of six disciplines. A discipline is a group of related powers that work in similar ways. Each of the disciplines is discussed below.

\subsubsection{Clairsentience}
Clairsentience powers enable you to learn secrets long forgotten, to glimpse the immediate future and predict the far future, to find hidden objects, and to know what is normally unknowable.

For the purpose of psionics-magic transparency, clairsentience powers are equivalent to powers of the divination school (thus, creatures immune to divination spells are also immune to clairsentience powers).

Many clairsentience powers have cone-shaped areas. These move with you and extend in the direction you look. The cone defines the area that you can sweep each round. If you study the same area for multiple rounds, you can often gain additional information, as noted in the descriptive text for the power.

\textbf{Scrying:} A power of the scrying subdiscipline creates an invisible sensor that sends you information. Unless noted otherwise, the sensor has the same powers of sensory acuity that you possess. This includes any powers or effects that target you, but not powers or effects that emanate from you. However, the sensor is treated as a separate, independent sensory organ of yours, and thus functions normally even if you have been blinded, deafened, or otherwise suffered sensory impairment. Any creature with an Intelligence score of 12 or higher can notice the sensor by making a DC 20 Intelligence check. The sensor can be dispelled as if it were an active power. Lead sheeting or psionic protection blocks scrying powers, and you sense that the power is so blocked.

\subsubsection{Metacreativity}
Metacreativity powers create objects, creatures, or some form of matter. Creatures you create usually, but not always, obey your commands.

A metacreativity power draws raw ectoplasm from the Astral Plane to create an object or creature in the place the psionic character designates (subject to the limits noted above). Objects created in this fashion are as solid and durable as normal objects, despite their originally diaphanous substance. Psionic creatures created with metacreativity powers are considered constructs, not outsiders.

A creature or object brought into being cannot appear inside another creature or object, nor can it appear floating in an empty space. It must arrive in an open location on a surface capable of supporting it. The creature or object must appear within the power's range, but it does not have to remain within the range.

For the purpose of psionics-magic transparency, metacreativity powers are equivalent to powers of the conjuration school (thus, creatures immune to conjuration spells are also immune to metacreativity powers).

\textbf{Creation:} A power of the creation subdiscipline creates an object or creature in the place the manifester designates (subject to the limits noted above). If the power has a duration other than instantaneous, psionic energy holds the creation together, and when the power ends, the created creature or object vanishes without a trace, except for a thin film of glistening ectoplasm that quickly evaporates. If the power has an instantaneous duration, the created object or creature is merely assembled through psionics. It lasts indefinitely and does not depend on psionics for its existence.

\subsubsection{Psychokinesis}
Psychokinesis powers manipulate energy or tap the power of the mind to produce a desired end. Many of these powers produce spectacular effects above and beyond the power's standard display, such as moving, melting, transforming, or blasting a target. Psychokinesis powers can deal large amounts of damage.

For the purpose of psionics-magic transparency, psychokinesis powers are equivalent to powers of the evocation school (thus, creatures immune to evocation spells are also immune to psychokinesis powers).

\subsubsection{Psychometabolism}
Psychometabolism powers change the physical properties of some creature, thing, or condition.

For the purpose of psionics-magic transparency, psychometabolism powers are equivalent to powers of the transmutation school (thus, creatures immune to transmutation spells are also immune to psychometabolism powers).

\textbf{Healing:} Psychometabolism powers of the healing subdiscipline can remove damage from creatures. However, psionic healing usually falls short of divine magical healing, in direct comparisons.

\subsubsection{Psychoportation}
Psychoportation powers move the manifester, an object, or another creature through space and time.

For the purpose of psionics-magic transparency, psychoportation powers do not have an equivalent school.

\textbf{Teleportation:} A power of the teleportation subdiscipline transports one or more creatures or objects a great distance. The most potent of these powers can cross planar boundaries. Usually the transportation is one-way (unless otherwise noted) and not dispellable. Teleportation is instantaneous travel through the Astral Plane. Anything that blocks astral travel also blocks teleportation.

\subsubsection{Telepathy}
Telepathy powers can spy on and affect the minds of others, influencing or controlling their behavior.

Most telepathy powers are mind-affecting.

For the purpose of psionics-magic transparency, telepathy powers are equivalent to powers of the enchantment school (thus, creatures resistant to enchantment spells are equally resistant to telepathy powers).

\textbf{Charm:} A power of the charm subdiscipline changes the way the subject views you, typically making it see you as a good friend.

\textbf{Compulsion:} A power of the compulsion subdiscipline forces the subject to act in some manner or changes the way her mind works. Some compulsion powers determine the subject's actions or the effects on the subject, some allow you to determine the subject's actions when you manifest them, and others give you ongoing control over the subject.