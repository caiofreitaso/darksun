\subsection{Concentration}
To manifest a power, you must concentrate. If something threatens to interrupt your concentration while you're manifesting a power, you must succeed on a \skill{Concentration} check or lose the power points without manifesting the power. The more distracting the interruption and the higher the level of the power that you are trying to manifest, the higher the DC. (Higher-level powers require more mental effort.)

\textbf{Injury:} Getting hurt or being affected by hostile psionics while trying to manifest a power can break your concentration and ruin a power. If you take damage while trying to manifest a power, you must make a \skill{Concentration} check (DC 10 + points of damage taken + the level of the power you're manifesting). The interrupting event strikes during manifestation if it occurs between when you start and when you complete manifesting a power (for a power with a manifesting time of 1 round or longer) or if it comes in response to your manifesting the power (such as an attack of opportunity provoked by the manifesting of the power or a contingent attack from a readied action).

If you are taking continuous damage half the damage is considered to take place while you are manifesting a power. You must make a \skill{Concentration} check (DC 10 + \onehalf the damage that the continuous source last dealt + the level of the power you're manifesting).

If the last damage dealt was the last damage that the effect could deal then the damage is over, and it does not distract you.

Repeated damage does not count as continuous damage.

\textbf{Power:} If you are affected by a power while attempting to manifest a power of your own, you must make a \skill{Concentration} check or lose the power you are manifesting. If the power affecting you deals damage, the \skill{Concentration} DC is 10 + points of damage + the level of the power you're manifesting. If the power interferes with you or distracts you in some other way, the \skill{Concentration} DC is the power's save DC + the level of the power you're manifesting. For a power with no saving throw, it's the DC that the power's saving throw would have if a save were allowed.

\textbf{Grappling or Pinned:} To manifest a power while grappling or pinned, you must make a \skill{Concentration} check (DC 20 + the level of the power you're manifesting) or lose the power.

\textbf{Vigorous Motion:} If you are riding on a moving mount, taking a bouncy ride in a wagon, on a small boat in rough water, belowdecks in a storm-tossed ship, or simply being jostled in a similar fashion, you must make a \skill{Concentration} check (DC 10 + the level of the power you're manifesting) or lose the power.

\textbf{Violent Motion:} If you are on a galloping horse, taking a very rough ride in a wagon, on a small boat in rapids or in a storm, on deck in a storm-tossed ship, or being tossed roughly about in a similar fashion, you must make a \skill{Concentration} check (DC 15 + the level of the power you're manifesting) or lose the power.

\textbf{Violent Weather:} If you are in a high wind carrying blinding rain or sleet, the DC is 5 + the level of the power you're manifesting. If you are in wind-driven hail, dust, or debris, the DC is 10 + the level of the power you're manifesting. In either case, you lose the power if you fail the \skill{Concentration} check. If the weather is caused by a power, use the rules in the Power subsection above.

\textbf{Manifesting Powers on the Defensive:} If you want to manifest a power without provoking attacks of opportunity, you need to dodge and weave. You must make a \skill{Concentration} check (DC 15 + the level of the power you're manifesting) to succeed. You lose the power points without successful manifestation if you fail.

\textbf{Entangled:} If you want to manifest a power while entangled in a net or while affected by a power with similar effects you must make a DC 15 \skill{Concentration} check to manifest the power. You lose the power if you fail.