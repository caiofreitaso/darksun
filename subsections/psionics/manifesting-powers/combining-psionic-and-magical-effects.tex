\subsection{Combining Psionic And Magical Effects}
Athasian magic works in a very different way than psionics and most forms of protection do not apply to both.

\textbf{Psionics-Magic Dissonance:} Magic and psionics are two very different sources of power that don't speak to each other. This creates a complex relationship between them.

\textit{Dispel:} The \spell{dispel magic} spell has no effect against psionic powers, abilities, or items. At the same time, the \psionic{dispel psionics} power has no effect against spells, spell-like abilities, or magic items.

\textit{Power Resistance and Spell Resistance:} Likewise, psionic powers are unimpeded by spell resistance, and spells are not affected by power resistance.

\textit{Antimagic and Null Psionics:} The \spell{antimagic field} does not block psionics, and \psionic{null psionics field} does not suppress magic.

\textbf{Multiple Effects:} Powers or psionic effects usually work as described no matter how many other powers, psionic effects, spells, or magical effects happen to be operating in the same area or on the same recipient. Except in special cases, a power does not affect the way another power or spell operates. Whenever a power has a specific effect on other powers or spells, the power description explains the effect (and vice versa for spells that affect powers). Several other general rules apply when powers, spells, magical effects, or psionic effects operate in the same place.

\textbf{Stacking Effects:} Powers that provide bonuses or penalties on attack rolls, damage rolls, saving throws, and other attributes usually do not stack with themselves. More generally, two bonuses of the same type don't stack even if they come from different powers, or one from a power and one from a spell. You use whichever bonus gives you the better result.

\textit{Different Bonus Types:} The bonuses or penalties from two different powers, or a power and a spell, stack if the effects are of different types. A bonus that isn't named (just a ``+2 bonus'' rather than a ``+2 insight bonus'') stacks with any bonus.

\textit{Same Effect More than Once in Different Strengths:} In cases when two or more similar or identical effects are operating in the same area or on the same target, but at different strengths, only the best one applies. If one power or spell is dispelled or its duration runs out, the other power or spell remains in effect (assuming its duration has not yet expired).

\textit{Same Effect with Differing Results:} The same power or spell can sometimes produce varying effects if applied to the same recipient more than once. The last effect in a series trumps the others. None of the previous spells or powers are actually removed or dispelled, but their effects become irrelevant while the final spell or power in the series lasts.

\textit{One Effect Makes Another Irrelevant:} Sometimes, a power can render another power irrelevant.

\textit{Multiple Mental Control Effects:} Sometimes psionic or magical effects that establish mental control render one another irrelevant. Mental controls that don't remove the recipient's ability to act usually do not interfere with one another, though one may modify another. If a creature is under the control of two or more creatures, it tends to obey each to the best of its ability, and to the extent of the control each effect allows. If the controlled creature receives conflicting orders simultaneously, the competing controllers must make opposed Charisma checks to determine which one the creature obeys.

\textbf{Powers and Spells with Opposite Effects:} Powers and spells with opposite effects apply normally, with all bonuses, penalties, or changes accruing in the order that they apply. Some powers and spells negate or counter each other. This is a special effect that is noted in a power's or spell's description.

\textbf{Instantaneous Effects:} Two or more magical or psionic effects with instantaneous durations work cumulatively when they affect the same object, place, or creature.