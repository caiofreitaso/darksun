\subsection{Tools and Skill Kits}

\ItemTable{Tools and Skill Kits}{
Alchemist's lab & 500 cp & 20 kg\\
Artisan's tools & 5 cp & 2.5 kg\\
Artisan's tools, masterwork & 55 cp & 2.5 kg\\
Book of poisons & 125 cp & 1 kg\\
Candle of rejuvenation & 50 cp & \\
Climber's kit & 80 cp & 2.5 kg1\\
Concealing weave & 5 cp & 1 kg\\
Disguise kit & 50 cp & 4 kg1\\
Healer's kit & 50 cp & 0.5 kg\\
Holly and mistletoe & & \\
Holy symbol, silver & 25 cp & 0.5 kg\\
Holy symbol, wooden & 1 cp & \\
Hourglass & 25 cp & 0.5 kg\\
Magnifying glass & 100 cp & \\
Meditative kit & 35 cp & 1.5 kg\\
Musical instrument, common & 5 cp & 1.5 kg1\\
Musical instrument, masterwork & 100 cp & 1.5 kg1\\
Navigator kit & 75 cp & 5 kg\\
Remote viewing kit & & 5 kg\\
Scale, merchant's & 2 cp & 0.5 kg\\
Spell component pouch & 5 cp & 1 kg\\
Spellbook, wizard's (blank) & 15 cp & 1.5 kg\\
Thieves' tools & 30 cp & 0.5 kg\\
Thieves' tools, masterwork & 100 cp & 1 kg\\
Tool, masterwork & 50 cp & 0.5 kg\\
Water clock & 1,000 cp & 100 kg\\
}

The items described below are particularly useful to characters that have certain specific skills or abilities and are used in specific situations.


\textbf{Alchemist's Lab:} An alchemist's lab always has the perfect tool for making alchemical items, so it provides advantage on \skill{Craft} (alchemy) checks. It has no bearing on the costs related to the \skill{Craft} (alchemy) skill. Without this lab, a character with the \skill{Craft} (alchemy) skill is assumed to have enough tools to use the skill but not enough to get the advantage that the lab provides.

\textbf{Artisan's Tools, Masterwork:} These tools serve the same purpose as artisan's tools (above), but masterwork artisan's tools are the perfect tools for the job, so you get advantage on \skill{Craft} checks made with them.

\textbf{Artisan's Tools:} These special tools include the items needed to pursue any craft. Without them, you have to use improvised tools (-2 penalty on \skill{Craft} checks), if you can do the job at all.

\textbf{Book of Poisons:} The original Book of Poisons is rumored to have been written by the half-elven bard Cabal, with current copies containing but fragments of the original poison recipes. This set of clay tablets is covered with markings, known mostly to bards, that can only be understood by making a \skill{Decipher Script} check (DC 15). Once deciphered, the reader can see that they contain a number of recipes that describe, step-by-step, tried-and-true methods for crafting specific poisons. The tablets grant the following benefits when used in conjunction with the crafting of poisons described in the set (normally 5 to 10 different poisons, of the DM's choice): advantage to \skill{Craft} (poisonmaking) checks and a +1 to the save DC of the poisons being crafted. This last bonus stacks with the scorpion's touch bardic ability.

\textbf{Candle of Rejuvenation:} This item allows a manifester to recover power points as if he were resting at night. The manifester recovers 10 power points at the end of each complete hour spent within 3 meters of a lit candle. By making an \skill{Autohypnosis} DC 15 check, this amount increases by one-half. Each candle burns for a total of eight hours.

\textbf{Climber's Kit:} This is the perfect tool for climbing and gives you advantage on \skill{Climb} checks.

\textbf{Concealing Weave:} This kit is composed of one or more related articles of clothing specifically made to camouflage a caster's arm and hand movements while casting a spell. This kit grants advantage on \skill{Bluff} checks made to conceal the casting of spells with a somatic component.

\textbf{Disguise Kit:} The kit is the perfect tool for disguise and provides advantage on \skill{Disguise} checks. A disguise kit is exhausted after ten uses.

\textbf{Healer's Kit:} It is the perfect tool for healing and provides advantage on \skill{Heal} checks. A healer's kit is exhausted after ten uses.

\textbf{Holy Symbol, Silver or Wooden:} A holy symbol focuses positive energy. A cleric uses it as the focus for his spells and as a tool for turning undead. Each religion has its own holy symbol.

 \textit{Unholy Symbols:} An unholy symbol is like a holy symbol except that it focuses negative energy and is used by evil clerics (or by neutral clerics who want to cast evil spells or command undead).

 \textit{Sorcerer-king's Sigil:} A sorcerer-king's sigil is like a holy symbol for templars. It is the sign of their rank and station within the templarate. It is unique to each city-state.

\textbf{Magnifying Glass:} This simple lens allows a closer look at small objects. It is also useful as a substitute for flint and steel when starting fires. Lighting a fire with a magnifying glass requires light as bright as sunlight to focus, tinder to ignite, and at least a full-round action. A magnifying glass grants advantage on \skill{Appraise} checks involving any item that is small or highly detailed.

\textbf{Meditative Kit:} This small and delicately carved crystal container produces an hypnotic rainbow-like effect while filled with clear water and struck by light. After 1 minute of uninterrupted observation of the rainbow pattern, the kit provides advantage to the next \skill{Autohypnosis} check made by the viewer within the next 10 minutes.

\textbf{Musical Instrument, Common or Masterwork:} A masterwork instrument grants advantage on \skill{Perform} checks involving its use.

\textbf{Navigator's Kit:} Prized posessions of many trading houses and frequent wanderers of the wastes, each of these kits is composed of a set of maps made of straight sticks representing roads, and small stones for villages, cities and others special locations, all lashed together by strings. If you succeed at a \skill{Knowledge} (geography) check DC 10 while using this kit, you gain a +4 bonus on \skill{Survival} checks made to keep from getting lost.

\textbf{Remote Viewing Kit:} This kit allows for a more potent use of the \psionic{remote viewing} power. Unlike other class or skill kits, this kit is created from local natural materials, effectively making it free in cost, but its user must recreate the kit before each use. Five ranks in \skill{Knowledge} (psionics), and 10 minutes, are required to create the kit. It must be created in silence, without distractions, and in a windless area. The kit takes the form of a 1.5-meter square patch of flat ground, covered with sand or particulate dirt to a depth of at least 2.5 centimeters, with 1d6 palm-sized stones deposited on it. Lines and circles are then traced around the stones and over the entire surface, creating a unique, maze-like pattern.

To gain the benefits of the kit, the user must focus on the patch of ground and succeed at a DC 15 \skill{Concentration} check after its creation; failure indicating that the user needs to recreate the kit anew. A successful check means the user's next manifestation of remove viewing, which must be within 1 minute of making the check, is altered in the following ways. First, the subject of the user's viewing attempt receives a $-2$ penalty to his Will saving throw against the \psionic{remote viewing}. Second, the user receives advantage to \skill{Concentration} checks made to manifest a power through \psionic{remote viewing}. Finally, the user receives advantage to \skill{Hide} checks to prevent his quasi-real viewpoint from being noticed by the subject he his viewing.

The effects of this kit can be made more potent if more than one character assists with its creation. Each character that helps adds another 1.5-meter square to the space taken by the kit, and another 10 minutes to the time required for the kit's completion. Each character that succeeds at a DC 15 \skill{Concentration} check at the time of the kit's completion can use the aid another action to help the user with skill checks made while the user is manifesting \psionic{remote viewing}. A kit created in this fashion is more complex, and as such requires two more ranks in \skill{Knowledge} (psionics) to create for each additional character that helped in its creation. Only a limited number of characters can help to create a remove viewing kit, equal to half the user's manifester level.

\textbf{Scale, Merchant's:} A scale grants advantage on \skill{Appraise} checks involving items that are valued by weight, including anything made of precious metals.

\textbf{Spell Component Pouch:} A spellcaster with a spell component pouch is assumed to have all the material components and focuses needed for spellcasting, except for those components that have a specific cost, divine focuses, and focuses that wouldn't fit in a pouch.

\textbf{Spellbook, Wizard's (Blank):} A spellbook has 100 pages of parchment, and each spell takes up one page per spell level (one page each for 0-level spells).

\textbf{Thieves' Tools, Masterwork:} This kit contains extra tools and tools of better make, which grant advantage on \skill{Disable Device} and \skill{Open Lock} checks.

\textbf{Thieves' Tools:} This kit contains the tools you need to use the \skill{Disable Device} and \skill{Open Lock} skills. Without these tools, you must improvise tools, and you have disadvantage on \skill{Disable Device} and \skill{Open Lock} checks.

\textbf{Tool, Masterwork:} This well-made item is the perfect tool for the job. It grants advantage on a related skill check (if any). Bonuses provided by multiple masterwork items used toward the same skill check do not stack.

\textbf{Water Clock:} This large, bulky contrivance gives the time accurate to within half an hour per day since it was last set. It requires a source of water, and it must be kept still because it marks time by the regulated flow of droplets of water.