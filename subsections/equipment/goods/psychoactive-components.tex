\subsection{Psychoactive Components}
\ItemTable{Psychoactive Components}{
Aviarag horn melange & 250 Cp & 1 lb.\\
Bouyan crystal & 400 Cp & 2 lb.\\
Cilops compound eye & 65 Cp & —\\
Dagorran crystal, diminutive & 100 Cp & —\\
Dagorran crystal, tiny & 300 Cp & 1⁄2 lb.\\
Tas’l worm & 100 Cp & —\\
T’chowb’s thalamus & 1,250 Cp & 1⁄2 lb.\\
}

Useful for manifesters, these items are components that have a chance of influencing the manifestation of certain powers when used properly. Unlike metempiric components, each psychoactive component must be used in a specific fashion in order to provide its benefits to the manifester. Unless otherwise noted, psychoactive components are consumed during the manifesting of a power.

Also included in this category is a creature sometimes used by psionic characters to access the residual psionic energy their mind's produce throughout the day.


\textbf{Aviarag Horn Melange:} The great horn of the noble aviarag, when crushed and mixed with a specially treated wine, preserves some of the psionic power of the aviarag and can expand the reach and power of the drinker's mind. A manifester drinking this psychoactive substance no more than 1 minute before using the mindwipe power increases the power's save DC by +2, and drinking it before manifesting mindlink doubles the power's range, as per the Enlarge Power feat (but without changing the power's effective level or requiring the expendature of a psionic focus). Drinking this melange is a standard action that provokes attacks of opportunity.

\textbf{Bouyan Crystal:} From salt formations found within the great salt plains, the clear gray bouyan stone is cut to a great degree of perfection, allowing a manifester to focus his mind upon it. A manifester that concentrates upon a held bouyan crystal when manifesting any telepathy power (a free action, assuming the manifester is already holding the component) increases the power's effective manifester level by +1.

\textbf{Cilops Compound Eye:} Crushing the central compound eye of a cilops produces a thick jelly that may extend the user's senses of his surroundings. If this jelly is spread over the eyes of a character (a full-round action) before he manifests object reading, and the temporary blindness it induces is endured for the extent of the power's duration, then this power always successfully identifies all other former owners of an object in sequence, with no chance that former owners will be skipped and thus not identified. Removing the jelly from one's eyes, thus negating the blindness, takes 1 minute. Eating the jelly before manifesting sensitivity to psychic impressions reduces the manifesting time of that power to 10 minutes.

\textbf{Dagorran Crystal:} This green crystal growth is extracted from the body of a daggoran. Manifesters use these crystals to harness the dagorran's ability to sense the psionic nature of creatures. A Diminutive daggoran crystal gives manifesters a +2 competence bonus to Psicraft checks made when using the psychic tracking psionic power, while a Tiny daggoran crystal gives a +5 competence bonus. Such a crystal is not consumed when used.

\textbf{Tas'l Worm:} Tas'l worms are worms of Diminutive size, similar in appearance to ock'n, but without eyestalks. Only found on living psionic creatures, these 1-inch long worms snake slowly between the skin and the skull of their host, accumulating and living off of residual psionic energy.

A tas'l worm can be removed by cutting open the skin. A character can remove a worm by taking 10 minutes to locate and extract it from the host's scalp. The extraction does 1d6 points of damage to the victim, and has a 75\% chance of killing the worm in the process. If a successful Heal check (DC 20) is made, the cutting damage is reduced to 1d2, with no chance of killing the worm.

A worm outside of its host lives for 1 hour. The worm, when put against the skin of a psionic creature, burrows into its flesh, causing 1 point of damage. Afterwards, the worm must stay within the host for at least 24 hours before it amasses enough residual energy to be used. Tas'l worms use the innocuous vermin statistics (see page 191 of Terrors of Athas). A worm linked to a host has a lifespan of 2d6 months. If more than one worm live on the same scalp, there is a 55\% chance for 1d2 worms to spawn each month thereafter.

The creature hosting tas'l worms can make a Concentration check (DC 20) as a free action, once per day, to tap the energy they contain. If the check is successful, each tas'l worm hosted by a creature provides 1 power point. All worms must be tapped at once, or none at all. These power points are considered a part of the creature's own power point reserve for the purpose of using stored power points. A failed check can be retried on the character's next turn.

The hosting creature receives a cumulative $-1$ circumstance penalty to Will saves against telepathic powers for each worm living within its body, as the worms make their host more responsive to outside psionic influence.

A tas'l worm registers as psionic to detect psionics.

\textbf{T'chowb's Thalamus:} The thalamus of a recently fed (within the last 24 hours) t'chowb is said to be seething with absorbed intelligence. If crushed in one's hand during the manifestation of a power (a free action, assuming the manifester is already holding the component), the t'chowb's thalamus gives a manifester a +10 circumstance bonus on manifester level checks made to overcome a target's power resistance.