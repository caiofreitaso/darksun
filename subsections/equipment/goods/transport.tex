\subsectionA{Transport}
Sometimes it is too hard or too dangerous to ride a kank---you'll need some other form of transportation. Some vehicles, such as the chariot and howdah, are moved by muscle power. The \skill{Handle Animal} skill is used only if that power comes from a team of draft animals. When the team consists of creatures with Intelligence scores of 3 or higher, the operative skill is \skill{Diplomacy}. When they are slaves or forced labor, the operative skill is \skill{Intimidate}.

\Table{Transport}{XR}{
\tableheader Transport & \tableheader Cost\\

\TableSubheader{Chariot} & \\
~ transport & 100 cp \\
~ light, war & 250 cp \\
~ heavy, war & 600 cp \\

% one driver and one Medium combatant
% 50% cover
% Poor Manuverability (Average with Ride check)

% one driver and three Medium combatants
% 75% cover
% driver takes $-4$ on attacks, if Ride check is successful
% Clumsy Manuverability (Poor, with Ride check)

\TableSubheader{Howdah} & \\
~ inix & 10 cp \\
~ mekillot & 20 cp \\

\TableSubheader{Howdah, War} & \\
~ inix & 100 cp \\
~ mekillot & 500 cp \\

\TableSubheader{Wagon, Open} & \\
~ 500 kg capacity & 20 cp \\
~ 1,250 kg capacity & 35 cp \\
~ 2,500 kg capacity & 50 cp \\
~ 5,000 kg capacity & 100 cp \\

\TableSubheader{Wagon, Enclosed} & \\
~ 500 kg capacity & 40 cp \\
~ 1,250 kg capacity & 70 cp \\
~ 2,500 kg capacity & 100 cp \\
~ 5,000 kg capacity & 200 cp \\
~ armored caravan & 1,000 cp \\

\TableSubheader{Silt Vehicles} & \\
~ schooner, trade\footnotemark[1] & 150 gp \\
~ schooner, war\footnotemark[1] & 150 gp \\
~ skiff\footnotemark[1] & 300 gp \\
~ skimmer & 200 gp \\

\TableNote{2}{1 \emph{Obsidian engine} is not included.}
}

\textbf{Chariot:} A chariot is a two-wheeled vehicle used for transportation, racing, war and processions. Transport chariots are very small and simple, requiring only a single animal to draw it. War chariots are significantly better constructed, and they provide cover to its occupants. A light war chariot is built for two Medium riders---generally one person will drive the chariot while the other uses a bow or other ranged weapon---and require two mounts, usually kanks or crodlus. A heavy war chariot is much larger than the other two kinds of chariots and are built to fit one driver and three Medium combatants. Heavy war chariots require four mounts to drive it.

\textbf{Howdah:} A howdah is an enclosure mounted on a riding animal containing space for one or more persons. Howdahs can be fitted on inix or mekillots, and provide shade and cover from the elements. An inix howdah usually has room for only one person, though the war howdah, built much stronger, can hold four. A mekillot howdah can hold one or two persons, but a war howdah is much bigger, consisting of two levels and holding up to sixteen warriors.

\textbf{Psionic Silt Vehicle:} It has a hull similar in shape to a conventional silt skimmer's, but it possesses no wheels. Its keel is flat-bottomed so the ship can rest level on the coast or while docked when not lifted by its shipfloater. In its center is installed a psionically powered piece of obsidian, the \emph{obsidian engine}, which is used to lift the vessel to the surface of the silt, so that it can be moved by use of the wind or by using poles against the siltbed to push the vehicle forward. These vessels can be used to cross any depth of silt.

\textit{Silt Schooner, Trade:} Silt schooner is the term used to describe a psionically-powered siltworthy ship used for trade, plying the sea between siltside cities and villages.

It has two masts and square sails. It needs a crew of 25 people, which is composed of a captain and his officers, a shipfloater and one or two apprentices, the rest being sailors who double as polers when the wind is dead. People within the bridge are not affected by the Gray Death condition that can prevail outside on the deck.

Such ship doesn't usually have space allowed for passengers, but accommodation can be made for passengers to take over ordinary cargo space. Rarely, a trade schooner can be fitted with a single light catapult or ballista instead of a corresponding amount of cargo space.

\textit{Silt Schooner, War:} It has three masts and square sails. It needs a crew of 60 people, which is composed of a captain and his officers, a shipfloater and one or two apprentices, the rest being sailors who double as polers when the wind is dead, and the catapult slave crews. People within the bridge are not affected by the Gray Death condition that can prevail outside on the deck.

A war schooner can be fitted with three heavy catapults or six light catapults or ballistas instead of a corresponding amount of cargo space. Most of a war schooner's cargo space is often converted into passenger space for soldiers.

\textit{Silt Skiff:} Silt skiff is the term used to describe a psionically-powered siltworthy ship used for coast hugging and as a means of revenue for sailors living in cities and villages near silt, such as Balic and Ledopolus.

It has a single mast and square sail. It needs a crew of 10 people, which is composed of a captain, a shipfloater and one or two apprentices, the rest crewing the riggings or manning the poles when no wind blows. Such ship do not usually have space allowed for passengers, but accommodation can be made for passengers to take over ordinary cargo space.

\textbf{Silt Skimmer:} It has a single mast with a triangular sail. It possesses four massive wheels, each one wide at its center but tapering to a fine point along the edge. The skimmer has wheels 9 meters in diameter that thread on the Sea of Silt's seabed, slicing through the silt. The skimmer can cross silt depth up to half its wheel's diameter. It needs a crew of 7 people and has a capacity of 2 metric tonnes.

\textbf{Wagon:} Wagons are an essential part of Athasian economy, as they facilitate the caravans that make life in the wastes possible. Open wagons are basic, open-topped wagons that can carry a certain amount of cargo. As Athasian wagons are built using little or no metal, there's a limit to how much cargo they can carry. 500-kg capacity wagons need only one kank to pull it, 1,250-kg and 2,500-kg require two and four kanks, respectively. 5,000-kg wagons require a single mekillot. Inix are not used to pull wagons because of their tails that get in the way.

\textit{Enclosed wagons:} They are more commonly used to transport people or fragile cargo that would otherwise be damaged by exposure to the elements.

\textit{Armored caravans:} They are primarily used by caravans traveling through areas plagued by dangerous monsters or raiders. It is an enclosed wagon with agafari wood used to strengthen the wagon throughout. There are also mount points for fixed crossbows on each side of that wagon that can swivel 180 degrees. Anyone using the crossbows or firing out of the rear of the wagon (when it is open) receives cover. Armored caravans require one mekillot to draw it, but are usually drawn by two to prevent stranding the wagon in the event of death of a mount.
