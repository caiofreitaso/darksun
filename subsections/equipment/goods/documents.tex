\subsection{Documents}
This class of items has a symbolic function, conveying authority or permission. No price is listed for these items, because their value is not inherent.

\textbf{Letter of Marque:} This letter bears the personal mark of the sorcerer-king, and bestows limited secular authority on the bearer, as if the bearer were a templar. The bearer of a letter of marque gains the authority to contest the actions of templars, using the bearer's \skill{Diplomacy} check. If the bearer is already a templar, then having the letter as additional authority grants the templar a +4 circumstance check towards authority contest checks. The letter of marque does not grant the authority to Intrude, Requisition, Accuse or Judge, but does grant power to contest such actions by templars. A letter of marque is limited by time. After a specified period (usually one year, and never longer than seven years) the letter loses its effectiveness. A sorcerer-king can also declare the letter invalid. Forging or fraudulently using a letter of marque is an unpardonable offense that brings a death sentence. Obviously, only the templars and other servants of the sorcerer-king that issued the letter of marque will honor its terms. A person who is caught with a king's letter of marque within another sorcerer-king's territory will have some explaining to do.

\textbf{Letter of Reprisal:} Like a letter of marque, this letter bears the personal mark of the sorcerer-king, and bestows limited secular authority on the bearer, as if the bearer were a templar. Unlike a letter of marque, a letter of reprisal has a limited scope to carrying out a specific mission, usually a reprisal or retaliation against a specific group of the King's enemies, for example, killing or capturing a specific enemy officer, capturing a particular enemy fortress or silt vessel, defiling a stretch of key farmland, or annihilating or enslaving a designated village. Depending on the bearer's \skill{Diplomacy} ranks, she can Requisition, Intrude, Accuse, Judge, but only if she can show that her request relates to fulfilling her assigned mission. She can attempt to contest the actions of other templars, but takes a $-4$ circumstance penalty on such attempts, since the opposing templars can argue (even if it is not true) that she is acting outside of the scope of the assigned reprisal mission. The $-4$ penalty also applies if templars contest any of her Requisition, Intrude, Accuse, or Judge actions.