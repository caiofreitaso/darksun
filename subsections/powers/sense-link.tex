\Psionic{Sense Link}{sense link}
{Telepathy [Mind-Affecting]}
{
	\textbf{Level:}
	Telepath 2\\
	\textbf{Manifesting Time:}
	1 round\\
	\textbf{Range:}
	Unlimited\\
	\textbf{Target:}
	One \emph{contacted} mind\\
	\textbf{Saving Throw:}
	None\\
	\textbf{Cost:}
	\psionic{contact}\\
	\textbf{Maintenance Cost:}
	2 pp/round\\
	\textbf{Prerequisites:}
	\psionic{mindlink}\\
	% \textbf{Critical Success:}
	% You link two senses at once.\\
	% \textbf{Critical Failure:}
	% You lose the linked sense for 1d4 hours.\\
}
{
	You perceive what the subject creature perceives using its sight, hearing, taste, or smell. Only one sense is \emph{linked}, and you cannot switch between senses with the same manifestation.

	You make any skill checks involving senses, such as \skill{Spot} or \skill{Listen}, as the subject, and only within the subject's field of view. You lose your Dexterity bonus to AC while directly sensing what the subject senses.

	Once \emph{sense link} is manifested, the \emph{link} persists even if the subject moves out of the range of the original manifestation (but the \emph{link} does not work across planes). You do not control the subject, nor can you communicate with it by means of this power.

	The strength of the subject's \emph{linked sense} could be enhanced by other powers or items, allowing you the same enhanced sense. You are subject to any gaze attack affecting the subject creature (if you \emph{linked} vision). If you are blinded or deafened, or suffer some other sensory deprivation, the \emph{linked} creature functions as an independent sensory organ, and provides you the benefit of the \emph{linked sense} from its perspective while this power's duration lasts.

	\textit{Augment:} You can augment this power in one or both of the following ways.
	\begin{enumerate*}
	\item If you spend 5 additional power points, you can have the subject perceive one of your senses instead of the other way around.
	\item If you spend 10 additional power points, you can \emph{link} to an additional sense of the same subject.
	\end{enumerate*}
}
