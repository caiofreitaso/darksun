\Psionic{Time Travel}{time travel}
{Psychoportation [Teleportation]}
{
	\textbf{Level:}
	Nomad 7\\
	\textbf{Manifesting Time:}
	1 mental action\\
	\textbf{Range:}
	Personal (see text)\\
	\textbf{Target:}
	You (see text)\\
	\textbf{Saving Throw:}
	None\\
	\textbf{Cost:}
	35 power points\\
	\textbf{Maintenance Cost:}
	7 pp/hour\\
	\textbf{Prerequisites:}
	\psionic{psionic teleport}, \psionic{personal time hop}\\
	% \textbf{Critical Success:}
	% No other effect.\\
	% \textbf{Critical Failure:}
	% You must make a Will save or dislocate to a random timeline and assume it is your native time.\\
}
{
	You extend your teleportation power into the time stream and journey to different a time. You can jump a day into the past or future. While you are gone, time keeps running in your normal setting---if you spend eight hours in the past, you will return to a point eight hours after the time you left.

	The DM should be guided by two principles: once an event has been changed once, it can never be changed again; and secondly. events tend to have a historical inertia. In other words. things have a way of working themselves out to be the same no matter what you do. The more important the event, the more difficult it is to change it.

	Whatever happens in the past or the future, the DM should use this power to make things more interesting. Ignore or apply paradoxes as desired to make the PC's life more entertaining and to keep the story going.

	\textbf{The Past:} In the short term, you may wish to alter recent events by warning someone not to do something that you know will turn out badly. You may travel back an hour to tell your companions (and your past self) not to storm the fortress, for example. The party may get a chance to replay the events in question---but you had better remember to go back and warn yourself, even if things do turn out better, otherwise you will never receive the warning.

	In the long term, you may try to recover lost information by speaking to people long dead. You may try to kill your enemies by assassinating their forebears. If you try to alter history, the DM should decide if you succeed or fail. Sometimes your actions may have unexpected ramifications. For example, by killing Kalak the Tyrant in his youth. you may pave the way for an even more terrible despot to arise.

	\textbf{The Future:} You can journey to the future to see how an action will turn out or to uncover information not available in your own day. Like the past, the future is malleable; even the fact of your visit changes the course of events in innumerable ways. Once you have glimpsed the future, historical inertia takes over and begins to bend events to follow the revealed timeline. This destiny can only be avoided with great difficulty once it comes into being.

	\textit{Augment:} You can augment this power in one or more of the following ways.
	\begin{enumerate*}
	\item If you know the \psionic{teleport other} power and spend 20 additional power points, you may also bring along one additional creature. You can augment the power this way up to one additional creature per manifester level.
	\item If you spend additional power points, you can increase this power's time distance. The distance also increases the Difficulty Class of the power check: 
	\end{enumerate*}

	\Table{}{XcC}{
	  \tableheader Distance
	& \tableheader Initial Cost
	& \tableheader Power Check DC Modifier \\
	1 day or less &  35 & +0 \\
	1 week        &  40 & +1 \\
	1 month       &  50 & +2 \\
	1 year        &  60 & +3 \\
	10 years      &  70 & +4 \\
	100 years     &  80 & +5 \\
	1,000 years   &  90 & +6 \\
	10,000 years  & 120 & +8 \\
	}
}