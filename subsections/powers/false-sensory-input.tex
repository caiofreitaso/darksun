\Psionic{False Sensory Input}{false sensory input}
{Telepathy (Figment)}
{
    \textbf{Level:}
    Telepath 3\\
    \textbf{Manifesting Time:}
    1 mental action\\
    \textbf{Range:}
    Unlimited\\
    \textbf{Target:}
    One \emph{contacted} mind\\
    \textbf{Saving Throw:}
    None\\
    \textbf{Cost:}
    \psionic{contact}\\
    \textbf{Maintenance Cost:}
    3 pp/round\\
    \textbf{Prerequisites:}
    \psionic{mindlink}\\
}
{
    You have a limited ability to falsify one of the subject's senses. The subject thinks she sees, hears, smells, tastes, or feels something other than what her senses actually report. You can't create a sensation where none exists, nor make the subject completely oblivious to a sensation, but you can replace the specifics of one sensation with different specifics. For instance, you could make a human look like a dwarf (or one human look like another specific human), a closed door look like it is open, a vat of acid smell like rose water, a parrot look like a bookend, stale rations taste like fresh fruit, a light pat feel like a dagger thrust, a scream sound like the howling wind, and so on.

    You can switch between senses you falsify round by round. You can't alter the size of an object by more than 50\% by using this power. Thus, you couldn't make a castle look like a hovel, but you could make it look like a different castle, or a rough hillock of approximately the same size. If this power is used to distract an enemy manifester who is attempting to use his powers, the enemy must make a \skill{Concentration} check as if grappling or pinned.

    Because you override a victim's senses, you can fool a victim who is using true seeing or some other method of gathering information, assuming you know that the victim is actively using such an effect and you can maintain concentration.
}
