\Psionic{Compact}{compact}
{Psychokinesis}
{
	\textbf{Level:}
	Kineticist 1\\
	\textbf{Manifesting Time:}
	1 mental action\\
	\textbf{Range:}
	9 meters\\
	\textbf{Target:}
	Unattended, nonmagical material of up to 0.3 m$^3$\\
	\textbf{Saving Throw:}
	None\\
	\textbf{Cost:}
	5 power points\\
	\textbf{Maintenance Cost:}
	None (Instantaneous)\\
	% \textbf{Critical Success:}
	% The material is compacted by a factor 20, instead.\\
	% \textbf{Critical Failure:}
	% The material does not compact and becomes extremely hot, causing 1d4 fire damage to those around it.\\
}
{
	You can compress material into a space 10\% its original volume, e.g., 0.3 m$^3$ (300 liters) are compressed to a space of 0.03 m$^3$ (30 liters). The material retains its original mass, and reacts to temperature changes normally. It can be gaseous, liquid, or solid. It will remain compact for 24 hours, then it quickly expands to its original size, exerting tremendous force.

	When expanding the material forces its way around existing objects. If carefully applied, each 0.03 m$^3$ of compacted materials can cause 25 points of damage on expansion. Hardness does not reduce this damage, nor is it halved as damage dealt to objects normally is. If the damage would reduce the hit points of the structure to zero, it collapses. Any creature caught inside a collapsing structure takes 8d6 points of bludgeoning damage (Reflex DC 15 half) and is pinned beneath the rubble (see below).

	If a creature ingests compacted materials, it takes 2d10 points of damage per 30 grams taken when the material reexpands. Expansion is quick but not explosive.
}