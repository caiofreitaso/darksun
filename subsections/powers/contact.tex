\Psionic{Contact}{contact}
{Telepathy [Mind-Affecting]}
{
	\textbf{Level:}
	Telepath 1\\
	\textbf{Manifesting Time:}
	1 mental action\\
	\textbf{Range:}
	See text\\
	\textbf{Target:}
	One non-psionic creature, or one willing psionic creature\\
	\textbf{Saving Throw:}
	None\\
	\textbf{Cost:}
	5 power points\\
	\textbf{Maintenance Cost:}
	1 pp/round\\
	% \textbf{Critical Success:}
	% \emph{Contact} is maintained for four rounds for free.\\
	% \textbf{Critical Failure:}
	% You are unable to use \emph{contact} on the target's mind for one month.\\
}
{
	With the \emph{contact} power, you can maintain mental contact with a creature of up to 5 Hit Dice.

	\emph{Contact} must be established before virtually any telepathic power can be used on another creature. It is just what its name implies---contact between your mind and another character's mind. \emph{Contact} does not allow communication by itself; it is merely the conduit for other telepathic exchanges.

	You can maintain \emph{contact} with more than one target at a time, but you must \emph{contact} each one individually, and pay a maintenance cost for each use of the power. For example, if you wish to inspire awe in two targets, you must establish and maintain \emph{contact} with each one separately.

	If you use another power that requires \emph{contact} on the same target, the cost of maintaining \emph{contact} is covered by the other power's cost.

	You cannot \emph{contact} a target that you know nothing about. In other words, you can't use this power to scan around and ``see what's out there.'' You must either have the target in line of effect or know specifically who or what you are looking for. You cannot, for example, try to \emph{contact} any random elf which may or may not be standing behind a closed door. However, you can try to \emph{contact} a particular elf which you have seen before.

	If you fail to establish \emph{contact}, he can try again the next round. Failure doesn't necessarily mean the target's mind cannot be contacted. Rather, it means the target has not been found yet. You can continue searching.

	\textbf{Range:} You can make \emph{contact} with any creature if you have line of effect with them. If you don't have line of effect, then the farther the target is from you, the more difficult it is to use this power. This power does not work across planes.

	\Table{}{CC}{
	  \tableheader Range
	& \tableheader Power Check DC Modifier \\
	600 meters        & +0 \\
	1.5 kilometer     & +1 \\
	15 kilometers     & +3 \\
	150 kilometers    & +5 \\
	1500 kilometers   & +7 \\
	15,000 kilometers & +9 \\
	}

	\textbf{Different Types:} You can use this power to \emph{contact} creatures with different a type than you. The more different the type of the creature is, the harder it is to use this power.

	\Table{}{XC}{
	  \tableheader Type
	& \tableheader Power Check DC Modifier \\
	Humanoid, Giant            & +0 \\
	Animal, Magical Beast      & +1 \\
	Monstrous Humanoid, Undead & +2 \\
	Dragon                     & +3 \\
	Elemental, Fey, Outsider   & +4 \\
	Aberration, Ooze           & +5 \\
	Construct, Vermin          & +6 \\
	Plant                      & +7 \\
	}

	Thri-kreen characters (and other insectoids, such as scrabs and thor-kreen) have a different mind and use the following table, instead.

	\Table{}{XC}{
	  \tableheader Type
  	& \tableheader Power Check DC Modifier \\
  	Monstrous Humanoid, Vermin & +0 \\
	Animal, Magical Beast      & +1 \\
	Humanoid, Giant, Undead    & +2 \\
	Dragon                     & +3 \\
	Elemental, Fey, Outsider   & +4 \\
	Aberration, Ooze           & +5 \\
	Construct                  & +6 \\
	Plant                      & +7 \\
	}

	\textit{Augment:} For each additional power point you spend, the maximum number of Hit Dice you can affect increases by 1.
}
