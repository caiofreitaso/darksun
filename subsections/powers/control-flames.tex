\Psionic{Control Flames}{control flames}
{Psychokinesis [Fire]}
{
	\textbf{Level:}
	Kineticist 2\\
	\textbf{Manifesting Time:}
	1 mental action\\
	\textbf{Range:}
	18 meters\\
	\textbf{Area:}
	One nonmagical fire source (see text)\\
	\textbf{Saving Throw:}
	See text\\
	\textbf{Cost:}
	3 power points\\
	\textbf{Maintenance Cost:}
	2 pp/round\\
	\textbf{Prerequisites:}
	\psionic{psionic telekinesis}\\
	% \textbf{Critical Success:}
	% You can increase up to 200\% or decrease to 0\%\\
	% \textbf{Critical Failure:}
	% You burn yourself, suffering 1d4 points of damage\\
}
{
	You pyrokinetically control the intensity or movements of one fire source. A nonmagical fire source can be controlled if it is equal to or smaller than the maximum size of fire you can control according to your power check, as noted on the accompanying table.	When your control over a fire source lapses, that fire immediately returns to its original state (or goes out if it has no fuel or has been moved away from its original location). With this power, you can artificially keep a fire burning that would normally expire for lack of fuel; even dousing a controlled flame with water does not put it out (though completely submerging the flame would). Normally, a creature at risk of catching on fire can avoid this fate by making a DC 15 Reflex saving throw, with success indicating that the fire has gone out. If the fire is one that has been kept burning by the use of \emph{control flames}, then the DC of the Reflex save needed to put out the flames increases to 25.

	You need to concentrate to change the fire in one of the following ways:

	\textit{Attack:} You can attack a creature with an animated fire. The animated fire can only reach adjacent creatures. It attacks as a touch attack, and its attack bonus is equal to your base attack bonus + your Intelligence bonus. The attack deals the indicated damage.

	\textit{Change Intensity:} By making a power check, you can increase or decrease the size of the animated fire by one category. The power check DC is equal to the highest DC between the current and target sizes. You can reduce a Tiny or smaller fire to nothing, extinguishing it. You can change intensity many times, as long as you keep succeeding your power checks.

	\textit{Move:} This power also enables you to make a fire move as if it were a living creature. You can animate only a naturally burning fire; if you attempt to animate one that has been increased or decreased in size by use1 of this power, the fire immediately returns to its original size. An animated fire moves at a speed of 9 meters. A fire that moves away from its fuel or its original location dies as soon as your control over it lapses.

	An animated fire can enter any square, even if a creature already occupies it. If an animated fire enters a square occupied by a creature, that creature can make a Reflex save to get out of the way. A successful Reflex save moves the creature to the nearest unoccupied square. The flames deal the indicated damage to any creature that is either on fire or surrounded by the flames (in the fire's space); see the accompanying table.

	At the start of your turn, the animated fire deals damage to any creature in its space, and the creature catches on fire unless it makes a Reflex save. A victim on fire takes 1d6 points of damage each round. Additional rounds in the same space as the animated fire occupies mean additional chances of ignition. The damage from multiple normal fires stacks, but the victim gets a saving throw each round to negate each fire. It is possible to switch control from the animated fire (causing it to disappear) to intensify flames that are already burning (thus denying the foe Reflex saves after the first).
}