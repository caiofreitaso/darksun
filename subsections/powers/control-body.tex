\Psionic{Control Body}{control body}
{Psychokinesis (Manipulation)}
{
	\textbf{Level:}
	Kineticist 4\\
	\textbf{Manifesting Time:}
	1 mental action\\
	\textbf{Range:}
	36 meters\\
	\textbf{Target:}
	One living creature\\
	\textbf{Saving Throw:}
	Fortitude negates\\
	\textbf{Cost:}
	20 power points\\
	\textbf{Maintenance Cost:}
	4 pp/round\\
	\textbf{Prerequisites:}
	\psionic{psionic telekinesis}\\
	% \textbf{Critical Success:}
	% The subject automatically fails the first Fortitude save\\
	% \textbf{Critical Failure:}
	% You suffer partial paralysis (an arm or leg) for 1d10 turns\\
}
{
	This power allows psychokinetic control of another person's body. In effect, the victim becomes a marionette. They know that someone else is pulling their strings, though, and they're probably mad as all get-out.

	\emph{Control body} doesn't require mental contact with the subject, since you are actually forcing limb movements independent of the target's mind. You can force the subject to stand up, sit down, walk, turn around, and so on, but operating the vocal cords is too difficult. You can also hold the subject immobile, rendering it helpless. You cannot force the subject to manifest powers, cast spells, or use any special ability that is not a function of just its body movements. If you lose line of sight to the subject, the effect of this power ends. If the body is forced to do something obviously suicidal, like walking off a cliff, the subject can make another Fortitude save to regain control.

	If you force the subject to engage in combat, it gains $-6$ penalty on its attack bonus. A subject of this power cannot make attacks of opportunity. The subject gains no benefit to Armor Class from its Dexterity, but it does gain a bonus to its AC equal to your Intelligence bonus.

	Although the subject's body is under your control, the subject's mind is not. Creatures capable of taking purely mental actions (such as manifesting powers) can do so.
}