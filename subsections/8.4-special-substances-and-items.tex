\subsection{Special Substances And Items}

\ItemTable{Special Substances and Items}{
Acid (flask) & 10 cp & 0.5 kg\\
Alchemist's fire (flask) & 20 cp & 0.5 kg\\
Antitoxin (vial) & 50 cp &\\
Balican sting & 5 Cp & 0.5 kg\\
Chitin ointment & 40 Cp & 0.5 kg\\
Draxia ointment & 20 Cp & 0.5 kg\\
Esperweed & 250 cp &\\
Everburning torch & 110 cp & 0.5 kg\\
Holy water (flask) & 25 cp & 0.5 kg\\
Hypnotic brew & 30 cp & 0.5 kg\\
Ignan tallgrass & 100 Cp &\\
Kuzza powder & 20 Cp &\\
Ranike sap (1 liter) & 2 Cp & 0.5 kg\\
Smokestick & 20 cp & 0.25 kg\\
\multicolumn{3}{l}{\textit{Splash-globe}}\\
~ Acid & 10 cp &\\
~ Kip pheromones & 30 cp &\\
~ Liquid darkness & 10 cp &\\
~ Liquid dust & 10 cp &\\
~ Liquid fire & 10 cp &\\
~ Liquid light & 10 cp &\\
~ Poison & Poison cost $\times$ 1.5 &\\
~ Ranike sap smoke & 10 cp &\\
~ Stench cloud & 50 cp &\\
~ Stun cloud & 35 cp &\\
Sunrod & 2 cp & 0.5 kg\\
Tanglefoot bag & 50 cp & 2 kg\\
Thunderstone & 30 cp & 0.5 kg\\
Tindertwig & 1 cp &\\
}

The following items are often, but not always available for sale in the Bard's Quarter of most city‐states. Contacting someone willing to sell these and other associated goods usually requires proficient use of the \skill{Bluff}, \skill{Diplomacy}, and/or \skill{Gather Information} skills.

Any of these substances except for the everburning torch and holy water can be made by a character with the \skill{Craft} (alchemy) skill.


\textbf{Acid:} You can throw a flask of acid as a splash weapon. Treat this attack as a ranged touch attack with a range increment of 10 feet. A direct hit deals 1d6 points of acid damage. Every creature within 5 feet of the point where the acid hits takes 1 point of acid damage from the splash.

\textbf{Alchemist's Fire:} You can throw a flask of alchemist's fire as a splash weapon. Treat this attack as a ranged touch attack with a range increment of 10 feet.

A direct hit deals 1d6 points of fire damage. Every creature within 5 feet of the point where the flask hits takes 1 point of fire damage from the splash. On the round following a direct hit, the target takes an additional 1d6 points of damage. If desired, the target can use a full-round action to attempt to extinguish the flames before taking this additional damage. Extinguishing the flames requires a DC 15 Reflex save. Rolling on the ground provides the target a +2 bonus on the save. Leaping into a lake or magically extinguishing the flames automatically smothers the fire.

\textbf{Antitoxin:} If you drink antitoxin, you get a +5 alchemical bonus on Fortitude saving throws against poison for 1 hour.

\textbf{Balican Sting:} This mixture of many vegetal irritants is used in conjunction with the flint-tipped javelin of the Balican fleet. Bards working for the late king Andropinis developed the substance to improve the damage done by his warriors fighting against the thick-skinned giants. This mixture, which is only effective against giants of the beasthead, crag, desert, or plains variety, causes the wound made by a balican javelin that breaks within it to itch. Unless a DC 15 Wisdom check is made by the giant on each of the following 1d4 rounds, he will scratch and inadvertently rub the shallow shards deeper, causing an additional 1d4 points of damage for each failed check.

\textbf{Chitin Ointment:} This salve is used to cure damaged chitin on kreens and other insectoid creatures. Once applied, as a standard action, this substance mends brittle or broken chitin, effectively stabilizing the creature if it had less than 0 hit points. Applying this substance to non-chitinous creatures produces no effects.

\textbf{Draxia Ointment:} The draxia weed grows on the islands of the Sea of Silt. It can be turned into an ointment that repels silt spawn by mixing the plant's juices with oil or fat. The ointment, when applied to the skin, emits a smell that repels silt spawn for two hours. Silt spawn will not come within 10 feet of a creature or object coated with draxia ointment. Although adult silt horrors find the smell irritating, they are usually unaffected by it. Sometimes silt horrors are irritated to such a level, however, that they may attack the creature or object giving off the smell. There is a 60\% chance that a silt horror will ignore all other targets and instead attack a character or object that smells of draxia weed.

\textbf{Everburning Torch:} This otherwise normal torch has a continual flame spell cast upon it. An everburning torch clearly illuminates a 20-foot radius and provides shadowy illumination out to a 40-foot radius.

\textbf{Holy Water:} Holy water damages undead creatures and evil outsiders almost as if it were acid. A flask of holy water can be thrown as a splash weapon.

Treat this attack as a ranged touch attack with a range increment of 3 meters. A flask breaks if thrown against the body of a corporeal creature, but to use it against an incorporeal creature, you must open the flask and pour the holy water out onto the target. Thus, you can douse an incorporeal creature with holy water only if you are adjacent to it. Doing so is a ranged touch attack that does not provoke attacks of opportunity.

A direct hit by a flask of holy water deals 2d4 points of damage to an undead creature or an evil outsider. Each such creature within 1.5 meter of the point where the flask hits takes 1 point of damage from the splash.

Temples to good deities sell holy water at cost (making no profit).

\textbf{Ignan Tallgrass:} A redish plant that grows in the Burning Plains near the Last Sea, ignan tallgrass can be harvested from the plains after flashfires, when they are easily spotted in small clumps untouched by the fires. Ignan tallgrass is tough and can be used to make mats and roofs of twinned fibers that stay fireproof for several months, if the harvesters are brave enough to face the flashfires to get to it, as the plant cannot be cultivated. If ignan tallgrass is sun-dried, crushed, and ingested within a week of it being picked, unless somehow magically kept fresh (as through the nurturing seeds spell), it confers resistance to fire 1 for one hour.

\textbf{Kuzza Powder:} Kuzza peppers are very hot. Typically, these vivid red peppers, when ripe, measure 2 to 2 1/2 inches long. These peppers are sometimes dried and ground into a powder by unscrupulous gladiators who use a blowpipe to blow the powder on a target, causing sever irritation. Treat this blowpipe as a blowgun with half the range increment. Filling a blowpipe is a move action that provokes attacks of opportunity. A direct hit blinds a creature for 1 round unless it makes a Fortitude DC 15. Every creature within 5 feet of the target takes a -2 penalty to Search and Spot checks for 5 rounds.

\textbf{Ranike Sap:} The sap of the ranike tree, which constantly runs down its bark, is toxic to insects. Gulg posesses the secret of safely extracting large quantities of sap from this tree, effectively milking the tree in a process called “bleeding”. If a liter of the sap is poured in a large receptacle, such as a brazier, and lit afire, a clear smoke that impairs neither vision or breathing forms, filling a 50-foot cube (a moderate or stronger wind dissipates the smoke in 5 rounds). The smoke repels mundane insects, while giant insects, or those creatures that can be categorised as insect-like (such as antloids, kanks, and thri-kreen), that breathe or contact the smoke must make a DC 15 Fortitude save each round for one minute; failure indicates that they are sickened for that round. The sap burns 1 hour for each liter of sap in the receptacle, after which the smoke dissipates naturally.

A shallow depression in the ground several feet wide can replace the need for a receptacle. The sap can also be used to deliniate an area---each liter poured on the ground can create a line a few inches wide and 10 feet long. When such a line is set afire, it burns for 1 minute and creates smoke in an area 10 feet long by 5 feet wide and high.

\textbf{Smokestick:} This alchemically treated wooden stick instantly creates thick, opaque smoke when ignited. The smoke fills a 10-foot cube (treat the effect as a fog cloud spell, except that a moderate or stronger wind dissipates the smoke in 1 round). The stick is consumed after 1 round, and the smoke dissipates naturally.

\textbf{Splash-globes:} Splash-globes are spherical glass jars containing contact poison or up to half a pint of some alchemical fluid. In addition to bursting on impact like any grenade, splash-globes can be placed in hinged pelota, thus giving the grenade additional range when fired through a splash-bow or dejada. The following types of splash-globes are available:

 \textit{Acid:} Standard flask acid can be placed in splash-globes.

 \textit{Contact Poison:} Any contact poison can be placed in a splash-globe.

 \textit{Kip Pheromones:} This splash-globe is commonly crafted by bards using kip pheromones collected by dwarven kip herders. The liquid contained within the globe is an alchemical mixture that turns into smoke on contact with air. The smoke produced is clear and does not impair vision or breathing, filling a 10-foot cube for one minute (a moderate or stronger wind dissipates the smoke in 1 round). Those within the smoke must make a DC 15 Fortitude save each round they are in contact with it or become fascinated for the as long as the smoke remains. Dwarves gain a +4 racial bonus on their Fortitude save against kip pheromones.

 \textit{Liquid Darkness:} Anyone struck directly by liquid darkness must make a Reflex save (DC 15) or be blinded for one minute. Those splashed with liquid darkness have their vision blurred for one minute if they fail a DC 15 Reflex save, granting their opponents concealment. In addition, all natural fires within the splash area are instantly extinguished. Liquid darkness immediately extinguishes liquid light.

 \textit{Liquid Dust:} The liquid from this splash-globe turns into dust on contact with the air. You can use this liquid to cover up to 20 1.5-meter squares of tracks. On impact, liquid dust forms a 4.5-meter diameter cloud, ten feet high that lasts one round. Alternately, liquid dust can be launched via slash-globes. Anyone struck directly by liquid dust must make a DC 15 Fortitude save each round for one minute; failure dictates that they are nauseated for that round. Those splashed with liquid dust suffer the same effect for one round if they fail a DC 15 Fortitude save.

 \textit{Liquid Fire:} Alchemist's fire can be placed in splash-globes.

 \textit{Liquid Light:} This splash-globe contains two liquids that mix together when the splash-globe is ruptured. The resulting mixture glows for eight hours. If you break the liquid light globe while it is still in its pouch, the pouch can serve as a light source just like a sunrod. Anyone struck directly by liquid light must make a DC 20 Fortitude save or be temporarily dazzled (–1 on all attack rolls) for 1 minute, and will glow in darkness for eight hours unless they somehow cover the affected areas. Creatures splashed with liquid light (see grenade rules) also glow in the darkness, but are not blinded.

 \textit{Ranike Sap Smoke:} The liquid from this splash-globe is an alchemical mixture of ranike sap that turns into smoke on contact with air. The smoke produced is clear and does not impair vision or breathing, filling a 10-foot cube (a moderate or stronger wind dissipates the smoke at the end of the character's action). The smoke repels mundane insects, while giant insects, or those creatures that can be categorised as insect-like (such as antloids, kanks, and thri-kreens), that breath or enter in contact with the smoke must make a DC 15 Fortitude save each round for one minute; failure indicates that they are sickened for that round. This small quantity of sap only reacts with the air for 1 round, after which the smoke dissipates naturally.

 \textit{Stench Cloud:} The liquid inside this splash-globe is crafted from fordorran musk and stinkweed extract. The foul liquid turns into smoke on contact with air. The smoke produced is clear and does not impair vision or breathing, filling a 10-foot cube for one minute (a moderate or stronger wind dissipates the smoke in 1 round). Those within the smoke must make a DC 15 Fortitude save each round they are in contact with it or become nauseated for as long as they remain in contact with the cloud.

 \textit{Stun Cloud:} The liquid inside this splash-globe is crafted from boiled floater jelly combined with the pulped spines from a poisonous cactus. The liquid turns into smoke on contact with air. The smoke produced is clear and does not impair vision or breathing, filling a 10-foot cube for one minute (a moderate or stronger wind dissipates the smoke in 1 round). Those within the smoke must make a DC 15 Fortitude save each round they are in contact with it or become stunned for as long as they remain in contact with the cloud.

\textbf{Sunrod:} This 1-foot-long, gold-tipped, iron rod glows brightly when struck. It clearly illuminates a 30-foot radius and provides shadowy illumination in a 60-foot radius. It glows for 6 hours, after which the gold tip is burned out and worthless.

\textbf{Tanglefoot Bag:} When you throw a tanglefoot bag at a creature (as a ranged touch attack with a range increment of 10 feet), the bag comes apart and the goo bursts out, entangling the target and then becoming tough and resilient upon exposure to air. An entangled creature takes a -2 penalty on attack rolls and a -4 penalty to Dexterity and must make a DC 15 Reflex save or be glued to the floor, unable to move. Even on a successful save, it can move only at half speed. Huge or larger creatures are unaffected by a tanglefoot bag. A flying creature is not stuck to the floor, but it must make a DC 15 Reflex save or be unable to fly (assuming it uses its wings to fly) and fall to the ground. A tanglefoot bag does not function underwater.

A creature that is glued to the floor (or unable to fly) can break free by making a DC 17 Strength check or by dealing 15 points of damage to the goo with a slashing weapon. A creature trying to scrape goo off itself, or another creature assisting, does not need to make an attack roll; hitting the goo is automatic, after which the creature that hit makes a damage roll to see how much of the goo was scraped off. Once free, the creature can move (including flying) at half speed. A character capable of spellcasting who is bound by the goo must make a DC 15 Concentration check to cast a spell. The goo becomes brittle and fragile after 2d4 rounds, cracking apart and losing its effectiveness. An application of universal solvent to a stuck creature dissolves the alchemical goo immediately.

\textbf{Thunderstone:} You can throw this stone as a ranged attack with a range increment of 20 feet. When it strikes a hard surface (or is struck hard), it creates a deafening bang that is treated as a sonic attack. Each creature within a 10-foot-radius spread must make a DC 15 Fortitude save or be deafened for 1 hour. A deafened creature, in addition to the obvious effects, takes a -4 penalty on initiative and has a 20\% chance to miscast and lose any spell with a verbal component that it tries to cast.

Since you don't need to hit a specific target, you can simply aim at a particular 5-foot square. Treat the target square as AC 5.

\textbf{Tindertwig:} The alchemical substance on the end of this small, wooden stick ignites when struck against a rough surface. Creating a flame with a tindertwig is much faster than creating a flame with flint and steel (or a magnifying glass) and tinder. Lighting a torch with a tindertwig is a standard action (rather than a full-round action), and lighting any other fire with one is at least a standard action.