\subsectionA{The Gray}
When a character enters the Gray, its vast emptiness stretches out before him. It can be extremely hard to keep one's bearing in the Gray, especially without a reference point. Distances are hard to determine, and even figuring out which direction one is moving can be a challenge!

The Gray is coexistent with the Material Plane but separate from the Elemental Planes. The Material Plane itself is visible from the Gray, but it appears muted and indistinct, its colors blurring into each other and its edges turning fuzzy. The Gray is usually invisible to those on the Material Plane, and creatures in the Gray cannot normally attack creatures on the Material Plane, and vice versa. A traveler in the Gray is invisible, incorporeal, and utterly silent to someone on the Material Plane.

Visibility in the Gray is reduced to twilight levels, but the dimness does not grant concealment to creatures. Low-light vision and darkvision function normally. A creature's range of vision to the Material Plane is limited to 18 meters in any direction. %Each minute a traveler moves deeper, his range of vision shrinks by 3 meters until he can only see 3 m around. At that point, the creature is at the Deep Gray (see below).

Living, corporeal creatures cast a faint glow that, though perceptible, fails to illuminate any of the ashen drear of the Gray. However, the warm body of a living creature appears as a beacon, visible up to a mile and often drawing spirits near'.

Creatures move at half speed in the Gray, though they may move in any direction.

\subsubsection{Gray Traits}
\begin{itemize*}
\item \textbf{No gravity.} Gravity does not exist in this plane of nothing. There is no concept of up or down in the Gray. The effect can be extremely disorienting.
\item \textbf{Mildly neutral-aligned.}
\item \textbf{No Elemental or Energy Traits.} The Gray does not pose an immediate danger to living creatures traveling within it.
\item \textbf{Enhanced Magic:} In the Gray, a wizard can draw energy for a spell from an incorporeal undead (whether or not it has been forced into corporeal form), as the undead acts as a battery of energy. As part of casting a spell, a wizard can make a touch attack against the undead, dealing 1d6 points of damage per level of the spell to be powered by its energy. This touch attack is a free action that provokes attacks of opportunity.
\item \textbf{Impeded Magic and Psionics:} Spells and powers that draw upon the power of the Black are completely useless, for there is no light or shadow in the Gray. Spells and powers of the shadow subschool or with the light or darkness descriptors fail, absorbed into the surroundings.\\

Since the Gray contains no plant life, wizards in the Gray cannot draw magic for their spells. Items that contain magical charges (rods, staves, wands and scrolls) still function, as the energy powering the magical effect is contained within the item.
\item \textbf{Alterable Morphic.} The plane contains little to alter, however.
\end{itemize*}

% \subsubsection{The Deep Gray}
% The Deep Gray is the lower layer of the Gray, the furthest from the Material Plane. In the Deep Gray is where lost travelers lose their lives, as the negative energy from the plane takes their vitality.

% Once a creature becomes lost in the Deep Gray, movement becomes abstract as the plane has no landmarks, so the concepts of distance or direction lose meaning. Only time matters.

% \Table{}{Xl}{
%   \tableheader Goal
% & \tableheader Time Needed\\
% Returning to the Gray & 1d10 minutes \\
% Finding a specific object on the Gray & 1d10$\times$100 hours \\
% }

% The Deep Gray shares the same traits as the Gray, except as noted here.
% \begin{itemize*}
% \item \textbf{Major negative-dominant:} Each round, those within must make a DC 25 Fortitude save or gain a negative level. A creature whose negative levels equal its current levels or Hit Dice is slain, becoming a wraith.\\

% Some areas within the plane have only the minor negative-dominant trait, and these islands tend to be inhabited.
% \item \textbf{Enhanced magic.} Spells and spell-like abilities that use negative energy are maximized (as if the Maximize Spell metamagic feat had been used on them, but the spells don't require higher-level slots). Spells and spell-like abilities that are already maximized are unaffected by this benefit. Class abilities that use negative energy, such as rebuking and controlling undead, gain a +10 bonus on the roll to determine Hit Dice affected.
% \item \textbf{Impeded magic} Spells and spell-like abilities that use positive energy, including \emph{cure} spells, are impeded. Characters on this plane take a $-10$ penalty on Fortitude saving throws made to remove negative levels bestowed by an energy drain attack.
% \end{itemize*}
