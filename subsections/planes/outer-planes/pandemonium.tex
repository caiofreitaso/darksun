\subsectionA{Windswept Depth of Pandemonium}
Pandemonium is a great mass of matter pierced by innumerable tunnels carved by the howling winds of the plane. It is windy, noisy, and dark, having no natural source of light. The wind quickly extinguishes normal fires, and lights that last longer draw attention of wights driven insane by the constant howling wind.

Every word, scream, or shout is caught by the wind and flung through all the layers of the plane. Conversation is accomplished by shouting, and even then words are spirited away by the wind beyond 3 meters. Likewise, spells and effects that rely on sonic energy have their range limited to 3 meters. Travelers are temporarily deafened after 1d10 rounds of exposure to the winds, and permanently deafened after 24 hours of exposure. Temporarily deafened characters regain their hearing after 1 hour spent out of the wind.

Ear plugs or similar devices negate the deafening effect. Of course, wearing ear-plugs effectively mimics the normal effects of being deafened.

The stale wind of Pandemonium is cold, and it steals the heat from travelers unprotected from its endless gale that buffets each inhabitant, blowing sand and dirt into eyes, snuffing torches, and carrying away loose items. In some places, the wind can howl so fiercely that it lifts creatures off their feet and carries them for miles before dashing their forms to lifeless pulp against some dark, unseen cliff face.

In a few relatively sheltered places, the wind dies down to just a breeze carrying haunting echoes from distant pans of the plane, though they are so distorted that they sound like cries of torment.

Erythnul, the Lord of Slaughter, makes his terrible domain on Pandemonium.

Pandemonium has four layers: Pandesmos, Cocytus, Phlegethon, and Agathion.

Pandemonium has the following traits.
\begin{itemize*}
\item \textbf{Objective Directional Gravity:} In the cavernous tunnels of Pandemonium, gravity is oriented toward whatever wall a creature is nearest. Thus, there is no normal concept of floor, wall and ceiling---any surface is a floor if you're near enough to it. Rare narrow tunnels exactly cancel out gravity, allowing a traveler to shoot through them at incredible speed. The layer of Phlegethon is an exception---there the normal gravity trait applies.
\item \textbf{Divinely Morphic:} Specific powerful beings such as the deity Erythnul can alter Pandemonium. Ordinary creatures find Pandemonium indistinguishable from the Material Plane (the alterable morphic trait, in other words). Spells and physical effort affect Pandemonium normally.
\item \textbf{Mildly Chaos-Aligned:} Lawful characters on the plane of Pandemonium suffer a $-2$ penalty on all Charisma-based checks.
\end{itemize*}

\subsubsection{Windstorms on Pandemonium}
The constant winds on Pandemonium can gust with howls so maddening and speeds so enormous that they become dangerous.

Those caught without shelter when one of Pandemonium's windstorms blows up are in trouble; both mind and body are in peril. A windstorm has a 10\% chance per day of blowing through a given area. Generally, a windstorm gusts through an area in 1 round.

\Table{Windstorms on Pandemonium}{lXp{25mm}}{
  \tableheader d\%
& \tableheader Effect
& \tableheader Saving Throw \\
01--10 & Flying pebbles deal 1d4 points of damage & Reflex DC 15 half \\
11--20 & Pelting stones deal 2d6 points of damage & Reflex DC 18 half \\
21--30 & Howling wind causes \spell{confusion} for 1d4+1 rounds & Will DC 15 negates \\
31--40 & Flying boulders deal 2d8 points of damage & Reflex DC 20 half \\
41--50 & Cacophonous wind causes \spell{confusion} for 2d4+1 rounds & Will DC 18 negates \\
51--60 & Wind picks up travelers, dashing them against rock wall for 2d10 points of damage & Reflex DC 22 half \\
61--70 & Screaming wind causes \spell{confusion} for 2d4+1 rounds & Will DC 20 negates \\
71--80 & Wind picks up travelers, dashing them against rock wall for 4d10 points of damage & Reflex DC 24 half \\
81--90 & Wind picks up travelers, dashing them against rock wall for 4d10 points of  damage, then blows them into a tributary of the River Styx & Reflex DC 24 half, then Reflex DC 20 negates \\
91--100 & Shrieking wind causes \spell{insanity} & Will DC 22 negates \\
}

\subsubsection{Pandesmos}
The first layer of Pandemonium has the largest caverns, with some big enough to hold entire nations. Large or small, most caverns are desolate and abandoned to the winds.

Several of Pandesmos's caverns and tunnels possess a feature in common besides the omnipresent wind. Streams of frigid water flow from cavern to cavern, some down the center of the tunnel in midair because the objective gravity exerted by each wall cancels out the others. Many of these streams, but not all, are tributaries of the River Styx.

\textbf{Madhouse:} A group of outsiders known as the Bleak Cabal maintains a citadel in Pandesmos that serves as a way station for travelers. The Madhouse is a sprawling edifice of haphazardly organized buildings divided by several circular stone walls. The citadel is so large it fills an entire cavern, covering every surface. The place is rife with travelers, petitioners, and natives. Available services include lodging and most other services one might expect in i normal city. However, a respectable percentage of the Madhouse's populace is insane, deaf, or both.

\textbf{Winter's Hall:} This region of Pandemonium is snowy and blizzard-ridden. Visibility, even when light can be had, is only a few feet. The snow never rests; the winds constantly whip it up so it coats tunnels and even creatures with a uniform layer of ice. Frost giants and winter wolves prowl the cold waste. These creatures serve a particularly cruel entity called many names but most often venerated as the Trickster.

\subsubsection{Cocytus}
The tunnels of Cocytus tend to be smaller than those of Pandesmos, which means that they funnel the winds more strongly. The resulting wails have earned Cocytus the nickname ``layer of lamentation.'' Strangely, the tunnels on this layer bear the marks of having been hand-chiseled, but such an undertaking must have occurred so long ago that years do not suffice as a measure.

\textbf{Howler's Crag:} A jagged spike of stone stands in the center of Cocytus. The Crag is a jumbled pile of stones, boulders, and worked stone, as if a giant's palace had collapsed in on itself. The Crag's top is mostly a level latform about eight feet in diameter, with a low wall surrounding it. The platform and those on it glow with an ephemeral blue radiance. The lower reaches of the Crag are riddled with small burrows. Some are merely dead ends, but others connect. The wall of every burrow is covered with lost alphabets that supposedly spell out strange psalms, liturgies, and strings of numerals or formulas.

Natives of Pandemonium say that anything yelled aloud from the top of the Crag finds the ears of the intended recipient, no matter where that recipient is on the Great Wheel. The words of the message are borne on a shrieking, frigid wind.

Demons of various sorts have learned that visitors constantly trickle to the crag. The visitors are usually archeologists, diviners, and those wishing to send a message to some lost friend or enemy. Most become the prey of the ambushing fiends.

\textbf{Harmonica:} Legend tells of a site in Cocytus called Harmonica. In this place, the winds whip through a cavern with holes and tubes chiseled into gargantuan rock columns, creating a noise worse than anywhere else in the plane. Somewhere within this mazelike realm of tortured cacophony lies the true secret of planewalking: the art of traveling the planes without a portal, spell, or device of any kind. In all likelihood, this secret is a legend with no basis in fact, but that doesn't stop the occasional seeker from finding, then dying among, the columns of Harmonica.

\subsubsection{Phlegethon}
The unrelenting noise of dripping water meshes with the howling winds of Phlegethon's narrow, twisting runnels. The rock itself absorbs light and heat. All light sources, natural and magic, only shine to half their normal distance. Unlike on the other layers, normal gravity applies in Phlegethon's tunnels, giving rise to intricate stalagmite and stalactite formations, which in turn are constantly weathered by the brutal wind.

\textbf{Windglum:} Windglum is a city of Banished in a cavern several miles wide and long, with enormous natural columns that hold up the cavern's ceiling. Hundreds of ever-burning globes provide light for the city, illuminating a disordered sprawl of individual homes. The homes in turn surround a fortification known locally as the Citadel of Loros.

Windglum is characterized by an aura of suspicion. The locals are unlikely to trust strangers, and many of Windglum's citizens are mentally unstable. However, one inn in Windglum welcomes strangers. Called the Scaly Dog, it's a place where a planar traveler can meet other wayfarers, hire mercenaries, gather information, or seek employment.

\textbf{Citadel of Slaughter:} Called ``The Many,'' the intermediate deity Erythnul is lord of envy, malice, panic, ugliness, and slaughter. Erythnul is a brutal deity who makes his home in what appears to be a tumbled ruin of some vast citadel. In fact, its tortuous passages channel cold winds on which can always be heard the sound of terrible battle. Battle-mad petitioners of all races infest the passages, and they desire nothing other than to hunt and slay each other in cold blood.

At the center of the pile is Erythnul himself, usually engaged in the slaughter of an endless stream of petitioners, as well as the occasional mortal captive. In battle, the deity's features change between human, gnoll, bugbear, ogre, and troll. If ever Erythnul's blood is spilled, it transforms into an allied creature of whatever form Erythnul currently wears.

No one goes to the Citadel of Slaughter on purpose, unless they serve Erythnul and seek to join in the deity's eternal slaughter.

\subsubsection{Agathion}
In the fourth layer, the narrowing tunnels finally constrict down to nothing, leaving behind an infinite number of closed-off spaces filled with stale air or vacuum surrounded by an infinitude of solid stone. The portals that connect Agathion to the rest of Pandemonium open into the otherwise unreachable bubbles, but the act of stepping through a portal always sets off a windstorm.

Unless you know where the portal is, the closed-off spaces of Agathion are almost impossible to find. For this reason, forgotten spaces have been used by deities (and other powerful entities that predate the current deities) as vaults where items are hidden away. Such items may include uncontrollable artifacts, precious mementos, lost languages, unborn cosmologies, and monsters of such cataclysmic power that they couldn't be slain or otherwise neutralized.
