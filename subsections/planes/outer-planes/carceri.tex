\subsectionA{Tarterian Depths of Carceri}
Carceri seems the least overtly dangerous of the lower planes, but that first impression quickly disappears. Acid seas and sulfurous atmospheres may be rare on this plane, and there are no areas of biting cold or infernos of raging heat. The danger of Carceri is a subtler thing.

The plane is a place of darkness and despair, of passions and poisons, and of kingdom-shattering betrayals. On Carceri, hatreds run like a deep, slow-moving river. And there's no telling what the flood of treachery is going to consume next. It is said that a prisoner on Carceri may only escape when she has become stronger than whatever imprisoned her there. That's a difficult task on a plane whose very nature breeds despair, betrayal, and self-hatred.

Unlike most inhabitants of Carceri, the deity Nerull makes his home on Carceri willfully, not because of exile.

Carceri consists of six layers. Each layer has a series of orbs like tiny planets, in a row. A gulf of air separates each orb from the next. On a particular layer, little distinguishes one orb from the next, and it's possible that the number of orblike planets on each layer is infinite.

It has the following traits.
\begin{itemize*}
\item \textbf{Divinely Morphic:} Nerull and any other entity of lesser deity power or greater can alter Carceri. More ordinary creatures find Carceri indistinguishable from the Material Plane; it responds to spells and physical effort normally.
\item \textbf{Mildly Evil-Aligned:} Good characters on Carceri suffer a $-2$ penalty on all Charisma-based checks.
\end{itemize*}

\subsubsection{Orthrys}
Orthrys, the first layer of Carceri, is a realm of vast bogs and quicksand. The River Styx runs freely through the layer, saturating the ground with its magic. Channels carved into the soft ground through eons of erosion are wide and deep. Where there is no river, there are swamps. Though patches of dry ground exist, they are rare and usually climb swiftly to rugged mountains where enraged titans dwell.

Mosquitoes swarm the air above the bogs, annoying travelers. Even more annoying are the smooth-talking petitioners that populate this dreary realm.

\textbf{Bastion of Last Hope:} A fortress made of black igneous rock squats in a mountain range of Orthrys. The ambient, reddish light of the plane lends the Bastion of Last Hope a brooding air of menace. Only one entrance offers itself, and those entering can't help but notice that the entrance strongly resembles the maw of some massive demonic toad.

No one person rules the Bastion. Instead, it serves as a sort of outpost for anarchists. Here a traveler can obtain all manner of forged documents, surgical alterations to aid a permanent disguise, and various other nefarious goods and services. It is a good place to find assassins, spies, and others of ill repute. But cunning travelers remember that they're on a plane full of traitors, so they trust no one within the Bastion's walls.

\textbf{Mount Orthrys:} The highest peaks of the mountain ranges on two of this layer's orbs reach ridiculously high, just bridging the planetary gulf between them. At their intersection is a titanic palace of white marble columns, amphitheaters, and galleries. Here lives a race of titans, banished from the Material Plane long ago. The titan lord of Mount Orthrys, Cronus, resides at the center of his palace in a throne room a mile wide. Visitors may seek audiences with Cronus to hear his wisdom, but those who seek such counsel must be always aware that the titan's eons-long anger at his confinement may lash out unexpectedly at those who can come and go at their leisure. Cronus has the power of a lesser deity for the purposes of altering Mount Orthrys.

\subsubsection{Cathrys}
The orbs in the second layer of Carceri are covered with fetid jungles and scarlet plains. The stench of decay fills the air, a rot fueled by acidic secretions of jungle plants. Those without immunity to acid are soon rendered down to their component materials if they stay too long amid the swaying trees. The jungle air deals 1d4 points of acid damage per minute, and some plants secrete more potent acids.

The plains of Cathrys are more habitable. Vast, windswept grasslands cover the planes. Some patches possess razor-sharp leaves, which can cut a traveler not mindful of them. Those who hustle (double move) or run on the plains must make a Reflex save (DC 20) each round or cut themselves for 1d4 points of damage.

\textbf{Apothecary of Sin:} Located deep in the fetid jungles of an orb of Cathrys is the Apothecary of Sin. The Apothecary is built from cunningly woven scrap wood atop the trunk of large tree, raising the one-story structure high above the waving branches of the acid-laden leaves below. Rope-suspended catwalks provide access above the treetops, though random sections are missing, possibly victims of caustic storms. Mundane and exotic poisons and acids are bought and sold in the Apothecary.

A demon called Sinmaker runs the Apothecary. Sinmaker is a glabrezu of average abilities, except for his special affinity for acids, poisons, and venoms. He delights in all things poisonous---the more diabolical, the better. All poisons are available in the Apothecary, as well as many special, unique concoctions bought by Sinmaker from travelers or synthesized in Sinmaker's own laboratory. Acid is also sold here, by the one-dose vial or by the thousand-dose keg. Neither the size of the purchase nor the nature of the buyer matters to Sinmaker.

\subsubsection{Minethys}
The third layer of Carceri is filled with sand. Stinging grit is driven so hard by the wind that it can strip an exposed being to the bone in a matter of hours, should one of the place's terrible windstorms spring up. Sandstorms are 10\% likely in any given area per 24 hours. All who dwell in this layer, mortal and fiend alike, cover themselves in cloth garments to block out the stinging sand.

Tornadoes are common on Minethys. To avoid these hazards, petitioners live in miserable sand-filled pits, dug by hand. Their crude pits must be constantly dug out to provide even the slightest shelter.

\textbf{Sand Tombs of Payratheon:} Payratheon is the name of a vanished city built on an orb of Minethys eons ago. That city is long buried, but its sand-drowned avenues, crumbled towers, and silted porticos still remain far below the shifting surface of the layer. Sometimes the shifting sands reveal Payratheon for an hour or a longer, but it is always engulfed again by the sands, smothering most creatures who were tempted by its appearance and entered the sand-blasted city.

Particularly resourceful adventurers have burrowed down to find outlying suburbs of the city during its phases of submersion. Tales of terror walk hand in hand with these accounts, which tell of dragonlike ``sand gorgons'' that swim through the sand as if water. Also mentioned are the remnants of former inhabitants that force their way through the streets as petrified undead, so weathered and eroded that little can be discerned of their race or original size.

\subsubsection{Colothys}
The fourth layer of Carceri is a realm of mountains so tall, rough, and cruel as to stagger the imagination of a traveler from the Material Plane. Travel on foot here is almost impossible, because the land is divided by canyons miles deep where it is not lifted to absurd heights by mighty tectonics. A few trading routes do exist, usually in the form of rickety bridges and cliff-face trails barely wide enough for one.

It's impossible to move normally away from the areas along the trading routes. Characters must make \skill{Climb} checks (DC 15) to move one-half their speed as a miscellaneous full-round action.

\textbf{Garden of Malice:} The hanging gardens of Colothys are found on a single orb of the layer that travelers would do well to avoid. To the inexperienced eye, many of the cliff faces and sheer slopes of this orb are home to thick vines and tubers that sprout a riot of beautiful flowers. Characters who attempt to collect samples for their botanical collections quickly learn that the vines are animate and determined to wring the life from any creature that would dare to use them as climbing aids, defoliate the flowers, or even move too close.

It may be that the animate vines represent one large organism that has grown through the eons to cover one whole orb. Once every six hundred days, the vines release tiny seeds into the air that look like dandelion fluff. The winds of the layer often send the seeds blowing across several hundred other orbs of the mountainous realm. Though many are eaten by vermin, many other seeds have also found nourishing soil, and have sprouted tubers in small nooks and forgotten cliff-faces on other orbs.

\subsubsection{Porphatys}
The fifth layer of Carceri is a realm where each orb is coated in a cold, shallow ocean fed by constant black snow. The snow and water are mildly acidic, automatically dealing 1d6 points of acid damage per 10 minutes of direct exposure.

Artificial structures do not last long in Porphatys. Small islands barely taller than sandbars rise above the waves. Most petitioners crow from atop the small sandbar islands, promising anything to those who can take them away. Despite their entreaties, they reward any charity with betrayal at the first opportunity.

Another exiled titan lives here, but even his palace is half sunken and slowly crumbling before the acidic waves.

\textbf{Ship of One Hundred:} A ship rides the cold swells of Porphatys's seas, called the Ship of One Hundred, though in some accounts it is referred to as the White Caravel. It appears as a ghost-white caravel unmanned by any visible crew. It wends between the islets of many orbs (somehow disappearing on one orb and appearing on another), picking up stranded souls and other travelers who are brave (or foolish) enough to brave passage.

Passengers soon discover that apparently no one moves on board the craft. The lower deck and hold are stuffed with exactly one hundred unadorned stone sarcophagi. No one has ever successfully opened a sarcophagus and lived to tell the tale. Any time this has been tried, some unrecorded calamity devours all creatures currently on board, and the next time the ship puts in at a new port it is utterly empty of life. Stories have it that the ship seeks to deliver its terrible cargo, but it waits for the end times to do so.

Between the ``cleansings'' that occur when the curious try to open a sarcophagus, travelers (mostly petitioners, demons, or other creatures) infest the ship. Some make it their temporary home, happy to move from place to place by whatever mysterious force steers the ship. These denizens take a very dim view of visitors who want to open a sarcophagus.

\subsubsection{Agathys}
The coldest layer of Carceri is also the lowest---or inner-most, given the nested nature of this plane. Unlike the other layers, Agathys has only a single orb: a sphere of black ice streaked with red.

The air is bitterly cold and deals 1d2 points of cold damage each round. This layer has the minor negative-dominant trait. Petitioners here are half imbedded in the ice, their lies frozen on their lips.

\textbf{Necromanteion:} A black citadel carved out of ice is the focus of the greater deity Nerull's realm. Nerull is a deity of death and is called the Reaper, the Foe of All Good, the Bringer of Darkness, and similar names. Petitioners are frozen flush into the floors, walls, and ceilings of the Necromanteion, just as they are in the surrounding ice.

The deserted entrance to the Necromanteion leads quickly to a wide hall called the Hidden Temple, which crawls with undead of all types. The pallid, green glow of gibbering ghoul-light lanterns illuminates the area. Hundreds of onyx altars are evenly spaced around the hall, and demonic clerics constantly chant stanzas of a ghastly necromantic ritual. Besides chanting, the demonic priests spend endless hours attending grotesque experiments on necrotic flesh piled on other altars.

Nerull's throne stands at the center of the Hidden Temple. Woe betide the character who disturbs Nerull, a rust-red skeleton wearing a dull black cloak. Always clutched in Nerull's skeletal hands is his sablewood staff. Lifecutter, which projects a scythelike blade of scarlet force that has the power to slay any creature.

The Hidden Temple has several satellite chambers. Some hold food and quarters for the demonic clerics, others have cells for living captives destined to be strapped onto an onyx altar (or become food for a hungry cleric), and in some are special vaults where the relics of Nerull's faith are sealed away.

Finally, small tunnels lead deeper into the ice of the layer, supposedly connecting to vaults of horror so ghastly that even the demonic priests shy from exploring their depths. Otherworldly wailing and whispers rise up from the depths.
