\subsectionA{Heroic Domains of Ysgard}
Ysgard is a plane on an epic scale, with soaring mountains, deep fjords, and dark caverns that hide the secret forges of the dwarves. A biting wind always blows at a hero's back. From the freezing water channels to the sacred groves of Alfheim's elves, Ysgard's terrain is grand and terrible. It is a place of sharp seasons: Winter is a time of darkness and killing cold, and a summer day is scorching and clear.

Most spectacular of all, the landscape floats atop immense rivers of earth flowing forever through an endless skyscape. The broadest earthen rivers are the size of continents, while smaller sections, called earthbergs, are island-sized. Fire rages under each river, but only a reddish glow penetrates to the continent's top. Of more concern is the occasional collision between rivers, which produces terrible quakes and sometimes spawns new mountain ranges.

Ysgard is the home of slain heroes who wage eternal battle on fields of glory. When these petitioners fall, they rise again the next morning to continue eternal warfare. Two deities make their homes on Ysgard: Kord, scion of Strength; and Olidammara, patron of thieves.

Ysgard has the following traits.
\begin{itemize*}
\item \textbf{Divinely Morphic:} Specific powerful beings (such as the deities Kord and Olidammara) can alter Ysgard with a thought. They can change vast areas, creating great realms for themselves.
\item \textbf{Minor Positive-Dominant:} Ysgard possesses a riotous explosion of life in all its forms. All individuals on a positive-dominant plane gain fast healing 2 and may even regrow lost limbs in time. Additionally, those slain in the never-ending conflicts on Ysgard's fields of battle rise each morning as if \spell{true resurrection} were cast on them, fully healed and ready to fight anew. Even petitioners, who as outsiders cannot be raised, awaken fully healed. Only those who suffer mortal wounds on Ysgard's battlefields get the \spell{true resurrection} effect; dead characters brought to Ysgard don't spontaneously revive.
\item \textbf{Mildly Chaos-Aligned:} Lawful creatures on Ysgard suffer a $-2$ penalty on all Charisma-based checks.
\end{itemize*}

\subsubsection{Ysgard}
The top layer of Ysgard, also called Ysgard, is far and away the most well known and well traveled of the three layers. Most of the inhabitants live in camps and rugged settlements with rough and wild conditions. The layer is dotted with dozens of huge halls, smoking battlefields, and hilly terrain leading down to cold seas. Few settlements exist along the edges of any of the earthbergs, except those interested in trade with communities on other earthbergs.

\textbf{Kord's Realm:} The deity of the strong and courageous, Kord the Brawler lives in the Hall of the Valiant on this plane. His grand hall is built of stout beams of wood hewn from a single massive ash tree. Within, Kord presides over a never-ending banquet where honored guests come and go, but the revelry never ends. The feast tables surround a great open space where valiant heroes wrestle for sport. Sometimes, Kord himself sets aside his intelligent dragon-slaying greatsword, Kelmar, and his dragon-hide accoutrements, and enters the square to the great delight of all assembled.

\textbf{Plain of Ida:} This great field is located near the Hall of the Valiant and the great free city of Himinborg, the largest population center on the layer. The Plain of Ida hosts daily festivals where warriors can flaunt their mettle. Here, bravery and skill in battle is valued over all else.

\textbf{Alfheim:} Elven petitioners populate this brilliant, sunlit region, as does a contingent of mortal elves. Alfheim is suffused with light and joy, and visitors cannot help but be buoyed by the happiness in the air. The lands are wild and beautiful, untouched by civilization. Wildlife is plentiful, and natural features such as streams, forests, and sunny hills are likewise bountiful.

The elven natives are friendly, but they care little for anything but games and meditative appreciation of their natural surroundings. While many elves live in harmony with nature among the trees and fields of the surface, some elves abide in glittering caves below the surface of Alfheim.

Alfheim has seasons. Summers are long and kind, and its winters are dark and unforgiving. During winter, the elves retreat into the glittering caves, the entrances to which are sealed off and buried during the season of snows.

\textbf{Den of Olidammara:} The god of rogues, Olidammara the Laughing Rogue is an intermediate deity who concerns himself with music, revels, wine, humor, and similar ideals. Wood, stone, and stranger substances create a grand but haphazard structure, as if several mansions of various cultures were mashed together.

On the inside, mazes, locked doors, blind hallways, and secret treasuries surround a grand hall where music and dancing are mandatory. Usually, the guests of this inmost den include rogues, bards, performers, and entertainers of all stripes and all places. Wine, romance, and song rule here, where Olidammara lounges at his ease on a grand divan---unless he is disguised as one of his many guests using his magic laughing mask. Because some terrible prank often draws him far away from his den, other deities treat Olidammara with deserved caution no matter where they are.


\subsubsection{Muspelheim}
The middle layer of Ysgard, Muspelheim, is made from ribbons of floating earth, some continent-sized or larger. Here, though, the ground smokes and burns, earning this layer the name ``Land of Fire.'' It's a hostile layer where even the ground is sharp volcanic rock. Most of Muspelheim has the fire-dominant trait.

Muspelheim's ground rolls toward a ridge of fiery mountains at the highest point. This range, called the Serpent Spine, is home to hundreds of clans of fire giants. Watchtowers and citadels defend the mountain passes against rival clans and unwanted visitors.

The Spire is a towering, needle-thin citadel of dark stone in the midst of the Serpent Spine mountains. Devout fire giant maidens are said to inhabit the tower, serving as clerics of a mysterious intermediate deity of fire giants.

\subsubsection{Nidavellir}
The third layer of Ysgard is Nidavellir. It is an ``underground'' realm crisscrossed by warm tunnels, heated by hot springs and geysers. The wild regions are crowded with underground forests of strange woods that need no sun, only heat, to grow. Vast caverns run through veins of clear quartz, and deep holds are studded with shining mica and pyrite. Precious and semiprecious minerals are strewn across the floor of some lengths of runnel and even entire caverns.

Dwarven and gnome kingdoms divide up most of Nidavellir. Most of the layer's inhabitants are mortals, but petitioners are common as well. It is a place of fiery furnaces, ringing anvils, and constant striving for perfection in the crafts of smithing, runecrafting, and magic. Its halls resound with the chanting voices of dwarves and the lilting songs of gnomes. Though the two races are rivals often given to war, they unite when confronted by their underground enemies: dark elves.

\textbf{Svartalfheim:} Drow have their own realm in Nidavellir. Though the gnomes and dwarves think the worst of the dark elves, the allegiances of these particular drow are not as evil as many travelers might think. Like others of this layer, they merely wish to be left alone and they don't take kindly to unannounced visitors or trespassers.
