\subsectionA{Ever-Changing Chaos of Limbo}
Limbo is a plane of pure chaos. Untended sections appear as a roiling soup of the four basic elements and all their combinations. Balls of fire, pockets of air, chunks of earth, and waves of water battle for ascendance until they in turn are overcome by yet another chaotic surge. However, landscapes similar to ones found on the Material Plane drift through the miasma: bits of forest, meadow, ruined castles, and small islands.

Limbo is inhabited by living natives. Most prominent of these are the githzerai and the slaadi. In Limbo, most petitioners take the form of unthinking, ghostly spheres of swirling chaos.

Limbo has no layers. Or if it does, the layers continually merge and part, each is as chaotic as the next, and even the wisest sages would be hard-pressed to distinguish one from another.

Maps are useless in the chaotic expanse. Over time, even solid, permanent structures drift in the chaotic currents of Limbo. The time it takes an individual or group of individuals to reach a particular area depends on how familiar they are with that area:

\Table{}{XX}{
  \tableheader Familiarity
& \tableheader Travel Time \\
Very familiar     & 2d6 hours \\
Studied carefully & 1d4 $\times$ 6 hours \\
Seen casually     & 1d4 $\times$ 10 hours \\
Viewed once       & 1d6 $\times$ 20 hours \\
Description only  & 1d10 $\times$ 50 hours \\
}

It has the following traits.
\begin{itemize*}
\item \textbf{Subjective Directional Gravity:} Inhabitants of the plane determine their own ``down'' direction. Objects not under the motive force of others do not move.
\item \textbf{Highly Morphic:} Limbo is continually changing, and keeping a particular area stable is difficult. A given area, unless magically stabilized somehow, can react to specific spells, sentient thought, or the force of will. Left alone, it continually changes. For more information on stabilization, see Controlling Limbo, below.
\item \textbf{Sporadic Element-Dominant:} No one element constantly dominates Limbo. Each element (Earth, Water, Air, or Fire) is dominant from time to time, so any given area is a chaotic, dangerous boil. The elemental dominance can change without warning.
\item \textbf{Strongly Chaos-Aligned:} Nonchaotic characters suffer a $-2$ penalty on all Charisma-, Wisdom-, and Intelligence-based checks. However, the strongly chaos-aligned trait disappears within the walls of githzerai monasteries (but not githzerai cities).
\item \textbf{Wild Magic:} Spells and spell-like abilities in Limbo function in wildly different ways. They function normally within permanent structures or on permanently stabilized landscapes in Limbo. But any spell or spell-like ability used in an untended area of Limbo, or an area temporarily controlled, has a chance to go awry. The spellcaster must make a level check (1d20 + spellcaster level) against a DC of 15 + the level of the attempted spell. If the caster fails the check, roll on \tabref{Wild Magic Effects}.
\end{itemize*}


\subsubsection{Controlling Limbo}
There are three kinds of terrain in Limbo: uncontrolled raw areas, controlled areas, and stabilized areas. Raw areas make up most of the plane, while the controlled areas (also called tended areas) and stabilized areas are tiny islands in comparison.

\textbf{Raw Limbo:} Uncontrolled areas of limbo are dangerous, but most sentient creatures can exert a localized calming influence (see Controlled Limbo, below). But sometimes there's no control, such as when a visitor first enters Limbo or when a traveler is knocked unconscious. When no one's trying to control a given area of Limbo, it exhibits the qualities noted on the table below. For the purposes of this table, an area is everything within a 9-meter-radius sphere, though areas can drift and move around randomly. For a given area, roll on the table once every 1d10 minutes.

\Table{Uncontrolled Limbo}{lX}{
  \tableheader d\%
& \tableheader Effect \\
01--10  & Air-dominant \\
11--20  & Earth-dominant \\
21--30  & Fire-dominant \\
31--40  & Water-dominant \\
41--50  & Mixed dominant: Air and earth \\
51--60  & Mixed dominant: Fire and earth \\
61--70  & Mixed dominant: Water and earth \\
71--80  & Mixed dominant: Water and air \\
81--90  & Mixed dominant: Air and fire \\
91--100 & Balance (as if air-dominant) \\

\TableNote{2}{\textbf{Element-Dominant:} The indicated type of element surges in the given area. The previous dominant element is wiped away in the first round, and the effects of the new dominant element come into play immediately. Limbo's subjective gravity trait overrides elemental gravity traits that conflict with it.}\\

\TableNote{2}{\textbf{Mixed Dominance:} Two elements mix together, creating a hybrid effect. All effects of both element-dominant traits simultaneously affect the area. In addition to the trait effects, the region develops a chaotic mix of both elements. For example, where earth and fire mix, a boiling ball of magma results.}\\

\TableNote{2}{\textbf{Balance:} The elemental forces come into exact balance, and tranquillity results (for 1d10 minutes). Treat a balanced area as air-dominant, because that trait has no dramatic effects.}\\
}


\textbf{Controlled Limbo:} Controlling a raw area of Limbo is an exercise of the mind. A Wisdom check (DC 16) establishes control within part of a raw area of limbo, and the check can be repeated once per round as a free action. A traveler who has failed checks twice in a row gains a +6 circumstance bonus on subsequent checks. If entering an area of raw Limbo from a controlled or stabilized area, a character can make a control check just prior to stepping into the boil.

If the Wisdom check succeeds, the creature has established control over part of the area and can reshape it as she desires, allowing a desired element or a mixture of elements to become dominant. A favorite among travelers from the Material Plane is a chunk of earth surrounded by a small atmosphere of air.

Consult the table below to determine how large an area a character can control.

\Table{Controlled Limbo}{cXX}{
  \tableheader Wisdom Score
& \tableheader Area of Control
& \tableheader Stabilized Area \\
1--3   & None                    &  \\
4--7   & 30-cm radius            &  \\
8--11  & 1.5-m radius            &  \\
12--15 & 3-m radius              &  \\
16--19 & 4.5-m radius            &  \\
20--23 & 6-m radius              & 1.5-m radius \\
24+    & +1.5 m per 4 Wis points & +1.5 m per 4 Wis points \\
}

\textbf{Stabilized Limbo:} A section of Limbo becomes stabilized if a creature of sufficiently high Wisdom creates it within an area of control. The stabilized area in the center of the area of control retains its traits. It drifts at the whim of Limbo's chaotic currents and, if not protected, is eventually eroded by repeated immersions in the elemental surges. For instance, a 1.5-meter-radius ball of fire could become stable if created by a creature with a Wisdom of 20 or higher. Over the course of several dunks in water, however, it is eroded and finally dissipated. However, industrious creatures can bring bits of stabilized earth together and use them as the foundation for permanent structures, especially if tended by guardians.
