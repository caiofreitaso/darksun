\subsectionA{Gray Waste of Hades}
Hades sits at the nadir of the lower planes, halfway
between two races of fiends each bent on the other's annihilation. Thus, it often sees its gray plains darkened by vast armies of demons battling equally vast armies of devils who neither ask nor give quarter. If any plane defines the nature of true evil, it is the Gray Waste.

In the Gray Waste of Hades, pure undiluted evil acts as a powerful spiritual force that drags all creatures down. Here, even the consuming rage of the Abyss and the devious plotting of the Nine Hells are subjugated to hopelessness. Apathy and despair seep into everything at the pole of evil. Hades slowly kills a visitor's dreams and desires, leaving the withered husk of what used to be a fiery sprit. Spend enough time in Hades, and visitors give up on things that used to matter, eventually giving in to total apathy.

A spiritual poison affects any creature (including outsiders) in Hades that does not possess spell resistance of 10 or more. Creatures without spell resistance 10 must make a Will save (DC 13) every twenty-four hours they spend in Hades.

A failed save deals 1 point of temporary Wisdom damage to the victim. A victim can be drained to a minimum Wisdom of 1 in this fashion. Unlike most ability score damage, Wisdom damage dealt by ``the grays'' does not heal until the victim has left Hades behind. Each point of Wisdom damage dealt in this fashion represents growing apathy, hopelessness, and despair.

This effect is concurrent with the entrapping trait of Hades. Wisdom damage taken from the grays makes it harder to make the weekly saving throws to resist the loss of all hope that the entrapping trait represents.

Hades has three layers called ``glooms.'' Uncaring malevolence that slowly crushes the spirit permeates each gloom.

It has the following traits.
\begin{itemize*}
\item \textbf{Divinely Morphic:} Entities of at least lesser deity status
can alter Hades, though few deities deign to reign in Hades. The Gray Waste has the alterable morphic trait for less powerful creatures; Hades responds normally to spells and physical effort.
\item \textbf{Strongly Evil-Aligned:} Nonevil characters in Hades suffer a –2 penalty on all Charisma-, Wisdom-, and Intelligence-based checks.
\item \textbf{Entrapping:} This is a special trait unique to Hades, although Elysium has a similar entrapping trait. A nonoutsider in Hades experiences increasing apathy and despair while there. Colors become grayer and less vivid, sounds duller, and even the demeanor of companions seems to be more hateful. At the conclusion of every week spent in Hades, any nonoutsider must make a will saving throw (DC 10 + the number of consecutive weeks in Hades). Failure indicates that the individual has fallen entirely under the control of the plane, becoming a petitioner of Hades. \\

Travelers entrapped by the inherent evil of Hades cannot leave the plane of their own volition and have no desire to do so. Memories of any previous life fade into nothingness, and it takes a wish or miracle spell to return such characters to normal.
\end{itemize*}

\subsubsection{Oinos}
The first gloom of Hades is a land of stunted trees, roving fiends, and virulent disease. But more than anything else, it is a plane ravaged by war. This is the central battlefield of the Blood War. Fiends, warrior-slaves, trained beasts, and hired mercenaries gather here to wage horrific battles on an epic scale. These battles despoil the already bleak terrain. The sounds of rending claws, clashing weapons, and screams echo across the entire layer.

\textbf{Khin-Oin the Wasting Tower:} A twenty-mile-high tower, Khin-Oin looks like nothing so much as a freestanding spinal column. Some say that's exactly what it is: the backbone of a deity slain by yugoloths. Khin-Oin plunges as deep into Oinos's gray soil as it ascends into the air, so the tower's sublevels tunnel twenty miles deep.

The Wasting Tower is ruled by an ultraloth prince named Mydianchlarus. In fact, some stories hint that the entire yugoloth race was birthed here, arising in a pit at the absolute bottom of Khin-Oin. None but yugoloths have ever held the tower, despite the constant array of fiendish armies outside.

The rooms and the floors of the tower seem to have no end. Spawning vats, magical laboratories, and meditation chambers can be found here, as can orreries, suites of rooms for yugoloths, and floors that are themselves battlegrounds and drill fields. Mydianchlarus rules from the tower's zenith, and the token of his rulership is the Siege Malicious.

Whoever rules the Wasting Tower is often referred to as the oinoloth. Any creature that can successfully invade the Wasting Tower and make it to the top chamber has the opportunity to claim the title for himself. Claiming the title involves defeating the current ruler, then sitting on the Siege Malicious. The Siege Malicious is a throne of artifact-level power, and as such, it may grant powers over the layer of Oinos.

\textbf{The Siege Malicious:} The Siege Malicious is a major artifact. It is a gargantuan, immovable throne carved from the stone of the Wasting Tower itself. The throne is inlaid with tarnished silver, base copper, and brass. A circular crown of rubies adorns the top of the high seat, which is just large enough to sit a Huge creature. (Many Medium-size creatures would look ridiculous sitting on the Siege Malicious with their legs dangling several feet off the floor.)

\textit{Powers of the Throne:} In order to operate the Siege Malicious, a character sitting on the throne must have defeated the previous oinoloth. If the previous oinoloth yet lives, the sitter suffers 3d6+6 points of permanent Charisma drain, as a consequence of being infected with a particularly virulent strain of the disease called gray wasting. Characters immune to disease don't take damage, but the Siege Malicious seems powerless to them.

If the character sitting on the throne has defeated the previous oinoloth, then the powers of the siege malicious are his. But the throne forever changes those who sit on it.

The Siege Malicious deals 1d4 points of permanent Charisma drain as part of the sitter's skin sloughs off in a rather grotesque manner. This disfigurement is the mark of the oinoloth and may not be magically healed without forsaking the title of oinoloth.

But with the disfigurement comes absolute control of disease on the layer of Oinos. The new oinoloth (whether yugoloth or not) commands the diseases of Oinos, creating, modifying, or nullifying diseases as he sees fit. New or modified diseases could potentially spread beyond the layer of Oinos, but the oinoloth only has this power while in Hades. The oinoloth has power over disease whether sitting in the Siege Malicious or not.

\textit{Creating or Modifying a Disease:} The oinoloth may conceive of or modify a disease at will as a free action (though
coming up with just the right name is an exercise of
intellect that could take longer). The important parameters
for creating or modifying a disease are infection, DC,
incubation, and damage; for more information, see Disease
in Chapter 3 of the D UNGEON M ASTER ’s Guide.
Generally, new or modified diseases must possess a
standard infection type, have a DC no higher than 20, have
an incubation time of no less than one day, and have
damage not greater than 1d8 temporary points of any
ability score damage except Constitution (1d6 if the
disease deals permanent ability drain). Secondary visual
effects of a new disease are up to the oinoloth. Secondary
effects can include deafness, blindness, muteness, and
other sensory deprivations (one per disease), on a second
failed saving throw against the initial disease DC.
Infecting: Once a disease is created or modified, the
oinoloth can set it loose. The oinoloth can infect a living
target within 300 feet as a standard action, and the target
gets no saving throw to avoid infection.

Niflheim
The second gloom of Hades is a layer of gray mists that
constantly twist and swirl among sickly trees and ominous
bluffs. The thin fog limits vision to 100 feet at best, muffles
sound, and eventually saturates everything with
dampness. Niflheim is not as war-ravaged as Oinos, prob-
ably because the mist hinders combat. Many predators
prowl the lands, hidden amid the mist, including fiendish
dire wolves and trolls.
Vision (including darkvision) is limited to 100 feet in
Niflheim, and Listen checks suffer a –4 circumstance
penalty due to the muffling nature of the fog.
Death of Innocence: A small town tucked away in the
misty pines, Death of Innocence is constructed of hewn
pine taken from the surrounding forest.
The town holds more than 5,000 mortals and (nonlarva)
petitioners, though they mostly remain inside their
dwellings, giving the city a vacant feel. Strangely, those
who live behind the protection of the town's walls
sometimes strive to improve their lot and break out of
apathy.
Great wooden gates bar entry to Death of Innocence,
and both the gates and the outer wall bristle with spikes.
Inside, a broad avenue leads to the town's center, where a
gray marble fountain stands. The wood of the buildings
and gates oozes blood, as if sap, confirming the belief that
petitioners are trapped within the wood. Neither the grays
nor the entrapping trait of Hades can penetrate the walls
of Death of Innocence.

Pluton
The third gloom of Hades is a layer of dying willows,
shriveled olive trees, and night-black poplars. It is a realm
where no one wants to be and no one can remember why
they came. Of course, petitioners have no choice in the
matter.
Usually, the Blood War does not reach this lowest
gloom, though some raids have occurred when one side or
the other wished to retrieve the spirit of a fallen mortal
captain who possessed particularly sharp tactical skills.
Underworld: The Underworld is contained within
walls of gray marble that stretch for hundreds of miles and
are visible for thousands of miles beyond that. A single
double gate pierces the marble walls of the realm.
Constructed of beaten bronze, the gates are dented and
scarred by heroes intent on getting past. However, the
gates are also guarded by a terrible fiendish beast, a
Gargantuan three-headed hound made from the
squirming, decaying bodies of hundreds of petitioners.
Beyond the gate, the inside of the realm appears much
like the outside. Blackened trees, stunted bushes, and
wasted ground dominate the landscape. Larvae are
everywhere, writhing in the dust, as are gray, wraithlike
petitioners who are on the verge of being sucked
completely dry of all emotion by the spiritual decay of the
plane. When they lose the last shred of emotion, their
remaining essence becomes one with the gloom of Pluton.
Sometimes, great heroes or desperate lovers from the
Material Plane travel to this layer via a tributary of the
River Styx or portals hidden in great volcanic fissures.
They come to the Underworld because they believe that
they can find the spirit of a friend or loved one and
extricate that spirit from a hopeless eternity. Besides
larvae, faded petitioners, and the occasional foolish
mortals, demons, yugoloths, and devils roam the land,
looking for choice morsels.