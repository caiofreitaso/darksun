\subsection{Magic Traits}
A plane's magic trait describes how magic works on the plane compared to how it works on the Material Plane. Particular locations on a plane (such as those under the direct control of deities) may be pockets where a different magic trait applies.

\textbf{Normal Magic:} This magic trait means that all spells and supernatural abilities function as written. Unless otherwise noted in a description, every plane has the normal magic trait.

\textbf{Wild Magic:} On a plane with the wild magic trait spells and spell-like abilities function in radically different and sometimes dangerous ways. Any spell or spell-like ability used on a wild magic plane has a chance to go awry. The caster must make a level check (DC 15 + the level of the spell or effect) for the magic to function normally. For spell-like abilities, use the level or HD of the creature employing the ability for the caster level check and the level of the spell-like ability to set the DC for the caster level check. Failure on this check means that something strange happens; roll d\% and consult the following table.

\Table{}{lX}{
  \tableheader d\%
& \tableheader Effect\\
01-19  & Spell rebounds on caster with normal effect. If the spell cannot affect the caster, it simply fails.\\
20-23  & A circular pit 4.5 meters wide opens under the caster's feet; it is 3 meters deep per level of the caster.\\
24-27  & The spell fails, but the target or targets of the spell are pelted with a rain of small objects (anything from flowers to rotten fruit), which disappear upon striking. The barrage continues for 1 round. During this time the targets are blinded and must make \skill{Concentration} checks (DC 15 + spell level) to cast spells.\\
28-31  & The spell affects a random target or area. Randomly choose a different target from among those in range of the spell or center the spell at a random place within range of the spell. To generate direction randomly, roll 1d8 and count clockwise around the compass, starting with south. To generate range randomly, roll 3d6. Multiply the result by 1.5 meter for close range spells, 6 meters for medium range spells, or 24 meters for long range spells.\\
32-35  & The spell functions normally, but any material components are not consumed. The spell is not expended from the caster's mind (a spell slot or prepared spell can be used again). An item does not lose charges, and the effect does not count against an item's or spell-like ability's use limit.\\
36-39  & The spell does not function. Instead, everyone (friend or foe) within 9 meters of the caster receives the effect of a heal spell.\\
40-43  & The spell does not function. Instead, a deeper darkness and a silence effect cover a 9-meter radius around the caster for 2d4 rounds.\\
44-47  & The spell does not function. Instead, a reverse gravity effect covers a 9-meter radius around the caster for 1 round.\\
48-51  & The spell functions, but shimmering colors swirl around the caster for 1d4 rounds. Treat this a glitterdust effect with a save DC of 10 + the level of the spell that generated this result.\\
52-59  & Nothing happens. The spell does not function. Any material components are used up. The spell or spell slot is used up, and charges or uses from an item are used up.\\
60-71  & Nothing happens. The spell does not function. Any material components are not consumed. The spell is not expended from the caster's mind (a spell slot or prepared spell can be used again). An item does not lose charges, and the effect does not count against an item's or spell-like ability's use limit.\\
72-98  & The spell functions normally.\\
99-100 & The spell functions strongly. Saving throws against the spell incur a $-2$ penalty. The spell has the maximum possible effect, as if it were cast with the \feat{Maximize Spell} feat. If the spell is already maximized with the feat, there is no further effect.\\
}

\textbf{Impeded Magic:} Particular spells and spell-like abilities are more difficult to cast on planes with this trait, often because the nature of the plane interferes with the spell.

To cast an impeded spell, the caster must make a \skill{Spellcraft} check (DC 20 + the level of the spell). If the check fails, the spell does not function but is still lost as a prepared spell or spell slot. If the check succeeds, the spell functions normally.

\textbf{Enhanced Magic:} Particular spells and spell-like abilities are easier to use or more powerful in effect on planes with this trait than they are on the Material Plane.

Natives of a plane with the enhanced magic trait are aware of which spells and spell-like abilities are enhanced, but planar travelers may have to discover this on their own.

If a spell is enhanced, certain metamagic feats can be applied to it without changing the spell slot required or the casting time. Spellcasters on the plane are considered to have that feat or feats for the purpose of applying them to that spell. Spellcasters native to the plane must gain the feat or feats normally if they want to use them on other planes as well.

\textbf{Limited Magic:} Planes with this trait permit only the use of spells and spell-like abilities that meet particular qualifications.

Magic can be limited to effects from certain schools or subschools, to effects with certain descriptors, or to effects of a certain level (or any combination of these qualities). Spells and spell-like abilities that don't meet the qualifications simply don't work.

\textbf{Dead Magic:} These planes have no magic at all. A plane with the dead magic trait functions in all respects like an antimagic field spell. Divination spells cannot detect subjects within a dead magic plane, nor can a spellcaster use teleport or another spell to move in or out. The only exception to the ``no magic'' rule is permanent planar portals, which still function normally.