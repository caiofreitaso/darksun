\subsection{Physical Traits}
The two most important natural laws set by physical traits are how gravity works and how time passes. Other physical traits pertain to the size and shape of a plane and how easily a plane's nature can be altered.

\textbf{Gravity:} The direction of gravity's pull may be unusual, and it might even change directions within the plane itself.

\textit{Normal Gravity:} Most planes have gravity similar to that of the Material Plane. The usual rules for ability scores, carrying capacity, and encumbrance apply. Unless otherwise noted in a description, it is assumed every plane has the normal gravity trait.

\textit{Heavy Gravity:} The gravity on a plane with this trait is much more intense than on the Material Plane. As a result, Strength- and Dexterity-based skill checks incur a $-2$ circumstance penalty, as do all attack rolls. All item weights are effectively doubled, which might affect a character's speed. Weapon ranges are halved. A character's Strength and Dexterity scores are not affected. Characters who fall on a heavy gravity plane take 1d10 points of damage for each 3 meters fallen, to a maximum of 20d10 points of damage.

\textit{Light Gravity:} The gravity on a plane with this trait is less intense than on the Material Plane. As a result, creatures find that they can lift more, but their movements tend to be ungainly. Characters on a plane with the light gravity trait take a $-2$ circumstance penalty on attack rolls and \skill{Balance}, \skill{Ride}, \skill{Swim}, and \skill{Tumble} checks. All items weigh half as much. Weapon ranges double, and characters gain a +2 circumstance bonus on \skill{Climb} and \skill{Jump} checks.

Strength and Dexterity don't change as a result of light gravity, but what you can do with such scores does change. These advantages apply to travelers from other planes as well as natives.

Falling characters on a light gravity plane take 1d4 points of damage for each 3 meters of the fall (maximum 20d4).

\textit{No Gravity:} Individuals on a plane with this trait merely float in space, unless other resources are available to provide a direction for gravity's pull.

\textit{Objective Directional Gravity:} The strength of gravity on a plane with this trait is the same as on the Material Plane, but the direction is not the traditional ``down'' toward the ground. It may be down toward any solid object, at an angle to the surface of the plane itself, or even upward.

In addition, objective directional gravity may change from place to place. The direction of ``down'' may vary.

\textit{Subjective Directional Gravity:} The strength of gravity on a plane with this trait is the same as on the Material Plane, but each individual chooses the direction of gravity's pull. Such a plane has no gravity for unattended objects and nonsentient creatures. This sort of environment can be very disorienting to the newcomer, but is common on ``weightless'' planes.

Characters on a plane with subjective directional gravity can move normally along a solid surface by imagining ``down'' near their feet. If suspended in midair, a character ``flies'' by merely choosing a ``down'' direction and ``falling'' that way. Under such a procedure, an individual ``falls'' 45 meters in the first round and 90 meters in each succeeding round. Movement is straight-line only. In order to stop, one has to slow one's movement by changing the designated ``down'' direction (again, moving 45 meters in the new direction in the first round and 90 meters per round thereafter).

It takes a DC 16 Wisdom check to set a new direction of gravity as a free action; this check can be made once per round. Any character who fails this Wisdom check in successive rounds receives a +6 bonus on subsequent checks until he or she succeeds.

\textbf{Time:} The rate of time's passage can vary on different planes, though it remains constant within any particular plane. Time is always subjective for the viewer. The same subjectivity applies to various planes. Travelers may discover that they'll pick up or lose time while moving among the planes, but from their point of view, time always passes naturally.

\textit{Normal Time:} This trait describes the way time passes on the Material Plane. One hour on a plane with normal time equals one hour on the Material Plane. Unless otherwise noted in a description, every plane has the normal time trait.

\textit{Timeless:} On planes with this trait, time still passes, but the effects of time are diminished. How the timeless trait can affect certain activities or conditions such as hunger, thirst, aging, the effects of poison, and healing varies from plane to plane.

The danger of a timeless plane is that once one leaves such a plane for one where time flows normally, conditions such as hunger and aging do occur retroactively.

\textit{Flowing Time:} On some planes, time can flow faster or slower. One may travel to another plane, spend a year there, then return to the Material Plane to find that only six seconds have elapsed. Everything on the plane returned to is only a few seconds older. But for that traveler and the items, spells, and effects working on him, that year away was entirely real.

When designating how time works on planes with flowing time, put the Material Plane's flow of time first, followed by the same flow in the other plane.

\textit{Erratic Time:} Some planes have time that slows down and speeds up, so an individual may lose or gain time as he moves between the two planes. The following is provided as an example.

\Table{}{lXX}{ 
  \tableheader d\%
& \tableheader Time on Material Plane
& \tableheader Time on Erratic Time Plane\\
01--10  & 1 day   & 1 round\\
11--40  & 1 day   & 1 hour\\
41--60  & 1 day   & 1 day\\
61--90  & 1 hour  & 1 day\\
91--100 & 1 round & 1 day\\
}

To the denizens of such a plane, time flows naturally and the shift is unnoticed.

If a plane is timeless with respect to magic, any spell cast with a noninstantaneous duration is permanent until dispelled.

\textbf{Shape and Size:} Planes come in a variety of sizes and shapes. Most planes are infinite, or at least so large that they may as well be infinite.

\textit{Infinite:} Planes with this trait go on forever, though they may have finite components within them. Or they may consist of ongoing expanses in two directions, like a map that stretches out infinitely.

\textit{Finite Shape:} A plane with this trait has defined edges or borders. These borders may adjoin other planes or hard, finite borders such as the edge of the world or a great wall. Demiplanes are often finite.

\textit{Self-Contained Shape:} On planes with this trait, the borders wrap in on themselves, depositing the traveler on the other side of the map. A spherical plane is an example of a self-contained, finite plane, but there can be cubes, toruses, and flat planes with magical edges that teleport the traveler to an opposite edge when he crosses them.

Some demiplanes are self-contained.

\textbf{Morphic Traits:} This trait measures how easily the basic nature of a plane can be changed. Some planes are responsive to sentient thought, while others can be manipulated only by extremely powerful creatures. And some planes respond to physical or magical efforts.

\textit{Alterable Morphic:} On a plane with this trait, objects remain where they are (and what they are) unless affected by physical force or magic. You can change the immediate environment as a result of tangible effort.

\textit{Highly Morphic:} On a plane with this trait, features of the plane change so frequently that it's difficult to keep a particular area stable. Such planes may react dramatically to specific spells, sentient thought, or the force of will. Others change for no reason.

\textit{Magically Morphic:} Specific spells can alter the basic material of a plane with this trait.

\textit{Divinely Morphic:} Specific unique beings (deities or similar great powers) have the ability to alter objects, creatures, and the landscape on planes with this trait. Ordinary characters find these planes similar to alterable planes in that they may be affected by spells and physical effort. But the deities may cause these areas to change instantly and dramatically, creating great kingdoms for themselves.

\textit{Static:} These planes are unchanging. Visitors cannot affect living residents of the plane, nor objects that the denizens possess. Any spells that would affect those on the plane have no effect unless the plane's static trait is somehow removed or suppressed. Spells cast before entering a plane with the static trait remain in effect, however.

Even moving an unattended object within a static plane requires a DC 16 Strength check. Particularly heavy objects may be impossible to move.

\textit{Sentient:} These planes are ones that respond to a single thought---that of the plane itself. Travelers would find the plane's landscape changing as a result of what the plane thought of the travelers, either becoming more or less hospitable depending on its reaction.