\subsectionA{Paraelemental Plane of Silt}
The Paraelemental Plane of Silt is a powdery field of gray that stretches for as far as the eye can see, which is a very short distance due to the clouds of silt that blanket the plane.. The silt is uniform throughout the plane, so one has trouble finding his way, or differentiating between this section of Gray Death and that section.

Settlements within the plane are formed from compressed silt into cities. These cities are thriving places where creatures rely on sonar and touch more than any other sense. Travelers need some way to survive, due to the lack of oxygen, the nature of silt, and the effects of Gray Death.

Being one of the more powerful Paraelemental Planes, the Armies of the Silt Lords are fierce. They flow forward like a tide. Anything that they catch is pulled apart and ground away by the thousands of tiny particles sandblasting the meat from a victim's bones.

The Paraelemental Plane of Silt has the following traits.
\begin{itemize*}
\item \textbf{Subjective Directional Gravity:} Inhabitants of the plane determine their own ``down'' direction. Objects not under the motive force of others do not move.
\item \textbf{Enhanced Magic:} Spells and spell-like abilities with the acid descriptor are both maximized and enlarged (as if the \feat{Maximize Spell} and \feat{Enlarge Spell} had been used on them, but the spells don't require higher-level slots). Spells and spell-like abilities that are already maximized or enlarged are unaffected by this benefit.
\item \textbf{Impeded Magic:} Spells and spell-like abilities that use or create water (including spells of the Water domain and spells that summon water elementals or outsiders with the water subtype) are impeded.
\end{itemize*}
