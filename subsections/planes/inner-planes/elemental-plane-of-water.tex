\subsectionA{Elemental Plane of Water}
The Elemental Plane of Water is a sea of green and blue. Lacking a floor or a surface, it is an entirely fluid environment lit by a diffuse glow. It is one of the more hospitable of the Elemental Planes once a traveler figures out how to breathe.

The eternal oceans of this plane vary between ice cold and boiling hot, between saline and fresh. The water is constantly in motion, wracked by currents and tides. The plane's permanent settlements form around bits of coral and other drifting things suspended within this endless liquid. These settlements drift on the tides of the Elemental Plane of Water.

Armies of the Elemental Plane of Water are strange and surprisingly plentiful. Within the waves that are water elementals ride fearsome aquatic creatures, from sharks to kraken. The armies of the plane are aware when an intruder lurks, and if one is injured in battle, the scent of blood will travel for miles, attracting more and more creatures to devour the interloper.

The plane boasts two layers: Quasielemental Mist and Quasielemental Ice.

\ElementalLayers{Mist}{Water}{Ice}
\Figure*{b}{images/planes-1.png}

\subsubsection{Elemental Plane of Water Traits}
\begin{itemize*}
\item \textbf{Subjective Directional Gravity:} Inhabitants of the plane determine their own ``down'' direction. Objects not under the motive force of others do not move.
\item \textbf{Water-Dominant:} Creatures made of fire take 1d10 points of damage each round.
\item \textbf{Enhanced Magic:} Spells and spell-like abilities that use or create water are both extended and enlarged (as if the \feat{Extend Spell} and \feat{Enlarge Spell} metamagic feats had been used on them, but the spells don't require higher-level slots). Spells and spell-like abilities that are already extended or enlarged are unaffected by this benefit.
\item \textbf{Impeded Magic:} Spells and spell-like abilities with the fire descriptor (including spells of the Fire domain) are impeded.
\end{itemize*}


\subsubsection{Quasielemental Mist}
The layer of Quasielemental Mist is where the Elemental Plane of Water is closest to the Positive Energy Plane, infusing the water with energy, until it turns to mist. This mist is not hot or cold, or even a dangerous hazard. Near the Elemental Water, the mist becomes so thick it seems more like bubbly water.

\textbf{Breathing Mist:} The Quasielemental Mist is barely breathable, creatures are \spellref{slow}{slowed} by breathing it. A \spell{worm's breath} spell removes this limitation.

The Quasielemental Mist has the following additional traits.
\begin{itemize*}
\item \textbf{Minor Positive-Dominant:} A creature in Quasielemental Mist gains fast healing 2 as an extraordinary ability.
\end{itemize*}

\subsubsection{Quasielemental Ice}
The layer of Quasielemental Ice is where the Elemental Plane of Water is closest to the Negative Energy Plane, removing all warmth from the water, only ice remaining.

\textbf{Unearthly Cold:} Every round, an unprotected creature in Quasielemental Ice takes 1d6 lethal damage and 1d4 nonlethal damage from the cold (no save allowed). Partially protected characters take damage once per 10 minutes, instead.

The Quasielemental Ice has the following additional traits.
\begin{itemize*}
\item \textbf{Minor Negative-Dominant:} A living creature in Quasielemental Ice take 1d6 points of damage per round. At 0 hit points or lower, they crumble into ash.\\

This trait replaces the Water-Dominant trait.
\end{itemize*}
