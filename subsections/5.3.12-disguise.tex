\Skill{Disguise}{Cha}
\textbf{Check:} Your Disguise check result determines how good the disguise is, and it is opposed by others’ Spot check results. If you don’t draw any attention to yourself, others do not get to make Spot checks. If you come to the attention of people who are suspicious (such as a guard who is watching commoners walking through a city gate), it can be assumed that such observers are taking 10 on their Spot checks.

You get only one Disguise check per use of the skill, even if several people are making Spot checks against it. The Disguise check is made secretly, so that you can’t be sure how good the result is.

The effectiveness of your disguise depends in part on how much you’re attempting to change your appearance.

\Table{Disguise Check Modifiers}{X r{3cm}}{
\tableheader Disguise & \tableheader Disguise Check Modifier\\
% These modifiers are cumulative; use any that apply.
% Per step of difference between your actual age category and your disguised age category. The steps are: young (younger than adulthood), adulthood, middle age, old, and venerable.
Minor details only & +5\\
Disguised as different gender1 & $-2$\\
Disguised as different race1 & $-2$\\
Disguised as different age category1 & $-22$
}

If you are impersonating a particular individual, those who know what that person looks like get a bonus on their Spot checks according to the table below. Furthermore, they are automatically considered to be suspicious of you, so opposed checks are always called for.

\Table{}{l R}{
\tableheader Familiarity & \tableheader Viewer’s Spot Check Bonus\\
Recognizes on sight & +4\\
Friends or associates & +6\\
Close friends & +8\\
Intimate & +10
}

Usually, an individual makes a Spot check to see through your disguise immediately upon meeting you and each hour thereafter. If you casually meet many different creatures, each for a short time, check once per day or hour, using an average Spot modifier for the group.

\textbf{Action:} Creating a disguise requires 1d3 $\times$ 10 minutes of work.

\textbf{Try Again:} Yes. You may try to redo a failed disguise, but once others know that a disguise was attempted, they’ll be more suspicious.

\textbf{Special:} Magic that alters your form, such as \emph{alter self}, \emph{disguise self}, \emph{polymorph}, or \emph{shapechange}, grants you a +10 bonus on Disguise checks (see the individual spell descriptions). You must succeed on a Disguise check with a +10 bonus to duplicate the appearance of a specific individual using the \emph{veil} spell. Divination magic that allows people to see through illusions (such as \emph{true seeing}) does not penetrate a mundane disguise, but it can negate the magical component of a magically enhanced one.

You must make a Disguise check when you cast a \emph{simulacrum} spell to determine how good the likeness is.

If you have the Deceitful feat, you get a +2 bonus on Disguise checks.

\textbf{Synergy:} If you have 5 or more ranks in Bluff, you get a +2 bonus on Disguise checks when you know that you’re being observed and you try to act in character.