\section{Drugs}

\BigTablePair{Drugs}{ll ccXlr{11mm}}{
\tableheader Name & \tableheader Fort DC & \tableheader Addiction & \tableheader Slaking Period & \tableheader Spiral Damage & \tableheader Craft DC & \tableheader Cost\\

\multicolumn{7}{l}{\textit{Contact}}\\
N'ko shard & DC 15 & DC 15 & 2 days & 1d4 Str, 1d4 Dex, 1d4 Con & DC 18 & 20 cp\\
\multicolumn{7}{l}{\textit{Inhaled}}\\
Bellaweed & DC 12 & DC 25 & 1 day & 1d4 Dex, 1d2 Int, 1d2 Wis, 1d6 Cha & DC 14 & 2 bits\\
Shrinebush fumes & DC 25 & DC 17 & 3 days & 1d6 Wis, 1d6 Cha & DC 32 & 3,500 cp\\
\multicolumn{7}{l}{\textit{Ingested}}\\
Weomre sap & DC 17 & DC 15 & 1 day & 1d3 Dex, 1d3 Con, 1d6 Wis & & 75 cp\\
Esperweed & DC 15 & DC 15 & 1 day & 1d6 Int, 1d6 Wis & & 250 cp\\
Esperweed, refined & DC 22 & DC 20 & 7 days & 1d4 Str, 1d4 Dex, 1d4 Con, 1d8 Int, 1d8
Wis & DC 25$^\star$ & 1,500 cp\\
}

Drugs are a special type of poison, one in which the user or victim afterwards craves more of the agent that influenced him.

Drugs are affected by magic and psionic powers the same way poisons are, as such \psionic{antidote simulation}, \spell{delay poison}, \spell{detect poison}, and \spell{neutralize poison} affect drugs as if they were poisons.

Using a drug causes secondary effects, but also provides effects that benefits the character, such as a sense of euphoria or bonuses to ability scores. These are the primary effects. Characters can attempt to use a drug without incuring its secondary effects, so as to only benefit from its use, by making a saving throw against the secondary effects DC. Characters that are addicted to a drug must add +5 to this DC. The primary effects always occur wether the saving throw against the secondary effects is successful or not; side effects also cannot be avoided. An overdose occurs when the save to avoid the secondary effects is failed by 10 or more.

A character affected by a drug, whether intentionally or not, must succeed at a save against its addiction DC or become addicted to the drug. An addicted character that cannot obtain more of the drug by the end of the slaking period takes the given spiral damage indicated for that drug. After taking the spiral damage, another save against the addiction DC is made. A successful save indicates the character has broken off his addiction to the drug, while a failed save means the character must again use the drug within the slaking time or suffer more spiral damage. Using a given drug more than once during its given slaking period causes the character to makes a new save to resist addiction, but with a +5 to the DC for each additional use.

Drugs are crafted using the \skill{Craft} (poisonmaking) skill, just like poisons. Some drugs require an additional \skill{Psicraft} or \skill{Spellcraft} check to create, due to their magic or psionic nature. A successful \skill{Craft} (poisonmaking) check allows you to create one dose, plus one additional dose for every four points by which you exceed the \skill{Craft} check.

The purchase and possession of most drugs is illegal in the city-states, but can be obtained in bard quarters and elven markets.

\subsection{Drugs Descriptions}
Here is the format for drug entries, as used below.

\textbf{Type:} The drug’s method of delivery (contact, ingested or inhaled)

\textit{Contact:} Contact drugs are very rare; the agent is placed directly on the skin for a certain amount of time. Contact drugs cannot be utilized in a combat situation. A creature with a natural armor bonus higher than +4 is unaffected by exposure to a contact drug as the agent cannot be absorbed through their skin.

\textit{Ingested:} Ingested drugs are the norm; the agent is masticated, eaten, drank, or otherwise left in the mouth to absorb through the sensitive flesh. Ingested drugs cannot be utilized in a combat situation. A bad-intentioned person can administer a dose to an unconscious creature or attempt to dupe someone into ingesting a drug mixed with food or drink.

\textit{Inhaled:} Inhaled drugs usually come in the form of smokesticks or small balls of gummy or crystalized material which are burned and whose smoke is then inhaled. Inhaled drugs cannot be utilized in a combat situation. In indoor conditions, one dose of an inhaled drug creates smoke that spreads to fill the volume of a 3-meter cube. If more than five doses are burned simultaneously within an enclosed space, even creatures that are not actively using the drug are affected. Each creature within the area must make a saving throw as if they were using the drug, but at a reduced DC, both for resisting secondary effects and addiction (substract 5 from the DC). (Holding one’s breath is ineffective against inhaled drugs as they can also affect the nasal membranes, tear ducts, and other parts of the body.)

\textbf{Fortitude save DC: }

\textbf{Side Effects:} Adverse effects from using the drug. These effects occur immediately upon taking the drug and cannot be avoided.

\textbf{Primary Effects:} The effects that are beneficial and often dsired by the character exposed to the drug.

\textbf{Secondary Effects:} These effects vary greatly from one drug to another, and can be beneficial or detrimental (or both) to the character exposed to the drug. These effects occur if the character fails his save. (Ability damage is temporary unless otherwise noted.)

\textbf{Overdose:} The effects of failing a save to avoid the drug’s secondary effects by 10 or more. (Ability damage is temporary unless otherwise noted.)

\textbf{Slaking Period:} The period of time that follows the taking of a drug, during which a character who takes another dose sees its save DC to avoid addiction increase by +5 for each additional dose taken. For example, a user that has just taken a dose of a drug with a slaking period of ten days must wait until the eleventh day before taking another dose. If he takes another dose during the slacking period he must save against the DC to avoid addiction with an increase of +5 to the DC. An additional dose after that adds +5 to the already modified DC, and so on. Each time a character takes a dose of a drug, the time that has passed towards reaching the end of the slacking period is reset to zero.

\textbf{Spiral Damage:} The effects upon an addicted character who reaches the end of a drug’s slaking period. (Ability damage is temporary unless otherwise noted.)

\textbf{Craft:} The \skill{Craft} (poisonmaking) DC to create one dose of the drug. Some drugs require an additional \skill{Psicraft} or \skill{Spellcraft} check in order to create.

\textbf{Price:} The cost of one dose of the drug. It is not possible to use a drug in any quantity smaller than one dose.