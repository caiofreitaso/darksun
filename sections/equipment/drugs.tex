\section{Drugs}

\BigTablePair{Drugs}{lcccXlr{11mm}}{
& \multicolumn{2}{c}{\tableheader Fortitude Save DC}\\
\cmidrule[0.5pt]{2-3}
\tableheader Name & \tableheader Secondary DC & \tableheader Addiction DC & \tableheader Slaking Period & \tableheader Spiral Damage & \tableheader \skill{Craft} DC & \tableheader Cost\\

\multicolumn{7}{l}{\TableSubheader{Contact}}\\
N'ko shard & DC 15 & DC 15 & 2 days & 1d4 Str, 1d4 Dex, 1d4 Con & DC 18 & 20 cp\\
\multicolumn{7}{l}{\TableSubheader{Inhaled}}\\
Bellaweed & DC 12 & DC 25 & 1 day & 1d4 Dex, 1d2 Int, 1d2 Wis, 1d6 Cha & DC 14 & 2 bits\\
Shrinebush fumes & DC 25 & DC 17 & 3 days & 1d6 Wis, 1d6 Cha & DC 32 & 3,500 cp\\
\multicolumn{7}{l}{\TableSubheader{Ingested}}\\
Weomre sap & DC 17 & DC 15 & 1 day & 1d3 Dex, 1d3 Con, 1d6 Wis & & 75 cp\\
Esperweed & DC 15 & DC 15 & 1 day & 1d6 Int, 1d6 Wis & & 250 cp\\
Esperweed, refined & DC 22 & DC 20 & 7 days & 1d4 Str, 1d4 Dex, 1d4 Con, 1d8 Int, 1d8
Wis & DC 25 $\star$ & 1,500 cp\\
\rowcolor{white}
\multicolumn{7}{l}{$\star$ Additionally, you must make a \skill{Psicraft} check DC 15.}
}

Drugs are a special type of poison, one in which the user or victim afterwards craves more of the agent that influenced him.

Drugs are affected by magic and psionic powers the same way poisons are, as such \psionic{antidote simulation}, \spell{delay poison}, \spell{detect poison}, and \spell{neutralize poison} affect drugs as if they were poisons.

Using a drug causes secondary effects, but also provides effects that benefits the character, such as a sense of euphoria or bonuses to ability scores. These are the primary effects. Characters can attempt to use a drug without incuring its secondary effects, so as to only benefit from its use, by making a saving throw against the secondary effects DC. Characters that are addicted to a drug must add +5 to this DC. The primary effects always occur wether the saving throw against the secondary effects is successful or not; side effects also cannot be avoided. An overdose occurs when the save to avoid the secondary effects is failed by 10 or more.

A character affected by a drug, whether intentionally or not, must succeed at a save against its addiction DC or become addicted to the drug. An addicted character that cannot obtain more of the drug by the end of the slaking period takes the given spiral damage indicated for that drug. After taking the spiral damage, another save against the addiction DC is made. A successful save indicates the character has broken off his addiction to the drug, while a failed save means the character must again use the drug within the slaking time or suffer more spiral damage. Using a given drug more than once during its given slaking period causes the character to makes a new save to resist addiction, but with a +5 to the DC for each additional use.

Drugs are crafted using the \skill{Craft} (poisonmaking) skill, just like poisons. Some drugs require an additional \skill{Psicraft} or \skill{Spellcraft} check to create, due to their magic or psionic nature. A successful \skill{Craft} (poisonmaking) check allows you to create one dose, plus one additional dose for every four points by which you exceed the \skill{Craft} check.

The purchase and possession of most drugs is illegal in the city-states, but can be obtained in bard quarters and elven markets.

Here is the format for the quick drug references (given as column headings on \tabref{Drugs}).

\textbf{Type}: The drug's method of delivery (contact, ingested or inhaled)

\textit{Contact}: Contact drugs are very rare; the agent is placed directly on the skin for a certain amount of time. Contact drugs cannot be utilized in a combat situation. A creature with a natural armor bonus higher than +4 is unaffected by exposure to a contact drug as the agent cannot be absorbed through their skin.

\textit{Ingested}: Ingested drugs are the norm; the agent is masticated, eaten, drank, or otherwise left in the mouth to absorb through the sensitive flesh. Ingested drugs cannot be utilized in a combat situation. A bad-intentioned person can administer a dose to an unconscious creature or attempt to dupe someone into ingesting a drug mixed with food or drink.

\textit{Inhaled}: Inhaled drugs usually come in the form of smokesticks or small balls of gummy or crystalized material which are burned and whose smoke is then inhaled. Inhaled drugs cannot be utilized in a combat situation. In indoor conditions, one dose of an inhaled drug creates smoke that spreads to fill the volume of a 3-meter cube. If more than five doses are burned simultaneously within an enclosed space, even creatures that are not actively using the drug are affected. Each creature within the area must make a saving throw as if they were using the drug, but at a reduced DC, both for resisting secondary effects and addiction (substract 5 from the DC). (Holding one's breath is ineffective against inhaled drugs as they can also affect the nasal membranes, tear ducts, and other parts of the body.)

\textbf{Fortitude save DC}: The character must make both saving throws each time they use a drug.

\textit{Secondary save DC}: Fortitude save to avoid secondary effects. 

\textit{Addiction save DC}: Fortitude save to avoid addiction.

\textbf{Slaking Period}: The period of time that follows the taking of a drug, during which a character who takes another dose sees its save DC to avoid addiction increase by +5 for each additional dose taken. For example, a user that has just taken a dose of a drug with a slaking period of ten days must wait until the eleventh day before taking another dose. If he takes another dose during the slacking period he must save against the DC to avoid addiction with an increase of +5 to the DC. An additional dose after that adds +5 to the already modified DC, and so on. Each time a character takes a dose of a drug, the time that has passed towards reaching the end of the slacking period is reset to zero.

\textbf{Spiral Damage}: The effects upon an addicted character who reaches the end of a drug's slaking period. (Ability damage is temporary unless otherwise noted.)

\textbf{Craft}: The \skill{Craft} (poisonmaking) DC to create one dose of the drug. Some drugs require an additional \skill{Psicraft} or \skill{Spellcraft} check in order to create.

\textbf{Price}: The cost of one dose of the drug. It is not possible to use a drug in any quantity smaller than one dose.

Drugs have effects and potential overdose that are more detailed below. Here is the format for the detailed drug entries, as used below.

\textbf{Side Effects}: Adverse effects from using the drug. These effects occur immediately upon taking the drug and cannot be avoided.

\textbf{Primary Effects}: The effects that are beneficial and often dsired by the character exposed to the drug.

\textbf{Secondary Effects}: These effects vary greatly from one drug to another, and can be beneficial or detrimental (or both) to the character exposed to the drug. These effects occur if the character fails his save. (Ability damage is temporary unless otherwise noted.)

\textbf{Overdose}: The effects of failing a save to avoid the drug's secondary effects by 10 or more. (Ability damage is temporary unless otherwise noted.)


\subsection{Drugs Descriptions}

\textbf{Bellaweed}: This drug is made from the leaves of the small bellaweed plant, a thriving desert vine that sports coarse, dark-green leaves and large, bell-shaped white blossoms. The leaves are dried and finely chopped, then mixed with the pulverized seeds of the plant. The mixture is then allowed to macerate in wine for a time before being dried, producing a paste that can be smoked, giving off a sickly odor.

Bellaweed is extremely addictive. Thrill seekers first try bellaweed to feel the sensation of well-being and the surealistic visions inhaling the smoke provides. But as more and more bellaweed is used, the taste for everything else, including all that was important to the user before, begins to lose ground before the need for the drug. Most of the people addicted to bellaweed finish as docile slaves, unable to shrug off their addiction, slowly losing their will, doing only simple and basic tasks while under the effect of the bellaweed their masters provide.

\textit{Side Effects}: The user of bellaweed is considered helpless for the duration of the primary effects.

\textit{Primary Effects}: After 1d6 rounds, the user enters a state of euphoria and begins to experience pleasing visions and hallucinations for the next 1d4 hours. The user is considered immune to fear effects and gaze attacks during that time.

\textit{Secondary Effects}: After the duration of the primary effects, real life---compared to the vision provided by the bellaweed---seems dull and flat. The user takes 1d2 points of permanent Wisdom and Intelligence damage until these scores get to 3, after which they don't go down more from using bellaweed.

\textit{Overdose}: The user becomes nauseated for 1d4 hours.

\textbf{Esperweed}: The esperweed is a rare plant that grows in the Forest Ridge and on some of the mudflats surrounding the Sea of Silt. The plant, which grows up to one meter high, has a brownish-green stalk that turns bright green near the flowers and leaves at its top. The flowers each sport six reddish-orange petals surrounding a bright red stamen. The root of the esperweed is a potent psionic stimulant that must be used within a week of being picked, unless somehow magically kept fresh (as through the \spell{nurturing seeds} spell). The roots of one plant give 2d4 doses.

\textit{Side Effects}: The user becomes nauseated for 1 minute.

\textit{Primary Effects}: The user begins to regain one power point per minute, up to 10\% above his normal power point limit (round down). The user also gains the empathy power. These effects last for one hour, after which he loses all remaining power points above his normal limit. These extra power points, if any, are spent first when manifesting powers (much like temporary hit points).

\textit{Secondary Effects}: The user takes 1d4 points of Strength damage.

\textit{Overdose}: The user takes a $-4$ penalty to Will saves for the next 2d6 hours.

\textbf{Esperweed, Refined}: This drug is obtained from the distillation of common esperweed roots. It takes 4 doses of normal esperweed to create one dose of refined esperweed.

\textit{Side Effects}: The user becomes nauseated for 2d10 minutes.

\textit{Primary Effects}: After the duration of the side effects, for the next 1d4+5 minutes, the user gains a +5 bonus to his effective manifester level. He also begins to regain one power point per round, up to 20\% above his normal power point limit (rounded down). After the duration of the primary effects end, he loses all remaining power points above his normal limit. These extra power points, if any, are spent first when manifesting powers (much like temporary hit points).

\textit{Secondary Effects}: The user takes 1d4$-1$ points of Strength, Dexterity and Constitution damage.

\textit{Overdose}: The user gains a $-2$ effective level penalty to every psionic classes he possess for 24 hours. Whenever the user's level is used in a die roll or calculation, reduce it by $-2$ for each psionic class he possesses. At the end of the 24 hours, the user must make a successful Will save (DC 25) or lose one level from any psionic class they possess (player's choice). This has the same effect as if the character suffered from a permanent level drain caused by an energy drain attack, except that the class level lost may not be the most recent level the character acquired.

\textbf{N'ko Shard}: This drug is obtained from the crystallization of the pale blue leaves of the n'ko'ma plant. This small plant is found throughout the Tyr Region, growing in bushes near oasises and water holes. The shard must be held in contact with the skin, often by means of wrappings, for 10 minutes before its effects can be felt. The crystal shard leaves a permanent pale-blue stain in the shape of the crystal where it was in contact with the skin; recurrent users have a great deal of skin surface tinted a telltale blue color. This drug is mostly used in the gladiatorial arena and by nomadic warriors.

A user of a n'ko shard gets an increase in his reflexes and an inordinate amount of split-second perception---a kind of supernatural sixth sense---resulting in heightened combat senses, at the cost of being less attuned to details surrounding him, resulting in a visual haze somewhat akin to that experienced by an individual with vision impairment.

\textit{Side Effects}: The user takes a $-4$ penalty to Spot and Search checks for the next 12 hours.

\textit{Primary Effects}: For the next 1d2 hours, the user gains a +1 bonus to melee attack rolls and a +1 dodge bonus to AC and Reflex saves.

\textit{Secondary Effects}: The user's vision becomes blurred, so all opponents are considered to have concealment (20\% miss chance) relative to the n'ko shard's user.

\textit{Overdose}: The user is blinded for 2d4 hours.

\textbf{Shrinebush Fumes}: This dangerous drug was used by Green Age religions in cathartic ceremonies that sometimes ended in the death of its participants, and was reserved for the grand priests of the time. The drug is made from the dried leaves of a mature shrinebush plant, which grows on graves and is said to have a connection to the Gray. The plant became more common with the advent of mass graves during the Cleansing Wars, but its use in the making of this drug was nearly forgotten during those conflicted times. To be used to contact a spirit---the most common reason---the drug must be inhaled within 3 meters of a corpse, or piece of corpse, whose spirit the user wishes to contact.

\textit{Side Effects}: The user falls unconscious for the duration of the primary effects.

\textit{Primary Effects}: For 1d10 minutes, the drug's user creates a temporary mindscape that it uses to attempt contact with spirits dwelling in the Gray, in a manner similar to the \spell{speak with dead} spell, but with only a 25\% chance of successfully contacting the targeted spirit. If successful, the user can ask up to one question per minute to the spirit. If unsuccessful, it is the other spirits near the mindscape that answer the questions and communicate with the user, lying and creating answers as their alignment and agendas---if any---allow.

A serious danger to creating such a mindscape is that any undead outside the mindscape can communicate with the user and will often try to lure him into the void. Due to imperfections in the manifestation of the mindscape such undead can enter its boundaries as a free action, and those that have the possession ability can attempt to possess the user's physical body.

\textit{Secondary Effects}: The mindscape is not as perfect as one created by a spell or power and lets some of the Gray energies pass through it, making such mindscapes rather unsafe to use. The user's life force is slowly sucked out into the Gray; upon failing his save he takes 1 negative energy level of damage, which is permanent if he fails a second save 24 hours later.

\textit{Overdose}: The user enters an unconscious state and produces a mindscape that is even more permeable than usual to the energies of the Gray. He is considered as being nearly in the void and he must make a Will save (DC 20 + the number of previous saves) each minute for 1d10 minutes or perish. Users cannot use the drug's primary effects while in an overdose-induced mindscape.