\section{Wealth and Money}
\subsection{Coins}
All prices in {\tableheader Dark Sun} are given in terms of ceramic pieces, the most common coin. Ceramics are made from glazed clay and baked in batches once a year in a secure process supervised by the high templar that supervises the city's treasury. Bits are literally one-tenth parts of a ceramic piece---the ceramic pieces break easily into ten bits. Some cities' ceramic pieces have small holes that can be threaded onto a bracelet or necklace. The lowest unit of Athasian trade is the lead bead (bd).

Each of the city-states of the Tablelands produces its own currency. All cities use ceramic pieces as the most common coin, but also mint silver coins and, in some cases, rare and highly prized gold coins.

\Table[\scriptsize]{Currency Conversions}{l C C C C}{
& \multicolumn{4}{c}{\tableheader Exchange Value}\\
& \tableheader bd & \tableheader bit & \tableheader cp & \tableheader sp\\
Lead bead (bd) & 1 & 1/10 & 1/100 & 1/1,000\\
Ceramic bit (bit) & 10 & 1 & 1/10 & 1/100\\
Ceramic piece (cp) & 100 & 10 & 1 & 1/10\\
Silver piece (sp) & 1,000 & 100 & 10 & 1\\
Gold piece (gp) & 10,000 & 1000 & 100 & 10\\
}

\subsubsection{Moneychangers}
Adventurers that travel between cities will need to change their currency for local currency at each city they visit. With a couple of exceptions, the city-states have moneychangers available for incoming visitors. Located near the city gates and in large market places, moneychangers denote their business by hanging a large purple banner from their shop. The banners are always purple, but the moneychangers in each city-state display a different emblem on the banner, based on their city's standard.

Moneychangers charge each customer a fee to change coins. The fees differ by city and are summarized on \tabref{Moneychangers}.

\Table{Moneychangers}{C C}{
\tableheader City & \tableheader Exchange Rate\\
Balic & 6\%\\
Draj & 8\%\\
Gulg & 10\%\\
Kurn & 16\%\\
Nibenay & 14\%\\
Raam & 12\%\\
Tyr & 12\%\\
Urik & 9\%\\
}

These fees are averages and may vary slightly. There are of course many unscrupulous money merchants who will charge as much as they can get away with. Moneychangers in Kurn are rare but a couple do exist. Since metal coins from any city-state are readily accepted by local merchants and no corresponding Kurnish coins exist there is little need to exchange such coins. There are, however, a few moneychangers willing to exchange ceramic pieces.

There are two cities that do not have moneychangers. Visitors to Celik have no need of a moneychangers, as merchants in that city take coins of all types. Nor are there any moneychangers in Eldaarich. Since Eldaarich has been shut off from the rest of the land for so long, visitors needing to exchange money have been nonexistent, so no moneychangers have set up business.

\subsection{Trade}
In general, the Athasian economy in the cities is relatively stable thanks to the Merchant Houses. Under normal conditions, supply is ample thanks to the caravans traveling back and forth between the cities. However, for smaller communities and trade outposts the price situation on certain goods can sway drastically. A raider attack or sandstorm can result in lack of necessities such as food and water, for which people will pay almost any amount of coin. Coins are not the only means of exchange. Barter and trade in commodities is widespread.

Dune traders commonly exchange trade goods without using currency, instead relying on a basic bartering system.


\Table{Trade Goods}{X r{1.3cm}}{
\tableheader Item & \tableheader Cost \\
One kilogram of salt & 4 bits \\
One kilogram of grain or faro & 6 bits \\
One kilogram of lead & 1 cp \\
One kilogram of nuts, or one kilogram of kank nectar & 2 cp \\
One square meter of cotton (cloth) & 4 cp \\
One kilogram of obsidian, or one square meter of silk, or one metric ton of water, or one erdlu & 10 cp \\
One herding kank, or one aprig & 50 cp \\
One kilogram of copper, or one male carru, or one inix & 100 cp \\
 One kilogram of iron, or one mekillot & 200 cp \\
One female carru & 300 cp \\
One kilogram of silver & 1,000 cp \\
One kilogram of gold & 10,000 cp \\
}

\subsection{Selling Loot}
In general, a character can sell something for half its listed price.

Trade goods are the exception to the half-price rule. A trade good, in this sense, is a valuable good that can be easily exchanged almost as if it were cash itself.