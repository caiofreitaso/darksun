\section{Special Materials}
A number of exotic materials can be found in the wilds and underground of the Tablelands and beyond, in addition to the more mundane ones. These substances have innate special properties. If you make a suit of armor or weapon out of more than one special material, you get the benefit of only the most prevalent material. However, you can build a double weapon with each head made of a different special material.

\subsection{Special Weapons Materials}
Each of the special materials described below has a definite game effect. Some creatures have damage reduction based on their creature type or core concept. Some are resistant to all but a special type of damage, such as that dealt by evil-aligned weapons or bludgeoning weapons. Others are vulnerable to weapons of a particular material. Characters may choose to carry several different types of weapons, depending upon the campaign and types of creatures they most commonly encounter.

\Table{Special Materials Cost Modifiers}{XR}{
\tableheader Type of Item & \tableheader Item Cost Modifier\\
\TableSubheader{Adamantine (dwarven steel)} &\\
Ammunition & +60 gp\\
Light armor & +5,000 gp\\
Medium armor & +10,000 gp\\
Heavy armor & +15,000 gp\\
Weapon & +3,000 gp\\
Shield & +2,000 gp\\
\TableSubheader{Agafari} &\\
Any wooden item & weight in kg $\times$ 20 cp\\
\TableSubheader{Dasl} &\\
Any weapon & $\times$ 10\\
\TableSubheader{Drakehide} &\\
Any armor or shield & masterwork price $\times$ 2\\
\TableSubheader{Drake Ivory} &\\
Any weapon & $\times$ 2\\
\TableSubheader{Iron} &\\
Any armor, shield, or weapon & $\times$ 100\\
\TableSubheader{Silver, alchemical} &\\
Ammunition & +2 cp\\
Light weapon & +20 cp\\
One-handed weapon, or one head of a double weapon & +90 cp\\
Two-handed weapon, or both heads of a double weapon & +180 cp\\
}

\textbf{Adamantine}: While weapons and armor made from adamantine, also called ``dwarven steel'', can be found on Athas, they are quite rare. Nearly all items made from adamantine are relics from a long past age, scavenged from ruins by elves and treasure hunters. Raw adamantine cannot be bought on the market, and weapons and armor constructed from adamantine are considered priceless relics.

This ultrahard metal adds to the quality of a weapon or suit of armor. Weapons fashioned from adamantine have a natural ability to bypass hardness when sundering weapons or attacking objects, ignoring hardness less than 20. Armor made from adamantine grants its wearer damage reduction of 1/-- if it's light armor, 2/-- if it's medium armor, and 3/-- if it's heavy armor. Adamantine is so costly that weapons and armor made from it are always of masterwork quality; the masterwork cost is included in the prices. Thus, adamantine weapons and ammunition have a +1 enhancement bonus on attack rolls, and the armor check penalty of adamantine armor is lessened by 1 compared to ordinary armor of its type. Items without metal parts cannot be made from adamantine. An arrow could be made of adamantine, but a quarterstaff could not.

Only weapons, armor, and shields normally made of metal can be fashioned from adamantine. Weapons, armor and shields normally made of steel that are made of adamantine have one-third more hit points than normal. Adamantine has 40 hit points per inch of thickness and hardness 20.

\textbf{Agafari}: This rare magic wood is as hard as normal wood but very light. Any wooden or mostly wooden item (such as a bow, an arrow, or a spear) made from agadari is considered a masterwork item and weighs only half as much as a normal wooden item of that type. Items not normally made of wood or only partially of wood (such as a battleaxe or a mace) either cannot be made from agadari or do not gain any special benefit from being made of agadari. The armor check penalty of a agadari shield is lessened by 2 compared to an ordinary shield of its type. To determine the price of a agadari item, use the original weight but add 20 cp per kilogram to the price of a masterwork version of that item.

For weapons affected by the inferior material rule, agafari is considered an inferior material.

Agafari has 10 hit points per inch of thickness and hardness 5.

\textbf{Dasl}: Dasl is a special kind of crystalline material created by thri-kreen and often used to manufacture their weapons. An item made from dasl is treated as if it was made from iron and is not considered to be made from inferior materials. However, for purposes of harming creatures with damage reduction, a dasl weapon is not treated as being made from metal. An item made from dasl costs ten times what it normally would. Thus, a dasl chatkcha costs 200 cp instead of the 20 cp a stone or bone chatkcha would.

Dasl has a 15 hit points per inch of thickness and hardness 7.

\textbf{Drakehide}: The hide of a drake is of such value that most sorcerer-kings forbid their sale. They have instructed templars to confiscate any such items that appear in the market in the name of their sorcerer-king. Because drakes are so rare it is easy for templars to claim the item was stolen from the sorcerer-king and have the seller put to death. Elves, of course, defy these edicts at every turn, and make a fair profit selling drake materials while just one step ahead of their templar pursuers.

Armorsmiths can work with the hides of drakes to produce armor or shields of masterwork quality. One drake produces enough hide for four suits of hide armor, three suits of banded mail, two suits of half-plate, or one suit of full plate for a Medium creature. Enough hide is available to produce a small or large masterwork shield in addition to the armor.

Because dragonhide armor isn't made of metal, druids can wear it without penalty.

Drakehide armor costs double what masterwork armor of that type ordinarily costs, but it takes no longer to make than ordinary armor of that type.

Drakehide has 10 hit points per inch of thickness and hardness 10.

\textbf{Drake Ivory}: Drake ivory is extraordinarily strong and easy to work compared to the bone that most Athasian weaponsmiths use. Since it can only be obtained from the claws and teeth of deadly drakes, it is rare, expensive, and forbidden (see drakehide, above).

Weapons made from drake ivory cost an additional 2,000 cp to enchant. Weapons crafted from drake ivory cost twice what they normally would. A double weapon that is only half crafted using drake ivory increases its cost by 50\%.

Drake ivory has 30 hit points per inch of thickness and hardness 10.

% Giant Hair: Giant hair is very strong and frequently woven together to form a very strong cord. While sometimes used in armor, it is most frequently used to create rope. This rope costs 50 cp for 15 meters, has a hardness of 5 and 2 hp per inch of thickness

\textbf{Iron}: Iron (and most other metals) is rare on Athas, but weapons and armor made of iron can still be found. In all of the city-states, there are at least a few craftsmen that are able to work iron. The only fresh source of iron is the mines in the city-state of Tyr. Many of the iron weapons and armor available for sale have been scavenged from ruins. Weapons made of iron (including iron-based compounds like steel) can bypass the damage reduction possessed by some Athasian monsters. Iron-based items cost 100 times what they normally would.

Iron has 30 hit points per inch of thickness and hardness 10.

\textbf{Silver, Alchemical}: The process of binding silver to weapons has been greatly refined on Athas. Very little silver is actually needed in the process, and it can be bound to weapons crafted from dasl, obsidian and bone, as well as those made from iron.

A complex process involving metallurgy and alchemy can bond silver to a weapon made of steel so that it bypasses the damage reduction of creatures such as lycanthropes.

On a successful attack with a silvered weapon, the wielder takes a $-1$ penalty on the damage roll (with the usual minimum of 1 point of damage). The alchemical silvering process can’t be applied to nonmetal items, and it doesn’t work on rare metals such as adamantine, cold iron, and mithral.

Alchemical silver has 10 hit points per inch of thickness and hardness 8.
