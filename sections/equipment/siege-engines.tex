\section{Siege Engines}
Siege engines are large weapons, temporary structures, or pieces of equipment traditionally used in besieging a castle or fortress.

\Table{Siege Engines}{lRCCCC}{
\tableheader Item & \tableheader Cost & \tableheader Damage & \tableheader Critical & \tableheader Range Increment & \tableheader Typical Crew\\
Catapult, heavy & 80 sp & 6d6 & & 60 m\footnotemark[1] & 4 \\
Catapult, light & 55 sp & 4d6 & & 45 m\footnotemark[1] & 2 \\
Ballista & 50 sp & 3d8 & 19--20 & 36 m & 1 \\
Ram & 100 sp & 3d8\footnotemark[2] & & & 10 \\
Siege tower & 200 sp & & & & 20 \\
\TableNote{6}{1 30 m minimum.}\\
\TableNote{6}{2 See description for special rules.}\\
}

\BigTablePair{Catapult Attack Modifiers}{XR}{
\tableheader Condition & \tableheader Modifier\\
No line of sight to target square & $-6$\\
Successive shots (crew can see where most recent misses landed) & Cumulative +2 per previous miss (maximum +10)\\
Successive shots (crew can't see where most recent misses landed, but observer is providing feedback) & Cumulative +1 per previous miss (maximum +5)\\
}

\textbf{Catapult, Heavy}: A heavy catapult is a massive engine capable of throwing rocks or heavy objects with great force. Because the catapult throws its payload in a high arc, it can hit squares out of its line of sight. To fire a heavy catapult, the crew chief makes a special check against DC 15 using only his base attack bonus, Intelligence modifier, range increment penalty, and the appropriate modifiers from the lower section of \tabref{Catapult Attack Modifiers}. If the check succeeds, the catapult stone hits the square the catapult was aimed at, dealing the indicated damage to any object or character in the square. Characters who succeed on a DC 15 Reflex save take half damage. Once a catapult stone hits a square, subsequent shots hit the same square unless the catapult is reaimed or the wind changes direction or speed.

If a catapult stone misses, roll 1d8 to determine where it lands. This determines the misdirection of the throw, with 1 being back toward the catapult and 2 through 8 counting clockwise around the target square. Then, count 3 squares away from the target square for every range increment of the attack.

Loading a catapult requires a series of full-round actions. It takes a DC 15 Strength check to winch the throwing arm down; most catapults have wheels to allow up to two crew members to use the aid another action, assisting the main winch operator. A DC 15 \skill{Profession} (siege engineer) check latches the arm into place, and then another DC 15 \skill{Profession} (siege engineer) check loads the catapult ammunition. It takes four full-round actions to reaim a heavy catapult (multiple crew members can perform these full-round actions in the same round, so it would take a crew of four only 1 round to reaim the catapult).

A heavy catapult takes up a space 4.5 meters across.

\textbf{Catapult, Light}: This is a smaller, lighter version of the heavy catapult. It functions as the heavy catapult, except that it takes a DC 10 Strength check to winch the arm into place, and only two full-round actions are required to reaim the catapult.

A light catapult takes up a space 3 meters across.

\textbf{Ballista}: A ballista is essentially a Huge heavy crossbow fixed in place. Its size makes it hard for most creatures to aim it. Thus, a Medium creature takes a $-4$ penalty on attack rolls when using a ballista, and a Small creature takes a $-6$ penalty. It takes a creature smaller than Large two full-round actions to reload the ballista after firing.

A ballista takes up a space 1.5 meter across.

\textbf{Ram}: This heavy pole is sometimes suspended from a movable scaffold that allows the crew to swing it back and forth against objects. As a full-round action, the character closest to the front of the ram makes an attack roll against the AC of the construction, applying the $-4$ penalty for lack of proficiency. (It's not possible to be proficient with this device.) In addition to the damage given on \tabref{Siege Engines}, up to nine other characters holding the ram can add their Strength modifier to the ram's damage, if they devote an attack action to doing so. It takes at least one Huge or larger creature, two Large creatures, four Medium-size creatures, or eight Small creatures to swing a ram. (Tiny or smaller creatures can't use a ram.)

A ram is typically 9 meters long. In a battle, the creatures wielding the ram stand in two adjacent columns of equal length, with the ram between them.

\textbf{Siege Tower}: This device is a massive wooden tower on wheels or rollers that can be rolled up against a wall to allow attackers to scale the tower and thus to get to the top of the wall with cover. The wooden walls are usually 30 centimeters thick.

A typical siege tower takes up a space 4.5 meters across. The creatures inside push it at a speed of 3 meters (and a siege tower can't run). The eight creatures pushing on the ground floor have total cover, and those on higher floors get improved cover and can fire through arrow slits.

