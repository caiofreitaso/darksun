\section{Pterrans}
\Quote{The people of the Tablelands know nothing of life. They choose no Path for themselves, and consume everything until they are dead.}{Keltruch, pterran ranger}

Pterrans are rarely seen in the Tablelands. They live their lives in the Hinterlands, rarely leaving the safety of their villages. However, the recent earthquake and subsequent storms have brought disruption into the pterran's lives. More pterrans now venture outside their homes, and come to the Tyr region to seek trade and information.

\textbf{Personality:} Among strangers, pterrans seem like subdued, cautious beings, but once others earn a pterran's trust, they will find an individual that is open, friendly, inquisitive, and optimistic. In other respects, a pterran's personality is largely shaped by her chosen life path: Pterrans who choose the path of the warrior are less disturbed by the brutality of the Tablelands; they are constantly examining their surroundings and considering how the terrain where they are standing could be defended; they take greatest satisfaction from executing a combat strategy that results in victory without friendly casualties. Pterrans who choose the path of the druid are most interested in plants, animals, and the state of the land; they take greatest satisfaction when they eliminate a threat to nature. Pterrans that choose the path of the mind are most interested in befriending and understanding other individuals and societies; these telepaths take greatest satisfaction from intellectual accomplishments such as solving mysteries, exposing deception, resolving quarrels between individuals, and establishing trade routes between communities.

\textbf{Physical Description:} Pterrans are 5 to 6 \onehalf feet tall reptiles with light brown scaly skin, sharp teeth, and a short tail. Pterrans wear little clothing, preferring belts and loincloths, or sashes. They walk upright, like humanoids, and have opposing thumbs and three-fingered, talon-clawed hands. Pterrans have two shoulder stumps, remnants of wings they possessed long ago, and a finlike growth juts out at the back of their heads. Pterrans weigh between 180 to 220 pounds. There is no visible distinction between male and female pterrans.

\textbf{Relations:} Pterrans are new to the Tablelands, and unaccustomed to cultures and practices of the region. They have learned to not judge too quickly. Their faith in the Earth Mother means they undertake their adventure with open minds, but they will remain subdued and guarded around people they do not trust. A pterran's respect for the Earth Mother governs all his behavior. Creatures that openly destroy the land or show disrespect for the creatures of the wastes are regarded suspiciously. Pterrans understand the natural cycle of life and death, but have difficulty with some aspects of the city life, such as cramped living spaces, piled refuse, and the smells of unwashed humanoids.

\textbf{Alignment:} Pterrans tend towards lawful, well-structured lives, and most of them are good. Evil pterran adventurers are usually outcasts who have committed some horrible offense.

\textbf{Pterran Lands:} Most adventuring pterrans come from one of two villages in the Hinterlands, southwest of the Tyr regions: Pterran Vale and Lost Scale.

\textbf{Magic:} The wizard's use of the environment as a source of power conflicts with a pterran's religious beliefs. Pterrans will cautiously tolerate members of other races who practice preserving magic, if the difference is explained to them.

\textbf{Psionics:} Virtually all pterrans have a telepathic talent, and pterran psions are nearly universally telepaths. Telepathy is considered one of the honored pterran ``life paths.''

\textbf{Religion:} Pterrans worship the Earth Mother, a representation of the whole world of Athas. Their devotion to the Earth Mother is deeply rooted in all aspects of their culture, and it defines a pterran's behavior. All rituals and religious events are related to their worship of the Earth Mother. Religious events include festivals honoring hunts or protection from storms, with a priest presiding over the celebration. Most pterran priests are druids.

\textbf{Language:} Pterran speak Saurian with an accent that is difficult for other races to understand. The long appendage at the back of their head enables them to create sounds that no other race in the Tablelands can reproduce. The sounds are low, and resonate through the pterran's crest. Humanoid vocal chords cannot reproduce such sounds. Pterrans learn the Common tongue easily, but speak it with a slight, odd accent.

\textbf{Names:} Pterrans earn their first name just after they hatch, based on the weather and season of their hatching. After the pterran has decided upon a Life Path and has completed their apprenticeship, she receives title that becomes the first part of her name. This marks her transition into pterran society. There are a number of traditional names associated with each Life Path, but names do not always come from these ranks.

\textbf{Male Names:} Airson, Darksun, Earthsong, Suntail, Goldeye, Onesight, Terrorclaw.

\textbf{Female Names:} Cloudrider, Greenscale, Lifehearth, Rainkeeper, Spiritally, Watertender.

\textbf{Path Name:} Aandu, Caril, Dsar, Everin, Illik, Myril, Odten, Qwes, Pex, Ptellac, Ristu, Ssrui, Tilla, Xandu.

\textbf{Tribe or Village Names:} Pterran Vale, Lost Scale

\textbf{Adventurers:} Pterrans adventure because they believe the recent earthquake and disturbing events are signs from the Earth Mother that they should get more involved in the planet's affairs. They believe that these recent upheavals of nature are signs that the Earth Mother needs help, and this is a call the pterrans will gladly accept. As such, the most brave and adventurous of the pterrans have begun to establish contact with Tyr and some merchant houses, hoping to expand their contacts and information.

\subsection{Pterran Society}
Pterran society is based largely on ceremony and celebrations. An area is set aside in the center of each village for ceremonies. Pterrans revere the world of Athas as the Earth Mother, and believe themselves to be her favored children. Throughout the day, they engage in a number of ceremonies that give thanks to the Earth Mother. These are led by druids who play a very important role in pterran society.

A pterran village is a collection of many smaller family dwellings. Pterrans always bear young in pairs.

At age 15 every pterran chooses a ``life path.'' The three main life paths are the path of the warrior, the path of the druid and the path of the psionicist, though lesser life paths exist as well.

More pterrans follow the path of the warrior than any of the other paths, and become protectors of their villages as well as the tribe's weapon makers.

Pterrans that choose the path of the druid provide an important role in the daily ceremonies to the Earth Mother.

Fewer pterrans choose the path of the psionicist than the other two major paths, as psionics are viewed as outside of nature. Psionicists are viewed with suspicion by the rest of the tribe; however, they do provide valuable skills to the tribe and are often the tribe's negotiators when they meet outsiders.
\vskip2em
Pterrans are omnivores. Much of their diet comes from hunting animals and raising crops. Kirre, id fiend, and flailer are all considered pterran delicacies.

\subsection{Roleplaying Suggestions}
Remember your character class is your ``life path.'' You think of yourself, and present yourself first and foremost as a druid, a warrior or a psion. Remember your daily celebrations and giving of thanks to the Earth Mother. You can usually find a reason to be grateful. Disrespect for the land angers you, since the whole land has withered under the disrespect of foolish humans and others. You celebrate with song and with dance. You have a good sense of humor but it does not extend to blasphemies such as defiling. In initial role-playing situations, you are unfamiliar with the customs and practices of the societies of the Tyr Region. However, you are not primitive by any definition of the word. You look upon differences with curiosity and a willingness to learn, as long as the custom doesn't harm the Earth Mother or her works.

\subsection{Pterran Racial Traits}
\begin{itemize*}
    \item $-2$ Dexterity, +2 Wisdom, +2 Charisma: Pterrans' strong confidence and keen instincts for others' motives make them keen diplomats, and when they take the path of the psion, powerful telepaths.
    \item Humanoid (psionic, reptilian): Pterrans are humanoid creatures with the psionic and reptilian subtypes.
    \item Medium: As Medium creatures, pterrans have no special bonuses or penalties due to size.
    \item Pterran base land speed is 30 feet.
    \item Poor Hearing: Pterrans have only slits for ears, and their hearing sense is diminished. Pterrans suffer a $-2$ penalty to Listen checks.
    \item Natural Weaponry: Pterrans can use their natural weapons instead of fighting with crafted weapons if they so choose. A pterran can rake with their primary claw attack for 1d3 of damage for each claw, and they bite for 1d4 points of damage as a secondary attack. For more on natural attacks, see MM section on natural weapons.
    \item Psi-Like Ability: At will---missive. All pterrans are gifted from the day they hatch with the ability to communicate telepathically, but only with their fellow reptiles. Manifester level is equal to \onehalf Hit Dice (minimum 1st).
    \item Weapon Familiarity: The following weapon is treated as martial rather than as an exotic weapon: thanak. This weapon is more common among pterrans than among other races.
    \item Automatic Languages: Saurian. Bonus Languages: Common, Dwarven, Elven, Halfling, Giant, Gith, Kreen, and Yuan-ti. Pterran know the languages of the few intelligent creatures that live in the Hinterlands.
    \item Life Path: A pterran's life path determines his favored class. Those following the Path of the Druid have druid as a favored class; the Path of the Mind gives psion as a favored class, while the Path of the Warrior gives ranger as a favored class. A Pterran chooses a life path upon coming of age, and the path cannot be changed once chosen at character creation time.
\end{itemize*}