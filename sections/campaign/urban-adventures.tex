\section{Urban Adventures}
At first glance, a city is much like a dungeon, made up of walls, doors, rooms, and corridors. Adventures that take place in cities have two salient differences from their dungeon counterparts, however. Characters have greater access to resources, and they must contend with law enforcement.

\textbf{Access to Resources:} Unlike in dungeons and the wilderness, characters can buy and sell gear quickly in a city. A large city or metropolis probably has high-level NPCs and experts in obscure fields of knowledge who can provide assistance and decipher clues. And when the PCs are battered and bruised, they can retreat to the comfort of a room at the inn.

The freedom to retreat and ready access to the marketplace means that the players have a greater degree of control over the pacing of an urban adventure.

\textbf{Law Enforcement:} The other key distinctions between adventuring in a city and delving into a dungeon is that a dungeon is, almost by definition, a lawless place where the only law is that of the jungle: Kill or be killed. A city, on the other hand, is held together by a code of laws, many of which are explicitly designed to prevent the sort of behavior that adventurers engage in all the time: killing and looting. Even so, most cities' laws recognize monsters as a threat to the stability the city relies on, and prohibitions about murder rarely apply to monsters such as aberrations or evil outsiders. Most evil humanoids, however, are typically protected by the same laws that protect all the citizens of the city. Having an evil alignment is not a crime (except in some severely theocratic cities, perhaps, with the magical power to back up the law); only evil deeds are against the law. Even when adventurers encounter an evildoer in the act of perpetrating some heinous evil upon the populace of the city, the law tends to frown on the sort of vigilante justice that leaves the evildoer dead or otherwise unable to testify at a trial.

\subsection{Weapon, Spell, and Power Restrictions}
Different cities have different laws about such issues as carrying weapons in public and restricting psions or spellcasters. In Athasian culture, generally the following laws will hold true.% no matter where a psion goes.

\begin{itemize*}
\item \textbf{Wizardry is a capital crime.} Wizards are blamed for the scarring of the world for the past 8,000 years. Because of this, using arcane magic in any settlement is considered high treason and wizards are hunted and killed, with very few exceptions such as the Free Wizards of Tyr.

\item \textbf{Crimes committed by psionic means are punished normally.} Killing someone by means of a \psionic{psychic crush} or by \psionic{control body} to march a victim off a rooftop is still murder, and will be treated as such by most authorities. Using psionics to obstruct investigations, resist arrest, or avoid the agents of the sorcerer-king is a crime as well.

\item \textbf{A psion bears the guilt for any crime committed by someone under his mental control.} This is difficult to prove, but there have been cases where a psion has been judged for the crimes of someone he dominated. There have been many more cases were criminals claimed that a mindbender made them break the law. Templars generally scoff at this plea unless some astounding evidence appears to the contrary.

\item \textbf{No one may read another's thoughts.} In most cities it is illegal for a psion to pry into someone's mind without consent. This law is almost impossible to enforce, but it is often used as a general charge against a psion who has angered the templars. Slaves aren't counted as people under Athasian law, however; slave owners may use any means to keep control of their property.

\item \textbf{No one may use the Way to influence another person's thoughts or actions.} This law is almost universal. Using the Way to dominate people or to implant post-hypnotic suggestions is considered the vilest of crimes against a free citizen. Most telepathic devotions fail under this category. %, including awe, daydream, aversion, repugnance, and so forth.

\item \textbf{No one may use the Way to spy on another.} Using \psionic{clairaudience} or \psionic{clairvoyance} to pry into the privacy of a free citizen is considered tema crime. Other means of psionic espionage might include using sight link or sound link through a third party. Again, this is difficult to enforce and even more difficult to prove before the templars.

\item \textbf{The summoning or contact of extraplanar powers is considered high treason.} Most city-dwellers take a very dim view of the reckless summoning of fiends or similar planar horrors.

\item \textbf{Officers of the court may use psionic interrogations in
due process of the law.} If necessary, the templars may summon a master of the Way to get to the bottom of almost any matter. This is usually a last resort by the templars, since even the most oppressive rulers respect their citizens' privacy of thought. 
\end{itemize*}
% The city's laws may not affect all characters equally. A monk isn't hampered at all by a law about peace-bonding weapons, but a cleric is reduced to a fraction of his power if all holy symbols are confiscated at the city's gates.





\subsection{Demographics}

\Table{Town Sizes}{lr{21mm}r{13mm}X}{
  \tableheader Town Size
& \tableheader Population\footnotemark[1]
& \tableheader CP Limit
& \tableheader Examples \\

Small tribe &           20--60 &      20 cp & Sortar's camp \\
Large tribe &          61--200 &      50 cp & Tenpug's band \\
Village     &       201--1,000 &     100 cp & Winter Nest \\
Small town  &     1,001--5,000 &     400 cp & Pterran Vale \\
Large town  &    5,001--10,000 &   1,500 cp &  \\
Small city  &   10,001--50,000 &   7,500 cp & City-states \\
Large city  &  50,001--100,000 &  20,000 cp &  \\
Metropolis  & 100,001--300,000 &  75,000 cp &  \\
Megalopolis & 300,001 or more  & 250,000 cp &  \\
}

\Table{NPCs Needed per Population}{Xcl}{
  \tableheader Class
& \tableheader Levels
& \tableheader For Every... \\

\multicolumn{2}{l}{\TableSubheader{Agriculture}} & 2,000 people\\
Earth cleric  & 5th to 8th     & \\
Ranger        & 11th or higher & \\
Druid         & 12th or higher & \\

\multicolumn{2}{l}{\TableSubheader{Health}} & 15,000 people\\
Cleric & 5th or higher  & \\
Druid  & 12th or higher & \\
}

\textbf{Agriculture:} A big enough settlement needs a spellcaster whose job is to cast \spell{plant growth} on the city's crops. One such spellcaster is required for every 2,000 people---it can be a earth cleric of 5th- to 8th-level, a ranger 11th-level or higher, or a druid 12th-level or higher.

Clerics of 9th-level or higher are called to help with planar duties from their elemental masters, rangers need to be of high enough level to be able cast the spell, while druids are required to wander until they reach the 12th-level.

\textbf{Health:} It is expected to exist a cleric or druid that can cast \spell{remove disease} for every 15,000 people. For smaller communities, there is one priest to attend to a big number of villages and tribes. For big cities, their priests would attend to the city and its satellite towns.

\subsubsection{Community Population and Wealth}
\subsubsection{NPCs in the Community}
\textbf{Highest-Level NPC in the Community for Each Class:} Use the following tables to determine the highest-level character in a given class for a given community. Determine the appropriate community modifier by consulting the first table below; then refer to the second table, roll the dice indicated for the class, and apply the modifier to get a result.

A result of 0 or lower for character level means that no characters of that kind can be found in the community. The maximum level for any class is 20th.


\Table{Community Modifiers}{lX}{
  \tableheader Community Size
& \tableheader Community Modifier \\
Small tribe & $-3$ \\
Large tribe & $-2$ \\
Village     & $-1$ \\
Small town  & +0 \\
Large town  & +1 \\
Small city  & +2 (roll twice)\footnotemark[1] \\
Large city  & +4 (roll thrice)\footnotemark[1] \\
Metropolis  & +8 (roll four times)\footnotemark[1] \\
Megalopolis & +16 (roll five times)\footnotemark[1] \\
\TableNote{2}{1 Cities this large can have more than one high-level NPC per class, each of whom generates lower-level characters of the same class.}\\
}

\Table{Highest-Level Locals}{lX}{
  \tableheader Class
& \tableheader Character Level \\

Barbarian & 1d4 (2) + community modifier\footnotemark[1] \\
Bard      & 1d6 (3) + community modifier \\
Cleric    & 1d6 (3) + community modifier \\
Druid     & 1d3 (2) + community modifier \\
Fighter   & 1d8 (4) + community modifier \\ 
Gladiator & 1d8 (4) + community modifier \\
Psion     & 1d4 (4) + community modifier\footnotemark[2] \\
Ranger    & 1d3 (2) + community modifier\footnotemark[3] \\
Scout     & 1d8 (4) + community modifier \\
Templar   & ---\footnotemark[2] \\
Thief     & 1d8 (4) + community modifier \\
Wizard    & 1d4 (2) + community modifier\footnotemark[2] \\

\TableNote{2}{1 In raiding tribes, level is 1d8 (4) + community modifier.}\\
\TableNote{2}{2 In city states, level is 20.}\\
\TableNote{2}{3 In hunting clans, level is 1d8 (4) + community modifier.}\\
}