\section{Designing Encounters}
The heart of any adventure is its encounters. An encounter is any event that puts a specific problem before the PCs that they must solve. Most encounters present combat with monsters or hostile NPCs, but there are many other types---a trapped corridor, a political interaction with a suspicious king, a dangerous passage over a rickety rope bridge, an awkward argument with a friendly NPC who suspects a PC has betrayed him, or anything that adds drama to the game. Brain-teasing puzzles, roleplaying challenges, and skill checks are all classic methods for resolving encounters, but the most complex encounters to build are the most common ones---combat encounters.

When designing an encounter, you first decide what level of challenge you want your PCs to face, then follow the steps outlined below.

\subsection{Determine Average Party Level}
Determine the average effective character level (ECL) of your player characters---this is their Average Party Level (APL for short). You should round this value to the nearest whole number (this is one of the few exceptions to the round down rule). Note that these encounter creation guidelines assume a group of four or five PCs. If your group contains six or more players, add one to their average level. If your group contains three or fewer players, subtract one from their average level. For example, if your group consists of six players, two of which are 4th level and four of which are 5th level, their APL is 6th (28 total levels, divided by six players, rounding up, and adding one to the final result).

\subsection{Determine Encounter Level}
Encounter Level (or EL) is a convenient number used to indicate the relative danger presented by a monster, trap, hazard, or other encounter---the higher the EL, the more dangerous the encounter. Refer to \tabref{Encounter Design} to determine the Encounter Level your group should face, depending on the difficulty of the challenge you want and the group's APL.

\Table{Encounter Design}{XX}{
  \tableheader Difficulty
& \tableheader Encounter Level Equals\\
Easy        & APL $-1$\footnotemark[1] \\
Average     & APL \\
Challenging & APL +1 \\
Hard        & APL +2 \\
Epic        & APL +3 \\
\TableNote{2}{1 The easy encounter for an average party level of 1 is \onehalf} \\
}

\subsection{Build the Encounter}
The Encounter Level informs the budget you have for the encounter. Use \tabref{Encounter Budget} to get the total number of points and how many points each CR is worth. To build an encounter simply add creatures, skill challenges, traps, and hazards whose combined points does not exceed the budget for the entire encounter. Its easiest to add the highest CR challenges to the encounter first, filling out the remaining total with lesser challenges.

Encounters should not be based on individual challenges (creatures, skill challenges, traps, or hazards) with Challenge Rating less than the APL$-6$, since this difference in level brings no challenge for the party. You can use such low-level challenges to enhance the environment, e.g. servants for a king, traps to delay the PCs. However, they should not be the focus of the encounter.


\Table{Encounter Budget}{CC}{
  \tableheader Challenge Rating
& \tableheader Points \\

APL $-6$ &  1 \\
APL $-5$ &  2 \\
APL $-4$ &  3 \\
APL $-3$ &  4 \\
APL $-2$ &  5 \\
APL $-1$ &  6 \\
APL      &  8 \\
APL  +1  & 12 \\
APL  +2  & 16 \\
APL  +3  & 24 \\
}

For example, let's say you want your group of three 12th-level PCs to face a hard encounter. The PCs have an APL of 11, and \tabref{Encounter Design} tells you that the appropriate encounter level for your APL 11 group would be EL 13. Given the options, you decide to use a pack of dune reapers: three dune reaper drones (CR 6), two dune reaper warriors (CR 7), and a dune reaper matron (CR 8). %To improve the environment, you also add to the encounter five dune reaper drones (CR 6). They are not supposed to be the actual problems for the PCs, but they complete the feel of a dune reaper pack.

% \subsection{Awarding Experience}
% Characters advance in level by defeating monsters, overcoming challenges, and completing adventures---in so doing, they earn experience points (XP for short). Although you can award experience points as soon as a challenge is overcome, this can quickly disrupt the flow of gameplay. It's easier to simply award experience points at the end of a game session---that way if a character earns enough XP to gain a level, the player won't disrupt the game while they level up their character. They can instead take the time between game sessions to do that.

% Keep a list of all the Encounter Levels for each encounter the PCs overcome. At the end of each session, award XP to each PC that participated. Simply 

%  Each monster, trap, and obstacle awards a set amount of XP, as determined by its CR, regardless of the level of the party in relation to the challenge, although you should never bother awarding XP for challenges that have a CR of 10 or more lower than the APL. Pure roleplaying encounters generally have a CR equal to the average level of the party (although particularly easy or difficult roleplaying encounters might be one higher or lower). There are two methods for awarding XP. While one is more exact, it requires a calculator for ease of use. The other is slightly more abstract.

% Exact XP: Once the game session is over, take your list of defeated CR numbers and look up the value of each CR on Table: Experience Point Awards under the “Total XP” column. Add up the XP values for each CR and then divide this total by the number of characters---each character earns an amount of XP equal to this number.

% Abstract XP: Simply add up the individual XP awards listed for a group of the appropriate size. In this case, the division is done for you---you need only total up all the awards to determine how many XP to award to each PC.

% Story Awards: Feel free to award Story Awards when players conclude a major storyline or make an important accomplishment. These awards should be worth double the amount of experience points for a CR equal to the APL. Particularly long or difficult story arcs might award even more, at your discretion as GM.
