\section{Skill Challenges}
Skill challenges are a series of skill checks made by a group of characters in order to obtain a certain objective. They must accumulate a number of successful skill uses before they have too many failures.

A DM should treat a skill challenge just as they treat a monster. Skill challenges have CR, so you can use skill challenge as the only thing happening in an encounter, or just another challenge with monsters and traps.


\subsection{Designing Skill Challenges}
Before designing a skill challenge, understand the context of the encounter. Where it will take place, if it will be a stand-alone challenge or will it be part of a combat?

Give as much attention to the setting of the skill challenge as you do to the setting of the rest of the adventure. You might not need a detailed map full of interesting terrain for a skill challenge, but an interesting setting helps set the tone for the encounter.

If the challenge involves any kind of interaction with nonplayer characters or monsters, detail those characters as well. In a complex social encounter, have a clear picture of the motivations, goals, and interests of the NPCs involved so you can tie them to character skill checks.

A skill challenge can serve as an encounter in and of itself, or it can be combined with monsters as part of a combat encounter.

\subsubsection{Set the Objective}
Define the goal of the challenge and what obstacles the characters face to accomplish that goal. The goal has everything to do with the overall story of the adventure. Success at the challenge should be important to the adventure, but not essential. You don't want a series of bad skill checks to bring the adventure to a grinding halt. At worst, failure at the challenge should send the characters on a long detour, thereby creating a new and interesting part of the adventure.

\subsubsection{Define the Complexity}
Challenge Rating and complexity determine how hard the challenge is for the characters to overcome. The skill challenge's Difficulty Class determine the base Challenge Rating for the encounter, while the complexity determines the number of successes the characters need to overcome the challenge, and how many failures to end the challenge. The complexity also modifies that base CR to give the final Challenge Rating.

Set the DC for the primary skills using \tabref{Base Challenge Rating for Skill Challenge}. As a starting point, use the Challenge Rating closest to the group's level. If you reduce the DCs by 2, reduce the base CR by 1. If you increase the DCs by 2, increase the base CR by 1.

\Table{Base Challenge Rating for Skill Challenge}{CCC}{
  \tableheader Difficulty Class for Primary Skills\textsuperscript{1}
& \tableheader Difficulty Class for Secondary Skills\textsuperscript{1}
& \tableheader Base Challenge Rating \\
DC 10 & DC 15 &  1 \\
DC 15 & DC 20 &  3 \\
DC 20 & DC 25 &  5 \\
DC 25 & DC 30 & 10 \\
DC 30 & DC 35 & 15 \\
DC 40 & DC 45 & 20 \\
\TableNote{3}{1 You can modify the DC by +2 or $-2$. Doing so changes the base CR by +1 or $-1$, respectively.}
}

Set the complexity based on how significant you want the challenge to be (see \tabref{Skill Challenge Complexity}). High complexity challenges are expected to have the same weight as combat encounters. Low complexity challenges are less significant, or work as part of a combat encounter. You can cut the number of failures in half, and increase the Challenge Rating by two.

\Table{Skill Challenge Complexity}{lCCl}{
  \tableheader Complexity Level
& \tableheader Successes
& \tableheader Failures\textsuperscript{1}
& \tableheader Final Challenge Rating \\
1 &  4 & 2 & CR$-4$ \\
2 &  6 & 3 & CR$-3$ \\
3 &  8 & 4 & CR$-2$ \\
4 & 10 & 5 & CR$-1$ \\
5 & 12 & 6 & CR     \\
\TableNote{4}{1 If you halve the number of failures, increase the CR by 2.}\\
}

\subsubsection{Select the Skills}
Certain skills lead to the natural solutions to the problem the challenge presents. These should serve as the primary skills in the challenge. Give some thought to which skills you select here, keeping in mind the goal of involving all the players in the action. You know what skills your player characters are good at, so make sure to include some chances for every character to shine.

Start with a list of the challenge's primary skills, then give some thought to what a character might do when using that skill. You don't need to make an exhaustive list, but try to define categories of actions the characters might take. Sometimes characters might decide to do exactly what you anticipate, but often you need to take what a player wants to do and find the closest match to the actions you've outlined.

When a player's turn comes up in a skill challenge, let that player's character use any skill the player wants. As long as the player or you can come up with
a way to let this secondary skill play a part in the challenge, go for it. If a player wants to use a skill you didn't identify as a primary skill in the challenge, however, then that secondary skill uses a DC 5 points higher. The use of the skill might win the day in unexpected ways, but the risk is greater as well. In addition, a secondary skill can never be used by a single character more than once in a challenge.

Always keep in mind that players can and will come up with ways to use skills you do not expect. Stay on your toes, and let whatever improvised skill uses they come up with guide the rewards and penalties you apply afterward. Remember that not everything has to be directly tied to the challenge. Tangential or unrelated benefits, such as finding a small, forgotten treasure, can also be fun.

\subsubsection{Decide the Consequences}
When the skill challenge ends, reward the characters for their success (with challenge-specific rewards, as well as experience points) or assess penalties for their failure.

A successful skill challenge should be rewarded as a combat encounter for the entire group that participated in it (see \tabref{Experience Awards from Combats}), and each successful skill check should be rewarded individually to each character (see \tabref{Experience Awards from Skill Checks}).

Beyond the fundamental reward, the characters' success should have a significant impact on the story of the adventure. Additional rewards might include information, clues, and favors, as well as simply moving the adventure forward.

If the characters fail the challenge, the story still has to move forward, but in a different direction and possibly by a longer, more dangerous route. You can think of it like a room in a dungeon. If the characters can't defeat the dragon in that room, they don't get the experience for killing it or the treasure it guards, and they can't go through the door on the opposite side of the room. They might still be able to get to the chamber behind the door, but by taking a different and more arduous path. In the same way, failure in a skill challenge should send the characters down a different route in the adventure, but not derail them entirely.

\subsection{Running a Skill Challenge}
A skill challenge should be run akin to a combat encounter (see \chapref{Combat}). Initiative should be rolled to establish an order of play. If the skill challenge is part of a combat encounter, work the challenge into the order just as you do monsters.

Skill challenges need to be described clearly to the players so they know what earns successes. You can't start a skill challenge until the PCs know their role in it, and that means giving them a couple of skills to start.

In a skill challenge encounter, every player character must make skill checks to contribute to the success or failure of the encounter. Characters must make a check on their turn using one of the identified primary skills or they must use a different skill, if they can come up with a way to contribute to the challenge (with a higher DC). A secondary skill can be used only once by a single character in any given skill challenge.

Always give the players the information they need to make smart choice. Make sure to tell them when a check result provides a success or a failure.

\subsubsection{Group Skill Checks}
Sometimes a skill challenge calls for a group skill check. In this case, allow one character to be the leader. This character makes the actual check to gain a success or failure. The others make checks to help the lead character, in effect aiding that character, but their checks provide neither a success nor a failure toward resolving the challenge. They use the Aid Another option, giving +2 circumstance bonus or advantage to the lead character's skill check (depending on the number of successful aids).

\subsection{Sample Skill Challenges}
Use the following skill challenge templates as the basis for skill challenges you design for your adventures. Adjust the complexity and Challenge Rating according to the needs of your adventure.

\subsubsection{Find the Veiled Alliance}
Without inside help, the characters need to find the Veiled Alliance in a foreign city, for goods or shelter against the templars.

\textbf{Objective:} Find the Veiled Alliance to escape from the templars, or to buy material components.

\textbf{Challenge Rating:} 15.

\textbf{Complexity:} Requires 8 successes before 2 failures.

\textbf{Primary Skills:} \skill{Diplomacy}, \skill{Gather Information}, \skill{Knowledge} (arcana), \skill{Listen}, \skill{Sense Motive}, \skill{Spot}.

\textit{Diplomacy (DC 30):} The character tries to bring a trader to help them as a guide to the city's market, where illegal material components can be found. First success with this skill opens up the use of \skill{Sense Motive}.

\textit{Gather Information (DC 30):} The character asks the common people from the city where they could find material components, in a non-incriminating way. Three successes with this skill inform the character the location with the precision of two blocks.

\textit{Knowledge (arcana) (DC 30):} The character understands the veiled code of members of the Alliance. This is available only after one character has gained a success using \skill{Listen}, and it can be used only once in this way during the challenge.

\textit{Listen (DC 35):} The character tries to eavesdrop conversations around the city. First success with this skill opens up the use of \skill{Knowledge} (arcana).

\textit{Sense Motive (DC 30):} The character empathize with the guide. First success with this skill reveals that any use of \skill{Intimidate} earns a failure.

\textit{Spot (DC 30):} The character checks for hidden signs left by the Veiled Alliance throughout the city. First success with this skill opens up the use of \skill{Knowledge} (arcana).

\textit{Intimidate:} Only the first NPC can be intimidated by the group. It counts as failure, however. The NPC will help, but they will alert the templars as well. Any other NPC refuses to be helpful with this request while being intimidated. They either don't know enough or simply won't cooperate.

\textbf{Success:} The characters find the hiding spot of the Veiled Alliance. This could mean they can come as veiled themselves.

\textbf{Failure:} The templars are alerted of the strangers trying to find the Alliance, and chase the PCs throughout the city.

\subsubsection{Urban Chase}
The Templarate is chasing the PCs as they try to escape. A typical chase plays out in rounds, but it could take from minutes to hours in the game world.

\textbf{Objective:} Escape from the templarate.

\textbf{Challenge Rating:} 12.

\textbf{Complexity:} Requires 12 successes before 3 failures.

\textbf{Primary Skills:} \skill{Balance}, \skill{Climb}, \skill{Jump}, \skill{Knowledge} (local), \skill{Spot}, \skill{Tumble}.

\textit{Balance (DC 25):} The character balances in the strait walls between the houses. This is available only after the character has obtained a success using \skill{Climb}.

\textit{Climb (DC 25):} The character climbs on a house and starts to run on the ceiling.

\textit{Jump (DC 25):} The character jumps over an obstacle, or leaps a fence. Failure indicates you stumble upon the obstacle the templarate gets closer. If the character has obtained success using \skill{Climb}, the character jumps over or down to a street. Failure in this jump and the character takes falling damage.

\textit{Knowledge (local) (DC 25):} The character knows enough about the layout of the city to use the environment to their best advantage.

\textit{Spot (DC 20):} The character spots a shortcut, notice a hiding space, or otherwise aid their cause. It doesn't count as a success or a failure but instead gives advantage to the next character's skill check.

\textit{Tumble (DC 35):} The character tumbles through the busy street to gain advantage in the chase.

\textbf{Success:} The PCs evade pursuit or lead their pursuers into an ambush (which leads directly to a combat).

\textbf{Failure:} The templarate catch up with the PCs, and a combat starts immediately.


\subsubsection{Interrogation}
The PCs want to extract information from a prisoner. An interrogation can take minutes, hours, or even days. This is a model of interrogation based on a 5th-level templar. You can use this as a base for interrogations, if you adjust the DCs accordingly to the prisoner in question.

\textbf{Objective:} Convince an NPC to give you the information.

\textbf{Challenge Rating:} 1.

\textbf{Complexity:} Requires 4 successes before 2 failures.

\textbf{Primary Skills:} \skill{Bluff}, \skill{Diplomacy}, \skill{Intimidate}.

\textit{Bluff (DC 20):} The character tricks the NPC into revealing a vital piece of information. Failure closes off this approach and increases the DCs for other checks by +5 for the duration of the challenge.

\textit{Diplomacy (DC 25):} The character reasons or bargains with the NPC, offering something in good faith for the information. At the end of the challenge, if three or more successes came from this approach, the character must do everything in their power to hold up their end of the bargain.

\textit{Intimidate (DC 20):} The character threatens the NPC. Failure closes off this approach and increases the DCs for other checks by +5 for the duration of the challenge.

\textbf{Success:} The PCs learn the desired information, and the NPC might agree to spy for them or otherwise give more information in the future.

\textbf{Failure:} The NPC refuses to give any useful information, or gives incorrect or dangerously inaccurate information. The PCs might walk into an ambush, or give the wrong code phrase, or otherwise run into trouble due to the false information they received.
