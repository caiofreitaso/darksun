\section{Traps}

\textbf{Types of Traps:} A trap can be either mechanical or magic in nature. Mechanical traps include pits, arrow traps, falling blocks, water-filled rooms, whirling blades, and anything else that depends on a mechanism to operate. A mechanical trap can be constructed by a PC through successful use of the \skill{Craft} (trapmaking) skill (see Designing a Trap, below, and the skill description).

Magic traps are further divided into spell traps and magic device traps. Magic device traps initiate spell effects when activated, just as wands, rods, rings, and other magic items do. Creating a magic device trap requires the \feat{Craft Wondrous Item} feat (see Designing a Trap and the feat description).

Spell traps are simply spells that themselves function as traps. Creating a spell trap requires the services of a character who can cast the needed spell or spells, who is usually either the character creating the trap or an NPC spellcaster hired for the purpose.

\subsection{Mechanical Traps}
Dungeons are frequently equipped with deadly mechanical (nonmagical) traps. A trap typically is defined by its location and triggering conditions, how hard it is to spot before it goes off, how much damage it deals, and whether or not the heroes receive a saving throw to mitigate its effects. Traps that attack with arrows, sweeping blades, and other types of weaponry make normal attack rolls, with a specific attack bonus dictated by the trap’s design.

Creatures who succeed on a DC 20 \skill{Search} check detect a simple mechanical trap before it is triggered. (A simple trap is a snare, a trap triggered by a tripwire, or a large trap such as a pit.)

A character with the trapfinding class feature who succeeds on a DC 21 (or higher) \skill{Search} check detects a well-hidden or complex mechanical trap before it is triggered. Complex traps are denoted by their triggering mechanisms and involve pressure plates, mechanisms linked to doors, changes in weight, disturbances in the air, vibrations, and other sorts of unusual triggers.

\subsection{Magic Traps}
Many spells can be used to create dangerous traps. Unless the spell or item description states otherwise, assume the following to be true.

\begin{itemize*}
\item A successful \skill{Search} check (DC 25 + spell level) made by a rogue (and only a rogue) detects a magic trap before it goes off. Other characters have no chance to find a magic trap with a \skill{Search} check.
\item Magic traps permit a saving throw in order to avoid the effect (DC 10 + spell level $\times$ 1.5).
\item Magic traps may be disarmed by a rogue (and only a rogue) with a successful \skill{Disable Device} check (DC 25 + spell level).
\end{itemize*}

\subsection{Elements of a Trap}
All traps---mechanical or magic---have the following elements: trigger, reset, \skill{Search} DC, \skill{Disable Device} DC, attack bonus (or saving throw or onset delay), damage/effect, and Challenge Rating. Some traps may also include optional elements, such as poison or a bypass. These characteristics are described below.

\subsubsection{Trigger}
A trap's trigger determines how it is sprung.

\textbf{Location Trigger:} A location trigger springs a trap when someone stands in a particular square.

\textbf{Proximity Trigger:} This trigger activates the trap when a creature approaches within a certain distance of it. A proximity trigger differs from a location trigger in that the creature need not be standing in a particular square. Creatures that are flying can spring a trap with a proximity trigger but not one with a location trigger. Mechanical proximity triggers are extremely sensitive to the slightest change in the air. This makes them useful only in places such as crypts, where the air is unusually still.

The proximity trigger used most often for magic device traps is the \spell{alarm} spell. Unlike when the spell is cast, an \spell{alarm} spell used as a trigger can have an area that's no larger than the area the trap is meant to protect.

Some magic device traps have special proximity triggers that activate only when certain kinds of creatures approach. For example, a \spell{detect good} spell can serve as a proximity trigger on an evil altar, springing the attached trap only when someone of good alignment gets close enough to it.

\textbf{Sound Trigger:} This trigger springs a magic trap when it detects any sound. A sound trigger functions like an ear and has a +15 bonus on Listen checks. A successful Move Silently check, magical silence, and other effects that would negate hearing defeat it. A trap with a sound trigger requires the casting of clairaudience during its construction.

\textbf{Visual Trigger:} This trigger for magic traps works like an actual eye, springing the trap whenever it ``sees'' something. A trap with a visual trigger requires the casting of \spell{arcane eye}, \spellref{clairvoyance/clairaudience}{clairvoyance}, or \spell{true seeing} during its construction. Sight range and the Spot bonus conferred on the trap depend on the spell chosen, as shown.

\Table{}{lXc}{
  \tableheader Spell
& \tableheader Sight Range
& \tableheader Spot Bonus \\
\spell{arcane eye} & Line of sight (unlimited range) & +20 \\
\spellref{clairaudience/clairvoyance}{clairaudience} & One preselected location & +15 \\
\spell{true seeing} & Line of sight (up to 36 m) & +30 \\
}

If you want the trap to ``see'' in the dark, you must either choose the \spell{true seeing} option or add \spell{darkvision} to the trap as well. (\spell{Darkvision} limits the trap's sight range in the dark to 18 meters.) If \spell{invisibility}, disguises, or illusions can fool the spell being used, they can fool the visual trigger as well.

\textbf{Touch Trigger:} A touch trigger, which springs the trap when touched, is one of the simplest kinds of trigger to construct. This trigger may be physically attached to the part of the mechanism that deals the damage or it may not. You can make a magic touch trigger by adding alarm to the trap and reducing the area of the effect to cover only the trigger spot.

\textbf{Timed Trigger:} This trigger periodically springs the trap after a certain duration has passed.

\textbf{Spell Trigger:} All spell traps have this kind of trigger. The appropriate spell descriptions explain the trigger conditions for traps that contain spell triggers.

\subsubsection{Reset}
A reset element is the set of conditions under which a trap becomes ready to trigger again.

\textbf{No Reset:} Short of completely rebuilding the trap, there's no way to trigger it more than once. Spell traps have no reset element.

\textbf{Repair Reset:} To get the trap functioning again, you must repair it.

\textbf{Manual Reset:} Resetting the trap requires someone to move the parts back into place. This is the kind of reset element most mechanical traps have.

\textbf{Automatic Reset:} The trap resets itself, either immediately or after a timed interval.

\subsubsection{Repairing And Resetting Mechanical Traps}
Repairing a mechanical trap requires a \skill{Craft} (trapmaking) check against a DC equal to the one for building it. The cost for raw materials is one-fifth of the trap's original market price. To calculate how long it takes to fix a trap, use the same calculations you would for building it, but use the cost of the raw materials required for repair in place of the market price.

Resetting a trap usually takes only a minute or so. For a trap with a more difficult reset method, you should set the time and labor required.

\subsubsection{Bypass (Optional Element)}
If the builder of a trap wants to be able to move past the trap after it is created or placed, it's a good idea to build in a bypass mechanism---something that temporarily disarms the trap. Bypass elements are typically used only with mechanical traps; spell traps usually have built-in allowances for the caster to bypass them.

\textbf{Lock:} A lock bypass requires a DC 30 \skill{Open Lock} check to open.

\textbf{Hidden Switch:} A hidden switch requires a DC 25 \skill{Search} check to locate.

\textbf{Hidden Lock:} A hidden lock combines the features above, requiring a DC 25 \skill{Search} check to locate and a DC 30 \skill{Open Lock} check to open.

\subsubsection{Search And Disable Device DCs}
The builder sets the \skill{Search} and Disable Device DCs for a mechanical trap. For a magic trap, the values depend on the highest-level spell used.

\textbf{Mechanical Trap:} The base DC for both \skill{Search} and \skill{Disable Device} checks is 20. Raising or lowering either of these DCs affects the base cost (Table: Cost Modifiers for Mechanical Traps) and possibly the CR (Table: CR Modifiers for Mechanical Traps).

\textbf{Magic Trap:} The DC for both \skill{Search} and \skill{Disable Device} checks is equal to 25 + the spell level of the highest-level spell used. Only characters with the trapfinding class feature can attempt a \skill{Search} check or a \skill{Disable Device} check involving a magic trap. These DCs do not affect the trap's cost or CR.

\subsubsection{Attack Bonus/Saving Throw DC}
A trap usually either makes an attack roll or forces a saving throw to avoid it. Occasionally a trap uses both of these options, or neither (see Never Miss).

\textbf{Pits:} These are holes (covered or not) that characters can fall into and take damage. A pit needs no attack roll, but a successful Reflex save (DC set by the builder) avoids it. Other save-dependent mechanical traps also fall into this category.

Pits in dungeons come in three basic varieties: uncovered, covered, and chasms. Pits and chasms can be defeated by judicious application of the Climb skill, the Jump skill, or various magical means.

Uncovered pits serve mainly to discourage intruders from going a certain way, although they cause much grief to characters who stumble into them in the dark, and they can greatly complicate a melee taking place nearby.

Covered pits are much more dangerous. They can be detected with a DC 20 \skill{Search} check, but only if the character is taking the time to carefully examine the area before walking across it. A character who fails to detect a covered pit is still entitled to a DC 20 Reflex save to avoid falling into it. However, if she was running or moving recklessly at the time, she gets no saving throw and falls automatically.

Trap coverings can be as simple as piled refuse (straw, leaves, sticks, garbage), a large rug, or an actual trapdoor concealed to appear as a normal part of the floor. Such a trapdoor usually swings open when enough weight (usually about 50 to 80 pounds) is placed upon it. Devious trap builders sometimes design trapdoors so that they spring back shut after they open. The trapdoor might lock once it's back in place, leaving the stranded character well and truly trapped. Opening such a trapdoor is just as difficult as opening a regular door (assuming the trapped character can reach it), and a DC 23 Strength check is needed to keep a spring-loaded door open.
% Trap coverings can be as simple as piled refuse (straw, leaves, sticks, garbage), a large rug, or an actual trapdoor concealed to appear as a normal part of the floor. Such a trapdoor usually swings open when enough weight (usually about 50 to 80 pounds) is placed upon it. Devious trap builders sometimes design trapdoors so that they spring back shut after they open. The trapdoor might lock once it's back in place, leaving the stranded character well and truly trapped. Opening such a trapdoor is just as difficult as opening a regular door (assuming the trapped character can reach it), and a DC 13 Strength check is needed to keep a spring-loaded door open.

Pit traps often have something nastier than just a hard floor at the bottom. A trap designer may put spikes, monsters, or a pool of acid, lava, or even water at the bottom. Spikes at the bottom of a pit deal damage as daggers with a +10 attack bonus and a +1 bonus on damage for every 10 feet of the fall (to a maximum bonus on damage of +5). If the pit has multiple spikes, a falling victim is attacked by 1d4 of them. This damage is in addition to any damage from the fall itself.

Monsters sometimes live in pits. Any monster that can fit into the pit might have been placed there by the dungeon's designer, or might simply have fallen in and not been able to climb back out.

A secondary trap, mechanical or magical, at the bottom of a pit can be particularly deadly. Activated by a falling victim, the secondary trap attacks the already injured character when she's least ready for it.

\textbf{Ranged Attack Traps:} These traps fling darts, arrows, spears, or the like at whoever activated the trap. The builder sets the attack bonus. A ranged attack trap can be configured to simulate the effect of a composite bow with a high strength rating which provides the trap with a bonus on damage equal to its strength rating.

\textbf{Melee Attack Traps:} These traps feature such obstacles as sharp blades that emerge from walls and stone blocks that fall from ceilings. Once again, the builder sets the attack bonus.

\subsubsection{Damage/Effect}
The effect of a trap is what happens to those who spring it. Usually this takes the form of either damage or a spell effect, but some traps have special effects.

\textbf{Pits:} Falling into a pit deals 1d6 points of damage per 10 feet of depth.

\textbf{Ranged Attack Traps:} These traps deal whatever damage their ammunition normally would. If a trap is constructed with a high strength rating, it has a corresponding bonus on damage.

\textbf{Melee Attack Traps:} These traps deal the same damage as the melee weapons they ``wield.'' In the case of a falling stone block, you can assign any amount of bludgeoning damage you like, but remember that whoever resets the trap has to lift that stone back into place.

A melee attack trap can be constructed with a built-in bonus on damage rolls, just as if the trap itself had a high Strength score.

\textbf{Spell Traps:} Spell traps produce the spell's effect. Like all spells, a spell trap that allows a saving throw has a save DC of 10 + spell level + caster's relevant ability modifier.

\textbf{Magic Device Traps:} These traps produce the effects of any spells included in their construction, as described in the appropriate entries. If the spell in a magic device trap allows a saving throw, its save DC is 10 + spell level $\times$ 1.5. Some spells make attack rolls instead.

\textbf{Special:} Some traps have miscellaneous features that produce special effects, such as drowning for a water trap or ability damage for poison. Saving throws and damage depend on the poison or are set by the builder, as appropriate.

\subsubsection{Miscellaneous Trap Features}
Some traps include optional features that can make them considerably more deadly. The most common such features are discussed below.

\textbf{Alchemical Item:} Mechanical traps may incorporate alchemical devices or other special substances or items, such as tanglefoot bags, alchemist's fire, thunderstones, and the like. Some such items mimic spell effects. If the item mimics a spell effect, it increases the CR as shown on Table: CR Modifiers for Mechanical Traps.

\textbf{Gas:} With a gas trap, the danger is in the inhaled poison it delivers. Traps employing gas usually have the never miss and onset delay features (see below).

\textbf{Liquid:} Any trap that involves a danger of drowning is in this category. Traps employing liquid usually have the never miss and onset delay features (see below).

\textbf{Multiple Target:} Traps with this feature can affect more than one character.

\textbf{Multiple Attacks:} A melee attack trap can make multiple attacks with the same weapon, slashing back and forth at careless adventurers before resuming its reset position. Each successive attack after the first is made with a cumulative $-5$ penalty on the attack roll, and the trap can make as many attacks as its attack bonus divided by 6 (rounded down, maximum 4). If the trap is also poisoned, the poison is effective only on the first attack that strikes a target, after which the blade is not envenomed again until the trap resets.

\textbf{Never Miss:} When the entire dungeon wall moves to crush you, your quick reflexes won't help, since the wall can't possibly miss. A trap with this feature has neither an attack bonus nor a saving throw to avoid, but it does have an onset delay (see below). Most traps involving liquid or gas are of the never miss variety.

\textbf{Onset Delay:} An onset delay is the amount of time between when the trap is sprung and when it deals damage. A never miss trap always has an onset delay.

\textbf{Poison:} Traps that employ poison are deadlier than their nonpoisonous counterparts, so they have correspondingly higher CRs. To determine the CR modifier for a given poison, consult Table: CR Modifiers for Mechanical Traps. Only injury, contact, and inhaled poisons are suitable for traps; ingested types are not. Some traps simply deal the poison's damage. Others deal damage with ranged or melee attacks as well.

\textbf{Pit Spikes:} Treat spikes at the bottom of a pit as daggers, each with a +10 attack bonus. The damage bonus for each spike is +1 per 10 feet of pit depth (to a maximum of +5). Each character who falls into the pit is attacked by 1d4 spikes. Pit spikes do not add to the average damage of the trap (see Average Damage, below).

\textbf{Pit Bottom:} If something other than spikes waits at the bottom of a pit, it's best to treat that as a separate trap (see Multiple Traps, below) with a location trigger that activates on any significant impact, such as a falling character.

\textbf{Touch Attack:} This feature applies to any trap that needs only a successful touch attack (melee or ranged) to hit.

\subsection{Sample Traps}
The costs listed for mechanical traps are market prices; those for magic traps are raw material costs. Caster level and class for the spells used to produce the trap effects are provided in the entries for magic device traps and spell traps. For all other spells used (in triggers, for example), the caster level is assumed to be the minimum required.

\subsubsection{CR 1 Traps}
\textbf{Basic Arrow Trap:} CR 1; mechanical; proximity trigger; manual reset; Atk +10 ranged (1d6/$\times$3, arrow); \skill{Search} DC 20; \skill{Disable Device} DC 20. Market Price: 2,000 cp.

\textbf{Camouflaged Pit Trap:} CR 1; mechanical; location trigger; manual reset; DC 20 Reflex save avoids; 3 m deep (1d6, fall); \skill{Search} DC 24; \skill{Disable Device} DC 20. Market Price: 1,800 cp.

\textbf{Deeper Pit Trap:} CR 1; mechanical; location trigger; manual reset; hidden switch bypass (\skill{Search} DC 25); DC 15 Reflex save avoids; 6 m deep (2d6, fall); multiple targets (first target in each of two adjacent 1.5 m squares); \skill{Search} DC 20; \skill{Disable Device} DC 23. Market Price: 1,300 cp.

\textbf{Fusillade of Darts:} CR 1; mechanical; location trigger; manual reset; Atk +10 ranged (1d4+1, dart); multiple targets (fires 1d4 darts at each target in two adjacent 1.5 m squares); \skill{Search} DC 14; \skill{Disable Device} DC 20. Market Price: 500 cp.

\textbf{Poison Dart Trap:} CR 1; mechanical; location trigger; manual reset; Atk +8 ranged (1d4 plus poison, dart); poison (bloodroot, DC 12 Fortitude save resists, 0/1d4 Con plus 1d3 Wis); \skill{Search} DC 20; \skill{Disable Device} DC 18. Market Price: 700 cp.

\textbf{Poison Needle Trap:} CR 1; mechanical; touch trigger; manual reset; Atk +8 ranged (1 plus greenblood oil poison); \skill{Search} DC 22; \skill{Disable Device} DC 20. Market Price: 1,300 cp.

\textbf{Portcullis Trap:} CR 1; mechanical; location trigger; manual reset; Atk +10 melee (3d6); \skill{Search} DC 20; \skill{Disable Device} DC 20. Note: Damage applies only to those underneath the portcullis. Portcullis blocks passageway. Market Price: 1,400 cp.

\textbf{Razor-Wire across Hallway:} CR 1; mechanical; location trigger; no reset; Atk +10 melee (2d6, wire); multiple targets (first target in each of two adjacent 1.5 m squares); \skill{Search} DC 22; \skill{Disable Device} DC 15. Market Price: 400 cp.

\textbf{Rolling Rock Trap:} CR 1; mechanical; location trigger; manual reset; Atk +10 melee (2d6, rock); \skill{Search} DC 20; \skill{Disable Device} DC 22. Market Price: 1,400 cp.

\textbf{Scything Blade Trap:} CR 1; mechanical; location trigger; automatic reset; Atk +8 melee (1d8/$\times$3); \skill{Search} DC 21; \skill{Disable Device} DC 20. Market Price: 1,700 cp.

\textbf{Spear Trap:} CR 1; mechanical; location trigger; manual reset; Atk +12 ranged (1d8/$\times$3, spear); \skill{Search} DC 20; \skill{Disable Device} DC 20. Note: 60 m max range, target determined randomly from those in its path. Market Price: 1,200 cp.

\textbf{Swinging Block Trap:} CR 1; mechanical; touch trigger; manual reset; Atk +5 melee (4d6, stone block); \skill{Search} DC 20; \skill{Disable Device} DC 20. Market Price: 500 cp.

\textbf{Wall Blade Trap:} CR 1; mechanical; touch trigger; automatic reset; hidden switch bypass (\skill{Search} DC 25); Atk +10 melee (2d4/$\times$4, scythe); \skill{Search} DC 22; \skill{Disable Device} DC 22. Market Price: 2,500 cp.

\subsubsection{CR 2 Traps}
\textbf{Box of Brown Mold:} CR 2; mechanical; touch trigger (opening the box); automatic reset; 1.5 m cold aura (3d6, cold nonlethal); \skill{Search} DC 22; \skill{Disable Device} DC 16. Market Price: 3,000 cp.

\textbf{Bricks from Ceiling:} CR 2; mechanical; touch trigger; repair reset; Atk +12 melee (2d6, bricks); multiple targets (all targets in two adjacent 1.5 m squares); \skill{Search} DC 20; \skill{Disable Device} DC 20. Market Price: 2,400 cp.

\textbf{Burning Hands Trap:} CR 2; magic device; proximity trigger (\spell{alarm}); automatic reset; spell effect (\spell{burning hands}, 1st-level wizard, 1d4 fire, DC 11 Reflex save half damage); \skill{Search} DC 26; \skill{Disable Device} DC 26. Cost: 500 cp, 40 XP.

\textbf{Camouflaged Pit Trap:} CR 2; mechanical; location trigger; manual reset; DC 20 Reflex save avoids; 6 m deep (2d6, fall); multiple targets (first target in each of two adjacent 1.5 m squares); \skill{Search} DC 24; \skill{Disable Device} DC 19. Market Price: 3,400 cp.

\textbf{Inflict Light Wounds Trap:} CR 2; magic device; touch trigger; automatic reset; spell effect (\spell{inflict light wounds}, 1st-level cleric, 1d8+1, DC 11 Will save half damage); \skill{Search} DC 26; \skill{Disable Device} DC 26. Cost: 500 cp, 40 XP.

\textbf{Javelin Trap:} CR 2; mechanical; location trigger; manual reset; Atk +16 ranged (1d6+4, javelin); \skill{Search} DC 20; \skill{Disable Device} DC 18. Market Price: 4,800 cp.

\textbf{Large Net Trap:} CR 2; mechanical; location trigger; manual reset; Atk +5 melee (see note); \skill{Search} DC 20; \skill{Disable Device} DC 25. Note: Characters in 3-m square are grappled by net (Str 18) if they fail a DC 14 Reflex save. Market Price: 3,000 cp.

\textbf{Pit Trap:} CR 2; mechanical, location trigger; manual reset; DC 20 Reflex save avoids; 12 m deep (4d6, fall); \skill{Search} DC 20; \skill{Disable Device} DC 20. Market Price: 2,000 cp.

\textbf{Poison Needle Trap:} CR 2; mechanical; touch trigger; repair reset; lock bypass (\skill{Open Lock} DC 30); Atk +17 melee (1 plus poison, needle); poison (blue whinnis, DC 14 Fortitude save resists (poison only), 1 Con/unconsciousness); \skill{Search} DC 22; \skill{Disable Device} DC 17. Market Price: 4,720 cp.

\textbf{Spiked Pit Trap:} CR 2; mechanical; location trigger; automatic reset; DC 20 Reflex save avoids; 6 m deep (2d6, fall); multiple targets (first target in each of two adjacent 1.5 m squares); pit spikes (Atk +10 melee, 1d4 spikes per target for 1d4+2 each); \skill{Search} DC 18; \skill{Disable Device} DC 15. Market Price: 1,600 cp.

\textbf{Tripping Chain:} CR 2; mechanical; location trigger; automatic reset; multiple traps (tripping and melee attack); Atk +15 melee touch (trip), Atk +15 melee (2d4+2, spiked chain); \skill{Search} DC 15; \skill{Disable Device} DC 18. Market Price: 3,800 cp. Note: This trap is really one CR 1 trap that trips and a second CR 1 trap that attacks with a spiked chain. If the tripping attack succeeds, a +4 bonus applies to the spiked chain attack because the opponent is prone.

\textbf{Well-Camouflaged Pit Trap:} CR 2; mechanical; location trigger; repair reset; DC 20 Reflex save avoids; 3 m deep (1d6, fall); \skill{Search} DC 27; \skill{Disable Device} DC 20. Market Price: 4,400 cp.

\subsubsection{CR 3 Traps}
\textbf{Burning Hands Trap:} CR 3; magic device; proximity trigger (\spell{alarm}); automatic reset; spell effect (\spell{burning hands}, 5th-level wizard, 5d4 fire, DC 11 Reflex save half damage); \skill{Search} DC 26; \skill{Disable Device} DC 26. Cost: 2,500 cp, 200 XP.

\textbf{Camouflaged Pit Trap:} CR 3; mechanical; location trigger; manual reset; DC 20 Reflex save avoids; 9 m deep (3d6, fall); multiple targets (first target in each of two adjacent squares); \skill{Search} DC 24; \skill{Disable Device} DC 18. Market Price: 4,800 cp.

\textbf{Ceiling Pendulum:} CR 3; mechanical; timed trigger; automatic reset; Atk +15 melee (1d12+8/$\times$3, greataxe); \skill{Search} DC 15; \skill{Disable Device} DC 27. Market Price: 14,100 cp.

\textbf{Fire Trap:} CR 3; spell; spell trigger; no reset; spell effect (\spell{fire trap}, 3rd-level druid, 1d4+3 fire, DC 13 Reflex save half damage); \skill{Search} DC 27; \skill{Disable Device} DC 27. Cost: 85 cp to hire NPC spellcaster.

\textbf{Extended Bane Trap:} CR 3; magic device; proximity trigger (\spell{detect good}); automatic reset; spell effect (\emph{extended} \spell{bane}, 3rd-level cleric, DC 13 Will save negates); \skill{Search} DC 27; \skill{Disable Device} DC 27. Cost: 3,500 cp, 280 XP.

\textbf{Ghoul Touch Trap:} CR 3; magic device; touch trigger; automatic reset; spell effect (\spell{ghoul touch}, 3rd-level wizard, DC 13 Fortitude save negates); \skill{Search} DC 27; \skill{Disable Device} DC 27. Cost: 3,000 cp, 240 XP.

\textbf{Hail of Needles:} CR 3; mechanical; location trigger; manual reset; Atk +20 ranged (2d4); \skill{Search} DC 22; \skill{Disable Device} DC 22. Market Price: 5,400 cp.

\textbf{Acid Arrow Trap:} CR 3; magic device; proximity trigger (\spell{alarm}); automatic reset; Atk +2 ranged touch; spell effect (\spell{acid arrow}, 3rd-level wizard, 2d4 acid/round for 2 rounds); \skill{Search} DC 27; \skill{Disable Device} DC 27. Cost: 3,000 cp, 240 XP.

\textbf{Pit Trap:} CR 3; mechanical, location trigger; manual reset; DC 20 Reflex save avoids; 18 m deep (6d6, fall); \skill{Search} DC 20; \skill{Disable Device} DC 20. Market Price: 3,000 cp.

\textbf{Poisoned Arrow Trap:} CR 3; mechanical; touch trigger; manual reset; lock bypass (\skill{Open Lock} DC 30); Atk +12 ranged (1d8 plus poison, arrow); poison (Large monstrous scorpion venom, DC 14 Fortitude save resists, 1d4 Con/1d4 Con); \skill{Search} DC 19; \skill{Disable Device} DC 15. Market Price: 2,900 cp.

\textbf{Spiked Pit Trap:} CR 3; mechanical; location trigger; manual reset; DC 20 Reflex save avoids; 6 m deep (2d6, fall); multiple targets (first target in each of two adjacent 1.5 m squares); pit spikes (Atk +10 melee, 1d4 spikes per target for 1d4+2 each); \skill{Search} DC 21; \skill{Disable Device} DC 20. Market Price: 3,600 cp.

\textbf{Stone Blocks from Ceiling:} CR 3; mechanical; location trigger; repair reset; Atk +10 melee (4d6, stone blocks); \skill{Search} DC 25; \skill{Disable Device} DC 20. Market Price: 5,400 cp.

\subsubsection{CR 4 Traps}
\textbf{Bestow Curse Trap:} CR 4; magic device; touch trigger (\spell{detect chaos}); automatic reset; spell effect (\spell{bestow curse}, 5th-level cleric, DC 14 Will save negates); \skill{Search} DC 28; \skill{Disable Device} DC 28. Cost: 8,000 cp, 640 XP.

\textbf{Camouflaged Pit Trap:} CR 4; mechanical; location trigger; manual reset; DC 20 Reflex save avoids; 12 m deep (4d6, fall); multiple targets (first target in each of two adjacent 1.5 m squares); \skill{Search} DC 25; \skill{Disable Device} DC 17. Market Price: 6,800 cp.

\textbf{Collapsing Column:} CR 4; mechanical; touch trigger (attached); no reset; Atk +15 melee (6d6, stone blocks); \skill{Search} DC 20; \skill{Disable Device} DC 24. Market Price: 8,800 cp.

\textbf{Glyph of Warding (Blast):} CR 4; spell; spell trigger; no reset; spell effect (\spell{glyph of warding} [blast], 5th-level cleric, 2d8 acid, DC 14 Reflex save half damage); multiple targets (all targets within 1.5 m); \skill{Search} DC 28; \skill{Disable Device} DC 28. Cost: 350 cp to hire NPC spellcaster.

\textbf{Lightning Bolt Trap:} CR 4; magic device; proximity trigger (\spell{alarm}); automatic reset; spell effect (\spell{lightning bolt}, 5th-level wizard, 5d6 electricity, DC 14 Reflex save half damage); \skill{Search} DC 28; \skill{Disable Device} DC 28. Cost: 7,500 cp, 600 XP.

\textbf{Pit Trap:} CR 4; mechanical, location trigger; manual reset; DC 20 Reflex save avoids; 24 m deep (8d6, fall); \skill{Search} DC 20; \skill{Disable Device} DC 20. Market Price: 4,000 cp.

\textbf{Poisoned Dart Trap:} CR 4; mechanical; location trigger; manual reset; Atk +15 ranged (1d4+4 plus poison, dart); multiple targets (1 dart per target in a 3-m-by-3-m area); poison (Small monstrous centipede poison, DC 10 Fortitude save resists, 1d2 Dex/1d2 Dex); \skill{Search} DC 21; \skill{Disable Device} DC 22. Market Price: 12,090 cp.

\textbf{Sepia Snake Sigil Trap:} CR 4; spell; spell trigger; no reset; spell effect (\spell{sepia snake sigil}, 5th-level wizard, DC 14 Reflex save negates); \skill{Search} DC 28; \skill{Disable Device} DC 28. Cost: 650 cp to hire NPC spellcaster.

\textbf{Spiked Pit Trap:} CR 4; mechanical; location trigger; automatic reset; DC 20 Reflex save avoids; 18 m deep (6d6, fall); pit spikes (Atk +10 melee, 1d4 spikes per target for 1d4+5 each); \skill{Search} DC 20; \skill{Disable Device} DC 20. Market Price: 4,000 cp.

\textbf{Wall Scythe Trap:} CR 4; mechanical; location trigger; automatic reset; Atk +20 melee (2d4+8/$\times$4, scythe); \skill{Search} DC 21; \skill{Disable Device} DC 18. Market Price: 17,200 cp.

\textbf{Water-Filled Room Trap:} CR 4; mechanical; location trigger; automatic reset; multiple targets (all targets in a 3-m-by-3-m room); never miss; onset delay (5 rounds); liquid; \skill{Search} DC 17; \skill{Disable Device} DC 23. Market Price: 11,200 cp.

\textbf{Wide-Mouth Spiked Pit Trap:} CR 4; mechanical; location trigger; manual reset; DC 20 Reflex save avoids; 6 m deep (2d6, fall); multiple targets (first target in each of two adjacent 1.5 m squares); pit spikes (Atk +10 melee, 1d4 spikes per target for 1d4+2 each); \skill{Search} DC 18; \skill{Disable Device} DC 25. Market Price: 7,200 cp.

\subsubsection{CR 5 Traps}
\textbf{Camouflaged Pit Trap:} CR 5; mechanical; location trigger; manual reset; DC 20 Reflex save avoids; 50 ft. deep (5d6, fall); multiple targets (first target in each of two adjacent 1.5 m squares); \skill{Search} DC 25; \skill{Disable Device} DC 17. Market Price: 8,500 cp.

\textbf{Doorknob Smeared with Contact Poison:} CR 5; mechanical; touch trigger (attached); manual reset; poison (nitharit, DC 13 Fortitude save resists, 0/3d6 Con); \skill{Search} DC 25; \skill{Disable Device} DC 19. Market Price: 9,650 cp.

\textbf{Falling Block Trap:} CR 5; mechanical; location trigger; manual reset; Atk +15 melee (6d6); multiple targets (can strike all characters in two adjacent specified squares); \skill{Search} DC 20; \skill{Disable Device} DC 25. Market Price: 15,000 cp.

\textbf{Fire Trap:} CR 5; spell; spell trigger; no reset; spell effect (\spell{fire trap}, 7th-level wizard, 1d4+7 fire, DC 16 Reflex save half damage); \skill{Search} DC 29; \skill{Disable Device} DC 29. Cost: 305 cp to hire NPC spellcaster.

\textbf{Fireball Trap:} CR 5; magic device; touch trigger; automatic reset; spell effect (\spell{fireball}, 8th-level wizard, 8d6 fire, DC 14 Reflex save half damage); \skill{Search} DC 28; \skill{Disable Device} DC 28. Cost: 12,000 cp, 960 XP.

\textbf{Flooding Room Trap:} CR 5; mechanical; proximity trigger; automatic reset; no attack roll necessary (see note below); \skill{Search} DC 20; \skill{Disable Device} DC 25. Note: Room floods in 4 rounds. Market Price: 17,500 cp.

\textbf{Fusillade of Darts:} CR 5; mechanical; location trigger; manual reset; Atk +18 ranged (1d4+1, dart); multiple targets (1d8 darts per target in a 3-m-by-3-m area); \skill{Search} DC 19; \skill{Disable Device} DC 25. Market Price: 18,000 cp.

\textbf{Moving Executioner Statue:} CR 5; mechanical; location trigger; automatic reset; hidden switch bypass (\skill{Search} DC 25); Atk +16 melee (1d12+8/$\times$3, greataxe); multiple targets (both arms attack); \skill{Search} DC 25; \skill{Disable Device} DC 18. Market Price: 22,500 cp.

\textbf{Phantasmal Killer Trap:} CR 5; magic device; proximity trigger (\spell{alarm} covering the entire room); automatic reset; spell effect (\spell{phantasmal killer}, 7th-level wizard, DC 16 Will save for disbelief and DC 16 Fort save for partial effect); \skill{Search} DC 29; \skill{Disable Device} DC 29. Cost: 14,000 cp, 1,120 XP.

\textbf{Pit Trap:} CR 5; mechanical, location trigger; manual reset; DC 20 Reflex save avoids; 30 m deep (10d6, fall); \skill{Search} DC 20; \skill{Disable Device} DC 20. Market Price: 5,000 cp.

\textbf{Poison Wall Spikes:} CR 5; mechanical; location trigger; manual reset; Atk +16 melee (1d8+4 plus poison, spike); multiple targets (closest target in each of two adjacent 1.5 m squares); poison (Medium monstrous spider venom, DC 12 Fortitude save resists, 1d4 Str/1d4 Str); \skill{Search} DC 17; \skill{Disable Device} DC 21. Market Price: 12,650 cp.

\textbf{Spiked Pit Trap:} CR 5; mechanical; location trigger; manual reset; DC 25 Reflex save avoids; 12 m deep (4d6, fall); multiple targets (first target in each of two adjacent 1.5 m squares); pit spikes (Atk +10 melee, 1d4 spikes per target for 1d4+4 each); \skill{Search} DC 21; \skill{Disable Device} DC 20. Market Price: 13,500 cp.

\textbf{Spiked Pit Trap (24 m Deep):} CR 5; mechanical; location trigger, manual reset; DC 20 Reflex save avoids; 24 m deep (8d6, fall), pit spikes (Atk +10 melee, 1d4 spikes for 1d4+5 each); \skill{Search} DC 20; \skill{Disable Device} DC 20. Market Price: 5,000 cp.

\textbf{Ungol Dust Vapor Trap:} CR 5; mechanical; location trigger; manual reset; gas; multiple targets (all targets in a 3-m-by-3-m room); never miss; onset delay (2 rounds); poison (ungol dust, DC 15 Fortitude save resists, 1 Cha/1d6 Cha plus 1 Cha drain); \skill{Search} DC 20; \skill{Disable Device} DC 16. Market Price: 9,000 cp.

\subsubsection{CR 6 Traps}
\textbf{Built-to-Collapse Wall:} CR 6; mechanical; proximity trigger; no reset; Atk +20 melee (8d6, stone blocks); multiple targets (all targets in a 3-m-by-3-m area); \skill{Search} DC 14; \skill{Disable Device} DC 16. Market Price: 15,000 cp.

\textbf{Compacting Room:} CR 6; mechanical; timed trigger; automatic reset; hidden switch bypass (\skill{Search} DC 25); walls move together (12d6, crush); multiple targets (all targets in a 3-m by 3-m room); never miss; onset delay (4 rounds); \skill{Search} DC 20; \skill{Disable Device} DC 22. Market Price: 25,200 cp.

\textbf{Flame Strike Trap:} CR 6; magic device; proximity trigger (\spell{detect magic}); automatic reset; spell effect (\spell{flame strike}, 9th-level cleric, 9d6 fire, DC 17 Reflex save half damage); \skill{Search} DC 30; \skill{Disable Device} DC 30. Cost: 22,750 cp, 1,820 XP.

\textbf{Fusillade of Spears:} CR 6; mechanical; proximity trigger; repair reset; Atk +21 ranged (1d8, spear); multiple targets (1d6 spears per target in a 3 m-by-3-m area); \skill{Search} DC 26; \skill{Disable Device} DC 20. Market Price: 31,200 cp.

\textbf{Glyph of Warding (Blast):} CR 6; spell; spell trigger; no reset; spell effect (\spell{glyph of warding} [blast], 16th-level cleric, 8d8 sonic, DC 14 Reflex save half damage); multiple targets (all targets within 1.5 m); \skill{Search} DC 28; \skill{Disable Device} DC 28. Cost: 680 cp to hire NPC spellcaster.

\textbf{Lightning Bolt Trap:} CR 6; magic device; proximity trigger (\spell{alarm}); automatic reset; spell effect (\spell{lightning bolt}, 10th-level wizard, 10d6 electricity, DC 14 Reflex save half damage); \skill{Search} DC 28; \skill{Disable Device} DC 28. Cost: 15,000 cp, 1,200 XP.

\textbf{Spiked Blocks from Ceiling:} CR 6; mechanical; location trigger; repair reset; Atk +20 melee (6d6, spikes); multiple targets (all targets in a 3-m-by-3-m area); \skill{Search} DC 24; \skill{Disable Device} DC 20. Market Price: 21,600 cp.

\textbf{Spiked Pit Trap (30 m Deep):} CR 6; mechanical; location trigger, manual reset; DC 20 Reflex save avoids; 30 m deep (10d6, fall); pit spikes (Atk +10 melee, 1d4 spikes per target for 1d4+5 each); \skill{Search} DC 20; \skill{Disable Device} DC 20. Market Price: 6,000 cp.

\textbf{Whirling Poison Blades:} CR 6; mechanical; timed trigger; automatic reset; hidden lock bypass (\skill{Search} DC 25, \skill{Open Lock} DC 30); Atk +10 melee (1d4+4/19-20 plus poison, dagger); poison (purple worm poison, DC 24 Fortitude save resists, 1d6 Str/2d6 Str); multiple targets (one target in each of three preselected 1.5 m squares); \skill{Search} DC 20; \skill{Disable Device} DC 20. Market Price: 30,200 cp.

\textbf{Wide-Mouth Pit Trap:} CR 6; mechanical; location trigger, manual reset; DC 25 Reflex save avoids; 12 m deep (4d6, fall); multiple targets (all targets within a 3-m-by-3-m area); \skill{Search} DC 26; \skill{Disable Device} DC 25. Market Price: 28,200 cp.

\textbf{Wyvern Arrow Trap:} CR 6; mechanical; proximity trigger; manual reset; Atk +14 ranged (1d8 plus poison, arrow); poison (wyvern poison, DC 17 Fortitude save resists, 2d6 Con/2d6 Con); \skill{Search} DC 20; \skill{Disable Device} DC 16. Market Price: 17,400 cp.

\subsubsection{CR 7 Traps}
\textbf{Acid Fog Trap:} CR 7; magic device; proximity trigger (\spell{alarm}); automatic reset; spell effect (\spell{acid fog}, 11th-level wizard, 2d6/round acid for 11 rounds); \skill{Search} DC 31; \skill{Disable Device} DC 31. Cost: 33,000 cp, 2,640 XP.

\textbf{Blade Barrier Trap:} CR 7; magic device; proximity trigger (\spell{alarm}); automatic reset; spell effect (\spell{blade barrier}, 11th-level cleric, 11d6 slashing, DC 19 Reflex save half damage); \skill{Search} DC 31; \skill{Disable Device} DC 31. Cost: 33,000 cp, 2,640 XP.

\textbf{Burnt Othur Vapor Trap:} CR 7; mechanical; location trigger; repair reset; gas; multiple targets (all targets in a 3-m-by-3-m room); never miss; onset delay (3 rounds); poison (burnt othur fumes, DC 18 Fortitude save resists, 1 Con drain/3d6 Con); \skill{Search} DC 21; \skill{Disable Device} DC 21. Market Price: 17,500 cp.

\textbf{Chain Lightning Trap:} CR 7; magic device; proximity trigger (\spell{alarm}); automatic reset; spell effect (\spell{chain lightning}, 11th-level wizard, 11d6 electricity to target nearest center of trigger area plus 5d6 electricity to each of up to eleven secondary targets, DC 19 Reflex save half damage); \skill{Search} DC 31; \skill{Disable Device} DC 31. Cost: 33,000 cp, 2,640 XP.

\textbf{Black Tentacles Trap:} CR 7; magic device; proximity trigger (\spell{alarm}); no reset; spell effect (\spell{black tentacles}, 7th-level wizard, 1d4+7 tentacles, Atk +7 melee [1d6+4, tentacle]); multiple targets (up to six tentacles per target in each of two adjacent 1.5 m squares); \skill{Search} DC 29; \skill{Disable Device} DC 29. Cost: 1,400 cp, 112 XP.

\textbf{Fusillade of Greenblood Oil Darts:} CR 7; mechanical; location trigger; manual reset; Atk +18 ranged (1d4+1 plus poison, dart); poison (greenblood oil, DC 13 Fortitude save resists, 1 Con/ 1d2 Con); multiple targets (1d8 darts per target in a 3-m-by-3-m area); \skill{Search} DC 25; \skill{Disable Device} DC 25. Market Price: 33,000 cp.

\textbf{Lock Covered in Dragon Bile:} CR 7; mechanical; touch trigger (attached); no reset; poison (dragon bile, DC 26 Fortitude save resists, 3d6 Str/0); \skill{Search} DC 27; \skill{Disable Device} DC 16. Market Price: 11,300 cp.

\textbf{Summon Monster VI Trap:} CR 7; magic device; proximity trigger (\spell{alarm}); no reset; spell effect (\spell{summon monster VI}, 11th-level wizard), \skill{Search} DC 31; \skill{Disable Device} DC 31. Cost: 3,300 cp, 264 XP.

\textbf{Water-Filled Room:} CR 7; mechanical; location trigger; manual reset; multiple targets (all targets in a 3-m-by-3-m room); never miss; onset delay (3 rounds); water; \skill{Search} DC 20; \skill{Disable Device} DC 25. Market Price: 21,000 cp.

\textbf{Well-Camouflaged Pit Trap:} CR 7; mechanical; location trigger; repair reset; DC 25 Reflex save avoids; 21 m deep (7d6, fall); multiple targets (first target in each of two adjacent 1.5 m squares); \skill{Search} DC 27; \skill{Disable Device} DC 18. Market Price: 24,500 cp.

\subsubsection{CR 8 Traps}
\textbf{Deathblade Wall Scythe:} CR 8; mechanical; touch trigger; manual reset; Atk +16 melee (2d4+8 plus poison, scythe); poison (deathblade, DC 20 Fortitude save resists, 1d6 Con/2d6 Con); \skill{Search} DC 24; \skill{Disable Device} DC 19. Market Price: 31,400 cp.

\textbf{Destruction Trap:} CR 8; magic device; touch trigger (alarm); automatic reset; spell effect (\spell{destruction}, 13th-level cleric, DC 20 Fortitude save for 10d6 damage); \skill{Search} DC 32; \skill{Disable Device} DC 32. Cost: 45,500 cp, 3,640 XP.

\textbf{Earthquake Trap:} CR 8; magic device; proximity trigger (\spell{alarm}); automatic reset; spell effect (\spell{earthquake}, 13th-level cleric, 65-ft. radius, DC 15 or 20 Reflex save, depending on terrain); \skill{Search} DC 32; \skill{Disable Device} DC 32. Cost: 45,500 cp, 3,640 XP.

\textbf{Insanity Mist Vapor Trap:} CR 8; mechanical; location trigger; repair reset; gas; never miss; onset delay (1 round); poison (insanity mist, DC 15 Fortitude save resists, 1d4 Wis/2d6 Wis); multiple targets (all targets in a 3-m-by-3-m room); \skill{Search} DC 25; \skill{Disable Device} DC 20. Market Price: 23,900 cp.

\textbf{Acid Arrow Trap:} CR 8; magic device; visual trigger (true seeing); automatic reset; multiple traps (two simultaneous acid arrow traps); Atk +9 ranged touch and +9 ranged touch; spell effect (\spell{acid arrow}, 18th-level wizard, 2d4 acid damage for 7 rounds); \skill{Search} DC 27; \skill{Disable Device} DC 27. Cost: 83,500 cp, 4,680 XP. Note: This trap is really two CR 6 acid arrow traps that fire simultaneously, using the same trigger and reset.

\textbf{Power Word Stun Trap:} CR 8; magic device; touch trigger; no reset; spell effect (\spell{power word stun}, 13th-level wizard), \skill{Search} DC 32; \skill{Disable Device} DC 32. Cost: 4,550 cp, 364 XP.

\textbf{Prismatic Spray Trap:} CR 8; magic device; proximity trigger (\spell{alarm}); automatic reset; spell effect (\spell{prismatic spray}, 13th-level wizard, DC 20 Reflex, Fortitude, or Will save, depending on effect); \skill{Search} DC 32; \skill{Disable Device} DC 32. Cost: 45,500 cp, 3,640 XP.

\textbf{Reverse Gravity Trap:} CR 8; magic device; proximity trigger (\spell{alarm}, 3-m area); automatic reset; spell effect (\spell{reverse gravity}, 13th-level wizard, 6d6 fall [upon hitting the ceiling of the 60-ft.- high room], then 6d6 fall [upon falling 18 m to the floor when the spell ends], DC 20 Reflex save avoids damage); \skill{Search} DC 32; \skill{Disable Device} DC 32. Cost: 45,500 cp, 3,640 XP.

\textbf{Well-Camouflaged Pit Trap:} CR 8; mechanical; location trigger; repair reset; DC 20 Reflex save avoids; 30 m deep (10d6, fall); \skill{Search} DC 27; \skill{Disable Device} DC 18. Market Price: 16,000 cp.

\textbf{Word of Chaos Trap:} CR 8; magic device; proximity trigger (\spell{detect law}); automatic reset; spell effect (\spell{word of chaos}, 13th-level cleric); \skill{Search} DC 32; \skill{Disable Device} DC 32. Cost: 46,000 cp, 3,680 XP.

\subsubsection{CR 9 Traps}
\textbf{Drawer Handle Smeared with Contact Poison:} CR 9; mechanical; touch trigger (attached); manual reset; poison (black lotus extract, DC 20 Fortitude save resists, 3d6 Con/3d6 Con); \skill{Search} DC 18; \skill{Disable Device} DC 26. Market Price: 21,600 cp.

\textbf{Dropping Ceiling:} CR 9; mechanical; location trigger; repair reset; ceiling moves down (12d6, crush); multiple targets (all targets in a 3-m-by-3-m room); never miss; onset delay (1 round); \skill{Search} DC 20; \skill{Disable Device} DC 16. Market Price: 12,600 cp.

\textbf{Incendiary Cloud Trap:} CR 9; magic device; proximity trigger (\spell{alarm}); automatic reset; spell effect (\spell{incendiary cloud}, 15th-level wizard, 4d6/round for 15 rounds, DC 22 Reflex save half damage); \skill{Search} DC 33; \skill{Disable Device} DC 33. Cost: 60,000 cp, 4,800 XP.

\textbf{Wide-Mouth Pit Trap:} CR 9; mechanical; location trigger; manual reset; DC 25 Reflex save avoids; 30 m deep (10d6, fall); multiple targets (all targets within a 3-m-by-3-m area); \skill{Search} DC 25; \skill{Disable Device} DC 25. Market Price: 40,500 cp.

\textbf{Wide-Mouth Spiked Pit with Poisoned Spikes:} CR 9; mechanical; location trigger; manual reset; hidden lock bypass (\skill{Search} DC 25, \skill{Open Lock} DC 30); DC 20 Reflex save avoids; 21 m deep (7d6, fall); multiple targets (all targets within a 3-m-by-3-m area); pit spikes (Atk +10 melee, 1d4 spikes per target for 1d4+5 plus poison each); poison (giant wasp poison, DC 14 Fortitude save resists, 1d6 Dex/1d6 Dex); \skill{Search} DC 20; \skill{Disable Device} DC 20. Market Price: 11,910 cp.

\subsubsection{CR 10 Traps}
\textbf{Crushing Room:} CR 10; mechanical; location trigger; automatic reset; walls move together (16d6, crush); multiple targets (all targets in a 3-m-by-3-m room); never miss; onset delay (2 rounds); \skill{Search} DC 22; \skill{Disable Device} DC 20. Market Price: 29,000 cp.

\textbf{Crushing Wall Trap:} CR 10; mechanical; location trigger; automatic reset; no attack roll required (18d6, crush); \skill{Search} DC 20; \skill{Disable Device} DC 25. Market Price: 25,000 cp.

\textbf{Energy Drain Trap:} CR 10; magic device; visual trigger (true seeing); automatic reset; Atk +8 ranged touch; spell effect (\spell{energy drain}, 17th-level wizard, 2d4 negative levels for 24 hours, DC 23 Fortitude save negates); \skill{Search} DC 34; \skill{Disable Device} DC 34. Cost: 124,000 cp, 7,920 XP.

\textbf{Forcecage and Summon Monster VII trap:} CR 10; magic device; proximity trigger (\spell{alarm}); automatic reset; multiple traps (one \spell{forcecage} trap and one \spell{summon monster VII} trap that summons a hamatula); spell effect (\spell{forcecage}, 13th-level wizard), spell effect (\spell{summon monster VII}, 13th-level wizard, hamatula); \skill{Search} DC 32; \skill{Disable Device} DC 32. Cost: 241,000 cp, 7,280 XP. Note: This trap is really one CR 8 trap that creates a \spell{forcecage} and a second CR 8 trap that summons a hamatula in the same area. If both succeed, the hamatula appears inside the \spell{forcecage}. These effects are independent of each other.

\textbf{Poisoned Spiked Pit Trap:} CR 10; mechanical; location trigger; manual reset; hidden lock bypass (\skill{Search} DC 25, \skill{Open Lock} DC 30); DC 20 Reflex save avoids; 50 ft. deep (5d6, fall); multiple targets (first target in each of two adjacent 1.5 m squares); pit spikes (Atk +10 melee, 1d4 spikes per target for 1d4+5 plus poison each); poison (purple worm poison, DC 24 Fortitude save resists, 1d6 Str/2d6 Str); \skill{Search} DC 16; \skill{Disable Device} DC 25. Market Price: 19,700 cp.

\textbf{Wail of the Banshee Trap:} CR 10; magic device; proximity trigger (\spell{alarm}); automatic reset; spell effect (\spell{wail of the banshee}, 17th-level wizard, DC 23 Fortitude save negates); multiple targets (up to 17 creatures); \skill{Search} DC 34; \skill{Disable Device} DC 34. Cost: 76,500 cp, 6,120 XP.

\subsection{Designing A Trap}
\textbf{Mechanical Traps:} Simply select the elements you want the trap to have and add up the adjustments to the trap's Challenge Rating that those elements require (see \tabref{CR Modifiers for Mechanical Traps}) to arrive at the trap's final CR. From the CR you can derive the DC of the \skill{Craft} (trapmaking) checks a character must make to construct the trap.


\textbf{Magic Traps:} As with mechanical traps, you don't have to do anything other than decide what elements you want and then determine the CR of the resulting trap (see \tabref{CR Modifiers for Magic Traps}). If a player character wants to design and construct a magic trap, he must have the \feat{Craft Wondrous Item} feat. In addition, he must be able to cast the spell or spells that the trap requires---or, failing that, he must be able to hire an NPC to cast the spells for him.

\Table{CR Modifiers for Mechanical Traps}{Xl}{
  \tableheader Feature
& \tableheader CR Modifier\\
\cmidrule[0pt]{1-2}
\multicolumn{2}{l}{\TableSubheader{\skill{Search} DC}}\\
15 or lower            & $-1$ \\
25--29                 & +1 \\
30 or higher           & +2 \\

\multicolumn{2}{l}{\TableSubheader{\skill{Disable Device} DC}}\\
15 or lower            & $-1$ \\
25--29                 & +1 \\
30 or higher           & +2 \\

\multicolumn{2}{l}{\TableSubheader{Reflex Save DC (Pit or Other Save-Dependent Trap)}}\\
15 or lower            & $-1$ \\
16--24                 & +0 \\
25--29                 & +1 \\
30 or higher           & +2 \\

\cmidrule[0pt]{1-2}
\multicolumn{2}{l}{\TableSubheader{Attack Bonus (Melee or Ranged Attack Trap)}}\\
+0 or lower            & -2 \\
+1 to +5               & $-1$ \\
+6 to +14              & +0 \\
+15 to +19             & +1 \\
+20 to +24             & +2 \\
+25 to +34             & +3 \\
+35 to +45             & +4 \\
+50 or higher          & +5 \\

\cmidrule[0pt]{1-2}
\multicolumn{2}{l}{\TableSubheader{Damage/Effect}}\\
Average damage         & +1/7 points\footnotemark[1] \\

% \cmidrule[0pt]{1-2}
\multicolumn{2}{l}{\TableSubheader{Miscellaneous Features}}\\
Alchemical device      & Level of spell mimicked \\
Liquid                 & +5 \\
Multiple target        & +1 (or 0 if never miss) \\
Multiple melee attack  & +2 \\
Onset delay 1 round    & +3 \\
Onset delay 2 rounds   & +2 \\
Onset delay 3 rounds   & +1 \\
Onset delay 4+ rounds  & $-1$ \\
Poison                 & See \tabref{CR Modifiers for Poisons} \\
Pit spikes             & +1 \\
Touch attack           & +1 \\
\TableNote{2}{1 Rounded to the nearest multiple of 7 (round up for an average that lies exactly between two numbers).}
}

\Table{CR Modifiers for Poisons}{XlXl}{
  \tableheader Poison
& \tableheader CR
& \tableheader Poison
& \tableheader CR\\

Antloid, soldier (contact)     & +3 & Mastyrial, black               & +4 \\
Antloid, soldier (injury)      & +3 & Mastyrial, desert              & +4 \\
Assassin bug                   & +1 & Medium spider venom            & +2 \\
Beetle, dragon\footnotemark[1] & +4 & Mulworm (contact)              & +2 \\
Black adder venom              & +2 & Mulworm (injury)               & +2 \\
Black lotus extract            & +8 & Nitharit                       & +3 \\
Bloodgrass (jungle)            & +3 & Pulp bee                       & +2 \\
Bloodgrass (plains)            & +1 & Purple worm poison             & +6 \\
Bloodroot                      & +1 & Random displays                & +2 \\
Blossomkiller                  & +2 & S'thag zagath                  & +3 \\
Blue whinnis                   & +1 & Sassone leaf residue           & +2 \\
Boneclaw, lesser               & +2 & Scarlet warden                 & +4 \\
Brain seed powder              & +3 & Scorpion, barbed               & +3 \\
Burnt othur fumes              & +5 & Scorpion, gold                 & +2 \\
Cha'thrang                     & +4 & Shadow essence                 & +4 \\
Cistern fiend                  & +6 & Silk wyrm                      & +2 \\
Deathblade                     & +5 & Silt serpent                   & +2 \\
Dragon bile                    & +6 & Silt serpent, giant            & +3 \\
Drik, high                     & +4 & Silt serpent, giant (immature) & +4 \\
Dune freak                     & +2 & Single-mindedness              & +2 \\
Dust glider                    & +2 & Small centipede poison         & +1 \\
Fordorran                      & +6 & Spider, crystal                & +3 \\
Gaj poison gas                 & +2 & Spider, dark defiler           & +1 \\
Giant wasp poison              & +4 & Spider, dark psion             & +2 \\
Greenblood oil                 & +1 & Spider, dark queen             & +5 \\
Hej-kin                        & +1 & Spider, dark warrior           & +2 \\
Insanity mist                  & +3 & T'chowb ichor                  & +2 \\
Jankx                          & +3 & Terinav root                   & +4 \\
Kank, soldier                  & +2 & Ungol dust                     & +2 \\
Large spider venom             & +4 & Wyvern poison                  & +5 \\
Malyss root paste              & +3 & Zik-trin'ta                    & +3 \\

\TableNote{4}{1 Only affects dray and creatures with the Dragon type.}\\
}


\Table{CR Modifiers for Magic Traps}{lX}{
  \tableheader Feature
& \tableheader CR Modifier\\
Highest-level spell & + Spell level OR\\
\cmidrule[0pt]{1-2}
                    & +1 per 7 points of average damage per round\footnotemark[1]\\
\TableNote{2}{1 Rounded to the nearest multiple of 7 (round up for an average that lies exactly between two numbers).}
}

\subsubsection{Challenge Rating Of A Trap}
To calculate the Challenge Rating of a trap, add all the CR modifiers (see the tables below) to the base CR for the trap type.

\textbf{Mechanical Trap:} The base CR for a mechanical trap is 0. If your final CR is 0 or lower, add features until you get a CR of 1 or higher.

\textbf{Magic Trap:} For a spell trap or magic device trap, the base CR is 1. The highest-level spell used modifies the CR (see Table: CR Modifiers for Magic Traps).

\textbf{Average Damage:} If a trap (either mechanical or magic) does hit point damage, calculate the average damage for a successful hit and round that value to the nearest multiple of 7. Use this value to adjust the Challenge Rating of the trap, as indicated on the tables below. Damage from poisons and pit spikes does not count toward this value, but damage from a high strength rating and extra damage from multiple attacks does.

For a magic trap, only one modifier applies to the CR---either the level of the highest-level spell used in the trap, or the average damage figure, whichever is larger.

\textbf{Multiple Traps:} If a trap is really two or more connected traps that affect approximately the same area, determine the CR of each one separately.

\textit{Multiple Dependent Traps:} If one trap depends on the success of the other (that is, you can avoid the second trap altogether by not falling victim to the first), they must be treated as separate traps.

\textit{Multiple Independent Traps:} If two or more traps act independently (that is, none depends on the success of another to activate), use their CRs to determine their combined Encounter Level as though they were monsters. The resulting Encounter Level is the CR for the combined traps.

\subsubsection{Mechanical Trap Cost}
The base cost of a mechanical trap is 1,000 cp. Apply all the modifiers from \tabref{Cost Modifiers for Mechanical Traps} for the various features you've added to the trap to get the modified base cost.

The final cost is equal to (modified base cost $\times$ Challenge Rating) + extra costs. The minimum cost for a mechanical trap is (CR $\times$ 100) cp.

After you've multiplied the modified base cost by the Challenge Rating, add the price of any alchemical items or poison you incorporated into the trap. If the trap uses one of these elements and has an automatic reset, multiply the poison or alchemical item cost by 20 to provide an adequate supply of doses.

\textbf{Multiple Traps:} If a trap is really two or more connected traps, determine the final cost of each separately, then add those values together. This holds for both multiple dependent and multiple independent traps (see the previous section).

\Table{Cost Modifiers for Mechanical Traps}{p{25mm}X}{
  \tableheader Feature
& \tableheader Cost Modifier\\

\cmidrule[0pt]{1-2}
\multicolumn{2}{l}{\TableSubheader{Trigger Type}}\\
Location                                  & --- \\
Proximity                                 & +1,000 cp\\
Touch                                     & --- \\
Touch (attached)                          & $-100$ cp\\
Timed                                     & +1,000 cp\\

\multicolumn{2}{l}{\TableSubheader{Reset Type}}\\
No reset                                  & $-500$ cp\\
Repair                                    & $-200$ cp\\
Manual                                    & --- \\
Automatic                                 & +500 cp (or 0 if trap has timed trigger)\\

\cmidrule[0pt]{1-2}
\multicolumn{2}{l}{\TableSubheader{Bypass Type}}\\
Lock                                      & +100 cp (\skill{Open Lock} DC 30)\\
Hidden switch                             & +200 cp (\skill{Search} DC 25)\\
Hidden lock                               & +300 cp (\skill{Open Lock} DC 30, \skill{Search} DC 25)\\

% \cmidrule[0pt]{1-2}
\multicolumn{2}{l}{\TableSubheader{\skill{Search} DC}}\\
19 or lower                               & $-100$ cp $\times$ (20 $-$ DC)\\
20                                        & --- \\
21 or higher                              & +200 cp $\times$ (DC $-$ 20)\\

% \cmidrule[0pt]{1-2}
\multicolumn{2}{l}{\TableSubheader{\skill{Disable Device} DC}}\\
19 or lower                               & $-100$ cp $\times$ (20 $-$ DC)\\
20                                        & --- \\
21 or higher                              & +200 cp $\times$ (DC $-$ 20)\\

% \cmidrule[0pt]{1-2}
\multicolumn{2}{l}{\TableSubheader{Reflex Save DC (Pit or Other Save-Dependent Trap)}}\\
19 or lower                               & $-100$ cp $\times$ (20 $-$ DC)\\
20                                        & --- \\
21 or higher                              & +300 cp $\times$ (DC $-$ 20)\\

% \cmidrule[0pt]{1-2}
\multicolumn{2}{l}{\TableSubheader{Attack Bonus (Melee or Ranged Attack Trap)}}\\
+9 or lower                               & $-100$ cp $\times$ (10 $-$ bonus)\\
+10                                       & --- \\
+11 or higher                             & +200 cp $\times$ (bonus $-$ 10)\\

\multicolumn{2}{l}{\TableSubheader{Damage Bonus}}\\
High strength rating (ranged attack trap) & +100 cp $\times$ bonus (max +4)\\
High Strength bonus (melee attack trap)   & +100 cp $\times$ bonus (max +8)\\

\cmidrule[0pt]{1-2}
\multicolumn{2}{l}{\TableSubheader{Miscellaneous Features}}\\
Never miss                                & +1,000 cp\\
Poison                                    & Cost of poison\footnotemark[1]\\
Alchemical item                           & Cost of item\footnotemark[1]\\


\TableNote{2}{1 Multiply cost by 20 if trap features automatic reset.}\\
}

\subsubsection{Magic Device Trap Cost}
Building a magic device trap involves the expenditure of experience points as well as gold pieces, and requires the services of a spellcaster. \tabref{Cost Modifiers for Magic Device Traps} summarizes the cost information for magic device traps. If the trap uses more than one spell (for instance, a sound or visual trigger spell in addition to the main spell effect), the builder must pay for them all (except alarm, which is free unless it must be cast by an NPC; see below).

The costs derived from \tabref{Cost Modifiers for Magic Device Traps} assume that the builder is casting the necessary spells himself (or perhaps some other PC is providing the spells for free). If an NPC spellcaster must be hired to cast them those costs must be factored in as well.

A magic device trap takes one day to construct per 500 cp of its cost.

\Table{Cost Modifiers for Magic Device Traps}{lX}{
  \tableheader Feature
& \tableheader Cost Modifier\\

\spell{Alarm} spell used in trigger & ---  \\

% \cmidrule[0pt]{1-2}
\multicolumn{2}{l}{\TableSubheader{One-Shot Trap}}\\

Each spell used in trap & +50 cp $\times$ caster level $\times$ spell level, +4 XP $\times$ caster level $\times$ spell level\\
Material components     & + Cost of all material components\\
XP components           & + Total of XP components $\times$ 5 cp\\

% \cmidrule[0pt]{1-2}
\multicolumn{2}{l}{\TableSubheader{Automatic Reset Trap}}\\
Each spell used in trap & +500 cp $\times$ caster level $\times$ spell level, +40 XP $\times$ caster level $\times$ spell level\\
Material components     & + Cost of all material components $\times$ 100 cp\\
XP components           & + Total of XP components $\times$ 500 cp\\
}

\subsubsection{Spell Trap Cost}
A spell trap has a cost only if the builder must hire an NPC spellcaster to cast it.

\subsubsection{Craft DCs For Mechanical Traps}
Once you know the Challenge Rating of a trap determine the \skill{Craft} (trapmaking) DC by referring to the table and the modifiers given below.

\Table{}{XX}{
  \tableheader Trap CR
& \tableheader Base \skill{Craft} (Trapmaking) DC \\
 1--3  & 20 \\
 4--6  & 25 \\
 7--10 & 30 \\
11--15 & 35 \\
16--20 & 40 \\
}
\Table{}{XX}{
  \tableheader Additional Components
& \tableheader Modifier to \skill{Craft} (Trapmaking) DC \\
Proximity trigger & +5 \\
Automatic reset   & +5 \\
}

\textbf{Making the Checks:} To determine how much progress a character makes on building a trap each week, that character makes a \skill{Craft} (trapmaking) check. See the \skill{Craft} skill description for details on \skill{Craft} checks and the circumstances that can affect them.


