\section{Armor}

While Athasian characters use all the varieties of armor, the armor they use incorporates materials commonly found in the world around them. Though most of the armors are made using various parts of common Athasian animals, the armor construction process makes use of several different reinforcement methods developed over time. Many of the armors are highly composite, made using the pieces of several different animals---no two suits of armor look quite alike. Through the use of hardening resins, shaped chitin and stiff leather backings, Athasian armorers can craft remarkably durable armors from the material at hand.

Thousands of years of tortuous heat have lead Athasian armorers to develop ingenious air ventilation and air circulation methods. This allows medium and heavy armors to be worn in the Athasian heat.


\begin{table*}[t!]
\caption{\label{tab:Armor and Shields}Armor and Shields}
\rowcolors{0}{TableColor}{}
\scriptsize
\begin{tabularx}{\textwidth}{X z{9mm}z{11mm}z{7mm}z{9mm}z{8mm}z{9mm}z{7mm}z{7mm}z{7mm}z{11mm}z{11mm}z{11mm}}
	\rowcolor{white}
& \multicolumn{2}{c}{\tableheader Cost}
& \multirow[b]{2}{7mm}{\centering \tableheader AC Bonus}
& \multirow[b]{2}{1cm}{\centering \tableheader Max. Dex Bonus}
& \multirow[b]{2}{8mm}{\centering \tableheader Armor Penalty}
& \multirow[b]{2}{9mm}{\centering \tableheader Failure Chance}
& \multicolumn{3}{c}{\tableheader Speed}
& \multicolumn{3}{c}{\tableheader Weight}\\
\cmidrule[0.5pt]{2-3}
\cmidrule[0.5pt]{8-13}

\footnotesize \tableheader Armor & \tableheader S/M & \tableheader L & & & & & \tableheader 6 m & \tableheader 9 m & \tableheader 12 m & \tableheader S & \tableheader M & \tableheader L\\

\multicolumn{13}{l}{\textit{Light Armor}}\\
Padded & 5 cp & 10 cp & +1 & +8 & +0 & 5\% & 6 m & 9 m & 12 m & 2.5 kg & 5 kg & 10 kg\\
Leather & 10 cp & 20 cp & +2 & +6 & +0 & 10\% & 6 m & 9 m & 12 m & 3.75 kg & 7.5 kg & 15 kg\\
Studded leather & 25 cp & 50 cp & +3 & +5 & $-1$ & 15\% & 6 m & 9 m & 12 m & 5 kg & 10 kg & 20 kg\\
Gladiator, light & 50 cp & 100 cp & +3 & +6 & $-1$ & 15\% & 6 m & 9 m & 12 m & 3 kg & 6 kg & 12 kg\\
Caravan & 75 cp & 150 cp & +4 & +3 & $-2$ & 20\% & 6 m & 9 m & 12 m & 4 kg & 8 kg & 16 kg\\
Chitin & 100 cp & 200 cp & +4 & +4 & $-2$ & 20\% & 6 m & 9 m & 12 m & 6.25 kg & 12.5 kg & 25 kg\\

\multicolumn{13}{l}{\textit{Medium Armor}}\\
Hide & 15 cp & 30 cp & +3 & +4 & $-3$ & 20\% & 4.5 m & 6 m & 9 m & 6.25 kg & 12.5 kg & 25 kg\\
Scale mail & 50 cp & 100 cp & +4 & +3 & $-4$ & 25\% & 4.5 m & 6 m & 9 m & 7.5 kg & 15 kg & 30 kg\\
Gladiator, medium & 100 cp & 200 cp & +4 & +5 & $-3$ & 20\% & 4.5 m & 6 m & 9 m & 4 kg & 8 kg & 16 kg\\
Shell & 150 cp & 300 cp & +5 & +2 & $-5$ & 30\% & 4.5 m & 6 m & 9 m & 10 kg & 20 kg & 40 kg\\
Breastplate & 200 cp & 400 cp & +5 & +3 & $-4$ & 25\% & 4.5 m & 6 m & 9 m & 7.5 kg & 15 kg & 30 kg\\

\multicolumn{13}{l}{\textit{Heavy armor}\footnotemark[1]}\\
Chitin warsuit & 165 cp & 330 cp & +5 & +1 & $-5$ & 40\% & 4.5 m & 6 m & 9 m & 6.25 kg & 12.5 kg & 25 kg\\
Splint mail & 200 cp & 400 cp & +6 & +0 & $-7$ & 40\% & 4.5 m & 6 m & 9 m & 11.25 kg & 22.5 kg & 45 kg\\
Banded mail & 250 cp & 500 cp & +6 & +1 & $-6$ & 35\% & 4.5 m & 6 m & 9 m & 8.75 kg & 17.5 kg & 35 kg\\
Tyrian warsuit & 410 cp & 810 cp & +6 & +1 & $-5$ & 30\% & 4.5 m & 6 m & 9 m & 10 kg & 20 kg & 40 kg\\
Half-plate & 6 gp & 12 gp & +7 & +0 & $-7$ & 40\% & 4.5 m & 6 m & 9 m & 12.5 kg & 25 kg & 50 kg\\
Full plate & 15 gp & 30 gp & +8 & +1 & $-6$ & 35\% & 4.5 m & 6 m & 9 m & 12.5 kg & 25 kg & 50 kg\\

\multicolumn{13}{l}{\textit{Shields}}\\
Buckler & 15 cp & 30 cp & +1 && $-1$ & 5\% &&&& 1.25 kg & 2.5 kg & 5 kg\\
Shield, light wooden & 3 cp & 6 cp & +1 && $-1$ & 5\% &&&& 1.25 kg & 2.5 kg & 5 kg\\
Shield, light steel & 9 gp & 18 gp & +1 && $-1$ & 5\% &&&& 1.5 kg & 3 kg & 6 kg\\
Shield, heavy wooden & 7 cp & 14 cp & +2 && $-2$ & 15\% &&&& 2.5 kg & 5 kg & 10 kg\\
Shield, heavy steel & 20 gp & 40 gp & +2 && $-2$ & 15\% &&&& 3.75 kg & 7.5 kg & 15 kg\\
Shield, long & 20 cp & 40 cp & +3 && $-4$ & 20\% &&&& 2.5 kg & 5 kg & 10 kg\\
Shield, tower & 30 cp & 60 cp & +4\footnotemark[2] & +2 & $-10$ & 50\% &&&& 11.25 kg & 22.5 kg & 
45 kg\\

\multicolumn{13}{l}{\textit{Extras}}\\
Armor spikes & +50 cp & +100 cp &&&& &&&& +2.5 kg & +5 kg & +10 kg\\
Gauntlet, locked & 8 cp & 16 cp &&& $\star$ & $\star$ &&&& +1.25 kg & +2.5 kg & +5 kg\\
Shield spikes & +10 cp & +20 cp &&&& &&&& +1.25 kg & +2.5 kg & +5 kg\\
\rowcolor{white}
\multicolumn{13}{l}{1 When running in heavy armor, you move only triple your speed, not quadruple.}\\
\rowcolor{white}
\multicolumn{13}{l}{2 A tower shield can instead grant you cover.}\\
\rowcolor{white}
\multicolumn{13}{l}{$\star$ Hand not free to cast spells or employ skills.}\\
\end{tabularx}
\end{table*}

\subsection{Armor Qualities}
To wear heavier armor effectively, a character can select the Armor Proficiency feats, but most classes are automatically proficient with the armors that work best for them.

Armor and shields can take damage from some types of attacks.

Here is the format for armor entries (given as column headings on \tabref{Armor and Shields}, below).

\textbf{Cost:} The cost of the armor for Small, Medium, or Large humanoid creatures. See Armor for Unusual Creatures, below, for armor prices for other creatures.

\textbf{Armor/Shield Bonus:} Each armor grants an armor bonus to AC, while shields grant a shield bonus to AC. The armor bonus from a suit of armor doesn't stack with other effects or items that grant an armor bonus. Similarly, the shield bonus from a shield doesn't stack with other effects that grant a shield bonus.

\textbf{Maximum Dex Bonus:} This number is the maximum Dexterity bonus to AC that this type of armor allows. Heavier armors limit mobility, reducing the wearer's ability to dodge blows. This restriction doesn't affect any other Dexterity-related abilities.

Even if a character's Dexterity bonus to AC drops to 0 because of armor, this situation does not count as losing a Dexterity bonus to AC.

Your character's encumbrance (the amount of gear he or she carries) may also restrict the maximum Dexterity bonus that can be applied to his or her Armor Class.

\textit{Shields:} Shields do not affect a character's maximum Dexterity bonus.

\textbf{Armor Check Penalty:} Any armor heavier than leather hurts a character's ability to use some skills. An armor check penalty number is the penalty that applies to \skill{Balance}, \skill{Climb}, \skill{Escape Artist}, \skill{Hide}, \skill{Jump}, \skill{Move Silently}, \skill{Sleight of Hand}, and \skill{Tumble} checks by a character wearing a certain kind of armor. Double the normal armor check penalty is applied to Swim checks. A character's encumbrance (the amount of gear carried, including armor) may also apply an armor check penalty.

\textit{Shields:} If a character is wearing armor and using a shield, both armor check penalties apply.

\textit{Nonproficient with Armor Worn:} A character who wears armor and/or uses a shield with which he or she is not proficient takes the armor's (and/or shield's) armor check penalty on attack rolls and on all Strength-based and Dexterity-based ability and skill checks. The penalty for nonproficiency with armor stacks with the penalty for nonproficiency with shields.

\textit{Sleeping in Armor:} A character who sleeps in medium or heavy armor is automatically fatigued the next day. He or she takes a -2 penalty on Strength and Dexterity and can't charge or run. Sleeping in light armor does not cause fatigue.

\textbf{Arcane Spell Failure:} Armor interferes with the gestures that a spellcaster must make to cast an arcane spell that has a somatic component. Arcane spellcasters face the possibility of arcane spell failure if they're wearing armor. Bards can wear light armor without incurring any arcane spell failure chance for their bard spells.

\textit{Casting an Arcane Spell in Armor:} A character who casts an arcane spell while wearing armor must usually make an arcane spell failure roll. The number in the Arcane Spell Failure Chance column on Table: Armor and Shields is the chance that the spell fails and is ruined. If the spell lacks a somatic component, however, it can be cast with no chance of arcane spell failure.

\textit{Shields:} If a character is wearing armor and using a shield, add the two numbers together to get a single arcane spell failure chance.

\textbf{Speed:} Medium or heavy armor slows the wearer down. The number on \tabref{Armor and Shields} is the character's speed while wearing the armor. When running in heavy armor, you move only triple your speed, not quadruple.

Elves, half-giants, and thri-kreen have an unencumbered speed of 12 meters. Humans, half-elves, muls, and pterrans have an unencumbered speed of 9 meters.

They use the first column. Aarakocras, dwarves, and halflings have an unencumbered speed of 6 meters. They use the second column. Remember, however, that a dwarf's land speed remains 6 meters even in medium or heavy armor or when carrying a medium or heavy load.

\textit{Shields:} Shields do not affect a character's speed.

\textbf{Weight:} This column gives the weight of the armor sized for a Medium wearer. Armor fitted for Small characters weighs half as much, and armor for Large characters weighs twice as much.

\subsection{Getting Into And Out Of Armor}
The time required to don armor depends on its type; see \tabref{Donning Armor}.

\textbf{Don:} This column tells how long it takes a character to put the armor on. (One minute is 10 rounds.) Readying (strapping on) a shield is only a move action.

\textbf{Don Hastily:} This column tells how long it takes to put the armor on in a hurry. The armor check penalty and armor bonus for hastily donned armor are each 1 point worse than normal.

\textbf{Remove:} This column tells how long it takes to get the armor off. Loosing a shield (removing it from the arm and dropping it) is only a move action.

\TransparentTable{Donning Armor}{l CCC}{
\tableheader Armor Type & \tableheader Don & \tableheader Don Hastily & \tableheader Remove\\
\rowcolor{TableColor}
Shield (any) & 1 move action & n/a & 1 move action\\
Padded & \multirow{6}{*}{1 minute} & \multirow{6}{*}{5 rounds} & \multirow{6}{*}{1 minute\footnotemark[1]}\\
Leather &&&\\
Hide &&&\\
Studded leather &&&\\
Chitin &&&\\
Chitin warsuit &&&\\
\rowcolor{TableColor}
Breastplate &&&\\
\rowcolor{TableColor}
Caravan &&&\\
\rowcolor{TableColor}
Light gladiator &&&\\
\rowcolor{TableColor}
Medium gladiator &&&\\
\rowcolor{TableColor}
Scale mail &&&\\
\rowcolor{TableColor}
Shell &&&\\
\rowcolor{TableColor}
Banded mail &&&\\
\rowcolor{TableColor}
Splint mail & \multirow{-8}{*}{4 minutes\footnotemark[1]} & \multirow{-8}{*}{1 minute} & \multirow{-8}{*}{1 minute\footnotemark[1]}\\
Half-plate & \multirow{3}{*}{4 minutes\footnotemark[2]} & \multirow{3}{*}{4 minutes\footnotemark[1]} & \multirow{3}{1.5cm}{\centering 1d4+1 minutes\footnotemark[1]}\\
Full plate &&&\\
Tyrian warsuit &&&\\
}
\TransparentTable{}{l X}{1 & If the character has some help, cut this time in half. A single character doing nothing else can help one or two adjacent characters. Two characters can't help each other don armor at the same time.\\
2 & The wearer must have help to don this armor. Without help, it can be donned only hastily.
}

\subsection{Armor For Unusual Creatures}
Armor and shields for unusually big creatures, unusually little creatures, and nonhumanoid creatures have different costs and weights from those given on \tabref{Armor and Shields}. Refer to the appropriate line on the table below and apply the multipliers to cost and weight for the armor type in question.

For creatures of size Tiny or smaller, divide armor bonus by 2.

\Table{}{l CCCC}{
 & \multicolumn{2}{c}{\tableheader Humanoid} & \multicolumn{2}{c}{\tableheader Nonhumanoid} \\
\cmidrule[0.5pt]{2-5}
\tableheader Size & \tableheader Cost & \tableheader Weight & \tableheader Cost & \tableheader Weight\\
Tiny or smaller & $\times$\onehalf & $\times$1/10 & $\times$1 & $\times$1/10\\
Small & $\times$1 & $\times$\onehalf & $\times$2 & $\times$\onehalf\\
Medium & $\times$1 & $\times$1 & $\times$2 & $\times$1\\
Large & $\times$2 & $\times$2 & $\times$4 & $\times$2\\
Huge & $\times$4 & $\times$5 & $\times$8 & $\times$5\\
Gargantuan & $\times$8 & $\times$8 & $\times$16 & $\times$8\\
Colossal & $\times$16 & $\times$12 & $\times$32 & $\times$12\\
}

\subsection{Armor Descriptions}
Any special benefits or accessories to the types of armor found on \tabref{Armor and Shields} are described below.

\textbf{Armor Spikes:} You can have spikes added to your armor, which allow you to deal extra piercing damage on a successful grapple attack. The spikes count as a martial weapon. If you are not proficient with them, you take a -4 penalty on grapple checks when you try to use them. You can also make a regular melee attack (or off-hand attack) with the spikes, and they count as a light weapon in this case. (You can't also make an attack with armor spikes if you have already made an attack with another off-hand weapon, and vice versa.)\\An enhancement bonus to a suit of armor does not improve the spikes' effectiveness, but the spikes can be made into magic weapons in their own right.

\textbf{Banded Mail:} The suit includes gauntlets.

\textbf{Breastplate:} It comes with a helmet and greaves.

\textbf{Buckler:} This small metal shield is worn strapped to your forearm. You can use a bow or crossbow without penalty while carrying it. You can also use your shield arm to wield a weapon (whether you are using an off-hand weapon or using your off hand to help wield a two-handed weapon), but you take a -1 penalty on attack rolls while doing so. This penalty stacks with those that may apply for fighting with your off hand and for fighting with two weapons. In any case, if you use a weapon in your off hand, you don't get the buckler's AC bonus for the rest of the round.

You can't bash someone with a buckler.

\textbf{Caravan Armor:} This suit of armor is a combination of several different materials. Thick chitin bracers provide efficient protection to the forearms, while thick leather protects the shins and knees. A leather kilt and and shirt of thick cord layers protect the body and provide sufficient cooling. This armor is so named because it's mostly used by caravan guards who need decent protection while not being slowed down by their armor. Light caravan armor comes with a turban made of thick cord.

This armor doesn't provide full-body protection and thus the wearer is more prone to critical hits; the AC against rolls to confirm a critical hit is reduced by 1.

\textbf{Chitin Warsuit:} This suit of armor comes with padded armor, which is worn beneath the actual armor, to prevent abrasions. A long shell shirt covers the torso and the waist, chitin sleeves over both arms and shoulders and end in chitin gauntlets, long chitin pants cover the legs, and a bone or chitin helmet, usually made of a creature's skull or head exoskeleton, covers the head. This armor offers good protection, but brings the usual problems with heat accumulation.

\textbf{Chitin:} This armor is skillfully made by interlocking hexagonal bits of chitin (usually carved from a kank's carapace).

A chitin armor comes with a chitin cap.

\textbf{Full Plate:} The suit includes gauntlets, heavy leather boots, a visored helmet, and a thick layer of padding that is worn underneath the armor. Each suit of full plate must be individually fitted to its owner by a master armorsmith, although a captured suit can be resized to fit a new owner at a cost of 200 to 800 (2d4$\times$100) ceramic pieces.

\textbf{Gauntlet, Locked:} This armored gauntlet has small chains and braces that allow the wearer to attach a weapon to the gauntlet so that it cannot be dropped easily. It provides a +10 bonus on any roll made to keep from being disarmed in combat. Removing a weapon from a locked gauntlet or attaching a weapon to a locked gauntlet is a full-round action that provokes attacks of opportunity.

The price given is for a single locked gauntlet. The weight given applies only if you're wearing a breastplate, light armor, or no armor. Otherwise, the locked gauntlet replaces a gauntlet you already have as part of the armor.

While the gauntlet is locked, you can't use the hand wearing it for casting spells or employing skills. (You can still cast spells with somatic components, provided that your other hand is free.)

Like a normal gauntlet, a locked gauntlet lets you deal lethal damage rather than nonlethal damage with an unarmed strike.

\textbf{Half-Plate:} The suit includes gauntlets.

\textbf{Light Gladiator Armor:} This suit of armor combines leather and bone to provide the gladiator with decent protection and minimal hindrance. Thick leather shinpads provide leg protection whithout hampering movement, while a breastplate of bone and leather skirt or loincoth protects the  gladiator's torso. A bone helmet protect the head and face, and a cuff of thick leather protects the gladiator's weapon hand. Gladiators that rely on high maneuverability prefer this kind of of armor; masterwork suits are highly desired and respected.

This armor's lightweight and area-specific coverage provides many openings for critical hits; thus, the AC against rolls to confirm critical hits is reduced by 2.

\textbf{Medium Gladiator Armor:} This suit of armor combines leather and chitin to provide the gladiator with good protection without hampering his freedom of movement too much. A vambrace made of chitin covers the gladiator's weapon arm and is held in place by a leather corselet, while a shoulder plate of chitin covers the off-hand shoulder. A thick leather skirt protects the gladiator's haunch and chitin shinpads protect his tibia. This suit of picemeal armor comes with a chitin helm that usually resembles a beast's head.

This armor provides many openings for critical hits; therefore, the AC against rolls to confirm critical hits is reduced by 2.

\textbf{Scale Mail:} Scale mail is usually made from the scales of an erdlu, inix or other naturally scaled creatures.

The suit includes gauntlets.

\textbf{Shell:} Shell armor is made by weaving giant's hair around the shells of various small creatures such as an aprig.

The suit includes gauntlets.

\textbf{Shield, Heavy:} You strap a shield to your forearm and grip it with your hand. A heavy shield is so heavy that you can't use your shield hand for anything else.

\textit{Wooden or Steel:} Wooden and steel shields offer the same basic protection, though they respond differently to special attacks.

\textit{Shield Bash Attacks:} You can bash an opponent with a heavy shield, using it as an off-hand weapon. See \tabref{Martial Weapons} for the damage dealt by a shield bash. Used this way, a heavy shield is a martial bludgeoning weapon. For the purpose of penalties on attack rolls, treat a heavy shield as a one-handed weapon. If you use your shield as a weapon, you lose its AC bonus until your next action (usually until the next round). An enhancement bonus on a shield does not improve the effectiveness of a shield bash made with it, but the shield can be made into a magic weapon in its own right.

\textbf{Shield, Light:} You strap a shield to your forearm and grip it with your hand. A light shield's weight lets you carry other items in that hand, although you cannot use weapons with it.

\textit{Wooden or Steel:} Wooden and steel shields offer the same basic protection, though they respond differently to special attacks.

\textit{Shield Bash Attacks:} You can bash an opponent with a light shield, using it as an off-hand weapon. See \tabref{Martial Weapons} for the damage dealt by a shield bash. Used this way, a light shield is a martial bludgeoning weapon. For the purpose of penalties on attack rolls, treat a light shield as a light weapon. If you use your shield as a weapon, you lose its AC bonus until your next action (usually until the next round). An enhancement bonus on a shield does not improve the effectiveness of a shield bash made with it, but the shield can be made into a magic weapon in its own right.

\textbf{Shield, Long:} This is a slim, two-handed shield commonly used by the kreen races of the northern kreen Empire; it is extremely rare to find a long shield in the hands of a nomadic kreen of the Tablelands, although they are occasionally spotted in the arena. Kreen usually hold the long shield with two arms from the same side. Long shields are made of bone, chitin, hide, or wood.

You need two hands to use a long shield. Two handed humanoids who use a long shield can do so by using it horizontally, but by doing so you cannot wield a weapon.

\textit{Shield Bash Attacks:} You can bash an opponent with a long shield, using it as an off-hand weapon. See \tabref{Martial Weapons} in the Player's Handbook for the damage dealt by a shield bash. Used this way, a long shield is a martial bludgeoning weapon. For the purpose of penalties on attack rolls, treat a long shield as a two-handed weapon. If you use your shield as a weapon, you lose its AC bonus until your next action (usually until the next round). An enhancement bonus on a shield does not improve the effectiveness of a shield bash made with it, but the shield can be made into a magic weapon in its own right.

\textbf{Shield, Tower:} This massive wooden shield is nearly as tall as you are. In most situations, it provides the indicated shield bonus to your AC. However, you can instead use it as total cover, though you must give up your attacks to do so. The shield does not, however, provide cover against targeted spells; a spellcaster can cast a spell on you by targeting the shield you are holding. You cannot bash with a tower shield, nor can you use your shield hand for anything else.

When employing a tower shield in combat, you take a -2 penalty on attack rolls because of the shield's encumbrance.

\textbf{Shield Spikes:} When added to your shield, these spikes turn it into a martial piercing weapon that increases the damage dealt by a shield bash as if the shield were designed for a creature one size category larger than you. You can't put spikes on a buckler or a tower shield. Otherwise, attacking with a spiked shield is like making a shield bash attack.

An enhancement bonus on a spiked shield does not improve the effectiveness of a shield bash made with it, but a spiked shield can be made into a magic weapon in its own right.

\textbf{Splint Mail:} The suit includes gauntlets.

\textbf{Tyrian Warsuit:} This armor combines metal and chitin. A chitin breatplate covers the front, back, shoulders and upper arms, while a long shell skirt protects the haunch. Metal shinpads, padded on the inside, are worn over leather boots, to avoid burns, while metal gauntlets, also padded on the inside, are worn over leather cuffs. This suit of armor comes with a full chitin helmet.

\subsection{Masterwork Armor}
Just as with weapons, you can purchase or craft masterwork versions of armor or shields. Such a well-made item functions like the normal version, except that its armor check penalty is lessened by 1.

A masterwork suit of armor or shield costs an extra 150 gp over and above the normal cost for that type of armor or shield.

The masterwork quality of a suit of armor or shield never provides a bonus on attack or damage rolls, even if the armor or shield is used as a weapon.

All magic armors and shields are automatically considered to be of masterwork quality.

You can't add the masterwork quality to armor or a shield after it is created; it must be crafted as a masterwork item.