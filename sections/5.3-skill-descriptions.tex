\section{Skill Descriptions}
This section describes each skill, including common uses and typical modifiers. Characters can sometimes use skills for purposes other than those noted here.

Here is the format for skill descriptions.

\subsection{Skill Name}
The skill name line includes (in addition to the name of the skill) the following information.

\textbf{Key Ability:} The abbreviation of the ability whose modifier applies to the skill check. Exception: Speak Language and Literacy have ``None'' as its key ability because the use of those skills does not require a check.

\textbf{Trained Only:} If this notation is included in the skill name line, you must have at least 1 rank in the skill to use it. If it is omitted, the skill can be used untrained (with a rank of 0). If any special notes apply to trained or untrained use, they are covered in the Untrained section (see below).

\textbf{Armor Check Penalty:} If this notation is included in the skill name line, an armor check penalty applies (when appropriate) to checks using this skill. If this entry is absent, an armor check penalty does not apply.

The skill name line is followed by a general description of what using the skill represents. After the description are a few other types of information:

\textbf{Check:} What a character (``you'' in the skill description) can do with a successful skill check and the check's DC.

\textbf{Action:} The type of action using the skill requires, or the amount of time required for a check.

\textbf{Try Again:} Any conditions that apply to successive attempts to use the skill successfully. If the skill doesn't allow you to attempt the same task more than once, or if failure carries an inherent penalty (such as with the Climb skill), you can't take 20. If this paragraph is omitted, the skill can be retried without any inherent penalty, other than the additional time required.

\textbf{Special:} Any extra facts that apply to the skill, such as special effects deriving from its use or bonuses that certain characters receive because of class, feat choices, or race.

\textbf{Synergy:} Some skills grant a bonus to the use of one or more other skills because of a synergistic effect. This entry, when present, indicates what bonuses this skill may grant or receive because of such synergies. See \tabref{Skill Synergies} for a complete list of bonuses granted by synergy between skills (or between a skill and a class feature).

\textbf{Restriction:} The full utility of certain skills is restricted to characters of certain classes or characters who possess certain feats. This entry indicates whether any such restrictions exist for the skill.

\textbf{Untrained:} This entry indicates what a character without at least 1 rank in the skill can do with it. If this entry doesn't appear, it means that the skill functions normally for untrained characters (if it can be used untrained) or that an untrained character can't attempt checks with this skill (for skills that are designated as ``Trained Only'').

\input{subsections/5.3-appraise.tex}
\input{subsections/5.3-autohypnosis.tex}
\Skill{Balance}{Dex; Armor Check Penalty}
\textbf{Check:} You can walk on a precarious surface. A successful check lets you move at half your speed along the surface for 1 round. A failure by 4 or less means you can't move for 1 round. A failure by 5 or more means you fall. The difficulty varies with the surface, as follows:

\Table{Balance DCs}{X Z{1cm} X Z{1cm}}{
\tableheader Narrow Surface & \tableheader Balance DC & \tableheader Difficult Surface & \tableheader Balance DC\\
7-12 inches wide & 10 & Uneven flagstone & 10\\
2-6 inches wide & 15 & Hewn stone floor & 10\\
Less than 2 inches wide & 20 & Sloped or angled floor & 10
}

\Table{Narrow Surface Modifiers}{X R}{
\tableheader Surface & \tableheader DC Modifier \\
Lightly obstructed & +2\\
Severely obstructed & +5\\
Lightly slippery & +2\\
Severely slippery & +5\\
Sloped or angled & +2
}

For narrow surfaces, use the DC given on \tabref{Balance DCs} and add modifiers from \tabref{Narrow Surface Modifiers}, as appropriate. Those modifiers stack.

Make checks for difficult surfaces only if running or charging. Failure by 4 or less means the character can't run or charge, but may otherwise act normally.

\textit{Being Attacked while Balancing:} You are considered flat-footed while balancing, since you can't move to avoid a blow, and thus you lose your Dexterity bonus to AC (if any). If you have 5 or more ranks in Balance, you aren't considered flat-footed while balancing. If you take damage while balancing, you must make another Balance check against the same DC to remain standing.

\textit{Accelerated Movement:} You can try to walk across a precarious surface more quickly than normal. If you accept a $-5$ penalty, you can move your full speed as a move action. (Moving twice your speed in a round requires two Balance checks, one for each move action used.) You may also accept this penalty in order to charge across a precarious surface; charging requires one Balance check for each multiple of your speed (or fraction thereof) that you charge.

\textbf{Action:} None. A Balance check doesn't require an action; it is made as part of another action or as a reaction to a situation.

\textbf{Special:} If you have the Agile feat, you get a +2 bonus on Balance checks.

\textbf{Synergy:} If you have 5 or more ranks in \skill{Tumble}, you get a +2 bonus on Balance checks.
\input{subsections/5.3-bluff.tex}
\input{subsections/5.3-climb.tex}
\Skill{Concentration}{Con}
\textbf{Check:} You must make a Concentration check whenever you might potentially be distracted (by taking damage, by harsh weather, and so on) while engaged in some action that requires your full attention. Such actions include casting a spell, concentrating on an active spell, directing a spell, using a spell-like ability, or using a skill that would provoke an attack of opportunity. In general, if an action wouldn't normally provoke an attack of opportunity, you need not make a Concentration check to avoid being distracted.

If the Concentration check succeeds, you may continue with the action as normal. If the check fails, the action automatically fails and is wasted. If you were in the process of casting a spell, the spell is lost. If you were concentrating on an active spell, the spell ends as if you had ceased concentrating on it. If you were directing a spell, the direction fails but the spell remains active. If you were using a spell-like ability, that use of the ability is lost. A skill use also fails, and in some cases a failed skill check may have other ramifications as well.

The table below summarizes various types of distractions that cause you to make a Concentration check. If the distraction occurs while you are trying to cast a spell, you must add the level of the spell you are trying to cast to the appropriate Concentration DC. If more than one type of distraction is present, make a check for each one; any failed Concentration check indicates that the task is not completed.

If you are trying to cast, concentrate on, or direct a spell when the distraction occurs, add the level of the spell to the indicated DC.

If the distracting spell or power allows no save, use the save DC it would have if it did allow a save.

\Table{}{Y{3cm} >{\raggedright\arraybackslash}X}{
\tableheader Concentration DC & \tableheader Distraction\\
% 
% Such as during the casting of a spell with a casting time of 1 round or more, or the execution of an activity that takes more than a single full-round action (such as Disable Device). Also, damage stemming from an attack of opportunity or readied attack made in response to the spell being cast (for spells with a casting time of 1 standard action) or the action being taken (for activities requiring no more than a full-round action).
% Such as from acid arrow.
% 
10 + damage dealt & Damaged during the action.\\
10 + half of continuous damage last dealt & Taking continuous damage during the action, such as from \emph{acid arrow}.\\
Distracting spell's save DC & Distracted by nondamaging spell.\\
10 & Vigorous motion (on a moving mount, taking a bouncy wagon ride, in a small boat in rough water, belowdecks in a stormtossed ship).\\
15 & Violent motion (on a galloping horse, taking a very rough wagon ride, in a small boat in rapids, on the deck of a storm-tossed ship).\\
20 & Extraordinarily violent motion (earthquake).\\
15 & Entangled.\\
20 & Grappling or pinned. (You can cast only spells without somatic components for which you have any required material component in hand.)\\
5 & Weather is a high wind carrying blinding rain or sleet.\\
10 & Weather is wind-driven hail, dust, or debris.\\
Distracting spell's save DC & Weather caused by a spell, such as \emph{storm of vengeance}.\\
% \hline
Distracting power's save DC & Distracted by nondamaging power.\\
15 + power level & Attempting to manifest a power without its display.\\
20 & Gain psionic focus.\\
20 & Grappling or pinned. (You can manifest powers normally unless you fail your Concentration check.)\\
Distracting power's save DC & Weather caused by power
}

\textit{Gain Psionic Focus:} Merely holding a reservoir of psionic power points in mind gives psionic characters a special energy. Psionic characters can put that energy to work without actually paying a power point cost---they can become psionically focused as a special use of the Concentration skill.

If you have 1 or more power points available, you can meditate to attempt to become psionically focused. The DC to become psionically focused is 20. Meditating is a full-round action that provokes attacks of opportunity. When you are psionically focused, you can expend your focus on any single Concentration check you make thereafter. When you expend your focus in this manner, your Concentration check is treated as if you rolled a 15. It's like taking 10, except that the number you add to your Concentration modifier is 15. You can also expend your focus to gain the benefit of a psionic feat---many psionic feats are activated in this way.

Once you are psionically focused, you remain focused until you expend your focus, become unconscious, or go to sleep (or enter a meditative trance, in the case of elans), or until your power point reserve drops to 0.

\textbf{Action:} Usually none. In most cases, making a Concentration check doesn't require an action; it is either a free action (when attempted reactively) or part of another action (when attempted actively). Meditating to gain psionic focus is a full-round action.

\textbf{Try Again:} Yes, though a success doesn't cancel the effect of a previous failure, such as the loss of a spell you were casting or the disruption of a spell you were concentrating on.

\textbf{Special:} You can use Concentration to cast a spell, use a spell-like ability, or use a skill defensively, so as to avoid attacks of opportunity altogether. This doesn't apply to other actions that might provoke attacks of opportunity.

The DC of the check is 15 (plus the spell's level, if casting a spell or using a spell-like ability defensively). If the Concentration check succeeds, you may attempt the action normally without provoking any attacks of opportunity. A successful Concentration check still doesn't allow you to take 10 on another check if you are in a stressful situation; you must make the check normally. If the Concentration check fails, the related action also automatically fails (with any appropriate ramifications), and the action is wasted, just as if your concentration had been disrupted by a distraction.

A character with the Combat Casting feat gets a +4 bonus on Concentration checks made to cast a spell or use a spell-like ability while on the defensive or while grappling or pinned.

You can use Concentration to manifest a power or use a psi-like ability defensively, so as to avoid attacks of opportunity altogether. The DC of the check is 15 + the power's level. If the Concentration check succeeds, you can manifest normally without provoking any attacks of opportunity. If the Concentration check fails, the power also automatically fails and the power points are wasted, just as if your concentration had been disrupted by a distraction.

A character with the Combat Manifestation feat gets a +4 bonus on Concentration checks made to manifest a power or use a psi-like ability while on the defensive or while grappling or pinned.

\textbf{Synergy:} If you have 5 or more ranks in Concentration, you get a +2 bonus on \skill{Autohypnosis} checks.
\input{subsections/5.3-craft.tex}
\input{subsections/5.3-decipher-script.tex}
\input{subsections/5.3-diplomacy.tex}
\input{subsections/5.3-disable-device.tex}
\Skill{Disguise}{Cha}
\textbf{Check:} Your Disguise check result determines how good the disguise is, and it is opposed by others' Spot check results. If you don't draw any attention to yourself, others do not get to make Spot checks. If you come to the attention of people who are suspicious (such as a guard who is watching commoners walking through a city gate), it can be assumed that such observers are taking 10 on their Spot checks.

You get only one Disguise check per use of the skill, even if several people are making Spot checks against it. The Disguise check is made secretly, so that you can't be sure how good the result is.

The effectiveness of your disguise depends in part on how much you're attempting to change your appearance.

\Table{Disguise Check Modifiers}{X r{3cm}}{
\tableheader Disguise & \tableheader Disguise Check Modifier\\
% 
Minor details only & +5\\
Disguised as different gender & $-2$\\
Disguised as different race & $-2$\\
Disguised as different age category & $-2$ per step of difference
}

Modifiers for gender, race, and age are cumulative; use any that apply.

Apply modifiers for age per step of difference between your actual age category and your disguised age category. The steps are: young (younger than adulthood), adulthood, middle age, old, and venerable.


If you are impersonating a particular individual, those who know what that person looks like get a bonus on their Spot checks according to the table below. Furthermore, they are automatically considered to be suspicious of you, so opposed checks are always called for.

\Table{}{l R}{
\tableheader Familiarity & \tableheader Viewer's Spot Check Bonus\\
Recognizes on sight & +4\\
Friends or associates & +6\\
Close friends & +8\\
Intimate & +10
}

Usually, an individual makes a Spot check to see through your disguise immediately upon meeting you and each hour thereafter. If you casually meet many different creatures, each for a short time, check once per day or hour, using an average Spot modifier for the group.

\textbf{Action:} Creating a disguise requires 1d3 $\times$ 10 minutes of work.

\textbf{Try Again:} Yes. You may try to redo a failed disguise, but once others know that a disguise was attempted, they'll be more suspicious.

\textbf{Special:} Magic that alters your form, such as \emph{alter self}, \emph{disguise self}, \emph{polymorph}, or \emph{shapechange}, grants you a +10 bonus on Disguise checks (see the individual spell descriptions). You must succeed on a Disguise check with a +10 bonus to duplicate the appearance of a specific individual using the \emph{veil} spell. Divination magic that allows people to see through illusions (such as \emph{true seeing}) does not penetrate a mundane disguise, but it can negate the magical component of a magically enhanced one.

You must make a Disguise check when you cast a \emph{simulacrum} spell to determine how good the likeness is.

If you have the Deceitful feat, you get a +2 bonus on Disguise checks.

\textbf{Synergy:} If you have 5 or more ranks in \skill{Bluff}, you get a +2 bonus on Disguise checks when you know that you're being observed and you try to act in character.
\input{subsections/5.3-escape-artist.tex}
\input{subsections/5.3-forgery.tex}
\input{subsections/5.3-gather-information.tex}
\input{subsections/5.3-handle-animal.tex}
\input{subsections/5.3-heal.tex}
\input{subsections/5.3-hide.tex}
\input{subsections/5.3-intimidate.tex}
\input{subsections/5.3-jump.tex}
\input{subsections/5.3-knowledge.tex}
\input{subsections/5.3-listen.tex}
\input{subsections/5.3-literacy.tex}
\input{subsections/5.3-move-silently.tex}
\input{subsections/5.3-open-lock.tex}
\input{subsections/5.3-perform.tex}
\input{subsections/5.3-profession.tex}
\input{subsections/5.3-psicraft.tex}
\input{subsections/5.3-ride.tex}
\Skill{Search}{Int}
\textbf{Check:} You generally must be within 10 feet of the object or surface to be searched. The table below gives DCs for typical tasks involving the Search skill.

\textbf{Action:} It takes a full-round action to search a 5-foot-by-5-foot area or a volume of goods 5 feet on a side.

\textbf{Special:} An elf has a +2 racial bonus on Search checks, and a half-elf has a +1 racial bonus.

If you have the \feat{Investigator} feat, you get a +2 bonus on Search checks.

The spells \spell{explosive runes}, \spell{fire trap}, \spell{glyph of warding}, \spell{symbol}, and \spell{teleportation circle} create magic traps that a rogue can find by making a successful Search check and then can attempt to disarm by using \skill{Disable Device}. Identifying the location of a snare spell has a DC of 23. \spell{Spike growth} and \spell{spike stones} create magic traps that can be found using Search, but against which \skill{Disable Device} checks do not succeed. See the individual spell descriptions for details.

Active abjuration spells within 10 feet of each other for 24 hours or more create barely visible energy fluctuations. These fluctuations give you a +4 bonus on Search checks to locate such abjuration spells.

\textbf{Synergy:} If you have 5 or more ranks in Search, you get a +2 bonus on \skill{Survival} checks to find or follow tracks.

If you have 5 or more ranks in \skill{Knowledge} (architecture and engineering), you get a +2 bonus on Search checks to find secret doors or hidden compartments.

\textbf{Restriction:} While anyone can use Search to find a trap whose DC is 20 or lower, only a rogue can use Search to locate traps with higher DCs. (Exception: The spell \spell{find traps} temporarily enables a cleric to use the Search skill as if he were a rogue.)
\input{subsections/5.3-sense-motive.tex}
\input{subsections/5.3-sleight-of-hand.tex}
\input{subsections/5.3-speak-language.tex}
\input{subsections/5.3-spellcraft.tex}