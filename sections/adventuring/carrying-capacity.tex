\section{Carrying Capacity}
Encumbrance rules determine how much a character's armor and equipment slow him or her down. Encumbrance comes in two parts: encumbrance by armor and encumbrance by total weight.

\Table{Carrying Capacity}{r{9mm}RRR} {
\tableheader Strength Score & \tableheader Light Load & \tableheader Medium Load & \tableheader Heavy Load\\
1 & up to 2 kg & up to 3 kg & up to 5 kg\\
2 & up to 3 kg & up to 7 kg & up to 10 kg\\
3 & up to 5 kg & up to 10 kg & up to 15 kg\\
4 & up to 7 kg & up to 13 kg & up to 20 kg\\
5 & up to 8 kg & up to 17 kg & up to 25 kg\\
6 & up to 10 kg & up to 20 kg & up to 30 kg\\
7 & up to 12 kg & up to 23 kg & up to 35 kg\\
8 & up to 13 kg & up to 27 kg & up to 40 kg\\
9 & up to 15 kg & up to 30 kg & up to 45 kg\\
10 & up to 17 kg & up to 33 kg & up to 50 kg\\
11 & up to 19 kg & up to 38 kg & up to 58 kg\\
12 & up to 22 kg & up to 43 kg & up to 65 kg\\
13 & up to 25 kg & up to 50 kg & up to 75 kg\\
14 & up to 29 kg & up to 58 kg & up to 88 kg\\
15 & up to 33 kg & up to 67 kg & up to 100 kg\\
16 & up to 38 kg & up to 77 kg & up to 115 kg\\
17 & up to 43 kg & up to 87 kg & up to 130 kg\\
18 & up to 50 kg & up to 100 kg & up to 150 kg\\
19 & up to 58 kg & up to 117 kg & up to 175 kg\\
20 & up to 67 kg & up to 133 kg & up to 200 kg\\
21 & up to 77 kg & up to 153 kg & up to 230 kg\\
22 & up to 87 kg & up to 173 kg & up to 260 kg\\
23 & up to 100 kg & up to 200 kg & up to 300 kg\\
24 & up to 117 kg & up to 233 kg & up to 350 kg\\
25 & up to 133 kg & up to 267 kg & up to 400 kg\\
26 & up to 153 kg & up to 307 kg & up to 760 kg\\
27 & up to 173 kg & up to 347 kg & up to 520 kg\\
28 & up to 200 kg & up to 400 kg & up to 600 kg\\
29 & up to 233 kg & up to 467 kg & up to 700 kg\\
+10 & $\times$4 & $\times$4 & $\times$4\\
}

\textbf{Encumbrance by Armor}: A character's armor defines his or her maximum Dexterity bonus to AC, armor check penalty, speed, and running speed. Unless your character is weak or carrying a lot of gear, that's all you need to know. The extra gear your character carries won't slow him or her down any more than the armor already does.

If your character is weak or carrying a lot of gear, however, then you'll need to calculate encumbrance by weight. Doing so is most important when your character is trying to carry some heavy object.

\textbf{Weight}: If you want to determine whether your character's gear is heavy enough to slow him or her down more than the armor already does, total the weight of all the character's items, including armor, weapons, and gear. Compare this total to the character's Strength on \tabref{Carrying Capacity}. Depending on how the weight compares to the character's carrying capacity, he or she may be carrying a light, medium, or heavy load. Like armor, a character's load affects his or her maximum Dexterity bonus to AC, carries a check penalty (which works like an armor check penalty), reduces the character's speed, and affects how fast the character can run, as shown on \tabref{Carrying Loads}. A medium or heavy load counts as medium or heavy armor for the purpose of abilities or skills that are restricted by armor. Carrying a light load does not encumber a character.

If your character is wearing armor, use the worse figure (from armor or from load) for each category. Do not stack the penalties.

\Table{Carrying Loads}{l *{11}{c}}{
& \tableheader Max& \tableheader Check & \multicolumn{8}{c}{\tableheader Speed (meters)} &\\
\cmidrule[0.5pt]{4-11}
\tableheader Load & \tableheader Dex & \tableheader Penalty & \tableheader 6 & \tableheader 9 & \tableheader 12 & \tableheader 15 & \tableheader 18 & \tableheader 21 & \tableheader 24 & \tableheader 27 & \tableheader Run\\

Medium & +3 & --3 & 4.5 & 6 & 9 & 10.5 & 12 & 15 & 16.5 & 18 & $\times$4\\
Heavy & +1 & --6 & 4.5 & 6 & 9 & 10.5 & 12 & 15 & 16.5 & 18 & $\times$3\\
}

\textbf{Lifting and Dragging}: A character can lift as much as his or her maximum load over his or her head.

A character can lift as much as double his or her maximum load off the ground, but he or she can only stagger around with it. While overloaded in this way, the character loses any Dexterity bonus to AC and can move only 1.5 meter per round (as a full-round action).

A character can generally push or drag along the ground as much as five times his or her maximum load. Favorable conditions can double these numbers, and bad circumstances can reduce them to one-half or less.

\textbf{Bigger and Smaller Creatures}: The figures on \tabref{Carrying Capacity} are for Medium bipedal creatures. A larger bipedal creature can carry more weight depending on its size category, as follows: Large $\times$2, Huge $\times$4, Gargantuan $\times$8, Colossal $\times$16. A smaller creature can carry less weight depending on its size category, as follows: Small $\times$\threequarters, Tiny $\times$\onehalf, Diminutive $\times$\onequarter, Fine $\times$1/8.

Quadrupeds can carry heavier loads than characters can. Instead of the multipliers given above, multiply the value corresponding to the creature's Strength score from \tabref{Carrying Capacity} by the appropriate modifier, as follows: Fine $\times$\onequarter, Diminutive $\times$\onehalf, Tiny $\times$\threequarters, Small $\times$1, Medium $\times$1\onehalf, Large $\times$3, Huge $\times$6, Gargantuan $\times$12, Colossal $\times$24.

\textbf{Tremendous Strength}: For Strength scores not shown on \tabref{Carrying Capacity}, find the Strength score between 20 and 29 that has the same number in the ``ones'' digit as the creature's Strength score does and multiply the numbers in that for by 4 for every ten points the creature's strength is above the score for that row.