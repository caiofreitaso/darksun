\Figure*{t}{images/city-1.png}
\section{Urban Adventures}
At first glance, a city is much like a dungeon, made up of walls, doors, rooms, and corridors. Adventures that take place in cities have two salient differences from their dungeon counterparts, however. Characters have greater access to resources, and they must contend with law enforcement.

\textbf{Access to Resources:} Unlike in dungeons and the wilderness, characters can buy and sell gear quickly in a city. A large city or metropolis probably has high-level NPCs and experts in obscure fields of knowledge who can provide assistance and decipher clues. And when the PCs are battered and bruised, they can retreat to the comfort of a room at the inn.

The freedom to retreat and ready access to the marketplace means that the players have a greater degree of control over the pacing of an urban adventure.

\textbf{Law Enforcement:} The other key distinctions between adventuring in a city and delving into a dungeon is that a dungeon is, almost by definition, a lawless place where the only law is that of the jungle: Kill or be killed. A city, on the other hand, is held together by a code of laws, many of which are explicitly designed to prevent the sort of behavior that adventurers engage in all the time: killing and looting. Even so, most cities' laws recognize monsters as a threat to the stability the city relies on, and prohibitions about murder rarely apply to monsters such as aberrations or evil outsiders. Most evil humanoids, however, are typically protected by the same laws that protect all the citizens of the city. Having an evil alignment is not a crime (except in some severely theocratic cities, perhaps, with the magical power to back up the law); only evil deeds are against the law. Even when adventurers encounter an evildoer in the act of perpetrating some heinous evil upon the populace of the city, the law tends to frown on the sort of vigilante justice that leaves the evildoer dead or otherwise unable to testify at a trial.

\subsection{Weapon And Spell Restrictions}
Different cities have different laws about such issues as carrying weapons in public and restricting spellcasters.

The city's laws may not affect all characters equally. A monk isn't hampered at all by a law about peace-bonding weapons, but a cleric is reduced to a fraction of his power if all holy symbols are confiscated at the city's gates.

\subsection{Urban Features}
Walls, doors, poor lighting, and uneven footing: In many ways a city is much like a dungeon. Some new considerations for an urban setting are covered below.

\subsubsection{Walls And Gates}
Many cities are surrounded by walls. A typical small city wall is a fortified stone wall 1.5 meter thick and 6 meters high. Such a wall is fairly smooth, requiring a DC 30 \skill{Climb} check to scale. The walls are crenellated on one side to provide a low wall for the guards atop it, and there is just barely room for guards to walk along the top of the wall. A typical small city wall has AC 3, hardness 8, and 450 hp per 3-meter section.

A typical large city wall is 3 meters thick and 9 meters high, with crenellations on both sides for the guards on top of the wall. It is likewise smooth, requiring a DC 30 \skill{Climb} check to scale. Such a wall has AC 3, hardness 8, and 720 hp per 3-meter section.

A typical metropolis wall is 4.5 meters thick and 12 meters tall. It has crenellations on both sides and often has a tunnel and small rooms running through its interior. Metropolis walls have AC 3, hardness 8, and 1,170 hp per 3-meter section.

Unlike smaller cities, metropolises often have interior walls as well as surrounding walls---either old walls that the city has outgrown, or walls dividing individual districts from each other. Sometimes these walls are as large and thick as the outer walls, but more often they have the characteristics of a large city's or small city's walls.

\subsubsection{Watch Towers}
Some city walls are adorned with watch towers set at irregular intervals. Few cities have enough guards to keep someone constantly stationed at every tower, unless the city is expecting attack from outside. The towers provide a superior view of the surrounding countryside as well as a point of defense against invaders.

Watch towers are typically 3 meters higher than the wall they adjoin, and their diameter is 5 times the thickness of the wall. Arrow slits line the outer sides of the upper stories of a tower, and the top is crenellated like the surrounding walls are. In a small tower (7.5 meter in diameter adjoining a 1.5-meter-thick wall), a simple ladder typically connect the tower's stories and the roof. In a larger tower, stairs serve that purpose.

Heavy wooden doors, reinforced with iron and bearing good locks (Open Lock DC 30), block entry to a tower, unless the tower is in regular use. As a rule, the captain of the guard keeps the key to the tower secured on her person, and a second copy is in the city's inner fortress or barracks.

\subsubsection{Gates}
A typical city gate is a gatehouse with two portcullises and murder holes above the space between them. In towns and some small cities, the primary entry is through iron double doors set into the city wall.

Gates are usually open during the day and locked or barred at night. Usually, one gate lets in travelers after sunset and is staffed by guards who will open it for someone who seems honest, presents proper papers, or offers a large enough bribe (depending on the city and the guards).

\subsubsection{Guards And Soldiers}
A city typically has full-time military personnel equal to 1\% of its adult population, in addition to militia or conscript soldiers equal to 5\% of the population. The full-time soldiers are city guards responsible for maintaining order within the city, similar to the role of modern police, and (to a lesser extent) for defending the city from outside assault. Conscript soldiers are called up to serve in case of an attack on the city.

A typical city guard force works on three eight-hour shifts, with 30\% of the force on a day shift (8 A.M. to 4 P.M.), 35\% on an evening shift (4 P.M. to 12 A.M.), and 35\% on a night shift (12 A.M. to 8 A.M.). At any given time, 80\% of the guards on duty are on the streets patrolling, while the remaining 20\% are stationed at various posts throughout the city, where they can respond to nearby alarms. At least one such guard post is present within each neighborhood of a city (each neighborhood consisting of several districts).

The majority of a city guard force is made up of warriors, mostly 1st level. Officers include higher-level warriors, fighters, a fair number of clerics, and wizards or sorcerers, as well as multiclass fighter/spellcasters.

\subsubsection{City Streets}
Typical city streets are narrow and twisting. Most streets average 4.5 to 6 meters wide [(1d4+1)$\times$1.5 meter)], while alleys range from 3 meters wide to only 1.5 meter. Cobblestones in good condition allow normal movement, but ones in poor repair and heavily rutted dirt streets are considered light rubble, increasing the DC of \skill{Balance} and \skill{Tumble} checks by 2.

Some cities have no larger thoroughfares, particularly cities that gradually grew from small settlements to larger cities. Cities that are planned, or perhaps have suffered a major fire that allowed authorities to construct new roads through formerly inhabited areas, might have a few larger streets through town. These main roads are 21.5 meter wide---offering room for wagons to pass each other---with 1.5-meter-wide sidewalks on either side.

\textbf{Crowds:} Urban streets are often full of people going about their daily lives. In most cases, it isn't necessary to put every 1st-level commoner on the map when a fight breaks out on the city's main thoroughfare. Instead just indicate which squares on the map contain crowds. If crowds see something obviously dangerous, they'll move away at 9 meters per round at initiative count 0. It takes 2 squares of movement to enter a square with crowds. The crowds provide cover for anyone who does so, enabling a Hide check and providing a bonus to Armor Class and on Reflex saves.

\textit{Directing Crowds:} It takes a DC 15 \skill{Diplomacy} check or DC 20 \skill{Intimidate} check to convince a crowd to move in a particular direction, and the crowd must be able to hear or see the character making the attempt. It takes a full-round action to make the \skill{Diplomacy} check, but only a free action to make the \skill{Intimidate} check.

If two or more characters are trying to direct a crowd in different directions, they make opposed \skill{Diplomacy} or \skill{Intimidate} checks to determine whom the crowd listens to. The crowd ignores everyone if none of the characters' check results beat the DCs given above.

\Figure{t}{images/city-2.png}
\subsubsection{Above And Beneath The Streets}
\textbf{Rooftops:} Getting to a roof usually requires climbing a wall (see the Walls section), unless the character can reach a roof by jumping down from a higher window, balcony, or bridge. Flat roofs, common only in warm climates (accumulated snow can cause a flat roof to collapse), are easy to run across. Moving along the peak of a roof requires a DC 20 \skill{Balance} check. Moving on an angled roof surface without changing altitude (moving parallel to the peak, in other words) requires a DC 15 \skill{Balance} check. Moving up and down across the peak of a roof requires a DC 10 \skill{Balance} check.

Eventually a character runs out of roof, requiring a long jump across to the next roof or down to the ground. The distance to the next closest roof is usually 1d3$\times$1.5 meter horizontally, but the roof across the gap is equally likely to be 1.5 meter higher, 1.5 meter lower, or the same height. Use the guidelines in the Jump skill (a horizontal jump's peak height is one-fourth of the horizontal distance) to determine whether a character can make a jump.

\textbf{Sewers:} To get into the sewers, most characters open a grate (a full-round action) and jump down 3 meters. Sewers are built exactly like dungeons, except that they're much more likely to have floors that are slippery or covered with water. Sewers are also similar to dungeons in terms of creatures liable to be encountered therein. Some cities were built atop the ruins of older civilizations, so their sewers sometimes lead to treasures and dangers from a bygone age.

\subsubsection{City Buildings}
Most city buildings fall into three categories. The majority of buildings in the city are two to five stories high, built side by side to form long rows separated by secondary or main streets. These row houses usually have businesses on the ground floor, with offices or apartments above.

Inns, successful businesses, and large warehouses---as well as millers, tanners, and other businesses that require extra space--- are generally large, free-standing buildings with up to five stories.

Finally, small residences, shops, warehouses, or storage sheds are simple, one-story wooden buildings, especially if they're in poorer neighborhoods.

Most city buildings are made of a combination of stone or clay brick (on the lower one or two stories) and timbers (for the upper stories, interior walls, and floors). Roofs are a mixture of boards, thatch, and slates, sealed with pitch. A typical lower-story wall is 30 centimeters thick, with AC 3, hardness 8, 90 hp, and a \skill{Climb} DC of 25. Upper-story walls are 6 inches thick, with AC 3, hardness 5, 60 hp, and a \skill{Climb} DC of 21. Exterior doors on most buildings are good wooden doors that are usually kept locked, except on public buildings such as shops and taverns.

\textbf{Buying Buildings:} Characters might want to buy their own buildings or even construct their own castle. Use the prices in \tabref{Buildings} directly, or as a guide when for extrapolating costs for more exotic structures.

\Table{Buildings}{lR}{
  \tableheader Item
& \tableheader Cost \\
Simple house     & 1,000 cp \\
Grand house      & 5,000 cp \\
Mansion          & 100,000 cp \\
Tower            & 50,000 cp \\
Keep             & 150,000 cp \\
Castle           & 500,000 cp \\
Huge castle      & 1,000,000 cp \\
Moat with bridge & 50,000 cp \\
\TableNote{2}{\textit{Simple House:} This one- to three-room house is made of wood and has a thatched roof.}\\
\TableNote{2}{\textit{Grand House:} This four- to ten-room house is made of wood and has a thatched roof.}\\
\TableNote{2}{\textit{Mansion:} This ten- to twenty-room residence has two or three stories and is made of wood and brick. It has a slate roof.}\\
\TableNote{2}{\textit{Tower:} This round or square, three-level tower is made of stone.}\\
\TableNote{2}{\textit{Keep:} This fortified stone building has fifteen to twenty-five rooms.}\\
\TableNote{2}{\textit{Castle:} A castle is a keep surrounded by a 4.5-meter stone wall with four towers. The wall is 3 meters thick.}\\
\TableNote{2}{\textit{Huge Castle:} A huge castle is a particularly large keep with numerous associated buildings (stables, forge, granaries, and so on) and an elaborate 6-meter-high wall that creates bailey and courtyard areas. The wall has six towers and is 3 meters thick.}\\
\TableNote{2}{\textit{Moat with Bridge:} The moat is 4.5 meters deep and 9 meters wide. The bridge may be a wooden drawbridge or a permanent stone structure.}\\
}

\subsubsection{City Lights}
If a city has main thoroughfares, they are lined with lanterns hanging at a height of 2 meters from building awnings. These lanterns are spaced 18 meters apart, so their illumination is all but continuous. Secondary streets and alleys are not lit; it is common for citizens to hire lantern-bearers when going out after dark.

Alleys can be dark places even in daylight, thanks to the shadows of the tall buildings that surround them. A dark alley in daylight is rarely dark enough to afford true concealment, but it can lend a +2 circumstance bonus on Hide checks.
