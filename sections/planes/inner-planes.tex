\section{Inner Planes}
The Inner Planes are places of raw power and pure elements, of ultimate states and extreme conditions. They are the building blocks of the rest of the universe and represent matter and energy in their primal states. As destinations go, the Inner Planes are the most inhospitable to creatures from the Material Plane. Unprepared travelers are at best helpless before such power, and at worst they are snuffed out like candles in a tornado.

Each of the Inner Planes is a region of roughly similar environment. The Elemental Plane of Earth, for example, is mostly solid matter, while flames overwhelm the Elemental Plane of Fire. On each of the planes are bits of other elements and substances, existing like islands in the otherwise overwhelming element or energy. Outsiders and visitors gravitate toward the comfort these islands provide. The City of Brass, on the Elemental Plane of Fire, is the best-known locale of this type.

The Inner Planes have the following structure.
\begin{itemize*}
\item All Elemental Planes are coterminous with the Positive Energy Plane and the Negative Energy Plane. They also have three layers: the main elemental layer, a layer close to the Positive Energy Plane, and another close to the Negative Energy Plane.
\item The Elemental Plane of Air is coterminous with the Paraelemental Planes of Rain and Sun, and the Elemental Planes of Fire and Water.
\item The Elemental Plane of Earth is coterminous with the Paraelemental Planes of Magma and Silt, and the Elemental Planes of Fire and Water.
\item The Elemental Plane of Fire is coterminous with the Paraelemental Planes of Magma and Sun, and the Elemental Planes of Air and Earth.
\item The Elemental Plane of Water is coterminous with the Paraelemental Planes of Rain and Silt, and the Elemental Planes of Air and Earth.
\end{itemize*}


\begin{figure}[t!]
\centering
\begin{tikzpicture}[blend mode=multiply, baseline=0]
\contourlength{.5mm}
\begin{scope}[blend group=normal]
\draw[line width=0.2mm, color=black, pattern=crosshatch dots, pattern color=black!20] (\halflength*0.5, 0) arc (0:360:\halflength*0.5);
\draw[line width=0.2mm, color=black, draw, fill=black!10] ( 0.34\halflength*0.5, 0.94\halflength*0.5) arc (70:110:\halflength*0.5) -- (0,0) --cycle;
\draw[line width=0.2mm, color=black, draw, fill=black!10] (-0.34\halflength*0.5,-0.94\halflength*0.5) arc (250:290:\halflength*0.5) -- (0,0) --cycle;
\draw[line width=0.2mm, color=black, draw, fill=black!10] (-0.94\halflength*0.5, 0.34\halflength*0.5) arc (160:200:\halflength*0.5) -- (0,0) --cycle;
\draw[line width=0.2mm, color=black, draw, fill=black!10] ( 0.94\halflength*0.5, 0.34\halflength*0.5) arc (20:-20:\halflength*0.5) -- (0,0) --cycle;
\draw[line width=0.2mm, color=black, draw, fill=white] (0,0) circle (\halflength*0.5/2);
\node[color=richblack] at (0, \baselineskip/2) {\tableheader Ethereal};
\node[color=richblack] at (0,-\baselineskip/2) {\tableheader Plane};
\node[color=richblack] at (-0.7071*0.75\halflength*0.5, 0.7071*0.75\halflength*0.5) {\tableheader \contour{white}{Air}};
\node[color=richblack] at ( 0.7071*0.75\halflength*0.5,-0.7071*0.75\halflength*0.5) {\tableheader \contour{white}{Earth}};
\node[color=richblack] at ( 0.7071*0.75\halflength*0.5, 0.7071*0.75\halflength*0.5) {\tableheader \contour{white}{Fire}};
\node[color=richblack] at (-0.7071*0.75\halflength*0.5,-0.7071*0.75\halflength*0.5) {\tableheader \contour{white}{Water}};
\node[color=richblack] at (0.75\halflength*0.5, 0) {\tableheader Magma};
\node[color=richblack] at (-0.75\halflength*0.5, 0) {\tableheader Rain};
\node[color=richblack] at (0, -0.75\halflength*0.5) {\tableheader Silt};
\node[color=richblack] at (0, 0.75\halflength*0.5) {\tableheader Sun};
\end{scope}
\begin{scope}[blend group=normal, shift={(2\halflength*0.5+1mm,0)}]
\path[pattern=dots, pattern color=black!40]
   (-0.7071\halflength*0.5,-0.7071\halflength*0.5) arc (225:210:\halflength*0.5)
-- (-.1mm  ,-0.5\halflength*0.5)
-- (-.1mm  ,-0.7071\halflength*0.5)
--cycle;
\path[pattern=dots, pattern color=black!40]
   (0.7071\halflength*0.5,-0.7071\halflength*0.5) arc (-45:-30:\halflength*0.5)
-- (.1mm  ,-0.5\halflength*0.5)
-- (.1mm  ,-0.7071\halflength*0.5)
--cycle;

\path[pattern=dots, pattern color=black!20]
   (-0.7071\halflength*0.5, 0.7071\halflength*0.5) arc (135:150:\halflength*0.5)
-- (-0.1mm   , 0.5\halflength*0.5)
-- (-0.1mm   , 0.7071\halflength*0.5)
--cycle;
\path[pattern=dots, pattern color=black!20]
   (0.7071\halflength*0.5, 0.7071\halflength*0.5) arc (45:30:\halflength*0.5)
-- (0.1mm   , 0.5\halflength*0.5)
-- (0.1mm   , 0.7071\halflength*0.5)
--cycle;

\draw[line width=0.2mm, color=black, draw]
   ( 0.7071\halflength*0.5, 0.7071\halflength*0.5) arc (45:-45:\halflength*0.5)
-- ( 0    ,-0.7071\halflength*0.5)
-- ( 0    , 0.7071\halflength*0.5)
-- ( 0.5\halflength*0.5, 0.7071\halflength*0.5)
--cycle;

\draw[line width=0.2mm, color=black, draw]
   (-0.7071\halflength*0.5,-0.7071\halflength*0.5) arc (225:135:\halflength*0.5)
-- ( 0    , 0.7071\halflength*0.5)
-- ( 0    ,-0.7071\halflength*0.5)
-- (-0.5\halflength*0.5,-0.7071\halflength*0.5)
--cycle;


\path[pattern=crosshatch dots, pattern color=black!20]
   (-0.866\halflength*0.5, 0.5\halflength*0.5) arc (150:210:\halflength*0.5)
-- (-0.1mm   ,-0.5\halflength*0.5)
-- (-0.1mm   , 0.5\halflength*0.5)
--cycle;
\path[pattern=crosshatch dots, pattern color=black!20]
   (0.866\halflength*0.5, 0.5\halflength*0.5) arc (30:-30:\halflength*0.5)
-- (0.1mm,-0.5\halflength*0.5)
-- (0.1mm, 0.5\halflength*0.5)
--cycle;
\draw[line width=0.2mm, color=black, draw, fill=white] ( 0.7071\halflength*0.5, 0.7071\halflength*0.5) arc  (45:135:\halflength*0.5) -- ( 0.7071\halflength*0.5, 0.7071\halflength*0.5) --cycle;
\draw[line width=0.2mm, color=black, draw, fill=black] (-0.7071\halflength*0.5,-0.7071\halflength*0.5) arc (225:315:\halflength*0.5) -- (-0.7071\halflength*0.5,-0.7071\halflength*0.5) --cycle;

\draw[line width=0.2mm, color=black, draw, fill=black!10] (0, 0) circle (\halflength*0.5/2);
\draw[line width=0.2mm, color=black, draw] (\halflength*0.5, 0) arc (0:360:\halflength*0.5);
\node[color=white] at (0,-0.85\halflength*0.5) {\tableheader Negative};
\node[color=richblack] at (0, 0.85\halflength*0.5) {\tableheader Positive};
\node[color=richblack,rotate=90] at (-0.75\halflength*0.5,0) {\tableheader \contour{white}{Elemental}};
\node[color=richblack] at (0,0) {\tableheader Paraelemental};
\node[color=richblack,rotate=270] at ( 0.75\halflength*0.5,0) {\tableheader \contour{white}{Elemental}};
\end{scope}
\end{tikzpicture}
\end{figure}

\newcommand{\ElementalLayers}[3]{
\begin{figure}[h!]
\centering
\begin{tikzpicture}[blend mode=multiply, baseline=0]
\contourlength{.5mm}
\begin{scope}[blend group=normal]
\path[pattern=dots, pattern color=black!20] (-\halflength, 7.5mm) rectangle (\halflength, 3.5mm);
\path[pattern=crosshatch dots, pattern color=black!20] (-\halflength, 3.5mm) rectangle (\halflength, -3.5mm);
\path[pattern=dots, pattern color=black!40] (-\halflength, -3.5mm) rectangle (\halflength, -7.5mm);
\node[color=richblack] at (0,  5.5mm) {\tableheader \contour{white}{Quasielemental #1}};
\node[color=richblack] at (0, 0)    {\tableheader \contour{white}{Elemental #2}};
\node[color=richblack] at (0, -5.5mm) {\tableheader \contour{white}{Quasielemental #3}};
\end{scope}
\end{tikzpicture}
\end{figure}

}

\subsectionA{Elemental Plane of Air}
The Elemental Plane of Air is an empty plane, consisting of sky above and sky below. Clouds billow up in bank after bank, swelling into grand thunderheads and dissipating into wisps like cotton candy. The wind pulls and tugs around the traveler, and rainbows glimmer in the distance.

The Elemental Plane of Air is the most comfortable and survivable of the Inner Planes, and it is the home of all manner of airborne creatures. Indeed, flying creatures find themselves at a great advantage on this plane. While travelers without flight can survive easily here, they are at a disadvantage.

Cities in the Elemental Plane of Air are vast floating fortresses composed of clouds and walls of wind. While non-natives will find these disorienting and weird, denizens of the plane have little trouble navigating the streets of these floating cities.

Armies of Air are strange indeed. While many air elementals are naturally invisible, elemental beasts and incarnations soar within the ranks. They may seem like smaller numbers than are actually there by sight, but the Armies of Air make a fearsome and terrible noise as they approach. If traveling on the plane, one should flee from the sound of great rushing wind, and thunderclaps that have no clouds or lightning.

The Elemental Plane of Air has the following traits.
\begin{itemize*}
\item \textbf{Subjective Directional Gravity:} Inhabitants of the plane determine their own ``down'' direction. Objects not under the motive force of others do not move.
\item \textbf{Air-Dominant.}
\item \textbf{Enhanced Magic:} Spells and spell-like abilities that use, manipulate, or create air (including spells of the Air domain) are both empowered and enlarged (as if the \feat{Empower Spell} and \feat{Enlarge Spell} metamagic feats had been used on them, but the spells don't require higher-level slots).
\item \textbf{Impeded Magic:} Spells and spell-like abilities that use or create earth (including spells of the Earth domain and spells that summon earth elementals or outsiders with the earth subtype) are impeded.
\end{itemize*}

\subsectionA{Elemental Plane of Earth}
The Elemental Plane of Earth is a solid place of rock, soil, and stone. An unwary and unprepared traveler may find himself entombed within this vast solidity of material and have his body crushed into powder so all that is left is dust to stand as a warning to any foolish enough to follow.

Despite its unyielding nature, the Elemental Plane of Earth is varied in its consistency, ranging from relatively soft soil to veins of heavier and more valuable metal. Moving within the areas of softer soil will lead to the occasional isolated pocket of air. These locations typically have settlements in them for interacting with and talking to non-natives. The Lords of the Elemental Plane of Earth understand their followers on Athas aren't as hardy as they, and when they seek an audience, it will be here.

When off to war, the inhabitants of the Plane of Earth come like an earthquake. Their forms are the earth itself, so all manner of stone, metal, gems, soil and sand rush forward to meet their foes. Earth is the most patient of the Elemental Planes, however, and a sign of surrender or intentions of peace will typically be accepted, even if only for a short while.

The plane boasts two layers: Quasielemental Mineral and Quasielemental Dust.

\ElementalLayers{Mineral}{Earth}{Dust}

\subsubsection{Elemental Plane of Earth Traits}
\begin{itemize*}
\item \textbf{Heavy Gravity:} Strength-based and Dexterity-based skill checks incur a $-2$ circumstance penalty, as do all attack rolls. All item weights are effectively doubled, which might affect a character's speed. Weapon ranges are halved. Characters who fall on a heavy gravity plane take 1d10 points of damage for each 3 meters fallen, to a maximum of 20d10 points of damage.
\item \textbf{Earth-Dominant.}
\item \textbf{Enhanced Magic:} Spells and spell-like abilities that use, manipulate, or create earth or stone (including those of the Earth domain) are both empowered and extended (as if the \feat{Empower Spell} and \feat{Extend Spell} metamagic feats had been used on them, but the spells don't require higher-level slots). Spells and spell-like abilities that are already empowered or extended are unaffected by this benefit.
\item \textbf{Impeded Magic:} Spells and spell-like abilities that use or create air (including spells of the Air domain and spells that summon air elementals or outsiders with the air subtype) are impeded.
\end{itemize*}


\subsubsection{Quasielemental Mineral}
The layer of Quasielemental Mineral is where the Elemental Plane of Earth is closest to the Positive Energy Plane, infusing the earth with energy. It is saturated with jewels, gems, and all manner of valuable crystals. As one moves into the Mineral layer, gemstones become more prevalent, and they glow like stars.

The Quasielemental Mineral has the following additional traits.
\begin{itemize*}
\item \textbf{Minor Positive-Dominant:} A creature in Quasielemental Mineral gains fast healing 2 as an extraordinary ability.
\end{itemize*}

\subsubsection{Quasielemental Dust}
The layer of Quasielemental Dust is where the Elemental Plane of Earth is closest to the Negative Energy Plane, removing the energy and substance of the earth. As one comes closer to Quasielemental Dust, the earth starts to break down. Rocks become soil, and soil becomes dust. This layer is an expanse of tenuous atmosphere filled with swirling particles of granular matter.

\textbf{Dust Devils:} In Quasielemental Dust, dust devils are specially dangerous. They function as windstorms, but failing the Fortitude save (DC 18) will turn anything to dust, as the \spell{disintegrate} spell.

The Quasielemental Dust has the following additional traits.
\begin{itemize*}
\item \textbf{Subjective Directional Gravity:} Inhabitants of the plane determine their own ``down'' direction. Objects not under the motive force of others do not move.\\

This trait replaces the Heavy Gravity trait.
\item \textbf{Minor Negative-Dominant:} A living creature in Quasielemental Dust take 1d6 points of damage per round. At 0 hit points or lower, they crumble into ash.\\

This trait replaces the Earth-Dominant trait.
\end{itemize*}

\subsectionA{Elemental Plane of Fire}
The Elemental Plane of Fire is a nightmare to behold. The ground is nothing more than great, ever shifting plates of compressed flame. The air ripples with the heat of continual firestorms. The oceans are made of liquid flame. Yet, many creatures call this place home. Non-natives who lack protection from fire will find themselves nothing more than cinders within minutes if not seconds.

Fire burns here without fuel or air, and flammables brought onto the plane are ignited and consumed. The cities of this plane are constructed of compressed flames and heavy metals, like brass. The Lords of Fire are some of the most volatile, yet weakest rulers of the Elemental Planes. Their passion burns like everything else and they change their minds in a flash. Beware the traveler who makes a deal with the Lords of Flames as the contract may go up in smoke.

Armies of Fire are masked by the rolling smoke that heralds their approach, and they radiate heat that will burn those that come too close. Travelers on the plane should flee if they encounter a war party, as Fire delights in setting those unburned alight.

The plane boasts two layers: Quasielemental Radiance and Quasielemental Ash.

\begin{figure}[h!]
\centering
\begin{tikzpicture}[blend mode=multiply, baseline=0]
\contourlength{.5mm}
\begin{scope}[blend group=normal]
\path[fill=black!20] (-\columnwidth/2-2mm, -3.5mm) rectangle (\columnwidth/2-2mm,-10.5mm);
\path[pattern=dots, pattern color=black!20] (-\columnwidth/2-2mm, 10.5mm) rectangle (\columnwidth/2-2mm, 3.5mm);
\path[pattern=crosshatch dots, pattern color=black!20] (-\columnwidth/2-2mm, 3.5mm) rectangle (\columnwidth/2-2mm, -3.5mm);
\path[pattern=crosshatch dots, pattern color=white] (-\columnwidth/2-2mm, -3.5mm) rectangle (\columnwidth/2-2mm, -10 .5mm);
\node[color=richblack] at (0,  7mm) {\tableheader \contour{white}{Quasielemental Radiance}};
\node[color=richblack] at (0, 0)    {\tableheader \contour{white}{Elemental Fire}};
\node[color=richblack] at (0, -7mm) {\tableheader \contour{white}{Quasielemental Ash}};
\end{scope}
\end{tikzpicture}
\end{figure}

\subsubsection{Elemental Plane of Fire Traits}
\begin{itemize*}
\item \textbf{Fire-Dominant:} Unprotected wood, paper, cloth, and other flammable materials catch fire almost immediately, and those wearing unprotected flammable clothing catch on fire. In addition, individuals take 3d10 points of fire damage every round. Those that are made of water take double damage each round.
\item \textbf{Enhanced Magic:} Spells and spell-like abilities with the fire descriptor are both maximized and enlarged (as if the \feat{Maximize Spell} and \feat{Enlarge Spell} had been used on them, but the spells don't require higher-level slots). Spells and spell-like abilities that are already maximized or enlarged are unaffected by this benefit.
\item \textbf{Impeded Magic:} Spells and spell-like abilities that use or create water (including spells of the Water domain and spells that summon water elementals or outsiders with the water subtype) are impeded.
\end{itemize*}


\subsubsection{Quasielemental Radiance}
The layer of Quasielemental Radiance is where the Elemental Plane of Fire is closest to the Positive Energy Plane, infusing the fire with energy. Fire burns harder, brighter, in all colors. This is a place where light comes to being.

\textbf{Blinding Light:} Creatures in Quasielemental Radiance must make a Fortitude saving throw (DC 12) to avoid being blinded for 5 rounds by the brilliance of their surroundings.

The Quasielemental Radiance has the following additional traits.
\begin{itemize*}
\item \textbf{Minor Positive-Dominant:} A creature in Quasielemental Radiance gains fast healing 2 as an extraordinary ability.
\item \textbf{Enhanced Magic:} Spells and spell-like abilities with the light descriptor are both maximized and enlarged (as if the \feat{Maximize Spell} and \feat{Enlarge Spell} had been used on them, but the spells don't require higher-level slots). Spells and spell-like abilities that are already maximized or enlarged are unaffected by this benefit.
\item \textbf{Impeded Magic:} Spells and spell-like abilities with the darkness or shadow descriptor are impeded.
\end{itemize*}

\subsubsection{Quasielemental Ash}
The layer of Quasielemental Ash is where the Elemental Plane of Fire is closest to the Negative Energy Plane, removing the energy from the fire, leaving only ashes. Where Quasielemental Ash meets the Elemental Fire, it is an accumulation of burnt matter pressed into a solid mass. This burnt matter starts to break down when closer to the center of the layer. This layer is filled with floating particles, as clouds from a volcano.

\textbf{No Air:} A creature who breathes air starts to suffocate in the Quasielemental Ash.

The Quasielemental Ash has the following traits.
\begin{itemize*}
\item \textbf{Subjective Directional Gravity:} Inhabitants of the plane determine their own ``down'' direction. Objects not under the motive force of others do not move.
\item \textbf{Minor Negative-Dominant:} A living creature in Quasielemental Ash take 1d6 points of damage per round. At 0 hit points or lower, they crumble into ash.
\item \textbf{Enhanced Magic:} Spells and spell-like abilities  with the fire descriptor are both maximized and enlarged (as if the \feat{Maximize Spell} and \feat{Enlarge Spell} had been used on them, but the spells don't require higher-level slots). Spells and spell-like abilities that are already maximized or enlarged are unaffected by this benefit.
\item \textbf{Impeded Magic:} Spells and spell-like abilities that use or create water (including spells of the Water domain and spells that summon water elementals or outsiders with the water subtype) are impeded.
\end{itemize*}

\subsectionA{Elemental Plane of Water}
The Elemental Plane of Water is a sea of green and blue. Lacking a floor or a surface, it is an entirely fluid environment lit by a diffuse glow. It is one of the more hospitable of the Elemental Planes once a traveler figures out how to breathe.

The eternal oceans of this plane vary between ice cold and boiling hot, between saline and fresh. The water is constantly in motion, wracked by currents and tides. The plane's permanent settlements form around bits of coral and other drifting things suspended within this endless liquid. These settlements drift on the tides of the Elemental Plane of Water.

Armies of the Elemental Plane of Water are strange and surprisingly plentiful. Within the waves that are water elementals ride fearsome aquatic creatures, from sharks to kraken. The armies of the plane are aware when an intruder lurks, and if one is injured in battle, the scent of blood will travel for miles, attracting more and more creatures to devour the interloper.

The plane boasts two layers: Quasielemental Mist and Quasielemental Ice.

\begin{figure}[h!]
\centering
\begin{tikzpicture}[blend mode=multiply, baseline=0]
\contourlength{.5mm}
\begin{scope}[blend group=normal]
\path[pattern=dots, pattern color=black!20] (-\halflength, 7mm) rectangle (\halflength, 3.5mm);
\path[pattern=crosshatch dots, pattern color=black!20] (-\halflength, 3.5mm) rectangle (\halflength, -3.5mm);
\path[pattern=dots, pattern color=black!40] (-\halflength, -3.5mm) rectangle (\halflength, -7mm);
\node[color=richblack] at (0,  5.25mm) {\tableheader \contour{white}{Quasielemental Mist}};
\node[color=richblack] at (0, 0)    {\tableheader \contour{white}{Elemental Water}};
\node[color=richblack] at (0, -5.25mm) {\tableheader \contour{white}{Quasielemental Ice}};
\end{scope}
\end{tikzpicture}
\end{figure}
\Figure*{b}{images/planes-1.png}

\subsubsection{Elemental Plane of Water Traits}
\begin{itemize*}
\item \textbf{Subjective Directional Gravity:} Inhabitants of the plane determine their own ``down'' direction. Objects not under the motive force of others do not move.
\item \textbf{Water-Dominant:} Creatures made of fire take 1d10 points of damage each round.
\item \textbf{Enhanced Magic:} Spells and spell-like abilities that use or create water are both extended and enlarged (as if the \feat{Extend Spell} and \feat{Enlarge Spell} metamagic feats had been used on them, but the spells don't require higher-level slots). Spells and spell-like abilities that are already extended or enlarged are unaffected by this benefit.
\item \textbf{Impeded Magic:} Spells and spell-like abilities with the fire descriptor (including spells of the Fire domain) are impeded.
\end{itemize*}


\subsubsection{Quasielemental Mist}
The layer of Quasielemental Mist is where the Elemental Plane of Water is closest to the Positive Energy Plane, infusing the water with energy, until it turns to mist. This mist is not hot or cold, or even a dangerous hazard. Near the Elemental Water, the mist becomes so thick it seems more like bubbly water.

\textbf{Breathing Mist:} The Quasielemental Mist is barely breathable, creatures are \spellref{slow}{slowed} by breathing it. A \spell{worm's breath} spell removes this limitation.

The Quasielemental Mist has the following additional traits.
\begin{itemize*}
\item \textbf{Minor Positive-Dominant:} A creature in Quasielemental Mist gains fast healing 2 as an extraordinary ability.
\end{itemize*}

\subsubsection{Quasielemental Ice}
The layer of Quasielemental Ice is where the Elemental Plane of Water is closest to the Negative Energy Plane, removing all warmth from the water, only ice remaining.

\textbf{Unearthly Cold:} Every round, an unprotected creature in Quasielemental Ice takes 1d6 lethal damage and 1d4 nonlethal damage from the cold (no save allowed). Partially protected characters take damage once per 10 minutes, instead.

The Quasielemental Ice has the following additional traits.
\begin{itemize*}
\item \textbf{Minor Negative-Dominant:} A living creature in Quasielemental Ice take 1d6 points of damage per round. At 0 hit points or lower, they crumble into ash.\\

This trait replaces the Water-Dominant trait.
\end{itemize*}

\subsectionA{Paraelemental Plane of Magma}
The Paraelemental Plane of Magma is a vast expanse of flowing lava. Different pockets of alternating temperatures make some places within the plane harder or softer. Occasionally, a pocket of toxic fumes will exist. Travelers on the planes who are not native will be burned to cinders then crushed under the weight of all that liquid stone.

Cities within the plane are made of obsidian and basalt. The hellish landscape leaves little room for error for those traveling here. Passions erupt like volcanoes, and violence spews forth in pyroclastic shows of force and magic.
 
The armies of the plane are fearsome. They drip with molten rock burning and crushing all those in their way. They are highly aggressive and militaristic. Those who fail their lords are consumed, and their remains return to the plane that gave them life. Those traveling here would be best served to make the trip a short one. If they don't, the inhabitants of the plane will.

The Paraelemental Plane of Magma has the following traits.
\begin{itemize*}
\item \textbf{Heavy Gravity:} Strength-based and Dexterity-based skill checks incur a $-2$ circumstance penalty, as do all attack rolls. All item weights are effectively doubled, which might affect a character's speed. Weapon ranges are halved. Characters who fall on a heavy gravity plane take 1d10 points of damage for each 3 meters fallen, to a maximum of 20d10 points of damage.
\item \textbf{Earth-Dominant.}
\item \textbf{Fire-Dominant:} Unprotected wood, paper, cloth, and other flammable materials catch fire almost immediately, and those wearing unprotected flammable clothing catch on fire. In addition, individuals take 3d10 points of fire damage every round. Those that are made of water take double damage each round.
\item \textbf{Enhanced Magic.} Spells and spell-like abilities with the fire or earth descriptor are both maximized and enlarged (as if the \feat{Maximize Spell} and \feat{Enlarge Spell} had been used on them, but the spells don't require higher-level slots). Spells and spell-like abilities that are already maximized or enlarged are unaffected by this benefit.
\item \textbf{Impeded Magic.} Spells and spell-like abilities that use or create water (including spells of the Water domain and spells that summon water elementals or outsiders with the water subtype) are impeded.
\end{itemize*}

\subsectionA{Paraelemental Plane of Rain}
The Paraelemental Plane of Rain is an endless expanse of clouds, lightning and, of course, rain. The plane is not uniform, so pockets of peaceful light drizzle give way to raging storms that blow with the force of a hurricane. The Paraelemental Plane of Rain also has varying temperatures, so that one place will have warm, tropical rains, while others will have storms that are the rarest of all on Athas: blizzards.

Settlements in the plane are made of clouds and ice, hardened into vast castles and cities. Lightning jumps from structure to structure, and the sound of thunder, both from the storms and from the voices of the inhabitance, echo for hundreds of miles. A sadness seems to infect the plane, as well as an anger that burst like a thunderclap.

The forces of the Plane of Rain are the weakest of all the Paraelemental planes. Due to the weakened condition of Athas and their connection to the planes, Rain has suffered. However, weakest does not mean weak, and to withstand the force unleashed by the denizens of the planes when off to war is to be electrified, deafened, and then to have your flesh ripped by your bones from the onslaught of hailstones. Travelers should be prepared to make their stay as short as possible.

\subsubsection{Paraelemental Plane of Rain Traits}
\begin{itemize*}
\item \textbf{Subjective Directional Gravity:} Inhabitants of the plane determine their own ``down'' direction. Objects not under the motive force of others do not move.
\item \textbf{Air-Dominant.}
\item \textbf{Enhanced Magic:} Spells and spell-like abilities that use, manipulate, or create air or water (including spells of the Air or water domains) are both empowered and enlarged (as if the \feat{Empower Spell} and \feat{Enlarge Spell} metamagic feats had been used on them, but the spells don't require higher-level slots).
\item \textbf{Impeded Magic:} Spells and spell-like abilities that use or create earth or fire (including spells of the Earth and Fire domains and spells that summon earth or fire elementals or outsiders with the earth or fire subtype) are impeded.
\end{itemize*}

\subsectionA{Paraelemental Plane of Silt}
The Paraelemental Plane of Silt is a powdery field of gray that stretches for as far as the eye can see, which is a very short distance due to the clouds of silt that blanket the plane.. The silt is uniform throughout the plane, so one has trouble finding his way, or differentiating between this section of Gray Death and that section.

Settlements within the plane are formed from compressed silt into cities. These cities are thriving places where creatures rely on sonar and touch more than any other sense. Travelers need some way to survive, due to the lack of oxygen, the nature of silt, and the effects of Gray Death.

Being one of the more powerful Paraelemental Planes, the Armies of the Silt Lords are fierce. They flow forward like a tide. Anything that they catch is pulled apart and ground away by the thousands of tiny particles sandblasting the meat from a victim's bones.

\textbf{Gray Death:} Every 10 minutes in the Paraelemental Plane of Silt, a character must save against the gray death disease. Expertly worn cloth gives +4 on the save. Inhaled---Fortitude DC 15, incubation period 1 day, damage 1d3 Str and 1d3 Dex. When damaged, character must succeed on another saving throw or become permanently fatigued.

\subsubsection{Paraelemental Plane of Silt Traits}
\begin{itemize*}
\item \textbf{Subjective Directional Gravity:} Inhabitants of the plane determine their own ``down'' direction. Objects not under the motive force of others do not move.
\item \textbf{Earth-Dominant.}
\item \textbf{Enhanced Magic:} Spells and spell-like abilities with the acid descriptor are both maximized and enlarged (as if the \feat{Maximize Spell} and \feat{Enlarge Spell} had been used on them, but the spells don't require higher-level slots). Spells and spell-like abilities that are already maximized or enlarged are unaffected by this benefit.
\item \textbf{Impeded Magic:} Spells and spell-like abilities that use or create water (including spells of the Water domain and spells that summon water elementals or outsiders with the water subtype) are impeded.
\end{itemize*}

\subsectionA{Paraelemental Plane of Sun}
The Paraelemental Plane of Sun is a vast plane of cloudless sky and an enormous sun, with heat and light radiating on all. There is no protection from the heat, nor do the denizens of the plane care to have it.

Settlements on the plane are cities of light, radiating like the sun on solidified rays of solar energy. While different colors exist, the structures tend to mimic the sun and have a dark red coloring. Every so often a building will have prisms, scattering the light into more colors than a traveler could imagine. Visitors will want protection from the exposure to so much light and heat.

The Armies of the Sun are blinding forces that radiate not only their anger and power, but also cast out any shadows or darkness. They travel fast and strike hard, vanishing in the blink of an eye, leaving their victims blind and burned.

The Paraelemental Plane of Sun has the following traits.
\begin{itemize*}
\item \textbf{Subjective Directional Gravity:} Inhabitants of the plane determine their own ``down'' direction. Objects not under the motive force of others do not move.
\item \textbf{Air-Dominant.}
\item \textbf{Fire-Dominant:} Unprotected wood, paper, cloth, and other flammable materials catch fire almost immediately, and those wearing unprotected flammable clothing catch on fire. In addition, individuals take 3d10 points of fire damage every round. Those that are made of water take double damage each round.
\item \textbf{Enhanced Magic:} Spells and spell-like abilities with the air or fire descriptor are both maximized and enlarged (as if the Maximize Spell and Enlarge had been used on them, but the spells don't require higher-level slots). Spells and spell-like abilities that are already maximized or enlarged are unaffected by this benefit.
\item \textbf{Impeded Magic:} Spells and spell-like abilities that use or create water, darkness or shadows (including spells of the Water domain and spells that summon water elementals or outsiders with the water or shadow or Black subtype) are impeded.
\end{itemize*}

\subsectionA{Negative Energy Plane}
To an observer, there's little to see on the Negative Energy Plane. It is a dark, empty place, an eternal pit where a traveler can fall until the plane itself steals away all light and life. The Negative Energy Plane is the most hostile of the Inner Planes, and the most uncaring and intolerant of life. Only creatures immune to its life-draining energies can survive there.

\subsubsection{Negative Energy Plane Traits}
\begin{itemize*}
\item \textbf{Subjective Directional Gravity:} Inhabitants of the plane determine their own ``down'' direction. Objects not under the motive force of others do not move.
\item \textbf{Major Negative-Dominant:} Each round, those within must make a DC 25 Fortitude save or gain a negative level. A creature whose negative levels equal its current levels or Hit Dice is slain, becoming a wraith. The \spell{death ward} spell protects a traveler from the damage and energy drain of the plane.

Some areas within the plane have only the minor negative-dominant trait, and these islands tend to be inhabited. On those areas, living creatures take 1d6 points of damage per round. At 0 hit points or lower, they crumble into ash.
\item \textbf{Enhanced Magic:} Spells and spell-like abilities that use negative energy are maximized (as if the \feat{Maximize Spell} metamagic feat had been used on them, but the spells don't require higher-level slots). Spells and spell-like abilities that are already maximized are unaffected by this benefit. Class abilities that use negative energy, such as rebuking and controlling undead, gain a +10 bonus on the roll to determine Hit Dice affected.
\item \textbf{Impeded Magic:} Spells and spell-like abilities that use positive energy, including \spellref{cure light wounds}{cure} spells, are impeded. Characters on this plane take a $-10$ penalty on Fortitude saving throws made to remove negative levels bestowed by an energy drain attack.
\end{itemize*}

\subsectionA{Positive Energy Plane}
The Positive Energy Plane has no surface and is akin to the Elemental Plane of Air with its wide-open nature. However, every bit of this plane glows brightly with innate power. This power is dangerous to mortal forms, which are not made to handle it. Despite the beneficial effects of the plane, it is one of the most hostile of the Inner Planes. An unprotected character on this plane swells with power as positive energy is force-fed into her. Then, her mortal frame unable to contain that power, she immolates as if she were a small planet caught at the edge of a supernova. Visits to the Positive Energy Plane are brief, and even then travelers must be heavily protected.

The Positive Energy Plane has the following traits.
\begin{itemize*}
\item \textbf{Subjective Directional Gravity:} Inhabitants of the plane determine their own ``down'' direction. Objects not under the motive force of others do not move.
\item \textbf{Major Positive-Dominant:} A creature on the Positive Energy Plane must make a DC 15 Fortitude save to avoid being blinded for 10 rounds by the brilliance of the surroundings. Simply being on the plane grants fast healing 5 as an extraordinary ability. In addition, those at full hit points gain 5 additional temporary hit points per round. These temporary hit points fade 1d20 rounds after the creature leaves the Positive Energy Plane. However, a creature must make a DC 20 Fortitude save each round that its temporary hit points exceed its normal hit point total. Failing the saving throw results in the creature exploding in a riot of energy, killing it.

Some regions of the plane have the minor positive-dominant trait instead, and those islands tend to be inhabited. All individuals in those regions gain fast healing 2 as an extraordinary ability.
\item \textbf{Enhanced Magic:} Spells and spell-like abilities that use positive energy, including \spellref{cure light wounds}{cure} spells, are maximized (as if the \feat{Maximize Spell} metamagic feat had been used on them, but the spells don't require higher-level slots). Spells and spell-like abilities that are already maximized are unaffected by this benefit. Class abilities that use positive energy, such as turning and destroying undead, gain a +10 bonus on the roll to determine Hit Dice affected. (Undead are almost impossible to find on this plane, however.)
\item \textbf{Impeded Magic:} Spells and spell-like abilities that use negative energy (including \spellref{inflict light wounds}{inflict} spells) are impeded.
\end{itemize*}

