\section{Outer Planes}
If the Inner Planes are the raw matter and energy that makes up the multiverse, the Outer Planes are the direction, thought and purpose for such construction. 
These are the planes where the ancient gods once thrived. The Outer Planes are so far from the Prime Material Plane that any of the existing gods cannot interact with Athas anymore. The bigger buffer the Gray has between Athas and the Astral Plane is enough to impede the formation of conduits connecting the Prime Material Plane to the Outer Planes.

% Accordingly, many sages refer to the Outer Planes as divine planes, spiritual planes, or godly planes, for the Outer Planes are best known as the homes of deities. Gods may live elsewhere, but they thrive on the Outer Planes. Other creatures inhabit the Outer Planes too, some that are the servants of deities and others that fiercely guard their independence.

To the novice traveler, the Outer Planes seem more hospitable and familiar to natives of the Material Plane than either the Transitive Planes or the Inner Planes. The notion is deceptive. While the landscape might look like that of the Material Plane, it can change at the whim of the powerful forces that live on the Outer Planes. The desires of the godly forces that dwell on these planes can remake them completely, effectively erasing and rebuilding existence itself to better fulfill their own needs.

\textbf{Traveling:} Each of the Outer Planes borders two others, forming the Great Wheel that encircles the Gray. It is possible to move between the Outer Planes of the Great Wheel through the Astral or through physical movement as one plane gives way to another near the coterminous border regions. These borders are loosely defined and always in a state of flux, so creatures sometimes move between planes without intending to.

\subsubsection{Godhood}
The Outer Planes are also the homes of powerful deities. These deities may exist on other planes, but the Outer Planes are ideal for them. The Outer Planes are divinely morphic, so deities can alter the landscape itself within a limited area. And deities benefit from inhabiting planes that match their alignment.

The basic categories of godhood are:

\textbf{Quasi-Deity/Hero-Deity:} Quasi-deities and hero-deities are only slightly ``divine,'' but they still possess a portion of deity-level power or attract worshipers. They lack the ability to manipulate the area around them. As a result, they tend to wander or occupy small realms on the Outer Planes that are constructed with brute labor, not divine power.

\textbf{Demideity:} This is the weakest category of fully divine deities. Indeed, some demideities were once legendary mortals who were rewarded with (or seized for themselves) the mantle of godhood. They are the most mortal-seeming of the deities. They also lack the power to manipulate the divinely morphic area around them. They also wander, construct their homes conventionally, or rely on more powerful deities to manipulate the planar landscape for them.

\textbf{Lesser Deity:} The deities in this category have several domains they grant their clerics and an area of responsibility called a portfolio that they oversee. Lesser deities tend to handle relatively ``small'' matters, and they often work for and with more powerful deities. They can begin to alter the Outer Plane where they live, changing how travelers reach their realms. Within a two-kilometer radius, a lesser deity can determine the nature of any planar connections. The deity can decide, for example, whether the area can be accessed from the Astral Plane. In addition, a lesser deity can mandate that planar portals only appear at a specific location and that creatures cannot be summoned from within the area.

\textbf{Intermediate Deity:} These deities are more powerful than the lesser deities and command more important portfolios. can make more dramatic and far-reaching effects. They can affect everything within a fifteen-kilometer radius just as a lesser deity can, plus an intermediate deity can apply the enhanced magic or impeded magic trait to up to four groups (selected by school, domain, or descriptor) of spells.

Many deities apply the enhanced magic trait to their domain spells, making them maximized (as the \feat{Maximize Spell} feat) within the boundaries of their realm. The impeded magic trait doesn't affect the intermediate deity's spells and spell-like abilities, but the deity can take advantage of enhanced magic within the realm.

In addition, the intermediate deity can erect buildings as desired and alter terrain within ten miles to become any terrain type found on the Material Plane.

\textbf{Greater Deity:} These are the most powerful deities that adventurers encounter. They have the most important portfolios and often have other deities reporting to them. Some lead entire groups of deities (called pantheons). A greater deity affects everything just as an intermediate deity does, but out to a two-hundred-kilometer radius. Within this area, the greater deity can also do anyone of the following:

\begin{itemize*}
\item Change or apply an elemental or energy trait of the area.
\item Change or apply a time trait within the area.
\item Change or apply a gravity trait within the area.
\item Apply the limited magic trait against a particular school, domain, or spell descriptor within the area. The greater deity's own spells and spell-like abilities are not limited by these restrictions.
\end{itemize*}

Using power to divinely morph a plane in such dramatic fashion is time-consuming and involves a great deal of the deity's divine power. As a result, a greater deity often uses major trait changes to cement his or her hold over a realm. Such places are not given up lightly.

\subsectionA{Heroic Domains of Ysgard}
Ysgard is a plane on an epic scale, with soaring mountains, deep fjords, and dark caverns that hide the secret forges of the dwarves. A biting wind always blows at a hero's back. From the freezing water channels to the sacred groves of Alfheim's elves, Ysgard's terrain is grand and terrible. It is a place of sharp seasons: Winter is a time of darkness and killing cold, and a summer day is scorching and clear.

Most spectacular of all, the landscape floats atop immense rivers of earth flowing forever through an endless skyscape. The broadest earthen rivers are the size of continents, while smaller sections, called earthbergs, are island-sized. Fire rages under each river, but only a reddish glow penetrates to the continent's top. Of more concern is the occasional collision between rivers, which produces terrible quakes and sometimes spawns new mountain ranges.

Ysgard is the home of slain heroes who wage eternal battle on fields of glory. When these petitioners fall, they rise again the next morning to continue eternal warfare. Two deities make their homes on Ysgard: Kord, scion of Strength; and Olidammara, patron of thieves.

Ysgard has the following traits.
\begin{itemize*}
\item \textbf{Divinely Morphic:} Specific powerful beings (such as the deities Kord and Olidammara) can alter Ysgard with a thought. They can change vast areas, creating great realms for themselves.
\item \textbf{Minor Positive-Dominant:} Ysgard possesses a riotous explosion of life in all its forms. All individuals on a positive-dominant plane gain fast healing 2 and may even regrow lost limbs in time. Additionally, those slain in the never-ending conflicts on Ysgard's fields of battle rise each morning as if \spell{true resurrection} were cast on them, fully healed and ready to fight anew. Even petitioners, who as outsiders cannot be raised, awaken fully healed. Only those who suffer mortal wounds on Ysgard's battlefields get the \spell{true resurrection} effect; dead characters brought to Ysgard don't spontaneously revive.
\item \textbf{Mildly Chaos-Aligned:} Lawful creatures on Ysgard suffer a $-2$ penalty on all Charisma-based checks.
\end{itemize*}

\subsubsection{Ysgard}
The top layer of Ysgard, also called Ysgard, is far and away the most well known and well traveled of the three layers. Most of the inhabitants live in camps and rugged settlements with rough and wild conditions. The layer is dotted with dozens of huge halls, smoking battlefields, and hilly terrain leading down to cold seas. Few settlements exist along the edges of any of the earthbergs, except those interested in trade with communities on other earthbergs.

\textbf{Kord's Realm:} The deity of the strong and courageous, Kord the Brawler lives in the Hall of the Valiant on this plane. His grand hall is built of stout beams of wood hewn from a single massive ash tree. Within, Kord presides over a never-ending banquet where honored guests come and go, but the revelry never ends. The feast tables surround a great open space where valiant heroes wrestle for sport. Sometimes, Kord himself sets aside his intelligent dragon-slaying greatsword, Kelmar, and his dragon-hide accoutrements, and enters the square to the great delight of all assembled.

\textbf{Plain of Ida:} This great field is located near the Hall of the Valiant and the great free city of Himinborg, the largest population center on the layer. The Plain of Ida hosts daily festivals where warriors can flaunt their mettle. Here, bravery and skill in battle is valued over all else.

\textbf{Alfheim:} Elven petitioners populate this brilliant, sunlit region, as does a contingent of mortal elves. Alfheim is suffused with light and joy, and visitors cannot help but be buoyed by the happiness in the air. The lands are wild and beautiful, untouched by civilization. Wildlife is plentiful, and natural features such as streams, forests, and sunny hills are likewise bountiful.

The elven natives are friendly, but they care little for anything but games and meditative appreciation of their natural surroundings. While many elves live in harmony with nature among the trees and fields of the surface, some elves abide in glittering caves below the surface of Alfheim.

Alfheim has seasons. Summers are long and kind, and its winters are dark and unforgiving. During winter, the elves retreat into the glittering caves, the entrances to which are sealed off and buried during the season of snows.

\textbf{Den of Olidammara:} The god of rogues, Olidammara the Laughing Rogue is an intermediate deity who concerns himself with music, revels, wine, humor, and similar ideals. Wood, stone, and stranger substances create a grand but haphazard structure, as if several mansions of various cultures were mashed together.

On the inside, mazes, locked doors, blind hallways, and secret treasuries surround a grand hall where music and dancing are mandatory. Usually, the guests of this inmost den include rogues, bards, performers, and entertainers of all stripes and all places. Wine, romance, and song rule here, where Olidammara lounges at his ease on a grand divan---unless he is disguised as one of his many guests using his magic laughing mask. Because some terrible prank often draws him far away from his den, other deities treat Olidammara with deserved caution no matter where they are.


\subsubsection{Muspelheim}
The middle layer of Ysgard, Muspelheim, is made from ribbons of floating earth, some continent-sized or larger. Here, though, the ground smokes and burns, earning this layer the name ``Land of Fire.'' It's a hostile layer where even the ground is sharp volcanic rock. Most of Muspelheim has the fire-dominant trait.

Muspelheim's ground rolls toward a ridge of fiery mountains at the highest point. This range, called the Serpent Spine, is home to hundreds of clans of fire giants. Watchtowers and citadels defend the mountain passes against rival clans and unwanted visitors.

The Spire is a towering, needle-thin citadel of dark stone in the midst of the Serpent Spine mountains. Devout fire giant maidens are said to inhabit the tower, serving as clerics of a mysterious intermediate deity of fire giants.

\subsubsection{Nidavellir}
The third layer of Ysgard is Nidavellir. It is an ``underground'' realm crisscrossed by warm tunnels, heated by hot springs and geysers. The wild regions are crowded with underground forests of strange woods that need no sun, only heat, to grow. Vast caverns run through veins of clear quartz, and deep holds are studded with shining mica and pyrite. Precious and semiprecious minerals are strewn across the floor of some lengths of runnel and even entire caverns.

Dwarven and gnome kingdoms divide up most of Nidavellir. Most of the layer's inhabitants are mortals, but petitioners are common as well. It is a place of fiery furnaces, ringing anvils, and constant striving for perfection in the crafts of smithing, runecrafting, and magic. Its halls resound with the chanting voices of dwarves and the lilting songs of gnomes. Though the two races are rivals often given to war, they unite when confronted by their underground enemies: dark elves.

\textbf{Svartalfheim:} Drow have their own realm in Nidavellir. Though the gnomes and dwarves think the worst of the dark elves, the allegiances of these particular drow are not as evil as many travelers might think. Like others of this layer, they merely wish to be left alone and they don't take kindly to unannounced visitors or trespassers.

\subsectionA{Ever-Changing Chaos of Limbo}
Limbo is a plane of pure chaos. Untended sections appear as a roiling soup of the four basic elements and all their combinations. Balls of fire, pockets of air, chunks of earth, and waves of water battle for ascendance until they in turn are overcome by yet another chaotic surge. However, landscapes similar to ones found on the Material Plane drift through the miasma: bits of forest, meadow, ruined castles, and small islands.

Limbo is inhabited by living natives. Most prominent of these are the githzerai and the slaadi. In Limbo, most petitioners take the form of unthinking, ghostly spheres of swirling chaos.

Limbo has no layers. Or if it does, the layers continually merge and part, each is as chaotic as the next, and even the wisest sages would be hard-pressed to distinguish one from another.

Maps are useless in the chaotic expanse. Over time, even solid, permanent structures drift in the chaotic currents of Limbo. The time it takes an individual or group of individuals to reach a particular area depends on how familiar they are with that area:

\Table{}{XX}{
  \tableheader Familiarity
& \tableheader Travel Time \\
Very familiar     & 2d6 hours \\
Studied carefully & 1d4 $\times$ 6 hours \\
Seen casually     & 1d4 $\times$ 10 hours \\
Viewed once       & 1d6 $\times$ 20 hours \\
Description only  & 1d10 $\times$ 50 hours \\
}

It has the following traits.
\begin{itemize*}
\item \textbf{Subjective Directional Gravity:} Inhabitants of the plane determine their own ``down'' direction. Objects not under the motive force of others do not move.
\item \textbf{Highly Morphic:} Limbo is continually changing, and keeping a particular area stable is difficult. A given area, unless magically stabilized somehow, can react to specific spells, sentient thought, or the force of will. Left alone, it continually changes. For more information on stabilization, see Controlling Limbo, below.
\item \textbf{Sporadic Element-Dominant:} No one element constantly dominates Limbo. Each element (Earth, Water, Air, or Fire) is dominant from time to time, so any given area is a chaotic, dangerous boil. The elemental dominance can change without warning.
\item \textbf{Strongly Chaos-Aligned:} Nonchaotic characters suffer a $-2$ penalty on all Charisma-, Wisdom-, and Intelligence-based checks. However, the strongly chaos-aligned trait disappears within the walls of githzerai monasteries (but not githzerai cities).
\item \textbf{Wild Magic:} Spells and spell-like abilities in Limbo function in wildly different ways. They function normally within permanent structures or on permanently stabilized landscapes in Limbo. But any spell or spell-like ability used in an untended area of Limbo, or an area temporarily controlled, has a chance to go awry. The spellcaster must make a level check (1d20 + spellcaster level) against a DC of 15 + the level of the attempted spell. If the caster fails the check, roll on \tabref{Wild Magic Effects}.
\end{itemize*}


\subsubsection{Controlling Limbo}
There are three kinds of terrain in Limbo: uncontrolled raw areas, controlled areas, and stabilized areas. Raw areas make up most of the plane, while the controlled areas (also called tended areas) and stabilized areas are tiny islands in comparison.

\textbf{Raw Limbo:} Uncontrolled areas of limbo are dangerous, but most sentient creatures can exert a localized calming influence (see Controlled Limbo, below). But sometimes there's no control, such as when a visitor first enters Limbo or when a traveler is knocked unconscious. When no one's trying to control a given area of Limbo, it exhibits the qualities noted on the table below. For the purposes of this table, an area is everything within a 9-meter-radius sphere, though areas can drift and move around randomly. For a given area, roll on the table once every 1d10 minutes.

\Table{Uncontrolled Limbo}{lX}{
  \tableheader d\%
& \tableheader Effect \\
01--10  & Air-dominant \\
11--20  & Earth-dominant \\
21--30  & Fire-dominant \\
31--40  & Water-dominant \\
41--50  & Mixed dominant: Air and earth \\
51--60  & Mixed dominant: Fire and earth \\
61--70  & Mixed dominant: Water and earth \\
71--80  & Mixed dominant: Water and air \\
81--90  & Mixed dominant: Air and fire \\
91--100 & Balance (as if air-dominant) \\

\TableNote{2}{\textbf{Element-Dominant:} The indicated type of element surges in the given area. The previous dominant element is wiped away in the first round, and the effects of the new dominant element come into play immediately. Limbo's subjective gravity trait overrides elemental gravity traits that conflict with it.}\\

\TableNote{2}{\textbf{Mixed Dominance:} Two elements mix together, creating a hybrid effect. All effects of both element-dominant traits simultaneously affect the area. In addition to the trait effects, the region develops a chaotic mix of both elements. For example, where earth and fire mix, a boiling ball of magma results.}\\

\TableNote{2}{\textbf{Balance:} The elemental forces come into exact balance, and tranquillity results (for 1d10 minutes). Treat a balanced area as air-dominant, because that trait has no dramatic effects.}\\
}


\textbf{Controlled Limbo:} Controlling a raw area of Limbo is an exercise of the mind. A Wisdom check (DC 16) establishes control within part of a raw area of limbo, and the check can be repeated once per round as a free action. A traveler who has failed checks twice in a row gains a +6 circumstance bonus on subsequent checks. If entering an area of raw Limbo from a controlled or stabilized area, a character can make a control check just prior to stepping into the boil.

If the Wisdom check succeeds, the creature has established control over part of the area and can reshape it as she desires, allowing a desired element or a mixture of elements to become dominant. A favorite among travelers from the Material Plane is a chunk of earth surrounded by a small atmosphere of air.

Consult the table below to determine how large an area a character can control.

\Table{Controlled Limbo}{cXX}{
  \tableheader Wisdom Score
& \tableheader Area of Control
& \tableheader Stabilized Area \\
1--3   & None                    &  \\
4--7   & 30-cm radius            &  \\
8--11  & 1.5-m radius            &  \\
12--15 & 3-m radius              &  \\
16--19 & 4.5-m radius            &  \\
20--23 & 6-m radius              & 1.5-m radius \\
24+    & +1.5 m per 4 Wis points & +1.5 m per 4 Wis points \\
}

\textbf{Stabilized Limbo:} A section of Limbo becomes stabilized if a creature of sufficiently high Wisdom creates it within an area of control. The stabilized area in the center of the area of control retains its traits. It drifts at the whim of Limbo's chaotic currents and, if not protected, is eventually eroded by repeated immersions in the elemental surges. For instance, a 1.5-meter-radius ball of fire could become stable if created by a creature with a Wisdom of 20 or higher. Over the course of several dunks in water, however, it is eroded and finally dissipated. However, industrious creatures can bring bits of stabilized earth together and use them as the foundation for permanent structures, especially if tended by guardians.

\subsectionA{Windswept Depth of Pandemonium}
Pandemonium is a great mass of matter pierced by innumerable tunnels carved by the howling winds of the plane. It is windy, noisy, and dark, having no natural source of light. The wind quickly extinguishes normal fires, and lights that last longer draw attention of wights driven insane by the constant howling wind.

Every word, scream, or shout is caught by the wind and flung through all the layers of the plane. Conversation is accomplished by shouting, and even then words are spirited away by the wind beyond 3 meters. Likewise, spells and effects that rely on sonic energy have their range limited to 3 meters. Travelers are temporarily deafened after 1d10 rounds of exposure to the winds, and permanently deafened after 24 hours of exposure. Temporarily deafened characters regain their hearing after 1 hour spent out of the wind.

Ear plugs or similar devices negate the deafening effect. Of course, wearing ear-plugs effectively mimics the normal effects of being deafened.

The stale wind of Pandemonium is cold, and it steals the heat from travelers unprotected from its endless gale that buffets each inhabitant, blowing sand and dirt into eyes, snuffing torches, and carrying away loose items. In some places, the wind can howl so fiercely that it lifts creatures off their feet and carries them for miles before dashing their forms to lifeless pulp against some dark, unseen cliff face.

In a few relatively sheltered places, the wind dies down to just a breeze carrying haunting echoes from distant pans of the plane, though they are so distorted that they sound like cries of torment.

Erythnul, the Lord of Slaughter, makes his terrible domain on Pandemonium.

Pandemonium has four layers: Pandesmos, Cocytus, Phlegethon, and Agathion.

Pandemonium has the following traits.
\begin{itemize*}
\item \textbf{Objective Directional Gravity:} In the cavernous tunnels of Pandemonium, gravity is oriented toward whatever wall a creature is nearest. Thus, there is no normal concept of floor, wall and ceiling---any surface is a floor if you're near enough to it. Rare narrow tunnels exactly cancel out gravity, allowing a traveler to shoot through them at incredible speed. The layer of Phlegethon is an exception---there the normal gravity trait applies.
\item \textbf{Divinely Morphic:} Specific powerful beings such as the deity Erythnul can alter Pandemonium. Ordinary creatures find Pandemonium indistinguishable from the Material Plane (the alterable morphic trait, in other words). Spells and physical effort affect Pandemonium normally.
\item \textbf{Mildly Chaos-Aligned:} Lawful characters on the plane of Pandemonium suffer a $-2$ penalty on all Charisma-based checks.
\end{itemize*}

\subsubsection{Windstorms on Pandemonium}
The constant winds on Pandemonium can gust with howls so maddening and speeds so enormous that they become dangerous.

Those caught without shelter when one of Pandemonium's windstorms blows up are in trouble; both mind and body are in peril. A windstorm has a 10\% chance per day of blowing through a given area. Generally, a windstorm gusts through an area in 1 round.

\Table{Windstorms on Pandemonium}{lXp{25mm}}{
  \tableheader d\%
& \tableheader Effect
& \tableheader Saving Throw \\
01--10 & Flying pebbles deal 1d4 points of damage & Reflex DC 15 half \\
11--20 & Pelting stones deal 2d6 points of damage & Reflex DC 18 half \\
21--30 & Howling wind causes \spell{confusion} for 1d4+1 rounds & Will DC 15 negates \\
31--40 & Flying boulders deal 2d8 points of damage & Reflex DC 20 half \\
41--50 & Cacophonous wind causes \spell{confusion} for 2d4+1 rounds & Will DC 18 negates \\
51--60 & Wind picks up travelers, dashing them against rock wall for 2d10 points of damage & Reflex DC 22 half \\
61--70 & Screaming wind causes \spell{confusion} for 2d4+1 rounds & Will DC 20 negates \\
71--80 & Wind picks up travelers, dashing them against rock wall for 4d10 points of damage & Reflex DC 24 half \\
81--90 & Wind picks up travelers, dashing them against rock wall for 4d10 points of  damage, then blows them into a tributary of the River Styx & Reflex DC 24 half, then Reflex DC 20 negates \\
91--100 & Shrieking wind causes \spell{insanity} & Will DC 22 negates \\
}

\subsubsection{Pandesmos}
The first layer of Pandemonium has the largest caverns, with some big enough to hold entire nations. Large or small, most caverns are desolate and abandoned to the winds.

Several of Pandesmos's caverns and tunnels possess a feature in common besides the omnipresent wind. Streams of frigid water flow from cavern to cavern, some down the center of the tunnel in midair because the objective gravity exerted by each wall cancels out the others. Many of these streams, but not all, are tributaries of the River Styx.

\textbf{Madhouse:} A group of outsiders known as the Bleak Cabal maintains a citadel in Pandesmos that serves as a way station for travelers. The Madhouse is a sprawling edifice of haphazardly organized buildings divided by several circular stone walls. The citadel is so large it fills an entire cavern, covering every surface. The place is rife with travelers, petitioners, and natives. Available services include lodging and most other services one might expect in i normal city. However, a respectable percentage of the Madhouse's populace is insane, deaf, or both.

\textbf{Winter's Hall:} This region of Pandemonium is snowy and blizzard-ridden. Visibility, even when light can be had, is only a few feet. The snow never rests; the winds constantly whip it up so it coats tunnels and even creatures with a uniform layer of ice. Frost giants and winter wolves prowl the cold waste. These creatures serve a particularly cruel entity called many names but most often venerated as the Trickster.

\subsubsection{Cocytus}
The tunnels of Cocytus tend to be smaller than those of Pandesmos, which means that they funnel the winds more strongly. The resulting wails have earned Cocytus the nickname ``layer of lamentation.'' Strangely, the tunnels on this layer bear the marks of having been hand-chiseled, but such an undertaking must have occurred so long ago that years do not suffice as a measure.

\textbf{Howler's Crag:} A jagged spike of stone stands in the center of Cocytus. The Crag is a jumbled pile of stones, boulders, and worked stone, as if a giant's palace had collapsed in on itself. The Crag's top is mostly a level latform about eight feet in diameter, with a low wall surrounding it. The platform and those on it glow with an ephemeral blue radiance. The lower reaches of the Crag are riddled with small burrows. Some are merely dead ends, but others connect. The wall of every burrow is covered with lost alphabets that supposedly spell out strange psalms, liturgies, and strings of numerals or formulas.

Natives of Pandemonium say that anything yelled aloud from the top of the Crag finds the ears of the intended recipient, no matter where that recipient is on the Great Wheel. The words of the message are borne on a shrieking, frigid wind.

Demons of various sorts have learned that visitors constantly trickle to the crag. The visitors are usually archeologists, diviners, and those wishing to send a message to some lost friend or enemy. Most become the prey of the ambushing fiends.

\textbf{Harmonica:} Legend tells of a site in Cocytus called Harmonica. In this place, the winds whip through a cavern with holes and tubes chiseled into gargantuan rock columns, creating a noise worse than anywhere else in the plane. Somewhere within this mazelike realm of tortured cacophony lies the true secret of planewalking: the art of traveling the planes without a portal, spell, or device of any kind. In all likelihood, this secret is a legend with no basis in fact, but that doesn't stop the occasional seeker from finding, then dying among, the columns of Harmonica.

\subsubsection{Phlegethon}
The unrelenting noise of dripping water meshes with the howling winds of Phlegethon's narrow, twisting runnels. The rock itself absorbs light and heat. All light sources, natural and magic, only shine to half their normal distance. Unlike on the other layers, normal gravity applies in Phlegethon's tunnels, giving rise to intricate stalagmite and stalactite formations, which in turn are constantly weathered by the brutal wind.

\textbf{Windglum:} Windglum is a city of Banished in a cavern several miles wide and long, with enormous natural columns that hold up the cavern's ceiling. Hundreds of ever-burning globes provide light for the city, illuminating a disordered sprawl of individual homes. The homes in turn surround a fortification known locally as the Citadel of Loros.

Windglum is characterized by an aura of suspicion. The locals are unlikely to trust strangers, and many of Windglum's citizens are mentally unstable. However, one inn in Windglum welcomes strangers. Called the Scaly Dog, it's a place where a planar traveler can meet other wayfarers, hire mercenaries, gather information, or seek employment.

\textbf{Citadel of Slaughter:} Called ``The Many,'' the intermediate deity Erythnul is lord of envy, malice, panic, ugliness, and slaughter. Erythnul is a brutal deity who makes his home in what appears to be a tumbled ruin of some vast citadel. In fact, its tortuous passages channel cold winds on which can always be heard the sound of terrible battle. Battle-mad petitioners of all races infest the passages, and they desire nothing other than to hunt and slay each other in cold blood.

At the center of the pile is Erythnul himself, usually engaged in the slaughter of an endless stream of petitioners, as well as the occasional mortal captive. In battle, the deity's features change between human, gnoll, bugbear, ogre, and troll. If ever Erythnul's blood is spilled, it transforms into an allied creature of whatever form Erythnul currently wears.

No one goes to the Citadel of Slaughter on purpose, unless they serve Erythnul and seek to join in the deity's eternal slaughter.

\subsubsection{Agathion}
In the fourth layer, the narrowing tunnels finally constrict down to nothing, leaving behind an infinite number of closed-off spaces filled with stale air or vacuum surrounded by an infinitude of solid stone. The portals that connect Agathion to the rest of Pandemonium open into the otherwise unreachable bubbles, but the act of stepping through a portal always sets off a windstorm.

Unless you know where the portal is, the closed-off spaces of Agathion are almost impossible to find. For this reason, forgotten spaces have been used by deities (and other powerful entities that predate the current deities) as vaults where items are hidden away. Such items may include uncontrollable artifacts, precious mementos, lost languages, unborn cosmologies, and monsters of such cataclysmic power that they couldn't be slain or otherwise neutralized.

\subsectionA{Infinite Layers of The Abyss}
The Abyss is all that is ugly, all that is evil, and all that is chaotic reflected in infinite variety through layers beyond count. Its virtually endless layers spiral downward into ever more atrocious forms. Conventional wisdom places the number of layers of the Abyss at 666, though there may be far more. The whole point of the Abyss, after all, is that it's far more terrible than conventional wisdom could ever encompass.

Each layer of the Abyss has its own unique, horrific environment. No theme unifies the multifarious layers other than their harsh, inhospitable nature. Lakes of caustic acid, clouds of noxious fumes, caverns of razor-sharp spikes, and landscapes of magma are all possibilities. So are less immediately deadly terrains such as parched salt deserts, subtly poisonous winds, and plains of biting insects.

The Abyss is home to demons, creatures devoted to death and destruction. A demon in the Abyss looks upon visitors as food or a source of amusement. Some see powerful visitors as potential recruits (willing or not) in the never-ending war that pits demons against devils: the Blood War.

Demon lords and deities inhabit the Abyss, including Demogorgon, Graz'zt, Pazuzu, Blibdoolpoolp (deity of kuo-toa), Diirinka (deity of the derro), the Great Mother (deity of beholders), Gruumsh (deity of orcs), Hruggek (deity of bugbears), and many others, including the well-known deity Lolth (draw deity and queen of the demonweb pits). Other demon princes include Yeenoghu, Alzrius, Baphomet, Eldanoth, Fraz Urblu, Juiblex, Kostchtchie, Lissa'aera, Lupercio, Lynkhab, Pale Night, Verin, and Vucarik.

As noted before, the Abyss has layers beyond count, though the top layer is well-known: the Plain of Infinite Portals.

It has the following traits.
\begin{itemize*}
\item \textbf{Divinely Morphic:} Entities at least as powerful as lesser deities can alter the Abyss. Less powerful creatures find the Abyss indistinguishable from a normal Material Plane (alterable morphic trait) in that the plane can be changed by spells and physical effort.
\item \textbf{Mixed Elemental and Energy Traits:} This trait varies widely from layer to layer. In the Abyss as a whole, no one element or energy constantly dominates, though certain layers have a dominant element or energy, or a mixture of two or more.
\item \textbf{Mildly Chaos-Aligned and Mildly Evil-Aligned:} Lawful characters in the Abyss suffer a $-2$ penalty on all Charisma-based checks, and good characters suffer the same penalty. Lawful good characters suffer a $-4$ penalty on all Charisma-based checks.
\end{itemize*}

\subsubsection{Plain of Infinite Portals}
This is the topmost of the uncountable Abyssal layers. It is a barren, dusty place without life or greenery, baking beneath a hell-red sun. The dusty plains are broken by three features: huge pits in the earth, great iron strongholds, and the River Styx.

The pits of this first layer are portals to deeper layers. Dropping down a given pit soon deposits the traveler into the associated layer, though jumping into random pits that lead to unknown planes of the Abyss is insanely dangerous. Most of the pits are two-way portals, but some are only one-way, leaving travelers stranded on the new layer.

Iron strongholds here most often house powerful demons and their court. Such fortresses often serve as a rallying point for demonic armies on their way to join the endless Blood War. Some of that war's greatest battles take plane in this layer, deeper layers, and nearby Outer Planes.

The River Styx flows a winding course on this layer. Some channels pour into the pits, while other pits well up with foul water, serving as tributaries of the mighty river.

A character entering a new, unknown layer of the Abyss via a pit (or another method) can wind up in almost any sort of terrain. Develop this layer of the Abyss yourself, or use \tabref{Random Abyssal Layers} to provide guidance.

\textbf{Broken Reach:} Red Shroud, a succubus sorcerer, rules the town of Broken Reach, which serves as a gathering point for Blood War mercenaries, a way spot for travelers insane enough to explore the Abyss, and a place for trade. The town is a set of crumbling towers surrounded by outworks of trenches, walls, and spiky barricades.

Several important precincts are underground. The portal to Plague-Mort, a town in the Outlands, is beneath the main hall, for example. The food stores, the arsenal, the interrogation halls, and the crypts are likewise underground, connected by narrow tunnels. Rooms for visiting mercenaries and merchants are above ground, off the main towered hall. The inhabitants are a mix of petitioner slaves, demons of all types, and mercenaries from the Material Plane and beyond.

\textbf{Ferrug:} An abandoned iron stronghold is situated near the Lakes of Molten Iron, a series of natural whitehot crucibles filled with molten iron. Ferrug's former demonic lord was slain as she lay senseless while astrally traveling to the Material Plane to corrupt mortal hearts. Since then, Ferrug has hosted countless armies of demons interested in gathering workable iron to build other iron strong- holds. Because the demons highly value iron, devil strike forces often attack the Lakes of Molten Iron, so Ferrug currently serves as a command center for a force of demons charged by Demogorgon to protect the lakes.

\subsubsection{Random Abyssal Layers}
What if your characters wind up being sent to the Abyss as a result of an adventure gone wrong? Or what if they flee powerful demons on the Plain of Infinite Portals by jumping down the nearest pit?

Use the following table to randomly determine the general terrain type of an unknown layer. If desired, roll twice (or more) and combine the results.

\Table{Random Abyssal Layers}{lX}{
  \tableheader d\%
& \tableheader Type of Layer\\
01--05  & Air-dominant \\
06--10  & Blood War battleground (demons against devils) \\
11--15  & Burning hellscape \\
16--20  & Demonic city \\
21--25  & Desert of sand, ice, salt, or ash \\
26--30  & Earth-dominant \\
31--35  & Fire-dominant \\
36--40  & Grass plain (filled with predators) \\
41--45  & Mixed elemental-dominant (as Limbo) \\
46--50  & Mountainous \\
51--55  & Negative-dominant (minor or major) \\
56--60  & Normal (as the Material Plane) \\
61--65  & Ocean of water \\
66--70  & Realm of powerful Abyssal entity \\
71--75  & Sea of acid \\
76--80  & Sea of insects \\
81--85  & Sea of magma \\
86--90  & Subterranean \\
91--95  & Undead realm \\
96--100 & Water-dominant \\
}

\subsectionA{Tarterian Depths of Carceri}
Carceri seems the least overtly dangerous of the lower planes, but that first impression quickly disappears. Acid seas and sulfurous atmospheres may be rare on this plane, and there are no areas of biting cold or infernos of raging heat. The danger of Carceri is a subtler thing.

The plane is a place of darkness and despair, of passions and poisons, and of kingdom-shattering betrayals. On Carceri, hatreds run like a deep, slow-moving river. And there's no telling what the flood of treachery is going to consume next. It is said that a prisoner on Carceri may only escape when she has become stronger than whatever imprisoned her there. That's a difficult task on a plane whose very nature breeds despair, betrayal, and self-hatred.

Unlike most inhabitants of Carceri, the deity Nerull makes his home on Carceri willfully, not because of exile.

Carceri consists of six layers. Each layer has a series of orbs like tiny planets, in a row. A gulf of air separates each orb from the next. On a particular layer, little distinguishes one orb from the next, and it's possible that the number of orblike planets on each layer is infinite.

It has the following traits.
\begin{itemize*}
\item \textbf{Divinely Morphic:} Nerull and any other entity of lesser deity power or greater can alter Carceri. More ordinary creatures find Carceri indistinguishable from the Material Plane; it responds to spells and physical effort normally.
\item \textbf{Mildly Evil-Aligned:} Good characters on Carceri suffer a $-2$ penalty on all Charisma-based checks.
\end{itemize*}

\subsubsection{Orthrys}
Orthrys, the first layer of Carceri, is a realm of vast bogs and quicksand. The River Styx runs freely through the layer, saturating the ground with its magic. Channels carved into the soft ground through eons of erosion are wide and deep. Where there is no river, there are swamps. Though patches of dry ground exist, they are rare and usually climb swiftly to rugged mountains where enraged titans dwell.

Mosquitoes swarm the air above the bogs, annoying travelers. Even more annoying are the smooth-talking petitioners that populate this dreary realm.

\textbf{Bastion of Last Hope:} A fortress made of black igneous rock squats in a mountain range of Orthrys. The ambient, reddish light of the plane lends the Bastion of Last Hope a brooding air of menace. Only one entrance offers itself, and those entering can't help but notice that the entrance strongly resembles the maw of some massive demonic toad.

No one person rules the Bastion. Instead, it serves as a sort of outpost for anarchists. Here a traveler can obtain all manner of forged documents, surgical alterations to aid a permanent disguise, and various other nefarious goods and services. It is a good place to find assassins, spies, and others of ill repute. But cunning travelers remember that they're on a plane full of traitors, so they trust no one within the Bastion's walls.

\textbf{Mount Orthrys:} The highest peaks of the mountain ranges on two of this layer's orbs reach ridiculously high, just bridging the planetary gulf between them. At their intersection is a titanic palace of white marble columns, amphitheaters, and galleries. Here lives a race of titans, banished from the Material Plane long ago. The titan lord of Mount Orthrys, Cronus, resides at the center of his palace in a throne room a mile wide. Visitors may seek audiences with Cronus to hear his wisdom, but those who seek such counsel must be always aware that the titan's eons-long anger at his confinement may lash out unexpectedly at those who can come and go at their leisure. Cronus has the power of a lesser deity for the purposes of altering Mount Orthrys.

\subsubsection{Cathrys}
The orbs in the second layer of Carceri are covered with fetid jungles and scarlet plains. The stench of decay fills the air, a rot fueled by acidic secretions of jungle plants. Those without immunity to acid are soon rendered down to their component materials if they stay too long amid the swaying trees. The jungle air deals 1d4 points of acid damage per minute, and some plants secrete more potent acids.

The plains of Cathrys are more habitable. Vast, windswept grasslands cover the planes. Some patches possess razor-sharp leaves, which can cut a traveler not mindful of them. Those who hustle (double move) or run on the plains must make a Reflex save (DC 20) each round or cut themselves for 1d4 points of damage.

\textbf{Apothecary of Sin:} Located deep in the fetid jungles of an orb of Cathrys is the Apothecary of Sin. The Apothecary is built from cunningly woven scrap wood atop the trunk of large tree, raising the one-story structure high above the waving branches of the acid-laden leaves below. Rope-suspended catwalks provide access above the treetops, though random sections are missing, possibly victims of caustic storms. Mundane and exotic poisons and acids are bought and sold in the Apothecary.

A demon called Sinmaker runs the Apothecary. Sinmaker is a glabrezu of average abilities, except for his special affinity for acids, poisons, and venoms. He delights in all things poisonous---the more diabolical, the better. All poisons are available in the Apothecary, as well as many special, unique concoctions bought by Sinmaker from travelers or synthesized in Sinmaker's own laboratory. Acid is also sold here, by the one-dose vial or by the thousand-dose keg. Neither the size of the purchase nor the nature of the buyer matters to Sinmaker.

\subsubsection{Minethys}
The third layer of Carceri is filled with sand. Stinging grit is driven so hard by the wind that it can strip an exposed being to the bone in a matter of hours, should one of the place's terrible windstorms spring up. Sandstorms are 10\% likely in any given area per 24 hours. All who dwell in this layer, mortal and fiend alike, cover themselves in cloth garments to block out the stinging sand.

Tornadoes are common on Minethys. To avoid these hazards, petitioners live in miserable sand-filled pits, dug by hand. Their crude pits must be constantly dug out to provide even the slightest shelter.

\textbf{Sand Tombs of Payratheon:} Payratheon is the name of a vanished city built on an orb of Minethys eons ago. That city is long buried, but its sand-drowned avenues, crumbled towers, and silted porticos still remain far below the shifting surface of the layer. Sometimes the shifting sands reveal Payratheon for an hour or a longer, but it is always engulfed again by the sands, smothering most creatures who were tempted by its appearance and entered the sand-blasted city.

Particularly resourceful adventurers have burrowed down to find outlying suburbs of the city during its phases of submersion. Tales of terror walk hand in hand with these accounts, which tell of dragonlike ``sand gorgons'' that swim through the sand as if water. Also mentioned are the remnants of former inhabitants that force their way through the streets as petrified undead, so weathered and eroded that little can be discerned of their race or original size.

\subsubsection{Colothys}
The fourth layer of Carceri is a realm of mountains so tall, rough, and cruel as to stagger the imagination of a traveler from the Material Plane. Travel on foot here is almost impossible, because the land is divided by canyons miles deep where it is not lifted to absurd heights by mighty tectonics. A few trading routes do exist, usually in the form of rickety bridges and cliff-face trails barely wide enough for one.

It's impossible to move normally away from the areas along the trading routes. Characters must make \skill{Climb} checks (DC 15) to move one-half their speed as a miscellaneous full-round action.

\textbf{Garden of Malice:} The hanging gardens of Colothys are found on a single orb of the layer that travelers would do well to avoid. To the inexperienced eye, many of the cliff faces and sheer slopes of this orb are home to thick vines and tubers that sprout a riot of beautiful flowers. Characters who attempt to collect samples for their botanical collections quickly learn that the vines are animate and determined to wring the life from any creature that would dare to use them as climbing aids, defoliate the flowers, or even move too close.

It may be that the animate vines represent one large organism that has grown through the eons to cover one whole orb. Once every six hundred days, the vines release tiny seeds into the air that look like dandelion fluff. The winds of the layer often send the seeds blowing across several hundred other orbs of the mountainous realm. Though many are eaten by vermin, many other seeds have also found nourishing soil, and have sprouted tubers in small nooks and forgotten cliff-faces on other orbs.

\subsubsection{Porphatys}
The fifth layer of Carceri is a realm where each orb is coated in a cold, shallow ocean fed by constant black snow. The snow and water are mildly acidic, automatically dealing 1d6 points of acid damage per 10 minutes of direct exposure.

Artificial structures do not last long in Porphatys. Small islands barely taller than sandbars rise above the waves. Most petitioners crow from atop the small sandbar islands, promising anything to those who can take them away. Despite their entreaties, they reward any charity with betrayal at the first opportunity.

Another exiled titan lives here, but even his palace is half sunken and slowly crumbling before the acidic waves.

\textbf{Ship of One Hundred:} A ship rides the cold swells of Porphatys's seas, called the Ship of One Hundred, though in some accounts it is referred to as the White Caravel. It appears as a ghost-white caravel unmanned by any visible crew. It wends between the islets of many orbs (somehow disappearing on one orb and appearing on another), picking up stranded souls and other travelers who are brave (or foolish) enough to brave passage.

Passengers soon discover that apparently no one moves on board the craft. The lower deck and hold are stuffed with exactly one hundred unadorned stone sarcophagi. No one has ever successfully opened a sarcophagus and lived to tell the tale. Any time this has been tried, some unrecorded calamity devours all creatures currently on board, and the next time the ship puts in at a new port it is utterly empty of life. Stories have it that the ship seeks to deliver its terrible cargo, but it waits for the end times to do so.

Between the ``cleansings'' that occur when the curious try to open a sarcophagus, travelers (mostly petitioners, demons, or other creatures) infest the ship. Some make it their temporary home, happy to move from place to place by whatever mysterious force steers the ship. These denizens take a very dim view of visitors who want to open a sarcophagus.

\subsubsection{Agathys}
The coldest layer of Carceri is also the lowest---or inner-most, given the nested nature of this plane. Unlike the other layers, Agathys has only a single orb: a sphere of black ice streaked with red.

The air is bitterly cold and deals 1d2 points of cold damage each round. This layer has the minor negative-dominant trait. Petitioners here are half imbedded in the ice, their lies frozen on their lips.

\textbf{Necromanteion:} A black citadel carved out of ice is the focus of the greater deity Nerull's realm. Nerull is a deity of death and is called the Reaper, the Foe of All Good, the Bringer of Darkness, and similar names. Petitioners are frozen flush into the floors, walls, and ceilings of the Necromanteion, just as they are in the surrounding ice.

The deserted entrance to the Necromanteion leads quickly to a wide hall called the Hidden Temple, which crawls with undead of all types. The pallid, green glow of gibbering ghoul-light lanterns illuminates the area. Hundreds of onyx altars are evenly spaced around the hall, and demonic clerics constantly chant stanzas of a ghastly necromantic ritual. Besides chanting, the demonic priests spend endless hours attending grotesque experiments on necrotic flesh piled on other altars.

Nerull's throne stands at the center of the Hidden Temple. Woe betide the character who disturbs Nerull, a rust-red skeleton wearing a dull black cloak. Always clutched in Nerull's skeletal hands is his sablewood staff. Lifecutter, which projects a scythelike blade of scarlet force that has the power to slay any creature.

The Hidden Temple has several satellite chambers. Some hold food and quarters for the demonic clerics, others have cells for living captives destined to be strapped onto an onyx altar (or become food for a hungry cleric), and in some are special vaults where the relics of Nerull's faith are sealed away.

Finally, small tunnels lead deeper into the ice of the layer, supposedly connecting to vaults of horror so ghastly that even the demonic priests shy from exploring their depths. Otherworldly wailing and whispers rise up from the depths.

\subsectionA{Gray Waste of Hades}
Hades sits at the nadir of the lower planes, halfway
between two races of fiends each bent on the other's annihilation. Thus, it often sees its gray plains darkened by vast armies of demons battling equally vast armies of devils who neither ask nor give quarter. If any plane defines the nature of true evil, it is the Gray Waste.

In the Gray Waste of Hades, pure undiluted evil acts as a powerful spiritual force that drags all creatures down. Here, even the consuming rage of the Abyss and the devious plotting of the Nine Hells are subjugated to hopelessness. Apathy and despair seep into everything at the pole of evil. Hades slowly kills a visitor's dreams and desires, leaving the withered husk of what used to be a fiery sprit. Spend enough time in Hades, and visitors give up on things that used to matter, eventually giving in to total apathy.

A spiritual poison affects any creature (including outsiders) in Hades that does not possess spell resistance of 10 or more. Creatures without spell resistance 10 must make a Will save (DC 13) every twenty-four hours they spend in Hades.

A failed save deals 1 point of temporary Wisdom damage to the victim. A victim can be drained to a minimum Wisdom of 1 in this fashion. Unlike most ability score damage, Wisdom damage dealt by ``the grays'' does not heal until the victim has left Hades behind. Each point of Wisdom damage dealt in this fashion represents growing apathy, hopelessness, and despair.

This effect is concurrent with the entrapping trait of Hades. Wisdom damage taken from the grays makes it harder to make the weekly saving throws to resist the loss of all hope that the entrapping trait represents.

Hades has three layers called ``glooms.'' Uncaring malevolence that slowly crushes the spirit permeates each gloom.

It has the following traits.
\begin{itemize*}
\item \textbf{Divinely Morphic:} Entities of at least lesser deity status
can alter Hades, though few deities deign to reign in Hades. The Gray Waste has the alterable morphic trait for less powerful creatures; Hades responds normally to spells and physical effort.
\item \textbf{Strongly Evil-Aligned:} Nonevil characters in Hades suffer a –2 penalty on all Charisma-, Wisdom-, and Intelligence-based checks.
\item \textbf{Entrapping:} This is a special trait unique to Hades, although Elysium has a similar entrapping trait. A nonoutsider in Hades experiences increasing apathy and despair while there. Colors become grayer and less vivid, sounds duller, and even the demeanor of companions seems to be more hateful. At the conclusion of every week spent in Hades, any nonoutsider must make a will saving throw (DC 10 + the number of consecutive weeks in Hades). Failure indicates that the individual has fallen entirely under the control of the plane, becoming a petitioner of Hades. \\

Travelers entrapped by the inherent evil of Hades cannot leave the plane of their own volition and have no desire to do so. Memories of any previous life fade into nothingness, and it takes a wish or miracle spell to return such characters to normal.
\end{itemize*}

\subsubsection{Oinos}
The first gloom of Hades is a land of stunted trees, roving fiends, and virulent disease. But more than anything else, it is a plane ravaged by war. This is the central battlefield of the Blood War. Fiends, warrior-slaves, trained beasts, and hired mercenaries gather here to wage horrific battles on an epic scale. These battles despoil the already bleak terrain. The sounds of rending claws, clashing weapons, and screams echo across the entire layer.

\textbf{Khin-Oin the Wasting Tower:} A twenty-mile-high tower, Khin-Oin looks like nothing so much as a freestanding spinal column. Some say that's exactly what it is: the backbone of a deity slain by yugoloths. Khin-Oin plunges as deep into Oinos's gray soil as it ascends into the air, so the tower's sublevels tunnel twenty miles deep.

The Wasting Tower is ruled by an ultraloth prince named Mydianchlarus. In fact, some stories hint that the entire yugoloth race was birthed here, arising in a pit at the absolute bottom of Khin-Oin. None but yugoloths have ever held the tower, despite the constant array of fiendish armies outside.

The rooms and the floors of the tower seem to have no end. Spawning vats, magical laboratories, and meditation chambers can be found here, as can orreries, suites of rooms for yugoloths, and floors that are themselves battlegrounds and drill fields. Mydianchlarus rules from the tower's zenith, and the token of his rulership is the Siege Malicious.

Whoever rules the Wasting Tower is often referred to as the oinoloth. Any creature that can successfully invade the Wasting Tower and make it to the top chamber has the opportunity to claim the title for himself. Claiming the title involves defeating the current ruler, then sitting on the Siege Malicious. The Siege Malicious is a throne of artifact-level power, and as such, it may grant powers over the layer of Oinos.

\textbf{The Siege Malicious:} The Siege Malicious is a major artifact. It is a gargantuan, immovable throne carved from the stone of the Wasting Tower itself. The throne is inlaid with tarnished silver, base copper, and brass. A circular crown of rubies adorns the top of the high seat, which is just large enough to sit a Huge creature. (Many Medium-size creatures would look ridiculous sitting on the Siege Malicious with their legs dangling several feet off the floor.)

\textit{Powers of the Throne:} In order to operate the Siege Malicious, a character sitting on the throne must have defeated the previous oinoloth. If the previous oinoloth yet lives, the sitter suffers 3d6+6 points of permanent Charisma drain, as a consequence of being infected with a particularly virulent strain of the disease called gray wasting. Characters immune to disease don't take damage, but the Siege Malicious seems powerless to them.

If the character sitting on the throne has defeated the previous oinoloth, then the powers of the siege malicious are his. But the throne forever changes those who sit on it.

The Siege Malicious deals 1d4 points of permanent Charisma drain as part of the sitter's skin sloughs off in a rather grotesque manner. This disfigurement is the mark of the oinoloth and may not be magically healed without forsaking the title of oinoloth.

But with the disfigurement comes absolute control of disease on the layer of Oinos. The new oinoloth (whether yugoloth or not) commands the diseases of Oinos, creating, modifying, or nullifying diseases as he sees fit. New or modified diseases could potentially spread beyond the layer of Oinos, but the oinoloth only has this power while in Hades. The oinoloth has power over disease whether sitting in the Siege Malicious or not.

\textit{Creating or Modifying a Disease:} The oinoloth may conceive of or modify a disease at will as a free action (though
coming up with just the right name is an exercise of
intellect that could take longer). The important parameters
for creating or modifying a disease are infection, DC,
incubation, and damage; for more information, see Disease
in Chapter 3 of the D UNGEON M ASTER ’s Guide.
Generally, new or modified diseases must possess a
standard infection type, have a DC no higher than 20, have
an incubation time of no less than one day, and have
damage not greater than 1d8 temporary points of any
ability score damage except Constitution (1d6 if the
disease deals permanent ability drain). Secondary visual
effects of a new disease are up to the oinoloth. Secondary
effects can include deafness, blindness, muteness, and
other sensory deprivations (one per disease), on a second
failed saving throw against the initial disease DC.
Infecting: Once a disease is created or modified, the
oinoloth can set it loose. The oinoloth can infect a living
target within 300 feet as a standard action, and the target
gets no saving throw to avoid infection.

Niflheim
The second gloom of Hades is a layer of gray mists that
constantly twist and swirl among sickly trees and ominous
bluffs. The thin fog limits vision to 100 feet at best, muffles
sound, and eventually saturates everything with
dampness. Niflheim is not as war-ravaged as Oinos, prob-
ably because the mist hinders combat. Many predators
prowl the lands, hidden amid the mist, including fiendish
dire wolves and trolls.
Vision (including darkvision) is limited to 100 feet in
Niflheim, and Listen checks suffer a –4 circumstance
penalty due to the muffling nature of the fog.
Death of Innocence: A small town tucked away in the
misty pines, Death of Innocence is constructed of hewn
pine taken from the surrounding forest.
The town holds more than 5,000 mortals and (nonlarva)
petitioners, though they mostly remain inside their
dwellings, giving the city a vacant feel. Strangely, those
who live behind the protection of the town's walls
sometimes strive to improve their lot and break out of
apathy.
Great wooden gates bar entry to Death of Innocence,
and both the gates and the outer wall bristle with spikes.
Inside, a broad avenue leads to the town's center, where a
gray marble fountain stands. The wood of the buildings
and gates oozes blood, as if sap, confirming the belief that
petitioners are trapped within the wood. Neither the grays
nor the entrapping trait of Hades can penetrate the walls
of Death of Innocence.

Pluton
The third gloom of Hades is a layer of dying willows,
shriveled olive trees, and night-black poplars. It is a realm
where no one wants to be and no one can remember why
they came. Of course, petitioners have no choice in the
matter.
Usually, the Blood War does not reach this lowest
gloom, though some raids have occurred when one side or
the other wished to retrieve the spirit of a fallen mortal
captain who possessed particularly sharp tactical skills.
Underworld: The Underworld is contained within
walls of gray marble that stretch for hundreds of miles and
are visible for thousands of miles beyond that. A single
double gate pierces the marble walls of the realm.
Constructed of beaten bronze, the gates are dented and
scarred by heroes intent on getting past. However, the
gates are also guarded by a terrible fiendish beast, a
Gargantuan three-headed hound made from the
squirming, decaying bodies of hundreds of petitioners.
Beyond the gate, the inside of the realm appears much
like the outside. Blackened trees, stunted bushes, and
wasted ground dominate the landscape. Larvae are
everywhere, writhing in the dust, as are gray, wraithlike
petitioners who are on the verge of being sucked
completely dry of all emotion by the spiritual decay of the
plane. When they lose the last shred of emotion, their
remaining essence becomes one with the gloom of Pluton.
Sometimes, great heroes or desperate lovers from the
Material Plane travel to this layer via a tributary of the
River Styx or portals hidden in great volcanic fissures.
They come to the Underworld because they believe that
they can find the spirit of a friend or loved one and
extricate that spirit from a hopeless eternity. Besides
larvae, faded petitioners, and the occasional foolish
mortals, demons, yugoloths, and devils roam the land,
looking for choice morsels.
% \input{subsections/planes/outer-planes/gehenna.tex}
% \input{subsections/planes/outer-planes/baator.tex}
% \input{subsections/planes/outer-planes/acheron.tex}
% \input{subsections/planes/outer-planes/mechanus.tex}
% \input{subsections/planes/outer-planes/arcadia.tex}
% \input{subsections/planes/outer-planes/celestia.tex}
% \input{subsections/planes/outer-planes/bytopia.tex}
% \input{subsections/planes/outer-planes/elysium.tex}
% \input{subsections/planes/outer-planes/beastlands.tex}
% \input{subsections/planes/outer-planes/arborea.tex}
% \input{subsections/planes/outer-planes/outlands.tex}
