\subsectionA{General Feats}

\Feat{Ability Focus}
{Choose one of the creature's special attacks.}
{Special attack.}
{Add +2 to the DC for all saving throws against the special attack on which the creature focuses.}
{}
{A creature can gain this feat multiple times. Its effects do not stack. Each time the creature takes the feat it applies to a different special attack.}

\Feat{Ancestral Knowledge}
{You know legends and facts about long past events that have been shrouded by the sands of time.}
{Int 13, \skill{Knowledge} (history) 10 ranks.}
{Choose one of the following time periods: Blue Age, Green Age, or Cleansing Wars. You gain a +10 bonus on \skill{Knowledge} (history) checks or bardic knowledge checks to gain information about the chosen category.}{}
{You can take this feat more than once, but the bonus doesn't stack. Each time you take this feat, you choose another time period.}

% \Feat{Antipsionic Magic}
% {Your spells are more potent when used against psionic characters and creatures.}
% {\skill{Spellcraft} 5 ranks.}
% {You get a get a +2 bonus on caster level checks made to overcome a psionic creature's power resistance.

% This bonus stacks with the bonus conferred by Spell Penetration and Greater Spell Penetration. Moreover, whenever a psionic creature attempts to dispel a spell you cast, it makes its manifester level check against a DC of 13 + its manifester level.

% The benefits of this feat apply only to power resistance.

% % The bonus does not apply to spell resistance. This is an exception to the psionics-magic transparency rule.
% }{}
% {You cannot take or use this feat if you have the ability to use powers (if you have a power point reserve or psi-like abilities).}

\Feat{Arena Clamor}
{With your savage blows, you can make your companions give their best.}
{Cha 13, \feat{Improved Critical}, \skill{Perform} (arena fighting) 5 ranks.}
{Whenever you confirm a critical hit, all allies within a 18-meter radius who have line of sight on you receive a +2 morale bonus on attack rolls for 1 round. This is a mind-affecting effect. This effect is not cumulative. Characters cannot be affected more than once in this way in the same combat.}{}{}

\GFeat{Arena Fighter}
{Martial prowess, stance (arena guile).}
{
Your gladiator and fighter levels stack for the purpose of determining your bonus from any martial prowess technique. For example, a 5th-level gladiator/7th-level fighter with bravery and tortoise style would gain +6 on Will saves against mind-affecting abilities and would improve the shield bonus by 3, as if he was a 12th-level fighter.

Your gladiator and fighter levels also stack for the purpose of determining which stances you have access to. For example, a 5th-level gladiator/7th-level fighter would have access to keen eye and scoundrel pose, as if he was a 12th-level gladiator.
}

\GFeat{Arena Performer}
{Bardic music, gladiatorial performance.}
{
Your bard and gladiator levels stack for the purpose of determining your bonus from inspire courage. For example, a 3rd-level bard/5th-level gladiator would give +2 morale bonus with inspire courage, as if she were an 8th-level bard.

Your bard and gladiator levels also stack for the purpose of determining the bonus and penalties from dirty trick. For example, a 3rd-level bard/5th-level gladiator could give +2 morale bonus or -2 morale penalty with dirty trick, as if she were an 8th-level gladiator.

Additionally, you may expend a bardic music daily use to start a gladiatorial performance, and you may use \skill{Perform} (arena fighting) to start a bardic music.
}

% \GFeat{Arena Psionicist}
% {Gladiatorial performance, psychic warrior level 1st.}
% {
% Your gladiator and psychic warrior levels stack for the purpose of determining which gladiatorial performances you have access. For example, a 8th-level gladiator/1st-level psychic warrior would have access to shake off, as if he was a 9th-level gladiator.

% Your gladiator and psychic warrior levels also stack for the purpose of determining your manifester level. For example, a 8th-level gladiator/1st-level psychic warrior with a Wisdom score of 16 would have 13 additional power points from his ability score and could spend 9 power points in a single psionic power, as if he was a 9th-level psychic warrior.
% }

\Feat{Armor Proficiency (Heavy)}
{}
{Armor Proficiency (light), Armor Proficiency (medium).}
{See Armor Proficiency (light).}
{See Armor Proficiency (light).}
{Fighters, psychic warriors, and clerics automatically have Armor Proficiency (heavy) as a bonus feat. They need not select it.}

\Feat{Armor Proficiency (Light)}
{}{}
{When you wear a type of armor with which you are proficient, the armor check penalty for that armor applies only to \skill{Balance}, \skill{Climb}, \skill{Escape Artist}, \skill{Hide}, \skill{Jump}, \skill{Move Silently}, \skill{Sleight of Hand}, and \skill{Tumble} checks.}
{A character who is wearing armor with which she is not proficient applies its armor check penalty to attack rolls and to all skill checks that involve moving, including Ride.}
{All characters except wizards, and psions automatically have Armor Proficiency (light) as a bonus feat. They need not select it.}

\Feat{Armor Proficiency (Medium)}
{}
{Armor Proficiency (light).}
{See Armor Proficiency (light).}
{See Armor Proficiency (light).}
{Fighters, barbarians, gladiators, psychic warriors, clerics, druids, and templars automatically have Armor Proficiency (medium) as a bonus feat. They need not select it.}

\Feat{Armored Stealth}
{}
{Dex 13, \skill{Hide} 4 ranks, \skill{Move Silently} 4 ranks, \feat{Armor Proficiency (Light)}, \feat{Stealthy}.}
{You do not apply armor check penalty to \skill{Hide} and \skill{Move Silently} checks, whenever you are wearing a light armor.}
{You apply the armor check penalty to all skills with the notation (see \chapref{Skills}), including \skill{Hide} and \skill{Move Silently}.}
{}

\GFeat{Augment Summoning}
{\feat{Spell Focus} (conjuration).}
{Each creature you conjure with any summon spell gains a +4 enhancement bonus to Strength and Constitution for the duration of the spell that summoned it.}

\Feat{Brutal Attack}
{Your decisive attacks are especially frightening for those who watch.}
{Cha 13, \feat{Improved Critical}, \skill{Perform} (arena fighting) 5 ranks.}
{Whenever you confirm a critical hit, all enemies within a 3-meter radius who have line of sight on you must make a Will save (DC 10 + \onehalf your character level + your Cha modifier) or become shaken for a number of rounds equal to your Cha modifier. This is a mind-affecting fear effect.

Whether or not the save is successful, that creature cannot be affected again by the same character's brutal attack for 24 hours.}{}{}

\Feat{Bug Trainer}
{You can train vermin creatures, such as kanks and cilops.}
{\skill{Handle Animal} 5 ranks, \skill{Knowledge} (nature) 5 ranks.}
{You can use the \skill{Handle Animal} skill for vermin as though they were animals with an Intelligence score of 1.}
{You can use the \skill{Handle Animal} skill only on creatures with an Intelligence score of 1 or 2.}
{}

\Feat{Chaotic Mind}
{The turbulence of your thoughts prevents others from gaining insight into your actions.}
{Chaotic alignment, Cha 15.}
{Creatures and characters who have an insight bonus on their attack rolls, an insight bonus to their Armor Class, or an insight bonus on skill checks or ability checks do not gain those bonuses against you.

The benefit of this feat applies only to insight bonuses gained from psionic powers and psi-like abilities. %This is an exception to the psionics-magic transparency rule.
}{}
{You cannot take or use this feat if you have the ability to use powers (if you have a power point reserve or psi-like abilities).}

\Feat{Cloak Dance}
{You are skilled at using optical tricks to make yourself seem to be where you are not.}
{\skill{Hide} 10 ranks, \skill{Perform} (dance) 2 ranks.}
{You can take a move action to obscure your exact position. Until your next turn, you have concealment. Alternatively, you can take a full-round action to entirely obscure your exact position. Until your next turn, you have total concealment.}{}{}

\Feat{Closed Mind}
{Your mind is better able to resist psionics than normal.}{}
{You get a +2 bonus on all saving throws to resist powers.

The benefit of this feat applies only to psionic powers and psi-like abilities. %This is an exception to the psionics-magic transparency rule.
}{}
{You cannot take or use this feat if you have the ability to use powers (if you have a power point reserve or psi-like abilities).}

\GFeat{Combat Casting}{}
{You get a +4 bonus on Concentration checks made to cast a spell or use a spell-like ability while on the defensive or while you are grappling or pinned.}

\Feat{Cornered Fighter}
{You fight better when you freedom is put at risk.}
{Base attack bonus +5.}
{You receive a +2 bonus on attack rolls and a +2 bonus to AC when fighting against opponents who flank you.}
{}{}

\Feat{Deadly Precision}
{You empty your mind of all distracting emotion, becoming an instrument of deadly precision.}
{Dex 15, base attack bonus +5.}
{You have deadly accuracy with your sneak attacks. You can reroll any result of 1 on your sneak attack's extra damage dice. You must keep the result of the reroll, even if it is another 1.}{}{}

\Feat{Defender of the Land}
{You share power with the spirit from your guarded land, to nurture and protect the land to which the spirit is tied.}
{Wild shape class feature.}
{You receive a +1 caster level on spells you cast against defilers and your spells damage is increased by 1 per die against defilers.}
{}{}

\Feat{Diehard}
{}
{\feat{Endurance}.}
{When reduced to between $-1$ and $-9$ hit points, you automatically become stable. You don't have to roll d\% to see if you lose 1 hit point each round.

When reduced to negative hit points, you may choose to act as if you were disabled, rather than dying. You must make this decision as soon as you are reduced to negative hit points (even if it isn't your turn). If you do not choose to act as if you were disabled, you immediately fall unconscious.

When using this feat, you can take either a single move or standard action each turn, but not both, and you cannot take a full round action. You can take a move action without further injuring yourself, but if you perform any standard action (or any other action deemed as strenuous, including some free actions, swift actions, or immediate actions, such as casting a quickened spell) you take 1 point of damage after completing the act. If you reach $-10$ hit points, you immediately die.}
{A character without this feat who is reduced to between $-1$ and $-9$ hit points is unconscious and dying.}
{}

\Feat{Dissimulated}
{Your ability to speak what others want to hear increases the credibility of your words.}
{Int 13, Cha 13, \skill{Bluff} 5 ranks.}
{In addition to your Charisma modifier, you can add your Intelligence modifier to your \skill{Bluff} checks.}
{}{}

\Feat{Drake's Child}
{You are what is known as a drake's child, an individual who shows both exceptional strength and wisdom.}
{Str 13, Wis 13.}
{You get a +1 bonus to Will saves and a +1 bonus to Fortitude saves. You gain an additional +1 bonus to saving throws against ability drain, ability damage, energy drain, and death effects.}
{}{}

\GFeat{Druidic Hunter}
{Animal companion, wild shape.}
{
Your druid and ranger levels stack for the purpose of determining your effective ranger level for the purpose of determining the bonus Hit Dice, extra tricks, special abilities, and other bonuses that your animal companion receives. For example, a 5th-level druid/4th-level ranger's animal companion would gain +4 HD and have a total of 3 bonus tricks, as if he was a 9th-level ranger.
}


\Feat{Elemental Cleansing}
{Undead you turn or rebuke suffer elemental damage.}
{Ability to turn or rebuke undead.}
{Any undead that you successfully turn or rebuke takes 2d6 points of energy damage in addition to the normal turning or rebuking effect. The type of damage dealt is the one associated with your patron element.}
{}{}

\Feat{Empower Spell-Like Ability}
{}
{Spell-like ability at caster level 6th or higher.}
{Choose one of the creature's spell-like abilities, subject to the restrictions below. The creature can use that ability as an empowered spell-like ability three times per day (or less, if the ability is normally usable only once or twice per day).

When a creature uses an empowered spell-like ability, all variable, numeric effects of the spell-like ability are increased by one half. Saving throws and opposed rolls are not affected. Spell-like abilities without random variables are not affected.

The creature can only select a spell-like ability duplicating a spell with a level less than or equal to half its caster level (round down) $-2$. For a summary, see the table in the description of the \feat{Quicken Spell-Like Ability} feat.}
{}
{This feat can be taken multiple times. Each time it is taken, the creature can apply it to a different one of its spell-like abilities.}

\Feat{Endurance}
{}
{You gain a +4 bonus on the following checks and saves: Swim checks made to resist nonlethal damage, Constitution checks made to continue running, Constitution checks made to avoid nonlethal damage from a forced march, Constitution checks made to hold your breath, Constitution checks made to avoid nonlethal damage from starvation or thirst, Fortitude saves made to avoid nonlethal damage from hot or cold environments, and Fortitude saves made to resist damage from suffocation. Also, you may sleep in light or medium armor without becoming fatigued.}
{A character without this feat who sleeps in medium or heavier armor is automatically fatigued the next day.}{}
{A ranger automatically gains Endurance as a bonus feat at 3rd level. He need not select it.}

\GFeat{Eschew Materials}{}
{You can cast any spell that has a material component costing 1 cp or less without needing that component. (The casting of the spell still provokes attacks of opportunity as normal.) If the spell requires a material component that costs more than 1 cp, you must have the material component at hand to cast the spell, just as normal.}

\Feat{Extra Music}
{}
{Bardic music ability.}
{Each time you take this feat, you can use your bardic music ability four more times per day than normal.}
{}
{You can gain Extra Music multiple times. Its effects stack. Each time you take the feat, you can use bardic music four additional times per day.}

\Feat{Extra Performance}
{You can make gladiatorial performances more often than normal.}
{Gladiatorial performance ability.}
{Each time you take this feat, you can use your gladiatorial performance ability four more times per day than normal.}
{}
{You can gain Extra Performance multiple times. Its effects stack. Each time you take the feat, you can use gladiatorial performance four additional times per day.}

\Feat{Extra Turning}
{}
{Ability to turn or rebuke creatures.}
{Each time you take this feat, you can use your ability to turn or rebuke creatures four more times per day than normal.

If you have the ability to turn or rebuke more than one kind of creature each of your turning or rebuking abilities gains four additional uses per day.}
{Without this feat, a character can typically turn or rebuke undead (or other creatures) a number of times per day equal to 3 + his or her Charisma modifier.}
{You can gain Extra Turning multiple times. Its effects stack. Each time you take the feat, you can use each of your turning or rebuking abilities four additional times per day.}

\Feat{Faithful Follower}
{You overcome your fears while being led.}{}
{You receive a +5 morale bonus on saving throws against fear effects whenever you are within 6 meters of an ally with the \feat{Leadership} feat.}{}{}

\Feat{Favorite}
{You have gained the graces of your sorcerer-monarch, receiving extra benefits.}
{\feat{Secular Authority}, \skill{Diplomacy} 10 ranks.}
{You can use your secular authority ability four more times per day than normal. Furthermore, whenever you contest or are contested in the use of secular authority, you receive a +2 bonus on your opposed \skill{Diplomacy} check.}
{Without this feat, a templar can typically use secular authority only once per day per templar level.}
{You can gain Favorite multiple times. Its effects stack. Each time you take the feat, you can use secular authority four additional times per day.}

\Feat{Fearsome}
{Your might frightens your foes.}
{Str 15.}
{You can use your Strength modifier instead of your Charisma modifier on \skill{Intimidate} checks. Additionally, you receive a +2 bonus on \skill{Intimidate} checks.}{}{}

\Feat{Flyby Attack}
{}
{Fly speed.}
{When flying, the creature can take a move action (including a dive) and another standard action at any point during the move. The creature cannot take a second move action during a round when it makes a flyby attack.}
{Without this feat, the creature takes a standard action either before or after its move.}
{}

\Feat{Force of Will}
{You are able to resist psionic attacks with extreme force of will.}
{\feat{Iron Will}.}
{Once per round, when targeted by a psionic effect that allows a Reflex save or a Fortitude save, you can instead make a Will saving throw to avoid the effect.

The benefit of this feat applies only to psionic powers and psi-like abilities. %This is an exception to the psionics-magic transparency rule.
}{}
{You cannot take or use this feat if you have the ability to use powers (if you have a power point reserve or psi-like abilities).}

\Feat{Greasing the Wheels}
{You can circumvent various official obstacles when a person in a position of trust or authority is willing to accept ``presents.''}
{Cha 13, \skill{Diplomacy} 7 ranks, \skill{Knowledge} (local) 5 ranks.}
{This feat grants a new use for the \skill{Diplomacy} skill. You must share a language with a creature in order to use this option. This option cannot be used during combat.

% \textit{Bribery Etiquette:} You can discern the timing of the offer, the amount that will most likely garner the wanted reaction, and the best way to disguise the bribe so that it doesn't draw attention from unwanted witnesses. An insulted character will have his attitude changed one step for the worse and might report you to the proper authorities (this can be negated by a successful \skill{Diplomacy} check, albeit with a $-10$ penalty).

% To bribe a character, you must give him a number of ceramic pieces (in coins, items or other valuables) as shown below.

\textit{Bribe Official:} You can discern the timing of the offer, the amount that will most likely garner the wanted reaction, and the best way to disguise the bribe so that it doesn't draw attention from unwanted witnesses.

You can help a \skill{Diplomacy} check to improve an NPC attitude by bribing. To determine the amount of ceramic pieces, roll the base bribe described in the table below and multiply it by the target DC (see \tabref{Diplomacy DCs}).

\Table{}{l R}{
\tableheader NPC station & \tableheader Base Bribe\\
Peasant or slave    &  2d4 cp\\
Freeman or soldier  &  3d8 cp\\
Merchant or officer & 5d10 cp\\
Noble or general    & 5d10 $\times$ 10 cp\\
% Templar or guard, low-level (1--4) & 12 cp\\
% Templar or guard, mid-level (5--8) & 20 cp\\
% Templar or guard, high-level (9+) & 30 cp
}

Several factors can affect the amount needed to bribe a character, but the DM may modify these values as he sees fit.

\Table{}{X c}{
\tableheader PC Background & \tableheader Bribe Modifier\\
Renown             & $-2$\\
Illegal profession & +5\\
PC is wanted       & $\times2$\\
Slave              & +2\\
Noble              & $-2$\\
Templar            & $\times$\onehalf\\
}
}{}{}

\GFeat{Great Fortitude}{}
{You get a +2 bonus on all Fortitude saving throws.}

\Feat{Greater Spell Focus}
{Choose a school of magic to which you already have applied the \feat{Spell Focus} feat.}{}
{Add +1 to the Difficulty Class for all saving throws against spells from the school of magic you select. This bonus stacks with the bonus from \feat{Spell Focus}.}{}
{You can gain this feat multiple times. Its effects do not stack. Each time you take the feat, it applies to a new school of magic to which you already have applied the \feat{Spell Focus} feat.}{}

\Feat{Greater Counterspell}
{}
{\skill{Spellcraft} 8 ranks, \feat{Improved Counterspell}.}
{You may counterspell using an immediate action, instead of readying action.}
{Without this feat, you may counter a spell only by choosing to ready action.}
{}

\GFeat{Greater Spell Penetration}
{\feat{Spell Penetration}.}
{You get a +2 bonus on caster level checks (1d20 + caster level) made to overcome a creature's spell resistance. This bonus stacks with the one from Spell Penetration.}

\GFeat{Grovel}
{Eldaarich, \skill{Perform} 1 rank.}
{By dramatically throwing yourself prone and helpless on the ground, you get a +3 bonus to all \skill{Diplomacy} checks and \skill{Bluff} checks.}

\Feat{Hard as Rock}
{You are resolute while fighting and particularly tough to kill.}
{Con 15, \feat{Diehard}, \feat{Endurance}.}
{You are immune to death from massive damage.

You can use your Constitution modifier in place of your Dexterity modifier on Reflex saves.}
{}{}

\Feat{Hostile Mind}
{Your mind recoils violently against those who use psionics against you.}
{Cha 15.}
{Whenever you are subject to a power from the telepathy discipline (regardless of whether the power is harmful or beneficial to you), the manifester must make a Will saving throw against a DC of 10 + \onehalf your character level + your Charisma bonus or take 2d6 points of damage.

The benefit of this feat applies only to psionic powers and psi-like abilities. %This is an exception to the psionics-magic transparency rule.
}{}
{You cannot take or use this feat if you have the ability to use powers (if you have a power point reserve or psi-like abilities).}

\Feat{Hover}
{}
{Fly speed.}
{When flying, the creature can halt its forward motion and hover in place as a move action. It can then fly in any direction, including straight down or straight up, at half speed, regardless of its maneuverability.

If a creature begins its turn hovering, it can hover in place for the turn and take a full-round action. A hovering creature cannot make wing attacks, but it can attack with all other limbs and appendages it could use in a full attack. The creature can instead use a breath weapon or cast a spell instead of making physical attacks, if it could normally do so.

If a creature of Large size or larger hovers within 6 meters of the ground in an area with lots of loose debris, the draft from its wings creates a hemispherical cloud with a radius of 18 meters. The winds so generated can snuff torches, small campfires, exposed lanterns, and other small, open flames of non-magical origin. Clear vision within the cloud is limited to 3 meters. Creatures have concealment at 4.5 to 6 meters (20\% miss chance). At 7.5 meters or more, creatures have total concealment (50\% miss chance, and opponents cannot use sight to locate the creature).

Those caught in the cloud must succeed on a \skill{Concentration} check (DC 10 + onehalf creature's HD) to cast a spell.}
{Without this feat, a creature must keep moving while flying unless it has perfect maneuverability.}
{}

\Feat{Improved Counterspell}
{}
{\skill{Concentration} 4 ranks, \skill{Knowledge} (arcana) 2 ranks, \skill{Spellcraft} 4 ranks.}
{When counterspelling, you may use a spell of the same school that is one or more spell levels higher than the target spell.}
{Without this feat, you may counter a spell only with the same spell or with a spell specifically designated as countering the target spell.}{}

\Feat{Improved Familiar}
{This feat allows spellcasters to acquire a new familiar from a nonstandard list, but only when they could normally acquire a new familiar.}
{Ability to acquire a new familiar, sufficiently high level (see below).}
{When choosing a familiar, the creatures listed below are also available to the spellcaster. The spellcaster may choose a familiar with an alignment up to one step away on each of the alignment axes (lawful through chaotic, good through evil).

\Table{Improved Familiars}{p{2cm} X Z{1.2cm}}{
\tableheader Familiar & \tableheader Condition & \tableheader Arcane Spellcaster Level\\
Black/Gray Touched & Ability to channel energy from the Black or the Grey & 3rd\\
Boneclaw, lesser &  Neutral Alignment & 3rd\\
Pterrax & Reptilian subtype or ability to manifest psionic powers & 3rd\\
Element-touched\footnotemark[1] & Matching subtype or patron element, or preserver & 5th\\
Paraelement-touched\footnotemark[1] & Matching subtype or patron element, or defiler & 5th\\
Tagster & Preserver & 5th\\
Dagorran & Neutral alignment & 5th\\
Elemental, Small & Matching subtype or patron element, or preserver & 5th\\
Paraelemental, Small & Matching subtype or patron element, or defiler & 5th\\
Boneclaw, greater & Neutral alignment or ability to manifest psionic powers & 7th\\
Tigone & Neutral alignment or preserver & 7th\\
Tembo & Defiler or ability to manifest psionic powers & 7th\\
Wall walker & Neutral alignment or defiler & 7th\\
Psionocus & Ability to manifest psionic powers (The master must first create the psionocus.) & 7th\\

\TableNote{3}{1 Apply the template to a familiar from the standard list.}\\
}



Improved familiars otherwise use the rules for regular familiars, with two exceptions: If the creature's type is something other than animal, its type does not change; and improved familiars do not gain the ability to speak with other creatures of their kind (although many of them already have the ability to communicate).}{}{}

\Feat{Improved Natural Armor}
{}
{Natural armor, Con 13.}
{Your natural armor bonus increases by 1.}
{}
{You can gain this feat multiple times. Its effects stack. Each time you take this feat, your natural armor bonus increases by another point.}

\Feat{Improved Natural Attack}
{}
{Natural weapon, base attack bonus +4.}
{Choose one of the your natural attack forms. The damage for this natural weapon increases by one step, as if the your size had increased by one category: 1d2, 1d3, 1d4, 1d6, 1d8, 2d6, 3d6, 4d6, 6d6, 8d6, 12d6.

A weapon or attack that deals 1d10 points of damage increases as follows: 1d10, 2d8, 3d8, 4d8, 6d8, 8d8, 12d8.}
{}
{You can gain this feat multiple times. Its effects do not stack. Each time you take the feat, it applies to a different natural attack.

You may choose your unarmed strike as natural attack.}

\Feat{Improved Sigil}
{You templar sigil has been imbued with special powers.}
{Sigil ability, \skill{Diplomacy} 9 ranks.}
{Choose two 1st-level divine spells on your spell list. You can use them once per day as a spell-like ability. You must grasp and hold your sigil to use this ability. The save DC for these spell-like abilities is 11 + your Charisma modifier.}{}{}

\GFeat{Improved Turning}
{Ability to turn or rebuke creatures.}
{You turn or rebuke creatures as if you were one level higher than you are in the class that grants you the ability.}

\Feat{Improviser}
{You are adept at using makeshift weapons.}
{Wis 13, base attack bonus +3.}
{Whenever using improvised weapons in combat, you suffer a $-1$ penalty on attack rolls made with them.}
{Whenever using improvised weapons in combat, you suffer a $-4$ penalty on attack rolls made with them.}{}

\Feat{Innate Hunter}
{You are an excellent hunter, capable to find sustenance even in the most desolate areas.}
{\feat{Track}, \skill{Survival} 5 ranks.}
{You receive a +4 insight bonus on \skill{Survival} checks involving hunting. You also receive a +1 insight bonus on attack rolls when fighting with creatures with the animal type.}{}{}

\GFeat{Iron Will}{}
{You get a +2 bonus on all Will saving throws.}

% \GFeat{Jaguar Roar}
% {Cha 13, Draj, \skill{Intimidate} 9 ranks.}
% {Making a jaguar roar is a swift action. All intelligent creatures who can hear you and who are within 9 meters may become shaken for 2d4 rounds. A creature in the affected area can resist the effect with a successful Will save (DC 10 + your level + Charisma modifier). Any creature that successfully resists the effect cannot be affected again by the same character's jaguar roar for 24 hours.}

\Feat{Kiltektet}
{The Kiltektet is a group consisting mostly, but not solely, of kreen dedicated to hunting for knowledge and spreading it.}{}
{All Knowledge skills are class skills for you.}{}{}

\GFeat{Leadership}
{Character level 6th.}
{Having this feat enables the character to attract loyal companions and devoted followers, subordinates who assist her. See the table below for what sort of cohort and how many followers the character can recruit.

\Table{Leadership}{p{2cm} C *{6}{Z{.4cm}}}{
\rowcolor{white}
\multirow{2}{2cm}{\tableheader Leadership Score} & \multirow{2}{*}{\tableheader Cohort Level} & \multicolumn{6}{c}{\tableheader Number of Followers by Level}\\
\cmidrule[0.5pt]{3-8}
             &      & 1st & 2nd & 3rd & 4th & 5th & 6th \\
1 or lower   &      &     &     &     &     &     &     \\
2            &  1st &     &     &     &     &     &     \\
3            &  2nd &     &     &     &     &     &     \\
4            &  3rd &     &     &     &     &     &     \\
5            &  3rd &     &     &     &     &     &     \\
6            &  4th &     &     &     &     &     &     \\
7            &  5th &     &     &     &     &     &     \\
8            &  5th &     &     &     &     &     &     \\
9            &  6th &     &     &     &     &     &     \\
10           &  7th &   5 &     &     &     &     &     \\
11           &  7th &   6 &     &     &     &     &     \\
12           &  8th &   8 &     &     &     &     &     \\
13           &  9th &  10 &   1 &     &     &     &     \\
14           & 10th &  15 &   1 &     &     &     &     \\
15           & 10th &  20 &   2 &  1  &     &     &     \\
16           & 11th &  25 &   2 &  1  &     &     &     \\
17           & 12th &  30 &   3 &  1  &  1  &     &     \\
18           & 12th &  35 &   3 &  1  &  1  &     &     \\
19           & 13th &  40 &   4 &  2  &  1  &  1  &     \\
20           & 14th &  50 &   5 &  3  &  2  &  1  &     \\
21           & 15th &  60 &   6 &  3  &  2  &  1  &  1  \\
22           & 15th &  75 &   7 &  4  &  2  &  2  &  1  \\
23           & 16th &  90 &   9 &  5  &  3  &  2  &  1  \\
24           & 17th & 110 &  11 &  6  &  3  &  2  &  1  \\
25 or higher & 17th & 135 &  13 &  7  &  4  &  2  &  2  \\
}

\textit{Leadership Score:} A character's base Leadership score equals his level plus any Charisma modifier. In order to take into account negative Charisma modifiers, this table allows for very low Leadership scores, but the character must still be 6th level or higher in order to gain the Leadership feat. Outside factors can affect a character's Leadership score, as detailed above.

\textit{Cohort Level:} The character can attract a cohort of up to this level. Regardless of a character's Leadership score, he can only recruit a cohort who is two or more levels lower than himself. The cohort should be equipped with gear appropriate for its level. A character can try to attract a cohort of a particular race, class, and alignment. The cohort's alignment may not be opposed to the leader's alignment on either the law-vs-chaos or good-vs-evil axis, and the leader takes a Leadership penalty if he recruits a cohort of an alignment different from his own.

Cohorts earn XP as follows:

\begin{enumerate*}
\item The cohort does not count as a party member when determining the party's XP.
\item Divide the cohort's level by the level of the PC with whom he or she is associated (the character with the Leadership feat who attracted the cohort).
\item Multiply this result by the total XP awarded to the PC and add that number of experience points to the cohort's total.
\end{enumerate*}

If a cohort gains enough XP to bring it to a level one lower than the associated PC's character level, the cohort does not gain the new level---its new XP total is 1 less than the amount needed attain the next level.

\textit{Number of Followers by Level:} The character can lead up to the indicated number of characters of each level. Followers are similar to cohorts, except they're generally low-level NPCs. Because they're generally five or more levels behind the character they follow, they're rarely effective in combat.

Followers don't earn experience and thus don't gain levels. However, when a character with Leadership attains a new level, the player consults the table above to determine if she has acquired more followers, some of which may be higher level than the existing followers. (You don't consult the table to see if your cohort gains levels, however, because cohorts earn experience on their own.)

\textit{Leadership Modifiers:} Several factors can affect a character's Leadership score, causing it to vary from the base score (character level + Cha modifier). A character's reputation (from the point of view of the cohort or follower he is trying to attract) raises or lowers his Leadership score, see \tabref{Reputation}.

\Table{Reputation}{X c}{
\tableheader Leader's Reputation & \tableheader Modifier\\
Great renown & +2\\
Fairness and generosity & +1\\
Special power & +1\\
Failure & $-1$\\
Aloofness & $-1$\\
Cruelty & $-2$
}

Other modifiers may apply when the character tries to attract a cohort, see \tabref{Attracting Cohorts}.

\Table{Attracting Cohorts}{X c}{
\tableheader The Leader... & \tableheader Modifier\\
Has a familiar, special mount, or animal companion & $-2$\\
Recruits a cohort of a different alignment & $-1$\\
Caused the death of a cohort & $-2$ per cohort killed
}

Followers have different priorities from cohorts. When the character tries to attract a new follower, use any of the modifiers that apply on \tabref{Attracting Followers}.

\Table{Attracting Followers}{X c}{
\tableheader The Leader... & \tableheader Modifier\\
Has a stronghold, base of operations, guildhouse, or the like & +2\\
Moves around a lot & $-1$\\
Caused the death of other followers & $-1$
}}

\GFeat{Lightning Reflexes}{}
{You get a +2 bonus on all Reflex saving throws.}

\Feat{Linguist}
{You have an ear for language.}{}
{Speak Language is a class skill to you. You can also speak 2 additional languages.}{}
{This feat must be selected at 1st level.}

\GFeat{Martial Performer}
{Bardic music, martial prowess.}
{
Your bard and fighter levels stack for the purpose of determining your bonus from any martial prowess technique. For example, a 4th-level bard/8th-level fighter with bravery and tortoise style would gain +6 on Will saves against mind-affecting abilities and would improve the shield bonus by 3, as if he was a 12th-level fighter.

Your bard and fighter levels also stack for the purpose of determining the bonuses given by inspire courage. For example, a 4th-level bard/8th-level fighter would give +2 morale bonus with inspire courage, as if he was a 12th-level bard.

Additionally, you may forgo your attack with the lowest base attack bonus in a total attack to start or concentrate on a bardic music.
}

% \GFeat{Martial Psionicist}
% {Martial prowess, psychic warrior level 1st.}
% {
% Your fighter and psychic warrior levels stack for the purpose of determining your bonus from any martial prowess technique. For example, a 5th-level fighter/1st-level psychic warrior with tortoise style would improve the shield bonus by 3, as if he was a 6th-level fighter.

% Your fighter and psychic warrior levels also stack for the purpose of determining your manifester level. For example, a 5th-level fighter/1st-level psychic warrior with a Wisdom score of 16 would have 9 additional power points from his ability score and could spend 6 power points in a single psionic power, as if he was a 6th-level psychic warrior.
% }

\Feat{Martial Weapon Proficiency}
{Choose a type of martial weapon. You understand how to use that type of martial weapon in combat.}
{}
{You make attack rolls with the selected weapon normally.}
{When using a weapon with which you are not proficient, you take a $-4$ penalty on attack rolls.}
{Barbarians, fighters, gladiators, psychic warriors, and rangers are proficient with all martial weapons. They need not select this feat.

You can gain Martial Weapon Proficiency multiple times. Each time you take the feat, it applies to a new type of weapon.

A Hamanu's templar, because of the the War domain, automatically gains the Martial Weapon Proficiency feat related to his sorcerer-monarchs's favored weapon as a bonus feat, the longsword. He need not select it.}

\Feat{Mastyrial Blood}
{You have an uncanny resistance against toxic substances.}
{Con 13.}
{You receive a +4 bonus on saving throws against poison.}{}
{This feat must be selected at 1st level.}

\Feat{Mental Resistance}
{Your mind is armored against mental intrusion.}
{Base Will save bonus +2.}
{Against psionic attacks that do not employ an energy type to deal damage you gain damage reduction 3/--. In addition, when you are hit with ability damage (but not ability drain or ability burn damage) from a psionic attack, you take 3 points less than you would normally take.

The benefit of this feat applies only to psionic powers and psi-like abilities. %This is an exception to the psionics-magic transparency rule.
}{}
{You cannot take or use this feat if you have the ability to use powers (if you have a power point reserve or psi-like abilities).}

\Feat{Mind Over Body}
{Your ability damage heals more rapidly.}
{Con 13.}
{You heal ability damage and ability burn damage more quickly than normal. You heal a number of ability points per day equal to 1 + your Constitution bonus.}
{You heal ability damage and ability burn damage at a rate of 1 point per day.}{}

\Feat{Multiattack}
{}
{Three or more natural attacks.}
{The creature's secondary attacks with natural weapons take only a $-2$ penalty.}
{Without this feat, the creature's secondary attacks with natural weapons take a $-5$ penalty.}
{}

\Feat{Multiweapon Fighting}
{}
{Dex 13, three or more hands.}
{Penalties for fighting with multiple weapons are reduced by 2 with the primary hand and reduced by 6 with off hands.}
{A creature without this feat takes a $-6$ penalty on attacks made with its primary hand and a $-10$ penalty on attacks made with its off hands. (It has one primary hand, and all the others are off hands.) See \feat{Two-Weapon Fighting}.}
{This feat replaces the \feat{Two-Weapon Fighting} feat for creatures with more than two arms.}

\GFeat{Natural Spell}
{Wis 13, wild shape ability.}
{You can complete the verbal and somatic components of spells while in a wild shape. You substitute various noises and gestures for the normal verbal and somatic components of a spell.

You can also use any material components or focuses you possess, even if such items are melded within your current form. This feat does not permit the use of magic items while you are in a form that could not ordinarily use them, and you do not gain the ability to speak while in a wild shape.}

\Feat{Open Minded}
{You are naturally able to reroute your memory, mind, and skill expertise.}{}
{You immediately gain an extra 5 skill points. You spend these skill points as normal. If you spend them on a cross-class skills they count as onehalf ranks. You cannot exceed the normal maximum ranks for your level in any skill.}{}
{You can gain this feat multiple times. Each time, you immediately gain another 5 skill points.}

\Feat{Protective}
{You know that your gear could save your life, and you will do anything to protect it.}{}
{Gear on your person gains a +4 bonus to saving throws. If an item takes damage while you're holding it in your hands, you may make a Reflex save DC 10 + the amount of damage the item takes (after subtracting hardness) to transfer the damage to yourself.}{}{}

\Feat{Psionic Hole}
{You are anathema to psionic creatures and characters.}
{Con 15.}
{When a foe strikes you in melee combat, the foe immediately loses its psionic focus, if any. Also, if you are the target of a power, the manifester of the power must spend an additional number of power points equal to your Wisdom bonus, or the power fails (all the power points spent on the power are still lost). This extra cost does not count toward the maximum power points a manifester can spend on a single power.}{}
{You cannot take or use this feat if you have the ability to use powers (if you have a power point reserve or psi-like abilities).}

\Feat{Psionic Mimicry}
{Due to your study of psionic powers, you can pass off your spells as such.}
{\skill{Bluff} 8 ranks, \skill{Knowledge} (psionics) 4 ranks, \skill{Psicraft} 4 ranks.}
{You can disguise your spells as psionic powers by making a successful \skill{Bluff} check (DC 10 + spell level). An onlooker suspecting the nature of your spellcasting can attempt to identify a spell being cast using the \skill{Spellcraft} skill, but your check DC increases by 2.}{}{}

\Feat{Psionic Schooling}
{In your homeland, all who show some skill in the Way may receive training as a psion.}{}
{Psion, psychic warrior, or wilder is now a favored class for you (pick one), in addition to any other favored class you already possess. It does not count when determining multiclass XP penalties.}
{A character can have one favored class.}
{This feat must be selected at 1st level.}

% \GFeat{Psychic Hunter}
% {Favored enemy, psychic warrior level 1st.}
% {
% Your ranger and psychic warrior levels stack for the purpose of determining your manifester level. For example, a 5th-level ranger/1st-level psychic warrior with a Wisdom score of 16 would have 9 additional power points from his ability score and could spend 6 power points in a single psionic power, as if he was a 6th-level psychic warrior.

% As a swift action, you can sacrifice any number of power points (up to a maximum equal to your manifester level) to add a bonus to your attack and damage rolls against a favored enemy. This bonus is equal to half that number.
% }

\Feat{Quicken Spell-Like Ability}
{}
{Spell-like ability at caster level 10th or higher.}
{Choose one of the creature's spell-like abilities, subject to the restrictions described below. The creature can use that ability as a quickened spell-like ability three times per day (or less, if the ability is normally usable only once or twice per day).

Using a quickened spell-like ability is a swift action that does not provoke an attack of opportunity. The creature can perform another action---including the use of another spell-like ability---in the same round that it uses a quickened spell-like ability. The creature may use only one quickened spell-like ability per round.

The creature can only select a spell-like ability duplicating a spell with a level less than or equal to half its caster level (round down) $-4$. For a summary, see the associated table.

In addition, a spell-like ability that duplicates a spell with a casting time greater than 1 full round cannot be quickened.}
{Normally the use of a spell-like ability requires a standard action and provokes an attack of opportunity unless noted otherwise.}
{This feat can be taken multiple times. Each time it is taken, the creature can apply it to a different one of its spell-like abilities.}

\Feat{Raised by Beasts}
{Abandoned when you were very young, you were raised by wild animals.}{}
{Choose a kind of animal (amphibian, avian, mammal, fish, or reptile). You receive the wild empathy ability with animals of that kind. You also receive a +2 insight bonus on all \skill{Handle Animal} checks with animals of that kind.}{}
{This feat must be selected at 1st level.}

\Feat{Rapid Metabolism}
{Your wounds heal rapidly.}
{Con 13.}
{You naturally heal a number of hit points per day equal to the standard healing rate + double your Constitution bonus. You heal even if you do not rest. This healing replaces your normal natural healing. If you are tended successfully by someone with the Heal skill, you instead regain double the normal amount of hit points + double your Constitution bonus.}{}{}

\Feat{Reckless Offense}
{You can shift your focus from defense to offense.}
{Base attack bonus +1.}
{When you use the attack action or full attack action in melee, you can take a penalty of $-4$ to your Armor Class and add a +2 bonus on your melee attack roll. The bonus on attack rolls and penalty to Armor Class last until the beginning of your next turn.}{}{}

\GFeat{Reign of Terror}
{Raam, \skill{Intimidate} 5 ranks.}
{You gain a +4 bonus on \feat{Secular Authority} checks.}

\GFeat{Rogue Performer}
{Bardic music, sneak attack +2d6.}
{
Your bard and rogue levels stack for the purpose of determining the number of times per day that you can use your bardic music. For example, a 3rd-level bard/5th-level rogue could use her bardic music 8 times per day, as if she were an 8th-level bard.

Your bard and rogue levels stack for the purpose of determining your sneak attack bonus damage. For example, a 3rd-level bard/5th-level rogue would deal an extra 4d6 points of damage with her sneak attack, as if she were an 8th-level rogue.
}

\Feat{Run}
{}{}
{When running, you move five times your normal speed (if wearing medium, light, or no armor and carrying no more than a medium load) or four times your speed (if wearing heavy armor or carrying a heavy load). If you make a jump after a running start (see the \skill{Jump} skill description), you gain a +4 bonus on your \skill{Jump} check. While running, you retain your Dexterity bonus to AC.}
{You move four times your speed while running (if wearing medium, light, or no armor and carrying no more than a medium load) or three times your speed (if wearing heavy armor or carrying a heavy load), and you lose your Dexterity bonus to AC.}{}

\Feat{Secular Authority}
{You can use your authority within your city-state to order slaves to do your bidding, requisition troops, enter the homes of freemen and nobles, and have them arrested.}
{Cha 13, \skill{Diplomacy} 6 ranks, \feat{Negotiator}, accepted into city-state's templarate.}
{This feat grants four new uses for the \skill{Diplomacy} skill. None of them functions during combat.

\textit{Requisition:} You can draw upon the resources of your city, gaining the use of any slave, overriding the wishes of its owner.

\textit{Intrude:} You can, at any time, search the home, person or possessions of a slave. You may search and impound any evidence of wrongdoing, if found. Your authority does not extend to confiscating items for personal use.

\textit{Accuse:} You may have a slave imprisoned indefinitely, awaiting the gathering of evidence against him. You may only imprison one suspect in such a manner.

\textit{Judge:} You may pass judgment on a slave. This includes setting fines, prison sentences, death sentences or anything else you wish, within the laws of your city-state.

As you gain more ranks in the \skill{Diplomacy} ranks, you gain the authority to take these actions against progressively higher social rankings, as described on the table below.

\Table{Secular Authority abilities}{c X}{
\tableheader Ranks & \tableheader Ability\\
2 & Requisition slave\\
3 & Intrude on slave\\
4 & Accuse slave\\
5 & Requisition troops\\
6 & Intrude on freeman\\
7 & Judge slave\\
8 & Accuse freeman\\
9 & Requisition gear\\
10 & Intrude on noble\\
11 & Judge freeman\\
12 & Accuse noble\\
13 & Requisition spellcaster/manifester\\
14 & Intrude on templar\\
15 & Judge noble\\
16 & Accuse templar\\
17 & Requisition property\\
18+ & Judge templar
}

Failure to comply with these demands is usually sanctioned with fines, imprisonment, outlaw status, and possibly execution. Any of this ability can be contested by another person with the Secular Authority feat, and move to have the action reversed with an opposed \skill{Diplomacy} check. If the challenger wins the opposed roll, the defending templar's action is reversed (for example an imprisoned freeman is set free). If the defender wins the opposed roll nothing happens. Secular Authority can be contested in a particular case only once. A defending character who loses the opposed roll may not contest the result. Nor can he use Secular Authority to repeat the action that was contested against the same target.

You may use Secular Authority once per day for every four levels you have attained (but see Special), but only
within your city-state.}
{}
{A templar automatically gains Secular Authority as a bonus feat. He need not select it. A templar may use Secular Authority a number of times per day equal to his templar level, plus one more time per day for every four levels he has in classes other than templar.}

\Feat{Shield Proficiency}
{}{}
{You can use a shield and take only the standard penalties.}
{When you are using a shield with which you are not proficient, you take the shield's armor check penalty on attack rolls and on all skill checks that involve moving, including \skill{Ride} checks.}
{Barbarians, clerics, druids, fighters, gladiators, psychic warriors, rangers, and templars automatically have Shield Proficiency as a bonus feat. They need not select it.}

\Feat{Sidestep Charge}
{You are skilled at dodging past charging opponents and taking advantage when they miss.}
{Dex 13, \feat{Dodge}.}
{You get a +4 dodge bonus to Armor Class against charge attacks. If a charging opponent fails to make a successful attack against you, you gain an immediate attack of opportunity. This feat does not grant you more attacks of opportunity than you are normally allowed in a round. If you are flat-footed or otherwise denied your Dexterity bonus to Armor Class, you do not gain the benefit of this feat.}{}{}

\Feat{Simple Weapon Proficiency}
{}{}
{You make attack rolls with simple weapons normally.}
{When using a weapon with which you are not proficient, you take a $-4$ penalty on attack rolls.}
{All characters except for druids, psions, and wizards are automatically proficient with all simple weapons. They need not select this feat.}

\Feat{Sniper}
{You are better at hiding when firing missile weapons and trying to stay hidden.}
{Dex 13, \skill{Hide} 1 rank.}
{You receive a +5 competence bonus to \skill{Hide} checks when firing missiles while trying to stay hidden.}{}{}

\Feat{Skill Focus}
{Choose a skill.}{}
{You get a +3 bonus on all checks involving that skill. This skill is treated as a class skill in all respects for all classes you have levels in, both current and future.}{}
{You can gain this feat multiple times. Its effects do not stack. Each time you take the feat, it applies to a new skill.}

\Feat{Spell Focus}
{Choose a school of magic.}{}
{Add +1 to the Difficulty Class for all saving throws against spells from the school of magic you select.}{}
{You can gain this feat multiple times. Its effects do not stack. Each time you take the feat, it applies to a new school of magic.}

\Feat[Special]{Spell Mastery}{}
{Wizard level 1st.}
{Each time you take this feat, choose a number of spells equal to your Intelligence modifier that you already know. From that point on, you can prepare these spells without referring to a spellbook.}
{Without this feat, you must use a spellbook to prepare all your spells, except read magic.}{}

\GFeat{Spell Penetration}{}
{You get a +2 bonus on caster level checks (1d20 + caster level) made to overcome a creature's spell resistance.}

\Feat{Stand Still}
{You can prevent foes from fleeing or closing.}
{Str 13.}
{When a foe's movement out of a square you threaten grants you an attack of opportunity, you can give up that attack and instead attempt to stop your foe in his tracks. Make your attack of opportunity normally. If you hit your foe, he must succeed on a Reflex save against a DC of 10 + your damage roll (the opponent does not actually take damage), or immediately halt as if he had used up his move actions for the round.

Since you use the Stand Still feat in place of your attack of opportunity, you can do so only a number of times per round equal to the number of times per round you could make an attack of opportunity (normally just one).}
{Attacks of opportunity cannot halt your foes in their tracks.}{}

\Feat{Toughness}
{}{}
{You gain +3 hit points.}{}
{A character may gain this feat multiple times. Its effects stack.}

\Feat{Tower Shield Proficiency}
{}
{\feat{Shield Proficiency}.}
{You can use a tower shield and suffer only the standard penalties.}
{A character who is using a shield with which he or she is not proficient takes the shield's armor check penalty on attack rolls and on all skill checks that involve moving, including Ride.}
{Fighters automatically have Tower Shield Proficiency as a bonus feat. They need not select it.}

\Feat{Track}
{}{}
{To find tracks or to follow them for 1.5 kilometer requires a successful \skill{Survival} check. You must make another \skill{Survival} check every time the tracks become difficult to follow.

You move at half your normal speed (or at your normal speed with a $-5$ penalty on the check, or at up to twice your normal speed with a $-20$ penalty on the check). The DC depends on the surface and the prevailing conditions, as given on \tabref{Track DC}.


\Table{Track DC}{X Z{1.4cm} X Z{1.4cm}}{
\tableheader Surface & \tableheader \skill{Survival} DC & \tableheader Surface & \tableheader \skill{Survival} DC\\
Very soft ground & 5 & Firm ground & 15\\
Soft ground & 10 & Hard ground & 20
}

\textit{Very Soft Ground:} Any surface (fresh snow, thick dust, wet mud) that holds deep, clear impressions of footprints.

\textit{Soft Ground:} Any surface soft enough to yield to pressure, but firmer than wet mud or fresh snow, in which a creature leaves frequent but shallow footprints.

\textit{Firm Ground:} Most normal outdoor surfaces (such as lawns, fields, woods, and the like) or exceptionally soft or dirty indoor surfaces (thick rugs and very dirty or dusty floors). The creature might leave some traces (broken branches or tufts of hair), but it leaves only occasional or partial footprints.

\textit{Hard Ground:} Any surface that doesn't hold footprints at all, such as bare rock or an indoor floor. Most streambeds fall into this category, since any footprints left behind are obscured or washed away. The creature leaves only traces (scuff marks or displaced pebbles).

Several modifiers may apply to the \skill{Survival} check, as given on \tabref{Track DC Modifiers}.

\Table{Track DC Modifiers}{X Z{1.4cm}}{
\tableheader Condition & \tableheader \skill{Survival} DC Modifier\\
Every three creatures in the group being tracked & $-1$\\
Size of creature or creatures being tracked & \\
~ Fine & +8\\
~ Diminutive & +4\\
~ Tiny & +2\\
~ Small & +1\\
~ Medium & 0\\
~ Large & $-1$\\
~ Huge & $-2$\\
~ Gargantuan & $-4$\\
~ Colossal & $-8$\\
Every 24 hours since the trail was made & +1\\
Every hour of rain since the trail was made & +1\\
Fresh snow cover since the trail was made & +10\\
Poor visibility (Apply only the largest modifier from this category.) & \\
~ Overcast or moonless night & +6\\
~ Moonlight & +3\\
~ Fog or precipitation & +3\\
Tracked party hides trail (and moves at half speed) & +5
}

For a group of mixed sizes, apply only the modifier for the largest size category.

If you fail a \skill{Survival} check, you can retry after 1 hour (outdoors) or 10 minutes (indoors) of searching.}
{Without this feat, you can use the \skill{Survival} skill to find tracks, but you can follow them only if the DC for the task is 10 or lower. Alternatively, you can use the \skill{Search} skill to find a footprint or similar sign of a creature's passage using the DCs given above, but you can't use Search to follow tracks, even if someone else has already found them.}
{A ranger automatically has Track as a bonus feat. He need not select it.

This feat does not allow you to find or follow the tracks made by a subject of a pass without trace spell.}

\Feat{Urban Tracking}
{You can track down the location of missing persons or wanted individuals within communities.}{}
{
To find the trail of an individual or to follow it for 1 hour requires a \skill{Gather Information} check. You must make another \skill{Gather Information} check every hour of the search, as well as each time the trail becomes difficult to follow, such as when it moves to a different area of town.

\Table{Urban Tracking DC}{Xcl}{
\tableheader Community Size & \tableheader DC & \tableheader Checks Required\\
Thorp, hamlet, or village & 5 & 1d3\\
Small or large town & 10 & 1d4+1\\
Small or large city & 15 & 2d4\\
Metropolis & 20 & 2d4+2\\
}

The DC of the check, and the number of checks required to track down your quarry, depends on the community size and the conditions:

\Table{Urban Tracking Modifiers}{X Z{1.4cm}}{
\tableheader Conditions & \tableheader DC Modifier\\
Every three creatures in the group being sought & $-1$\\
Every 24 hours party has been missing/sought & +1\\
Tracked party ``lies low'' & +5\\
Tracked party matches community's primary racial demographic & +2\\
Tracked party does not match community's primary, or secondary racial demographic & -2\\
}
If you fail a \skill{Gather Information} check, you can retry after 1 hour of questioning. The DM should roll the number of checks required secretly, so that the player doesn't know exactly how long the task will require.

}
{A character without this feat can use \skill{Gather Information} to find out information about a particular individual, but each check takes 1d4+1 hours and doesn't allow effective trailing.}
{A character with 5 ranks in \skill{Knowledge} (local) gains a +2 bonus on the \skill{Gather Information} check to use this feat.

You can cut the time between \skill{Gather Information} checks in half (to 30 minutes per check rather than 1 hour), but you take a $-5$ penalty on the check.}

\Feat{Wastelander}
{You are an experienced survivor of the wastes.}{}
{You get a +1 bonus to Fortitude saves and a +2 bonus to \skill{Survival} checks.}{}{}

\Feat{Wild Talent}
{Your mind wakes to a previously unrealized talent for psionics.}
{}
{Your latent power of psionics flares to life, conferring upon you the designation of a psionic character. You learn one psionic power that does not have prerequistes. You must have the relevant key ability score 10 + the power's level (see the \class{Psion} class).

As a psionic character, you gain a reserve of power points equal to 5 $\times$ the power's level. If the chosen power can be maintained, you gain additional power points equal to 4 $\times$ the power's level.

You can take psionic feats, metapsionic feats, and psionic item creation feats.}
{}
{You cannot take or use this feat if you have the ability to use powers or cast spells.

This feat must be taken at 1st level.}
% {Your mind wakes to a previously unrealized talent for psionics.}{}
% {Your latent power of psionics flares to life, conferring upon you the designation of a psionic character. As a psionic character, you gain a reserve of 2 power points and can take psionic feats, metapsionic feats, and psionic item creation feats. You do not, however, gain the ability to manifest powers simply by virtue of having this feat.}{}{}

\GFeat{Wingover}
{Fly speed.}
{A flying creature with this feat can change direction quickly once each round as a free action. This feat allows it to turn up to 180 degrees regardless of its maneuverability, in addition to any other turns it is normally allowed. A creature cannot gain altitude during a round when it executes a wingover, but it can dive.

The change of direction consumes 3 meters of flying movement.}