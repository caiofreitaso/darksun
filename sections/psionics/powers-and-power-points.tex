\section{Powers and Power Points}
Psionic characters manifest powers, which involve the direct manipulation of personal mental energy. These manipulations require natural talent and personal meditation. A psionic character's level limits the number of power points available to manifest powers. A psionic character's relevant high score might allow him to gain extra power points. He can manifest the same power more than once, but each manifestation subtracts power points from his daily limit. Manifesting a power is an arduous mental task. To do so, a psionic character must have a key ability score of at least 10 + the power's level.

\textbf{Daily Power Point Acquisition}: To regain used daily power points, a psionic character must have a clear mind. To clear his mind, he must first sleep for 8 hours. The character does not have to slumber for every minute of the time, but he must refrain from movement, combat, manifesting powers, skill use, conversation, or any other demanding physical or mental task during the rest period. If his rest is interrupted, each interruption adds 1 hour to the total amount of time he has to rest to clear his mind, and he must have at least 1 hour of rest immediately prior to regaining lost power points. If the character does not need to sleep for some reason, he still must have 8 hours of restful calm before regaining power points.

\textbf{Recent Manifesting Limit/Rest Interruptions}: If a psionic character has manifested powers recently, the drain on his resources reduces his capacity to regain power points. When he regains power points for the coming day, all power points he has used within the last 8 hours count against his daily limit.

\textbf{Peaceful Environment}: To regain power points, a psionic character must have enough peace, quiet, and comfort to allow for proper concentration. The psionic character's surroundings need not be luxurious, but they must be free from overt distractions, such as combat raging nearby or other loud noises. Exposure to inclement weather prevents the necessary concentration, as does any injury or failed saving throw the character might incur while concentrating on regaining power points.

\textbf{Regaining Power Points}: Once the character has rested in a suitable environment, it takes only an act of concentration spanning 1 full round to regain all power points of the psionic character's daily limit.

\textbf{Death and Power Points}: If a character dies, all daily power points stored in his mind are wiped away. A potent effect (such as reality revision) can recover the lost power points when it recovers the character.

\subsection{Adding Powers}
Psionic characters can learn new powers when they attain a new level. A psion can learn any power from the psion/wilder list and powers from his chosen discipline's list. A wilder can learn any power from the psion/wilder list. A psychic warrior can learn any power from the psychic warrior list.

\textbf{Powers Gained at a New Level}: Psions and other psionic characters perform a certain amount of personal meditation between adventures in an attempt to unlock latent mental abilities. Each time a psionic character attains a new level, he or she learns additional powers according to his class description. Psions, psychic warriors, and wilders learn new powers of their choice in this fashion. These powers represent abilities unlocked from latency. The powers must be of levels the characters can manifest.

\textbf{Independent Research}: A psion also can research a power independently, duplicating an existing power or creating an entirely new one. If characters are allowed to develop new powers, use these guidelines to handle the situation.

Any kind of manifester can create a new power. The research involved requires access to a retreat conducive to uninterrupted meditation. Research involves an expenditure of 200 XP per week and takes one week per level of the power. At the end of that time, the character makes a \skill{Psicraft} check (DC 10 + spell level). If that check succeeds, the character learns the new power if her research produced a viable power. If the check fails, the character must go through the research process again if she wants to keep trying.

\subsubsection{Manifest an Unknown Power from Another's Powers Known}
A psionic character can attempt to manifest a power from a source other than his own knowledge (usually a power stone or another willing psionic character). To do so, the character must first make contact (a process similar to addressing a power stone, requiring a \skill{Psicraft} check against a DC of 15 + the highest level power in the power stone or repertoire). A psionic character can make contact with only a willing psionic character or creature (unconscious creatures are considered willing, but not psionic characters under the effects of other immobilizing conditions). Characters who can't use power stones for any reason are also banned from attempting to manifest powers from the knowledge of other psionic characters. Mental contact requires 1 full round of physical contact, which can provoke attacks of opportunity. Once contact is achieved, the character becomes aware of all the powers stored in the power stone or all the powers the other character knows up to the highest level of power the contactor knows himself.

Next, the psionic character must choose one of the powers and make a second \skill{Psicraft} check (DC 15 + the power's level) to see if he understands it. If the power is not on his class list, he automatically fails this check.

Upon successfully making contact with another willing psionic character or creature and learning what he can of one power in particular, the character can immediately attempt to manifest that power even if he doesn't know it (and assuming he has power points left for the day). He can attempt to manifest the power normally on his next turn, and he succeeds if he makes one additional \skill{Psicraft} check (DC 15 + the power's level). He retains the ability to manifest the selected power for only 1 round. If he doesn't manifest the power, fails the \skill{Psicraft} check, or manifests a different power, he loses his chance to manifest that power for the day.

\subsection{Using Stored Power Points}
A variety of psionic items exist to store power points for later use, in particular a storage device called a cognizance crystal. Regardless of what sort of item stores the power points, all psionic characters must follow strict rules when tapping stored power points.

\textbf{A Single Source}: When using power points from a storage item to manifest a power, a psionic character may not pay the power's cost with power points from more than one source. He must either use an item, his own power point reserve, or some other discrete power point source to pay the manifestation cost.

\textbf{Recharging}: Most power point storage devices allow psionic characters to ``recharge'' the item with their own power points. Doing this depletes the character's power point reserve on a 1-for-1 basis as if he had manifested a power; however, those power points remain indefinitely stored. The opposite is not true---psionic characters may not use power points stored in a storage item to replenish their own power point reserves.

