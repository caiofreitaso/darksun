\section{Power Descriptions}
The description of each power is presented in a standard format. Each category of information is explained and defined below.

\subsection{Name}
The first line of every power description gives the name by which the power is generally known. A power might be known by other names in some locales, and specific manifesters might have names of their own for their powers.

\subsection{Discipline (Subdiscipline)}
Beneath the power name is a line giving the discipline (and the subdiscipline in parentheses, if appropriate) that the power belongs to.

Every power is associated with one of six disciplines. A discipline is a group of related powers that work in similar ways. Each of the disciplines is discussed below.

\subsubsection{Clairsentience}
Clairsentience powers enable you to learn secrets long forgotten, to glimpse the immediate future and predict the far future, to find hidden objects, and to know what is normally unknowable.

For the purpose of psionics-magic transparency, clairsentience powers are equivalent to powers of the divination school (thus, creatures immune to divination spells are also immune to clairsentience powers).

Many clairsentience powers have cone-shaped areas. These move with you and extend in the direction you look. The cone defines the area that you can sweep each round. If you study the same area for multiple rounds, you can often gain additional information, as noted in the descriptive text for the power.

\textbf{Scrying}: A power of the scrying subdiscipline creates an invisible sensor that sends you information. Unless noted otherwise, the sensor has the same powers of sensory acuity that you possess. This includes any powers or effects that target you, but not powers or effects that emanate from you. However, the sensor is treated as a separate, independent sensory organ of yours, and thus functions normally even if you have been blinded, deafened, or otherwise suffered sensory impairment. Any creature with an Intelligence score of 12 or higher can notice the sensor by making a DC 20 Intelligence check. The sensor can be dispelled as if it were an active power. Lead sheeting or psionic protection blocks scrying powers, and you sense that the power is so blocked.

\subsubsection{Metacreativity}
Metacreativity powers create objects, creatures, or some form of matter. Creatures you create usually, but not always, obey your commands.

A metacreativity power draws raw ectoplasm from the Astral Plane to create an object or creature in the place the psionic character designates (subject to the limits noted above). Objects created in this fashion are as solid and durable as normal objects, despite their originally diaphanous substance. Psionic creatures created with metacreativity powers are considered constructs, not outsiders.

A creature or object brought into being cannot appear inside another creature or object, nor can it appear floating in an empty space. It must arrive in an open location on a surface capable of supporting it. The creature or object must appear within the power's range, but it does not have to remain within the range.

For the purpose of psionics-magic transparency, metacreativity powers are equivalent to powers of the conjuration school (thus, creatures immune to conjuration spells are also immune to metacreativity powers).

\textbf{Creation}: A power of the creation subdiscipline creates an object or creature in the place the manifester designates (subject to the limits noted above). If the power has a duration other than instantaneous, psionic energy holds the creation together, and when the power ends, the created creature or object vanishes without a trace, except for a thin film of glistening ectoplasm that quickly evaporates. If the power has an instantaneous duration, the created object or creature is merely assembled through psionics. It lasts indefinitely and does not depend on psionics for its existence.

\subsubsection{Psychokinesis}
Psychokinesis powers manipulate energy or tap the power of the mind to produce a desired end. Many of these powers produce spectacular effects above and beyond the power's standard display, such as moving, melting, transforming, or blasting a target. Psychokinesis powers can deal large amounts of damage.

For the purpose of psionics-magic transparency, psychokinesis powers are equivalent to powers of the evocation school (thus, creatures immune to evocation spells are also immune to psychokinesis powers).

\subsubsection{Psychometabolism}
Psychometabolism powers change the physical properties of some creature, thing, or condition.

For the purpose of psionics-magic transparency, psychometabolism powers are equivalent to powers of the transmutation school (thus, creatures immune to transmutation spells are also immune to psychometabolism powers).

\textbf{Healing}
Psychometabolism powers of the healing subdiscipline can remove damage from creatures. However, psionic healing usually falls short of divine magical healing, in direct comparisons.

\subsubsection{Psychoportation}
Psychoportation powers move the manifester, an object, or another creature through space and time.

For the purpose of psionics-magic transparency, psychoportation powers do not have an equivalent school.

\textbf{Teleportation}: A power of the teleportation subdiscipline transports one or more creatures or objects a great distance. The most potent of these powers can cross planar boundaries. Usually the transportation is one-way (unless otherwise noted) and not dispellable. Teleportation is instantaneous travel through the Astral Plane. Anything that blocks astral travel also blocks teleportation.

\subsubsection{Telepathy}
Telepathy powers can spy on and affect the minds of others, influencing or controlling their behavior.

Most telepathy powers are mind-affecting.

For the purpose of psionics-magic transparency, telepathy powers are equivalent to powers of the enchantment school (thus, creatures resistant to enchantment spells are equally resistant to telepathy powers).

\textbf{Charm}: A power of the charm subdiscipline changes the way the subject views you, typically making it see you as a good friend.

\textbf{Compulsion}: A power of the compulsion subdiscipline forces the subject to act in some manner or changes the way her mind works. Some compulsion powers determine the subject's actions or the effects on the subject, some allow you to determine the subject's actions when you manifest them, and others give you ongoing control over the subject.

\subsection{[Descriptor]}
Appearing on the same line as the discipline and subdiscipline (when applicable) is a descriptor that further categorizes the power in some way. Some powers have more than one descriptor.

The descriptors that apply to powers are acid, cold, death, electricity, evil, fire, force, good, language-dependent, light, mind-affecting, and sonic.

Most of these descriptors have no game effect by themselves, but they govern how the power interacts with other powers, with spells, with special abilities, with unusual creatures, with alignment, and so on.

A language-dependent power uses intelligible language as a medium.

A mind-affecting power works only against creatures with an Intelligence score of 1 or higher.

\subsection{Level}
The next line of the power description gives a power's level, a number between 1 and 9 that defines the power's relative strength. This number is preceded by the name of the class whose members can manifest the power. If a power is part of a discipline's list instead of the psion's general power list, this will be indicated by the name of the discipline's specialist. The specialists a power can be associated with include Egoist (psychometabolism), Kineticist (psychokinesis), Nomad (psychoportation), Seer (clairsentience), Shaper (metacreativity), and Telepath (telepathy).

\subsection{Display}
When a power is manifested, a display may accompany the primary effect. This secondary effect may be auditory, material, mental, olfactory, or visual. No power's display is significant enough to create consequences for the psionic creatures, allies, or opponents during combat. The secondary effect for a power occurs only if the power's description indicates it. If multiple powers with similar displays are in effect simultaneously, the displays do not necessary become more intense. Instead, the overall display remains much the same, though with minute spikes in intensity. A Psicraft check (DC 10 + 1 per additional power in use) reveals the exact number of simultaneous powers in play.

\textbf{Dispense with Displays}: Despite the fact that almost every power has a display, a psionic character can always choose to manifest the power without the flashy accompaniment. To manifest a power without any display (no matter how many displays it might have), a manifester must make a Concentration check (DC 15 + the level of the power). This check is part of the action of manifesting the power. If the check is unsuccessful, the power manifests normally with its display.

Even if a manifester manifests a power without a display, he is still subject to attacks of opportunity in appropriate circumstances. (Of course, another Concentration check can be made as normal to either manifest defensively or maintain the power if attacked.)

\textit{Auditory}: A bass-pitched hum issues from the manifester's vicinity or in the vicinity of the power's subject (manifester's choice), eerily akin to many deep-pitched voices. The sound grows in a second from hardly noticeable to as loud as a shout strident enough to be heard within 100 feet. At the manifester's option, the instantaneous sound can be so soft that it can be heard only within 15 feet with a successful DC 10 Listen check. Some powers describe unique auditory displays.

\textit{Material}: The subject or the area is briefly slicked with a translucent, shimmering substance. The glistening substance evaporates after 1 round regardless of the power's duration. Sophisticated psions recognize the material as ectoplasmic seepage from the Astral Plane; this substance is completely inert.

\textit{Mental}: A subtle chime rings once in the minds of creatures within 15 feet of either the manifester or the subject (at the manifester's option). At the manifester's option, the chime can ring continuously for the power's duration. Some powers describe unique mental displays.

\textit{Olfactory}: An odd but familiar odor brings to mind a brief mental flash of a long-buried memory. The scent is difficult to pin down, and no two individuals ever describe it the same way. The odor originates from the manifester and spreads to a distance of 20 feet, then fades in less than a second (or lasts for the duration, at the manifester's option).

\textit{Visual}: The manifester's eyes burn like points of silver fire while the power remains in effect. A rainbow-flash of light sweeps away from the manifester to a distance of 5 feet and then dissipates, unless a unique visual display is described. This is the case when the Display entry includes ``see text,'' which means that a visual effect is described somewhere in the text of the power.

\subsection{Manifesting Time}
Most powers have a manifesing time of 1 standard action. Others take 1 round or more, while a few require only a free action.

A power that takes 1 round to manifest requires a full-round action. It comes into effect just before the beginning of your turn in the round after you began manifesting the power. You then act normally after the power is completed.

A power that takes 1 minute to manifest comes into effect just before your turn 1 minute later (and for each of those 10 rounds, you are manifesting a power as a full-round action, as noted above for 1-round manifesting times). These actions must be consecutive and uninterrupted, or the power points are lost and the power fails.

When you use a power that takes 1 round or longer to manifest, you must continue the concentration from the current round to just before your turn in the next round (at least). If you lose concentration before the manifesting time is complete, the power points are lost and the power fails.

You make all pertinent decisions about a power (range, target, area, effect, version, and so forth) when the power comes into effect.

\subsection{Range}
A power's range indicates how far from you it can reach, as defined in the Range entry of the power description. A power's range is the maximum distance from you that the power's effect can occur, as well as the maximum distance at which you can designate the power's point of origin. If any portion of the area would extend beyond the range, that area is wasted. Standard ranges include the following.

\textbf{Personal}: The power affects only you.

\textbf{Touch}: You must touch a creature or object to affect it. A touch power that deals damage can score a critical hit just as a weapon can. A touch power threatens a critical hit on a natural roll of 20 and deals double damage on a successful critical hit. Some touch powers allow you to touch multiple targets. You can touch as many willing targets as you can reach, but all targets of the spell must be touched in the same round that you manifest the power.

\textbf{Close}: The power reaches as far as 7.5 meters away from you. The maximum range increases 1.5 meter for every two manifester levels you have.

\textbf{Medium}: The power reaches as far as 30 meters + 3 meters per manifester level.

\textbf{Long}: The power reaches as far as 120 meters + 12 meters per manifester level.

\textbf{Range Expressed in Meters}: Some powers have no standard range category, just a range expressed in meters.

\subsection{Aiming A Power}
You must make some choice about whom the power is to affect or where the power's effect is to originate, depending on the type of power. The next entry in a power description defines the power's target (or targets), its effect, or its area, as appropriate.

\textbf{Target or Targets}: Some powers have a target or targets. You manifest these powers on creatures or objects, as defined by the power itself. You must be able to see or touch the target, and you must specifically choose that target. However, you do not have to select your target until you finish manifesting the power.

If you manifest a targeted power on the wrong type of target the power has no effect. If the target of a power is yourself (the power description has a line that reads ``Target: You''), you do not receive a saving throw and power resistance does not apply. The Saving Throw and Power Resistance lines are omitted from such powers.

Some powers can be manifested only on willing targets. Declaring yourself as a willing target is something that can be done at any time (even if you're flat-footed or it isn't your turn). Unconscious creatures are automatically considered willing, but a character who is conscious but immobile or helpless (such as one who is bound, cowering, grappling, paralyzed, pinned, or stunned) is not automatically willing. The Saving Throw and Power Resistance lines are usually omitted from such powers, since only willing subjects can be targeted.

\textbf{Effect}: Some powers, such as most metacreativity powers, create things rather than affect things that are already present. Unless otherwise noted in the power description, you must designate the location where these things are to appear, either by seeing it or defining it. Range determines how far away an effect can appear, but if the effect is mobile, it can move regardless of the power's range once created.

\textit{Ray}: Some effects are rays. You aim a ray as if using a ranged weapon, though typically you make a ranged touch attack rather than a normal ranged attack. As with a ranged weapon, you can fire into the dark or at an invisible creature and hope you hit something. You don't have to see the creature you're trying to hit, as you do with a targeted power. Intervening creatures and obstacles, however, can block your line of sight or provide cover for the creature you're aiming at.

If a ray power has a duration, it's the duration of the effect that the ray causes, not the length of time the ray itself persists.

If a ray power deals damage, you can score a critical hit just as if it were a weapon. A ray power threatens a critical hit on a natural roll of 20 and deals double damage on a successful critical hit.

\textit{Spread}: Some effects spread out from a point of origin (which may be a grid intersection, or may be the manifester) to a distance described in the power. The effect can extend around corners and into areas that you can't see. Figure distance by actual distance traveled, taking into account turns the effect may take. When determining distance for spread effects, count around walls, not through them. As with movement, do not trace diagonals across corners. You must designate the point of origin for such an effect (unless the effect is centered on you), but you need not have line of effect (see below) to all portions of the effect.

\textit{(S) Shapeable}: If an Effect line ends with ``(S)'' you can shape the power. A shaped effect can have no dimension smaller than 10 feet.

\textbf{Area}: Some powers affect an area. Sometimes a power description specifies a specially defined area, but usually an area falls into one of the categories defined below.

Regardless of the shape of the area, you select the point where the power originates, but otherwise you usually don't control which creatures or objects the power affects. The point of origin of a power that affects an area is always a grid intersection. When determining whether a given creature is within the area of a power, count out the distance from the point of origin in squares just as you do when moving a character or when determining the range for a ranged attack. The only difference is that instead of counting from the center of one square to the center of the next, you count from intersection to intersection.

You can count diagonally across a square, but every second diagonal counts as 2 squares of distance. If the far edge of a square is within the power's area, anything within that square is within the power's area. If the power's area touches only the near edge of a square, however, anything within that square is unaffected by the power.

\textit{Burst, Emanation, or Spread}: Most powers that affect an area function as a burst, an emanation, or a spread. In each case, you select the power's point of origin and measure its effect from that point. A burst power affects whatever it catches in its area, even including creatures that you can't see. It can't affect creatures with total cover from its point of origin (in other words, its effects don't extend around corners). The default shape for a burst effect is a sphere, but some burst powers are specifically described as cone-shaped.

A burst's area defines how far from the point of origin the power's effect extends.

An emanation power functions like a burst power, except that the effect continues to radiate from the point of origin for the duration of the power.

A spread power spreads out like a burst but can turn corners. You select the point of origin, and the power spreads out a given distance in all directions. Figure the area the power effect fills by taking into account any turns the effect takes.

\textit{Cone, Line, or Sphere}: Most powers that affect an area have a particular shape, such as a cone, line, or sphere. A cone-shaped power shoots away from you in a quarter-circle in the direction you designate. It starts from any corner of your square and widens out as it goes. Most cones are either bursts or emanations (see above), and thus won't go around corners.

A line-shaped power shoots away from you in a line in the direction you designate. It starts from any corner of your square and extends to the limit of its range or until it strikes a barrier that blocks line of effect. A line-shaped power affects all creatures in squares that the line passes through or touches.

A sphere-shaped power expands from its point of origin to fill a spherical area. Spheres may be bursts, emanations, or spreads.

\textit{Other}: A power can have a unique area, as defined in its description.

\textbf{Line of Effect}: A line of effect is a straight, unblocked path that indicates what a power can affect. A solid barrier cancels a line of effect, but it is not blocked by fog, darkness, and other factors that limit normal sight. You must have a clear line of effect to any target that you manifest a power on or to any space in which you wish to create an effect. You must have a clear line of effect to the point of origin of any power you manifest.

A burst, cone, or emanation power affects only an area, creatures, or objects to which it has line of effect from its origin (a spherical burst's center point, a cone-shaped burst's starting point, or an emanation's point of origin). An otherwise solid barrier with a hole of at least 1 square foot through it does not block a power's line of effect. Such an opening means that the 5-foot length of wall containing the hole is no longer considered a barrier for the purpose of determining a power's line of effect.

\subsection{Duration}
A power's Duration line tells you how long the psionic energy of the power lasts.

Timed Durations: Many durations are measured in rounds, minutes, hours, or some other increment. When the time is up, the psionic energy sustaining the effect fades, and the power ends. If a power's duration is variable it is rolled secretly.

\textbf{Instantaneous}: The psionic energy comes and goes the instant the power is manifested, though the consequences might be long-lasting.

\textbf{Permanent}: The energy remains as long as the effect does. This means the power is vulnerable to dispel psionics.

\textbf{Concentration}: The power lasts as long as you concentrate on it. Concentrating to maintain a power is a standard action that does not provoke attacks of opportunity. Anything that could break your concentration when manifesting a power can also break your concentration while you're maintaining one, causing the power to end. You can't manifest a power while concentrating on another one. Some powers may last for a short time after you cease concentrating. In such a case, the power keeps going for the given length of time after you stop concentrating, but no longer. Otherwise, you must concentrate to maintain the power, but you can't maintain it for more than a stated duration in any event. If a target moves out of range, the power reacts as if your concentration had been broken.

\textbf{Subjects, Effects, and Areas}: If the power affects creatures directly the result travels with the subjects for the power's duration. If the power creates an effect, the effect lasts for the duration. The effect might move or remain still. Such an effect can be destroyed prior to when its duration ends. If the power affects an area then the power stays with that area for its duration. Creatures become subject to the power when they enter the area and are no longer subject to it when they leave.

\textbf{Touch Powers and Holding the Charge}: In most cases, if you don't discharge a touch power on the round you manifest it, you can hold the charge (postpone the discharge of the power) indefinitely. You can make touch attacks round after round. If you touch anything with your hand while holding a charge, the power discharges. If you manifest another power, the touch power dissipates.

Some touch powers allow you to touch multiple targets as part of the power. You can't hold the charge of such a power; you must touch all the targets of the power in the same round that you finish manifesting the power. You can touch one friend (or yourself) as a standard action or as many as six friends as a full round action.

\textbf{Discharge}: Occasionally a power lasts for a set duration or until triggered or discharged.

\textbf{(D) Dismissible}: If the Duration line ends with ``(D),'' you can dismiss the power at will. You must be within range of the power's effect and must mentally will the dismissal, which causes the same display as when you first manifested the power. Dismissing a power is a standard action that does not provoke attacks of opportunity. A power that depends on concentration is dismissible by its very nature, and dismissing it does not take an action or cause a display, since all you have to do to end the power is to stop concentrating on your turn.

\subsection{Saving Throw}
Usually a harmful power allows a target to make a saving throw to avoid some or all of the effect. The Saving Throw line in a power description defines which type of saving throw the power allows and describes how saving throws against the power work.

\textbf{Negates}: The power has no effect on a subject that makes a successful saving throw.

\textbf{Partial}: The power causes an effect on its subject, such as death. A successful saving throw means that some lesser effect occurs (such as being dealt damage rather than being killed).

\textbf{Half}: The power deals damage, and a successful saving throw halves the damage taken (round down).

\textbf{None}: No saving throw is allowed.

\textbf{(object)}: The power can be manifested on objects, which receive saving throws only if they are psionic or if they are attended (held, worn, grasped, or the like) by a creature resisting the power, in which case the object uses the creature's saving throw bonus unless its own bonus is greater. (This notation does not mean that a power can be manifested only on objects. Some powers of this sort can be manifested on creatures or objects.) A psionic item's saving throw bonuses are each equal to 2 + one-half the item's manifester level.

\textbf{(harmless)}: The power is usually beneficial, not harmful, but a targeted creature can attempt a saving throw if it desires.

\textbf{Saving Throw Difficulty Class}: A saving throw against your power has a DC 10 + the level of the power + your key ability modifier (Intelligence for a psion, Wisdom for a psychic warrior, or Charisma for a wilder). A power's level can vary depending on your class. Always use the power level applicable to your class.

\textbf{Succeeding on a Saving Throw}: A creature that successfully saves against a power that has no obvious physical effects feels a hostile force or a tingle, but cannot deduce the exact nature of the attack. Likewise, if a creature's saving throw succeeds against a targeted power you sense that the power has failed. You do not sense when creatures succeed on saves against effect and area powers.

\textbf{Failing a Saving Throw against Mind-Affecting Powers}: If you fail your save, you are unaware that you have been affected by a power.

\textbf{Automatic Failures and Successes}: A natural 1 (the d20 comes up 1) on a saving throw is always a failure, and the power may deal damage to exposed items (see Items Surviving after a Saving Throw, below). A natural 20 (the d20 comes up 20) is always a success.

\textbf{Voluntarily Giving up a Saving Throw}: A creature can voluntarily forego a saving throw and willingly accept a power's result. Even a character with a special resistance to psionics can suppress this quality.

\textbf{Items Surviving after a Saving Throw}: Unless the descriptive text for the power specifies otherwise, all items carried or worn by a creature are assumed to survive a psionic attack. If a creature rolls a natural 1 on its saving throw against the effect, however, an exposed item is harmed (if the attack can harm objects). Refer to Table: Items Affected by Psionic Attacks.

Determine which four objects carried or worn by the creature are most likely to be affected and roll randomly among them. The randomly determined item must make a saving throw against the attack form or take whatever damage the attack deals.

\Table{Items Affected by Psionic Attacks}{lX}{
\tableheader Order\footnotemark[1] & \tableheader Item\\
1st & Shield\\
2nd & Armor\\
3rd & Psionic or magic helmet, or psicrown\\
4th & Item in hand (including weapon, dorje, or the like)\\
5th & Psionic or magic cloak\\
6th & Stowed or sheathed weapon\\
7th & Psionic or magic bracers\\
8th & Psionic or magic clothing\\
9th & Psionic or magic jewelry (including rings)\\
10th & Anything else\\
\rowcolor{white}
\multicolumn{2}{l}{1 In order of most likely to least likely to be affected.}\\
}

\subsection{Power Resistance}
Power resistance is a special defensive ability. If your power is being resisted by a creature with power resistance, you must make a manifester level check (d20 + manifester level) at least equal to the creature's power resistance for the power to affect that creature. The defender's power resistance functions like an Armor Class against psionic attacks. Spell resistance is equivalent to power resistance unless the Psionics Is Different option is in use. Include any adjustments to your manifester level on this manifester level check.

The Power Resistance line and the descriptive text of a power description tell you whether power resistance protects creatures from the power. In many cases, power resistance applies only when a resistant creature is targeted by the power, not when a resistant creature encounters a power that is already in place.

The terms ``object'' and ``harmless'' mean the same thing for power resistance as they do for saving throws. A creature with power resistance must voluntarily lower the resistance (a standard action) to be affected by a power noted as harmless. In such a case, you do not need to make the manifester level check described above.

\subsection{Power Points}
All powers have a Power Points line, indicating the power's cost.

The psionic character class tables show how many power points a character has access to each day, depending on level.

A power's cost is determined by its level, as shown below. Every power's cost is noted in its description for ease of reference.

\Table{Power Points by Power Level}{l *{9}{C}}
{
\textbf{Power Level} & 1st & 2nd & 3rd & 4th & 5th & 6th & 7th & 8th & 9th\\
\textbf{Power Point Cost} & 1 & 3 & 5 & 7 & 9 & 11 & 13 & 15 & 17\\
}

\textbf{Power Point Limit}: Some powers allow you to spend more than their base cost to achieve an improved effect, or augment the power. The maximum number of points you can spend on a power (for any reason) is equal to your manifester level.

\textbf{XP Cost (XP)}: On the same line that the power point cost of a power is indicated, the power's experience point cost, if any, is noted. Particularly powerful effects entail an experience point cost to you. No spell or power can restore XP lost in this manner. You cannot spend so much XP that you lose a level, so you cannot manifest a power with an XP cost unless you have enough XP to spare. However, you can, on gaining enough XP to attain a new level, use those XP for manifesting a power rather than keeping them and advancing a level. The XP are expended when you manifest the power, whether or not the manifestation succeeds.

\subsection{Descriptive Text}
This portion of a power description details what the power does and how it works. If one of the previous lines in the description included ``see text,'' this is where the explanation is found. If the power you're reading about is based on another power you might have to refer to a different power for the ``see text'' information. If a power is the equivalent of a spell an entry of ``see spell text'' directs you to the appropriate spell description.

\textbf{Augment}: Many powers have variable effects based on the number of power points you spend when you manifest them. The more points spent, the more powerful the manifestation. How this extra expenditure affects a power is specific to the power. Some augmentations allow you to increase the number of damage dice, while others extend a power's duration or modify a power in unique ways. Each power that can be augmented includes an entry giving how many power points it costs to augment and the effects of doing so. However, you can spend only a total number of points on a power equal to your manifester level.

Augmenting a power takes place as part of another action (manifesting a power). Unless otherwise noted in the Augment section of an individual power description, you can augment a power only at the time you manifest it.

