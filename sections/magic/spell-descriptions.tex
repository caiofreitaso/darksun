\section{Spell Descriptions}
The description of each spell is presented in a standard format. Each category of information is explained and defined below.

\subsection{Name}
The first line of every spell description gives the name by which the spell is generally known.

\subsection{School (Subschool)}
Beneath the spell name is a line giving the school of magic (and the subschool, if appropriate) that the spell belongs to.

Almost every spell belongs to one of eight schools of magic. A school of magic is a group of related spells that work in similar ways. A small number of spells (arcane mark, limited wish, permanency, prestidigitation, and wish) are universal, belonging to no school.

\textbf{Abjuration}: Abjurations are protective spells. They create physical or magical barriers, negate magical or physical abilities, harm trespassers, or even banish the subject of the spell to another plane of existence.

If one abjuration spell is active within 10 feet of another for 24 hours or more, the magical fields interfere with each other and create barely visible energy fluctuations. The DC to find such spells with the Search skill drops by 4.

If an abjuration creates a barrier that keeps certain types of creatures at bay, that barrier cannot be used to push away those creatures. If you force the barrier against such a creature, you feel a discernible pressure against the barrier. If you continue to apply pressure, you end the spell.

\textbf{Conjuration}: Each conjuration spell belongs to one of five subschools. Conjurations bring manifestations of objects, creatures, or some form of energy to you (the summoning subschool), actually transport creatures from another plane of existence to your plane (calling), heal (healing), transport creatures or objects over great distances (teleportation), or create objects or effects on the spot (creation). Creatures you conjure usually, but not always, obey your commands.

A creature or object brought into being or transported to your location by a conjuration spell cannot appear inside another creature or object, nor can it appear floating in an empty space. It must arrive in an open location on a surface capable of supporting it.

The creature or object must appear within the spell's range, but it does not have to remain within the range.

\textit{Calling}: A calling spell transports a creature from another plane to the plane you are on. The spell grants the creature the one-time ability to return to its plane of origin, although the spell may limit the circumstances under which this is possible. Creatures who are called actually die when they are killed; they do not disappear and reform, as do those brought by a summoning spell (see below). The duration of a calling spell is instantaneous, which means that the called creature can't be dispelled.

\textit{Creation}: A creation spell manipulates matter to create an object or creature in the place the spellcaster designates (subject to the limits noted above). If the spell has a duration other than instantaneous, magic holds the creation together, and when the spell ends, the conjured creature or object vanishes without a trace. If the spell has an instantaneous duration, the created object or creature is merely assembled through magic. It lasts indefinitely and does not depend on magic for its existence.

\textit{Healing}: Certain divine conjurations heal creatures or even bring them back to life.

\textit{Summoning}: A summoning spell instantly brings a creature or object to a place you designate. When the spell ends or is dispelled, a summoned creature is instantly sent back to where it came from, but a summoned object is not sent back unless the spell description specifically indicates this. A summoned creature also goes away if it is killed or if its hit points drop to 0 or lower. It is not really dead. It takes 24 hours for the creature to reform, during which time it can't be summoned again.

When the spell that summoned a creature ends and the creature disappears, all the spells it has cast expire. A summoned creature cannot use any innate summoning abilities it may have, and it refuses to cast any spells that would cost it XP, or to use any spell-like abilities that would cost XP if they were spells.

\textit{Teleportation}: A teleportation spell transports one or more creatures or objects a great distance. The most powerful of these spells can cross planar boundaries. Unlike summoning spells, the transportation is (unless otherwise noted) one-way and not dispellable.

Teleportation is instantaneous travel through the Astral Plane. Anything that blocks astral travel also blocks teleportation.

\textbf{Divination}: Divination spells enable you to learn secrets long forgotten, to predict the future, to find hidden things, and to foil deceptive spells.

Many divination spells have cone-shaped areas. These move with you and extend in the direction you look. The cone defines the area that you can sweep each round. If you study the same area for multiple rounds, you can often gain additional information, as noted in the descriptive text for the spell.

\textit{Scrying}: A scrying spell creates an invisible magical sensor that sends you information. Unless noted otherwise, the sensor has the same powers of sensory acuity that you possess. This level of acuity includes any spells or effects that target you, but not spells or effects that emanate from you. However, the sensor is treated as a separate, independent sensory organ of yours, and thus it functions normally even if you have been blinded, deafened, or otherwise suffered sensory impairment.

Any creature with an Intelligence score of 12 or higher can notice the sensor by making a DC 20 Intelligence check. The sensor can be dispelled as if it were an active spell.

Lead sheeting or magical protection blocks a scrying spell, and you sense that the spell is so blocked.

\textbf{Enchantment}: Enchantment spells affect the minds of others, influencing or controlling their behavior.

All enchantments are mind-affecting spells. Two types of enchantment spells grant you influence over a subject creature.

\textit{Charm}: A charm spell changes how the subject views you, typically making it see you as a good friend.

\textit{Compulsion}: A compulsion spell forces the subject to act in some manner or changes the way her mind works. Some compulsion spells determine the subject's actions or the effects on the subject, some compulsion spells allow you to determine the subject's actions when you cast the spell, and others give you ongoing control over the subject.

\textbf{Evocation}: Evocation spells manipulate energy or tap an unseen source of power to produce a desired end. In effect, they create something out of nothing. Many of these spells produce spectacular effects, and evocation spells can deal large amounts of damage.

\textbf{Illusion}: Illusion spells deceive the senses or minds of others. They cause people to see things that are not there, not see things that are there, hear phantom noises, or remember things that never happened.

\textit{Figment}: A figment spell creates a false sensation. Those who perceive the figment perceive the same thing, not their own slightly different versions of the figment. (It is not a personalized mental impression.) Figments cannot make something seem to be something else. A figment that includes audible effects cannot duplicate intelligible speech unless the spell description specifically says it can. If intelligible speech is possible, it must be in a language you can speak. If you try to duplicate a language you cannot speak, the image produces gibberish. Likewise, you cannot make a visual copy of something unless you know what it looks like.

Because figments and glamers (see below) are unreal, they cannot produce real effects the way that other types of illusions can. They cannot cause damage to objects or creatures, support weight, provide nutrition, or provide protection from the elements. Consequently, these spells are useful for confounding or delaying foes, but useless for attacking them directly.

A figment's AC is equal to 10 + its size modifier.

\textit{Glamer}: A glamer spell changes a subject's sensory qualities, making it look, feel, taste, smell, or sound like something else, or even seem to disappear.

\textit{Pattern}: Like a figment, a pattern spell creates an image that others can see, but a pattern also affects the minds of those who see it or are caught in it. All patterns are mind-affecting spells.

\textit{Phantasm}: A phantasm spell creates a mental image that usually only the caster and the subject (or subjects) of the spell can perceive. This impression is totally in the minds of the subjects. It is a personalized mental impression. (It's all in their heads and not a fake picture or something that they actually see.) Third parties viewing or studying the scene don't notice the phantasm. All phantasms are mind-affecting spells.

\textit{Shadow}: A shadow spell creates something that is partially real from extradimensional energy. Such illusions can have real effects. Damage dealt by a shadow illusion is real.

\textit{Saving Throws and Illusions (Disbelief)}: Creatures encountering an illusion usually do not receive saving throws to recognize it as illusory until they study it carefully or interact with it in some fashion.

A successful saving throw against an illusion reveals it to be false, but a figment or phantasm remains as a translucent outline.

A failed saving throw indicates that a character fails to notice something is amiss. A character faced with proof that an illusion isn't real needs no saving throw. If any viewer successfully disbelieves an illusion and communicates this fact to others, each such viewer gains a saving throw with a +4 bonus.

\textbf{Necromancy}: Necromancy spells manipulate the power of death, unlife, and the life force. Spells involving undead creatures make up a large part of this school.

\textbf{Transmutation}: Transmutation spells change the properties of some creature, thing, or condition.

\subsection{[Descriptor]}
Appearing on the same line as the school and subschool, when applicable, is a descriptor that further categorizes the spell in some way. Some spells have more than one descriptor.

The descriptors are acid, air, chaotic, cold, darkness, death, earth, electricity, evil, fear, fire, force, good, language-dependent, lawful, light, mind-affecting, sonic, and water.

Most of these descriptors have no game effect by themselves, but they govern how the spell interacts with other spells, with special abilities, with unusual creatures, with alignment, and so on.

A language-dependent spell uses intelligible language as a medium for communication. If the target cannot understand or cannot hear what the caster of a language-dependant spell says the spell fails.

A mind-affecting spell works only against creatures with an Intelligence score of 1 or higher.

\subsection{Level}
The next line of a spell description gives the spell's level, a number between 0 and 9 that defines the spell's relative power. This number is preceded by an abbreviation for the class whose members can cast the spell. The Level entry also indicates whether a spell is a domain spell and, if so, what its domain and its level as a domain spell are. A spell's level affects the DC for any save allowed against the effect.

Names of spellcasting classes are abbreviated as follows: cleric Clr; druid Drd; ranger Rgr; templar Tmp; wizard Wiz.

The domains a spell can be associated with include Agriculture, Air, Animal, Chaos, Charm, Cleansing, Cycle, Death, Decay, Destruction, Drought, Earth, Fire, Forecasting, Freedom, Glory, Growth, Knowledge, Law, Madness, Magic, Magma, Mind, Mirage, Nobility, Plant, Protection, Purity, Rain, Replenishment, Silt, Strength, Sun, Travel, Trickery, War, Water, and Wrath.

\subsection{Components}
A spell's components are what you must do or possess to cast it. The Components entry in a spell description includes abbreviations that tell you what type of components it has. Specifics for material, focus, and XP components are given at the end of the descriptive text. Usually you don't worry about components, but when you can't use a component for some reason or when a material or focus component is expensive, then the components are important.

\textbf{Verbal (V)}: A verbal component is a spoken incantation. To provide a verbal component, you must be able to speak in a strong voice. A silence spell or a gag spoils the incantation (and thus the spell). A spellcaster who has been deafened has a 20\% chance to spoil any spell with a verbal component that he or she tries to cast.

\textbf{Somatic (S)}: A somatic component is a measured and precise movement of the hand. You must have at least one hand free to provide a somatic component.

\textbf{Material (M)}: A material component is one or more physical substances or objects that are annihilated by the spell energies in the casting process. Unless a cost is given for a material component, the cost is negligible. Don't bother to keep track of material components with negligible cost. Assume you have all you need as long as you have your spell component pouch.

\textbf{Focus (F)}: A focus component is a prop of some sort. Unlike a material component, a focus is not consumed when the spell is cast and can be reused. As with material components, the cost for a focus is negligible unless a price is given. Assume that focus components of negligible cost are in your spell component pouch.

\textbf{Divine Focus (DF)}: A divine focus component is an item of spiritual significance. The divine focus for a cleric or a paladin is a holy symbol appropriate to the character's faith.

If the Components line includes F/DF or M/DF, the arcane version of the spell has a focus component or a material component (the abbreviation before the slash) and the divine version has a divine focus component (the abbreviation after the slash).

\textbf{XP Cost (XP)}: Some powerful spells entail an experience point cost to you. No spell can restore the XP lost in this manner. You cannot spend so much XP that you lose a level, so you cannot cast the spell unless you have enough XP to spare. However, you may, on gaining enough XP to attain a new level, use those XP for casting a spell rather than keeping them and advancing a level. The XP are treated just like a material component---expended when you cast the spell, whether or not the casting succeeds.
\subsection{Casting Time}
Most spells have a casting time of 1 standard action. Others take 1 round or more, while a few require only a free action.

A spell that takes 1 round to cast is a full-round action. It comes into effect just before the beginning of your turn in the round after you began casting the spell. You then act normally after the spell is completed.

A spell that takes 1 minute to cast comes into effect just before your turn 1 minute later (and for each of those 10 rounds, you are casting a spell as a full-round action, just as noted above for 1-round casting times). These actions must be consecutive and uninterrupted, or the spell automatically fails.

When you begin a spell that takes 1 round or longer to cast, you must continue the concentration from the current round to just before your turn in the next round (at least). If you lose concentration before the casting is complete, you lose the spell.

A spell with a casting time of 1 free action doesn't count against your normal limit of one spell per round. However, you may cast such a spell only once per round. Casting a spell with a casting time of 1 free action doesn't provoke attacks of opportunity.

You make all pertinent decisions about a spell (range, target, area, effect, version, and so forth) when the spell comes into effect.

\subsection{Range}
A spell's range indicates how far from you it can reach, as defined in the Range entry of the spell description. A spell's range is the maximum distance from you that the spell's effect can occur, as well as the maximum distance at which you can designate the spell's point of origin. If any portion of the spell's area would extend beyond this range, that area is wasted. Standard ranges include the following.

\textbf{Personal:} The spell affects only you.

\textbf{Touch:} You must touch a creature or object to affect it. A touch spell that deals damage can score a critical hit just as a weapon can. A touch spell threatens a critical hit on a natural roll of 20 and deals double damage on a successful critical hit. Some touch spells allow you to touch multiple targets. You can touch as many willing targets as you can reach as part of the casting, but all targets of the spell must be touched in the same round that you finish casting the spell.

\textbf{Close:} The spell reaches as far as 7.5 meters away from you. The maximum range increases by 1.5 meter for every two full caster levels.

\textbf{Medium:} The spell reaches as far as 30 meters + 3 meters per caster level.

\textbf{Long:} The spell reaches as far as 120 meters + 12 meters per caster level.

\textbf{Unlimited:} The spell reaches anywhere on the same plane of existence.

\textbf{Range Expressed in Meters:} Some spells have no standard range category, just a range expressed in meters.
\subsection{Aiming A Spell}
You must make some choice about whom the spell is to affect or where the effect is to originate, depending on the type of spell. The next entry in a spell description defines the spell's target (or targets), its effect, or its area, as appropriate.

\textbf{Target or Targets}: Some spells have a target or targets. You cast these spells on creatures or objects, as defined by the spell itself. You must be able to see or touch the target, and you must specifically choose that target. You do not have to select your target until you finish casting the spell.

If the target of a spell is yourself (the spell description has a line that reads Target: You), you do not receive a saving throw, and spell resistance does not apply. The Saving Throw and Spell Resistance lines are omitted from such spells.

Some spells restrict you to willing targets only. Declaring yourself as a willing target is something that can be done at any time (even if you're flat-footed or it isn't your turn). Unconscious creatures are automatically considered willing, but a character who is conscious but immobile or helpless (such as one who is bound, cowering, grappling, paralyzed, pinned, or stunned) is not automatically willing.

Some spells allow you to redirect the effect to new targets or areas after you cast the spell. Redirecting a spell is a move action that does not provoke attacks of opportunity.

\textbf{Effect}: Some spells create or summon things rather than affecting things that are already present.

You must designate the location where these things are to appear, either by seeing it or defining it. Range determines how far away an effect can appear, but if the effect is mobile it can move regardless of the spell's range.

\textit{Ray}: Some effects are rays. You aim a ray as if using a ranged weapon, though typically you make a ranged touch attack rather than a normal ranged attack. As with a ranged weapon, you can fire into the dark or at an invisible creature and hope you hit something. You don't have to see the creature you're trying to hit, as you do with a targeted spell. Intervening creatures and obstacles, however, can block your line of sight or provide cover for the creature you're aiming at.

If a ray spell has a duration, it's the duration of the effect that the ray causes, not the length of time the ray itself persists.

If a ray spell deals damage, you can score a critical hit just as if it were a weapon. A ray spell threatens a critical hit on a natural roll of 20 and deals double damage on a successful critical hit.

\textit{Spread}: Some effects, notably clouds and fogs, spread out from a point of origin, which must be a grid intersection. The effect can extend around corners and into areas that you can't see. Figure distance by actual distance traveled, taking into account turns the spell effect takes. When determining distance for spread effects, count around walls, not through them. As with movement, do not trace diagonals across corners. You must designate the point of origin for such an effect, but you need not have line of effect (see below) to all portions of the effect.

\textbf{Area}: Some spells affect an area. Sometimes a spell description specifies a specially defined area, but usually an area falls into one of the categories defined below.

Regardless of the shape of the area, you select the point where the spell originates, but otherwise you don't control which creatures or objects the spell affects. The point of origin of a spell is always a grid intersection. When determining whether a given creature is within the area of a spell, count out the distance from the point of origin in squares just as you do when moving a character or when determining the range for a ranged attack. The only difference is that instead of counting from the center of one square to the center of the next, you count from intersection to intersection.

You can count diagonally across a square, but remember that every second diagonal counts as 2 squares of distance. If the far edge of a square is within the spell's area, anything within that square is within the spell's area. If the spell's area only touches the near edge of a square, however, anything within that square is unaffected by the spell.

\textit{Burst, Emanation, or Spread}: Most spells that affect an area function as a burst, an emanation, or a spread. In each case, you select the spell's point of origin and measure its effect from that point.

A burst spell affects whatever it catches in its area, even including creatures that you can't see. It can't affect creatures with total cover from its point of origin (in other words, its effects don't extend around corners). The default shape for a burst effect is a sphere, but some burst spells are specifically described as cone-shaped. A burst's area defines how far from the point of origin the spell's effect extends.

An emanation spell functions like a burst spell, except that the effect continues to radiate from the point of origin for the duration of the spell. Most emanations are cones or spheres.

A spread spell spreads out like a burst but can turn corners. You select the point of origin, and the spell spreads out a given distance in all directions. Figure the area the spell effect fills by taking into account any turns the spell effect takes.

\textit{Cone, Cylinder, Line, or Sphere}: Most spells that affect an area have a particular shape, such as a cone, cylinder, line, or sphere.

A cone-shaped spell shoots away from you in a quarter-circle in the direction you designate. It starts from any corner of your square and widens out as it goes. Most cones are either bursts or emanations (see above), and thus won't go around corners.

When casting a cylinder-shaped spell, you select the spell's point of origin. This point is the center of a horizontal circle, and the spell shoots down from the circle, filling a cylinder. A cylinder-shaped spell ignores any obstructions within its area.

A line-shaped spell shoots away from you in a line in the direction you designate. It starts from any corner of your square and extends to the limit of its range or until it strikes a barrier that blocks line of effect. A line-shaped spell affects all creatures in squares that the line passes through.

A sphere-shaped spell expands from its point of origin to fill a spherical area. Spheres may be bursts, emanations, or spreads.

\textit{Creatures}: A spell with this kind of area affects creatures directly (like a targeted spell), but it affects all creatures in an area of some kind rather than individual creatures you select. The area might be a spherical burst, a cone-shaped burst, or some other shape.

Many spells affect ``living creatures,'' which means all creatures other than constructs and undead. Creatures in the spell's area that are not of the appropriate type do not count against the creatures affected.

\textit{Objects}: A spell with this kind of area affects objects within an area you select (as Creatures, but affecting objects instead).

\textit{Other}: A spell can have a unique area, as defined in its description.

\textit{(S) Shapeable}: If an Area or Effect entry ends with ``(S),'' you can shape the spell. A shaped effect or area can have no dimension smaller than 10 feet. Many effects or areas are given as cubes to make it easy to model irregular shapes. Three-dimensional volumes are most often needed to define aerial or underwater effects and areas.

\textbf{Line of Effect}: A line of effect is a straight, unblocked path that indicates what a spell can affect. A line of effect is canceled by a solid barrier. It's like line of sight for ranged weapons, except that it's not blocked by fog, darkness, and other factors that limit normal sight.

You must have a clear line of effect to any target that you cast a spell on or to any space in which you wish to create an effect. You must have a clear line of effect to the point of origin of any spell you cast.

A burst, cone, cylinder, or emanation spell affects only an area, creatures, or objects to which it has line of effect from its origin (a spherical burst's center point, a cone-shaped burst's starting point, a cylinder's circle, or an emanation's point of origin).

An otherwise solid barrier with a hole of at least 1 square foot through it does not block a spell's line of effect. Such an opening means that the 5-foot length of wall containing the hole is no longer considered a barrier for purposes of a spell's line of effect.
\subsection{Duration}
A spell's Duration entry tells you how long the magical energy of the spell lasts.

\textbf{Timed Durations:} Many durations are measured in rounds, minutes, hours, or some other increment. When the time is up, the magic goes away and the spell ends. If a spell's duration is variable the duration is rolled secretly (the caster doesn't know how long the spell will last).

\textbf{Instantaneous:} The spell energy comes and goes the instant the spell is cast, though the consequences might be long-lasting.

\textbf{Permanent:} The energy remains as long as the effect does. This means the spell is vulnerable to dispel magic.

\textbf{Concentration:} The spell lasts as long as you concentrate on it. Concentrating to maintain a spell is a standard action that does not provoke attacks of opportunity. Anything that could break your concentration when casting a spell can also break your concentration while you're maintaining one, causing the spell to end.

You can't cast a spell while concentrating on another one. Sometimes a spell lasts for a short time after you cease concentrating.

\textbf{Subjects, Effects, and Areas:} If the spell affects creatures directly the result travels with the subjects for the spell's duration. If the spell creates an effect, the effect lasts for the duration. The effect might move or remain still. Such an effect can be destroyed prior to when its duration ends. If the spell affects an area then the spell stays with that area for its duration.

Creatures become subject to the spell when they enter the area and are no longer subject to it when they leave.

\textbf{Touch Spells and Holding the Charge:} In most cases, if you don't discharge a touch spell on the round you cast it, you can hold the charge (postpone the discharge of the spell) indefinitely. You can make touch attacks round after round. If you cast another spell, the touch spell dissipates.

Some touch spells allow you to touch multiple targets as part of the spell. You can't hold the charge of such a spell; you must touch all targets of the spell in the same round that you finish casting the spell.

\textbf{Discharge:} Occasionally a spells lasts for a set duration or until triggered or discharged.

\textbf{(D) Dismissible:} If the Duration line ends with  ``(D),'' you can dismiss the spell at will. You must be within range of the spell's effect and must speak words of dismissal, which are usually a modified form of the spell's verbal component. If the spell has no verbal component, you can dismiss the effect with a gesture. Dismissing a spell is a standard action that does not provoke attacks of opportunity.

A spell that depends on concentration is dismissible by its very nature, and dismissing it does not take an action, since all you have to do to end the spell is to stop concentrating on your turn.
\subsection{Saving Throw}
Usually a harmful spell allows a target to make a saving throw to avoid some or all of the effect. The Saving Throw entry in a spell description defines which type of saving throw the spell allows and describes how saving throws against the spell work.

\textbf{Negates}: The spell has no effect on a subject that makes a successful saving throw.

\textbf{Partial}: The spell causes an effect on its subject. A successful saving throw means that some lesser effect occurs.

\textbf{Half}: The spell deals damage, and a successful saving throw halves the damage taken (round down).

\textbf{None}: No saving throw is allowed.

\textbf{Disbelief}: A successful save lets the subject ignore the effect.

\textbf{(object)}: The spell can be cast on objects, which receive saving throws only if they are magical or if they are attended (held, worn, grasped, or the like) by a creature resisting the spell, in which case the object uses the creature's saving throw bonus unless its own bonus is greater. (This notation does not mean that a spell can be cast only on objects. Some spells of this sort can be cast on creatures or objects.) A magic item's saving throw bonuses are each equal to 2 + one-half the item's caster level.

\textbf{(harmless)}: The spell is usually beneficial, not harmful, but a targeted creature can attempt a saving throw if it desires.

\textbf{Saving Throw Difficulty Class}: A saving throw against your spell has a DC of 10 + the level of the spell + your bonus for the relevant ability (Intelligence for a wizard, Charisma for a templar, or Wisdom for a cleric, druid, and ranger). A spell's level can vary depending on your class. Always use the spell level applicable to your class.

\textbf{Succeeding on a Saving Throw}: A creature that successfully saves against a spell that has no obvious physical effects feels a hostile force or a tingle, but cannot deduce the exact nature of the attack. Likewise, if a creature's saving throw succeeds against a targeted spell you sense that the spell has failed. You do not sense when creatures succeed on saves against effect and area spells.

\textbf{Automatic Failures and Successes}: A natural 1 (the d20 comes up 1) on a saving throw is always a failure, and the spell may cause damage to exposed items (see Items Surviving after a Saving Throw, below). A natural 20 (the d20 comes up 20) is always a success.

\textbf{Voluntarily Giving up a Saving Throw}: A creature can voluntarily forego a saving throw and willingly accept a spell's result. Even a character with a special resistance to magic can suppress this quality.

\textbf{Items Surviving after a Saving Throw}: Unless the descriptive text for the spell specifies otherwise, all items carried or worn by a creature are assumed to survive a magical attack. If a creature rolls a natural 1 on its saving throw against the effect, however, an exposed item is harmed (if the attack can harm objects). Refer to \tabref{Items Affected by Magical Attacks}. Determine which four objects carried or worn by the creature are most likely to be affected and roll randomly among them. The randomly determined item must make a saving throw against the attack form and take whatever damage the attack deal.

If an item is not carried or worn and is not magical, it does not get a saving throw. It simply is dealt the appropriate damage.

\Table{Items Affected by Magical Attacks}{lX}{
\tableheader Order\footnotemark[1] & \tableheader Item\\
1st & Shield\\
2nd & Armor\\
3rd & Magic helmet, hat, or headband\\
4th & Item in hand (including weapon, wand, or the like)\\
5th & Magic cloak\\
6th & Stowed or sheathed weapon\\
7th & Magic bracers\\
8th & Magic clothing\\
9th & Magic jewelry (including rings)\\
10th & Anything else\\

\TableNote{2}{1 In order of most likely to least likely to be affected.}\\
}

\subsection{Spell Resistance}
Spell resistance is a special defensive ability. If your spell is being resisted by a creature with spell resistance, you must make a caster level check (1d20 + caster level) at least equal to the creature's spell resistance for the spell to affect that creature. The defender's spell resistance is like an Armor Class against magical attacks. Include any adjustments to your caster level to this caster level check.

The Spell Resistance entry and the descriptive text of a spell description tell you whether spell resistance protects creatures from the spell. In many cases, spell resistance applies only when a resistant creature is targeted by the spell, not when a resistant creature encounters a spell that is already in place.

The terms ``object'' and ``harmless'' mean the same thing for spell resistance as they do for saving throws. A creature with spell resistance must voluntarily lower the resistance (a standard action) in order to be affected by a spell noted as harmless. In such a case, you do not need to make the caster level check described above.

\subsection{Descriptive Text}
This portion of a spell description details what the spell does and how it works. If one of the previous entries in the description included ``see text,'' this is where the explanation is found.