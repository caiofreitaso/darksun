\section{Casting Spells}
Whether a spell is arcane or divine, and whether a character prepares spells in advance or chooses them on the spot, casting a spell works the same way.

\subsection{Choosing A Spell}
First you must choose which spell to cast. If you're a cleric, druid, experienced paladin, experienced ranger, or wizard, you select from among spells prepared earlier in the day and not yet cast (see Preparing Wizard Spells and Preparing Divine Spells).

If you're a bard or sorcerer, you can select any spell you know, provided you are capable of casting spells of that level or higher.

To cast a spell, you must be able to speak (if the spell has a verbal component), gesture (if it has a somatic component), and manipulate the material components or focus (if any). Additionally, you must concentrate to cast a spell.

If a spell has multiple versions, you choose which version to use when you cast it. You don't have to prepare (or learn, in the case of a bard or sorcerer) a specific version of the spell.

Once you've cast a prepared spell, you can't cast it again until you prepare it again. (If you've prepared multiple copies of a single spell, you can cast each copy once.) If you're a bard or sorcerer, casting a spell counts against your daily limit for spells of that spell level, but you can cast the same spell again if you haven't reached your limit.

\subsection{Concentration}
To cast a spell, you must concentrate. If something interrupts your concentration while you're casting, you must make a \skill{Concentration} check or lose the spell. The more distracting the interruption and the higher the level of the spell you are trying to cast, the higher the DC is. If you fail the check, you lose the spell just as if you had cast it to no effect.

\textbf{Injury}: If while trying to cast a spell you take damage, you must make a \skill{Concentration} check (DC 10 + points of damage taken + the level of the spell you're casting). If you fail the check, you lose the spell without effect. The interrupting event strikes during spellcasting if it comes between when you start and when you complete a spell (for a spell with a casting time of 1 full round or more) or if it comes in response to your casting the spell (such as an attack of opportunity provoked by the spell or a contingent attack, such as a readied action).

If you are taking continuous damage half the damage is considered to take place while you are casting a spell. You must make a \skill{Concentration} check (DC 10 + \onehalf the damage that the continuous source last dealt + the level of the spell you're casting). If the last damage dealt was the last damage that the effect could deal then the damage is over, and it does not distract you.

Repeated damage does not count as continuous damage.

\textbf{Spell}: If you are affected by a spell while attempting to cast a spell of your own, you must make a \skill{Concentration} check or lose the spell you are casting. If the spell affecting you deals damage, the DC is 10 + points of damage + the level of the spell you're casting.

If the spell interferes with you or distracts you in some other way, the DC is the spell's saving throw DC + the level of the spell you're casting. For a spell with no saving throw, it's the DC that the spell's saving throw would have if a save were allowed.

\textbf{Grappling or Pinned}: The only spells you can cast while grappling or pinned are those without somatic components and whose material components (if any) you have in hand. Even so, you must make a \skill{Concentration} check (DC 20 + the level of the spell you're casting) or lose the spell.

\textbf{Vigorous Motion}: If you are riding on a moving mount, taking a bouncy ride in a wagon, on a small boat in rough water, below-decks in a storm-tossed ship, or simply being jostled in a similar fashion, you must make a \skill{Concentration} check (DC 10 + the level of the spell you're casting) or lose the spell.

\textbf{Violent Motion}: If you are on a galloping horse, taking a very rough ride in a wagon, on a small boat in rapids or in a storm, on deck in a storm-tossed ship, or being tossed roughly about in a similar fashion, you must make a \skill{Concentration} check (DC 15 + the level of the spell you're casting) or lose the spell.

\textbf{Violent Weather}: You must make a \skill{Concentration} check if you try to cast a spell in violent weather. If you are in a high wind carrying blinding rain or sleet, the DC is 5 + the level of the spell you're casting. If you are in wind-driven hail, dust, or debris, the DC is 10 + the level of the spell you're casting. In either case, you lose the spell if you fail the \skill{Concentration} check. If the weather is caused by a spell, use the rules in the Spell subsection above.

\textbf{Casting Defensively}: If you want to cast a spell without provoking any attacks of opportunity, you must make a \skill{Concentration} check (DC 15 + the level of the spell you're casting) to succeed. You lose the spell if you fail.

\textbf{Entangled}: If you want to cast a spell while entangled in a net or by a tanglefoot bag or while you're affected by a spell with similar effects, you must make a DC 15 \skill{Concentration} check to cast the spell. You lose the spell if you fail.
\subsection{Counterspells}
It is possible to cast any spell as a counterspell. By doing so, you are using the spell's energy to disrupt the casting of the same spell by another character. Counterspelling works even if one spell is divine and the other arcane.

\textbf{How Counterspells Work:} To use a counterspell, you must select an opponent as the target of the counterspell. You do this by choosing the ready action. In doing so, you elect to wait to complete your action until your opponent tries to cast a spell. (You may still move your speed, since ready is a standard action.)

If the target of your counterspell tries to cast a spell, make a \skill{Spellcraft} check (DC 15 + the spell's level). This check is a free action. If the check succeeds, you correctly identify the opponent's spell and can attempt to counter it. If the check fails, you can't do either of these things.

To complete the action, you must then cast the correct spell. As a general rule, a spell can only counter itself. If you are able to cast the same spell and you have it prepared (if you prepare spells), you cast it, altering it slightly to create a counterspell effect. If the target is within range, both spells automatically negate each other with no other results.

\textbf{Counterspelling Metamagic Spells:} Metamagic feats are not taken into account when determining whether a spell can be countered

\textbf{Specific Exceptions:} Some spells specifically counter each other, especially when they have diametrically opposed effects.

\textbf{Dispel Magic as a Counterspell:} You can use dispel magic to counterspell another spellcaster, and you don't need to identify the spell he or she is casting. However, dispel magic doesn't always work as a counterspell.

\subsection{Caster Level}
A spell's power often depends on its caster level, which for most spellcasting characters is equal to your class level in the class you're using to cast the spell.

You can cast a spell at a lower caster level than normal, but the caster level you choose must be high enough for you to cast the spell in question, and all level-dependent features must be based on the same caster level.

In the event that a class feature, domain granted power, or other special ability provides an adjustment to your caster level, that adjustment applies not only to effects based on caster level (such as range, duration, and damage dealt) but also to your caster level check to overcome your target's spell resistance and to the caster level used in dispel checks (both the dispel check and the DC of the check).

\textbf{Caster Level Checks}: To make a caster level check, roll 1d20 and add your caster level (in the relevant class). If the result equals or exceeds the DC (or the spell resistance, in the case of caster level checks made for spell resistance), the check succeeds.

\subsection{Spell Failure}
If you ever try to cast a spell in conditions where the characteristics of the spell cannot be made to conform, the casting fails and the spell is wasted.

Spells also fail if your concentration is broken and might fail if you're wearing armor while casting a spell with somatic components.

\subsection{The Spell's Result}
Once you know which creatures (or objects or areas) are affected, and whether those creatures have made successful saving throws (if any were allowed), you can apply whatever results a spell entails.

\subsection{Special Spell Effects}
Many special spell effects are handled according to the school of the spells in question Certain other special spell features are found across spell schools.

\textbf{Attacks:} Some spell descriptions refer to attacking. All offensive combat actions, even those that don't damage opponents are considered attacks. Attempts to turn or rebuke undead count as attacks. All spells that opponents resist with saving throws, that deal damage, or that otherwise harm or hamper subjects are attacks. Spells that summon monsters or other allies are not attacks because the spells themselves don't harm anyone.

\textbf{Bonus Types:} Usually, a bonus has a type that indicates how the spell grants the bonus. The important aspect of bonus types is that two bonuses of the same type don't generally stack. With the exception of dodge bonuses, most circumstance bonuses, and racial bonuses, only the better bonus works (see Combining Magical Effects, below). The same principle applies to penalties---a character taking two or more penalties of the same type applies only the worst one.

\textbf{Bringing Back the Dead:} Several spells have the power to restore slain characters to life.

When a living creature dies, its soul departs its body, leaves the Material Plane, travels through the Astral Plane, and goes to abide on the plane where the creature's deity resides. If the creature did not worship a deity, its soul departs to the plane corresponding to its alignment. Bringing someone back from the dead means retrieving his or her soul and returning it to his or her body.

\textit{Level Loss:} Any creature brought back to life usually loses one level of experience. The character's new XP total is midway between the minimum needed for his or her new (reduced) level and the minimum needed for the next one. If the character was 1st level at the time of death, he or she loses 2 points of Constitution instead of losing a level.

This level loss or Constitution loss cannot be repaired by any mortal means, even \spell{wish} or \spell{miracle}. A revived character can regain a lost level by earning XP through further adventuring. A revived character who was 1st level at the time of death can regain lost points of Constitution by improving his or her Constitution score when he or she attains a level that allows an ability score increase.

\textit{Preventing Revivification:} Enemies can take steps to make it more difficult for a character to be returned from the dead. Keeping the body prevents others from using \spell{raise dead} or \spell{resurrection} to restore the slain character to life. Casting \spell{trap the soul} prevents any sort of revivification unless the soul is first released.

\textit{Revivification against One's Will:} A soul cannot be returned to life if it does not wish to be. A soul knows the name, alignment, and patron deity (if any) of the character attempting to revive it and may refuse to return on that basis.


\subsection{Combining Magical Effects}
Spells or magical effects usually work as described, no matter how many other spells or magical effects happen to be operating in the same area or on the same recipient. Except in special cases, a spell does not affect the way another spell operates. Whenever a spell has a specific effect on other spells, the spell description explains that effect. Several other general rules apply when spells or magical effects operate in the same place:

\textbf{Stacking Effects:} Spells that provide bonuses or penalties on attack rolls, damage rolls, saving throws, and other attributes usually do not stack with themselves. More generally, two bonuses of the same type don't stack even if they come from different spells (or from effects other than spells; see Bonus Types, above).

\textit{Different Bonus Names:} The bonuses or penalties from two different spells stack if the modifiers are of different types. A bonus that isn't named stacks with any bonus.

\textit{Same Effect More than Once in Different Strengths:} In cases when two or more identical spells are operating in the same area or on the same target, but at different strengths, only the best one applies.

\textit{Same Effect with Differing Results:} The same spell can sometimes produce varying effects if applied to the same recipient more than once. Usually the last spell in the series trumps the others. None of the previous spells are actually removed or dispelled, but their effects become irrelevant while the final spell in the series lasts.

\textit{One Effect Makes Another Irrelevant:} Sometimes, one spell can render a later spell irrelevant. Both spells are still active, but one has rendered the other useless in some fashion.

\textit{Multiple Mental Control Effects:} Sometimes magical effects that establish mental control render each other irrelevant, such as a spell that removes the subjects ability to act. Mental controls that don't remove the recipient's ability to act usually do not interfere with each other. If a creature is under the mental control of two or more creatures, it tends to obey each to the best of its ability, and to the extent of the control each effect allows. If the controlled creature receives conflicting orders simultaneously, the competing controllers must make opposed Charisma checks to determine which one the creature obeys.

\textbf{Spells with Opposite Effects:} Spells with opposite effects apply normally, with all bonuses, penalties, or changes accruing in the order that they apply. Some spells negate or counter each other. This is a special effect that is noted in a spell's description.

\textbf{Instantaneous Effects:} Two or more spells with instantaneous durations work cumulatively when they affect the same target.