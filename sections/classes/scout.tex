\Class{Scout}
{Raiding tribe. Twenty footmen, three on kanks. They seem to be from the Mekillot Mountains. Be careful with the one wearing a kank's head.}
{Lobuu Airhunter, elf scout}

\Figure[\columnwidth+2mm]{t}{images/adventurer-4.png}
Scouts are invaluable assets for venturing the wastes, as any caravan master can tell. They can provide information about the surrounding area in time for any incoming battle. Scouts accomplish this by traveling faster than normal across difficult terrain, and specializing in keeping themselves unseen while detecting the foes and dangers ahead.

Merchant Houses are the primary employers for scouts around Athas, as they need to secure each caravan against not only the dangers of the wasteland but against ambushes from enemy houses. Each city-state also have some kind of division for scouts. Wars are a always about intelligence and logistics, and templars do not wish to fall in disgrace because of some blunder that could be avoided by deploying scouts. Scouts are also the primary military force on nomad tribes, since they need haste to move to new fields.

\WarriorTable{The Scout}{
1st  & +0         & +0 & +2  & +0 & Fast movement, skirmish (+1d6), trapfinding \\
2nd  & +1         & +0 & +3  & +0 & Uncanny dodge                               \\
3rd  & +2         & +1 & +3  & +1 & Skirmish (+1d6/+1 AC), trackless step       \\
4th  & +3         & +1 & +4  & +1 & Evasion                                     \\
5th  & +3         & +1 & +4  & +1 & Skirmish (+2d6/+1 AC)                       \\
6th  & +4         & +2 & +5  & +2 & Trade secret                                \\
7th  & +5         & +2 & +5  & +2 & Skirmish (+2d6/+2 AC)                       \\
8th  & +6/+1      & +2 & +6  & +2 & Bonus feat, camouflage                      \\
9th  & +6/+1      & +3 & +6  & +3 & Skirmish (+3d6/+2 AC)                       \\
10th & +7/+2      & +3 & +7  & +3 & Special ability, trade secret               \\
11th & +8/+3      & +3 & +7  & +3 & Skirmish (+3d6/+3 AC)                       \\
12th & +9/+4      & +4 & +8  & +4 & Bonus feat                                  \\
13th & +9/+4      & +4 & +8  & +4 & Skirmish (+4d6/+3 AC)                       \\
14th & +10/+5     & +4 & +9  & +4 & Hide in plain sight, trade secret           \\
15th & +11/+6/+1  & +5 & +9  & +5 & Special ability, skirmish (+4d6/+4 AC)      \\
16th & +12/+7/+2  & +5 & +10 & +5 & Bonus feat                                  \\
17th & +12/+7/+2  & +5 & +10 & +5 & Skirmish (+5d6/+4 AC)                       \\
18th & +13/+8/+3  & +6 & +11 & +6 & Free movement, trade secret                 \\
19th & +14/+9/+4  & +6 & +11 & +6 & Skirmish (+5d6/+5 AC)                       \\
20th & +15/+10/+5 & +6 & +12 & +6 & Bonus feat, special ability                 \\
}

\subsection{Making a Scout}
Since scouts are more fragile and have less combat effectiveness than a mul fighter or a half-giant barbarian, they are not meant to fight fairly. Scouts focus on always being at an advantage: they are always on the move, stay hidden whenever possible, and use difficult terrain as safety against their enemies. Most of their class features aren't unique to them, but their combination makes scout the best survivalist.

\textbf{Races:} Elves make the majority of Athasian scouts, due to their faster movement and their nomad culture that relies on scouts to warn the tribe of any dangers further ahead. Human and half-elf scouts are usually hired by merchant houses to attend to caravans and prevent them to stumble upon any monsters or ambushes. Halflings also employ scouts in the Forest Ridge, as a way to prepare ambushes against any big-folk trying to invade their land. Thri-kreen use scouts in their all their packs: hunting packs need scouts to search for fresh quarry, while raiding packs look for vulnerable (and tasty) prey. Dwarves, muls, and half-giants don't usually are scouts as their races are too bulky to be successful scouts.

\textbf{Alignment:} Scouts can fit into any society, and can be of any alignment. Those in a military organization are usually lawful, while those in nomad tribes are usually chaotic. But scouts have no restriction whatsoever.

\subsection{Game Rule Information}
\textbf{Hit Die:} d8.

\Figure{t}{images/ranger-1.png}
\subsubsection{Class Skills}
\skill{Balance} (Dex),
\skill{Climb} (Str),
\skill{Craft} (Int),
\skill{Disable Device} (Int),
\skill{Escape Artist} (Dex),
\skill{Hide} (Dex),
\skill{Jump} (Str),
\skill{Knowledge} (dungeoneering) (Int),
\skill{Knowledge} (geography) (Int),
\skill{Knowledge} (nature) (Int),
\skill{Knowledge} (warcraft) (Int),
\skill{Listen} (Wis),
\skill{Move Silently} (Dex),
\skill{Ride} (Wis),
\skill{Search} (Int),
\skill{Sense Motive} (Wis),
\skill{Speak Language} (N/A),
\skill{Spot} (Wis),
\skill{Survival} (Wis),
\skill{Tumble} (Dex),
and \skill{Use Rope} (Dex).

\textbf{Skill Points per Level:} 8 + Int modifier ($\times4$ at 1st level).

\subsubsection{Class Features}
\textbf{Weapon and Armor Proficiency:} Scouts are proficient with all simple weapons, plus the handaxe, short sword, and shortbow. Scouts are proficient with light armor, but not with shields.


\textbf{Fast Movement (Ex):} A scout's land speed is faster than the norm for his race by +3 meters. A scout loses this benefit when wearing medium or heavy armor or when carrying a medium or heavy load.


\textbf{Skirmish (Ex):} A scout relies on mobility to deal extra damage and improve his defense. He deals an extra 1d6 points of damage on all attacks he makes during any round in which he moves at least 3 meters away from where she was at the start of her turn. The extra damage applies only to attacks made after the scout has moved at least 3 meters. The extra damage applies only to attacks taken during the scout's turn. This extra damage increases by 1d6 for every four levels gained above 1st (2d6 at 5th, 3d6 at 9th, 4d6 at 13th, and 5d6 at 17th level).

The extra damage only applies against living creatures that have a discernible anatomy. Undead, constructs, oozes, plants, incorporeal creatures, and creatures immune to extra damage from critical hits are not vulnerable to this additional damage. The scout must be able to see the target well enough to pick out a vital spot and must be able to reach such a spot. Scouts can apply this extra damage to ranged attacks made while skirmishing, but only if the target is within 9 meters.

At 3rd level, a scout gains a +1 competence bonus to Armor Class during any round in which he moves at least 3 meters. The bonus applies as soon as the scout has moved 3 meters, and lasts until the start of his next turn. This bonus improves by 1 for every four levels gained above 3rd (+2 at 7th, +3 at 11th, +4 at 15th, and +5 at 19th level).

A scout loses this ability when wearing medium or heavy armor or when carrying a medium or heavy load. If he gains the skirmish ability from another source, the bonuses stack.


\textbf{Trapfinding:} Scouts can use the \skill{Search} skill to locate traps when the task has a Difficulty Class higher than 20.

Finding a nonmagical trap has a DC of at least 20, or higher if it is well hidden. Finding a magic trap has a DC of 25 + the level of the spell used to create it.

Scouts can use the \skill{Disable Device} skill to disarm magic traps. A magic trap generally has a DC of 25 + the level of the spell used to create it.

A scout who beats a trap's DC by 10 or more with a \skill{Disable Device} check can study a trap, figure out how it works, and bypass it (with his party) without disarming it.


\textbf{Uncanny Dodge (Ex):} Starting at 2nd level, a scout can react to danger before his senses would normally allow him to do so. He retains his Dexterity bonus to AC (if any) even if he is caught flat-footed or struck by an invisible attacker. However, he still loses his Dexterity bonus to AC if immobilized.

If a scout already has uncanny dodge from a different class he automatically gains improved uncanny dodge instead.


\textbf{Trackless Step (Ex):} Starting at 3rd level, a scout leaves no trail in natural surroundings and cannot be tracked. He may choose to leave a trail if so desired.


\textbf{Evasion (Ex):} At 4th level and higher, a scout can avoid even magical and unusual attacks with great agility. If he makes a successful Reflex saving throw against an attack that normally deals half damage on a successful save, he instead takes no damage. Evasion can be used only if the scout is wearing light armor or no armor. A helpless scout does not gain the benefit of evasion.


\textbf{Trade Secrets:} At 6th level and every four levels thereafter (10th, 14th, and 18th level), a scout learns a trade secret chosen from the list below.

\textit{Accurate (Ex):} When a scout with this trade secret attacks, his accuracy allows him to ignore a number of points of armor or natural armor bonus to AC equal to \onequarter his scout level.

\textit{Battle Fortitude (Ex):} A scout adds \onequarter his scout level as a competence bonus on Fortitude saves and initiative checks. A scout loses this benefit when wearing medium or heavy armor or when carrying a medium or heavy load.

\textit{Deadly Range (Ex):} A scout with this talent increases the range at which he can deal skirmish damage by 6 meters. A scout can select this talent more than once; its effects stack.

\textit{Faster Movement (Ex):} A scout increases his land speed by +3 meters. This accumulates with the fast movement ability. A scout loses this benefit when wearing medium or heavy armor or when carrying a medium or heavy load.

\textit{Flawless Stride (Ex):} A scout can move through any sort of terrain that slows movement (such as undergrowth, rubble, and similar terrain) at his normal speed and without taking damage or suffering any other impairment.

This ability does not let him move more quickly through terrain that requires a \skill{Climb} or \skill{Swim} check to navigate, nor can he move more quickly through terrain or undergrowth that has been magically manipulated to impede motion.

A scout loses this benefit when wearing medium or heavy armor or when carrying a medium or heavy load.

\textit{Poison Use:} A scout never risk accidentally poisoning himself when applying poison to a blade.

\textit{Resilient (Ex):} A scout with this trade secret adds \onehalf his scout level as a luck bonus to saving throws against poisons, spells, and spell-like abilities.

\textit{Seasoned Explorer (Ex):} A scout of 14th level or higher can make \skill{Survival} checks while moving at his full overland speed without the $-30$ penalty.

\textit{Scout Feat:} A scout may select a luck feat\textsuperscript{CS}, a tactical feat\textsuperscript{CW}, or \feat{Skill Focus} in place of a trade secret. He must meet all the prerequisites for the feat. This trade secret may be chosen more than once, each time the scout must select a different feat.

\textit{Skilled:} A scout with this trade secret adds \onequarter your scout level as a competence bonus to one of the following skills: \skill{Balance}, \skill{Climb}, \skill{Hide}, \skill{Listen}, \skill{Move Silently}, \skill{Ride}, \skill{Search}, \skill{Spot}, \skill{Survival}. This trade secret may be chosen more than once, each time it applies to a different skill.

\textit{Versatile:} A scout with this trade secret selects any two non-class skills to be considered class skills.


\textbf{Bonus Feat:} At 8th level and every four levels thereafter (12th, 16th, and 20th level), a scout gains a bonus feat, which must be selected from the following list:
\feat{Blind-Fight},
Bounding Assault\textsuperscript{PH2},
Brachiation\textsuperscript{CAd},
\feat{Combat Expertise},
Danger Sense\textsuperscript{CAd},
Dash\textsuperscript{CW},
\feat{Dodge},
\feat{Endurance},
\feat{Far Shot},
Hear the Unseen\textsuperscript{CAd},
\feat{Improved Initiative},
Improved Skirmish\textsuperscript{CS},
Improved Swimming\textsuperscript{CAd},
Keen-Eared Scout\textsuperscript{PH2},
Melee Evasion\textsuperscript{PH2},
\feat{Mobility},
\feat{Point Blank Shot},
\feat{Precise Shot},
\feat{Quick Draw},
Quick Reconnoiter\textsuperscript{CAd},
\feat{Rapid Reload},
\feat{Shot On The Run},
\feat{Spring Attack},
\feat{Swift Ambusher}\textsuperscript{CS},
\feat{Swift Hunter}\textsuperscript{CS},
\feat{Track},
\feat{Wastelander}.
He must meet all the prerequisites for the feat.


\textbf{Camouflage (Ex):} A scout of 8th level or higher can use the \skill{Hide} skill in any sort of natural terrain, even if the terrain doesn't grant cover or concealment.


\textbf{Special Ability:} On attaining 10th level, and at every five levels thereafter (15th, and 20th), a scout gains a special ability of his choice from among the following options.

\textit{Blindsense (Ex):} A scout gains the blindsense ability out to 9 meters. He does not need to make \skill{Spot} or \skill{Listen} checks to notice and locate creatures within range of his blindsense ability, provided that he has line of effect to that creature. Any opponent the scout cannot see has total concealment (50\% miss chance) against him, and the scout still has the normal miss chance when attacking foes that have concealment. Visibility still affects the scout's movement. A scout is still denied its Dexterity bonus to Armor Class against attacks from creatures he cannot see.

\textit{Defensive Roll (Ex):} A scout learns how to avoid a potentially lethal blow to take less damage from it than he otherwise would. Once per day, when he would be reduced to 0 or fewer hit points by damage in combat (from a weapon or other blow, not a spell or special ability), the scout can attempt to roll with the damage. To use this ability, the scout must attempt a Reflex saving throw (DC = damage dealt). If the save succeeds, he takes only half damage from the blow; if it fails, he takes full damage. He must be aware of the attack and able to react to it in order to execute his defensive roll---if he is denied his Dexterity bonus to AC, he can't use this ability. Since this effect would not normally allow a character to make a Reflex save for half damage, the rogue's evasion ability does not apply to the defensive roll.

\textit{Favored Terrain (Ex):} A scout chooses a terrain given on \tabref{Athasian Terrains} to specialize. A scout receives a +2 bonus on \skill{Hide}, \skill{Listen}, \skill{Move Silently}, \skill{Search}, \skill{Spot} and \skill{Survival} checks and initiative checks made within his favored terrain. Likewise, he gets +2 bonus on \skill{Knowledge} (geography) and \skill{Knowledge} (nature) checks about his favored terrain.

If a scout also has ranger levels, this ability uses the same graduated progression that the favored terrain ability receives.

\textit{Improved Evasion (Ex):} This ability works like evasion, except that while the scout still takes no damage on a successful Reflex saving throw against attacks henceforth she takes only half damage on a failed save. A helpless scout does not gain the benefit of improved evasion.

\textit{Trap Sense (Ex):} If a scout passes within 1.5 meter of a trap, she is entitled to a \skill{Search} check to notice it as if she was actively looking for it.

\textit{Feat:} A scout may gain a bonus feat in place of a special ability.


\textbf{Hide in Plain Sight (Ex):} While in any sort of natural terrain, a scout of 14th level or higher can use the \skill{Hide} skill even while being observed.


\textbf{Free Movement (Ex):} At 18th level and higher, a scout can slip out of bonds, grapples, and even the effects of confining spells easily. This ability duplicates the effect of a \spell{freedom of movement} spell, except that it is always active. A scout loses this benefit when wearing medium or heavy armor or when carrying a medium or heavy load.



\subsection{Playing a Scout}
Scouts trained in the military rarely have the time to go in adventures. An adventuring scout usually comes from nomad tribes or rural villages, or even ex-military who left the service behind. They respect the dangers ahead, as their function is primarily to detect dangers and not to face them.

\subsubsection{Religion}
Scouts are generally not devoted to any element. Scouts in service of any infantry might pay homage to the element their cleric is devoted, while those in nomad tribes pay homage to the Spirits of the Land.

\subsubsection{Other Classes}
Scouts don't have problems with any class, as long as they get to do their job unimpeded. When they need to run ahead, they will not slow down because of fighter on a heavy armor. Scouts have great affinity with barbarians and rangers. They can either keep up with scouts as they travel, or they can stay undetected while they make any observation.

\subsubsection{Combat}
You serve as a backup melee combatant or ranged expert in battle. Your role is to support any warrior in the group, with your unique combat style. Scouts shine best in large battlefields where you can be always on the move, and in difficult terrains where you can walk in greater rate than your enemies. As a scout, you are not expected to fight fair, so use the environment as best as you can to hit and not be hit.

\subsubsection{Advancement}
Depending on your role and what is expected in your adventures, taking assigning skill points to the available \skill{Knowledge} fields is essential to be able to understand what dangers actually lie ahead. Likewise, \skill{Hide}, \skill{Listen}, \skill{Move Silently}, and \skill{Spot} are a vital part of scouting since detecting trouble from further distances while being undetected improves your chances to survive another day. Melee scouts tend to depend on \skill{Tumble} to move within their enemies' reach.

Due to your combat style, it is natural to take either the \feat{Shot on the Run} feat or the \feat{Spring Attack} feat, and all their prerequisites. Taking Improved Skirmish\textsuperscript{CS} is also expected at higher levels. If you wander too much on the wilderness, consider becoming a \class{Master Scout}.

\subsection{Starting Packages}
\subsubsection{The Archer}
Elf Scout

\textbf{Ability Scores:} Str 14, Dex 17, Con 10, Int 10, Wis 13, Cha 8.

\textbf{Skills:} \skill{Balance}, \skill{Climb}, \skill{Hide}, \skill{Listen}, \skill{Move Silently}, \skill{Search}, \skill{Spot}, \skill{Survival}, \skill{Tumble}, \skill{Use Rope}.

\textbf{Languages:} Common, Elven.

\textbf{Feat:} \feat{Point Blank Shot}.

\textbf{Weapons:} Dagger (1d4/19--20)

Shortbow with 20 arrows (1d6/$\times$3, 30 m).

\textbf{Armor:} Studded leather (+3 AC).

\textbf{Other Gear:} Standard adventurer's kit, 19 cp.

\subsubsection{The Dervish}
Halfling Scout

\textbf{Ability Scores:} Str 11, Dex 17, Con 12, Int 10, Wis 14, Cha 8.

\textbf{Skills:} \skill{Balance}, \skill{Escape Artist}, \skill{Hide}, \skill{Jump}, \skill{Listen}, \skill{Move Silently}, \skill{Sense Motive}, \skill{Spot}, \skill{Survival}, \skill{Tumble}.

\textbf{Languages:} Halfling.

\textbf{Feat:} \feat{Dodge}.

\textbf{Weapons:} Dagger (1d3/19--20)

Small longspear (1d6/$\times$3)

Five javelins (1d4, 9 m).

\textbf{Armor:} Studded leather (+3 AC).

\textbf{Other Gear:} Standard adventurer's kit, 65 cp.

\subsubsection{The Pathfinder}
Thri-kreen Scout

\textbf{Ability Scores:} Str 14, Dex 19, Con 14, Int 8, Wis 15, Cha 4.

\textbf{Skills:} \skill{Climb}, \skill{Hide}, \skill{Knowledge} (nature), \skill{Knowledge} (warcraft), \skill{Listen}, \skill{Move Silently}, \skill{Search}, \skill{Spot}, \skill{Survival}.

\textbf{Languages:} Kreen.

\textbf{Feat:} \feat{Wastelander}.

\textbf{Weapons:} Gythka (1d8/1d8)

Three chatkchas (1d6, 6 m).

\textbf{Armor:} Studded leather (+3 AC).

\textbf{Other Gear:} Standard adventurer's kit, 5 cp.

\subsection{Scouts on Athas}
\Quote{I need you four to run ahead and tell me what you see ten kilometers in every direction. The winds have told me to be careful early this morning and we don't want to lose any more kanks to raiders. See that you come back in two hours: we need to move.}{Krensh, leader of a nomad tribe}

Scouts are part of any group that needs to move frequently: city-state armies, merchant houses, nomad tribes, even halflings in the Forest Ridge or the Kreen. They are the go-to for reconnaissance all over Athas.

\subsubsection{Daily Life}
Scouts are adventurers because of their trade: their job is to venture ahead of their group, to discover what is at the edge of the horizon and come back to tell their tale. They do this while being hunters, soldiers, or guides.

Their class doesn't define as much of their live as their employers. An army scout has a rigid routine, while a nomad hunter has to adapt to their new environment each day.

\subsubsection{Notables}
The Kreen Empire uses scouts to gather information on the Tablelands, in case any sorcerer-king make a move against the Empire. Some of those scouts stay in the city-states: those are the thor-kreen.

\subsubsection{Organizations}
Scouts are never in an organization of their own, but as part of many greater ones. Joining an organization is part of their trade, as this is the easiest way for most scouts to guarantee their livelihood. Only those who are alone in the wastes by choice have no need for a patron organization.

\subsubsection{NPC Reactions}
Scouts are of little interest for the common people, since they don't have a glorious job as gladiators, nor have any impressive feat to accomplish. But all bards know the value of the information scouts provide and are quite fond of drunk scouts in any tavern. Nomad tribesmen also appreciate information and services of scouts to protect their tribes.

\subsubsection{Scout Lore}
Characters with ranks in \skill{Knowledge} (warcraft) can research scouts to learn more about them. When a character makes a skill check, read or paraphrase the following, including the information from lower DCs.

\textbf{DC 10:} Scouts serve to alert you of dangers ahead. They can spot hidden traps.

\textbf{DC 15:} Scouts are fast and move with ease through the harsh terrains that others find dangerous or impassable.

\textbf{DC 20:} Highly skilled scouts leave no trail and can disappear from sight at a moment's notice.
