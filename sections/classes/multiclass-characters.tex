\section{Multiclass Characters}
A character may add new classes as he or she progresses in level, thus becoming a multiclass character. The class abilities from a character's different classes combine to determine a multiclass character's overall abilities. Multiclassing improves a character's versatility at the expense of focus.

\subsection{Prerequisites}
To qualify for a new class, you must meet the ability score prerequisites for both your current class and your new one, as shown in \tabref{Multiclassing Prerequisites}. For example, a barbarian who decides to multiclass into the druid class must have Wisdom score of 15 or higher, and both Strength and Constitution scores of 13 or higher. Without the full training that a beginning character receives, you must be a quick study in your new class, having a natural aptitude that is reflected by higher-than-average ability scores.

\Table{Multiclassing Prerequisites}{lL}{
\tableheader Class & \tableheader Ability Score Minimum \\
Barbarian       & Strength 13, Constitution 13 \\
Bard            & Dexterity or Intelligence 13, Charisma 13 \\
Cleric          & Wisdom 13, Charisma 13 \\
Druid           & Wisdom 15 \\
Fighter         & Strength or Dexterity 15 \\
Gladiator       & Strength 13, Intelligence or Charisma 13 \\
Psion           & Intelligence 13, Constitution or Wisdom 13 \\
% Psychic Warrior & Strength or Dexterity 13, Wisdom 13 \\
Ranger          & Strength or Dexterity 13, Wisdom 13 \\
Rogue           & Dexterity 13, Intelligence or Charisma 13 \\
Templar         & Charisma 15 \\
% Wilder          & Charisma 15 \\
Wizard          & Intelligence 15 \\
}

\subsection{Class And Level Features}
As a general rule, the abilities of a multiclass character are the sum of the abilities of each of the character's classes.

\textbf{Level:} ``Character level'' is a character's total number of levels. It is used to determine when feats and ability score boosts are gained.

``Class level'' is a character's level in a particular class. For a character whose levels are all in the same class, character level and class level are the same.

\textbf{Hit Points:} A character gains hit points from each class as his or her class level increases, adding the new hit points to the previous total.

\textbf{Base Attack Bonus:} Add the base attack bonuses acquired for each class to get the character's base attack bonus. A resulting value of +6 or higher provides the character with multiple attacks.

\textbf{Saving Throws:} Add the base save bonuses for each class together.

\textbf{Skills:} If a skill is a class skill for any of a multiclass character's classes, then character level determines a skill's maximum rank. (The maximum rank for a class skill is 3 + character level.)

If a skill is not a class skill for any of a multiclass character's classes, the maximum rank for that skill is one-half the maximum for a class skill.

\textbf{Class Features:} A multiclass character gets all the class features of all his or her classes but must also suffer the consequences of the special restrictions of all his or her classes.

In the special case of turning undead, both clerics and experienced templars have the same ability. If the character's templar level is 4th or higher, her effective turning level is her cleric level plus her templar level.

In the special case of uncanny dodge, both experienced barbarians and experienced rogues have the same ability. When a barbarian/rogue would gain uncanny dodge a second time (for her second class), she instead gains improved uncanny dodge, if she does not already have it. Her barbarian and rogue levels stack to determine the rogue level an attacker needs to flank her.

\textbf{Feats:} A multiclass character gains feats based on character levels, regardless of individual class level

\textbf{Ability Increases:} A multiclass character gains ability score increases based on character level, regardless of individual class level.

\textbf{Spells:} The character gains spells from all of his or her spellcasting classes and keeps a separate spell list for each class. If a spell's effect is based on the class level of the caster, the player must keep track of which class's spell list the character is casting the spell from.

\textbf{Power Points:} If you have levels in more than one psionic class, you combine your power points from each class to make up your reserve. You can use these power points to manifest powers from any psionic class you have.

While you maintain a single reserve of power points from your class, race, and feat selections, you are still limited by the manifester level you have achieved with each power you know.