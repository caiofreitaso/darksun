\Class{Fighter}
{Any wastelander can pick up a bone and call it a club, but try pitting fifty of those against one dozen trained soldiers, and maybe you'll have an even match.}{Nikolos, human fighter}

From the small forts in sandy wastes of Athas to the guards of the merchant houses in the city-states, fighters are Athas' most common sight. Whether it is as mercenaries for the sorcerer-kings or as hired guards protecting the wealth of the nobility, fighters can be found everywhere in the Tablelands. Athas' fighters are trained to fight in small groups or huge units. Those that have proven themselves become the commanders in the city-states' armies, commanding hundreds or even thousands of men into war.

\subsection{Making a Fighter}
Fighters receive the best allotment of fighting skills and abilities. They learn the use of most weapons, the best armors and shields, as well as gaining special abilities to use with these weapons and armor.

Some fighters specialize in using a single weapon, and become masters at its use and deadliness. Other fighters will prefer more rounded skills, learning to shoot from far with bows and arrows, or nets or spears. Regardless, the fighter is to be feared.

\textbf{Races:} All of Athas' races can become fighters. Humans are usually the most numerous, though, since they are the most numerous of the races of the Tablelands. Dwarves make good fighters, even though they are smaller than most races; their inborn toughness and great strength more than makes up for their smaller stature. The half-giants are also seen very often as fighters, since their great strength and size are perfect for the job. Muls, with the inherited traits of both humans and dwarves, are also great fighters. Elves, with their long legs and frail constitution, are not often seen as fighters. Athas' intelligent insects, the Thri-kreen, make excellent warriors, with their four arms and the fact they do not need to sleep. Many of the savage races of the Tablelands are fighters, although most become rangers in order to survive.

\textbf{Alignment:} Fighters come from all walks of life, and can be of any alignment. Good fighters are usually seen as the protectors of small villagers or are part of renegade slave tribes, helping their tribe to survive in the harsh desert. Or they can be found as a dwarf perhaps, whose focus it is to guard his fellows. Evil fighters are often part of mercenary bands or under the control of a sorcerer-king; these beings often fight for power and money. Evil fighters can also be found as the rulers of small forts, guarding their oasis and exacting a hefty toll for its use.

\Figure{t}{images/fighter-3.png}

\MiniWarriorTable{The Fighter}{
1st  & +1             & +2  & +0 & +0 & Bonus feat \\
2nd  & +2             & +3  & +0 & +0 & Bonus feat \\
3rd  & +3             & +3  & +1 & +1 &  \\
4th  & +4             & +4  & +1 & +1 & Bonus feat \\
5th  & +5             & +4  & +1 & +1 & Martial prowess \\
6th  & +6/+1          & +5  & +2 & +2 & Bonus feat \\
7th  & +7/+2          & +5  & +2 & +2 & Martial prowess \\
8th  & +8/+3          & +6  & +2 & +2 & Bonus feat \\
9th  & +9/+4          & +6  & +3 & +3 & Martial prowess \\
10th & +10/+5         & +7  & +3 & +3 & Bonus feat \\
11th & +11/+6/+1      & +7  & +3 & +3 & Martial prowess \\
12th & +12/+7/+2      & +8  & +4 & +4 & Bonus feat \\
13th & +13/+8/+3      & +8  & +4 & +4 & Martial prowess \\
14th & +14/+9/+4      & +9  & +4 & +4 & Bonus feat \\
15th & +15/+10/+5     & +9  & +5 & +5 & Martial prowess \\
16th & +16/+11/+6/+1  & +10 & +5 & +5 & Bonus feat \\
17th & +17/+12/+7/+2  & +10 & +5 & +5 & Martial prowess \\
18th & +18/+13/+8/+3  & +11 & +6 & +6 & Bonus feat \\
19th & +19/+14/+9/+4  & +11 & +6 & +6 & Martial prowess \\
20th & +20/+15/+10/+5 & +12 & +6 & +6 & Bonus feat \\
}

\subsection{Game Rule Information}
\textbf{Hit Die:} d10.

\subsubsection{Class Skills}
\skill{Autohypnosis} (Wis), \skill{Climb} (Str), \skill{Craft} (Int), \skill{Handle Animal} (Cha), \skill{Intimidate} (Cha), \skill{Jump} (Str), \skill{Knowledge} (warcraft) (Int), \skill{Ride} (Dex), and \skill{Spot} (Wis).

\textbf{Skill Points per Level:} 4 + Int modifier ($\times4$ at 1st level).

\subsubsection{Class Features}
\textbf{Weapon and Armor Proficiency:} A fighter is proficient with all simple and martial weapons and with all armor (heavy, medium, and light) and shields (including tower shields).

\textbf{Bonus Feats:} At 1st level, a fighter gets a bonus combat-oriented feat in addition to the feat that any 1st-level character gets and the bonus feat granted to a human character. The fighter gains an additional bonus feat at 2nd level and every two fighter levels thereafter (4th, 6th, 8th, 10th, 12th, 14th, 16th, 18th, and 20th). These bonus feats must be drawn from the feats noted as fighter bonus feats. A fighter must still meet all prerequisites for a bonus feat, including ability score and base attack bonus minimums.

These bonus feats are in addition to the feat that a character of any class gets from advancing levels. A fighter is not limited to the list of fighter bonus feats when choosing these feats.

\textbf{Martial Prowess:} At 5th level and every two levels thereafter, a fighter improves his repertoire with new techniques. He may choose one of the following options.

\textit{Active Defense (Ex):} A fighter gains a dodge bonus to AC equal to \onequarter his fighter levels when wielding a shield and fighting defensively or using the \feat{Combat Expertise} feat. When using total defense, this bonus increases to \onehalf his fighter levels.

\textit{Aim (Ex):} A fighter can take full-round action to make a \skill{Spot} check to improve his next attack against a specific foe. DC is equal to his target's AC. His next attack against the target ignore armor bonus and natural armor bonus. This attack must be made within a number of rounds equal to \onequarter his fighter levels.

\Figure*{t}{images/battle-1.png}

\textit{Ambidextrous (Ex):} A fighter may add his full Strength modifier to his off-hand weapon damage, instead of \onehalf his Strength bonus.

\textit{Bravery (Ex):} A fighter with this ability gains a bonus equal to \onehalf his fighter levels on Will saves against mind-affecting abilities.

\textit{Close Combat Shot (Ex):} A fighter can attack with a ranged weapon while in a threatened square and not provoke an attack of opportunity.

\textit{Commanding Strike (Ex):} A fighter may forgo his attack with the lowest base attack bonus in a total attack to create an opening for an ally. The designated ally may use one of his attacks of opportunity to strike a foe. The ally may use ranged attacks---this is an exception for the rules for attacks of opportunity.

\textit{Coordinate Allies (Ex):} A fighter can use a full-round action to identify and apply an effective tactic for his allies. Each creature to be affected must be able to see and hear him, and able to pay attention to him. To coordinate, the fighter must make a \skill{Knowledge} (warcraft) check with a DC equal to 15 + the number of allies affected. If the check succeeds, all affected allies gain a competence bonus on attack rolls or a dodge bonus to AC equal to \onequarter his fighter levels. He chooses which of the two benefits to impart and must impart the same benefit to all affected allies. The benefits last for 1 round. The fighter cannot use this ability on himself.

\textit{Crossbowman (Ex):} Whenever a fighter attacks with a crossbow as a readied action, his target is denied its Dexterity bonus to its AC.

\textit{Defensive Tactics (Ex):} A fighter may use a move action to coordinate his allies to a defensive maneuver. Each creature to be affected must be able to see and hear him, and able to pay attention to him. To coordinate, the fighter must make a \skill{Knowledge} (warcraft) check with a DC equal to 10 + the number of allies affected. If the check succeeds, all affected allies gain a moral bonus to AC against attacks of opportunity equal to \onequarter his fighter levels. The benefits last for 1 round.

\textit{Double Opportunity (Ex):} When a fighter makes an attack of opportunity, he may attack once with both his primary and secondary weapons. The penalties for attacking with two weapons apply normally. The fighter must be at least 11th level to select this technique.

\textit{Flanking Tactics (Ex):} A fighter may use a move action to coordinate his allies to a flanking maneuver against a target. Each creature to be affected must be able to see and hear him, and able to pay attention to him. To coordinate, the fighter must make a \skill{Knowledge} (warcraft) check with a DC equal to the target's Base Attack Bonus. If the check succeeds, all affected allies gain a competence bonus on damage rolls equal to \onequarter his fighter levels when they flank the target. The benefits last for 1 round. The fighter cannot use this ability on himself.

\textit{Harden Resolve (Ex):} A fighter may use a standard action to improve his allies' morale. Each creature to be affected must be able to see and hear him, and able to pay attention to him. To coordinate, the fighter must make a \skill{Knowledge} (warcraft) check with a DC equal to 25 + 2 for each ally affected. If the check succeeds, all affected allies gain damage reduction equal to \onequarter his fighter levels. If an ally already has damage reduction, it improves by the same amount. The benefits last for 1 round. A fighter must be at least 11th level to select this technique.

\textit{Hawkeye (Ex):} Whenever a fighter isn't flat-footed, he has advantage on \skill{Spot} checks.

\textit{Lead the Charge (Ex):} A fighter may make a \skill{Knowledge} (warcraft) check during a charge against one specific foe. The check DC is equal to 20 + the number of allies affected. If the check succeeds, all affected allies that charge the same foe gain a competence bonus on attack and damage rolls equal to \onequarter his fighter levels. The benefits last for 1 round.

\textit{Leadership:} A fighter may gain the \feat{Leadership} feat as a bonus feat.

\textit{Maneuvering Attack (Ex):} A fighter may forgo his attack with the lowest base attack bonus in a total attack to maneuver one of his comrades into a more advantageous position. The fighter chooses a friendly creature that can see and hear him. That creature can use an immediate action to move up to half its speed.

\textit{Martial Versatility (Ex):} A fighter can apply any feat he has that affects only one weapon type (such as, \feat{Weapon Focus} or \feat{Rapid Reload}) to any simple weapon. A fighter must be at least 11th level to select this technique.

\textit{Mirror Move (Ex):} A fighter gains \onequarter his fighter levels as an insight bonus to AC when attacked by any weapon with which he has the \feat{Weapon Specialization} feat.

\textit{Overhand Chop (Ex):} When a fighter makes a single attack (with the attack action or a charge) with a two-handed weapon, he adds triple his Strength bonus on damage rolls, instead of 1\onehalf his Strength bonus.

\textit{Phalanx (Ex):} As long as a fighter is wielding a shield, he may use any two-handed polearm or spear as if it was an one-handed weapon. A fighter cannot use this ability with a buckler.

\textit{Pressing Shield (Ex):} A fighter may use his shield to help with his bull rush and overrun attempts. He adds his shield bonus on Strength checks made to push back the defender, and on Strength checks to knock down his opponent. A fighter cannot use this ability with a buckler.

\textit{Quick Aid (Ex):} A fighter may use the aid another action with a full attack. He must use his attack with the highest base attack bonus and roll against AC 20.

\textit{Ready Polearm (Ex):} A fighter can ready any weapon that deals increased damage against charge as an immediate action. A fighter must be at least 11th level to select this technique.

\textit{Second Wind (Ex):} A fighter with this ability can dig his resolve and endurance to find an extra burst of vitality. He can use a standard action to gain a number of temporary hit points equal to his Constitution modifier $\times$ his fighter levels. These hit points last until for the end of the encounter. He can use this ability only once per encounter.

\textit{Shield Another (Ex):} A fighter can use his shield to protect an adjacent ally. When using aid another to grant AC to an ally, he can choose a number to add as penalty to his shield bonus until his next turn. The ally target of the aid another action gains improves their shield bonus by that number. The fighter can't choose a number greater than his shield bonus or his base attack bonus. A fighter cannot use this ability with a buckler.

\textit{Shorten Haft (Ex):} A fighter may threaten adjacent squares when wielding reach weapons.

\textit{Skirmisher (Ex):} Whenever a fighter moves 3 or more meters in his turn, he gains a dodge bonus against ranged attacks equal to \onequarter his fighter.

\textit{Tortoise Style (Ex):} While wielding a shield, the fighter improves his shield bonus by \onequarter his fighter levels. A fighter cannot use this ability with a buckler.

\textit{Two-handed Style (Ex):} When the fighter rolls a 1 or 2 on a damage die for an attack made with a melee weapon wielded with two hands, he can reroll the die and must use the new roll even if the new roll is a 1 or 2. The fighter cannot use this benefit with light weapons.

\textit{Uncanny Dodge (Ex):} A fighter can react to danger before his senses would normally allow his to do so. He retains his Dexterity bonus to AC (if any) even if he is caught flat-footed or struck by an invisible attacker. However, he still loses his Dexterity bonus to AC if immobilized. If a fighter already has uncanny dodge from a different class he automatically gains improved uncanny dodge instead.

\textit{Weapon Training (Ex):} A fighter improves his aptitude in martial weapons. He gains a competence bonus on his attack rolls equal to \onequarter his fighter levels, whenever he is wielding a simple or martial weapon.


% \subsection{Playing a Fighter}
% Playing an Athasian fighter is not much different than playing one in other settings, the only difference is that the extreme heat makes most armor less than desirable on Athas.

% As a fighter, you undertake adventures according to the dictates of your cause, your faith, or your own selfish needs. You might find yourself on the hot, sandy field of battle, charging shoulder to shoulder with peasants and soldiers, raising pitchforks and shields against the defilers of the enemy army.

% \subsubsection{Religion}
% There are no gods on Athas, but many fighters worship the sorcerer-king of their respective cities as gods. Some fighters pay homage to the elemental forces of the Tablelands, asking their favored element for luck before entering the battlefield.

% \subsubsection{Other Classes}
% Fighters get along with most other classes. The rangers of the Tablelands often receive the highest of the respect for their ability to survive the wastes. Gladiators and fighters are often at each other's throats, since both share great combat abilities but differ in their methodology; they often try to show how each is better than the other is. Elemental clerics are welcome for their healing abilities as well as the help they can provide in battle.

% Fighters are uneasy around wizards; like the rest of the population they distrust magic. Templars are also distrusted, for the same reasons everyone else distrusts templars. Rogues are usually scorned by fighters; they prefer open battle to the rogue's sneaky ways.

% \subsubsection{Combat}
% Your specific tactics in battle depend on your role in the party and your weapon of choice. However, certain tactics are common to all fighters.

% You are generally at the forefront of any battle. Fighting on the front line allows you maximize your combat feats. Furthermore, if opponents focus on you, they cannot injure your allies. As a fighter, you're at your best when you can take on the monster or opponent that deals the most damage.

% \subsubsection{Advancement}
% When looking at feats to select as you gain levels, you have two basic paths. You can focus on your fighting skills, or you can attempt to expand your capabilities to serve as the party's leader. The former option is best when you are the group's primary combat specialist. If the party includes another fighter or suitable melee character, you can afford to dabble in Charisma-based skills. Although Diplomacy is not a class skill for you, the Field Office feat combined with a few cross-class ranks makes you a serviceable emissary.

% When it comes to combat feats, look to ones that improve your ability to deal damage. Feats such as Power Attack, Weapon Focus, and so forth are excellent options to boost your offense. Concentrated Fire, Rotate Lines, Shield Wall, and Spear Wall are excellent feats for army fighters.

% Improved Initiative is a critically important feat, since it allows you to act first, move forward and defend or guide your allies. The sooner you find a place in the front line, the longer you can hold back the enemy.


% \subsection{Starting Packages}
% \subsubsection{The Archer}
% Elf Fighter

% \textbf{Ability Scores:} Str 15, Dex 16, Con 11, Int 10, Wis 12, Cha 8.

% \textbf{Skills:} \skill{Jump}, \skill{Spot} (cc).

% \textbf{Languages:} Common, Elven.

% \textbf{Feat:} \feat{Point Blank Shot}, \feat{Precise Shot}.

% \textbf{Weapons:} Macahuitl (1d8/19-20)

% Dagger (1d4/19-20, 3 m)

% Longbow with 40 arrows (1d8/$\times$3, 30 m).

% \textbf{Armor:} Chitin armor (+4 AC).

% \textbf{Gear:} Standard adventurer's kit, 11 cp.

% \subsubsection{The Defender}
% Dwarf Fighter

% \textbf{Ability Scores:} Str 15, Dex 13, Con 16, Int 10, Wis 12, Cha 8.

% \textbf{Skills:} \skill{Craft} (weaponsmithing), \skill{Knowledge} (warcraft), \skill{Intimidate}.

% \textbf{Languages:} Common, Dwarven.

% \textbf{Feat:} \feat{Disciplined}, \feat{Weapon Focus} (dwarven waraxe).

% \textbf{Weapons:} Dwarven waraxe (1d10/$\times$3)

% Shortbow with 20 arrows (1d6/$\times$3, 18 m).

% \textbf{Armor:} Scale mail (+4 AC), heavy wooden shield (+2 AC).

% \textbf{Gear:} Standard adventurer's kit, 42 cp.

% \subsubsection{The Leader}
% Human Fighter

% \textbf{Ability Scores:} Str 15, Dex 8, Con 13, Int 10, Wis 12, Cha 14.

% \textbf{Skills:} \skill{Diplomacy} (cc), \skill{Knowledge} (warcraft), \skill{Intimidate}.

% \textbf{Languages:} Common.

% \textbf{Feat:} \feat{Field Officer}, \feat{Inspiring Presence}, \feat{Weapon Focus} (great macahuitl).

% \textbf{Weapons:} Great macahuitl (2d6, 19-20)

% Shortbow with 20 arrows (1d6/$\times$3, 18).

% \textbf{Armor:} Scale mail (+4 AC).

% \textbf{Gear:} Standard adventurer's kit, 19 cp.


\subsection{Fighters on Athas}
\Quote[-.em]{Yeah, he was alright with a sword, but he would wet himself every time we walked out onto the sand of the arena. I think it was the crowd... or the goat-headed giant they paired us against. Poor little weed, he never saw that club coming.}{Grek the Grand, talking about his onetime matched pair contest with Slavek Thydor}

Fighters bring clashing weapons, stirring speeches, and mass combat to the campaign. On Athas, the fighter is a trained warrior, a soldier skilled in mass warfare. Every society on Athas maintains an army of fighters to protect itself from attack or to wage wars of plunder and annihilation against its neighbors. Fighters are both the commanders and soldiers in these armies, and at higher levels are experts in both individual and formation combat, leadership, and morale.

\subsubsection{Daily Life}
A fighter adventures to prove his superior skill at arms, to advance the cause of whatever master he might serve, and to further his own aims.

Once he has reached a respectable level of accomplishment, a fighter might take the Leadership feat and start building his own army. As word spreads, less experienced warriors who are eager to fight for the same causes seek him out as the desperate peoples of Athas constantly look for great commanders, warriors who will lead them.

\subsubsection{Notables}
Fighters can notoriety for their deeds, whether triumphs in combat, selfless acts of great honor, or great tyranny. Many an adventurer grew up on stories such as that of the Crimson Legion, and how it managed to defeat Urik's previously undefeated army.

Another legend tells of about the rise and fall of General Zanthiros, the leader of the Balican army who managed to save the city from an onslaught of beast-headed giants more than once, and after losing the elections, left the city with hundreds of soldiers loyal to him and formed a raiding tribe.

\subsubsection{Organizations}
Fighters often band together into small armies or as mercenary groups working for trade houses. These organizations typically have different credos and values, but they allow their members to focus their time on their individual quests.

\subsubsection{NPC Reactions}
Individuals react to fighters based on their previous interactions with other members of the class. A brave fighter meets cold silence and contempt around the Barrier Wastes where evil fighters oppress the populace.

Gladiators usually talk down on fighters, saying that gladiators are the true masters of combat. Fighters usually reply that gladiators are nothing without a crowd looking. Because of that, their initial reaction is one step towards unfriendly than normal.

A fighter who has lived long enough to retire from adventuring typically acquires some position of authority, with commensurate political power, whether as a caravan leader, army general, or ruler of a raiding or slave tribe.

% \subsubsection{Fighter Lore}
% Characters with ranks in \skill{Knowledge} (warcraft) can research fighters to learn more about them. When a character makes a skill check, read or paraphrase the following, including the information from lower DCs.

% \textbf{DC 10:} Fighters may not be as flashy as gladiators in combat, but they surely are more effective in mass combat.

% \textbf{DC 15:} Fighters are combat-oriented characters adept at hand-on-hand combat just as well as commanding entire armies.

% \textbf{DC 20:} A fighter's mere presence in the battlefield can be enough to inspire his soldiers and weaken the resolve of his enemies.
