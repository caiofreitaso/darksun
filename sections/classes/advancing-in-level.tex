\section{Character Progress}

\subsection{Advancing in Level}
DMs use adventures to offer a number of challenges (or encounters) for the characters to overcome. As the characters overcome these encounters, they should improve and gain new abilities to overcome greater challenges. The rate of encounters is called the pace of the adventure, or the pace of the campaign. \tabref{Paces of Campaign} shows how many encounters or sessions are expected for each type of campaign.

\Table{Paces of Campaign}{X R R}{
\tableheader Pace of Campaign & \tableheader Encounters per Level & \tableheader Sessions per Level\\
Rushed & 5 encounters & 1 session \\
Fast & 10 encounters & 2 sessions \\
Default & 12 encounters & 3 sessions \\
Slow & 15 encounters & 4 sessions \\
Dragged & 20 encounters & 5+ sessions \\
}

Instead of the standard of using Experience Points (XP) to attain new character levels, we recommend that players gain levels based on a fixed number of sessions. This way the DM can tailor the experience based on the frequency of the sessions, e.g., if your group can only play once per month it may be better to speed up the progress of the characters.

If you would rather use a traditional leveling system via XP, you can use \tabref{Traditional Experience and Wealth}. In this table you can also see the expected wealth of player characters for any given level.

\Table{Traditional Experience and Wealth}{X X R}{
\tableheader ECL & \tableheader XP for next level & \tableheader Wealth\\
1 & 1,500 & 3d4 $\times$ 10 cp\\
2 & 3,000 & 600 cp\\
3 & 4,500 & 1,500 cp\\
4 & 6,000 & 3,000 cp\\
5 & 7,500 & 6,000 cp\\
6 & 9,000 & 9,000 cp\\
7 & 10,500 & 12,000 cp\\
8 & 12,000 & 18,000 cp\\
9 & 13,500 & 24,000 cp\\
10 & 15,000 & 30,000 cp\\
11 & 16,500 & 42,000 cp\\
12 & 18,000 & 54,000 cp\\
13 & 19,500 & 66,000 cp\\
14 & 21,000 & 90,000 cp\\
15 & 22,500 & 114,000 cp\\
16 & 24,000 & 138,000 cp\\
17 & 25,500 & 186,000 cp\\
18 & 27,000 & 234,000 cp\\
19 & 28,500 & 282,000 cp\\
20 & --- & 370,000 cp\\
}

\subsubsection{Level-Dependent Benefits}
Besides the bonuses given by their character classes, all characters gain additional feats and increase their abilities from advancing in level as seen in \tabref{Level-Dependent Benefits}. This does not count a race's level adjustment---only character classes.

\textbf{Feat:} Every character gets their first feat at 1st level. At 3rd level and every other 3 levels, characters gain another feat. More information on feats on \chapref{Feats}.

\textbf{Ability Score Increase:} At 4th level and every 4 levels thereafter, the player chooses one of their character's ability scores to increase by 1 point. This improvement is permanent.

\Table{Level-Dependent Benefits}{X C C}{
\tableheader Character Level & \tableheader Feat & \tableheader Ability Score Increase\\
1 & 1st & \\
2 & & \\
3 & 2nd & \\
4 & & 1st \\
5 & & \\
6 & 3rd & \\
7 & & \\
8 & & 2nd \\
9 & 4th & \\
10 & & \\
11 & & \\
12 & 5th & 3rd \\
13 & & \\
14 & & \\
15 & 6th & \\
16 & & 4th \\
17 & & \\
18 & 7th & \\
19 & & \\
20 & & 5th \\
}

\subsection{Gaining Experience}
Award players for combat and role-playing. These are some guidelines to award experience points players for their roleplay. Awards are only for successful attempts of each action.

\subsubsection{Awards per Encounter}
Each pace of campaign has a different award---faster campaigns have bigger rewards. To determine the XP award for an encounter, follow these steps.
\begin{enumerate*}
	\item Determine each character's effective character level (ECL).
	\item For each monster or trap defeated, determine its Challenge Rating (CR).
	\item Use \tabref{Experience Awards from Encounters} to cross-reference the campaign's pace with the CR of each monster or trap to find the base XP award.
	\item Divide the base XP award by the number of characters in the party.
	\item Add up all the XP awards for all the monsters or traps the character helped defeat.
	\item Repeat the process for each character.
\end{enumerate*}

Creatures summoned or that otherwise are related to an enemy's ability (such as animal companion) do not award XP. These abilities are already taken into account in the enemy's CR.

\Table{Experience Awards from Encounters}{X *{5}{C}} {
\rowcolor{white}
\multirow[l]{2}{1cm}{\tableheader Challenge Rating} & \multicolumn{5}{c}{\tableheader Experience Points Award (per monster)}\\
\cmidrule[0.5pt]{2-6}
& \tableheader Rushed & \tableheader Fast & \tableheader Normal & \tableheader Slow & \tableheader Dragged \\

1 & 1,200 & 600 & 500 & 400 & 300 \\
2 & 2,400 & 1,200 & 1,000 & 800 & 600 \\
3 & 3,600 & 1,800 & 1,500 & 1,200 & 900 \\
4 & 4,800 & 2,400 & 2,000 & 1,600 & 1,200 \\
5 & 6,000 & 3,000 & 2,500 & 2,000 & 1,500 \\
6 & 7,200 & 3,600 & 3,000 & 2,400 & 1,800 \\
7 & 8,400 & 4,200 & 3,500 & 2,800 & 2,100 \\
8 & 9,600 & 4,800 & 4,000 & 3,200 & 2,400 \\
9 & 10,800 & 5,400 & 4,500 & 3,600 & 2,700 \\
10 & 12,000 & 6,000 & 5,000 & 4,000 & 3,000 \\
11 & 13,200 & 6,600 & 5,500 & 4,400 & 3,300 \\
12 & 14,400 & 7,200 & 6,000 & 4,800 & 3,600 \\
13 & 15,600 & 7,800 & 6,500 & 5,200 & 3,900 \\
14 & 16,800 & 8,400 & 7,000 & 5,600 & 4,200 \\
15 & 18,000 & 9,000 & 7,500 & 6,000 & 4,500 \\
16 & 19,200 & 9,600 & 8,000 & 6,400 & 4,800 \\
17 & 20,400 & 10,200 & 8,500 & 6,800 & 5,100 \\
18 & 21,600 & 10,800 & 9,000 & 7,200 & 5,400 \\
19 & 22,800 & 11,400 & 9,500 & 7,600 & 5,700 \\
20 & 24,000 & 12,000 & 10,000 & 8,000 & 6,000 \\
}

\subsubsection{Awards per Race}
These awards are for roleplaying some of the stereotypical aspects of athasian character races. Players should remember that races are more than just stats on their character sheets, they have culture and history which are what these stereotypes try to enforce. The judgement of good roleplaying ultimately lies with the DM, and they must be familiar with the nuances of the character races. The communication between the DM and the players should be clear so that a good roleplaying experience can arise, and the nature of \textbf{Dark Sun} can be emphasized.

If you are using experience points to advance levels, multiply these awards by the character's effective character level (ECL). Remember that this will quicken the rate of progress of the characters.

\textbf{Aarakocra}:
\XPTable{Aarakocras}{

}

\textbf{Dwarf}: Dwarves live by their foci. A focus must take at least a week to complete. If a focus takes a least a year to complete, it becomes a major focus. Focus can be changed in very rare circumstances. These circumstances must be agreed between the player and the DM.

\XPTable{Dwarves}{
Pursue present focus & 5 XP $\times$ days pursuing\\
Ignore present focus & -50 XP $\times$ days ignoring\\
Complete major focus & 500 XP\\
}

\textbf{Elf}: Roleplaying an elf is centered around trust. Elves are self-reliant and do not want to gain friendship with every character they meet. They test redeemable outsiders (in the elvish perspective) to see if they are trustworthy.

\textit{Examples of subtle tests of trust}:
\begin{itemize*}
	\item entrust with confidential information,
	\item leave a valuable item easy for taking to see the outsider takes it,
	\item ask to deliver a message or item.
\end{itemize*}

\textit{Examples of life-threatening tests of trust}:
\begin{itemize*}
	\item let themselves get captured to see if there is a rescue attempt,
	\item fake unconsciousness after a battle to see what care is provided,
	\item cut supplies to see if they get a fair share.
\end{itemize*}

\XPTable{Elves}{
Subtle test of trust & 5 XP\\
Life-threatening test of trust & 50 XP\\
Refuse animal or magical transport & 10 XP\\
Continuous run & 5 XP $\times$ distance in km\\
}

\textbf{Half-Elf}:
\XPTable{Half-Elves}{
Observe human or elven custom & 5 XP\\
Better a human or elf in custom & 25 XP\\
}

\textbf{Half-Giant}:
\XPTable{Half-Giants}{
Imitate charismatic friend & 5 XP $\times$ days imitating\\
Shift alignment per influence & 5 XP\\
}

\textbf{Halfling}:
\XPTable{Halflings}{
Practice another race's custom & 5 XP\\
Aid another halfling & 10 XP\\
}

\XPTable{Muls}{
Heavy exertion & 10 XP $\times$ day of work\\
}

\textbf{Pterran}:
\XPTable{Pterrans}{

}

\textbf{Thri-Kreen}:
\XPTable{Thri-Kreens}{
Defeat creature for food & 5 XP\\
Paralyze creature & 10 XP\\
}

% \subsubsection{Awards per Class}
% Multiclass characters must choose which class to consider when receiving awards---you can only gain award for a single class.

% \textbf{\class{Barbarian}}: Barbarians are survivalists, so beyond defeating living creatures, defeating traps and natural hazards on their might alone award them bonus experience points.

% \XPTable{Barbarians}{
% Use special attack & 5 XP\\
% Use rage & 10 XP\\
% Defeat a creature & 10 XP $\times$ creature's CR\\
% Defeat trap or natural hazard & 10 XP $\times$ trap's CR\\
% }

% \textbf{\class{Bard}}: Bards gain bonus experience points for successful use of their bardic abilities. However they also gain XP for using poison against a creature---to weaken or kill the victim.

% \XPTable{Bards}{
% Use bardic music & 10 XP\\
% Use bardic knowledge & 25 XP\\
% Use poison effectively & 5 XP $\times$ creature's CR\\
% Defeat a creature & 5 XP $\times$ creature's CR\\
% Obtain treasure & 5 XP $\times$ value in cp\\
% }

% \textbf{\class{Cleric}}: Using elements with finesse and flair to overcome an obstacle should reaward clerics bonus experience points.

% \XPTable{Clerics}{
% Use domain power & 10 XP\\
% Use element creatively & 50 XP\\
% Cast spell & 5 XP $\times$ spell level\\
% Cast Healing spell & 10 XP $\times$ spell level\\
% Turn/rebuke undead & 25 XP $\times$ undead's CR\\
% Destroy/command undead & 50 XP $\times$ undead's CR\\
% Create magic item & 200 XP\\
% }

% \textbf{\class{Druid}}:

% \XPTable{Druids}{
% Cast spell & 5 XP $\times$ spell level\\
% Case Healing spell & 10 XP $\times$ spell level\\
% Use wild empathy & 25 XP\\
% Use wild shape & 10 XP\\
% Defeat defiler & 50 XP $\times$ defiler's CR\\
% Create magic item & 200 XP\\
% }

% \textbf{\class{Fighter}}: The fighter's role in society is about mass warfare, so being a good soldier during these times will award her additional experience points. Fighters do not gain experience points for spending weeks in reserve, even if they're commanding followers.

% \XPTable{Fighters}{
% Use special attack & 5 XP\\
% Defeat a creature alone & 5 XP $\times$ creature's CR\\
% Defeat a creature with a group & 10 XP $\times$ creature's CR\\
% Follow commands in battle & 25 XP\\
% Command a battle & 50 XP\\
% Build a war machine & 100 XP\\
% }

% \textbf{\class{Gladiator}}: Gladiators desire to be in the spotlight of an arena, to be the victor of a duel. Therefore, they receive additional experience points for defeating creatures in an arena without outside aid. The glory must be their alone.

% \XPTable{Gladiators}{
% Use special attack & 5 XP\\
% Use gladiatorial perfomance & 10 XP\\
% Defeat a creature & 5 XP $\times$ creature's CR\\
% Defeat a creature alone in an arena & 10 XP $\times$ creature's CR\\
% }

% \textbf{\class{Psion}}:

% \XPTable{Psions}{
% Defeat a psionic creature & 5 XP $\times$ creature's CR\\
% Research new psionic knowledge & 50 XP\\
% Manifest a power & 5 XP $\times$ power level\\
% Manifest a power to avoid combat & 10 XP $\times$ power level\\
% Create psionic item & 200 XP\\
% }

% \textbf{\class{Psychic Warrior}}:

% \XPTable{Psychic Warriors}{
% Defeat a creature & 5 XP $\times$ creature's CR\\
% Defeat a psionic creature & 5 XP $\times$ creature's CR\\
% Manifest a power & 5 XP $\times$ power level\\
% }

% \textbf{\class{Ranger}}: Rangers track their foes and hunt favored enemies. Accomplishing these taks award them additional experience points.

% \XPTable{Ranger}{
% Cast spell & 5 XP $\times$ spell level\\
% Defeat a creature & 5 XP $\times$ creature's CR\\
% Defeat a creature in a favorite territory & 5 XP $\times$ creature's CR\\
% Defeat a favorite enemy & 10 XP $\times$ enemy's CR\\
% Track a creature & 10 XP\\
% Track a creature in a favorite territory & 10 XP\\
% Track a favorite enemy & 10 XP\\
% Use wild empathy & 25 XP\\
% }

% \textbf{\class{Rogue}}:

% \XPTable{Rogue}{
% Defeat a creature & 5 XP $\times$ creature's CR\\
% Disable trap & 25 XP $\times$ trap's CR\\
% Obtain treasure & 5 XP $\times$ value in cp\\
% Obtain treasure for patron & 5 XP $\times$ value in cp\\
% Use sneak attack & 5 XP\\
% }

% \textbf{\class{Templar}}:

% \XPTable{Templar}{
% Cast spell & 5 XP $\times$ spell level\\
% Use secular authority on slave & 5 XP\\
% Use secular authority on freeman & 10 XP\\
% Use secular authority on noble & 25 XP\\
% Use secular authority on templar & 50 XP\\
% Fulfill sorcerer-king's mission & 100 XP\\
% }

% \textbf{\class{Wilder}}:

% \XPTable{Wilders}{
% Defeat a psionic creature & 5 XP $\times$ creature's CR\\
% Create psionic item & 200 XP\\
% Surge without enervation & 10 XP $\times$ bonus level\\
% Manifest a power & 5 XP $\times$ power level\\
% Manifest a power to avoid combat & 10 XP $\times$ power level\\
% }

% \textbf{\class{Wizard}}: Preservers and defilers serve different roles in the athasian society. Preservers want to remain hidden from the eyes of the templars, so they gain additional experience points for keeping spellcasting secret. Defilers on the other hand are aligned with the \emph{status quo} and are awarded additional experience points for carrying out the business of their sorcerer-monarchs.

% \XPTable{Wizards}{
% Cast spell & 5 XP $\times$ spell level\\
% Cast spell for sorcerer-king (defiler) & 5 XP $\times$ spell level\\
% Find new spell to spellbook & 50 XP $\times$ spell level\\
% Keep spellcasting secret (preserver) & 25 XP\\
% Create magic item & 200 XP\\
% }


\subsubsection{Awards per Skill Check}