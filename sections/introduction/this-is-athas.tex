\section{This is Athas}
Athas' savage, primal landscape is the result of long centuries of ecological and magical abuses. The world is dying. It breathes its last gasps as water turns to silt, grasslands become sandy wastes, and jungles decay into stony barrens. Still, life finds ways to endure even in these hellish conditions. In fact, it thrives.

Children growing up beneath the crimson sun don't aspire to become heroes. True heroes who champion causes or seek to make the world a better place are as rare as steel on Athas. Living to see the next dawn is more important than defending a set of beliefs, so survival ultimately motivates all living creatures---not virtue or righteousness.

But heroes are desperately needed in this harsh, savage world... Heroes like the ones who stepped forward to destroy the sorcerer-king Kalak and set Tyr free. Heroes like those who risked everything to kill the Dragon and keep Rajaat the Warbringer from devastating the land.

Today, Athas rushes toward its future. If the course of destruction is to be diverted, of Athas is to be restored, then more heroes must grab the reins of destiny and give new hope and promise to the world.

\subsection{Ten Things You Need to Know}
Every Dungeon Master and player needs to know and remember these facts about the world of Athas.

\textbf{Dark Sun is Different from Traditional D\&D.} Many monsters, prestige classes, spells or magic items from the core rulebooks simply are not available in Athas. Many races were extinguished from Athas during the Cleansing Wars. This is because Athas has a very different background than most D\&D settings. Check with your DM to see which options you have to choose from before building your character.

\textbf{Tone and Attitude:} Athas puts the survival of the fittest concept to its fullest. Those who cannot adapt to endure the tyrannical sorcerer-kings, the unrelenting sun, or the many dangers of the wastes will certainly perish. Illiteracy and slavery are commonplace, while magic is feared and hated. The term ``hero'' has a very different meaning on Athas.

\textbf{A Burnt World:} Thousands of years of reckless spellcasting and epic wars have turned Athas into a barren world, on the verge of an ecological collapse. From the first moments of dawn until the last twinkling of dusk, the crimson sun shimmers in the olive-tinged sky like a fiery puddle of blood, creating temperatures up to 65$^\circ$ C (150$^\circ$ F) by late afternoon. Waters is scarce, so most Athasians need to come up with alternative solutions for dealing with the heat or perish.

\textbf{A World Without Metal:} Metals are very rare on Athas. Its scarcity has forced Athasians to rely on barter and different materials, such as ceramic, to use as currency. It also hampers industrial and economic development as well; mills and workshops rarely have quality tools to produce everyday products. Even though most Athasians have developed ways of creating weapons and armor made of nonmetallic components, but the advantage of having metal equipment in battle is huge.

\textbf{The Will and The Way:} From the lowliest slave to the most powerful sorcerer-king, psionics pervade all levels of Athasian society. Virtually every individual has some mental ability, and every city-state has some sort of psionic academy available. Athasians use the term Will to refer to someone's innate ability for psionics and the Way for the study of psionics.

\textbf{A World Without Gods:} Athas is a world without true deities. Powerful sorcerer-kings often masquerade as gods but, though their powers are great and their worshipers many, they are not true gods. Arcane magic require life force, either from plants or animals, to be used. All divine power comes from the Elemental planes and the spirits of the land that inhabit geographic features.

\textbf{Planar Insulation:} Barriers exist between Athas and other planes. In the case of other planes of existence, the Gray impedes planar travel, except to the Elemental Planes. Consequently, travel via spelljamming is impossible, and planar travel is much more difficult. The same holds true for those trying to contact or reach Athas. The barrier formed by the Gray impedes travel in both directions.

\textbf{The Struggle For Survival:} The basic necessities of life are scarce on Athas. This means that every society must devote itself to attaining food and safeguarding its water supply, while protecting themselves from raiding tribes, Tyr-storms, and other city-states. This essentially means that most Athasian must devout a large deal of their lives just to survive.

\textbf{The Seven City-States:} The Tyr Region is the center of the world of Athas, at least as far as the people of the seven city-states are concerned. It's here, along the shores of the Silt Sea and in the shadows of the Ringing Mountains that civilization clings to a few scattered areas of fertile land and fresh water. The majority of the population lives in the city-states of Tyr, Urik, Raam, Draj, Nibenay, Gulg, and Balic. The remainder lives in remote villages built around oases and wells, or wanders about in nomadic tribes searching for what they need to survive.

\textbf{New and Changed Races:} The common races found in D\&D changed with the harsh planet, elves are tireless nomads, dwarves are bald and have a focused mind, half-elves struggle to fit in the society, and halflings are cannibals living in the jungles. In addition to those races, players can choose to play half-giants, muls, and thri-kreen. Half-giants are creatures with great strength, but dull wits. Muls are a hybrid race that combines the natural dwarven resilience and stubbornness with the adaptability from humans. Thri-kreen are insectoid creatures that roam the Athasian wastes in search for prey.

\Figure*{b}{images/fort-1.png}
\subsection{The Five Ages of Play}
{\tableheader Dark Sun} 3.5 supports adventures and campaigns set in many different ages, five of which are detailed in this book. You can set your campaign right after the events of the Prism Pentad. Known as the Age of Heroes, this is a period that fundamentally changed the world, when individuals begun fighting back all the tyranny and oppression, ending up with several sorcerer-kings dead and the first free city of the Tablelands appeared.

Or, you can go backward in time to the classic period where most sorcerer-kings were still alive and play during the Dragon Age or the Age of Sorcerer-Kings, when the world was becoming more and more a wasteland by defiling magic, and the Dragon of Tyr was almighty.

Or, you can go even more backward in time and play during the Blood Age, when Rajaat unleashed his human armies and his Champions in order to wipe out all other intelligent races from the face of Athas.

Or, you can go to Green Age, when the New Races began populating the lands left unscathed by the receding waves, and the first great cities were found, and psionics started to show its true power.

Or, you can go to the very first age, known as the Blue Age, when the world was still young and the only intelligent races where the rhulisti, the ancient halflings, and the kreen, lived in a world filled with oceans and a blue sun, and magic was nonexistent.

In addition, the rules set in this book can be used to support campaigns set in other ages. For example, you could forward to several hundred years into the future, in a world that could be either devastated by the Kreen invasion, or that has just begun to heal from most of the damage it suffered since Rajaat discovered arcane magic. Although these ages are not covered in this book, the rules herein can be used as a basis for play in them.
