\section{The Core Mechanic}
Whenever you attempt an action that has some chance of failure, you roll a twenty-sided die (d20). To determine if your character succeeds at a task you do this:

\begin{itemize*}
\item Roll a d20.
\item Add any relevant modifiers.
\item Compare the result to a target number.
\item If the result equals or exceeds the target number, your character succeeds. If the result is lower than the target number, you fail.
\end{itemize*}

\subsection{Dice}
Dice rolls are described with expressions such as ``3d4+3,'' which means ``roll three four-sided dice and add 3'' (resulting in a number between 6 and 15). The first number tells you how many dice to roll (adding the results together). The number immediately after the ``d'' tells you the type of die to use. Any number after that indicates a quantity that is added or subtracted from the result.

\textbf{d\%:} Percentile dice work a little differently. You generate a number between 1 and 100 by rolling two different ten-sided dice. One (designated before you roll) is the tens digit. The other is the ones digit. Two 0s represent 100.

\subsection{Modifiers}
A modifier is any bonus or penalty applying to a die roll. A positive modifier is a bonus, and a negative modifier is a penalty.



\subsubsection{Stacking}
In most cases, modifiers to a given check or roll stack (combine for a cumulative effect) if they come from different sources and have different types (or no type at all), but do not stack if they have the same type or come from the same source (such as the same spell cast twice in succession). If the modifiers to a particular roll do not stack, only the best bonus and worst penalty applies. Dodge bonuses and circumstance bonuses however, do stack with one another unless otherwise specified.



\Figure{t}{images/tower-1.png}



\subsubsection{Modifier Types}
\textbf{Ability Modifier:} The bonus or penalty associated with a particular ability score. Ability modifiers apply to die rolls for character actions involving the corresponding abilities.

\textbf{Alchemical Bonus:} An alchemical bonus is granted by the use of a nonmagical, alchemical substance such as antitoxin.

\textbf{Armor Bonus:} An armor bonus applies to Armor Class and is granted by armor or by a spell or magical effect that mimics armor. Armor bonuses stack with all other bonuses to Armor Class (even with natural armor bonuses) except other armor bonuses. An armor bonus doesn't apply against touch attacks, except for armor bonuses granted by force effects (such as the mage armor spell) which apply against incorporeal touch attacks, such as that of a shadow.

\textbf{Circumstance Modifier:} A circumstance bonus (or penalty) arises from specific conditional factors impacting the success of the task at hand. Circumstance bonuses stack with all other bonuses, including other circumstance bonuses, unless they arise from essentially the same source.

\textbf{Competence Modifier:} A competence bonus (or penalty) affects a character's performance of a particular task, as in the case of the bardic ability to inspire competence. Such a bonus may apply on attack rolls, saving throws, skill checks, caster level checks, or any other checks to which a bonus relating to level or skill ranks would normally apply. It does not apply on ability checks, damage rolls, initiative checks, or other rolls that aren't related to a character's level or skill ranks. Multiple competence bonuses don't stack; only the highest bonus applies.

\textbf{Deflection Bonus:} A deflection bonus affects Armor Class and is granted by a spell or magic effect that makes attacks veer off harmlessly. Deflection bonuses stack with all other bonuses to AC except other deflection bonuses. A deflection bonus applies against touch attacks.

\textbf{Dodge Bonus:} A dodge bonus improves Armor Class (and sometimes Reflex saves) resulting from physical skill at avoiding blows and other ill effects. Dodge bonuses are never granted by spells or magic items. Any situation or effect (except wearing armor) that negates a character's Dexterity bonus also negates any dodge bonuses the character may have. Dodge bonuses stack with all other bonuses to AC, even other dodge bonuses. Dodge bonuses apply against touch attacks.

\textbf{Enhancement Bonus:} An enhancement bonus represents an increase in the sturdiness and/or effectiveness of armor or natural armor, or the effectiveness of a weapon, or a general bonus to an ability score. Multiple enhancement bonuses on the same object (in the case of armor and weapons), creature (in the case of natural armor), or ability score do not stack. Only the highest enhancement bonus applies. Since enhancement bonuses to armor or natural armor effectively increase the armor or natural armor's bonus to AC, they don't apply against touch attacks.

\textbf{Insight Bonus:} An insight bonus improves performance of a given activity by granting the character an almost precognitive knowledge of what might occur. Multiple insight bonuses on the same character or object do not stack. Only the highest insight bonus applies.

\textbf{Luck Modifier:} A luck modifier represents good (or bad) fortune. Multiple luck bonuses on the same character or object do not stack. Only the highest luck bonus applies.

\textbf{Morale Modifier:} A morale bonus represents the effects of greater hope, courage, and determination (or hopelessness, cowardice, and despair in the case of a morale penalty). Multiple morale bonuses on the same character do not stack. Only the highest morale bonus applies. Nonintelligent creatures (creatures with an Intelligence of 0 or no Intelligence at all) cannot benefit from morale bonuses.

\textbf{Natural Armor Bonus:} A natural armor bonus improves Armor Class resulting from a creature's naturally tough hide. Natural armor bonuses stack with all other bonuses to Armor Class (even with armor bonuses) except other natural armor bonuses. Some magical effects (such as the barkskin spell) grant an enhancement bonus to the creature's existing natural armor bonus, which has the effect of increasing the natural armor's overall bonus to Armor Class. A natural armor bonus doesn't apply against touch attacks.

\textbf{Profane Modifier:} A profane bonus (or penalty) stems from the power of evil. Multiple profane bonuses on the same character or object do not stack. Only the highest profane bonus applies.

\textbf{Racial bonus:} A bonus granted because of the culture a particular creature was brought up in or because of innate characteristics of that type of creature. If a creature's race changes (for instance, if it dies and is reincarnated), it loses all racial bonuses it had in its previous form.

\textbf{Resistance Bonus:} A resistance bonus affects saving throws, providing extra protection against harm. Multiple resistance bonuses on the same character or object do not stack. Only the highest resistance bonus applies.

\textbf{Sacred Modifier:} A sacred bonus (or penalty) stems from the power of good. Multiple sacred bonuses on the same character or object do not stack. Only the highest sacred bonus applies.

\textbf{Shield Bonus:} A shield bonus improves Armor Class and is granted by a shield or by a spell or magic effect that mimics a shield. Shield bonuses stack with all other bonuses to AC except other shield bonuses. A magic shield typically grants an enhancement bonus to the shield's shield bonus, which has the effect of increasing the shield's overall bonus to AC. A shield bonus granted by a spell or magic item typically takes the form of an invisible, tangible field of force that protects the recipient. A shield bonus doesn't apply against touch attacks.

\textbf{Size Modifier:} A size bonus or penalty is derived from a creature's size category. Size modifiers of different kinds apply to Armor Class, attack rolls, Hide checks, grapple checks, and various other checks.



\subsubsection{Rounding Fractions}
In general, if you wind up with a fraction, round down, even if the fraction is one-half or larger.

\textit{Exception:} Certain rolls, such as damage and hit points, have a minimum of 1.

% \Figure*{b}{images/wizard-3.png}

\subsection{Multiplying}
Sometimes a rule makes you multiply a number or a die roll. As long as you're applying a single multiplier, multiply the number normally. When two or more multipliers apply to any abstract value (such as a modifier or a die roll), however, combine them into a single multiple, with each extra multiple adding 1 less than its value to the first multiple. Thus, a double ($\times$2) and a double ($\times$2) applied to the same number results in a triple ($\times$3, because 2 + 1 = 3).

When applying multipliers to real-world values (such as weight or distance), normal rules of math apply instead. A creature whose size doubles (thus multiplying its weight by 8) and then is turned to stone (which would multiply its weight by a factor of roughly 3) now weighs about 24 times normal, not 10 times normal. Similarly, a blinded creature attempting to negotiate difficult terrain would count each square as 4 squares (doubling the cost twice, for a total multiplier of $\times$4), rather than as 3 squares (adding 100\% twice).

\subsection{Advantage and Disadvantage}
Sometimes a special ability or spell tells you that you have advantage or disadvantage on an ability check, a saving throw, or an attack roll. When that happens, you roll a second d20 when you make the roll. Use the higher of the two rolls if you have advantage, and use the lower roll if you have disadvantage. For example, if you have disadvantage and roll a 17 and a 5, you use the 5. If you instead have advantage and roll those numbers, you use the 17.

If multiple situations affect a roll and each one grants advantage or imposes disadvantage on it, you don't roll more than one additional d20. If two favorable situations grant advantage, for example, you still roll only one additional d20.

If circumstances cause a roll to have both advantage and disadvantage, you are considered to have neither of them, and you roll one d20. This is true even if multiple circumstances impose disadvantage and only one grants advantage or vice versa. In such a situation, you have neither advantage nor disadvantage.
