\section{Alternate Class Features}
Choosing a class fixes key aspects of a character: their role, their abilities, and their attack and save bonuses. However, you can change a class feature to provide a new experience. Variant versions of several of class features are presented below. If you prefer the variant to the standard class feature, ask your game master if he approves of your swapping out your class feature for the variant version.

These alternate class features can exist side by side with the standard class features---some bards might work alone while their comrades favor having contacts all over Athas---or can completely replace the standard features. The balance between the standard class features and the alternate features is up to the game master.

\subsection{Bard}
\AlternateClassFeature{Contact}
{Instead of relying on their own wits like a traditional bard, a \emph{draqoman} relies on the help of others more capable in specific matters. Draqomen serve as guides and translators for newcomers to the cities. Local laws and customs vary by location, and a local resident who knows the city like the back of his hand and can call in a favor if necessary is definitely worth his pay. A draqoman will usually be loyal as long as he is paid and fed well. Mistreat your draqoman and you will soon find yourself on the wrong end of someone’s blade or in the prisons run by the templars.}
{\skill{Gather Information} 4 ranks, \skill{Knowledge} (local) 4 ranks.}
{6th, 11th, and 16th.}
{You do not gain quick thinking, or its improvements at 11th and 16th levels.}
{
	Once per week you may call acquaintances to do favors for you. The DM has final say on the extent of favors that may be extracted. The following list provides sample uses of contact.

\begin{itemize*}
\item Additional 5\% discount on purchased goods.
\item Access to purchase and sell black market goods.
\item Access to hire mercenary of trader's desired race and class (see Hirelings).
\item Access to purchasing spellcasting services.
\item Access to information (equal to \skill{Gather Information} DC 20).
\item Access to forged materials (equal to \skill{Forgery} DC 20).
\item Access to decipher (equal to \skill{Decipher Script} DC 20).
\item Access to other expert (skill check DC 20, at DM's discretion).
\item Appointment or meeting with an NPC (templar, noble, gladiatorial slave, chieftain, etc. At DM's discretion).
\item Access to a place to stay hidden for three days.
\item Avoid templar inspection.
\end{itemize*}

	You gain one additional use of contact at 11th level, and another again at 16th level.
}

\subsection{Ranger}
\AlternateClassFeature{Desert Runner}
{Desert runners are elves that have devoted themselves to the run, pushing themselves to the limit of Elven running ability. They are the scouts and messengers of a tribe. They run ahead of the rest, or alone through the desert to deliver important messages or items between clans with the greatest of speed. They are also some of the more competent hunters of the tribe able to track quarry silently while moving quickly through the desert.}
{Elf.}
{4th.}
{You do not gain an animal companion.}
{
	You gain +1 bonus per ranger level on all \skill{Concentration} checks related to the elf run ability, and all other checks to continue running. This includes skill checks to avoid tripping or falling and saving throws to resist effects that would directly slow or impede movement (such as a Will save to resist a \spell{slow} spell). It does not however include indirectly related checks, such as a \skill{Spot} check to notice a pit trap.

	You gain +10 feet of land speed while wearing no armor or light armor, and not carrying heavy load. Your speed increases by 10 feet for every three ranger levels thereafter (+20ft. at 7th, +30ft. at 10th, +40ft. at 13th, +50ft. at 16th, and +60ft. at 19th).

	You gain +1 AC dodge bonus when running. This bonus increases by 1 for every three ranger levels thereafter (+2 at 7th, +3 at 10th, +4 at 13th, +5 at 16th, and +6 at 19th).
}
\AlternateClassFeature{Sand Chitin}
{Usually selected from those thri-kreen that are slightly smaller and quicker, these kreen are trained extensively in the lore of the hunt and the way of stealth, honing their natural skills beyond those of the norm of their kind.}
{Thri-kreen.}
{1st.}
{You do not gain wild empathy.}
{
	Your racial \skill{Hide} bonus increases by 2 (+6 total). For every four ranger levels thereafter, it increases by another 2 (+8 at 5th, +10 at 9th, +12 at 13th, and +14 at 17th).
}

% \subsection{Templar}
% \AlternateClassFeature{Bureau Specialization}
% {}
% {Must be from the templarate of Tyr.}
% {8th and 12th.}
% {To select this class feature, you must permanently sacrifice one of your 4th-level spell slots, then another 6th-level spell slot at 12th level.}
% {Choose a templar class skill. You gain +3 bonus on checks of the chosen skill at 8th level.

% At 12th level, this bonus increases to +6 and you can take 10 even if stress and distraction would normally prevent you from doing so.}

\subsection{Wilder}
\AlternateClassFeature{Psionic Ritual}
{Bereft of formal training in the Way, psionically talented individuals from the tribal and nomadic peoples of the Tablelands and beyond must make do with their own understanding of the psionic arts. Seen by formally trained psions as an aberration of proper psionic practice, these self trained individuals can sometimes produce effects that leave their detractors speechless.}
{\skill{Profession} (herbalist) 4 ranks, \skill{Survival} 4 ranks.}
{7th.}
{You do not improve wild surge at 7th level.}
{
	Once per day, you can perform such a ritual to temporarily increase your Will. Each psionic ritual is unique, being of your own design, but all take one hour to complete and require a DC 20 \skill{Concentration} check. If the ritual is successful, you gains 1d4+1 temporary power points over and above your normal maximum. These power points remain available for one day or until they are used.
}

\subsection{Wizard}
\AlternateClassFeature{Exegete}
{As a result of their endless research and studies, arcanists end up knowing a little about a lot of different things. They are consulted often, becoming experts and advisers for their tribes.}
{Elf, \skill{Knowledge} (history) 10 ranks.}
{10th.}
{You do not gain a bonus feat.}
{
	You are considered to be trained in all forms of Knowledge and can choose to take 10 in \skill{Knowledge} checks which you have at least 10 ranks in.
}
\AlternateClassFeature{Phantasmal Guardian}
{Halfling protectors are masters of illusions that can aid their tribes and bring doom to their enemies in many strange ways.}
{Halfling, \skill{Knowledge} (nature) 10 ranks.}
{10th.}
{You do not gain a bonus feat.}
{
	You can summon a non-corporeal shadow figure that wards an area with a radius of 100 ft. + 10 ft. per level. Any other creature entering the warded area, except you and those you designate, will be attacked by the phantasmal guardian, as per the \spell{phantasmal killer} spell. The guardian can only attack once, where upon it dissipates. Summoning the guardian takes 1 minute, and it remains in the area until you dispels it (move action, unlimited range), or until it attacks someone. You can only have one guardian summoned at a time.

	You can use this ability 3 times per day.

	This is a spell-like ability.
}