\section{Environment Effects}
Environmental hazards specific to one kind of terrain (such as an avalanche, which occurs in the mountains) are described in \chapref{Adventuring} for each type of terrain, and again in \chapref{Campaigns} with a summary of all environmental hazards with a Challenge Rating.

Environmental effects that are common to more than one setting are detailed below. These environmental effects are not challenges as hazards are, but the mere characteristics of certain situations.

\subsection{Acid Effects}
Corrosive acids deals 1d6 points of damage per round of exposure except in the case of total immersion (such as into a vat of acid), which deals 10d6 points of damage per round. An attack with acid, such as from a hurled vial or a monster's spittle, counts as a round of exposure.

The fumes from most acids are inhaled poisons. Those who come close enough to a large body of acid to dunk a creature in it must make a DC 13 Fortitude save or take 1 point of Constitution damage. All such characters must make a second save 1 minute later or take another 1d4 points of Constitution damage.

Creatures immune to acid's caustic properties might still drown in it if they are totally immersed.


\subsection{Fire Effects}
Characters exposed to burning oil, bonfires, and noninstantaneous magic fires might find their clothes, hair, or equipment on fire. Spells with an instantaneous duration don't normally set a character on fire, since the heat and flame from these come and go in a flash.

Characters at risk of catching fire are allowed a DC 15 Reflex save to avoid this fate. If a character's clothes or hair catch fire, he takes 1d6 points of damage immediately. In each subsequent round, the burning character must make another Reflex saving throw. Failure means he takes another 1d6 points of damage that round. Success means that the fire has gone out. (That is, once he succeeds on his saving throw, he's no longer on fire.)

A character on fire may automatically extinguish the flames by jumping into enough water to douse himself. If no body of water is at hand, rolling on the ground or smothering the fire with cloaks or the like permits the character another save with a +4 bonus.

Those unlucky enough to have their clothes or equipment catch fire must make DC 15 Reflex saves for each item. Flammable items that fail take the same amount of damage as the character.


\subsection{Darkness}
Darkvision allows many characters and monsters to see perfectly well without any light at all, but characters with normal vision (or low-light vision, for that matter) can be rendered completely blind by putting out the lights. Torches or lanterns can be blown out by sudden gusts of subterranean wind, magical light sources can be dispelled or countered, or magical traps might create fields of impenetrable darkness.

In many cases, some characters or monsters might be able to see, while others are blinded. For purposes of the following points, a blinded creature is one who simply can't see through the surrounding darkness.

\begin{itemize*}
\item Creatures blinded by darkness lose the ability to deal extra damage due to precision (for example, a sneak attack).
\item Blinded creatures are hampered in their movement, and pay 2 squares of movement per square moved into (double normal cost). Blinded creatures can't run or charge.
\item All opponents have total concealment from a blinded creature, so the blinded creature has a 50\% miss chance in combat. A blinded creature must first pinpoint the location of an opponent in order to attack the right square; if the blinded creature launches an attack without pinpointing its foe, it attacks a random square within its reach. For ranged attacks or spells against a foe whose location is not pinpointed, roll to determine which adjacent square the blinded creature is facing; its attack is directed at the closest target that lies in that direction.
\item A blinded creature loses its Dexterity adjustment to AC and takes a $-2$ penalty to AC.
\item A blinded creature takes a $-4$ penalty on Search checks and most Strength- and Dexterity-based skill checks, including any with an armor check penalty. A creature blinded by darkness automatically fails any skill check relying on vision.
\item Creatures blinded by darkness cannot use gaze attacks and are immune to gaze attacks.
\end{itemize*}

A creature blinded by darkness can make a \skill{Listen} check as a free action each round in order to locate foes (DC equal to opponents' Move Silently checks). A successful check lets a blinded character hear an unseen creature “over there somewhere.” It's almost impossible to pinpoint the location of an unseen creature. A \skill{Listen} check that beats the DC by 20 reveals the unseen creature's square (but the unseen creature still has total concealment from the blinded creature).

\begin{itemize*}
\item A blinded creature can grope about to find unseen creatures. A character can make a touch attack with his hands or a weapon into two adjacent squares using a standard action. If an unseen target is in the designated square, there is a 50\% miss chance on the touch attack. If successful, the groping character deals no damage but has pinpointed the unseen creature's current location. (If the unseen creature moves, its location is once again unknown.)
\item If a blinded creature is struck by an unseen foe, the blinded character pinpoints the location of the creature that struck him (until the unseen creature moves, of course). The only exception is if the unseen creature has a reach greater than 1.5 meter (in which case the blinded character knows the location of the unseen opponent, but has not pinpointed him) or uses a ranged attack (in which case, the blinded character knows the general direction of the foe, but not his location).
\item A creature with the scent ability automatically pinpoints unseen creatures within 1.5 meter of its location.
\end{itemize*}


\subsection{Falling}
\textbf{Falling Damage:} The basic rule is simple: 1d6 points of damage per 3 meters fallen, to a maximum of 20d6.

If a character deliberately jumps instead of merely slipping or falling, the damage is the same but the first 1d6 is nonlethal damage. A DC 15 \skill{Jump} check or DC 15 \skill{Tumble} check allows the character to avoid any damage from the first 3 meters fallen and converts any damage from the second 3 meters to nonlethal damage. Thus, a character who slips from a ledge 9 meters up takes 3d6 damage. If the same character deliberately jumped, he takes 1d6 points of nonlethal damage and 2d6 points of lethal damage. And if the character leaps down with a successful \skill{Jump} or \skill{Tumble} check, he takes only 1d6 points of nonlethal damage and 1d6 points of lethal damage from the plunge.

Falls onto yielding surfaces (soft ground, mud) also convert the first 1d6 of damage to nonlethal damage. This reduction is cumulative with reduced damage due to deliberate jumps and the \skill{Jump} skill.

\textbf{Falling into Water:} Falls into water are handled somewhat differently. If the water is at least 3 meters deep, the first 6 meters of falling do no damage. The next 6 meters do nonlethal damage (1d3 per 3-meter increment). Beyond that, falling damage is lethal damage (1d6 per additional 3-meter increment).

Characters who deliberately dive into water take no damage on a successful DC 15 Swim check or DC 15 \skill{Tumble} check, so long as the water is at least 3 meters deep for every 9 meters fallen. However, the DC of the check increases by 5 for every 15 meters of the dive.


\subsection{Falling Objects}
Just as characters take damage when they fall more than 3 meters, so too do they take damage when they are hit by falling objects.

Objects that fall upon characters deal damage based on their weight and the distance they have fallen.

For each 100 kilograms of an object's weight, the object deals 1d6 points of damage, provided it falls at least 3 meters. Distance also comes into play, adding an additional 1d6 points of damage for every 3-meter increment it falls beyond the first (to a maximum of 20d6 points of damage).

Objects smaller than 100 kilograms also deal damage when dropped, but they must fall farther to deal the same damage. Use \tabref{Damage from Falling Objects} to see how far an object of a given weight must drop to deal 1d6 points of damage.

\Table{Damage from Falling Objects}{CC}{
  \tableheader Object Weight
& \tableheader Distance \\

100--51 kg &  6 m \\
 50--26 kg &  9 m \\
 25--16 kg & 12 m \\
  15--6 kg & 15 m \\
   5--3 kg & 18 m \\
   2--1 kg & 21 m \\
}

For each additional increment an object falls, it deals an additional 1d6 points of damage.

Objects weighing less than 1 pound do not deal damage to those they land upon, no matter how far they have fallen.


\subsection{Lava Effects}
Lava or magma deals 2d6 points of damage per round of exposure, except in the case of total immersion (such as when a character falls into the crater of an active volcano), which deals 20d6 points of damage per round.

Damage from magma continues for 1d3 rounds after exposure ceases, but this additional damage is only half of that dealt during actual contact (that is, 1d6 or 10d6 points per round).

An immunity or resistance to fire serves as an immunity to lava or magma. However, a creature immune to fire might still drown if completely immersed in lava.


\subsection{Smoke Effects}
A character who breathes heavy smoke must make a Fortitude save each round (DC 15, +1 per previous check) or spend that round choking and coughing. A character who chokes for 2 consecutive rounds takes 1d6 points of nonlethal damage.

Smoke obscures vision, giving concealment (20\% miss chance) to characters within it.


\subsection{Suffocation}
A character who has no air to breathe can hold her breath for 2 rounds per point of Constitution. After this period of time, the character must make a DC 10 Constitution check in order to continue holding her breath. The save must be repeated each round, with the DC increasing by +1 for each previous success.

When the character fails one of these Constitution checks, she begins to suffocate. In the first round, she falls unconscious (0 hit points). In the following round, she drops to $-1$ hit points and is dying. In the third round, she suffocates.

\subsubsection{Slow Suffocation}
A Medium character can breathe easily for 6 hours in a sealed chamber measuring 3 meters on a side. After that time, the character takes 1d6 points of nonlethal damage every 15 minutes. Each additional Medium character or significant fire source (a torch, for example) proportionally reduces the time the air will last. When a character falls unconscious from this nonlethal damage, she drops to $-1$ hit points and is dying. In the next round, she suffocates.

Small characters consume half as much air as Medium characters. A larger volume of air, of course, lasts for a longer time.

\subsection{Water Dangers}
Any character can wade in relatively calm water that isn't over his head, no check required. Similarly, swimming in calm water only requires skill checks with a DC of 10. Trained swimmers can just take 10. (Remember, however, that armor or heavy gear makes any attempt at swimming much more difficult.)

By contrast, fast-moving water is much more dangerous. On a successful DC 15 \skill{Swim} check or a DC 15 Strength check, it deals 1d3 points of nonlethal damage per round (1d6 points of lethal damage if flowing over rocks and cascades). On a failed check, the character must make another check that round to avoid going under.

Very deep water is not only generally pitch black, posing a navigational hazard, but worse, it deals water pressure damage of 1d6 points per minute for every 30 meters the character is below the surface. A successful Fortitude save (DC 15, +1 for each previous check) means the diver takes no damage in that minute. Very cold water deals 1d6 points of nonlethal damage from hypothermia per minute of exposure.

\subsubsection{Drowning}
Any character can hold her breath for a number of rounds equal to twice her Constitution score. After this period of time, the character must make a DC 10 Constitution check every round in order to continue holding her breath. Each round, the DC increases by 1. (See \skill{Swim} skill description.)

When the character finally fails her Constitution check, she begins to drown. In the first round, she falls unconscious (0 hp). In the following round, she drops to $-1$ hit points and is dying. In the third round, she drowns.

It is possible to drown in substances other than water, such as sand, quicksand, fine dust, and silos full of grain.
