\section{Type Summary}
% TODO: Change text
Type determines how magic affects a creature; for example, the \spell{hold animal} spell affects only creatures of the animal type. Type determines certain features, such as Hit Dice size, base attack bonus, base saving throw bonuses, and skill points.

Subtypes describe a creature's intrinsic affinities to other planes, their race, or even their form.

All types and some subtypes also impose inherent traits for any creature, such as a fixed alignment, or proficiency in weapons or armor.

\subsection{Types}
\textbf{Features:} For creatures, their racial hit dice function as classes, granting a number of features. Each racial hit die grants base attack bonus and skill points. Racial hit dice also have a number of good saves. Class skills for monster are the skills described in each monster's entry.

Below are the description of each creature type.

\subsubsection{Aberration Type}
An aberration has a bizarre anatomy, strange abilities, an alien mindset, or any combination of the three.

\textbf{Features:} An aberration has the following features.
\begin{itemize*}
\item 8-sided Hit Dice.
\item Base attack bonus equal to \threequarters total Hit Dice (as cleric).
\item Good Will saves.
\item Skill points equal to (2 + Int modifier, minimum 1) per Hit Die, with quadruple skill points for the first Hit Die.
\end{itemize*}

\textbf{Traits:} An aberration possesses the following traits (unless otherwise noted in a creature's entry).
\begin{itemize*}
\item Darkvision out to 18 meters.
\item Proficient with its natural weapons. If generally humanoid in form, proficient with all simple weapons and any weapon it is described as using.
\item Proficient with whatever type of armor (light, medium, or heavy) it is described as wearing, as well as all lighter types. Aberrations not indicated as wearing armor are not proficient with armor. Aberrations are proficient with shields if they are proficient with any form of armor.
\item Aberrations eat, sleep, and breathe.
\end{itemize*}

\subsubsection{Animal Type}
An animal is a living, nonhuman creature, usually a vertebrate with no magical abilities and no innate capacity for language or culture.

\textbf{Features:} An animal has the following features (unless otherwise noted in a creature's entry).
\begin{itemize*}
\item 8-sided Hit Dice.
\item Base attack bonus equal to \threequarters total Hit Dice (as cleric).
\item Good Fortitude and Reflex saves (certain animals have different good saves, for instance dire animals have good Fortitude, Reflex, and Will saves).
\item Skill points equal to (2 + Int modifier, minimum 1) per Hit Die, with quadruple skill points for the first Hit Die.
\end{itemize*}

\textbf{Traits:} An animal possesses the following traits (unless otherwise noted in a creature's entry).
\begin{itemize*}
\item Intelligence score of 1 or 2 (no creature with an Intelligence score of 3 or higher can be an animal).
\item Low-light vision.
\item \emph{Alignment:} Always neutral.
\item \emph{Treasure:} None.
\item Proficient with its natural weapons only. A noncombative herbivore uses its natural weapons as a secondary attack. Such attacks are made with a $-5$ penalty on the creature's attack rolls, and the animal receives only \onehalf its Strength modifier as a damage adjustment.
\item Proficient with no armor unless trained for war.
\item Animals eat, sleep, and breathe.
\end{itemize*}

\subsubsection{Construct Type}
A construct is an animated object or artificially constructed creature.

\textbf{Features:} A construct has the following features.
\begin{itemize*}
\item 10-sided Hit Dice.
\item Base attack bonus equal to \threequarters total Hit Dice (as cleric).
\item No good saving throws.
\item Skill points equal to (2 + Int modifier, minimum 1) per Hit Die, with quadruple skill points for the first Hit Die, if the construct has an Intelligence score. However, most constructs are mindless and gain no skill points or feats.
\end{itemize*}

\textbf{Traits:} A construct possesses the following traits (unless otherwise noted in a creature's entry).
\begin{itemize*}
\item No Constitution score.
\item Low-light vision.
\item Darkvision out to 18 meters.
\item Immunity to all mind-affecting effects (charms, compulsions, phantasms, patterns, and morale effects).
\item Immunity to poison, sleep effects, paralysis, stunning, disease, death effects, and necromancy effects.
\item Cannot heal damage on their own, but often can be repaired by exposing them to a certain kind of effect (see the creature's description for details) or through the use of the Craft Construct feat. A construct with the fast healing special quality still benefits from that quality.
\item Not subject to critical hits, nonlethal damage, ability damage, ability drain, fatigue, exhaustion, or energy drain.
\item Immunity to any effect that requires a Fortitude save (unless the effect also works on objects, or is harmless).
\item Not at risk of death from massive damage. Immediately destroyed when reduced to 0 hit points or less.
\item Since it was never alive, a construct cannot be raised or resurrected.
\item Because its body is a mass of unliving matter, a construct is hard to destroy. It gains bonus hit points based on size, as shown on the table.

\Table{}{*2{p{23mm}c}}{
  \tableheader Construct Size
& \tableheader Bonus HP
& \tableheader Construct Size
& \tableheader Bonus HP \\
Tiny or smaller &    & Huge       & 40 \\
Small           & 10 & Gargantuan & 60 \\
Medium          & 20 & Colossal   & 80 \\
Large           & 30 &&\\
}
\item Proficient with its natural weapons only, unless generally humanoid in form, in which case proficient with any weapon mentioned in its entry.
\item Proficient with no armor.
\item Constructs do not eat, sleep, or breathe.
\end{itemize*}

\subsubsection{Dragon Type}
A dragon is a reptilelike creature, usually winged, with magical or unusual abilities.

\textbf{Features:} A dragon has the following features.
\begin{itemize*}
\item 12-sided Hit Dice.
\item Base attack bonus equal to total Hit Dice (as fighter).
\item Good Fortitude, Reflex, and Will saves.
\item Skill points equal to (6 + Int modifier, minimum 1) per Hit Die, with quadruple skill points for the first Hit Die.
\end{itemize*}

\textbf{Traits:} A dragon possesses the following traits (unless otherwise noted in the description of a particular kind).
\begin{itemize*}
\item Darkvision out to 18 meters and low-light vision.
\item Immunity to magic sleep effects and paralysis effects.
\item Proficient with its natural weapons only unless humanoid in form (or capable of assuming humanoid form), in which case proficient with all simple weapons and any weapons mentioned in its entry.
\item Proficient with no armor.
\item Dragons eat, sleep, and breathe.
\end{itemize*}

\subsubsection{Elemental Type}
An elemental is a being composed of one of the four classical elements: air, earth, fire, or water.

\textbf{Features:} An elemental has the following features.
\begin{itemize*}
\item 8-sided Hit Dice.
\item Base attack bonus equal to \threequarters total Hit Dice (as cleric).
\item Good saves depend on the element: Fortitude (earth, water) or Reflex (air, fire).
\item Skill points equal to (2 + Int modifier, minimum 1) per Hit Die, with quadruple skill points for the first Hit Die.
\end{itemize*}

\textbf{Traits:} An elemental possesses the following traits (unless otherwise noted in a creature's entry).
\begin{itemize*}
\item Darkvision out to 18 meters.
\item Immunity to poison, sleep effects, paralysis, and stunning.
\item Not subject to critical hits or flanking.
\item Unlike most other living creatures, an elemental does not have a dual nature—its soul and body form one unit. When an elemental is slain, no soul is set loose. Spells that restore souls to their bodies, such as \spell{raise dead}, \spell{reincarnate}, and \spell{resurrection}, don't work on an elemental. It takes a different magical effect, such as \spell{limited wish}, \spell{wish}, \spell{miracle}, or \spell{true resurrection}, to restore it to life.
\item Proficient with natural weapons only, unless generally humanoid in form, in which case proficient with all simple weapons and any weapons mentioned in its entry.
\item Proficient with whatever type of armor (light, medium, or heavy) that it is described as wearing, as well as all lighter types. Elementals not indicated as wearing armor are not proficient with armor. Elementals are proficient with shields if they are proficient with any form of armor.
\item Elementals do not eat, sleep, or breathe.
\end{itemize*}

\subsubsection{Fey Type}
A fey is a creature with supernatural abilities and connections to nature or to some other force or place. Fey are usually human-shaped.

\textbf{Features:} A fey has the following features.
\begin{itemize*}
\item 6-sided Hit Dice.
\item Base attack bonus equal to \onehalf total Hit Dice (as wizard).
\item Good Reflex and Will saves.
\item Skill points equal to (6 + Int modifier, minimum 1) per Hit Die, with quadruple skill points for the first Hit Die.
\end{itemize*}

\textbf{Traits:} A fey possesses the following traits (unless otherwise noted in a creature's entry).
\begin{itemize*}
\item Low-light vision.
\item Proficient with all simple weapons and any weapons mentioned in its entry.
\item Proficient with whatever type of armor (light, medium, or heavy) that it is described as wearing, as well as all lighter types. Fey not indicated as wearing armor are not proficient with armor. Fey are proficient with shields if they are proficient with any form of armor.
\item Fey eat, sleep, and breathe.
\end{itemize*}

\subsubsection{Giant Type}
A giant is a humanoid-shaped creature of great strength, usually of at least Large size.

\textbf{Features:} A giant has the following features.
\begin{itemize*}
\item 8-sided Hit Dice.
\item Base attack bonus equal to \threequarters total Hit Dice (as cleric).
\item Good Fortitude saves.
\item Skill points equal to (2 + Int modifier, minimum 1) per Hit Die, with quadruple skill points for the first Hit Die.
\end{itemize*}

\textbf{Traits:} A giant possesses the following traits (unless otherwise noted in a creature's entry).
\begin{itemize*}
\item Low-light vision.
\item Proficient with all simple and martial weapons, as well as any natural weapons.
\item Proficient with whatever type of armor (light, medium or heavy) it is described as wearing, as well as all lighter types. Giants not described as wearing armor are not proficient with armor. Giants are proficient with shields if they are proficient with any form of armor.
\item Giants eat, sleep, and breathe.
\end{itemize*}

\subsubsection{Humanoid Type}
A humanoid usually has two arms, two legs, and one head, or a humanlike torso, arms, and a head. Humanoids have few or no supernatural or extraordinary abilities, but most can speak and usually have well-developed societies. They usually are Small or Medium. Every humanoid creature also has a subtype.

Humanoids with 1 Hit Die exchange the features of their humanoid Hit Die for the class features of a PC or NPC class. Humanoids of this sort are presented as 1st-level warriors, which means that they have average combat ability and poor saving throws.

Humanoids with more than 1 Hit Die are the only humanoids who make use of the features of the humanoid type.

\textbf{Features:} A humanoid has the following features (unless otherwise noted in a creature's entry).
\begin{itemize*}
\item 8-sided Hit Dice, or by character class.
\item Base attack bonus equal to \threequarters total Hit Dice (as cleric).
\item Good Reflex saves (usually; a humanoid's good save varies).
\item Skill points equal to (2 + Int modifier, minimum 1) per Hit Die, with quadruple skill points for the first Hit Die, or by character class.
\end{itemize*}

\textbf{Traits:} A humanoid possesses the following traits (unless otherwise noted in a creature's entry).
\begin{itemize*}
\item Proficient with all simple weapons, or by character class.
\item Proficient with whatever type of armor (light, medium, or heavy) it is described as wearing, or by character class. If a humanoid does not have a class and wears armor, it is proficient with that type of armor and all lighter types. Humanoids not indicated as wearing armor are not proficient with armor. Humanoids are proficient with shields if they are proficient with any form of armor.
\item Humanoids breathe, eat, and sleep.
\end{itemize*}

\subsubsection{Magical Beast Type}
Magical beasts are similar to animals but can have Intelligence scores higher than 2. Magical beasts usually have supernatural or extraordinary abilities, but sometimes are merely bizarre in appearance or habits.

\textbf{Features:} A magical beast has the following features.
\begin{itemize*}
\item 10-sided Hit Dice.
\item Base attack bonus equal to total Hit Dice (as fighter).
\item Good Fortitude and Reflex saves.
\item Skill points equal to (2 + Int modifier, minimum 1) per Hit Die, with quadruple skill points for the first Hit Die.
\end{itemize*}

\textbf{Traits:} A magical beast possesses the following traits (unless otherwise noted in a creature's entry).
\begin{itemize*}
\item Darkvision out to 18 meters and low-light vision.
\item Proficient with its natural weapons only.
\item Proficient with no armor.
\item Magical beasts eat, sleep, and breathe.
\end{itemize*}

\subsubsection{Monstrous Humanoid Type}
Monstrous humanoids are similar to humanoids, but with monstrous or animalistic features. They often have magical abilities as well.

\textbf{Features:} A monstrous humanoid has the following features.
\begin{itemize*}
\item 8-sided Hit Dice.
\item Base attack bonus equal to total Hit Dice (as fighter).
\item Good Reflex and Will saves.
\item Skill points equal to (2 + Int modifier, minimum 1) per Hit Die, with quadruple skill points for the first Hit Die.
\end{itemize*}

\textbf{Traits:} A monstrous humanoid possesses the following traits (unless otherwise noted in a creature's entry).
\begin{itemize*}
\item Darkvision out to 18 meters.
\item Proficient with all simple weapons and any weapons mentioned in its entry.
\item Proficient with whatever type of armor (light, medium, or heavy) it is described as wearing, as well as all lighter types. Monstrous humanoids not indicated as wearing armor are not proficient with armor. Monstrous humanoids are proficient with shields if they are proficient with any form of armor.
\item Monstrous humanoids eat, sleep, and breathe.
\end{itemize*}

\subsubsection{Ooze Type}
An ooze is an amorphous or mutable creature, usually mindless.

\textbf{Features:} An ooze has the following features.
\begin{itemize*}
\item 10-sided Hit Dice.
\item Base attack bonus equal to \threequarters total Hit Dice (as cleric).
\item No good saving throws.
\item Skill points equal to (2 + Int modifier, minimum 1) per Hit Die, with quadruple skill points for the first Hit Die, if the ooze has an Intelligence score. However, most oozes are mindless and gain no skill points or feats.
\end{itemize*}

\textbf{Traits:} An ooze possesses the following traits (unless otherwise noted in a creature's entry).
\begin{itemize*}
\item \textit{Mindless:} No Intelligence score, and immunity to all mind-affecting effects (charms, compulsions, phantasms, patterns, and morale effects).
\item Blind (but have the blindsight special quality), with immunity to gaze attacks, visual effects, illusions, and other attack forms that rely on sight.
\item Immunity to poison, sleep effects, paralysis, polymorph, and stunning.
\item Some oozes have the ability to deal acid damage to objects. In such a case, the amount of damage is equal to 10 + \onehalf ooze's HD + ooze's Con modifier per full round of contact.
\item Not subject to critical hits or flanking.
\item Proficient with its natural weapons only.
\item Proficient with no armor.
\item Oozes eat and breathe, but do not sleep.
\end{itemize*}

\subsubsection{Outsider Type}
An outsider is at least partially composed of the essence (but not necessarily the material) of some plane other than the Material Plane. Some creatures start out as some other type and become outsiders when they attain a higher (or lower) state of spiritual existence.

\textbf{Features:} An outsider has the following features.
\begin{itemize*}
\item 8-sided Hit Dice.
\item Base attack bonus equal to total Hit Dice (as fighter).
\item Good Fortitude, Reflex, and Will saves.
\item Skill points equal to (8 + Int modifier, minimum 1) per Hit Die, with quadruple skill points for the first Hit Die.
\end{itemize*}

\textbf{Traits:} An outsider possesses the following traits (unless otherwise noted in a creature's entry).
\begin{itemize*}
\item Darkvision out to 18 meters.
\item Unlike most other living creatures, an outsider does not have a dual nature—its soul and body form one unit. When an outsider is slain, no soul is set loose. Spells that restore souls to their bodies, such as \spell{raise dead}, \spell{reincarnate}, and \spell{resurrection}, don't work on an outsider. It takes a different magical effect, such as \spell{limited wish}, \spell{wish}, \spell{miracle}, or \spell{true resurrection} to restore it to life. An outsider with the native subtype can be raised, reincarnated, or resurrected just as other living creatures can be.
\item Proficient with all simple and martial weapons and any weapons mentioned in its entry.
\item Proficient with whatever type of armor (light, medium, or heavy) it is described as wearing, as well as all lighter types. Outsiders not indicated as wearing armor are not proficient with armor. Outsiders are proficient with shields if they are proficient with any form of armor.
\item Outsiders breathe, but do not need to eat or sleep (although they can do so if they wish). Native outsiders breathe, eat, and sleep.
\end{itemize*}

\subsubsection{Plant Type}
This type comprises vegetable creatures. Note that regular plants, such as one finds growing in gardens and fields, lack Wisdom and Charisma scores (see Nonabilities) and are not creatures, but objects, even though they are alive.

\textbf{Features:} A plant creature has the following features.
\begin{itemize*}
\item 8-sided Hit Dice.
\item Base attack bonus equal to \threequarters total Hit Dice (as cleric).
\item Good Fortitude saves.
\item Skill points equal to (2 + Int modifier, minimum 1) per Hit Die, with quadruple skill points for the first Hit Die, if the plant creature has an Intelligence score. However, some plant creatures are mindless and gain no skill points or feats.
\end{itemize*}

\textbf{Traits:} A plant creature possesses the following traits (unless otherwise noted in a creature's entry).
\begin{itemize*}
\item Low-light vision.
\item Immunity to all mind-affecting effects (charms, compulsions, phantasms, patterns, and morale effects).
\item Immunity to poison, sleep effects, paralysis, polymorph, and stunning.
\item Not subject to critical hits.
\item Proficient with its natural weapons only.
\item Proficient with no armor.
\item Plants breathe and eat, but do not sleep.
\end{itemize*}

\subsubsection{Undead Type}
Undead are once-living creatures animated by spiritual or supernatural forces.

\textbf{Features:} An undead creature has the following features.
\begin{itemize*}
\item 12-sided Hit Dice.
\item Base attack bonus equal to \onehalf total Hit Dice (as wizard).
\item Good Will saves.
\item Skill points equal to (4 + Int modifier, minimum 1) per Hit Die, with quadruple skill points for the first Hit Die, if the undead creature has an Intelligence score. However, many undead are mindless and gain no skill points or feats.
\end{itemize*}

\textbf{Traits:} An undead creature possesses the following traits (unless otherwise noted in a creature's entry).
\begin{itemize*}
\item No Constitution score.
\item Darkvision out to 18 meters.
\item Immunity to all mind-affecting effects (charms, compulsions, phantasms, patterns, and morale effects).
\item Immunity to poison, sleep effects, paralysis, stunning, disease, and death effects.
\item Not subject to critical hits, nonlethal damage, ability drain, or energy drain. Immune to damage to its physical ability scores (Strength, Dexterity, and Constitution), as well as to fatigue and exhaustion effects.
\item Cannot heal damage on its own if it has no Intelligence score, although it can be healed. Negative energy (such as an \spell{inflict} spell) can heal undead creatures. The fast healing special quality works regardless of the creature's Intelligence score.
\item Immunity to any effect that requires a Fortitude save (unless the effect also works on objects or is harmless).
\item Uses its Charisma modifier for Concentration checks.
\item Not at risk of death from massive damage, but when reduced to 0 hit points or less, it is immediately destroyed.
\item Not affected by \spell{raise dead} and \spell{reincarnate} spells or abilities. \spell{Resurrection} and \spell{true resurrection} can affect undead creatures. These spells turn undead creatures back into the living creatures they were before becoming undead.
\item Proficient with its natural weapons, all simple weapons, and any weapons mentioned in its entry.
\item Proficient with whatever type of armor (light, medium, or heavy) it is described as wearing, as well as all lighter types. Undead not indicated as wearing armor are not proficient with armor. Undead are proficient with shields if they are proficient with any form of armor.
\item Undead do not breathe, eat, or sleep.
\end{itemize*}

\subsubsection{Vermin Type}
This type includes insects, arachnids, other arthropods, worms, and similar invertebrates.

\textbf{Features:} Vermin have the following features.
\begin{itemize*}
\item 8-sided Hit Dice.
\item Base attack bonus equal to \threequarters total Hit Dice (as cleric).
\item Good Fortitude saves.
\item Skill points equal to (2 + Int modifier, minimum 1) per Hit Die, with quadruple skill points for the first Hit Die, if the vermin has an Intelligence score. However, most vermin are mindless and gain no skill points or feats.
\end{itemize*}

\textbf{Traits:} Vermin possess the following traits (unless otherwise noted in a creature's entry).

\begin{itemize*}
\item \textit{Mindless:} No Intelligence score, and immunity to all mind-affecting effects (charms, compulsions, phantasms, patterns, and morale effects).
\item Darkvision out to 18 meters.
\item Proficient with their natural weapons only.
\item Proficient with no armor.
\item Vermin breathe, eat, and sleep.
\end{itemize*}

\subsection{Subtypes}
Below are the description of each creature subtype.

\subsubsection{Air Subtype}
This subtype usually is used for elementals and outsiders with a connection to the Elemental Plane Air. Air creatures always have fly speeds and usually have perfect maneuverability.

\subsubsection{Angel Subtype}
Angels are a race of celestials, or good outsiders, native to the good-aligned Outer Planes.

\textbf{Traits:} An angel possesses the following traits (unless otherwise noted in a creature's entry).
\begin{itemize*}
\item Darkvision out to 18 meters and low-light vision.
\item Immunity to acid, cold, and petrification.
\item Resistance to electricity 10 and fire 10.
\item +4 racial bonus on saves against poison.
\item \textit{Protective Aura (Su):} Against attacks made or effects created by evil creatures, this ability provides a +4 deflection bonus to AC and a +4 resistance bonus on saving throws to anyone within 6 meters of the angel. Otherwise, it functions as a magic circle against evil effect and a lesser globe of invulnerability, both with a radius of 6 meters (caster level equals angel's HD). (The defensive benefits from the circle are not included in an angel's statistics block.)
\item \textit{Tongues (Su):} All angels can speak with any creature that has a language, as though using a \spell{tongues} spell (caster level equal to angel's Hit Dice). This ability is always active.
\end{itemize*}

\subsubsection{Aquatic Subtype}
These creatures always have swim speeds and thus can move in water without making Swim checks. An aquatic creature can breathe underwater. It cannot also breathe air unless it has the amphibious special quality.

\subsubsection{Archon Subtype}
Archons are a race of celestials, or good outsiders, native to lawful good-aligned Outer Planes.

\textbf{Traits:} An archon possesses the following traits (unless otherwise noted in a creature's entry).
\begin{itemize*}
\item Darkvision out to 18 meters and low-light vision.
\item \textit{Aura of Menace (Su):} A righteous aura surrounds archons that fight or get angry. Any hostile creature within a 6-meter radius of an archon must succeed on a Will save to resist its effects. The save DC varies with the type of archon, is Charisma-based, and includes a +2 racial bonus. Those who fail take a $-2$ penalty on attacks, AC, and saves for 24 hours or until they successfully hit the archon that generated the aura. A creature that has resisted or broken the effect cannot be affected again by the same archon's aura for 24 hours.
\item Immunity to electricity and petrification.
\item +4 racial bonus on saves against poison.
\item \textit{Magic Circle against Evil (Su):} A magic circle against evil effect always surrounds an archon (caster level equals the archon's Hit Dice). (The defensive benefits from the circle are not included in an archon's statistics block.)
\item \textit{Teleport (Su):} Archons can use \spell{greater teleport} at will, as the spell (caster level 14th), except that the creature can transport only itself and up to 50 pounds of objects.
\item \textit{Tongues (Su):} All archons can speak with any creature that has a language, as though using a \spell{tongues} spell (caster level 14th). This ability is always active.
\end{itemize*}

\subsubsection{Augmented Subtype}
A creature receives this subtype whenever something happens to change its original type. Some creatures (those with an inherited template) are born with this subtype; others acquire it when they take on an acquired template. The augmented subtype is always paired with the creature's original type. A creature with the augmented subtype usually has the traits of its current type, but the features of its original type.

\subsubsection{Chaotic Subtype}
A subtype usually applied only to outsiders native to the chaotic-aligned Outer Planes. Most creatures that have this subtype also have chaotic alignments; however, if their alignments change they still retain the subtype. Any effect that depends on alignment affects a creature with this subtype as if the creature has a chaotic alignment, no matter what its alignment actually is. The creature also suffers effects according to its actual alignment. A creature with the chaotic subtype overcomes damage reduction as if its natural weapons and any weapons it wields were chaotic-aligned (see Damage Reduction, below).

\subsubsection{Cold Subtype}
A creature with the cold subtype has immunity to cold. It has vulnerability to fire, which means it takes half again as much (+50\%) damage as normal from fire, regardless of whether a saving throw is allowed, or if the save is a success or failure.

\subsubsection{Demon Subtype}
Demons are fiends, or evil outsiders, native to chaotic evil-aligned Outer Planes.

\textbf{Traits:} Most demons possess the following traits (unless otherwise noted in a creature's entry).
\begin{itemize*}
\item Immunity to electricity and poison.
\item Resistance to acid 10, cold 10, and fire 10.
\item \textit{Summon (Sp):} Many demons share the ability to summon others of their kind (the success chance and type of demon summoned are noted in each monster description). Demons are often reluctant to use this power until in obvious peril or extreme circumstances.
\item Telepathy.
\end{itemize*}

\subsubsection{Devil Subtype}
Devils are fiends, or evil outsiders, native to lawful evil-aligned Outer Planes.

\textbf{Traits:} A devil possesses the following traits (unless otherwise noted in a creature's entry).
\begin{itemize*}
\item Immunity to fire and poison.
\item Resistance to acid 10 and cold 10.
\item \textit{See in Darkness (Su):} Some devils can see perfectly in darkness of any kind, even that created by a deeper darkness spell.
\item \textit{Summon (Sp):} Some devils share the ability to summon others of their kind (the success chance and type of devils summoned are noted in each monster description).
\item Telepathy.
\end{itemize*}

\subsubsection{Earth Subtype}
This subtype usually is used for elementals and outsiders with a connection to the Elemental Plane of Earth. Earth creatures usually have burrow speeds, and most earth creatures can burrow through solid rock.

\subsubsection{Evil Subtype}
A subtype usually applied only to outsiders native to the evil-aligned Outer Planes. Evil outsiders are also called fiends. Most creatures that have this subtype also have evil alignments; however, if their alignments change, they still retain the subtype. Any effect that depends on alignment affects a creature with this subtype as if the creature has an evil alignment, no matter what its alignment actually is. The creature also suffers effects according to its actual alignment. A creature with the evil subtype overcomes damage reduction as if its natural weapons and any weapons it wields were evil-aligned (see Damage Reduction).

\subsubsection{Extraplanar Subtype}
A subtype applied to any creature when it is on a plane other than its native plane. A creature that travels the planes can gain or lose this subtype as it goes from plane to plane. Monster entries assume that encounters with creatures take place on the Material Plane, and every creature whose native plane is not the Material Plane has the extraplanar subtype (but would not have when on its home plane). Every extraplanar creature in this book has a home plane mentioned in its description. Creatures not labeled as extraplanar are natives of the Material Plane, and they gain the extraplanar subtype if they leave the Material Plane. No creature has the extraplanar subtype when it is on a transitive plane, such as the Astral Plane, the Ethereal Plane, and the Plane of Shadow.

\subsubsection{Fire Subtype}
A creature with the fire subtype has immunity to fire. It has vulnerability to cold, which means it takes half again as much (+50\%) damage as normal from cold, regardless of whether a saving throw is allowed, or if the save is a success or failure.

\subsubsection{Goblinoid Subtype}
Goblinoids are stealthy humanoids who live by hunting and raiding and who all speak Goblin.

\subsubsection{Good Subtype}
A subtype usually applied only to outsiders native to the good-aligned Outer Planes. Most creatures that have this subtype also have good alignments; however, if their alignments change, they still retain the subtype. Any effect that depends on alignment affects a creature with this subtype as if the creature has a good alignment, no matter what its alignment actually is. The creature also suffers effects according to its actual alignment. A creature with the good subtype overcomes damage reduction as if its natural weapons and any weapons it wields were good-aligned (see Damage Reduction, above).

\subsubsection{Incorporeal Subtype}
An incorporeal creature has no physical body. It can be harmed only by other incorporeal creatures, magic weapons or creatures that strike as magic weapons, and spells, spell-like abilities, or supernatural abilities. It is immune to all nonmagical attack forms. Even when hit by spells or magic weapons, it has a 50\% chance to ignore any damage from a corporeal source (except for positive energy, negative energy, force effects such as \spell{magic missile}, or attacks made with \emph{ghost touch} weapons). Although it is not a magical attack, holy water can affect incorporeal undead, but a hit with holy water has a 50\% chance of not affecting an incorporeal creature.

An incorporeal creature has no natural armor bonus but has a deflection bonus equal to its Charisma bonus (always at least +1, even if the creature's Charisma score does not normally provide a bonus).

An incorporeal creature can enter or pass through solid objects, but must remain adjacent to the object's exterior, and so cannot pass entirely through an object whose space is larger than its own. It can sense the presence of creatures or objects within a square adjacent to its current location, but enemies have total concealment (50\% miss chance) from an incorporeal creature that is inside an object. In order to see farther from the object it is in and attack normally, the incorporeal creature must emerge. An incorporeal creature inside an object has total cover, but when it attacks a creature outside the object it only has cover, so a creature outside with a readied action could strike at it as it attacks. An incorporeal creature cannot pass through a force effect.

An incorporeal creature's attacks pass through (ignore) natural armor, armor, and shields, although deflection bonuses and force effects (such as mage armor) work normally against it. Incorporeal creatures pass through and operate in water as easily as they do in air. Incorporeal creatures cannot fall or take falling damage. Incorporeal creatures cannot make trip or grapple attacks, nor can they be tripped or grappled. In fact, they cannot take any physical action that would move or manipulate an opponent or its equipment, nor are they subject to such actions. Incorporeal creatures have no weight and do not set off traps that are triggered by weight.

An incorporeal creature moves silently and cannot be heard with Listen checks if it doesn't wish to be. It has no Strength score, so its Dexterity modifier applies to both its melee attacks and its ranged attacks. Nonvisual senses, such as scent and blindsight, are either ineffective or only partly effective with regard to incorporeal creatures. Incorporeal creatures have an innate sense of direction and can move at full speed even when they cannot see.

\subsubsection{Lawful Subtype}
A subtype usually applied only to outsiders native to the lawful-aligned Outer Planes. Most creatures that have this subtype also have lawful alignments; however, if their alignments change, they still retain the subtype. Any effect that depends on alignment affects a creature with this subtype as if the creature has a lawful alignment, no matter what its alignment actually is. The creature also suffers effects according to its actual alignment. A creature with the lawful subtype overcomes damage reduction as if its natural weapons and any weapons it wields were lawful-aligned (see Damage Reduction, above).

\subsubsection{Native Subtype}
A subtype applied only to outsiders. These creatures have mortal ancestors or a strong connection to the Material Plane and can be raised, reincarnated, or resurrected just as other living creatures can be. Creatures with this subtype are native to the Material Plane (hence the subtype's name). Unlike true outsiders, native outsiders need to eat and sleep.

\subsubsection{Reptilian Subtype}
These creatures are scaly and usually coldblooded. The reptilian subtype is only used to describe a set of humanoid races, not all animals and monsters that are truly reptiles.

\subsubsection{Shapechanger Subtype}
A shapechanger has the supernatural ability to assume one or more alternate forms. Many magical effects allow some kind of shape shifting, and not every creature that can change shape has the shapechanger subtype.

\textbf{Traits:} A shapechanger possesses the following traits (unless otherwise noted in a creature's entry).
\begin{itemize*}
\item Proficient with its natural weapons, with simple weapons, and with any weapons mentioned in the creature's description.
\item Proficient with any armor mentioned in the creature's description, as well as all lighter forms. If no form of armor is mentioned, the shapechanger is not proficient with armor. A shapechanger is proficient with shields if it is proficient with any type of armor.
\end{itemize*}

\subsubsection{Swarm Subtype}
A swarm is a collection of Fine, Diminutive, or Tiny creatures that acts as a single creature. A swarm has the characteristics of its type, except as noted here. A swarm has a single pool of Hit Dice and hit points, a single initiative modifier, a single speed, and a single Armor Class. A swarm makes saving throws as a single creature. A single swarm occupies a square (if it is made up of nonflying creatures) or a cube (of flying creatures) 10 feet on a side, but its reach is 0 feet, like its component creatures. In order to attack, it moves into an opponent's space, which provokes an attack of opportunity. It can occupy the same space as a creature of any size, since it crawls all over its prey. A swarm can move through squares occupied by enemies and vice versa without impediment, although the swarm provokes an attack of opportunity if it does so. A swarm can move through cracks or holes large enough for its component creatures.

A swarm of Tiny creatures consists of 300 nonflying creatures or 1,000 flying creatures. A swarm of Diminutive creatures consists of 1,500 nonflying creatures or 5,000 flying creatures. A swarm of Fine creatures consists of 10,000 creatures, whether they are flying or not. Swarms of nonflying creatures include many more creatures than could normally fit in a 10-foot square based on their normal space, because creatures in a swarm are packed tightly together and generally crawl over each other and their prey when moving or attacking. Larger swarms are represented by multiples of single swarms. The area occupied by a large swarm is completely shapeable, though the swarm usually remains in contiguous squares.

\textbf{Traits:} A swarm has no clear front or back and no discernable anatomy, so it is not subject to critical hits or flanking. A swarm made up of Tiny creatures takes half damage from slashing and piercing weapons. A swarm composed of Fine or Diminutive creatures is immune to all weapon damage. Reducing a swarm to 0 hit points or lower causes it to break up, though damage taken until that point does not degrade its ability to attack or resist attack. Swarms are never staggered or reduced to a dying state by damage. Also, they cannot be tripped, grappled, or bull rushed, and they cannot grapple an opponent.

A swarm is immune to any spell or effect that targets a specific number of creatures (including single-target spells such as disintegrate), with the exception of mind-affecting effects (charms, compulsions, phantasms, patterns, and morale effects) if the swarm has an Intelligence score and a hive mind. A swarm takes half again as much damage (+50\%) from spells or effects that affect an area, such as splash weapons and many evocation spells.

Swarms made up of Diminutive or Fine creatures are susceptible to high winds such as that created by a \spell{gust of wind} spell. For purposes of determining the effects of wind on a swarm, treat the swarm as a creature of the same size as its constituent creatures. A swarm rendered unconscious by means of nonlethal damage becomes disorganized and dispersed, and does not reform until its hit points exceed its nonlethal damage.

\textbf{Swarm Attack:} Creatures with the swarm subtype don't make standard melee attacks. Instead, they deal automatic damage to any creature whose space they occupy at the end of their move, with no attack roll needed. Swarm attacks are not subject to a miss chance for concealment or cover. A swarm's statistics block has ``swarm'' in the Attack and Full Attack entries, with no attack bonus given. The amount of damage a swarm deals is based on its Hit Dice, as shown in the table.

\Table{}{XC}{
  \tableheader Swarm HD
& \tableheader Swarm Base Damage \\
1--5       & 1d6 \\
6--10      & 2d6 \\
11--15     & 3d6 \\
16--20     & 4d6 \\
21 or more & 5d6 \\
}

A swarm's attacks are nonmagical, unless the swarm's description states otherwise. Damage reduction sufficient to reduce a swarm attack's damage to 0, being incorporeal, and other special abilities usually give a creature immunity (or at least resistance) to damage from a swarm. Some swarms also have acid, poison, blood drain, or other special attacks in addition to normal damage.

Swarms do not threaten creatures in their square, and do not make attacks of opportunity with their swarm attack. However, they distract foes whose squares they occupy, as described below.

\textbf{Distraction (Ex):} Any living creature vulnerable to a swarm's damage that begins its turn with a swarm in its square is nauseated for 1 round; a Fortitude save (DC 10 + \onehalf swarm's HD + swarm's Con modifier; the exact DC is given in a swarm's description) negates the effect. Spellcasting or concentrating on spells within the area of a swarm requires a Concentration check (DC 20 + spell level). Using skills that involve patience and concentration requires a DC 20 Concentration check.

\subsubsection{Water Subtype}
This subtype usually is used for elementals and outsiders with a connection to the Elemental Plane of Water. Creatures with the water subtype always have swim speeds and can move in water without making Swim checks. A water creature can breathe underwater and usually can breathe air as well.

