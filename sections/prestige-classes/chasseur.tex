\PrestigeClass{Chasseur}
{All wizards must sleep sooner or later.}{Kael Stormseeker, elven chasseur}
{
Chasseurs, or huntsmen, are wizards who specialize in hunting down others of their kind. Some chasseurs are found working for the Veiled Alliance hunting down defilers, while others are employed by the templars and the sorcerer-kings to hunt down preservers. Some do not answer to any authority at all, but have their own agendas. Whether a defiler or a preserver, a chasseur uses skill and magic to hunt down and kill or capture other wizards.
}
{d6}
{a}
{
Chasseurs are usually elf, half-elf or human, though the rare aarakocra chasseur makes a dangerous airborne hunter. Many are multi-classed characters. Ranger/wizards excel at hunting down wizards traversing the wastes, while druid/wizards make vengeful and versatile enemies of all defilers. Urban rogue/wizards who become chasseurs sometimes take levels in the assassin class to focus on striking swiftly with deadly effect.
}
{
\textbf{Skills:} \skill{Disguise} 4 ranks, \skill{Gather Information} 4 ranks, \skill{Knowledge} (arcana) 6 ranks, \skill{Spellcraft} 8 ranks, \skill{Survival} 4 ranks.

\textbf{Feats:} \feat{Blind-Fight}, \feat{Urban Tracking}.

\textbf{Spells:} Able to cast 3rd-level arcane spells.
}
{\skill{Bluff} (Cha), \skill{Concentration} (Con), \skill{Craft} (Int), \skill{Decipher Script} (Int), \skill{Diplomacy} (Cha), \skill{Disguise} (Cha), \skill{Gather Information} (Cha), \skill{Hide} (Dex), \skill{Intimidate} (Cha), \skill{Knowledge} (arcana) (Int), \skill{Knowledge} (local) (Int), \skill{Listen} (Wis), \skill{Move Silently} (Dex), \skill{Profession} (Wis), \skill{Sense Motive} (Wis), \skill{Spellcraft} (Int), \skill{Spot} (Wis), and \skill{Survival} (Wis).}
{4}
{\PrestigeSpellTable}{
1 & +1 & +0 & +2 & +2 & Mage-hunter, anti-mage spellcasting & +1 level of existing arcane spellcasting class \\
2 & +2 & +0 & +3 & +3 & \feat{Improved Counterspell}        & \\
3 & +3 & +1 & +3 & +3 & Dispelling strike 1/day             & +1 level of existing arcane spellcasting class \\
4 & +4 & +1 & +4 & +4 & Armored mage                        & \\
5 & +5 & +1 & +4 & +4 & \feat{Greater Counterspell}         & +1 level of existing arcane spellcasting class \\
}
{
\textbf{Weapon and Armor Proficiencies:} Chasseurs are proficient with all simple and martial weapons, net, light armor, and shields (except tower shields).

\textbf{Spellcasting:} At each odd-numbered level, a chasseur gains new spells per day as if he had also gained a level in whatever arcane spellcasting class in which he could cast 3rd-level spells before he added the prestige class. He does not, however, gain any other benefit a character of that class would have gained. This essentially means that he adds the level of chasseur to the level of whatever other arcane spellcasting class he has, and then determines spells per day and caster level accordingly. If he had more than one arcane spellcasting class before he became a chasseur, he must decide to which class he adds each level of chasseur for the purpose of determining spells per day.

\textbf{Mage-Hunter (Ex):} At 1st level, a chasseur can add his class level to \skill{Bluff}, \skill{Gather Information}, \skill{Listen}, \skill{Sense Motive}, \skill{Spellcraft}, \skill{Spot}, and \skill{Survival} checks when using these skills against arcane spellcasters. Likewise, he gets a +2 bonus on weapon damage rolls against such creatures.

A chasseur can also add his class level as bonus to \skill{Diplomacy} checks made with non-wizards. When dealing with arcane spellcasters or when his spellcasting is revealed this bonus transforms into penalty, and he subtracts his class level from those checks, instead.

\textbf{Anti-mage Spellcasting (Su):} At 1st level, a chasseur adds his class level to the DC of any arcane spell he casts that target another arcane spellcaster. This does not apply to spells that do not have target, or spells that target creatures other than arcane spellcasters.

\textbf{Cry Witch (Ex):} At 2nd level, a chasseur can incite the common people against a target once a week. He must show his insignia and loudly accuse the target of wizardry. All common folk and local guard will try to seize and restrain the target by all means necessary. This is a mind-affecting, language-dependent effect.

The first time a chasseur cries witch in a community, the local folk believes blindly in his word. After the first time, the local folk makes a special DC 20 Will save. The DC decreases by 2 for each cry beyond the first. The DC decreases by 5 for each person falsely accused. Treat the listeners as a single entity with a +15 Will save for this effect.

He gains another weekly use at 4th level.

\textbf{Improved Counterspell:} At 2nd level, a chasseur gains \feat{Improved Counterspell} as a bonus feat.

\textbf{Dispelling Strike (Su):} Once per day, a chasseur of 3rd level or higher can attempt a dispelling strike with one normal melee attack. If he hits, he deals normal damage, and the victim is subject to a targeted \spell{greater dispel magic}. The chasseur's dispel check is 1d20 + arcane caster level + class level. If a chasseur makes a dispelling strike against a creature with no spells or effects to dispel, the dispelling strike has no effect, but the ability is used up for that day.

\textbf{Armored Mage (Ex):} A chasseur of 4th level or higher ignores arcane spell failure chances when wearing light armor.

\textbf{Immediate Counterspell (Ex):} At 5th level, a chasseur gains \feat{Greater Counterspell} as a bonus feat.
}