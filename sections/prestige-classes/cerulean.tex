\PrestigeClass{Cerulean}
{I am the eye of the storm.}{Krekara, aarakocra cerulean}
{
Ceruleans are mages who have discovered how to draw energy for their spells from the Cerulean storm. They are explorers and researchers who have discovered a new power source and seek to exploit it. For whatever reason, cerulean wizards are not content to practice magic as the generations before them have done. Whether they sought more power, an end to the destruction of the world's plant life, an outlet for their creativity or simply craved new knowledge; the cerulean wizards have found a new way to use magic. A minority of others have been lured to power by Tithian, the one-time king of Tyr who is trapped within the storm and seeks a means to escape.
}
{d4}
{a}
{
Most races that take to wizardry can become cerulean wizards. Aarakocra have an affinity for high, cold places, and thus find the class to be of particular interest, although they are somewhat reluctant to use the storm's destructive power. Ceruleans tend to avoid the Veiled Alliance and the agents of the sorcerer-kings. Both parties fear the ceruleans can cause disaster by their tampering with the forces of the Cerulean Storm. NPC ceruleans usually have various agendas. Some seek to release Tithian from his captivity in the Cerulean Storm. Ceruleans can often be found where Tyr Storms frequently wreak havoc, especially in and around Draj.
}
{
\textbf{Alignment}: Any chaotic.

\textbf{Skills}: \skill{Concentration} 6 ranks, \skill{Craft} (optics) 4 ranks, \skill{Spellcraft} 6 ranks.

% \textbf{Feats}: \feat{Empower Spell}.

\textbf{Spells}: Able to cast 2nd level arcane spells.

\textbf{Special}: The character must craft a blue lens focus (see below).
}
{\skill{Bluff} (Cha), \skill{Concentration} (Con), \skill{Craft} (Int), \skill{Decipher Script} (Int), \skill{Disguise} (Cha), \skill{Knowledge} (all skills, taken individually) (Int), \skill{Literacy} (N/A), \skill{Profession} (Wis), \skill{Speak Language} (N/A), \skill{Spellcraft} (Int) and \skill{Survival} (Wis).}
{2}
{\PrestigeSpellTable}{
1 & +0 & +0 & +0 & +2 & Cerulean casting, blue lens focus +1 & +1 level of existing arcane spellcasting class\\
2 & +1 & +0 & +0 & +3 && +1 level of existing arcane spellcasting class\\
3 & +1 & +1 & +1 & +3 & \emph{Call lightning} 1/day & +1 level of existing arcane spellcasting class\\
4 & +2 & +1 & +1 & +4 & \feat{Empower Spell} & +1 level of existing arcane spellcasting class\\
5 & +2 & +1 & +1 & +4 && +1 level of existing arcane spellcasting class\\
6 & +3 & +2 & +2 & +5 & \emph{Call lightning} 2/day & +1 level of existing arcane spellcasting class\\
7 & +3 & +2 & +2 & +5 & Rain domain & +1 level of existing arcane spellcasting class\\
8 & +4 & +2 & +2 & +6 && +1 level of existing arcane spellcasting class\\
9 & +4 & +3 & +3 & +6 & \emph{Call lightning} 3/day & +1 level of existing arcane spellcasting class\\
10 & +5 & +3 & +3 & +7 & \emph{Control weather} 1/day & +1 level of existing arcane spellcasting class\\
}
{
\textbf{Weapon and Armor Proficiencies}: Ceruleans gain no proficiency with any weapon or armor.

\textbf{Spellcasting:} When a new cerulean level is gained, you gain new spells per day as if you had also gained a level in whatever arcane spellcasting class in which they could cast 2nd-level spells before they added the prestige class. They do not, however, gain any other benefit a character of that class would have gained. This essentially means that you add the level of cerulean to the level of whatever other arcane spellcasting class you have, and then determines spells per day and caster level accordingly. If you had more than one arcane spellcasting class before you became a cerulean, you must decide to which class you add each level of arch defiler for the purpose of determining spells per day.

\textbf{Cerulean Casting}: Ceruleans have discovered how to draw energy from the Cerulean storm to fuel their spells. They can choose whether to utilize plant energy or cerulean energy when casting spells. Cerulean energy has no impact on the environment.

\textbf{Blue Lens Focus}: A cerulean can channel the energy of the Cerulean Storm through a Blue Lens to substitute material components for arcane spells. However, drawing upon the power of the Cerulean Storm is not without its perils. Depending upon the value of the material component to be substituted, the cerulean suffers an amount of damage that cannot be redirected, absorbed or otherwise avoided. This damage is applied immediately after a spell is cast.

\Table{}{p{1cm} X R p{1cm}}{
& \tableheader Component cost & \tableheader Damage &\\
& 1 cp or less & 0 & \\
& 1-50 cp & 5 & \\
& 51-300 cp & 11 & \\
& 301-750 cp & 17 & \\
& 750+ cp & 23 & \\
}

Crafting a blue lens requires raw materials worth 50 ceramic pieces. The \skill{Craft} (optics) DC is 20. The lens has 5 hit points and hardness 1. It uses the cerulean's save values.

\textbf{Call Lightning (Sp)}: At 3rd level, a cerulean can use the \spell{call lightning} as a spell-like ability, once per day. They can use use this ability twice per day at 6th level, and three times per day at 9th level.

\textbf{Empower Spell}: At 4th level, a cerulean gains \feat{Empower Spell} as a bonus feat.

\textbf{Rain Domain}: At 7th level, a cerulean gains access to the Rain domain. They are affected by its granted power---they are automatically successful in all Fortitude saves against wind effects from natural weather and are never hit by natural lightning.

In order to learn a spell from the Rain domain, a cerulean must find or purchase a scroll of that spell and pay the usual price to scribe the spell into their spellbook. 

When preparing spells, they can select one spell slot per spell level---and only one slot per spell level---to memorize a Rain domain spell. A cerulean can never prepare more than one domain spell of each level.

\textbf{Control Weather (Sp)}: At 10th level, a cerulean can use the \spell{control weather} as a spell-like ability once per day.
}
{}
{arcane}
{}
{}
{}