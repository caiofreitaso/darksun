\PrestigeSpellTable{The Battle Dancer}{
 1st & +0 & +0 & +2 & +0 & Martial dance, inspire courage & \\
 2nd & +1 & +0 & +3 & +0 & Battle dance +1, \feat{Two-Weapon Fighting} & +1 level of existing divine spellcasting class \\
 3rd & +2 & +1 & +3 & +1 & & +1 level of existing divine spellcasting class \\
 4th & +3 & +1 & +4 & +1 & & +1 level of existing divine spellcasting class \\
 5th & +3 & +1 & +4 & +1 & Ambidexterity 1, movement mastery & \\
 6th & +4 & +2 & +5 & +2 & Battle dance +2 & +1 level of existing divine spellcasting class \\
 7th & +5 & +2 & +5 & +2 & Dance of elemental glory & +1 level of existing divine spellcasting class \\
 8th & +6 & +2 & +6 & +2 & & +1 level of existing divine spellcasting class \\
 9th & +6 & +3 & +6 & +3 & Ambidexterity 2, inspire greatness & \\
10th & +7 & +3 & +7 & +3 & Battle dance +3, swift inspiration & +1 level of existing divine spellcasting class \\
}

\PrestigeClass{Battle Dancer}
{Death to our enemies. Death to the ones who have wronged us.}{Santhaal Wind Dancer, elven chieftain}
{A battle dancer is basically a specialized ritual dancer, one who uses his or her talents not to tell stories but to defend the tribe. In elf tribes, those who decide to combine the skills of the warrior arts with priestly endeavors become battle dancers.

In times of peace, they use their skills of dance and ritual storytelling to inspire their tribes and to help their elemental singers administer to the spiritual well-being of the elves under their care. When combat opportunities present themselves, the battle dancers welcome them. They feel closest to their chosen elements when combining warrior skills and clerical magic into a deadly dance of violence.}
{d8}
{a}
{Battle dancers are always of elven blood. The vast majority are full-blooded elves, but occasionally a tribal half-elf aspires to the status of battle dancer. Battle dancers are single- or multi-classed clerics. Multi-classed clerics often take rogue or fighter levels to diversify their abilities or further augment their combat abilities.}
{
\textbf{Race:} Elf or half-elf.

\textbf{Base Attack Bonus:} +3.

\textbf{Skills:} \skill{Perform} (dance) 8 ranks, \skill{Knowledge} (religion) 6 ranks, \skill{Spellcraft} 5 ranks, \skill{Tumble} 4 ranks.

\textbf{Feats:} \feat{Blind-Fight}, \feat{Weapon Focus} (elven longblade).

\textbf{Spells:} Able to cast 2nd-level divine spells.

\textbf{Special:} Must have an element as patron.
}
{\skill{Concentration} (Con), \skill{Craft} (Int), \skill{Disguise} (Cha), \skill{Heal} (Wis), \skill{Knowledge} (history) (Int), \skill{Knowledge} (religion) (Int), \skill{Literacy} (N/A), \skill{Perform} (Cha), \skill{Profession} (Wis), \skill{Speak Language} (N/A), \skill{Spellcraft} (Int), \skill{Survival} (Wis), and \skill{Tumble} (Dex).}
{4}
{}{}
{
\textbf{Weapon and Armor Proficiencies:} Battle dancers are proficient with all simple and martial weapons. Battle dancers are proficient with light armor, but not shields.

\textbf{Spellcasting:} When a new battle dancer level is gained---except 1st and every four levels thereafter (5th and 9th)---the character gains new spells per day as if he had also gained a level in whatever divine spellcasting class in which he could cast 2nd-level spells before he added the prestige class. He does not, however, gain any other benefit a character of that class would have gained. This essentially means that he adds the level of battle dancer to the level of whatever other divine spellcasting class he has, and then determines spells per day and caster level accordingly. If he had more than one divine spellcasting class before he became a battle dancer, he must decide to which class he adds each level of battle dancer for the purpose of determining spells per day.

\textbf{Martial Dance:} Once per day per class level, a battle dancer can use his voice to product magical effects on those around his. Each ability requires a minimum battle dancer level, and some require a minimum number of ranks in \skill{Perform} (dance) to qualify.

If a battle dancer has bard levels, he can add his bard levels to his battle dancer levels to determine the number of daily uses of his martial dance, the bonus of his inspire courage, and at which level the battle dancer gains inspire greatness. For example, a 4th-level bard/3rd-level water cleric/5th-level battle dancer could use martial dance nine times per day, his inspire courage ability would grant a +2 morale bonus on the appropriate rolls, and he would be able to inspire greatness if he has the requires ranks in \skill{Perform} (dance).

\textit{Inspire Courage (Su):} A battle dancer can use his martial dance to inspire courage in his allies (including herself), bolstering them against fear and improving their combat abilities. To be affected, an ally must be able to hear the dancer perform. The effect lasts for as long as the ally hears the dancer perform and for 5 rounds thereafter. An affected ally receives a +1 morale bonus on saving throws against charm and fear effects and a +1 morale bonus on attack and weapon damage rolls. Inspire courage is a mind-affecting ability.

A battle dancer's ability to inspire courage doesn't normally improve with level. However, if a battle dancer already has this ability from another class (such as from bard levels), add together his class levels from all classes that grant this ability to determine the morale bonus granted by this ability. For example, a 2nd-level bard/3rd-level water cleric/6th-level battle dancer would grant a +2 morale bonus on the appropriate rolls.

\textit{Battle Dance (Su):} A battle dancer of 2nd level or higher with 10 or more ranks in \skill{Perform} (dance) can enter in a battle dance. The battle dance lasts a number of rounds equal to half of his ranks in \skill{Perform} (dance), rounded down. The battle dancer gains +1 insight bonus on his AC, attack and damage rolls. While in this state, the battle dancer can use inspire courage or inspire greatness as a swift action. At every four levels beyond the 2nd, this bonus increases by +1 (+2 at 6th, and +3 at 10th).

\textit{Dance of Elemental Glory (Su):} A battle dancer of 7th level or higher with 15 or more ranks in \skill{Perform} (dance) can call upon their elemental patrons for inspiration. He can prepare to cast a divine spell by adding his dance technique to the somatic components of the spell. This dance is used with the casting and increases the time to cast the spell, in order to adjust the somatic components. If a spell doesn't normally have somatic components, it gains one with this dance. Spells that take an immediate action to cast cannot be improved this way. Spells prepared with \feat{Quicken Spell}, or cast as a swift action with swift inspiration, are treated as having an original casting time of a swift action.

\Table{}{XX}{
\tableheader Original Casting Time & \tableheader New Casting Time\\
1 swift action & 1 move action \\
1 standard action & 1 full round \\
1 full round or longer & Double the time \\
}

The spell he casts with this dance gains a bonus to its caster level based on the result of the battle dancer's \skill{Perform} (dance) check:

\Table{}{XC}{
\tableheader Perform Check Result & \tableheader Caster Level Increase\\
9 or lower & +0 \\
10 to 19 & +1 \\
20 to 29 & +2 \\
30 to 39 & +3 \\
40 to 49 & +4 \\
50 or higher & +5 \\
}

\textit{Inspire Greatness (Su):} A battle dancer of 9th level or higher with 12 or more ranks in a \skill{Perform} (dance) skill can use his martial dance to inspire greatness in himself or a single willing ally within 9 meters, granting them extra fighting capability. To inspire greatness, a battle dancer must dance and an ally must hear him dance. The effect lasts for as long as the ally hears the battle dancer dance and for 5 rounds thereafter. A creature inspired with greatness gains 2 bonus Hit Dice (d10s), the commensurate number of temporary hit points (apply the target's Constitution modifier, if any, to these bonus Hit Dice), a +2 competence bonus on attack rolls, and a +1 competence bonus on Fortitude saves. The bonus Hit Dice count as regular Hit Dice for determining the effect of spells that are Hit Dice dependent. Inspire greatness is a mind-affecting ability.

A battle dancer's ability to inspire greatness is normally granted at 9th level. However, if a battle dancer already has bard levels, add together his class levels from all classes that grant this ability to determine the morale bonus granted by this ability. For example, a 2nd-level bard/3rd-level water cleric/7th-level battle dancer would be able to inspire greatness.

\textbf{Two-Weapon Fighting:} At 2nd level, a battle dancer gains \feat{Two-Weapon Fighting} as a bonus feat.

\textbf{Ambidexterity (Ex):} At 5th level, a battle dancer lessen the penalties for fighting with two weapons by 1 (from $-4$ to $-3$, or from $-2$ to $-1$ if the off-hand weapon is a light weapon). At 9th level, the penalties are lessened by another 1 (from $-3$ to $-2$, or from $-1$ to $+0$ if the off-hand weapon is a light weapon).

\textbf{Movement Mastery (Ex):} At 5th level, a battle dancer is so certain of his movements that he is unaffected by adverse conditions. When making a \skill{Jump}, \skill{Perform} (dance), or \skill{Tumble} check, he may take 10 even if stress and distraction would normally prevent him from doing so.

\textbf{Swift Inspiration (Su):} At 10th level, a battle dancer can expend one turn or rebuke undead daily attempt to cast a divine spell that has a personal range as a swift action, if he is under the effect of battle dance, inspire greatness, or inspire heroics.
}
