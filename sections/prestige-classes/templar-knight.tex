\PrestigeClass{Templar Knight}
{Halt in the name of the Lion King.}{Talon, human templar knight}
{
Templars knights are usually experienced warriors with strong personalities who serve a sorcerer-king. Their duties differ, but many lead patrols of templar guards, act as bodyguards for higher ranking templars, or are charged with missions requiring a single capable warrior and templar.

In return for their services, the templar knights receive combat enhancing powers and spells from their sorcerer-king. However, due to their focus on combat prowess, the amount and selection of spells is limited.
}
{d10}
{a}
{Any character willing to draw power and authority from his sorcerer-king can become a templar knight. The majority hail from combat-oriented classes, especially fighters, but occasionally a psychic warrior or barbarian train enough to qualify for the class.}
{
\textbf{Base Attack Bonus:} +5.

\textbf{Skills:} \skill{Diplomacy} 2 ranks.

\textbf{Special:} Must be proficient with at least one martial weapon. Must be accepted into the templarate.
}
{\skill{Bluff} (Cha), \skill{Climb} (Str), \skill{Concentration} (Con), \skill{Craft} (Int), \skill{Diplomacy} (Cha), \skill{Handle Animal} (Cha), \skill{Heal} (Wis), \skill{Intimidate} (Cha), \skill{Jump} (Str), \skill{Knowledge} (religion) (Int), \skill{Knowledge} (warcraft) (Int), \skill{Literacy} (N/A), \skill{Profession} (Wis), \skill{Ride} (Dex), \skill{Sense Motive} (Wis), and \skill{Spellcraft} (Int).}
{4}
{\HalfSpellcasterTable[6mm]}{
 1st & +1  & +2 & +0 & +0 & \feat{Secular Authority}, sigil, smite opponents 1/day & 0 &&&\\
 2nd & +2  & +3 & +0 & +0 & Fearless presence                                      & 1 &&&\\
 3rd & +3  & +3 & +1 & +1 & Bonus feat                                             & 1 & 0 &&\\
 4th & +4  & +4 & +1 & +1 &                                                        & 1 & 1 &&\\
 5th & +5  & +4 & +1 & +1 & Spell channeling, smite opponents 2/day                & 1 & 1 & 0 &\\
 6th & +6  & +5 & +2 & +2 & Bonus feat                                             & 1 & 1 & 1 &\\
 7th & +7  & +5 & +2 & +2 & Smite opponents 3/day                                  & 2 & 1 & 1 & 0 \\
 8th & +8  & +6 & +2 & +2 &                                                        & 2 & 1 & 1 & 1 \\
 9th & +9  & +6 & +3 & +3 & Bonus feat                                             & 2 & 2 & 1 & 1 \\
10th & +10 & +7 & +3 & +3 & Spell channeling (full attack), smite opponents 3/day  & 2 & 2 & 2 & 1\\
}
{
\textbf{Weapon and Armor Proficiency:} Templar knights are proficient with all martial weapons, as well as all armors and shields (including tower shields).

\textbf{Spellcasting:} You cast divine spells, which are drawn from the templar knight spell list (see below). When you gain access to a new level of spells, you automatically know all the spells for that level on the templar knight's spell list. You can cast any spell you know without preparing it ahead of time. Essentially, your spell list is the same as your spells known list.

To cast a templar knight spell, you must have a Wisdom score of 10 + the spell's level. The Difficulty Class for a saving throw against a templar knight's spell is 10 + the spell's level + the templar knight's Wis modifier. Like other spellcasters, a templar knight can cast only a certain number of spells of each level per day. The base daily allotment is given on \tabref{The Templar Knight}. In addition, you receive bonus spells for a high Wisdom score.

A templar knight need not prepare spells in advance. You can cast any spell you know at any time, assuming you have not yet used up your spells per day for that spell level.

You use your sorcerer-king's sigil as divine focus, but unlike templars, your sigil cannot be used to cast certain orisons at will.

\textbf{Secular Authority:} At 1st level, you gain \feat{Secular Authority} as a bonus feat.

\textit{Sigil} \textbf{(Sp):} You can use the spell-like powers \spell{defiler scent}, \spell{detect magic}, and \spell{slave scent} a combined total of times per day equal to 3 + your Wis modifier. These spell-like powers do not count against your total of spells per day.

\textbf{Smite Opponent (Su):} Once per day, you may attempt to smite an opponent with one normal melee attack. You add your Charisma bonus (if any) to your attack roll and deals 1 extra point of damage per templar knight level. At 5th and 10th level, you may smite opponents one additional time per day.

\textbf{Fearless Presence (Su):} Beginning at 2nd level, you are immune to fear (magical or otherwise). Each ally within 3 meters of you gains a +4 morale bonus on saving throws against fear effects. This ability functions while you are conscious, but not if you are unconscious or dead.

\textbf{Bonus Feat:} At 3rd, 6th, and 9th level, you get a bonus combat-oriented feat. These bonus feats must be drawn from the feats noted as fighter bonus feats. You must still meet all prerequisites for the bonus feat, including ability score and base attack bonus minimums as well as class requirements. You cannot choose feats that specifically require levels in the fighter class unless you are a multiclass character with the requisite levels in the fighter class.

\textbf{Spell Channeling (Su):} Beginning at 5th level, you can use a standard action to cast any touch spell you know and deliver the spell through your weapon with a melee attack. Casting a spell in this manner does not provoke an attack of opportunity. The spell must have a casting time of 1 standard action or less. If the melee attack is successful, the attack deals damage normally; then the effect of the spell is resolved. At 10th level, you can cast any touch spell you know as part of a full attack action, and the spell affects each target you hit in melee combat that round. Doing so discharges the spell at the end of the round, in the case of a touch spell that would otherwise last longer than 1 round.

\subsubsection{Templar Knight Spell List}
The templar knight spell list appears below.

\textbf{1st:} \spell{cause fear}, \spell{command}, \spell{cure light wounds}, \spell{detect magic}, \spell{divine favor}, \spell{doom}, \spell{entropic shield}, \spell{inflict light wounds}, \spell{magic weapon}, \spell{shield of faith}, \spell{true strike}.

\textbf{2nd:} \spell{bear's endurance}, \spell{bull's strength}, \spell{cure moderate wounds}, \spell{death knell}, \spell{hold person}, \spell{inflict moderate wounds}, \spell{rage}, \spell{resist energy}.

\textbf{3rd:} \spell{bestow curse}, \spell{cure serious wounds}, \spell{dispel magic}, \spell{inflict serious wounds}, \spell{invisibility purge}, \spell{magic vestment}, \spell{protection from energy}, \spell{searing light}, \spell{speak with dead}.

\textbf{4th:} \spell{cure critical wounds}, \spell{divine power}, \spell{freedom of movement}, \spell{inflict critical wounds}, \spell{greater magic weapon}, \spell{wrath of the sorcerer-king}.

\subsubsection{Ex-Templar Knights}
A templar knight who abandons his sorcerer-monarch, or whose sorcerer-monarch dies, loses all templar knight spellcasting abilities. If you later become the templar knight of another sorcerer-monarch, you immediately regain your full templar knight spellcasting abilities.
}
% {}
% {religion}
% {Oh, yeah, I have heard of them. Soldiers who got promoted to templars and now cast spells and offer to be bribed just like all templars do.}
% {Templar knights develop the ability to deliver spell damage through weapon strikes.}
% {Characters who achieve this level of success can learn important details about specific templar knights in your campaign, including notable members, the areas where they operate, and the kinds of activities they undertake.}

% PCs who wish to meet with a templar knight need
% only to go to his templarate. Although the information
% might not be shared easily, it could be easily
% accomplished with a few silvers.