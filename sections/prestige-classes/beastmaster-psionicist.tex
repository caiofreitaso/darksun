\PrestigeClass{Beastmaster Psionicist}
{Come here, little guy. Come. Let me see what you're thinking.}{Hormandar, elf beastmaster psionicist}
{The beastmaster is a psionicist with an affinity for animals of all kinds. He can calm a raging mekillot with a few soothing words, or turn the most loyal guard creatures to his side. He is an outsider who is more comfortable with his animal charges than with any humanoid companionship.

Most beastmasters are not natives of the cities, but instead come from small villages or nomadic tribes---a background in which the young psionicist has extensive experience and contact with the animals he understands so well.}
{d8}
{a}
{Beastmasters tend to be multi-classed psychic warrior, with levels on either druid or ranger---they share the love of the natural beasts with these classes. There are also single-classed psychic warriors or telepaths, but these are rarer.}
{
\textbf{Base Attack Bonus:} +3.

\textbf{Skills:} \skill{Handle Animal} 4 ranks, \skill{Knowledge} (nature) 2 ranks, \skill{Knowledge} (psionics) 8 ranks, \skill{Survival} 4 ranks.

\textbf{Feat:} \feat{Endurance}.

\textbf{Powers:} Must be able to manifest at least two telepathy powers.

\textbf{Psionics:} Manifester level 4th.
}
{
\skill{Autohypnosis} (Wis), \skill{Concentration} (Con), \skill{Craft} (Int), \skill{Disable Device} (Int), \skill{Handle Animal} (Cha), \skill{Knowledge} (nature) (Int), \skill{Knowledge} (psionics) (Int), \skill{Psicraft} (Int), \skill{Ride} (Dex), \skill{Search} (Int), \skill{Survival} (Wis), and \skill{Use Rope} (Dex).
}
{4}
{\PrestigePowerTable}{
1st  & +1 & +0 & +2 & +2 & Animal companion, trapfinding, wild empathy & +1 level of existing manifesting class\\
2nd  & +2 & +0 & +3 & +3 & \emph{Animal mindlink}, uncanny dodge & \\
3rd  & +3 & +1 & +3 & +3 & \feat{Track}, wasteland trap sense +1 & +1 level of existing manifesting class\\
4th  & +4 & +1 & +4 & +4 & Evasion, \emph{speak with animals} & \\
5th  & +5 & +1 & +4 & +4 & Improved uncanny dodge & +1 level of existing manifesting class\\
6th  & +6 & +2 & +5 & +5 & \emph{Animal suggestion}, wasteland trap sense +2 & \\
7th  & +7 & +2 & +5 & +5 & Swift tracker & +1 level of existing manifesting class\\
8th  & +8 & +2 & +6 & +6 & \emph{Animal dominate} & \\
9th  & +9 & +3 & +6 & +6 & Wasteland trap sense +3 & +1 level of existing manifesting class\\
10th & +10 & +3 & +7 & +7 & Practiced telepath & \\
}
{
\textbf{Weapon and Armor Proficiency:} Beastmaster psionicists gain proficiency in all simple weapons, plus bolas, lasso, net, and whip. Beastmaster psionicists are proficient with light and medium armors, but not shields.

\textbf{Manifesting:} At every odd-numbered level, a beastmaster psionicist gains additional power points per day and access to new powers as if he had also gained a level in whatever manifesting class he belonged to before he added the prestige class. He does not, however, gain any other benefit a character of that class would have gained (bonus feats, metapsionic or item creation feats, and so on). This essentially means that he adds the level of beastmaster psionicist to the level of whatever manifesting class the character has, then determines power points per day, powers known, and manifester level accordingly.

If a character had more than one manifesting class before he became a beastmaster psionicist, he must decide to which class he adds the new level of beastmaster psionicist for the purpose of determining power points per day, powers known, and manifester level.

\textbf{Animal Companion (Ex):} A beastmaster psionicist gains the service of a loyal animal companion. See the ranger class feature. Treat the beastmaster psionicist as a ranger whose level is equal to the beastmaster psionicist's class level + 6. A beastmaster psionicist can select one of the animals available to a 4th-level ranger and then apply the modifications as appropriate for a 7th-level ranger's animal companion, or he can select a typical version of one of the animals available to a 7th-level ranger.

As a beastmaster psionicist gains class levels, his animal companion gains Hit Dice and other special abilities just as a ranger's animal companion does. Use the beastmaster psionicist's class level + 6 to determine the animal companion's special abilities.

If a beastmaster psionicist already has an animal companion from another class, his beastmaster psionicist class levels stack with class levels from all other classes that grant an animal companion. For example, a 4th-level ranger/4th-level psychic warrior/2nd-level beastmaster psionicist would be treated as a 12th-level ranger for the purpose of improving the statistics of his animal companion (and which alternative animal companions he could select).

\textbf{Trapfinding (Ex):} A beastmaster psionicist can find, disarm, or bypass traps with a DC of 20 or higher. He can use the \skill{Search} skill to find, and the \skill{Disable Device} skill to disarm, magic traps (DC 25 + the level of the spell used to create it). If his \skill{Disable Device} result exceeds the trap's DC by 10 or more, he discovers how to bypass the trap without triggering or disarming it.

\textbf{Wild Empathy (Ex):} A beastmaster psionicist can improve the attitude of an animal. See the druid class feature. If the beastmaster psionicist has wild empathy from another class, his levels stack for determining the bonus.

\textit{Animal Mindlink (Ps):} At 2nd level, a beastmaster psionicist adds \emph{animal mindlink} to his repertoire. It functions exactly like \psionic{mindlink} but it can only target animals or magical beasts. Once per day, he can manifest \emph{animal mindlink} at a reduced power point cost. The cost of \emph{animal mindlink} is reduced by the beastmaster psionicist's class level, to a minimum of 1 power point. The effect of this power is still restricted by the beastmaster psionicist's manifester level.

\textbf{Track:} At 3rd level, a beastmaster psionicist gains \feat{Track} as a bonus feat.

\textbf{Uncanny Dodge (Ex):} Starting at 3rd level, a beastmaster psionicist retains his Dexterity bonus to AC (if any) regardless of being caught flat-footed or struck by an invisible attacker. (He still loses any Dexterity bonus to AC if immobilized.)

If a character gains uncanny dodge from a second class the character automatically gains improved uncanny dodge (see below).

\textbf{Wasteland Trap Sense (Ex):} At 3rd level, a beastmaster psionicist is adept at evading the effects of traps and natural hazards. See the barbarian class feature. The bonuses rise by 1 for every three additional beastmaster psionicist levels gained (+2 at 6th level, +3 at 9th level) and stack with similar bonuses granted by other classes.

\textbf{Evasion (Ex):} At 4th level, a beastmaster psionicist gains the evasion ability. If he makes a successful Reflex saving throw against an attack that normally deals half damage on a successful save, he instead takes no damage. Evasion can be used only if the beastmaster psionicist is wearing light armor or no armor. A helpless beastmaster psionicist (such as one who is unconscious or paralyzed) does not gain the benefit of the ability.

If the character already has the evasion ability, he gains improved evasion instead. Improved evasion works like evasion, except that while the character still takes no damage on a successful Reflex saving throw, he takes only half damage on a failed save.

\textit{Speak with Animals (Sp):} At 4th level, a beastmaster psionicist can use \spell{speak with animals} as the spell once per day per beastmaster psionicist level. His caster level is equal to his class level.

\textbf{Improved Uncanny Dodge (Ex):} At 5th level, a beastmaster psionicist can no longer be flanked, since he can react to opponents on opposite sides of him as easily as he can react to a single attacker. This defense denies rogues the ability to use flank attacks to sneak attack the beastmaster psionicist. The exception to this defense is that a rogue at least four levels higher than the beastmaster psionicist can flank him (and thus sneak attack him).

If a character gains uncanny dodge (see above) from a second class the character automatically gains improved uncanny dodge, and the levels from those classes stack to determine the minimum rogue level required to flank the character.

\textit{Animal Suggestion (Ps):} At 6th level, a beastmaster psionicist adds \emph{animal suggestion} to his repertoire. It functions exactly like \psionic{psionic suggestion} but it can only target animals or magical beasts. Once per day, he can manifest \emph{animal suggestion} at a reduced power point cost. The cost of \emph{animal suggestion} is reduced by the beastmaster psionicist's class level, to a minimum of 1 power point. The effect of this power is still restricted by the beastmaster psionicist's manifester level.

\textbf{Swift Tracker (Ex):} Beginning at 7th level, a beastmaster psionicist can move at his normal speed while following tracks. See the ranger class feature.

\textit{Animal Dominate (Ps):} At 8th level, a beastmaster psionicist adds \emph{animal dominate} to his repertoire. It functions exactly like \psionic{psionic dominate} with the following exceptions. \emph{Animal dominate} can only target animals or magical beasts, without needing to augment the power by 2 power points. \emph{Animal dominate} cannot be augmented to affect other creature types.

\textbf{Practiced Telepath (Ex):} At 10th level, a beastmaster psionicist gains access to the telepath psionic discipline, if he doesn't already have it. He adds one power of that discipline for each level he can manifest to his repertoire. Whenever he gains access to a new power level, he may add one power of that psionic discipline to his repertoire.

}