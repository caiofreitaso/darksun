\PrestigeClass{Dune Trader}
{Don't trust anything with pointy ears. It will either cheat you or try to eat you.}{Marek, human trader}
{Wagons pulled by mekillots and kanks travel along dusty roads, carrying slaves, weapons, food and other goods between the city-states and the villages of the wastes. Caravans of all sizes trek across the Tablelands and beyond, owned by powerful Merchant Houses. Trade ensures the survival of many small villages and is vital to the economy of the city-states of the sorcerer-kings. Dune traders are agents for the merchant houses. Some even aspire to become invited family members of the ancient merchant dynasties.}
{d6}
{a}
{Because of the requirements for entry, the dune trader can appeal to a wide range of characters. Rogues, bards, rangers, and other skill-focused characters are the most likely to enter the prestige class, but the entry requirements are well within the reach of intelligent members of any class.}
{
\textbf{Skills:} \skill{Appraise} 5 ranks, \skill{Bluff} 5 ranks, \skill{Diplomacy} 7 ranks, \skill{Profession} (merchant) 2 ranks, \skill{Sense Motive} 5 ranks.

\textbf{Feats:} \feat{Trader}.

\textbf{Special:} Must be accepted into a merchant house.
}
{
\skill{Appraise} (Int), \skill{Bluff} (Cha), \skill{Craft} (Int), \skill{Decipher Script} (Int), \skill{Diplomacy} (Cha), \skill{Disguise} (Cha), \skill{Forgery} (Int), \skill{Gather Information} (Cha), \skill{Hide} (Dex), \skill{Intimidate} (Cha), \skill{Listen} (Wis), \skill{Move Silently} (Dex), \skill{Open Lock} (Dex), \skill{Profession} (Wis), \skill{Ride} (Dex), \skill{Search} (Int), \skill{Sense Motive} (Wis), \skill{Sleight of Hand} (Dex), \skill{Speak Language} (N/A), and \skill{Spot} (Wis).
}
{8}
{\PrestigeWarriorTable}{
1 & +0 & +0 & +0 & +2 & Fast talk\\
2 & +1 & +0 & +0 & +3 & Open arms, contact 1/week\\
3 & +2 & +1 & +1 & +3 & Bargaining 10\%/+2\\
4 & +3 & +1 & +1 & +4 & Dazzle (1 creature), linguist\\
5 & +3 & +1 & +1 & +4 & Agent\\
6 & +4 & +2 & +2 & +5 & Improved fast talk, contact 2/week\\
7 & +5 & +2 & +2 & +5 & Dazzle (2 creatures), bargaining 20\%/+4\\
8 & +6 & +2 & +2 & +6 & Taunt\\
9 & +6 & +3 & +3 & +6 & Allies\\
10 & +7 & +3 & +3 & +7 & Contact 3/week, dazzle (3 creatures)}
{
\textbf{Weapon and Armor Proficiency:} Dune traders gain no proficiency with any weapon or armor.

\textbf{Fast Talk (Ex):} You can retry \skill{Bluff} and \skill{Diplomacy} checks at a $-5$ penalty, but only once per check.

\textbf{Open Arms:} Beginning at 2nd level, a dune trader becomes skilled at initiating peaceful (and not so peaceful) negotiations. The dune trader adds a competence bonus equal to one-half of his dune trader level (rounded down) on all \skill{Diplomacy} checks.

\textbf{Contact:} At 2nd level, dune traders gain the privilege of acquaintances that will do favors for them once per week. The DM has final say on the extent of favors that may be extracted. The following list provides sample uses of contact.

\begin{itemize*}
\item Additional 5\% discount on purchased goods.
\item Access to purchase and sell black market goods.
\item Access to hire mercenary of trader's desired race and class (see Hirelings).
\item Access to purchasing spellcasting services.
\item Access to information (equal to \skill{Gather Information} DC 20).
\item Access to forged materials (equal to \skill{Forgery} DC 20).
\item Access to decipher (equal to \skill{Decipher Script} DC 20).
\item Access to other expert (skill check DC 20, at DM's discretion).
\item Appointment or meeting with an NPC (templar, noble, gladiatorial slave, chieftain, etc. At DM's discretion).
\item Access to a place to stay hidden for three days.
\item Avoid templar inspection.
\end{itemize*}

A dune trader gains a new weekly use of contact at 6th and 10th levels.

\textbf{Bargaining:} At 3rd level, a dune trader can purchase goods with a 10\% rebate off the listed price. This stacks with the discount granted by Contact above. Agents of the major merchant houses listed below gain a +2 circumstance bonus to a skill embedded in the merchant house's culture and organization. Agents of other, smaller houses gain a +1 circumstance bonus to a skill of their choice.

\textit{Inika:} \skill{Gather Information}

\textit{M'ke:} \skill{Sense Motive}

\textit{Shom:} \skill{Bluff}

\textit{Stel:} \skill{Knowledge} (warcraft)

\textit{Tsalaxa:} \skill{Intimidate}

\textit{Vordon:} \skill{Appraise}

\textit{Wavir:} \skill{Diplomacy}

\textit{Other:} Skill of choice

At 7th level, the discount increases to 20\%, major merchant bonus increase to +4, and small house bonus incresce to +2.

\textbf{Dazzle (Ex):} At 4th level, a dune trader gains the ability to dazzle a creature through sheer force of personality, a winning smile, and fast-talking. The creature to be fascinated must be within 27 meters, able to see, hear and understand the dune trader, and able to pay attention to him. The dune trader must also be able to see the creature. The distraction of nearby combat or other dangers prevents the ability from working.

As a move action, the dune trader makes an opposed Bluff check. If he succeeds on the check, the creature receives a $-1$ penalty on attack rolls, ability checks, skill checks, and saving throws for a number of rounds equal to his dune trader level.

At 7th and 10th level, a dune trader can target one additional creature with a single use of this ability.

This is a mind-affecting, language-dependent ability.

\textbf{Linguist:} At 4th level, a dune trader becomes a master linguist. Whenever he encounters a new language, either spoken or written, that he does not know, he can make an Intelligence check (for a spoken language) or a \skill{Decipher Script} check (for a written language) to determine if he can understand it.

The DC for the check is DC 15 if the language is commonly spoken or DC 25 if the language is ancient or unique. Success means he can glean enough meaning from a conversation or document to ascertain the basic message, but this ability in no way simulates actually being able to converse or fluently read and write in a given language. A single check covers roughly one minute of a spoken language or one page of a written language.

\textbf{Agent:} At 5th level, a dune trader gains a cohort as per the \feat{Leadership} feat. His Leadership score is his level plus his Charisma bonus. If he possesses the \feat{Leadership} feat, he is entitled to two cohorts. In the case of multiple cohorts, their combined level may not exceed his character level + dune trader level (add his dune trader level twice).

\textbf{Improved Fast Talk (Ex):} Beginning at 5th level, a dune trader may make a rushed \skill{Diplomacy} check as a full-round action with a $-5$ penalty instead of the normal $-10$ penalty.

\textbf{Taunt (Ex):} At 8th level, a dune trader gains the ability to temporarily rattle a creature through the use of insults and goading. As a move action, he may taunt a target able to see, hear and understand him, with an Intelligence score of 3 or higher. The opponent resists the taunt by making a Will saving throw (DC 10 + his dune trader level + his Cha modifier). If the save fails, he is the only creature it can make melee attacks against for 1 round.

This is a mind-affecting, language-dependent ability.

\textbf{Allies:} Beginning at 9th level, a dune trader gains the favor of an organization, tribe, planar creature or a powerful individual---in the most extreme case, a sorcerer-king. The frequency and extent of favors a trader may call upon will vary (for example, spending the night under the protected tents of an Elven tribe is a small favor, while asking for a caravan raid in which several tribe members will perish is a large favor). The DM determines how often the trader can call upon his allies for aid without losing their favor. The maximum monetary value of the favor cannot exceed 1,000 cp.
}
{}
{local}
{Dune traders? Oh, bless them, by all the elements! If it were not for the traders, we would surely perish in a season!}
{Dune traders are masters of glibness, commerce, and information.}
{Characters who achieve this level of success can learn important details about specific dune traders, in your campaign, including a notable individual, the area in which he operates, and the kinds of activities he undertakes.}

PCs who try to establish contact with a dune trader should make a DC 15 Gather Information check to find a House Emporium, through which contact can be arranged, or a DC 20 Gather information to track a dune trader down directly. If the PCs are trying to make a deal with a dune trader, give them a +2 circumstance bonus on the check.