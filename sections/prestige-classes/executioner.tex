\PrestigeClass{Executioner}
{To find Harmony, the people must understand that the King has all power and that the people have none. The spectacle of public execution teaches that the King has power over life and death. The executioner represents the King, while the perpetrator represents the people.}{excerpt from Finding Harmony in these Troubled Times, T'karei Khala, Haleban High Templar}
{While death for entertainment and public executions are far from uncommon in the Tyr region, only in Eldaarich has public execution become the predominant form of performance art. The Haleban encourages this trend, believing executions bring harmony to society by teaching proper fear of the King.

Executioners receive many privileges within Eldaarich: an honorary exception to the laws forbidding persons to wear armor and carry weapons in the city, and some are even allowed the privilege of disguising themselves as King Daskinor while carrying out an arena execution.}
{d12}
{an}
{Most executioners begin as either fighters or gladiators. Fighters are often drawn to the melee aspect of it. Most gladiators who become executioners do so because they just want the crowd around while they kill someone.}
{\textbf{Alignment:} Lawful evil.

\textbf{Skills:} \skill{Perform} (acting or oratory) 8 ranks, \skill{Heal} 4 ranks.

\textbf{Feats:} \feat{Weapon Finesse}.

\textbf{Special:} Must have composed and popularized a new style of execution; must have executed someone to entertain a crowd of at least 100 people. Must be approved by the Haleban Order.}
{
\skill{Bluff} (Cha), \skill{Craft} (Int), \skill{Diplomacy} (Cha), \skill{Disguise} (Cha), \skill{Escape Artist} (Dex), \skill{Handle Animal} (Cha), \skill{Heal} (Wis), \skill{Intimidate} (Cha), \skill{Jump} (Str), \skill{Perform} (Cha), \skill{Ride} (Dex), \skill{Sense Motive} (Wis), \skill{Sleight of Hand} (Dex), \skill{Spot} (Wis), \skill{Tumble} (Dex), and \skill{Use Rope} (Dex).
}
{4}
{\PrestigeWarriorTable}{
1st & +1 & +2 & +0 & +0 & Status, exact agony\\
2nd & +2 & +3 & +0 & +0 & Gruesome trophy\\
3rd & +3 & +3 & +1 & +1 & Crippling strike\\
4th & +4 & +4 & +1 & +1 & Die again\\
5th & +5 & +4 & +1 & +1 & Exact status, live to die another day
}
{
\textbf{Weapon and Armor Proficiency:} Executioners are proficient with all axes, bard's garrote, net, and lasso.

\textbf{Status (Ex):} By making a move-equivalent action and a \skill{Spot} check (DC 15 or the creature's \skill{Bluff} check, whichever is higher), you can discern the conditions affecting any one living creature within 3 meters of you. Executioners regard this ability as a sacred mystery, and would never share this information with any other person, including allies and superiors.

\textbf{Exact Agony (Su):} You can set a damage cap for the damage on your weapon damage rolls. Regardless of the die roll, the victim will not take more than the designated cap damage. Additionally, you can choose to deal nonlethal damage against one target rather than lethal damage, with either weapons, spells, or psionic powers, without any attack roll penalty, higher spell slot or additional power point expenditure (anyone other than the specified target that is affected by the attack takes lethal damage as normal).

\textbf{Gruesome Trophy (Sp):} Beginning at 2nd level, you learn the mysterious Eldaarish craft of shrinking heads, and you're able to create magical shrunken heads from enemies you have personally executed, even if you're not  a spellcaster. Additionally, while displaying the head of one of your victims, you gain a circumstance bonus on \skill{Intimidate} checks equal to you executioner level, against anyone who witnessed the execution or who knew the victim.

\textbf{Crippling Strike (Ex):} This is exactly like the rogue special ability of the same name, except no sneak attack is required and it only inflicts 1 Str damage. Hence, the executioner does not need to deny the target's Dex bonus in order to make the crippling strike.

\textbf{Die Again (Sp):} You can animate any single deceased character within 9 meters as a swift action into a zombie. This ability otherwise works like the animate dead spell, except you can't have more HD of undead than twice your executioner level and no material component is needed. Executioners often use this ability to give the appearance that the executed person has gotten back up and is attacking him from behind, so that the executioner can whirl around and hack the body's head off right  before it the zombie strikes him, electrifying the crowd.

\textbf{Exact Status (Ex):} At 5th level, you can use your status ability to identify the exact hit points left and nonlethal damage of any one living creature. Executioners use this ability, in conjunction with the exact agony ability, to make it appear that they have absolute power over life and death. Obviously, executioners regard this ability as sacred as the exact agony one. Any executioner who discloses information learned through this ability loses all executioner abilities (but not weapon proficiencies). He may not progress any farther in levels as an executioner. He regains his abilities and advancement potential if he atones for his violations (see the atonement spell description) to a Haleban templar, as appropriate.

\textbf{Live to Die Another Day (Sp):} At 5th level, you can produce a \spell{raise dead} effect, as the spell, once per week. The recipient must have died within the last 5 minutes for the ability to be successful.
}
% {}
% {local [Eldaarich]}
% {Executioners are some kind of gladiator that always seem to know when his victims are about to die.}
% {This is a much-demanded occupation in Eldaarich and most dens in the city host executions.}
% {The most experienced executioners are able to bring their victims back from the Grey, only to ruthless execute them again.}