\PrestigeClass{Elemental Champion}
{The very wind screams for vengeance. Who am I not to listen?}{Grall, half-elf elemental champion}
{Rare warriors devoted to the elemental powers of this dying world, elemental champions are created for a single purpose---to protect the waning vitality of their elemental patron at any cost. Chosen from among the strongest warriors, elemental champions are a desperate measure taken against the continued devastation of the natural world. Imbued with the power of their patrons, elemental champions protect the few remaining pristine places on Athas and often serve as bodyguards or agents for clerics who serve their chosen element. The main goal of every champion though, is the execution of defilers. Each and every defiler they leave bloodied and broken in their wake is one less parasite draining away the essence of their patron, and some champions take this ideology even farther to extend their wrath to all wizards. With foes like these, however, few champions live long enough to truly make a difference.

Not all champions despise wizardry though, as the paraelemental powers have their champions as well.
Although they are very rare, paraelemental champions serve the same and opposite roles of the elemental counterparts, guarding and serving paraelemental clerics and defilers while seeking to expand the influence of their twisted masters. Paraelemental champions are otherwise the same as elemental champions in every respect except their allegiances.}
{d10}
{an}
{Almost all elemental champions come from the ranks of the warrior classes, but fighters make up the bulk of the class due to the ease that they can be the prerequisites. Barbarians and rangers also make excellent elemental champions. Of the common races, dwarves, mul, and half-giants are often chosen to become elemental champions due to their sheer strength and endurance, although a half-giant's notoriously feeble mentality can be a significant problem. Halflings are also common as elemental champions, particularly in the Forest Ridge.}
{
\textbf{Base attack bonus:} +6.

\textbf{Skills:} \skill{Knowledge} (religion) 3 ranks, \skill{Knowledge} (warcraft) 3 ranks.

\textbf{Feats:} \feat{Diehard}, \feat{Endurance}, \feat{Power Attack}.

\textbf{Special:} A prospective elemental champion must ritually petition his chosen elemental patron in order to take up the mantle of a champion of the element.
}
{\skill{Concentration} (Con), \skill{Craft} (Int), \skill{Heal} (Wis), \skill{Intimidate} (Cha), \skill{Knowledge} (religion) (Int), \skill{Knowledge} (the planes) (Int), \skill{Knowledge} (warcraft) (Int), \skill{Profession} (Wis), \skill{Ride} (Dex), \skill{Spellcraft} (Int), \skill{Survival} (Wis).}
{2}
{\HalfSpellcasterTable}{
1st & +1 & +2 & +0 & +2 & Defiler scent, smite 1/day & 0 &&&\\
2nd & +2 & +3 & +1 & +3 & Domain, elemental blessing & 1 &&&\\
3rd & +3 & +3 & +1 & +3 & Mettle & 1 & 1 &&\\
4th & +4 & +4 & +1 & +4 & Smite 2/day & 1 & 1 &&\\
5th & +5 & +4 & +2 & +4 & Aura of sacrifice & 1 & 1 & 1 &\\
6th & +6 & +5 & +2 & +5 & Bonus feat & 2 & 1 & 1 &\\
7th & +7 & +5 & +2 & +5 & Smite 3/day & 2 & 1 & 1 & 1 \\
8th & +8 & +6 & +3 & +6 & & 2 & 1 & 1 & 1 \\
9th & +9 & +6 & +3 & +6 & Bonus feat & 2 & 2 & 1 & 1 \\
10th & +10 & +7 & +3 & +7 & Smite 4/day & 2 & 2 & 2 & 1\\
}
{
\textbf{Weapon and Armor Proficiency:} Elemental champions are proficient in all simple and martial weapons. They are also proficient with all light and medium armor, and with shields (but not tower shields).

\textbf{Spellcasting:} An elemental champion has the ability to cast a small number of divine spells. To cast an elemental champion spell, you must have a Wisdom score of at least 10 + the spell's level, so an elemental champion with a Wisdom of 10 or lower cannot cast these spells.

Elemental champion bonus spells are based on Wisdom, and saving throws against these spells have a DC of 10 + spell level + your Wisdom modifier. When you get 0 spells per day of a given spell level you gain only the bonus spells you would be entitled to based on your Wisdom score for that spell level. The elemental champion's spell list appears below. An elemental champion has access to any spell on the list and can freely choose which to prepare, just as a cleric. An elemental champion prepares and casts spells just as a cleric does (though an elemental champion cannot spontaneously cast cure or inflict spells).

\textbf{Defiler Scent (Sp):} At 1st level, you gain the ability to use the \spell{defiler scent} spell at will as a spell-like ability.

\textbf{Smite (Su):} Once a day, an elemental champion of 1st level or higher may attempt to smite a creature with directly opposed alignment with one normal melee attack. You adds your Charisma modifier (if positive) to your attack roll and deals 1 extra point of damage per elemental champion level. If you accidentally smite a creature that is of opposite alignment, the smite has no effect but it is still used up for that day. At 4th level, 7th and 10th level, you may smite one additional time per day.

\textbf{Domain:} At 2nd level, an elemental champion receives one domain from the domains available to those who follow your patron element. You gain the domain's class skill(s), weapon and armor proficiencies, and granted power. You also can prepare and cast spells for the chosen domain in addition to the spells on the elemental champion spell list, up to the maximum spell level you can cast. You do not receive any additional spell slots due to this ability, however.

\textbf{Elemental Blessing (Su):} You apply your Charisma modifier (if positive) as a bonus on all saving throws.

\textbf{Mettle (Ex):} At 3rd level, you gain the ability to resist magical compulsions and other unusual attacks with uncommon willpower and stamina. If you make a successful Fortitude or Will save against an attack that would normally have a lesser effect with a successful save (such as a spell with a saving throw entry of Will half or Fortitude partial), the effect is instead completely negated. This ability only works as long as you are conscious, however.

\textbf{Aura of Sacrifice (Su):} By 5th level, your power has grown strong enough that you can protect the natural world from the ravages of defiling magic. If someone defiles within 9 meters of you, you can react to protect the land through sacrificing part of your own life force. This nullifies a wizard's defiling radius and any effects it entails, including those of Raze feats. You lose 1 hit point per 1.5 meter of defiling radius nullified.

\textbf{Bonus Feat:} You receive a bonus feat at 6th and 9th level. This feat can be chosen only from the fighter bonus feat list.

\subsubsection{Elemental Champion Spell List}
Elemental champions choose their spells from the following list:

\textbf{1st level:} \spell{bane}, \spell{bless}, \spell{bless element}, \spell{cause fear}, \spell{create element}, \spell{cure light wounds}, \spell{curse element}, \spell{detect poison}, \spell{divine favor}, \spell{endure elements}, \spell{guidance}, \spell{inflict light wounds}, \spell{magic weapon}, \spell{purify food and drink}, \spell{remove fear}, \spell{resistance}, \spell{shield of faith}, \spell{true strike}.

\textbf{2nd level:} \spell{align weapon}, \spell{augury}, \spell{bear's endurance}, \spell{bull's strength}, \spell{cat's grace}, \spell{cure moderate wounds}, \spell{eagle's splendor}, \spell{inflict moderate wounds}, \spell{resist energy}.

\textbf{3rd level:} \spell{breathing}, \spell{cure serious wounds}, \spell{dispel magic}, \spell{elemental armor}, \spell{elemental weapon}, \spell{greater magic weapon}, \spell{haste}, \spell{inflict serious wounds}, \spell{keen edge}, \spell{surface walk}.

\textbf{4th level:} \spell{bestow curse}, \spell{cure critical wounds}, \spell{death ward}, \spell{divination}, \spell{divine power}, \spell{freedom of movement}, \spell{inflict critical wounds}.

\subsubsection{Ex-Elemental Champions}
An elemental champion who defiles (even if he only defiles once) loses all elemental champion spells and abilities (but not weapon, armor, and shield proficiencies). He may not progress any farther in levels as an elemental champion. You regain your abilities and advancement potential if you atone for your violation (see the atonement spell description), as appropriate. Paraelemental champions, on the other hand, do not suffer from this restriction.
}
{}
{religion}
{}
{}
{}