\PrestigeClass{Grove Master}
{Like the silk wyrm, I strike at night under the guise of the moons. Like the tembo, I wear down my enemy, and like the dune reaper I hold no remorse.}{Jak, dwarven grove master}
{Grove masters are experienced druids devoted to protecting a certain area from the destruction of defilers and other intruders that would harm the land. In return for their devotion, the Spirit of the Land bestows additional powers on the grove masters. Grove masters are formidable enemies while on their guarded lands.

Most grove masters allow travelers free passage as long as they do not harm the ecosystem, but some view all as intruders and enemies. The latter are typically grove masters who have sworn to protect an endangered plant or species from eradication, and will go to any lengths to do so.}
{d8}
{a}
{Druids are natural candidates for this class, but members of this prestige class can come from a variety of backgrounds. A grove master is chosen more for his devotion to the wilderness than for any specific skill he possesses. Sometimes a very dedicated ranger manages to gain the trust of a Spirit of the Land and become a grove master.}
{
\textbf{Skills:} \skill{Knowledge} (nature) 10 ranks, \skill{Survival} 5 ranks, \skill{Hide} 4 ranks.

\textbf{Feats:} \feat{Wastelander}.

\textbf{Spells:} Able to cast 3rd‐level divine spells.

\textbf{Special:} Must gain the trust of a Spirit of the Land.
}
{\skill{Concentration} (Con), \skill{Craft} (Int), \skill{Diplomacy} (Cha), \skill{Disguise} (Cha), \skill{Handle Animal} (Cha), \skill{Heal} (Wis), \skill{Hide} (Dex), \skill{Knowledge} (nature) (Int), \skill{Listen} (Wis), \skill{Move Silently} (Dex), \skill{Profession} (Wis), \skill{Ride} (Dex), \skill{Spellcraft} (Int), \skill{Spot} (Wis), and \skill{Survival} (Wis).
}
{4}
{\PrestigeSpellTable}{
1 & +0 & +0 & +0 & +2 & Animal companion, wild shape, guarded lands, sacrifice & +1 level of existing divine spellcasting class\\
2 & +1 & +0 & +0 & +3 & Smite intruder 1/day & +1 level of existing divine spellcasting class \\
3 & +1 & +1 & +1 & +3 & Sustenance, \emph{invisibility} 1/day & +1 level of existing divine spellcasting class\\
4 & +2 & +1 & +1 & +4 & \emph{Teleport} 1/day & +1 level of existing divine spellcasting class \\
5 & +2 & +1 & +1 & +4 & Nondetection & +1 level of existing divine spellcasting class \\
6 & +3 & +2 & +2 & +5 & \emph{Invisibility} 2/day & +1 level of existing divine spellcasting class \\
7 & +3 & +2 & +2 & +5 & Smite intruder 2/day & +1 level of existing divine spellcasting class \\
8 & +4 & +2 & +2 & +6 & \emph{Greater teleport} 2/day & +1 level of existing divine spellcasting class \\
9 & +4 & +3 & +3 & +6 & \emph{Greater\ invisibility} 1/day & +1 level of existing divine spellcasting class \\
10 & +5 & +3 & +3 & +7 & Timeless body & +1 level of existing divine spellcasting class\\
}
{
\textbf{Spellcasting:} When a new grove master level is gained, you gain new spells per day as if you had also gained a level in whatever divine spellcasting class you belonged to before you added the prestige class. You do not, however, gain any other benefit a character of that class would have gained. This essentially means that you add the level of grove master to the level of whatever other divine spellcasting class you have, and then determines spells per day and caster level accordingly.

If you had more than one divine spellcasting class before you became a grove master, you must decide to which class you add each level of grove master for the purpose of determining spells per day.

\textbf{Animal Companion:} Your effective druid levels stack with your grove master levels for the purpose of determining the abilities of your animal companion.

\textbf{Wild Shape (Su):} Your druid levels stack with your grove master levels for the purpose of determining the number of daily uses, the maximum Hit Dice, size (but not creature type), and the duration of your wild shape ability.

\textbf{Guarded Lands (Su):} You choose a single area of up to 20 square miles to become your guarded lands. If someone defiles when casting arcane spells on your guarded lands, you instinctively know of the act and where on your lands it takes place.

\textbf{Sacrifice (Su):} If someone defiles on your guarded lands, you can react to protect your lands through sacrificing part of your own life force. This nullifies a wizard's defiling radius and any effects it entails, such as penalties to attacks and saving throws, and damage to plants. It also nullifies the benefits of any Raze feats used in the casting of the defiling spell. You lose 1 hit point per 1.5 meter of defiling radius nullified.

\textbf{Smite Intruder (Su):} The local Spirit of the Land infuses you with the power to smite intruders. Once a day, you may attempt to smite intruders with one normal melee attack. You add your Charisma modifier (if positive) to your attack roll and deals 1 extra point of damage per grove master level.

At 7th level, you may smite intruders one additional time per day.

\textbf{Sustenance (Su):} At 3rd level, you need not eat or drink after spending 24 hours in your guarded lands as long as you remain on your guarded lands. While on your guarded lands, the Spirit of the Land provides you with nutrition. 

\textbf{Invisibility (Sp):} Starting at 3rd level, you can become invisible as the \spell{invisibility} spell while on your guarded lands, once per day. Your caster level is equal to your grove master level. If you move outside the boundaries of your guarded lands while invisible, the effect is dispelled.

At 6th level, you may use this ability one additional time per day.

\textbf{Teleport (Sp):} At 4th level, you can teleport as the \spell{teleport} spell to any location within your guarded lands, once per day. You cannot teleport to a location beyond the boundaries of your guarded lands, nor can you teleport back to your lands if you move outside your guarded lands.

At 8th level, you use the effect of \spell{greater teleport} spell instead, and you gain an additional use per day.

\textbf{Nondetection (Sp):} While on your guarded lands, you become difficult to locate through magical means. You are treated as if under the effects of the \spell{nondetection} spell.

\textbf{Greater Invisibility (Sp):} You can become invisible and strike without revealing yourself as the \spell{greater invisibility} spell, once per day. Your caster level is equal to your grove master level. This ability functions only on your guarded lands. If you move outside the boundaries of your guarded lands while invisible, the effect is dispelled.

\textbf{Timeless Body (Ex):} You no longer suffer ability score penalties from aging and cannot be magically aged. Any penalties you may have already incurred, however, remain in place. Bonuses still accrue, and you will die of old age when your time is up.
}
{}
{nature or religion}
{Grove masters are experienced druids devoted to protecting a certain area from harm. In return, they receive additional powers from their Spirit of the Land.}
{Within their guarded lands, grove masters are almost undetectable and can sense defiling from miles away.}
{Characters who achieve this level of success can learn important details about a specific grove master in your campaign, including the areas where he operates, and the kinds of activities he undertakes.}

Due to their lack of any central organization, and a tendency of being loners, finding a grove master is no small feat. The best PCs might manage is to visit places known to have Spirits of the Land inhabiting them and hope that a grove master hears of their interest.