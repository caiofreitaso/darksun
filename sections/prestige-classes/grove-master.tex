\PrestigeClass{Grove Master}
{Like the silk wyrm, I strike at night under the guise of the moons. Like the tembo, I wear down my enemy, and like the dune reaper I hold no remorse.}{Jak, dwarven grove master}
{Grove masters are experienced druids devoted to protecting a certain area from the destruction of defilers and other intruders that would harm the land. In return for their devotion, the Spirit of the Land bestows additional powers on the grove masters. Grove masters are formidable enemies while on their guarded lands.

Most grove masters allow travelers free passage as long as they do not harm the ecosystem, but some view all as intruders and enemies. The latter are typically grove masters who have sworn to protect an endangered plant or species from eradication, and will go to any lengths to do so.}
{d8}
{a}
{Druids are natural candidates for this class, but members of this prestige class can come from a variety of backgrounds. A grove master is chosen more for his devotion to the wilderness than for any specific skill he possesses. Sometimes a very dedicated ranger manages to gain the trust of a Spirit of the Land and become a grove master.}
{
\textbf{Skills:} \skill{Knowledge} (nature) 10 ranks, \skill{Survival} 5 ranks, \skill{Hide} 4 ranks.

\textbf{Feats:} \feat{Defender of the Land}, \feat{Wastelander}.

\textbf{Spells:} Able to cast 3rd-level divine spells.

\textbf{Special:} Must gain the trust of a Spirit of the Land.
}
{\skill{Concentration} (Con), \skill{Craft} (Int), \skill{Diplomacy} (Cha), \skill{Disguise} (Cha), \skill{Handle Animal} (Cha), \skill{Heal} (Wis), \skill{Hide} (Dex), \skill{Knowledge} (nature) (Int), \skill{Listen} (Wis), \skill{Move Silently} (Dex), \skill{Profession} (Wis), \skill{Ride} (Dex), \skill{Spellcraft} (Int), \skill{Spot} (Wis), and \skill{Survival} (Wis).
}
{2}
{\PrestigeSpellTable}{
1 & +0 & +0 & +0 & +2 & Guarded lands, sacrifice, wild shape & +1 level of existing divine spellcasting class\\
2 & +1 & +0 & +0 & +3 & Green speech, smite intruder 1/day & +1 level of existing divine spellcasting class \\
3 & +1 & +1 & +1 & +3 & Sustenance, \emph{invisibility} 1/day & +1 level of existing divine spellcasting class\\
4 & +2 & +1 & +1 & +4 & Bonus domain, \emph{teleport} 1/day & +1 level of existing divine spellcasting class \\
5 & +2 & +1 & +1 & +4 & \emph{Nondetection} & +1 level of existing divine spellcasting class \\
6 & +3 & +2 & +2 & +5 & \emph{Invisibility} 2/day & +1 level of existing divine spellcasting class \\
7 & +3 & +2 & +2 & +5 & Smite intruder 2/day & +1 level of existing divine spellcasting class \\
8 & +4 & +2 & +2 & +6 & \emph{Teleport} 2/day & +1 level of existing divine spellcasting class \\
9 & +4 & +3 & +3 & +6 & Bonus domain & +1 level of existing divine spellcasting class \\
10 & +5 & +3 & +3 & +7 & Timeless body & +1 level of existing divine spellcasting class\\
}
{
\textbf{Spellcasting:} When a new grove master level is gained, the character gains new spells per day as if she had also gained a level in whatever divine spellcasting class she belonged to before she added the prestige class. You do not, however, gain any other benefit a character of that class would have gained. This essentially means that she adds the level of grove master to the level of whatever other divine spellcasting class she has, and then determines spells per day and caster level accordingly.

If she had more than one divine spellcasting class before she became a grove master, she must decide to which class she adds each level of grove master for the purpose of determining spells per day.

\textbf{Guarded Lands (Su):} A grove master chooses a single area of up to 30 square kilometers to become her guarded lands. If someone defiles when casting arcane spells on her guarded lands, she instinctively knows of the act and where on her lands it takes place.

\textbf{Sacrifice (Su):} If someone defiles on her guarded lands, she can react to protect her lands through sacrificing part of her own life force. This nullifies a wizard's defiling radius and any effects it entails, such as penalties to attacks and saving throws, and damage to plants. It also nullifies the benefits of any raze feats used in the casting of the defiling spell. She loses 1 hit point per 1.5 meter of defiling radius nullified.

\textbf{Wild Shape (Su):} A grove master's druid levels stack with her grove master levels for the purpose of determining the number of daily uses, the maximum Hit Dice, size (but not creature type), and the duration of her wild shape ability.

\textbf{Green Speech (Su):} At 2nd level, a grove master can speak with plants on her guarded lands, as if under the effects of the \spell{speak with plants} spell.

\textbf{Smite Intruder (Su):} At 2nd level, the local Spirit of the Land infuses a grove master with the power to smite intruders. Once a day, she may attempt to smite intruders with one normal melee attack. She adds her Charisma modifier (if positive) to her attack roll and deals 1 extra point of damage per grove master level.

At 7th level, she may smite intruders one additional time per day.

\textbf{Sustenance (Su):} At 3rd level, a grove master needs not eat or drink after spending 24 hours in her guarded lands as long as she remains on her guarded lands. While on her guarded lands, the Spirit of the Land provides her with nutrition. 

\textit{Invisibility (Sp):} Starting at 3rd level, a grove master can become invisible as the \spell{invisibility} spell while on her guarded lands, once per day. The grove master's caster level is equal to her grove master level. If she moves outside the boundaries of her guarded lands while invisible, the effect is dispelled.

At 6th level, she may use this ability one additional time per day.

\textbf{Bonus Domain:} At 4th level, a grove master gains access to a domain related to the most abundant element of her guarded lands. For example, if her guarded land is a stream or an oasis she is restricted to the water element. She may not choose a paraelement as the most abundant element.

She is affected by the domain's granted power, and when preparing spells, she can select one spell slot per spell level---and only one slot per spell level---to memorize a domain spell. A grove master can never prepare more than one domain spell of each level.

At 9th level, she gains access to another domain of the same element.

\textit{Teleport (Sp):} At 4th level, a grove master can teleport as the \spell{teleport} spell to any location within her guarded lands, once per day. She cannot teleport to a location beyond the boundaries of her guarded lands, nor can she teleport back to her lands if she moves outside her guarded lands.

At 8th level, she may use this ability one additional time per day.

\textit{Nondetection (Sp):} Beginning on 5th level, a grove master becomes difficult to locate through magical means while on her guarded lands. She is treated as if under the effects of the \spell{nondetection} spell.

\textbf{Timeless Body (Ex):} At 10th level, a grove master no longer suffers ability score penalties from aging and cannot be magically aged. Any penalties she may have already incurred, however, remain in place. Bonuses still accrue, and she will die of old age when her time is up.
}
% {}
% {nature or religion}
% {Grove masters are experienced druids devoted to protecting a certain area from harm. In return, they receive additional powers from their Spirit of the Land.}
% {Within their guarded lands, grove masters are almost undetectable and can sense defiling from miles away.}
% {Characters who achieve this level of success can learn important details about a specific grove master in your campaign, including the areas where he operates, and the kinds of activities he undertakes.}

% Due to their lack of any central organization, and a tendency of being loners, finding a grove master is no small feat. The best PCs might manage is to visit places known to have Spirits of the Land inhabiting them and hope that a grove master hears of their interest.