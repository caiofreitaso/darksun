\BigTableBottom{Psionic Aura Strength}{L*4C}{
& \tableheader Faint
& \tableheader Moderate
& \tableheader Strong
& \tableheader Overwhelming \\
Manifester level & 5th or lower & 6th--11th & 12th--20th & 21st+ (artifact) \\
}

\section{Psionic Item Descriptions}
In the following sections, each type of psionic item, such as cognizance crystals or psionic tattoos, has a general description, followed by descriptions of specific items.

General descriptions include notes on activation, random generation, and other information. The Armor Class, hardness, hit points, and break DC are given for typical examples of some types of psionic items. The Armor Class assumes that the item is unattended and includes a $-5$ penalty for the item's effective Dexterity of 0. If a creature holds the item, use the creature's Dexterity modifier as an adjustment to Armor Class in place of the $-5$ penalty.

Some individual items, notably those that simply store psionic powers, don't get full-blown descriptions. Simply reference the power's description. Assume that the power is manifested at the minimum level required to manifest it, unless otherwise noted. Increasing the manifester level so increases the cost of the item. The main reason to make the manifester level higher, or course, would be to increase the potency of the power. Raising the manifester level is common for powers such as \psionic{astral construct}, the duration of which increases with the level of the manifester.

Items with full descriptions have their abilities detailed, and each of the following aspects of these items is summarized at the end of the description.

\textbf{Aura:} Most of the time, a detect psionics power will reveal the discipline associated with a psionic item and the strength of the aura an item emits. This information (when applicable) is given at the beginning of the item's notational entry in the form of a phrase. See \tabref{Psionic Aura Strength} for more information.% See the \psionic{detect psionics} power description for more information.

\textbf{Manifester Level:} The next entry in the summary indicates the level of the creator (or the manifester level of the powers placed within the item, if this level is lower than the actual level of the creator). The manifester level provides the item's saving throw bonus, as well as range and other level-dependent aspects of the powers of the item (if variable).

% It also determines the level that must be contended with should the item come under the effect of a dispel psionics power or a similar situation.
This information is given in the form ``ML x,'' where ``ML'' is an abbreviation for manifester level and ``x'' is an ordinal number representing the manifester level itself. The item itself determines the manifester level, so the creator's manifester level must be as high as the item's manifester level (and prerequisites may effectively put a higher minimum on the creator's level).

\textbf{Prerequisites:} Certain requirements must be met in order for a character to create a psionic item. These include feats, powers, and miscellaneous requirements such as level, alignment, and race or kind. The prerequisites for creation of an item are given in the summary immediately following the item's manifester level. A power prerequisite can be provided by a character who knows the power or a psi-like ability that produces the desired power effect.

It is possible for more than one character to cooperate in the creation of an item, with each participant providing one or more of the prerequisites. In some cases, cooperation may even be necessary, such as if one character knows some of the powers necessary to create an item and another character knows the rest.

If two or more characters cooperate to create an item, they must agree among themselves who will be considered the creator for the purpose of determinations where the creator's level must be known. (It's sensible, although not mandatory, for the highest-level character involved to be considered the creator.) The character designated as the creator pays the experience points required to make the item.

Typically, a list of prerequisites includes one feat and one or more powers (or some other requirement in addition to the feat). When two powers at the end of a list are separated by ``or,'' one of those powers is required in addition to every other power mentioned prior to the last two.

\textbf{Market Price:} This gold piece value, given in the summary following the word ``Price,'' represents the price someone should expect to pay to buy the item. Market prices are also included on the random generation tables for easy reference. The market price of an item that can be constructed with a psionic item creation feat is usually equal to the base price + the price for any components (special materials or experience point expenditure).

\textbf{Cost to Create:} The cost in gold pieces and experience points to create the item is given in the summary following the word ``Cost.'' This information appears only for items with components (material or experience points) that make their market prices higher than their base prices. The cost to create includes the costs derived from the base cost plus the cost of the components. Items without components do not have a ``Cost'' entry. For them, the market price and base price are the same. The cost in gold pieces is \onehalf the market price, and the cost in experience points is 1/25 the market price.

\textbf{Weight:} The notational entry for many items ends with a value for the item's weight. When a weight figure is not given, the item has no weight worth noting (for the purpose of determining how much of a load a character can carry).
