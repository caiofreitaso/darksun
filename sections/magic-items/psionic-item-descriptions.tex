\BigTableBottom{Psionic Aura Strength}{L*4C}{
& \tableheader Faint
& \tableheader Moderate
& \tableheader Strong
& \tableheader Overwhelming \\
Manifester level & 5th or lower & 6th--11th & 12th--20th & 21st+ (artifact) \\
}

\section{Psionic Item Descriptions}
In the following sections, each type of psionic item, such as cognizance crystals or psionic tattoos, has a general description, followed by descriptions of specific items.

General descriptions include notes on activation, random generation, and other information. The Armor Class, hardness, hit points, and break DC are given for typical examples of some types of psionic items. The Armor Class assumes that the item is unattended and includes a $-5$ penalty for the item's effective Dexterity of 0. If a creature holds the item, use the creature's Dexterity modifier as an adjustment to Armor Class in place of the $-5$ penalty.

Some individual items, notably those that simply store psionic powers, don't get full-blown descriptions. Simply reference the power's description. Assume that the power is manifested at the minimum level required to manifest it, unless otherwise noted. Increasing the manifester level so increases the cost of the item. The main reason to make the manifester level higher, or course, would be to increase the potency of the power. Raising the manifester level is common for powers such as \psionic{astral construct}, the duration of which increases with the level of the manifester.

Items with full descriptions have their abilities detailed, and each of the following aspects of these items is summarized at the end of the description.

\textbf{Aura:} Most of the time, a detect psionics power will reveal the discipline associated with a psionic item and the strength of the aura an item emits. This information (when applicable) is given at the beginning of the item's notational entry in the form of a phrase. See \tabref{Psionic Aura Strength} for more information.% See the \psionic{detect psionics} power description for more information.

\textbf{Manifester Level:} The next entry in the summary indicates the level of the creator (or the manifester level of the powers placed within the item, if this level is lower than the actual level of the creator). The manifester level provides the item's saving throw bonus, as well as range and other level-dependent aspects of the powers of the item (if variable).

% It also determines the level that must be contended with should the item come under the effect of a dispel psionics power or a similar situation.
This information is given in the form ``ML x,'' where ``ML'' is an abbreviation for manifester level and ``x'' is an ordinal number representing the manifester level itself. The item itself determines the manifester level, so the creator's manifester level must be as high as the item's manifester level (and prerequisites may effectively put a higher minimum on the creator's level).

\textbf{Prerequisites:} Certain requirements must be met in order for a character to create a psionic item. These include feats, powers, and miscellaneous requirements such as level, alignment, and race or kind. The prerequisites for creation of an item are given in the summary immediately following the item's manifester level. A power prerequisite can be provided by a character who knows the power or a psi-like ability that produces the desired power effect.

It is possible for more than one character to cooperate in the creation of an item, with each participant providing one or more of the prerequisites. In some cases, cooperation may even be necessary, such as if one character knows some of the powers necessary to create an item and another character knows the rest.

If two or more characters cooperate to create an item, they must agree among themselves who will be considered the creator for the purpose of determinations where the creator's level must be known. (It's sensible, although not mandatory, for the highest-level character involved to be considered the creator.) The character designated as the creator pays the experience points required to make the item.

Typically, a list of prerequisites includes one feat and one or more powers (or some other requirement in addition to the feat). When two powers at the end of a list are separated by ``or,'' one of those powers is required in addition to every other power mentioned prior to the last two.

\textbf{Market Price:} This gold piece value, given in the summary following the word ``Price,'' represents the price someone should expect to pay to buy the item. Market prices are also included on the random generation tables for easy reference. The market price of an item that can be constructed with a psionic item creation feat is usually equal to the base price + the price for any components (special materials or experience point expenditure).

\textbf{Cost to Create:} The cost in gold pieces and experience points to create the item is given in the summary following the word ``Cost.'' This information appears only for items with components (material or experience points) that make their market prices higher than their base prices. The cost to create includes the costs derived from the base cost plus the cost of the components. Items without components do not have a ``Cost'' entry. For them, the market price and base price are the same. The cost in gold pieces is \onehalf the market price, and the cost in experience points is 1/25 the market price.

\textbf{Weight:} The notational entry for many items ends with a value for the item's weight. When a weight figure is not given, the item has no weight worth noting (for the purpose of determining how much of a load a character can carry).

\subsectionA{Cognizance Crystals}
\emph{Cognizance crystals} store power points that psionic characters can use to pay for manifesting their powers.

\Table{Cognizance Crystals}{CR}{
  \tableheader Maximum Power Point Storage
& \tableheader Market Price \\
 5 & 1,000 cp \\
10 & 4,000 cp \\
15 & 9,000 cp \\
20 & 16,000 cp \\
25 & 25,000 cp \\
30 & 36,000 cp \\
35 & 49,000 cp \\
40 & 64,000 cp \\
45 & 81,000 cp \\
}

\textbf{Physical Description:} A \emph{cognizance crystal} consists of a core crystal and two or three smaller crystals arranged in specific positions around it on strands of silver wire. The crystals give off a faint glow. A typical \emph{cognizance crystal} weighs approximately 1 pound, has AC 7, 10 hit points, a hardness of 8, and a break DC of 16.

\textbf{Activation:} The user must merely hold or have a crystal on her person for a period of at least 10 minutes (which is long enough to attune oneself to the crystal). Thereafter, the owner can use power points stored in the crystal to manifest powers she knows.

The maximum number of points a \emph{cognizance crystal} can store is always a number multiple of five and is never more than 45. It can store only as many power points as its original maximum, set at the time of its creation. When a \emph{cognizance crystal}'s power points are used up, the glow of the crystal dims. However, the user can recharge it by paying power points on a 1-for-1 basis. While doing this depletes the user's own power point reserve for the day, those power points remain available in the \emph{cognizance crystal} until used.

A user cannot directly replenish her personal power points from those stored in a \emph{cognizance crystal}, nor can she draw power points from more than one source to manifest a power. See \chapref{Psionics} for more information.

Faint to strong psychokinesis; ML equal to maximum power point storage divided by 5; \feat{Craft Cognizance Crystal}; Weight 0.5 kg.

\subsectionA{Intelligent Items}
Items of power (either magical or psionic) sometimes have intelligence of their own. Psionically imbued with sentience, these items think and feel the same way characters do and should be treated as NPCs. Intelligent items have extra abilities and sometimes extraordinary powers and special purposes.

Intelligent items can actually be considered creatures because they have Intelligence, Wisdom, and Charisma scores. Treat them as constructs with the psionic subtype. Intelligent items often have the ability to illuminate their surroundings at will (as psionic weapons do); many cannot see otherwise.

Intelligent items can activate their own powers without waiting for a command word from their owner. Intelligent items act during their owner's turn in the initiative order.

\subsubsection{Intelligent Item Alignment}
Any item with intelligence has an alignment. Note that intelligent weapons already have alignments, either stated or by implication. If you're generating a random intelligent weapon, that weapon's alignment must fit with any alignment-oriented special abilities it has.

Any character whose alignment does not correspond to that of the item (except as noted on the table) gains one negative level if he or she so much as picks up the item. Although this negative level never results in actual level loss, it remains as long as the item is in hand and cannot be overcome in any way (including restoration spells). This negative level is cumulative with any other penalties the item might already place on inappropriate wielders. Items with Ego scores (see below) of 20 to 29 bestow two negative levels. Items with Ego scores of 30 or higher bestow three negative levels.

\Table{Intelligent Item Alignment}{cX}{
  \tableheader d\%
& \tableheader Alignment of Item\\

01--05  & Chaotic good \\
06--15  & Chaotic neutral\footnotemark[1] \\
16--20  & Chaotic evil \\
21--25  & Neutral evil\footnotemark[1] \\
26--30  & Lawful evil \\
31--55  & Lawful good \\
56--60  & Lawful neutral\footnotemark[1] \\
61--80  & Neutral good\footnotemark[1] \\
81--100 & Neutral \\
\TableNote{2}{1 The item can also be used by any character whose alignment corresponds to the nonneutral portion of the item's alignment.}
}


\subsubsection{Languages Spoken By Item}
Like a character, an intelligent item speaks Common plus one additional language per point of Intelligence bonus. Choose appropriate languages, taking into account the item's origin and purposes.

\Table{Intelligent Item Psionic Powers}{cXr{14mm}}{
  \tableheader d\%
& \tableheader Psionic Power
& \tableheader Base Cost Modifier \\

00--00 & 1st-level power & +1,000 cp \\
00--00 & 2nd-level power & +3,000 cp \\
00--00 & 3rd-level power & +6,000 cp \\
}

\subsubsection{Intelligent Item Powers}
Intelligent items are a representation of the state of the psionic character when they created it. As a psionic character, intelligent items have manifester level, a pool of power points per day, and a number of known psionic powers. Intelligent items follow the same rules for manifesting.

\textbf{Manifester Level:} Intelligent items have a manifester level that determines the strength of some effects, like the Difficulty Class for psionic attacks, and its ability to know more powerful psionic powers.

\textbf{Power Points per Day:} To regain used daily power points, intelligent items need the same clear mind as any other manifester.

\textbf{Powers Known:} Intelligent items have a number of known psionic powers. They must fulfill all prerequisites to know a power, including having a key ability score of at least 10 + the power's level.

\textit{Personal Range:} Powers with personal range can target the wielder, if the item desires.

\textit{Key Ability Score:} For the psychometabolism and psychoportation disciplines, intelligent items use their Charisma as the key ability score. They use their Charisma score for power checks, and their Charisma modifier for saving throw DCs.

\subsubsection{Intelligent Item Skills}
The \tabref{Item Ability Scores and Capabilities} determines how many psionic powers an intelligent item has, and some items may have skill ranks. To find the item's specific ranks in skills, choose or roll on the table below.

If the same skill is rolled twice, roll again.

\Table{Intelligent Item Skills}{cXr{13mm}}{
  \tableheader d\%
& \tableheader Skill Ranks
& \tableheader Base Price Modifier \\
01--03  & Item has 10 ranks in \skill{Appraise}                           & +5,000 cp \\
04--13  & Item has 10 ranks in \skill{Bluff}                              & +5,000 cp \\
14--15  & Item has 10 ranks in \skill{Decipher Script}                    & +5,000 cp \\
16--20  & Item has 10 ranks in \skill{Diplomacy}                          & +5,000 cp \\
21--30  & Item has 10 ranks in \skill{Intimidate}                         & +5,000 cp \\
31--45  & Item has 10 ranks in \skill{Knowledge} (any one category)       & +5,000 cp \\
46--50  & Item has 10 ranks in \skill{Listen}                             & +5,000 cp \\
51--60  & Item has 10 ranks in \skill{Perform} (comedy, oratory, or sing) & +5,000 cp \\
61--75  & Item has 10 ranks in \skill{Psicraft}                           & +5,000 cp \\
76--80  & Item has 10 ranks in \skill{Search}                             & +5,000 cp \\
81--90  & Item has 10 ranks in \skill{Sense Motive}                       & +5,000 cp \\
91--92  & Item has 10 ranks in \skill{Spellcraft}                         & +5,000 cp \\
93--97  & Item has 10 ranks in \skill{Spot}                               & +5,000 cp \\
98--100 & Item has 10 ranks in \skill{Survival}                           & +5,000 cp \\
}

\subsubsection{Special Purpose Items}
Many of the great intelligent items in Athas were created with a special purpose in mind, which gives them and their wielder access to new powers. Most of them were created during the Preserver Jihad and the Cleansing Wars (see \chapref{Other Ages of Play}).

\textbf{Purpose:} An item's purpose must suit the type and alignment of the item and should always be treated reasonably. A purpose of ``defeat/slay arcane spellcasters'' doesn't mean that the sword forces the wielder to kill every wizard she sees. Nor does it mean that the sword believes it is possible to kill every wizard in the world. It does mean that the item hates arcane spellcasters and wants to bring the local wizard's cabal to ruin, as well as end the rule of a sorceress-queen in a nearby land. Likewise, a purpose of ``defend elves'' doesn't mean that if the wielder is an elf, he only wants to help himself. It means that the item wants to be used in furthering the cause of elves, stamping out their enemies and aiding their leaders. A purpose of ``defeat/slay all'' isn't just a matter of self-preservation. It means that the item won't rest (or let its wielder rest) until it places itself above all others.

\Table{Intelligent Item Purpose}{cX}{
  \tableheader d\%
& \tableheader Purpose \\
01--20 & Defeat/slay diametrically opposed alignment\footnotemark[1] \\
21--30 & Defeat/slay arcane spellcasters (including spellcasting monsters and those that use spell-like abilities) \\
31--40 & Defeat/slay divine spellcasters (including divine entities and servitors) \\
41--45 & Defeat/slay manifesters (including manifesting monsters and those that use psi-like abilities) \\
46--50 & Defeat/slay nonspellcasters \\
51--55 & Defeat/slay a particular creature type (see the \emph{bane} special ability for choices) \\
56--60 & Defeat/slay a particular race or kind of creature \\
61--70 & Defend a particular race or kind of creature \\
71--80 & Defeat/slay the servants of a specific deity \\
81--90 & Defend the servants and interests of a specific deity \\
91--95 & Defeat/slay all (other than the item and the wielder) \\
96--100 &  Choose one \\
\TableNote{2}{1 The purpose of the neutral (N) version of this item is to preserve the balance by defeating/slaying powerful beings of the extreme alignments (LG, LE, CG, CE).}
}


\subsubsection{Dedicated Powers}
A dedicated power operates only when an intelligent item is in pursuit of its special purpose. This determination is always made by the item. It should always be easy and straightforward to see how the ends justify the means. Unlike its other powers, an intelligent item can refuse to use its dedicated power even if the owner is dominant (see Items against Characters, below).

\textit{Note:} An intelligent item can have more than one power for its special purpose.

\Table{Special Purpose Item Dedicated Powers}{cXr{15mm}}{
  \tableheader d\%
& \tableheader Dedicated Power
& \tableheader Base Price Modifier \\

00--00 & Item can manifest a 4th-level psionic power & +10,000 cp \\
00--00 & Item can manifest a 5th-level psionic power & +15,000 cp \\
00--00 & Item can manifest a 6th-level psionic power & +21,000 cp \\
00--00 & Item can manifest a 7th-level psionic power & +28,000 cp \\
00--00 & Item can manifest a 8th-level psionic power & +36,000 cp \\
00--00 & Item can manifest a 9th-level psionic power & +45,000 cp \\
00--00 & Wielder gets +2 luck bonus on attacks, saves, and checks & +80,000 cp \\
}

\subsubsection{Item Ego}
Ego is a measure of the total power and force of personality that an item possesses. Only after all aspects of an item have been generated can its Ego score be calculated. An item's Ego score helps determine whether the item or the character is dominant in their relationship, as detailed below.

\Table{Item Ego}{Lc}{
  \tableheader Attribute of Item
& \tableheader Ego Score Modifier \\
Each +1 of item's enhancement bonus    & +1 \\
Each +1 of bonus for special abilities & +1 \\
Special purpose                        & +2 \\
Each dedicated power                   & +2 \\
}

\subsubsection{Items Against Characters}
When an item has an Ego of its own, it has a will of its own. The item is, of course, absolutely true to its alignment. If the character who possesses the item is not true to that alignment's goals or the item's special purpose, personality conflict---item against character---results. Similarly, any item with an Ego score of 20 or higher always considers itself superior to any character, and a personality conflict results if the possessor does not always agree with the item.

When a personality conflict occurs, the possessor must make a Will saving throw (DC = item's Ego). If the possessor succeeds, she is dominant. If she fails, the item is dominant. Dominance lasts for one day or until a critical situation occurs (such as a major battle, a serious threat to either the item or the character, and so on). Should an item gain dominance, it resists the character's desires and demands concessions such as any of the following.

\begin{itemize*}
\item Removal of associates or items whose alignment or personality is distasteful to the item.
\item The character divesting herself of all other magic items or items of a certain type.
\item Obedience from the character so the item can direct where they go for its own purposes.
\item Immediate seeking out and slaying of creatures hateful to the item.
\item Magical protections and devices to protect the item from molestation when it is not in use.
\item That the character carry the item with her on all occasions.
\item That the character relinquish the item in favor of a more suitable possessor due to alignment differences or conduct.
\end{itemize*}

In extreme circumstances, the item can resort to even harsher measures, such as the following acts:
\begin{itemize*}
\item Force its possessor into combat.
\item Refuse to strike opponents.
\item Strike at its wielder or her associates.
\item Force its possessor to surrender to an opponent.
\item Cause itself to drop from the character's grasp.
\end{itemize*}

Naturally, such actions are unlikely when harmony reigns between the character's and item's alignments or when their purposes and personalities are well matched. Even so, an item might wish to have a lesser character possess it in order to easily establish and maintain dominance over him, or a higher-level possessor so as to better accomplish its goals.

All magic items with personalities desire to play an important role in whatever activity is under way, particularly combat. Such items are rivals of each other, even if they are of the same alignment. No intelligent item wants to share its wielder with others. An intelligent item is aware of the presence of any other intelligent item within 60 feet, and most intelligent items try their best to mislead or distract their host so that she ignores or destroys the rival. Of course, alignment might change this sort of behavior.

Items with personalities are never totally controlled or silenced by the characters who possess them, even though they may never successfully control their possessors. They may be powerless to force their demands but remain undaunted and continue to air their wishes and demands.

\subsubsection{Intelligent Item Examples}
These items are commonly made in Athas. Because these items are easily controllable and they do not speak, they are the favorites of Athasian aristocrats. They are for psionic items what the settler is for characters: the commoner you are most likely to find.

\Table{Intelligent Item Examples}{LR}{
  \tableheader Intelligent Item
& \tableheader Price \\
Carpet anchor          &  2,000 cp \\
Sun cloak              &  2,000 cp \\
Boots of equilibrium   &  4,000 cp \\
Helm of iron will      &  4,000 cp \\
Horn of clairaudience  &  4,000 cp \\
Bag of faces           &  6,000 cp \\
Gauntlets of war       & 11,000 cp \\
Robe of the beast      & 11,000 cp \\
Spirit pipes           & 11,000 cp \\
Mirror of clairvoyance & 14,000 cp \\
}


\textbf{Bag of Faces:}
AL N;
Int 10, Wis 13, Cha 13;
Empathy, 18 m vision, hearing;
ML 3rd;
32 pp;
Ego score 5.

The \emph{bag of faces} is made from thick hide. The owner places the bag over their head for one full turn. When they take it off, they look like someone else. The owner must keep the bag on their person for it to maintain the disguise.

\textit{Known Powers:}
    \psionic{cell adjustment}, %1st
    \psionic{psionic disguise self}. %2nd

Faint psychometabolism;
ML 3rd;
\feat{Craft Psionic Item},
\psionic{cell adjustment},
\psionic{psionic disguise self};
Price 6,000 cp.



\textbf{Boots of Equilibrium:}
AL N;
Int 10, Wis 13, Cha 13;
Empathy, 18 m vision, hearing;
ML 3rd;
32 pp;
Ego score 5.

The \emph{boots of equilibrium} can vary in color from a deep red to a dark, mossy green. Those are the colors of the hej-kin hide used to make them.

\textit{Known Powers:}
    \psionic{body equilibrium}, %1st
    \psionic{catfall}. %1st

Faint psychometabolism;
ML 3rd;
\feat{Craft Psionic Item},
\psionic{body equilibrium},
\psionic{catfall};
Price 4,000 cp.



\textbf{Carpet Anchor:}
AL N;
Int 10, Wis 12, Cha 12;
Empathy, 9 m vision, hearing;
ML 1st;
20 pp;
Ego score 2.

A \emph{carpet anchor} looks like a normal carpet, usually measure 3 by 2.5 meters. It can be made of any material, but must be woven telekinetically.

\textit{Known Powers:}
    \psionic{spacetime anchor}. %1st

Faint psychometabolism;
ML 3rd;
\feat{Craft Psionic Item},
\psionic{spacetime anchor};
Price 2,000 cp.



\textbf{Gauntlets of War:}
AL N;
Int 10, Wis 13, Cha 13;
Empathy, 18 m vision, hearing;
ML 3rd;
32 pp;
Ego score 5.

The \emph{gauntlets of war} are made from bits of bone sewn together with Kirre gut, with the most noticeable feature being the distinctive hide covering. Curious characters may discover that the hide is from a nightmare beast.

\textit{Known Powers:}
    \psionic{adrenaline control}. %2nd
    \psionic{claws of the beast}. %3rd

Faint psychometabolism;
ML 3rd;
\feat{Craft Psionic Item},
\psionic{adrenaline control},
\psionic{claws of the beast};
Price 11,000 cp.



\textbf{Helm of Iron Will:}
AL N;
Int 12, Wis 12, Cha 10;
Empathy, 9 m vision, hearing;
ML 1st;
20 pp;
Ego score 2.

The \emph{helm of iron will} is the rune covered, bony skull plate of an id fiend. The inside has been laced with a fine web of iron threads. Greedy adventurers have been known to strip the iron out of the \emph{helm}, never knowing it's true worth.

\textit{Known Powers:}
    \psionic{tower of iron will}. %2nd

Faint universal;
ML 1st;
\feat{Craft Psionic Item},
\psionic{tower of iron will};
Price 4,000 cp.



\textbf{Horn of Clairaudience:}
AL N;
Int 10, Wis 12, Cha 12;
Empathy, 9 m vision, hearing;
ML 1st;
20 pp;
Ego score 2.

Frequently mistaken for a musical instrument, the horn looks like a simple bone trumpet. However, it is made from the bones of a silt runner. When the small end is placed at its owner's ear, it can use \psionic{clairaudience} to listen in on remote places.

\textit{Known Powers:}
    \psionic{clairaudience}. %2nd

Faint clairscience;
ML 1st;
\feat{Craft Psionic Item},
\psionic{clairaudience};
Price 4,000 cp.



\textbf{Mirror of Clairvoyance:}
AL N;
Int 10, Wis 13, Cha 13;
Empathy, 18 m vision, hearing;
ML 3rd;
32 pp;
Ego score 5.

The reflective paint used to coat the back of the glass is mixed with the ground skull of a psionic cat. When the mirror manifests \psionic{clairvoyance}, it can see things far away. Anyone else looking into the mirror sees the distant scene.

\textit{Known Powers:}
    \psionic{clairvoyance}, %3rd
    \psionic{environment}. %3rd

Faint clairscience;
ML 3rd;
\feat{Craft Psionic Item},
\psionic{clairvoyance},
\psionic{environment};
Price 14,000 cp.



\textbf{Robe of the Beast:}
AL N;
Int 10, Wis 13, Cha 13;
Empathy, 18 m vision, hearing;
ML 3rd;
32 pp;
Ego score 5.

These robes can be made from the hide of virtually any psionic animal. A \emph{robe of the beast} allows its wearer to discover and use their animal aura, with \psionic{animal affinity}. It is important to remember that the cloak does not choose the animal. That comes from the unconscious mind of the wearer.

\textit{Known Powers:}
    \psionic{animal affinity}, %3rd
    \psionic{biofeedback}. %2nd

Faint psychometabolism;
ML 3rd;
\feat{Craft Psionic Item},
\psionic{animal affinity},
\psionic{biofeedback};
Price 11,000 cp.



\textbf{Spirit Pipes:}
AL N;
Int 10, Wis 13, Cha 13;
Empathy, 18 m vision, hearing;
ML 3rd;
32 pp;
Ego score 5.

Made from the fingers of any once animated skeleton, the \emph{spirit pipes} are not meant to be played. When activated, they manifest \psionic{detect spirits} to detect the presence of incorporeal creatures. If the area is frequented by such spirits, or any spirit is that close, it plays a haunting tune all by itself.

\textit{Known Powers:}
    \psionic{detect spirits}, %3rd
    \psionic{watcher's ward}. %2nd

Faint clairscience;
ML 3rd;
\feat{Craft Psionic Item},
\psionic{detect spirits},
\psionic{watcher's ward};
Price 11,000 cp.



\textbf{Sun Cloak:}
AL N;
Int 10, Wis 12, Cha 12;
Empathy, 9 m vision, hearing;
ML 1st;
20 pp;
Ego score 2.

These cloaks are always some shade of green and made from a variety of plant fibers.

\textit{Known Powers:}
    \psionic{photosynthesis}. %1st

Faint psychometabolism;
ML 1st;
\feat{Craft Psionic Item},
\psionic{photosynthesis};
Price 2,000 cp.

\subsubsection{Unique Intelligent Items}
On a world with cultures as rich and varied as those on Athas, there are bound to be many unique and powerful magical and psionic creations. The following section contains but a few of these.

\Table{Unique Intelligent Items}{LR}{
  \tableheader Intelligent Item
& \tableheader Price \\
Periapt of Tierna              & 39,000 cp \\
Talisman of Torr'ack the Cruel & 54,000 cp \\
Agafari Rod                    & 58,300 cp \\
Red Crystal of Tyr             & 115,000 cp \\
}

\textbf{Agafari Rod:}
\emph{+2 club};
AL LN;
Int 18, Wis 18, Cha 10;
Speech, telepathy, 36 m darkvision, blindsense, hearing;
ML 13th;
120 pp;
Ego score 8.

The \emph{Agafari Rod} is nearly 1 meter long and carved in the artistic fashion of Gulg, with totem-like creatures climbing its length. A tuft of exotic feathers decorates its head.

\textit{Known Powers:}
	\psionic{ballistic attack}, %3rd
	\psionic{control flames}, %2nd
	\psionic{deflect}, %1st
	\psionic{psionic levitate}, %3rd
	\psionic{psionic telekinesis}, %1st

\textit{Special Purpose:} Defend the Crescent Forest.

\textit{Dedicated Powers:}
	\psionic{inertial barrier}, %4th
	\psionic{psionic wall of force}. %5th

\textit{Personality:} The \emph{Agafari Rod} was crafted by a powerful druid, known as the Keeper, to defend the Crescent Forest against Nibenese loggers, occasionally aiding the Gulgan army. However, Nibenese templars cut a ruined swath through the druid's grove and killed him in battle. The weapon was taken to Nibenay's High Templar trophy chamber, where it stayed for five years. Now it's resentful and angry at Nibenese people, specially their loggers and templars.

Strong psychokinesis;
CL 6th,
ML 13th;
\feat{Craft Magic Arms and Armor},
\feat{Craft Psionic Item},
\psionic{ballistic attack},
\psionic{control flames},
\psionic{deflect},
\psionic{inertial barrier},
\psionic{psionic levitate},
\psionic{psionic telekinesis},
\psionic{psionic wall of force};
Price 58,300 cp.


%%%%


\textbf{Periapt of Tierna:}
AL LG;
Int 10, Wis 16, Cha 16;
Speech, 18 m darkvision, hearing;
ML 9th;
86 pp;
Ego score 19.

The \emph{Periapt of Tierna} is a pale green gemstone with a white star in its center.

\textit{Known Powers:}
	\psionic{adapt body}, %1st
	\psionic{cell adjustment}, %1st
	\psionic{photosynthesis}, %1st
	\psionic{psionic displacement}, %3rd
	\psionic{sustenance}. %2nd

\textit{Special Purpose:} Defend those in need.

\textit{Dedicated Power:} \psionic{personal heal}. %6th

\textit{Personality:} The \emph{Periapt of Tierna} can speak, but only does so if someone nearby is suffering and its owner does not know of its healing abilities. It grieves for Tierna's apprentice, Relia, and has been saddened by the violence that surrounds it. Most of its owners have been savage bandits or marauders, and the \emph{Periapt of Tierna} wants to be placed in the hands of a healer to perform its original mission.

Moderate psychometabolism;
ML 9th;
\feat{Craft Psionic Item},
\psionic{adapt body},
\psionic{cell adjustment},
\psionic{personal heal},
\psionic{photosynthesis},
\psionic{psionic displacement},
\psionic{sustenance};
Price 39,000 cp.


%%%%


\textbf{Red Crystal of Tyr:}
AL LE;
Int 18, Wis 18, Cha 10;
Speech, telepathy, 36 m darkvision, blindsense, hearing;
ML 13th;
120 pp;
Ego score 30.

The \emph{Red Crystal of Tyr} is a large, jagged shard of unidentified stone, about the size of a sword hilt. Its facets are razor-sharp and can easily injure a careless handler. The crystal is circled by two bands of copper and suspended from a copper chain.

\textit{Known Powers:}
	\psionic{clairaudience}, %2nd
	\psionic{clairvoyance}, %3rd
	\psionic{detect spirits}, %3rd
	\psionic{psionic detect magic}. %2nd

\textit{Special Purpose:} Amass power.

\textit{Dedicated Powers:}
	\psionic{precognition}, %6th
	\psionic{predestination}, %7th
	\psionic{spirit lore}. %8th

\textit{Personality:} The \emph{Red Crystal of Tyr} is a vicious thing with a sarcastic, mocking manner. It urges its possessor to do whatever it takes to amass power. It provides its bearer with remarkably accurate knowledge of the future, but delights in showing possible failures and death to its unfortunate owner. The \emph{Red Crystal of Tyr} attempts to master any who claim it.

Strong clairscience;
ML 13th;
\feat{Craft Psionic Item},
\psionic{clairaudience},
\psionic{clairvoyance},
\psionic{detect spirits},
\psionic{precognition},
\psionic{predestination},
\psionic{psionic detect magic},
\psionic{spirit lore};
Price 115,000 cp.


%%%%


\textbf{Talisman of Torr'ack the Cruel:}
AL CE;
Int 15, Wis 10, Cha 15;
Speech, 18 m darkvision, hearing;
ML 7th;
64 pp;
Ego score 15.

The \emph{Talisman of Torr'ack the Cruel} is an obsidian circle covered in cryptic carvings, with a silver and iron necklace attached.

\textit{Known Powers:}
	\psionic{aging}, %3rd
	\psionic{double pain}, %1st
	\psionic{life draining}. %3rd

\textit{Special Purpose:} Defeat/slay all.

\textit{Dedicated Powers:}
	\psionic{death field}.  %8th

\textit{Personality:} Once a person of some skill comes into contact with the \emph{Talisman of Torr'ack the Cruel}, the device begins manipulating its owner to commit brutal and inhuman feats to those of less power. If the \emph{Talisman} comes into the possession of someone without the competence or power to use it properly, the device either remains inert or convinces its owner to ``seek out'' someone more capable of serving its needs. The \emph{Talisman} takes great pleasure in twisting innocents to its own ends.

Strong psychometabolism;
ML 7th;
\feat{Craft Psionic Item},
\psionic{aging},
\psionic{double pain},
\psionic{life draining},
\psionic{death field};
Price 54,000 cp.


\subsectionA{Psionic Tattoos}
Psionic tattoos are designs scribed on the skin that manifest powers on their wearers. The wearer doesn't get to make any decisions about the tattoo's effect---the manifester who scribed it has already done so. The tattoo's effect is continuous.

Psionic tattoos do not use the normal body slots for magic items, they use what available space there is in your body. Psionic tattoos can vary in size based on its power level, and a creature can hold a total of 78 power levels.

\textbf{Physical Description:} A typical psionic tattoo is a colorful pattern of tiny, interlacing lines within a larger design. This design can be as simple as a circle or a star, or as complex as an artist wishes to make it. Once it is scribed, a tattoo's design does not change. The size of the tattoo indicates its power level.

\textbf{Scribing Tattoos:} The manifester level of a psionic tattoo is the minimum level required to manifest the scribed power.

% \textit{Tattoo Location:} Different areas of the body can hold varying number of manifester levels, depending on their size (see \tabref{Psionic Tattoo Areas}). A psionic tattoo that requires more manifester levels than an area can hold, it can be scribed into an adjacent area. For example, an augmented \psionic{offensive precognition} with 13th manifester level can be tattooed in one whole arm---hand, forearm and upper arm.

Attempting to add one more total psionic power levels than the maximum allowed in any area causes all previously scribed tattoos in your body to simultaneously resonate and explode, each tattoo dealing 2d10 per psionic power level. All tattoos are lost when this happens.

\Table{Psionic Tattoo Areas}{CC} {
  \tableheader Body Area
& \tableheader Total Power Level \\
Face        & 2 \\
Scalp       & 4 \\
Hand        & 1 each \\
Forearm     & 3 each \\
Upper arm   & 4 each \\
Chest       & 6 \\
Abdomen     & 6 \\
Back, upper & 10 \\
Back, lower & 4 \\
Leg, upper  & 8 each \\
Leg, lower  & 5 each \\
Foot        & 2 each \\
}

\textbf{Suppressing a Tattoo Effect:} A psionically tattooed character can suppress any tattoo effect with a standard action that provokes attacks of opportunity. The character must use one action for each tattoo. Reactivating a tattoo is another standard action that provokes attacks of opportunity.

% \Table{Psionic Tattoos (1st--9th ML)}{XZ{11mm}r{13mm}}{
\Table{Psion Tattoos I}{X r{13mm}}{
  \tableheader Psionic Power
& \tableheader Cost \\
%COST = (power level + DC modifier) * manifester level * 4000 * duration modifier (minute = 2, round = 4)
%MAX POWER LEVEL = 14

\psionic{detect remote viewing}                 &   4,000 cp \\ %  1   *  1 * 4000
\psionic{adapt body} (one environment)          &   8,000 cp \\ %  1   *  1 * 4000 * 2
\psionic{body equilibrium}                      &   8,000 cp \\ %  1   *  1 * 4000 * 2
\psionic{photosynthesis}                        &   8,000 cp \\ %  1   *  1 * 4000 * 2
\psionic{anchored navigation}                   &  16,000 cp \\ %  4   *  1 * 4000
\psionic{carapace}                              &  16,000 cp \\ %  1   *  1 * 4000 * 4
\psionic{chameleon}                             &  16,000 cp \\ %  2   *  1 * 4000 * 2
\psionic{psionic disguise self}                 &  16,000 cp \\ %  2   *  1 * 4000 * 2
\psionic{psionic spider climb}                  &  16,000 cp \\ %  2   *  1 * 4000 * 2
\psionic{danger sense} (uncanny dodge)          &  24,000 cp \\ %  3   *  1 * 4000 * 2
\psionic{precognition}\footnotemark[1]          &  24,000 cp \\ %  6   *  1 * 4000
\psionic{psionic levitate}                      &  24,000 cp \\ %  3   *  1 * 4000 * 2
\psionic{biofeedback}                           &  32,000 cp \\ %  2   *  1 * 4000 * 4
\psionic{duo-dimension}                         &  32,000 cp \\ %  2   *  1 * 4000 * 4
\psionic{energy adaptation}                     &  32,000 cp \\ %  4   *  1 * 4000 * 2
\psionic{expansion} (one size category)         &  32,000 cp \\ %  2   *  1 * 4000 * 4
\psionic{flesh armor} (+1)                      &  32,000 cp \\ %  4   *  1 * 4000 * 2
\psionic{personal haste}                        &  32,000 cp \\ %  2   *  1 * 4000 * 4
\psionic{reduction} (one size category)         &  32,000 cp \\ %  2   *  1 * 4000 * 4
\psionic{adapt body} (any environment)          &  40,000 cp \\ %(1+4) *  1 * 4000 * 2
\psionic{danger sense} (improved uncanny dodge) &  40,000 cp \\ %(3+2) *  1 * 4000 * 2
\psionic{flesh armor} (+2)                      &  40,000 cp \\ %(4+1) *  1 * 4000 * 2
\psionic{metamorphosis} (\onehalf HD)           &  40,000 cp \\ %  5   *  1 * 4000 * 2
\psionic{animal affinity}                       &  48,000 cp \\ %  3   *  1 * 4000 * 4
\psionic{energy adaptation} (5 resistance)      &  48,000 cp \\ %(4+2) *  1 * 4000 * 2
\psionic{flesh armor} (+3)                      &  48,000 cp \\ %(4+2) *  1 * 4000 * 2
\psionic{graft weapon}                          &  48,000 cp \\ %  3   *  1 * 4000 * 4
\psionic{psionic displacement}                  &  48,000 cp \\ %  3   *  1 * 4000 * 4
\psionic{synesthete}                            &  48,000 cp \\ %  3   *  1 * 4000 * 4
\psionic{teleport trigger}                      &  48,000 cp \\ %  2   *  6 * 4000
\psionic{ubiquitous vision}                     &  48,000 cp \\ %  3   *  1 * 4000 * 4
\psionic{combat mind} (1)                       &  64,000 cp \\ %  4   *  1 * 4000 * 4
\psionic{ectoplasmic form}                      &  64,000 cp \\ %  4   *  1 * 4000 * 4
\psionic{energy adaptation} (10 resistance)     &  64,000 cp \\ %(4+4) *  1 * 4000 * 2
\psionic{expansion} (two size categories)       &  64,000 cp \\ %(2+2) *  1 * 4000 * 4
\psionic{flesh armor} (+4)                      &  64,000 cp \\ %(4+4) *  1 * 4000 * 2
\psionic{mind bar}                              &  64,000 cp \\ %  4   *  1 * 4000 * 4
\psionic{psionic fly}                           &  64,000 cp \\ %  4   *  1 * 4000 * 4
\psionic{reduction} (two size categories)       &  64,000 cp \\ %(2+2) *  1 * 4000 * 4
\psionic{flesh armor} (+5)                      &  72,000 cp \\ %(4+5) *  1 * 4000 * 2
\psionic{energy adaptation} (20 resistance)     &  80,000 cp \\ %(4+6) *  1 * 4000 * 2
\psionic{immovability}                          &  80,000 cp \\ %  5   *  1 * 4000 * 4
\psionic{metamorphosis} (1 HD)                  &  80,000 cp \\ %  5   *  2 * 4000 * 2
\psionic{safe path} (+2 AC)                     &  80,000 cp \\ %  5   *  1 * 4000 * 4
\psionic{second chance}                         &  80,000 cp \\ %  5   *  1 * 4000 * 4
\psionic{shadow body}                           &  80,000 cp \\ %  5   *  1 * 4000 * 4
\psionic{flesh armor} (+6)                      &  88,000 cp \\ %(4+7) *  1 * 4000 * 2

\TableNote{2}{1 Dischargeable.}
}

\Table{Psion Tattoos II}{X r{13mm}}{
  \tableheader Psionic Power
& \tableheader Cost \\

\psionic{cannibalize}                           &  96,000 cp \\ %  1   *  6 * 4000 * 4
\psionic{energy adaptation} (30 resistance)     &  96,000 cp \\ %(4+8) *  1 * 4000 * 2
\psionic{flesh armor} (+7)                      &  96,000 cp \\ %(4+8) *  1 * 4000 * 2
\psionic{kinetic control}                       &  96,000 cp \\ %  6   *  1 * 4000 * 4
\psionic{psionic blink}                         &  96,000 cp \\ %  6   *  1 * 4000 * 4
\psionic{safe path} (+3 AC)                     &  96,000 cp \\ %(5+1) *  1 * 4000 * 4
\psionic{flesh armor} (+8)                      & 104,000 cp \\ %(4+9) *  1 * 4000 * 2
\psionic{safe path} (+4 AC)                     & 112,000 cp \\ %(5+2) *  1 * 4000 * 4
\psionic{safe path} (+5 AC)                     & 128,000 cp \\ %(5+3) *  1 * 4000 * 4
\psionic{spirit lore}                           & 128,000 cp \\ %  8   *  1 * 4000 * 4
\psionic{psionic true seeing}                   & 128,000 cp \\ %  8   *  1 * 4000 * 4
\psionic{safe path} (+6 AC)                     & 144,000 cp \\ %(5+4) *  1 * 4000 * 4
\psionic{metamorphosis} (2 HD)                  & 160,000 cp \\ %  5   *  4 * 4000 * 2
\psionic{safe path} (+7 AC)                     & 160,000 cp \\ %(5+5) *  1 * 4000 * 4
\psionic{safe path} (+8 AC)                     & 176,000 cp \\ %(5+6) *  1 * 4000 * 4
\psionic{psionic freedom of movement}           & 192,000 cp \\ %  4   *  6 * 4000 * 2
\psionic{safe path} (+9 AC)                     & 192,000 cp \\ %(5+7) *  1 * 4000 * 4
\psionic{psionic arcane sight}                  & 200,000 cp \\ %  5   *  5 * 4000 * 2
\psionic{safe path} (+10 AC)                    & 208,000 cp \\ %(5+8) *  1 * 4000 * 4
\psionic{safe path} (+11 AC)                    & 224,000 cp \\ %(5+9) *  1 * 4000 * 4
\psionic{metamorphosis} (3 HD)                  & 240,000 cp \\ %  5   *  6 * 4000 * 2
\psionic{elemental composition}                 & 288,000 cp \\ %  9   *  2 * 4000 * 4
\psionic{metamorphosis} (4 HD)                  & 320,000 cp \\ %  5   *  8 * 4000 * 2
\psionic{reddopsi}                              & 336,000 cp \\ %  7   *  6 * 4000 * 2
\psionic{metamorphosis} (5 HD)                  & 400,000 cp \\ %  5   * 10 * 4000 * 2
\psionic{metamorphosis} (6 HD)                  & 480,000 cp \\ %  5   * 12 * 4000 * 2
\psionic{metamorphosis} (7 HD)                  & 560,000 cp \\ %  5   * 14 * 4000 * 2
\psionic{metamorphosis} (8 HD)                  & 640,000 cp \\ %  5   * 16 * 4000 * 2
\psionic{metamorphosis} (9 HD)                  & 720,000 cp \\ %  5   * 18 * 4000 * 2
\psionic{metamorphosis} (10 HD)                 & 800,000 cp \\ %  5   * 20 * 4000 * 2
% \psionic{cosmic awareness} (3 m)                & 1,584,000 cp \\ %  9   * 11 * 4000 * 4
}

