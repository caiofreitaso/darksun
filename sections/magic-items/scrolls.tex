\subsectionA{Scrolls}
A scroll is a spell (or collection of spells) that has been stored in written form. A spell on a scroll can be used only once. The writing vanishes from the scroll when the spell is activated. Using a scroll is basically like casting a spell.

\textbf{Physical Description:} A scroll is a heavy sheet of fine vellum or high-quality paper. An area about 20 centimeters wide and 30 centimeters long is sufficient to hold one spell. The sheet is reinforced at the top and bottom with strips of leather slightly longer than the sheet is wide. A scroll holding more than one spell has the same width (about 20 centimeters) but is an extra 30 centimeters long for each extra spell. Scrolls that hold three or more spells are usually fitted with reinforcing rods at each end rather than simple strips of leather. A scroll has AC 9, 1 hit point, hardness 0, and a break DC of 8.

To protect it from wrinkling or tearing, a scroll is rolled up from both ends to form a double cylinder. (This also helps the user unroll the scroll quickly.) The scroll is placed in a tube of ivory, jade, leather, metal, or wood. Most scroll cases are inscribed with magic symbols which often identify the owner or the spells stored on the scrolls inside. The symbols often hide magic traps.

\textbf{Activation:} To activate a scroll, a spellcaster must read the spell written on it. Doing so involves several steps and conditions.

\textit{Decipher the Writing:} The writing on a scroll must be deciphered before a character can use it or know exactly what spell it contains. This requires a \spell{read magic} spell or a successful \skill{Spellcraft} check (DC 20 + spell level).

Deciphering a scroll to determine its contents does not activate its magic unless it is a specially prepared cursed scroll. A character can decipher the writing on a scroll in advance so that he or she can proceed directly to the next step when the time comes to use the scroll.

\textit{Activate the Spell:} Activating a scroll requires reading the spell from the scroll. The character must be able to see and read the writing on the scroll. Activating a scroll spell requires no material components or focus. (The creator of the scroll provided these when scribing the scroll.) Note that some spells are effective only when cast on an item or items. In such a case, the scroll user must provide the item when activating the spell. Activating a scroll spell is subject to disruption just as casting a normally prepared spell would be. Using a scroll is like casting a spell for purposes of arcane spell failure chance.

To have any chance of activating a scroll spell, the scroll user must meet the following requirements.

\begin{itemize*}
\item The spell must be of the correct type (arcane or divine). Arcane spellcasters (assassins and wizards) can only use scrolls containing arcane spells, and divine spellcasters (clerics, druids, and rangers) can only use scrolls containing divine spells. (The type of scroll a character creates is also determined by his or her class.)
\item The user must have the spell on his or her class list.
\item The user must have the requisite ability score.
\end{itemize*}

If the user meets all the requirements noted above, and her caster level is at least equal to the spell's caster level, she can automatically activate the spell without a check. If she meets all three requirements but her own caster level is lower than the scroll spell's caster level, then she has to make a caster level check (DC = scroll's caster level + 1) to cast the spell successfully. If she fails, she must make a DC 5 Wisdom check to avoid a mishap (see Scroll Mishaps, below). A natural roll of 1 always fails, whatever the modifiers.

\textit{Determine Effect:} A spell successfully activated from a scroll works exactly like a spell prepared and cast the normal way. Assume the scroll spell's caster level is always the minimum level required to cast the spell for the character who scribed the scroll (usually twice the spell's level, minus 1), unless the caster specifically desires otherwise.

The writing for an activated spell disappears from the scroll.

\textit{Scroll Mishaps:} When a mishap occurs, the spell on the scroll has a reversed or harmful effect. Possible mishaps are given below.

\begin{itemize*}
\item A surge of uncontrolled magical energy deals 1d6 points of damage per spell level to the scroll user.
\item Spell strikes the scroll user or an ally instead of the intended target, or a random target nearby if the scroll user was the intended recipient.
\item Spell takes effect at some random location within spell range.
\item Spell's effect on the target is contrary to the spell's normal effect.
\item The scroll user suffers some minor but bizarre effect related to the spell in some way. Most such effects should last only as long as the original spell's duration, or 2d10 minutes for instantaneous spells.
\item Some innocuous item or items appear in the spell's area.
\item Spell has delayed effect. Sometime within the next 1d12 hours, the spell activates. If the scroll user was the intended recipient, the spell takes effect normally. If the user was not the intended recipient, the spell goes off in the general direction of the original recipient or target, up to the spell's maximum range, if the target has moved away.
\end{itemize*}

Several arcane spells are different in level for wizards than they are for assassins. Such spells appear on the table at the level appropriate to a wizard (considered the default because assassins typically don't involve themselves in scribing scrolls).

Likewise, some divine spells are different in level for clerics and druids than they are for rangers. Such spells appear at the level appropriate to a cleric or druid (considered the default because rangers typically don't involve themselves in scribing scrolls).
% Several divine spells are different in level for clerics and druids than they are for rangers. Such spells appear at the level appropriate to a cleric or druid (considered the default because rangers typically don't involve themselves in scribing scrolls).

If a divine spell is cast at different levels by clerics and druids, it appears at the level appropriate to a cleric (considered the default choice between clerics and druids).

Many spells are either arcane or divine, depending on the class of the caster. Such spells appear on both lists at the level appropriate to the class of the arcane or divine caster. 