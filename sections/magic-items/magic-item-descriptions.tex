\section{Magic Item Descriptions}
Each general type of magic item gets an overall description, followed by descriptions of specific items.

General descriptions include notes on activation, random generation, and other material. The AC, hardness, hit points, and break DC are given for typical examples of some magic items. The AC assumes that the item is unattended and includes a $-5$ penalty for the item's effective Dexterity of 0. If a creature holds the item, use the creature's Dexterity modifier in place of the $-5$ penalty.

Some individual items, notably those that simply store spells and nothing else, don't get full-blown descriptions. Reference the spell's description for details, modified by the form of the item (potion, scroll, wand, and so on). Assume that the spell is cast at the minimum level required to cast it.

Items with full descriptions have their powers detailed, and each of the following topics is covered in notational form at the end of the description.

\textbf{Aura:} Most of the time, a detect magic spell will reveal the school of magic associated with a magic item and the strength of the aura an item emits. This information (when applicable) is given at the beginning of the item's notational entry. See the detect magic spell description for details.

\textbf{Caster Level:} The next item in a notational entry gives the caster level of the item, indicating its relative power. The caster level determines the item's saving throw bonus, as well as range or other level-dependent aspects of the powers of the item (if variable). It also determines the level that must be contended with should the item come under the effect of a dispel magic spell or similar situation. This information is given in the form ``CL x,'' where ``CL'' is an abbreviation for caster level and ``x'' is an ordinal number representing the caster level itself.

For potions, scrolls, and wands, the creator can set the caster level of an item at any number high enough to cast the stored spell and not higher than her own caster level. For other magic items, the caster level is determined by the creator. The minimum caster level is that which is needed to meet the prerequisites given.

\textbf{Prerequisites:} Certain requirements must be met in order for a character to create a magic item. These include feats, spells, and miscellaneous requirements such as level, alignment, and race or kind. The prerequisites for creation of an item are given immediately following the item's caster level.

A spell prerequisite may be provided by a character who has prepared the spell (or who knows the spell, in the case of a sorcerer or bard), or through the use of a spell completion or spell trigger magic item or a spell-like ability that produces the desired spell effect. For each day that passes in the creation process, the creator must expend one spell completion item or one charge from a spell trigger item if either of those objects is used to supply a prerequisite.

It is possible for more than one character to cooperate in the creation of an item, with each participant providing one or more of the prerequisites. In some cases, cooperation may even be necessary.

If two or more characters cooperate to create an item, they must agree among themselves who will be considered the creator for the purpose of determinations where the creator's level must be known. The character designated as the creator pays the XP required to make the item.

Typically, a list of prerequisites includes one feat and one or more spells (or some other requirement in addition to the feat).

When two spells at the end of a list are separated by ``or,'' one of those spells is required in addition to every other spell mentioned prior to the last two.

\textbf{Market Price:} This gold piece value, given following the word ``Price,'' represents the price someone should expect to pay to buy the item. The market price for an item that can be constructed with an item creation feat is usually equal to the base price plus the price for any components (material or XP).

\textbf{Cost to Create:} The next part of a notational entry is the cost in gp and XP to create the item, given following the word ``Cost.'' This information appears only for items with components (material or XP), which make their market prices higher than their base prices. The cost to create includes the costs derived from the base cost plus the costs of the components.

Items without components do not have a ``Cost'' entry. For them, the market price and the base price are the same. The cost in gp is \onehalf the market price, and the cost in XP is 1/25 the market price.

\textbf{Weight:} The notational entry for many wondrous items ends with a value for the item's weight. When a weight figure is not given, the item has no weight worth noting (for purposes of determining how much of a load a character can carry). 