\section{Magic Item Descriptions}
Each general type of magic item gets an overall description, followed by descriptions of specific items.

General descriptions include notes on activation, random generation, and other material. The AC, hardness, hit points, and break DC are given for typical examples of some magic items. The AC assumes that the item is unattended and includes a $-5$ penalty for the item's effective Dexterity of 0. If a creature holds the item, use the creature's Dexterity modifier in place of the $-5$ penalty.

Some individual items, notably those that simply store spells and nothing else, don't get full-blown descriptions. Reference the spell's description for details, modified by the form of the item (potion, scroll, wand, and so on). Assume that the spell is cast at the minimum level required to cast it.

Items with full descriptions have their powers detailed, and each of the following topics is covered in notational form at the end of the description.

\textbf{Aura:} Most of the time, a detect magic spell will reveal the school of magic associated with a magic item and the strength of the aura an item emits. This information (when applicable) is given at the beginning of the item's notational entry. See the detect magic spell description for details.

\textbf{Caster Level:} The next item in a notational entry gives the caster level of the item, indicating its relative power. The caster level determines the item's saving throw bonus, as well as range or other level-dependent aspects of the powers of the item (if variable). It also determines the level that must be contended with should the item come under the effect of a dispel magic spell or similar situation. This information is given in the form ``CL x,'' where ``CL'' is an abbreviation for caster level and ``x'' is an ordinal number representing the caster level itself.

For potions, scrolls, and wands, the creator can set the caster level of an item at any number high enough to cast the stored spell and not higher than her own caster level. For other magic items, the caster level is determined by the creator. The minimum caster level is that which is needed to meet the prerequisites given.

\textbf{Prerequisites:} Certain requirements must be met in order for a character to create a magic item. These include feats, spells, and miscellaneous requirements such as level, alignment, and race or kind. The prerequisites for creation of an item are given immediately following the item's caster level.

A spell prerequisite may be provided by a character who has prepared the spell (or who knows the spell, in the case of a sorcerer or bard), or through the use of a spell completion or spell trigger magic item or a spell-like ability that produces the desired spell effect. For each day that passes in the creation process, the creator must expend one spell completion item or one charge from a spell trigger item if either of those objects is used to supply a prerequisite.

It is possible for more than one character to cooperate in the creation of an item, with each participant providing one or more of the prerequisites. In some cases, cooperation may even be necessary.

If two or more characters cooperate to create an item, they must agree among themselves who will be considered the creator for the purpose of determinations where the creator's level must be known. The character designated as the creator pays the XP required to make the item.

Typically, a list of prerequisites includes one feat and one or more spells (or some other requirement in addition to the feat).

When two spells at the end of a list are separated by ``or,'' one of those spells is required in addition to every other spell mentioned prior to the last two.

\textbf{Market Price:} This gold piece value, given following the word ``Price,'' represents the price someone should expect to pay to buy the item. The market price for an item that can be constructed with an item creation feat is usually equal to the base price plus the price for any components (material or XP).

\textbf{Cost to Create:} The next part of a notational entry is the cost in cp and XP to create the item, given following the word ``Cost.'' This information appears only for items with components (material or XP), which make their market prices higher than their base prices. The cost to create includes the costs derived from the base cost plus the costs of the components.

Items without components do not have a ``Cost'' entry. For them, the market price and the base price are the same. The cost in cp is \onehalf the market price, and the cost in XP is 1/25 the market price.

\textbf{Weight:} The notational entry for many wondrous items ends with a value for the item's weight. When a weight figure is not given, the item has no weight worth noting (for purposes of determining how much of a load a character can carry). 

\input{subsections/magic-items/magic-item-descriptions/armors.tex}
\section{Weapons}
Magic weapons have enhancement bonuses ranging from +1 to +5. They apply these bonuses to both attack and damage rolls when used in combat. All magic weapons are also masterwork weapons, but their masterwork bonus on attack rolls does not stack with their enhancement bonus on attack rolls.

Weapons come in two basic categories: melee and ranged. Some of the weapons listed as melee weapons can also be used as ranged weapons. In this case, their enhancement bonus applies to either type of attack.

In addition to an enhancement bonus, weapons may have special abilities. Special abilities count as additional bonuses for determining the market value of the item, but do not modify attack or damage bonuses (except where specifically noted). A single weapon cannot have a modified bonus (enhancement bonus plus special ability bonus equivalents) higher than +10. A weapon with a special ability must have at least a +1 enhancement bonus.

A weapon or a kind of ammunition may be made of an unusual material. Roll d\%: 01--95 indicates that the item is of a standard sort, and 96--100 indicates that it is made of a special material.

\textbf{Caster Level for Weapons:} The caster level of a weapon with a special ability is given in the item description. For an item with only an enhancement bonus and no other abilities, the caster level is three times the enhancement bonus. If an item has both an enhancement bonus and a special ability, the higher of the two caster level requirements must be met.

\textbf{Additional Damage Dice:} Some magic weapons deal additional dice of damage. Unlike other modifiers to damage, additional dice of damage are not multiplied when the attacker scores a critical hit.

\textbf{Ranged Weapons and Ammunition:} The enhancement bonus from a ranged weapon does not stack with the enhancement bonus from ammunition. Only the higher of the two enhancement bonuses applies.

Ammunition fired from a projectile weapon with an enhancement bonus of +1 or higher is treated as a magic weapon for the purpose of overcoming damage reduction. Similarly, ammunition fired from a projectile weapon with an alignment gains the alignment of that projectile weapon (in addition to any alignment it may already have).

\textbf{Magic Ammunition and Breakage:} When a magic arrow, crossbow bolt, shuriken, or sling bullet misses its target, there is a 50\% chance it breaks or otherwise is rendered useless. A magic arrow, bolt, bullet, or shuriken that hits is destroyed.

\textbf{Light Generation:} Fully 30\% of magic weapons shed light equivalent to a light spell (bright light in a 6-meter radius, shadowy light in a 12-meter radius). These glowing weapons are quite obviously magical. Such a weapon can't be concealed when drawn, nor can its light be shut off. Some of the specific weapons detailed below always or never glow, as defined in their descriptions.

\textbf{Hardness and Hit Points:} Each +1 of enhancement bonus adds 2 to a weapon's or shield's hardness and +10 to its hit points.

\textbf{Activation:} Usually a character benefits from a magic weapon in the same way a character benefits from a mundane weapon---by attacking with it. If a weapon has a special ability that the user needs to activate then the user usually needs to utter a command word (a standard action).

\textbf{Magic Weapons and Critical Hits:} Some weapon qualities and some specific weapons have an extra effect on a critical hit. This special effect functions against creatures not subject to critical hits, such as undead, elementals, and constructs. When fighting against such creatures, roll for critical hits as you would against humanoids or any other creature subject to critical hits. On a successful critical roll, apply the special effect, but do not multiply the weapon's regular damage.

\textbf{Weapons for Unusually Sized Creatures:} The cost of weapons for creatures who are neither Small nor Medium varies. The cost of the masterwork quality and any magical enhancement remains the same.

\textbf{Special Qualities:} Roll d\%. If the item is a melee weapon, a 01--30 result indicates that the item sheds light, 31--45 indicates that something (a design, inscription, or the like) provides a clue to the weapon's function, and 46--100 indicates no special qualities.

If the item is a ranged weapon, a 01--15 result indicates that something (a design, inscription, or the like) provides a clue to the weapon's function, and 16--100 indicates no special qualities.

\subsectionA{Potions and Oils}
A potion is a magic liquid that produces its effect when imbibed. Magic oils are similar to potions, except that oils are applied externally rather than imbibed. A potion or oil can be used only once. It can duplicate the effect of a spell of up to 3rd level that has a casting time of less than 1 minute.

Potions are like spells cast upon the imbiber. The character taking the potion doesn't get to make any decisions about the effect---the caster who brewed the potion has already done so. The drinker of a potion is both the effective target and the caster of the effect (though the potion indicates the caster level, the drinker still controls the effect).

The person applying an oil is the effective caster, but the object is the target.

\textbf{Physical Description:} A typical potion or oil consists of 30 mililiters of liquid held in a ceramic or glass vial fitted with a tight stopper. The stoppered container is usually no more than 1 inch wide and 2 inches high. The vial has AC 13, 1 hit point, hardness 1, and a break DC of 12. Vials hold 30 mililiters of liquid.

\textbf{Identifying Potions:} In addition to the standard methods of identification, PCs can sample from each container they find to attempt to determine the nature of the liquid inside. An experienced character learns to identify potions by memory---for example, the last time she tasted a liquid that reminded her of almonds, it turned out to be a potion of cure moderate wounds.

\textbf{Activation:} Drinking a potion or applying an oil requires no special skill. The user merely removes the stopper and swallows the potion or smears on the oil. The following rules govern potion and oil use.

Drinking a potion or using an oil on an item of gear is a standard action. The potion or oil takes effect immediately. Using a potion or oil provokes attacks of opportunity. A successful attack (including grappling attacks) against the character forces a Concentration check (as for casting a spell). If the character fails this check, she cannot drink the potion. An enemy may direct an attack of opportunity against the potion or oil container rather than against the character. A successful attack of this sort can destroy the container.

A creature must be able to swallow a potion or smear on an oil. Because of this, incorporeal creatures cannot use potions or oils.

Any corporeal creature can imbibe a potion. The potion must be swallowed. Any corporeal creature can use an oil.

A character can carefully administer a potion to an unconscious creature as a full-round action, trickling the liquid down the creature's throat. Likewise, it takes a full-round action to apply an oil to an unconscious creature.

\subsubsection{Potion Descriptions}
The caster level for a standard potion is the minimum caster level needed to cast the spell (unless otherwise specified).

\Table{Potions and Oils I}{Xr{15mm}} {
\tableheader Potion or Oil & \tableheader Market Price \\
\spell{cure light wounds} (potion)                & 50 cp \\
\spell{endure elements} (potion)                  & 50 cp \\
\spell{hide from animals} (potion)                & 50 cp \\
\spell{hide from undead} (potion)                 & 50 cp \\
\spell{jump} (potion)                             & 50 cp \\
\spell{mage armor} (potion)                       & 50 cp \\
\spell{magic fang} (potion)                       & 50 cp \\
\spell{magic stone} (oil)                         & 50 cp \\
\spell{magic weapon} (oil)                        & 50 cp \\
\spell{pass without trace} (potion)               & 50 cp \\
 \emph{protection from (alignment)} (potion)      & 50 cp \\
\spell{remove fear} (potion)                      & 50 cp \\
\spell{sanctuary} (potion)                        & 50 cp \\
\spell{shield of faith} +2 (potion)               & 50 cp \\
\spell{shillelagh} (oil)                          & 50 cp \\
\spell{bless weapon} (oil)                        & 100 cp \\
\spell{enlarge person} (potion)                   & 250 cp \\
\spell{reduce person} (potion)                    & 250 cp \\
\spell{aid} (potion)                              & 300 cp \\
\spell{barkskin} +2 (potion)                      & 300 cp \\
\spell{bear's endurance} (potion)                 & 300 cp \\
\spell{blur} (potion)                             & 300 cp \\
\spell{bull's strength} (potion)                  & 300 cp \\
\spell{cat's grace} (potion)                      & 300 cp \\
\spell{cure moderate wounds} (potion)             & 300 cp \\
\spell{darkness} (oil)                            & 300 cp \\
\spell{darkvision} (potion)                       & 300 cp \\
\spell{delay poison} (potion)                     & 300 cp \\
\spell{eagle's splendor} (potion)                 & 300 cp \\
\spell{fox's cunning} (potion)                    & 300 cp \\
\spell{invisibility} (potion or oil)              & 300 cp \\
\spell{lesser restoration} (potion)               & 300 cp \\
\spell{levitate} (potion or oil)                  & 300 cp \\
\spell{misdirection} (potion)                     & 300 cp \\
\spell{owl's wisdom} (potion)                     & 300 cp \\
\spell{protection from arrows} 10/magic (potion)  & 300 cp \\
\spell{remove paralysis} (potion)                 & 300 cp \\
\spell{resist energy} (type) 10 (potion)          & 300 cp \\
\spell{shield of faith} +3 (potion)               & 300 cp \\
\spell{spider climb} (potion)                     & 300 cp \\
\spell{undetectable alignment} (potion)           & 300 cp \\
\spell{barkskin} +3 (potion)                      & 600 cp \\
\spell{shield of faith} +4 (potion)               & 600 cp \\
\spell{resist energy} (type) 20 (potion)          & 700 cp \\
}

\Table{Potions and Oils II}{Xr{15mm}} {
\tableheader Potion or Oil & \tableheader Market Price \\
\spell{cure serious wounds} (potion)              & 750 cp \\
\spell{daylight} (oil)                            & 750 cp \\
\spell{displacement} (potion)                     & 750 cp \\
\spell{flame arrow} (oil)                         & 750 cp \\
\spell{fly} (potion)                              & 750 cp \\
\spell{gaseous form} (potion)                     & 750 cp \\
\spell{greater magic fang} +1 (potion)            & 750 cp \\
\spell{greater magic weapon} +1 (oil)             & 750 cp \\
\spell{haste} (potion)                            & 750 cp \\
\spell{heroism} (potion)                          & 750 cp \\
\spell{keen edge} (oil)                           & 750 cp \\
 \emph{magic circle against (alignment)} (potion) & 750 cp \\
\spell{magic vestment} +1 (oil)                   & 750 cp \\
\spell{neutralize poison} (potion)                & 750 cp \\
\spell{nondetection} (potion)                     & 750 cp \\
\spell{protection from energy} (type) (potion)    & 750 cp \\
\spell{rage} (potion)                             & 750 cp \\
\spell{remove blindness/deafness} (potion)        & 750 cp \\
\spell{remove curse} (potion)                     & 750 cp \\
\spell{remove disease} (potion)                   & 750 cp \\
\spell{tongues} (potion)                          & 750 cp \\
\spell{water breathing} (potion)                  & 750 cp \\
\spell{water walk} (potion)                       & 750 cp \\
\spell{barkskin} +4 (potion)                      & 900 cp \\
\spell{shield of faith} +5 (potion)               & 900 cp \\
\spell{good hope} (potion)                        & 1,050 cp \\
\spell{resist energy} (type) 30 (potion)          & 1,100 cp \\
\spell{barkskin} +5 (potion)                      & 1,200 cp \\
\spell{greater magic fang} +2 (potion)            & 1,200 cp \\
\spell{greater magic weapon} +2 (oil)             & 1,200 cp \\
\spell{magic vestment} +2 (oil)                   & 1,200 cp \\
\spell{protection from arrows} 15/magic (potion)  & 1,500 cp \\
\spell{greater magic fang} +3 (potion)            & 1,800 cp \\
\spell{greater magic weapon} +3 (oil)             & 1,800 cp \\
\spell{magic vestment} +3 (oil)                   & 1,800 cp \\
\spell{greater magic fang} +4 (potion)            & 2,400 cp \\
\spell{greater magic weapon} +4 (oil)             & 2,400 cp \\
\spell{magic vestment} +4 (oil)                   & 2,400 cp \\
\spell{greater magic fang} +5 (potion)            & 3,000 cp \\
\spell{greater magic weapon} +5 (oil)             & 3,000 cp \\
\spell{magic vestment} +5 (oil)                   & 3,000 cp \\
}

\subsectionA{Magic Rings}
Rings bestow magical powers upon their wearers. Only a rare few have charges. Anyone can use a ring.

A character can only effectively wear two magic rings. A third magic ring doesn't work if the wearer is already wearing two magic rings.

\textbf{Physical Description:} Rings have no appreciable weight. Although exceptions exist that are forged from metal---even precious metals such as gold, silver, and platinum---the vast majority of rings are crafted from glass, wood or bone. A ring has AC 13, 2 hit points, hardness 10, and a break DC of 25.

\textbf{Activation:} Usually, a ring's ability is activated by a command word (a standard action that does not provoke attacks of opportunity) or it works continually. Some rings have exceptional activation methods, according to their descriptions.

\Table{Rings}{Xr{15mm}}{
\tableheader Ring & \tableheader Market Price \\
Telepathic defense +1      & 1,000 cp \\
Protection +1              & 2,000 cp \\
Telepathic attack +1       & 2,000 cp \\
Feather falling            & 2,200 cp \\
Sustenance                 & 2,500 cp \\
Climbing                   & 2,500 cp \\
Jumping                    & 2,500 cp \\
% Swimming                   & 2,500 cp \\
Counterspells              & 4,000 cp \\
Telepathic defense +2      & 4,000 cp \\
Mind shielding             & 8,000 cp \\
Protection +2              & 8,000 cp \\
Telepathic attack +2       & 8,000 cp \\
Force shield               & 8,500 cp \\
Ram                        & 8,600 cp \\
Telepathic defense +3      & 9,000 cp \\
Climbing, improved         & 10,000 cp \\
Jumping, improved          & 10,000 cp \\
% Swimming, improved         & 10,000 cp \\
Animal friendship          & 10,800 cp \\
Energy resistance, minor   & 12,000 cp \\
Chameleon power            & 12,700 cp \\
Silt walking               & 15,000 cp \\
% Water walking              & 15,000 cp \\
Telepathic defense +4      & 16,000 cp \\
Protection +3              & 18,000 cp \\
Telepathic attack +3       & 18,000 cp \\
Spell storing, minor       & 18,000 cp \\
Invisibility               & 20,000 cp \\
Wizardry (I)               & 20,000 cp \\
Evasion                    & 25,000 cp \\
Telepathic defense +5      & 25,000 cp \\
X-ray vision               & 25,000 cp \\
Blinking                   & 27,000 cp \\
Meld into Stone            & 27,000 cp \\
Energy resistance, major   & 28,000 cp \\
Protection +4              & 32,000 cp \\
Telepathic attack +4       & 32,000 cp \\
Inanimate friend           & 36,400 cp \\
Wizardry (II)              & 40,000 cp \\
Freedom of movement        & 40,000 cp \\
Energy resistance, greater & 44,000 cp \\
Friend shield (pair)       & 50,000 cp \\
Protection +5              & 50,000 cp \\
Shooting stars             & 50,000 cp \\
Spell storing              & 50,000 cp \\
Telepathic attack +5       & 50,000 cp \\
Wizardry (III)             & 70,000 cp \\
Telekinesis                & 75,000 cp \\
Regeneration               & 90,000 cp \\
Three wishes               & 97,950 cp \\
Spell turning              & 98,280 cp \\
Wizardry (IV)              & 100,000 cp \\
Djinni calling             & 125,000 cp \\
Life                       & 132,000 cp \\
% Elemental command (air)    & 200,000 cp \\
% Elemental command (earth)  & 200,000 cp \\
% Elemental command (fire)   & 200,000 cp \\
% Elemental command (water)  & 200,000 cp \\
Spell storing, major       & 200,000 cp \\
}

\subsubsection{Ring Descriptions}
Standard rings are described below.


\textbf{Animal Friendship:} On command, this ring affects an animal as if the wearer had cast \spell{charm animal}.

Faint enchantment; CL 3rd; \feat{Forge Ring}, \spell{charm animal}; Price 10,800 cp.


\textbf{Blinking:} On command, this ring makes the wearer blink, as with the \spell{blink} spell.

Moderate transmutation; CL 7th; \feat{Forge Ring}, \spell{blink}; Price 27,000 cp.


\textbf{Chameleon Power:} As a free action, the wearer of this ring can gain the ability to magically blend in with the surroundings. This provides a +10 competence bonus on her \skill{Hide} checks. As a standard action, she can also command the ring to utilize the spell \spell{disguise self} as often as she wants.

Faint illusion; CL 3rd; \feat{Forge Ring}, \spell{disguise self}, \spell{invisibility}; Price 12,700 cp.


\textbf{Climbing:} This ring is actually a magic leather cord that ties around a finger. It continually grants the wearer a +5 competence bonus on \skill{Climb} checks.

Faint transmutatation; CL 5th; \feat{Forge Ring}, creator must have 5 ranks in the \skill{Climb} skill; Price 2,500 cp.


\textbf{Climbing, Improved:} As \emph{climbing}, except it grants a +10 competence bonus on its wearer's \skill{Climb} checks.

Faint transmutation; CL 5th; \feat{Forge Ring}, creator must have 10 ranks in the \skill{Climb} skill; Price 10,000 cp.


\textbf{Counterspells:} This ring might seem to be a \emph{ring of spell storing} upon first examination. However, while it allows a single spell of 1st through 6th level to be cast into it, that spell cannot be cast out of the ring again. Instead, should that spell ever be cast upon the wearer, the spell is immediately countered, as a counterspell action, requiring no action (or even knowledge) on the wearer's part. Once so used, the spell cast within the ring is gone. A new spell (or the same one as before) may be placed in it again.

Moderate evocation; CL 11th; \feat{Forge Ring}, \spell{imbue with spell ability}; Price 4,000 cp.


\textbf{Djinni Calling:} One of the many rings of fable, this ``genie'' ring is most useful indeed. It serves as a special \spell{gate} by means of which a specific djinni can be called from the Elemental Plane of Air. When the ring is rubbed (a standard action), the call goes out, and the djinni appears on the next round. The djinni faithfully obeys and serves the wearer of the ring, but never for more than 1 hour per day. If the djinni of the ring is ever killed, the ring becomes nonmagical and worthless.

Strong conjuration; CL 17th; \feat{Forge Ring}, \spell{gate}; Price 125,000 cp.


% \textbf{Elemental Command:} All four kinds of elemental command rings are very powerful. Each appears to be nothing more than a lesser magic ring until fully activated (by meeting a special condition, such as single-handedly slaying an elemental of the appropriate type or exposure to a sacred material of the appropriate element), but each has certain other powers as well as the following common properties.

% Elementals of the plane to which the ring is attuned can't attack the wearer, or even approach within 1.5 meter of him. If the wearer desires, he may forego this protection and instead attempt to charm the elemental (as \spell{charm monster}, Will DC 17 negates). If the charm attempt fails, however, absolute protection is lost and no further attempt at charming can be made.

% Creatures from the plane to which the ring is attuned who attack the wearer take a $-1$ penalty on their attack rolls. The ring wearer makes applicable saving throws from the extraplanar creature's attacks with a +2 resistance bonus. He gains a +4 morale bonus on all attack rolls against such creatures. Any weapon he uses bypasses the damage reduction of such creatures, regardless of any qualities the weapon may or may not have.

% The wearer of the ring is able to converse with creatures from the plane to which his ring is attuned. These creatures recognize that he wears the ring. They show a healthy respect for the wearer if alignments are similar. If alignments are opposed, creatures fear the wearer if he is strong. If he is weak, they hate and desire to slay him.

% The possessor of a ring of elemental command takes a saving throw penalty as follows:

% \Table{}{lX}{
% \tableheader Element & \tableheader Saving Throw Penalty \\
% Air   & $-2$ against earth-based effects \\
% Earth & $-2$ against air- or electricity-based effects \\
% Fire  & $-2$ against water- or cold-based effects \\
% Water & $-2$ against fire-based effects \\
% }

% In addition to the powers described above, each specific ring gives its wearer the following abilities according to its kind.

% \textit{Ring of Elemental Command (Air):}
% \begin{itemize*}
% \item \spell{Feather Fall} (unlimited use, wearer only)
% \item \spell{Resist Energy} (electricity) (unlimited use, wearer only)
% \item \spell{Gust of Wind} (twice per day)
% \item \spell{Wind Wall} (unlimited use)
% \item \spell{Air Walk} (once per day, wearer only)
% \item \spell{Chain Lightning} (once per week)
% \end{itemize*}

% The ring appears to be a \emph{ring of feather falling} until a certain condition is met to activate its full potential. It must be reactivated each time a new wearer acquires it.

% \textit{Ring of Elemental Command (Earth):}
% \begin{itemize*}
% \item \spell{Meld into Stone} (unlimited use, wearer only)
% \item \spell{Soften Earth and Stone} (unlimited use)
% \item \spell{Stone Shape} (twice per day)
% \item \spell{Stoneskin} (once per week, wearer only)
% \item \spell{Passwall} (twice per week)
% \item \spell{Wall of Stone} (once per day)
% \end{itemize*}

% The ring appears to be a \emph{ring of meld into stone} until the established condition is met.

% \textit{Ring of Elemental Command (Fire):}
% \begin{itemize*}
% \item \spell{Resist Energy} (fire) (as a major ring of energy resistance [fire])
% \item \spell{Burning Hands} (unlimited use)
% \item \spell{Flaming Sphere} (twice per day)
% \item \spell{Pyrotechnics} (twice per day)
% \item \spell{Wall of Fire} (once per day)
% \item \spell{Flame Strike} (twice per week)
% \end{itemize*}

% The ring appears to be a \emph{major ring of energy resistance (fire)} until the established condition is met.

% \textit{Ring of Elemental Command (Water):}
% \begin{itemize*}
% \item \spell{Water Walk} (unlimited use)
% \item \spell{Create Water} (unlimited use)
% \item \spell{Water Breathing} (unlimited use)
% \item \spell{Wall of Ice} (once per day)
% \item \spell{Ice Storm} (twice per week)
% \item \spell{Control Water} (twice per week)
% \end{itemize*}

% The ring appears to be a \emph{ring of water walking} until the established condition is met.

% Strong conjuration; CL 15th; \feat{Forge Ring}, \spell{summon monster VI}, all appropriate spells; Price 200,000 cp.


\textbf{Energy Resistance:} This reddish iron ring continually protects the wearer from damage from one type of energy---acid, cold, electricity, fire, or sonic (chosen by the creator of the item; determine randomly if found as part of a treasure hoard). Each time the wearer would normally take such damage, subtract the ring's resistance value from the damage dealt.

A \emph{minor ring of energy resistance} grants 10 points of resistance. A \emph{major ring of energy resistance} grants 20 points of resistance. A \emph{greater ring of energy resistance} grants 30 points of resistance.

Faint (minor or major) or moderate (greater) abjuration; CL 3rd (minor), 7th (major), or 11th (greater); \feat{Forge Ring}, \spell{resist energy}; Price 12,000 cp (minor), 28,000 cp (major), 44,000 cp (greater).


\textbf{Evasion:} This ring continually grants the wearer the ability to avoid damage as if she had evasion. Whenever she makes a Reflex saving throw to determine whether she takes half damage, a successful save results in no damage.

Moderate transmutation; CL 7th; \feat{Forge Ring}, \spell{jump}; Price 25,000 cp.


\textbf{Feather Falling:} This ring is crafted with a feather pattern all around its edge. It acts exactly like a \spell{feather fall} spell, activated immediately if the wearer falls more than 1.5 meter.

Faint transmutation; CL 1st; \feat{Forge Ring}, \spell{feather fall}; Price 2,200 cp.


\textbf{Force Shield:} An iron band, this simple ring generates a shield-sized (and shield-shaped) \spell{wall of force} that stays with the ring and can be wielded by the wearer as if it were a heavy shield (+2 AC). This special creation has no armor check penalty or arcane spell failure chance since it is weightless and encumbrance-free. It can be activated and deactivated at will as a free action.

Moderate evocation; CL 9th; \feat{Forge Ring}, \spell{wall of force}; Price 8,500 cp.


\textbf{Freedom of Movement:} This gold ring allows the wearer to act as if continually under the effect of a \spell{freedom of movement} spell.

Moderate abjuration; CL 7th; \feat{Forge Ring}, \spell{freedom of movement}; Price 40,000 cp.


\textbf{Friend Shield:} These curious rings always come in pairs. A \emph{friend shield ring} without its mate is useless. Either wearer of one of a pair of the rings can, at any time, command his or her ring to cast a \spell{shield other} spell with the wearer of the mated ring as the recipient. This effect has no range limitation.

Moderate abjuration; CL 10th; \feat{Forge Ring}, \spell{shield other}; Price 50,000 cp (for a pair).


\textbf{Inanimate Friend:} The wearer of this ring can instantaneously transform his familiar into a sandstone statuette by touching it and uttering a command word. The effect is similar to the \spell{statue} spell and can be used up to three times per day, lasting for 1 hour per use. The ring's effect can be dismissed before the duration expires by touching the statuette to the ring (a move-equivalent action). During the time spent as a \spell{statue} the familiar confers no benefits to its master and loses its familiar special abilities.

Strong transmutation; CL 13th; \feat{Forge Ring}, \spell{statue}; Price 36,400 cp.


\textbf{Invisibility:} By activating this simple silver ring, the wearer can benefit from \spell{invisibility}, as the spell.

Faint illusion; CL 3rd; \feat{Forge Ring}, \spell{invisibility}; Price 20,000 cp.


\textbf{Jumping:} This ring continually allows the wearer to leap about, providing a +5 competence bonus on all his \skill{Jump} checks.

Faint transmutation; CL 2nd; \feat{Forge Ring}, creator must have 5 ranks in the \skill{Jump} skill; Price 2,500 cp.


\textbf{Jumping, Improved:} As \emph{jumping}, except it grants a +10 competence bonus on its wearer's \skill{Jump} check.

Moderate transmutation; CL 7th; \feat{Forge Ring}, creator must have 10 ranks in the \skill{Jump} skill; Price 10,000 cp.


\textbf{Life:} This ring, made of aviarag ivory, protects the wearer from the effects of being caught in the defiling radius of a spellcasting defiler. The wearer is immune to all penalties associated with the defiling radius, even when augmented with Raze feats, magical items, or class abilities. Each morning, regardless of the wearer's activities during the past 24 hours, he regains 1 hit point per Hit Die as if he had slept for a full 8 hours, up to his maximum normal hit point total.

This benefit of the ring does not negate any other penalties associated with the previous day's activities, nor does it allow the character to memorize spells or regain power points as if they had slept for a full 8 hours

Moderate abjuration; CL 11th; \feat{Forge Ring}, \spell{allegiance of the land}; Price 132,000 cp.


\textbf{Meld into Stone:} This ring allows the wearer to use the spell \spell{meld into stone} on command.

Faint enchantment; CL 5th; \feat{Forge Ring}, \spell{meld into stone}; Price 27,000 cp.


\textbf{Mind Shielding:} This ring is usually of fine workmanship and wrought from heavy gold. The wearer is continually immune to \spell{detect thoughts}, \spell{discern lies}, and any attempt to magically discern her alignment.

Faint abjuration; CL 3rd; \feat{Forge Ring}, \spell{nondetection}; Price 8,000 cp.


\textbf{Protection:} This ring offers continual magical protection in the form of a deflection bonus of +1 to +5 to AC.

Faint abjuration; CL 5th; \feat{Forge Ring}, \spell{shield of faith}, caster must be of a level at least three times greater than the bonus of the ring; Price 2,000 cp (ring +1); 8,000 cp (ring +2); 18,000 cp (ring +3); 32,000 cp (ring +4); 50,000 cp (ring +5).


\textbf{Ram:} The \emph{ring of the ram} is an ornate ring forged of hard metal, usually iron or an iron alloy. It has the head of a ram as its device.

The wearer can command the ring to give forth a ramlike force, manifested as a vaguely discernible shape that resembles the head of a ram or a goat. This force strikes a single target, dealing 1d6 points of damage if 1 charge is expended, 2d6 points if 2 charges are used, or 3d6 points if 3 charges (the maximum) are used. Treat this as a ranged attack with a 15-meter maximum range and no penalties for distance.

The force of the blow is considerable, and those struck by the ring are subject to a bull rush if within 9 meters of the ring-wearer. (The ram has Strength 25 and is Large.) The ram gains a +1 bonus on the bull rush attempt if 2 charges are expended, or +2 if 3 charges are expended.

In addition to its attack mode, the \emph{ring of the ram} also has the power to open doors as if it were a character with Strength 25. If 2 charges are expended, the effect is equivalent to a character with Strength 27. If 3 charges are expended, the effect is that of a character with Strength 29.

A newly created ring has 50 charges. When all the charges are expended, the ring becomes a nonmagical item.

Moderate transmutation; CL 9th; \feat{Forge Ring}, \spell{bull's strength}, \spell{telekinesis}; Price 8,600 cp.


\textbf{Regeneration:} This white gold ring continually allows a living wearer to heal 1 point of damage per level every hour rather than every day. (This ability cannot be aided by the \skill{Heal} skill.) Nonlethal damage heals at a rate of 1 point of damage per level every 5 minutes. If the wearer loses a limb, an organ, or any other body part while wearing this ring, the ring \emph{regenerates} it as the spell. In either case, only damage taken while wearing the ring is regenerated.

Strong conjuration; CL 15th; \feat{Forge Ring}, \spell{regenerate}; Price 90,000 cp.


\textbf{Shooting Stars:} This ring has two modes of operation, one for being in shadowy darkness or outdoors at night and a second one when the wearer is underground or indoors at night.

During the night under the open sky or in areas of shadow or darkness, the \emph{ring of shooting stars} can perform the following functions on command.

\begin{itemize*}
\item \spell{Dancing Lights} (once per hour)
\item \spell{Light} (twice per night)
\item \spell{Ball lightning} (special, once per night)
\item \spell{Shooting stars} (special, three per week)
\end{itemize*}

The first special function, \emph{ball lightning}, releases one to four balls of lightning (ring wearer's choice). These glowing globes resemble dancing lights, and the ring wearer controls them in the same fashion (see the dancing lights spell description). The spheres have a 36-meter range and a duration of 4 rounds. They can be moved at 36 meters per round. Each sphere is about 1 meter in diameter, and any creature who comes within 1.5 meter of one causes its charge to dissipate, taking electricity damage in the process according to the number of balls created.

\Table{}{XX}{
\tableheader Number of Balls & \tableheader Damage per Ball \\
4 lightning balls & 1d6 points of damage each \\
3 lightning balls & 2d6 points of damage each \\
2 lightning balls & 3d6 points of damage each \\
1 lightning ball  & 4d6 points of damage \\
}

Once the \emph{ball lightning} function is activated, the balls can be released at any time before the sun rises. (Multiple balls can be released in the same round.)

The second special function produces three shooting stars that can be released from the ring each week, simultaneously or one at a time. They impact for 12 points of damage and spread (as a \spell{fireball}) in a 1.5-meter-radius sphere for 24 points of fire damage.

Any creature struck by a \emph{shooting star} takes full damage from impact plus full fire damage from the spread unless it makes a DC 13 Reflex save. Creatures not struck but within the spread ignore the impact damage and take only half damage from the fire spread on a successful DC 13 Reflex save. Range is 21 meters, at the end of which the \emph{shooting star} explodes, unless it strikes a creature or object before that. A \emph{shooting star} always follows a straight line, and any creature in its path must make a save or be hit by the projectile.

Indoors at night, or underground, the \emph{ring of shooting stars} has the following properties.

\begin{itemize*}
\item \spellref{faerie fire}{Faerie fire} (twice per day)
\item \emph{Spark shower} (special, once per day)
\end{itemize*}

The \emph{spark shower} is a flying cloud of sizzling purple sparks that fan out from the ring for a distance of 6 meters in an arc 3 meters wide. Creatures within this area take 2d8 points of damage each if not wearing metal armor or carrying a metal weapon. Those wearing metal armor and/or carrying a metal weapon take 4d8 points of damage.

Strong evocation; CL 12th; \feat{Forge Ring}, \spell{light}, \spell{faerie fire}, \spell{fireball}, \spell{lightning bolt}; Price 50,000 cp.


\textbf{Silt Walking:} This ring, carved from a silt drake tooth, allows the wearer to continually utilize the effects of the \spell{surface walk} spell.

Moderate transmutation; CL 9th; \feat{Forge Ring}, \spell{surface walk}; Price 15,000 cp.


\textbf{Spell Storing, Minor:} A \emph{minor ring of spell storing} contains up to three levels of spells that the wearer can cast. Each spell has a caster level equal to the minimum level needed to cast that spell. The user need not provide any material components or focus, or pay an XP cost to cast the spell, and there is no arcane spell failure chance for wearing armor (because the ring wearer need not gesture). The activation time for the ring is same as the casting time for the relevant spell, with a minimum of 1 standard action.

For a randomly generated ring, treat it as a scroll to determine what spells are stored in it. If you roll a spell that would put the ring over the three-level limit, ignore that roll; the ring has no more spells in it. (Not every newly discovered ring need be fully charged.)

A spellcaster can cast any spells into the ring, so long as the total spell levels do not add up to more than three. Metamagic versions of spells take up storage space equal to their spell level modified by the metamagic feat. A spellcaster can use a scroll to put a spell into the \emph{minor ring of spell storing}.

The ring magically imparts to the wearer the names of all spells currently stored within it.

Faint evocation; CL 5th; \feat{Forge Ring}, \spell{imbue with spell ability}; Price 18,000 cp.


\textbf{Spell Storing:} As the \emph{minor ring of spell storing}, except it holds up to five levels of spells.

Moderate evocation; CL 9th; \feat{Forge Ring}, \spell{imbue with spell ability}; Price 50,000 cp.


\textbf{Spell Storing, Major:} As the \emph{minor ring of spell storing}, except it holds up to ten levels of spells.

Strong evocation; CL 17th; \feat{Forge Ring}, \spell{imbue with spell ability}; Price 200,000 cp.


\textbf{Spell Turning:} Up to three times per day on command, this simple platinum band automatically reflects the next nine levels of spells cast at the wearer, exactly as if spell turning had been cast upon the wearer.

Strong abjuration; CL 13th; \feat{Forge Ring}, \spell{spell turning}; Price 98,280 cp.


\textbf{Sustenance:} This ring provides its wearer with life-sustaining nourishment, so he needs only half of his daily needs of food and water. The ring also refreshes the body and mind, so that its wearer needs only sleep 2 hours per day to gain the benefit of 8 hours of sleep. The ring must be worn for a full week before it begins to work. If it is removed, the owner must wear it for another week to reattune it to himself.

Faint conjuration; CL 5th; \feat{Forge Ring}, \spell{lesser restoration}; Price 2,500 cp.


% \textbf{Sustenance:} This ring continually provides its wearer with life-sustaining nourishment. The ring also refreshes the body and mind, so that its wearer needs only sleep 2 hours per day to gain the benefit of 8 hours of sleep. The ring must be worn for a full week before it begins to work. If it is removed, the owner must wear it for another week to reattune it to himself.

% Faint conjuration; CL 5th; \feat{Forge Ring}, \spell{create food and water}; Price 2,500 cp.


% \textbf{Swimming:} This silver ring has a wave pattern etched into the band. It continually grants the wearer a +5 competence bonus on \skill{Swim} checks.

% Faint transmutation; CL 2nd; \feat{Forge Ring}, creator must have 5 ranks in the \skill{Swim} skill; Price 2,500 cp.


% \textbf{Swimming, Improved:} As swimming, except it grants a +10 competence bonus on its wearer's \skill{Swim} checks.

% Moderate transmutation; CL 7th; \feat{Forge Ring}, creator must have 10 ranks in the \skill{Swim} skill; Price 10,000 cp.


\textbf{Telekinesis:} This ring allows the wearer to use the spell \spell{telekinesis} on command.

Moderate transmutation; CL 9th; \feat{Forge Ring}, \spell{telekinesis}; Price 75,000 cp.


\textbf{Telepathic Attack:} This ring offers continual enhancement bonus of +1 to +5 to power checks made for any telepathic attack.

Faint abjuration; CL 5th; \feat{Forge Ring}, \spell{dweomer of transference}, caster must be a manifester, caster must be of a level at least three times greater than the bonus of the ring; Price 2,000 cp (ring +1); 8,000 cp (ring +2); 18,000 cp (ring +3); 32,000 cp (ring +4); 50,000 cp (ring +5).


\textbf{Telepathic Defense:} This ring offers continual resistance bonus of +1 to +5 to Will saves made against telepathic attacks.

Faint abjuration; CL 5th; \feat{Forge Ring}, \spell{dweomer of transference}, caster must be a manifester, caster must be of a level at least three times greater than the bonus of the ring; Price 1,000 cp (ring +1); 4,000 cp (ring +2); 9,000 cp (ring +3); 16,000 cp (ring +4); 25,000 cp (ring +5).


\textbf{Three Wishes:} This ring is set with three rubies. Each ruby stores a \spell{wish} spell, activated by the ring. When a \spell{wish} is used, that ruby disappears. For a randomly generated ring, roll 1d3 to determine the remaining number of rubies. When all the wishes are used, the ring becomes a nonmagical item.

Strong evocation (if \spell{miracle} is used); CL 20th; \feat{Forge Ring}, \spell{wish} or \spell{miracle}; Price 97,950 cp; Cost 11,475 cp + 15,918 XP.


% \textbf{Water Walking:} This ring, set with an opal, allows the wearer to continually utilize the effects of the spell \spell{water walk}.

% Moderate transmutation; CL 9th; \feat{Forge Ring}, \spell{water walk}; Price 15,000 cp.


\textbf{Wizardry:} This special ring comes in four kinds (\emph{ring of wizardry I}, \emph{ring of wizardry II}, \emph{ring of wizardry III}, and \emph{ring of wizardry IV}), all of them useful only to arcane spellcasters. The wearer's arcane spells per day are doubled for one specific spell level. A \emph{ring of wizardry I} doubles 1st-level spells, a \emph{ring of wizardry II} doubles 2nd-level spells, a \emph{ring of wizardry III} doubles 3rd-level spells, and a \emph{ring of wizardry IV} doubles 4th-level spells. Bonus spells from high ability scores or school specialization are not doubled.

Moderate (wizardry I) or strong (wizardry II--IV) (no school); CL 11th (I), 14th (II), 17th (III), 20th (IV); \feat{Forge Ring}, \spell{limited wish}; Price 20,000 cp (I), 40,000 cp (II), 70,000 cp (III), 100,000 cp (IV).


\textbf{X-Ray Vision:} On command, this ring gives its possessor the ability to see into and through solid matter. Vision range is 6 meters, with the viewer seeing as if he were looking at something in normal light even if there is no illumination. X-ray vision can penetrate 30 centimeters of stone, 2.5 centimeters of common metal, or up to 1 meter of wood or dirt. Thicker substances or a thin sheet of lead blocks the vision.

Using the ring is physically exhausting, causing the wearer 1 point of Constitution damage per minute after the first 10 minutes of use in a single day.

Moderate divination; CL 6th; \feat{Forge Ring}, \spell{true seeing}; Price 25,000 cp.


\subsectionA{Rods}
Rods are scepterlike devices that have unique magical powers and do not usually have charges. Anyone can use a rod.
Physical Description

Rods weigh approximately 2.5 kilograms.

They range from 50 centimeters to 1 meter long and are usually made of iron or some other metal. (Many, as noted in their descriptions, can function as light maces or clubs due to their sturdy construction.)

These sturdy items have AC 9, 10 hit points, hardness 10, and a break DC of 27.

\textbf{Activation:} Details relating to rod use vary from item to item. See the individual descriptions for specifics.

\Table{Rods}{Xr{15mm}}{
\tableheader Rod & \tableheader Market Price \\
Metamagic, Enlarge, lesser   & 3,000 cp \\
Metamagic, Extend, lesser    & 3,000 cp \\
Metamagic, Silent, lesser    & 3,000 cp \\
Water divining               & 3,000 cp \\
Immovable                    & 5,000 cp \\
Metamagic, Empower, lesser   & 9,000 cp \\
Metal and mineral detection  & 10,500 cp \\
Cancellation                 & 11,000 cp \\
Elemental                    & 11,000 cp \\
Metamagic, Enlarge           & 11,000 cp \\
Metamagic, Extend            & 11,000 cp \\
Metamagic, Silent            & 11,000 cp \\
Wonder                       & 12,000 cp \\
Python                       & 13,000 cp \\
Metamagic, Maximize, lesser  & 14,000 cp \\
Flame extinguishing          & 15,000 cp \\
Viper                        & 19,000 cp \\
Enemy detection              & 23,500 cp \\
Metamagic, Enlarge, greater  & 24,500 cp \\
Metamagic, Extend, greater   & 24,500 cp \\
Metamagic, Silent, greater   & 24,500 cp \\
Splendor                     & 25,000 cp \\
Withering                    & 25,000 cp \\
Metamagic, Empower           & 32,500 cp \\
Thunder and lightning        & 33,000 cp \\
Metamagic, Quicken, lesser   & 35,000 cp \\
Negation                     & 37,000 cp \\
% Absorption                   & 50,000 cp \\
Flailing                     & 50,000 cp \\
Metamagic, Maximize          & 54,000 cp \\
Desiccating                  & 60,000 cp \\
Rulership                    & 60,000 cp \\
Security                     & 61,000 cp \\
% Lordly might                 & 70,000 cp \\
Metamagic, Empower, greater  & 73,000 cp \\
Guardianship                 & 75,000 cp \\
Metamagic, Quicken           & 75,500 cp \\
Alertness                    & 85,000 cp \\
Metamagic, Maximize, greater & 121,500 cp \\
Metamagic, Quicken, greater  & 170,000 cp \\
}

\subsubsection{Rod Descriptions}
Although all rods are generally scepterlike, their configurations and abilities run the magical gamut. Standard rods are described below.

\textbf{Absorption:} This rod acts as a magnet, drawing spells or spell-like abilities into itself. The magic absorbed must be a single-target spell or a ray directed at either the character possessing the rod or her gear. The rod then nullifies the spell's effect and stores its potential until the wielder releases this energy in the form of spells of her own. She can instantly detect a spell's level as the rod absorbs that spell's energy. Absorption requires no action on the part of the user if the rod is in hand at the time.

A running total of absorbed (and used) spell levels should be kept. The wielder of the rod can use captured spell energy to cast any spell she has prepared, without expending the preparation itself. The only restrictions are that the levels of spell energy stored in the rod must be equal to or greater than the level of the spell the wielder wants to cast, that any material components required for the spell be present, and that the rod be in hand when casting. For casters such as templars who do not prepare spells, the rod's energy can be used to cast any spell of the appropriate level or levels that they know.

A \emph{rod of absorption} absorbs a maximum of fifty spell levels and can thereafter only discharge any remaining potential it might have. The rod cannot be recharged. The wielder knows the rod's remaining absorbing potential and current amount of stored energy.

To determine the absorption potential remaining in a newly found rod, roll d\% and divide the result by 2. Then roll d\% again: On a result of 71-100, half the levels already absorbed by the rod are still stored within.

Strong abjuration; CL 15th; \feat{Craft Rod}, \spell{spell turning}; Price 50,000 cp.

\textbf{Alertness:} This rod is indistinguishable from a \emph{+1 light mace}. It has eight flanges on its macelike head. The rod bestows a +1 insight bonus on initiative checks. If grasped firmly, the rod enables the holder to use \spell{detect evil}, \spell{detect good}, \spell{detect chaos}, \spell{detect law}, \spell{detect magic}, \spell{discern lies}, \spell{light}, or \spell{see invisibility}. Each different use is a standard action.

If the head of a \emph{rod of alertness} is planted in the ground, and the possessor wills it to alertness (a standard action), the rod senses any creature within 36 meters who intends to harm the possessor. At the same time, the rod creates the effect of a prayer spell upon all creatures friendly to the possessor in a 6-meter radius. Immediately thereafter, the rod sends forth a mental alert to these friendly creatures, warning them of possible danger from the unfriendly creature or creatures within the 36-meter radius. These effects last for 10 minutes, and the rod can perform this function once per day. Last, the rod can be used to simulate the casting of an animate objects spell, utilizing any eleven (or fewer) Small objects located roughly around the perimeter of a 1.5-meter-radius circle centered on the rod when planted in the ground. Objects remain animated for 11 rounds. The rod can perform this function once per day.

Moderate abjuration, divination, enchantment, and evocation; CL 11th; \feat{Craft Rod}, \spell{alarm}, \spell{detect chaos}, \spell{detect evil}, \spell{detect good}, \spell{detect law}, \spell{detect magic}, \spell{discern lies}, \spell{light}, \spell{see invisibility}, \spell{prayer}, \spell{animate objects}; Price 85,000 cp.

\textbf{Cancellation:} This dreaded rod is a bane to magic items, for its touch drains an item of all magical properties. The item touched must make a DC 23 Will save to prevent the rod from draining it. If a creature is holding it at the time, then the item can use the holder's Will save bonus in place of its own if the holder's is better. In such cases, contact is made by making a melee touch attack roll. Upon draining an item, the rod itself becomes brittle and cannot be used again. Drained items are only restorable by \spell{wish} or \spell{miracle}. (If a \emph{sphere of annihilation} and a \emph{rod of cancellation} negate each other, nothing can restore either of them.)

Strong abjuration; CL 17th; \feat{Craft Rod}, \spell{mage's disjunction}; Price 11,000 cp.

\textbf{Desiccating:} This rod is crafted from the thigh-bone of a thrax, with a series of gaunt, dehydrated faces carved into it in a spiral. The \emph{desiccating rod} is often used by silt and sun clerics to drain water from living creatures. This rod is wielded as a \emph{+1 club} that deals 1d6+5 points of temporary Constitution damage to any creature touched (by making a melee touch attack) instead of the usual hit point damage. Oozes, plants and creatures with the aquatic subtype are more susceptible to this attack and instead take 1d8+5 points of Constitution damage. In either case, the defender negates the effect with a DC 22 Fortitude save.

Against creatures immune to ability score damage or who have no Constitution score, this rod causes damage as normal for a \emph{+1 club}.

Strong necromancy; CL 15th; \feat{Craft Rod}, \feat{Craft Magic Arms and Armor}, \spell{horrid wilting}; Price 60,000 cp.

\textbf{Elemental:} These widely varied looking rods are crafted by the clerics of all elemental affiliations to help them in their quest to restore or augment the power of their patron element. Although these rods are made specifically by and for elemental clerics, any abilties confered by a rod and unrelated to spellcasting are accessible to anybody possessing one, unless otherwise noted. One specific rod exists for every domain available to Athasian clerics.

The possessor of the rod gains a +4 bonus to the class skill added by the possession of the corresponding domain.

% An elemental rod allows the casting, once per day, of any one of the domain spells on the corresponding domain list, up to the maximum current level achieved by the wielder, but only if the wielder has access to said domain.

Finally, once per day, the rod can be used as a divine focus for a spell on the domain spell list. While used so, the spell cast by the wearer has its the save DC increased by +1. If no saving throw applies or is allowed, it instead adds +1 to the effective caster level of the effect. This ability can be used by any caster provided he has the domain spell on his spell list.

Strong evocation; CL 12th; \feat{Craft Rod}, \spell{imbue with spell ability}, access to the associated domain; Price 11,000 cp.

\textbf{Enemy Detection:} This device pulses in the wielder's hand and points in the direction of any creature or creatures hostile to the bearer of the device (nearest ones first). These creatures can be invisible, ethereal, hidden, disguised, or in plain sight. Detection range is 18 meters. If the bearer of the rod concentrates for a full round, the rod pinpoints the location of the nearest enemy and indicates how many enemies are within range. The rod can be used three times each day, each use lasting up to 10 minutes. Activating the rod is a standard action.

Moderate divination; CL 10th; \feat{Craft Rod}, \spell{true seeing}; Price 23,500 cp.

\textbf{Flailing:} Upon the command of its possessor, the rod activates, changing from a normal-seeming rod to a \emph{+3 dire flail}. The dire flail is a double weapon, which means that each of the weapon's heads can be used to attack. The wielder can gain an extra attack (with the second head) at the cost of making all attacks at a $-2$ penalty (as if she had the \feat{Two-Weapon Fighting} feat).

Once per day the wielder can use a free action to cause the rod to grant her a +4 deflection bonus to Armor Class and a +4 resistance bonus on saving throws for 10 minutes. The rod need not be in weapon form to grant this benefit.

Transforming it into a weapon or back into a rod is a move action.

Moderate enchantment; CL 9th; \feat{Craft Rod}, \feat{Craft Magic Arms and Armor}, \spell{bless}; Price 50,000 cp.

\textbf{Flame Extinguishing:} This rod can extinguish Medium or smaller nonmagical fires with simply a touch (a standard action). For the rod to be effective against other sorts of fires, the wielder must expend 1 or more of the rod's charges.

Extinguishing a Large or larger nonmagical fire, or a magic fire of Medium or smaller (such as that of a flaming weapon or a \spell{burning hands} spell), expends 1 charge. Continual magic flames, such as those of a weapon or a fire creature, are suppressed for 6 rounds and flare up again after that time. To extinguish an instantaneous fire spell, the rod must be within the area of the effect and the wielder must have used a ready action, effectively countering the entire spell.

When applied to Large or larger magic fires, such as those caused by \spell{fireball}, \spell{flame strike}, or \spell{wall of fire}, extinguishing the flames expends 2 charges from the rod.

If the device is used upon a fire creature (a melee touch attack), it deals 6d6 points of damage to the creature. This use requires 3 charges.

A \emph{rod of flame extinguishing} has 10 charges when found. Spent charges are renewed every day, so that a wielder can expend up to 10 charges in any 24-hour period.

Strong transmutation; CL 12th; \feat{Craft Rod}, \spell{pyrotechnics}; Price 15,000 cp.

\textbf{Guardianship:} Made from the gnarled root of an ancient \emph{tree of life}, this 1-meter long rod appears to pulse with life energy and is used to prevent defiling from further destroying the ecology of Athas. The wood that composes the rod must be taken from a living \emph{tree of life} as its ability to combat defiling is tied to the tree, and only functions while the tree still lives. As such, if the \emph{tree of life} from which this rod is crafted is ever destroyed, the rod loses all its powers.

Simply holding the rod while within the defiling radius of a wizard dampens the gathering of plant life energy, effectively making the terrain type for the wizard one step worse. As such, wizards cannot defile on desolate terrain. Furthermore, the rod prevents affected wizards from using Raze feats or extending the casting time of their spells for the purpose of increasing their effective caster level.

Three times per day, as an immediate action, the rod holder---when within the defiling radius of a casting wizard---can reduce the defiling radius to a distance equal to that between his location and that of the defiler, effectively limiting the defiler from using his highest level spells by reducing the maximum radius from which he can summon plant life energy. For example, a character holding a \emph{rod of guardianship} and standing 9 meters from a wizard could limit him to only casting 6th-level spells, as they defile a circular area 9 meters in radius; a 9th-level spell, with its 13.5 meters radius, would ``pass'' the rod wielder, and thus could be blocked. When this ability is used, a wizard attempting to cast a spell with a defiling radius that extends past the rod wielder simply fails in his action, losing the spell.

Finally, once per day as an immediate action, the rod holder---when within the defiling radius of a casting wizard---can completely negate the energy gathering process, disrupting the casting and causing the wizard to lose the spell.

Strong abjuration; CL 15th; \feat{Craft Rod}, \spell{conversion}; Price 75,000 cp.

\textbf{Immovable Rod:} This rod is a flat iron bar with a small button on one end. When the button is pushed (a move action), the rod does not move from where it is, even if staying in place defies gravity. Thus, the owner can lift or place the rod wherever he wishes, push the button, and let go. Several \emph{immovable rods} can even make a ladder when used together (although only two are needed). An \emph{immovable rod} can support up to 4,000 kilograms before falling to the ground. If a creature pushes against an \emph{immovable rod}, it must make a DC 30 Strength check to move the rod up to 3 meters in a single round.

Moderate transmutation; CL 10th; \feat{Craft Rod}, \spell{levitate}; Price 5,000 cp.

% \textbf{Lordly Might:} This rod has functions that are spell-like, and it can also be used as a magic weapon of various sorts. It also has several more mundane uses. The \emph{rod of lordly might} is metal, thicker than other rods, with a flanged ball at one end and six studlike buttons along its length. (Pushing any of the rod's buttons is equivalent to drawing a weapon.) It weighs 5 kilograms.

% The following spell-like functions of the rod can each be used once per day.

% \begin{itemize*}
% \item \spell{Hold Person} upon touch, if the wielder so commands (Will DC 14 negates). The wielder must choose to use this power and then succeed on a melee touch attack to activate the power. If the attack fails, the effect is lost.
% \item \spell{Fear} upon all enemies viewing it, if the wielder so desires (3-meter maximum range, Will DC 16 partial). Invoking this power is a standard action.
% \item Deal 2d4 hit points of damage to an opponent on a successful touch attack (Will DC 17 half) and cure the wielder of a like amount of damage. The wielder must choose to use this power before attacking, as with \spell{hold person}.
% \end{itemize*}

% The following weapon functions of the rod have no limit on the number of times they can be employed.

% \begin{itemize*}
% \item In its normal form, the rod can be used as a \emph{+2 light mace}.
% \item When button 1 is pushed, the rod becomes a \emph{+1 flaming longsword}. A blade springs from the ball, with the ball itself becoming the sword's hilt. The weapon lengthens to an overall length of 1.2 meter.
% \item When button 2 is pushed, the rod becomes a \emph{+4 battleaxe}. A wide blade springs forth at the ball, and the whole lengthens to 1.2 meter.
% \item When button 3 is pushed, the rod becomes a \emph{+3 shortspear} or \emph{+3 longspear}. The spear blade springs forth, and the handle can be lengthened up to 3.6 meters (wielder's choice), for an overall length of from 1.8 meter to 4.5 meters. At its 4.5-meter length, the rod is suitable for use as a lance.
% \end{itemize*}

% The following other functions of the rod also have no limit on the number of times they can be employed.

% \begin{itemize*}
% \item Climbing pole/ladder. When button 4 is pushed, a spike that can anchor in granite is extruded from the ball, while the other end sprouts three sharp hooks. The rod lengthens to anywhere between 1.5 and 15 meters in a single round, stopping when button 4 is pushed again. Horizontal bars 8 centimeters long fold out from the sides, 30 centimeters apart, in staggered progression. The rod is firmly held by the spike and hooks and can bear up to 2,000 kilograms. The wielder can retract the pole by pushing button 5.
% \item The ladder function can be used to force open doors. The wielder plants the rod's base 9 meters or less from the portal to be forced and in line with it, then pushes button 4. The force exerted has a Strength modifier of +12.
% \item When button 6 is pushed, the rod indicates magnetic north and gives the wielder a knowledge of his approximate depth beneath the surface or height above it.
% \end{itemize*}

% Strong enchantment, evocation, necromancy, and transmutation; CL 19th; \feat{Craft Rod}, \feat{Craft Magic Arms and Armor}, \spell{inflict light wounds}, \spell{bull's strength}, \spell{flame blade}, \spell{hold person}, \spell{fear}; Price 70,000 cp.

\textbf{Metal and Mineral Detection:} This rod pulses in the wielder's hand and points to the largest mass of metal within 9 meters. However, the wielder can concentrate on a specific metal or mineral. If the specific mineral is within 9 meters, the rod points to any places it is located, and the rod wielder knows the approximate quantity as well. If more than one deposit of the specified metal or mineral is within range, the rod points to the largest cache first. Each operation requires a full-round action.

Moderate divination; CL 9th; \feat{Craft Rod}, \spell{locate object}; Price 10,500 cp.

\textbf{Metamagic Rods:} \emph{Metamagic rods} hold the essence of a metamagic feat but do not change the spell slot of the altered spell. All the rods described here are use-activated (but casting spells in a threatened area still draws an attack of opportunity). A caster may only use one \emph{metamagic rod} on any given spell, but it is permissible to combine a rod with metamagic feats possessed by the rod's wielder. In this case, only the feats possessed by the wielder adjust the spell slot of the spell being cast.

Possession of a \emph{metamagic rod} does not confer the associated feat on the owner, only the ability to use the given feat a specified number of times per day. A templar still must take a full-round action when using a metamagic rod, just as if using a metamagic feat he possesses.

\textit{Lesser and Greater Metamagic Rods:} Normal metamagic rods can be used with spells of 6th level or lower. Lesser rods can be used with spells of 3rd level or lower, while greater rods can be used with spells of 9th level or lower.

% \begin{itemize*}
\textbf{Metamagic, Empower:} The wielder can cast up to three spells per day that are empowered as though using the \feat{Empower Spell} feat.

Strong (no school); CL 17th; \feat{Craft Rod}, \feat{Empower Spell}; Price 9,000 cp (lesser), 32,500 cp (normal), 73,000 cp (greater).

\textbf{Metamagic, Enlarge:} The wielder can cast up to three spells per day that are enlarged as though using the \feat{Enlarge Spell} feat.

Strong (no school); CL 17th; \feat{Craft Rod}, \feat{Enlarge Spell}; Price 3,000 cp (lesser), 11,000 cp (normal), 24,500 cp (greater).

\textbf{Metamagic, Extend:} The wielder can cast up to three spells per day that are extended as though using the \feat{Extend Spell} feat.

Strong (no school); CL 17th; \feat{Craft Rod}, \feat{Extend Spell}; Price 3,000 cp (lesser), 11,000 cp (normal), 24,500 cp (greater).

\textbf{Metamagic, Maximize:} The wielder can cast up to three spells per day that are maximized as though using the \feat{Maximize Spell} feat.

Strong (no school); CL 17th; \feat{Craft Rod}, \feat{Maximize Spell} feat; Price 14,000 cp (lesser), 54,000 cp (normal), 121,500 cp (greater).

\textbf{Metamagic, Quicken:} The wielder can cast up to three spells per day that are quickened as though using the \feat{Quicken Spell} feat.

Strong (no school); CL 17th; \feat{Craft Rod}, \feat{Quicken Spell}; Price 35,000 cp (lesser), 75,500 cp (normal), 170,000 cp (greater).

\textbf{Metamagic, Silent:} The wielder can cast up to three spells per day without verbal components as though using the \feat{Silent Spell} feat.

Strong (no school); CL 17th; \feat{Craft Rod}, \feat{Silent Spell}; Price 3,000 cp (lesser), 11,000 cp (normal), 24,500 cp (greater).
% \end{itemize*}

\textbf{Negation:} This device negates the spell or spell-like function or functions of magic items. The wielder points the rod at the magic item, and a pale gray beam shoots forth to touch the target device, attacking as a ray (a ranged touch attack). The ray functions as a \spell{greater dispel magic} spell, except it only affects magic items. To negate instantaneous effects from an item, the rod wielder needs to have used a ready action. The dispel check uses the rod's caster level (15th). The target item gets no saving throw, although the rod can't negate artifacts (even minor artifacts). The rod can function three times per day.

Strong varied; CL 15th; \feat{Craft Rod}, \spell{dispel magic}, and \spell{limited wish} or \spell{miracle}; Price 37,000 cp.

\textbf{Python:} This rod is longer than normal rods. It is about 1.2 meter long and weighs 5 kilograms. It strikes as a \emph{+1/+1 quarterstaff}. If the user throws the rod to the ground (a standard action), it grows to become a giant constrictor snake by the end of the round. The python obeys all commands of the owner. (In animal form, it retains the +1 enhancement bonus on attacks and damage possessed by the rod form.) The serpent returns to rod form (a full-round action) whenever the wielder desires, or whenever it moves farther than 30 meters from the owner. If the snake form is slain, it returns to rod form and cannot be activated again for three days. A \emph{python rod} only functions if the possessor is good.

Moderate transmutation; CL 10th; \feat{Craft Rod}, \feat{Craft Magic Arms and Armor}, \spell{baleful polymorph}, creator must be good; Price 13,000 cp.

\textbf{Security:} This item creates a nondimensional space, a pocket paradise. There the rod's possessor and as many as 199 other creatures can stay in complete safety for a period of time, up to 200 days divided by the number of creatures affected. All fractions are rounded down.

In this pocket paradise, creatures don't age, and natural healing take place at twice the normal rate. Fresh water and food (fruits and vegetables only) are in abundance. The climate is comfortable for all creatures involved.

Activating the rod (a standard action) causes the wielder and all creatures touching the rod to be transported instantaneously to the paradise. Members of large groups can hold hands or otherwise maintain physical contact, allowing all connected creatures in a circle or a chain to be affected by the rod. Unwilling creatures get a DC 17 Will save to negate the effect. If such a creature succeeds on its save, other creatures beyond that point in a chain can still be affected by the rod.

When the rod's effect expires or is dispelled, all the affected creatures instantly reappear in the location they occupied when the rod was activated. If something else occupies the space that a traveler would be returning to, then his body is displaced a sufficient distance to provide the space required for reentry. The rod's possessor can dismiss the effect whenever he wishes before the maximum time period expires, but the rod can only be activated once per week.

Strong conjuuration; CL 20th; \feat{Craft Rod}, \spell{gate}; Price 61,000 cp.

\textbf{Splendor:} The possessor of this rod has her Charisma score become 19 for as long as she holds or carries the item. Once per day, the rod creates and garbs her in clothing of the finest fabrics, plus adornments of furs and jewels.

Apparel created by the magic of the rod remains in existence for 12 hours. However, if the possessor attempts to sell or give away any part of it, to use it for a spell component, or the like, all the apparel immediately disappears. The same applies if any of it is forcibly taken from her.

The value of noble garb created by the rod ranges from 7,000 to 10,000 cp (1d4+6 $\times$ 1,000 cp)---1,000 cp for the fabric alone, 5,000 cp for the furs, and the rest for the jewel trim (maximum of twenty gems, maximum value 200 cp each).

In addition, the rod has a second special power, usable once per week. Upon command, it creates a palatial tent---a huge pavilion of silk 18 meters across. Inside the tent are temporary furnishings and food suitable to the splendor of the pavilion and sufficient to entertain as many as one hundred persons. The tent and its trappings last for one day. At the end of that time, the tent and all objects associated with it (including any items that were taken out of the tent) disappear.

Strong conjuration and transmutation; CL 12th; \feat{Craft Rod}, \spell{eagle's splendor}, \spell{fabricate}, \spell{major creation}; Price 25,000 cp.

\textbf{Thunder and Lightning:} Constructed of iron set with silver rivets, this rod has the properties of a \emph{+2 light mace}. Its other magical powers are as follows.
\begin{itemize*}
\item \textit{Thunder:} Once per day, the rod can strike as a \emph{+3 light mace}, and the opponent struck is stunned from the noise of the rod's impact (Fortitude DC 16 negates). Activating this power counts as a free action, and it works if the wielder strikes an opponent within 1 round.
\item \textit{Lightning:} Once per day, when the wielder desires, a short spark of electricity can leap forth when the rod strikes an opponent to deal the normal damage for a \emph{+2 light mace} (1d6+2) and an extra 2d6 points of electricity damage. Even when the rod might not score a normal hit in combat, if the roll was good enough to count as a successful melee touch attack hit, then the 2d6 points of electricity damage still applies. The wielder activates this power as a free action, and it works if he strikes an opponent within 1 round.
\item \textit{Thunderclap:} Once per day as a standard action, the wielder can cause the rod to give out a deafening noise, just as a \spell{shout} spell (Fortitude DC 16 partial, 2d6 points of sonic damage, target deafened for 2d6 rounds).
\item \textit{Lightning Stroke:} Once per day as a standard action, the wielder can cause the rod to shoot out a 1.5-meter-wide \spell{lightning bolt} (9d6 points of electricity damage, Reflex DC 16 half) to a range of 60 meters.
\item \textit{Thunder and Lightning:} Once per week as a standard action, the wielder of the rod can combine the thunderclap described above with a \spell{lightning bolt}, as in the lightning stroke. The thunderclap affects all within 3 meters of the bolt. The lightning stroke deals 9d6 points of electricity damage (count rolls of 1 or 2 as rolls of 3, for a range of 27 to 54 points), and the thunderclap deals 2d6 points of sonic damage. A single DC 16 Reflex save applies for both effects.
\end{itemize*}

Moderate evocation; CL 9th; \feat{Craft Rod}, \feat{Craft Magic Arms and Armor}, \spell{lightning bolt}, \spell{shout}; Price 33,000 cp.

\textbf{Viper:} This rod strikes as a \emph{+2 heavy mace}. Once per day, upon command, the head of the rod becomes that of an actual serpent for 10 minutes. During this period, any successful strike with the rod deals its usual damage and also poisons the creature hit. The poison deals 1d10 points of Constitution damage immediately (Fortitude DC 14 negates) and another 1d10 points of Constitution damage 1 minute later (Fortitude DC 14 negates). The rod only functions if its possessor is evil.

Moderate necromancy; CL 10th; \feat{Craft Rod}, \feat{Craft Magic Arms and Armor}, \spell{poison}, creator must be evil; Price 19,000 cp.

\textbf{Water Divining:} This rod is a small Y-shaped stick that must be held in both hands to use. Three times per day as a standard action, the rod locates and pull its holder toward the largest accumulation of water of at least four liters within 900 meters. The end of the rod will point toward the water and gently pull the character that way. The quality of the water need not be such that the character can easily obtain it. For instance, the rod might point down to an underground water source up to 900 meters beneath the wielder.

Moderate divination; CL 9th; \feat{Craft Rod}; Price 3,000 cp.

\textbf{Withering:} A \emph{rod of withering} acts as a \emph{+1 light mace} that deals no hit point damage. Instead, the wielder deals 1d4 points of Strength damage and 1d4 points of Constitution damage to any creature she touches with the rod (by making a melee touch attack). If she scores a critical hit, the damage from that hit is permanent ability drain. In either case, the defender negates the effect with a DC 17 Fortitude save.

Strong necromancy; CL 13th; \feat{Craft Rod}, \feat{Craft Magic Arms and Armor}, \spell{contagion}; Price 25,000 cp.

\textbf{Wonder:} A \emph{rod of wonder} is a strange and unpredictable device that randomly generates any number of weird effects each time it is used. (Activating the rod is a standard action.) Typical powers of the rod include the following.

\Table{}{cX}{
\tableheader d\% & \tableheader Wondrous Effect\\
01--05  & Slow creature pointed at for 10 rounds (Will DC 15 negates). \\
06--10  & \spell{Faerie Fire} surrounds the target. \\
11--15  & Deludes wielder for 1 round into believing the rod functions as indicated by a second die roll (no save). \\
16--20  & \spell{Gust of Wind}, but at windstorm force (Fortitude DC 14 negates). \\
21--25  & Wielder learns target's surface thoughts (as with \spell{detect thoughts}) for 1d4 rounds (no save). \\
26--30  & \spell{Stinking Cloud} at 9-m range (Fortitude DC 15 negates). \\
31--33  & Heavy rain falls for 1 round in 18-m radius centered on rod wielder. \\
34--36  & Summon an animal---a rhino (01--25 on d\%), elephant (26--50), or mouse (51--100). \\
37--46  & \spell{Lightning Bolt} (21 m long, 1.5 m wide), 6d6 damage (Reflex DC 15 half). \\
47--49  & Stream of 600 large butterflies pours forth and flutters around for 2 rounds, blinding everyone (including wielder) within 7.5 m (Reflex DC 14 negates). \\
50--53  & \spell{Enlarge Person} if within 18 m of rod (Fortitude DC 13 negates). \\
54--58  & \spell{Darkness}, 9-m-diameter hemisphere, centered 30 ft. away from rod. \\
59--62  & Grass grows in 160-sq.-ft. area before the rod, or grass existing there grows to ten times normal size. \\
63--65  & Turn ethereal any nonliving object of up to 500 kg mass and up to 0.85 m$^3$ in size. \\
66--69  & Reduce wielder to 1/12 height (no save). \\
70--79  & \spell{Fireball} at target or 30 m straight ahead, 6d6 damage (Reflex DC 15 half). \\
80--84  & \spell{Invisibility} covers rod wielder. \\
85--87  & Leaves grow from target if within 18 m of rod. These last 24 hours. \\
88--90  & 10-40 gems, value 1 cp each, shoot forth in a 9-m-long stream. Each gem deals 1 point of damage to any creature in its path: Roll 5d4 for the number of hits and divide them among the available targets. \\
91--95  & Shimmering colors dance and play over a 40-ft.-by-9-m area in front of rod. Creatures therein are blinded for 1d6 rounds (Fortitude DC 15 negates). \\
96--97  & Wielder (50\% chance) or target (50\% chance) turns permanently blue, green, or purple (no save). \\
98--100 & \spell{Flesh to Stone} (or \spell{stone to flesh} if target is stone already) if target is within 18 m (Fortitude DC 18 negates). \\
}

Moderate enchantment; CL 10th; \feat{Craft Rod}, \spell{confusion}, creator must be chaotic; Price 12,000 cp.

\subsectionA{Scrolls}
A scroll is a spell (or collection of spells) that has been stored in written form. A spell on a scroll can be used only once. The writing vanishes from the scroll when the spell is activated. Using a scroll is basically like casting a spell.

\textbf{Physical Description:} A scroll is a heavy sheet of fine vellum or high-quality paper. An area about 20 centimeters wide and 30 centimeters long is sufficient to hold one spell. The sheet is reinforced at the top and bottom with strips of leather slightly longer than the sheet is wide. A scroll holding more than one spell has the same width (about 20 centimeters) but is an extra 30 centimeters long for each extra spell. Scrolls that hold three or more spells are usually fitted with reinforcing rods at each end rather than simple strips of leather. A scroll has AC 9, 1 hit point, hardness 0, and a break DC of 8.

To protect it from wrinkling or tearing, a scroll is rolled up from both ends to form a double cylinder. (This also helps the user unroll the scroll quickly.) The scroll is placed in a tube of ivory, jade, leather, metal, or wood. Most scroll cases are inscribed with magic symbols which often identify the owner or the spells stored on the scrolls inside. The symbols often hide magic traps.

\textbf{Activation:} To activate a scroll, a spellcaster must read the spell written on it. Doing so involves several steps and conditions.

\textit{Decipher the Writing:} The writing on a scroll must be deciphered before a character can use it or know exactly what spell it contains. This requires a \spell{read magic} spell or a successful \skill{Spellcraft} check (DC 20 + spell level).

Deciphering a scroll to determine its contents does not activate its magic unless it is a specially prepared cursed scroll. A character can decipher the writing on a scroll in advance so that he or she can proceed directly to the next step when the time comes to use the scroll.

\textit{Activate the Spell:} Activating a scroll requires reading the spell from the scroll. The character must be able to see and read the writing on the scroll. Activating a scroll spell requires no material components or focus. (The creator of the scroll provided these when scribing the scroll.) Note that some spells are effective only when cast on an item or items. In such a case, the scroll user must provide the item when activating the spell. Activating a scroll spell is subject to disruption just as casting a normally prepared spell would be. Using a scroll is like casting a spell for purposes of arcane spell failure chance.

To have any chance of activating a scroll spell, the scroll user must meet the following requirements.

\begin{itemize*}
\item The spell must be of the correct type (arcane or divine). Arcane spellcasters (assassins and wizards) can only use scrolls containing arcane spells, and divine spellcasters (clerics, druids, and rangers) can only use scrolls containing divine spells. (The type of scroll a character creates is also determined by his or her class.)
\item The user must have the spell on his or her class list.
\item The user must have the requisite ability score.
\end{itemize*}

If the user meets all the requirements noted above, and her caster level is at least equal to the spell's caster level, she can automatically activate the spell without a check. If she meets all three requirements but her own caster level is lower than the scroll spell's caster level, then she has to make a caster level check (DC = scroll's caster level + 1) to cast the spell successfully. If she fails, she must make a DC 5 Wisdom check to avoid a mishap (see Scroll Mishaps, below). A natural roll of 1 always fails, whatever the modifiers.

\textit{Determine Effect:} A spell successfully activated from a scroll works exactly like a spell prepared and cast the normal way. Assume the scroll spell's caster level is always the minimum level required to cast the spell for the character who scribed the scroll (usually twice the spell's level, minus 1), unless the caster specifically desires otherwise.

The writing for an activated spell disappears from the scroll.

\textit{Scroll Mishaps:} When a mishap occurs, the spell on the scroll has a reversed or harmful effect. Possible mishaps are given below.

\begin{itemize*}
\item A surge of uncontrolled magical energy deals 1d6 points of damage per spell level to the scroll user.
\item Spell strikes the scroll user or an ally instead of the intended target, or a random target nearby if the scroll user was the intended recipient.
\item Spell takes effect at some random location within spell range.
\item Spell's effect on the target is contrary to the spell's normal effect.
\item The scroll user suffers some minor but bizarre effect related to the spell in some way. Most such effects should last only as long as the original spell's duration, or 2d10 minutes for instantaneous spells.
\item Some innocuous item or items appear in the spell's area.
\item Spell has delayed effect. Sometime within the next 1d12 hours, the spell activates. If the scroll user was the intended recipient, the spell takes effect normally. If the user was not the intended recipient, the spell goes off in the general direction of the original recipient or target, up to the spell's maximum range, if the target has moved away.
\end{itemize*}

Several arcane spells are different in level for wizards than they are for assassins. Such spells appear on the table at the level appropriate to a wizard (considered the default because assassins typically don't involve themselves in scribing scrolls).

Likewise, some divine spells are different in level for clerics and druids than they are for rangers. Such spells appear at the level appropriate to a cleric or druid (considered the default because rangers typically don't involve themselves in scribing scrolls).
% Several divine spells are different in level for clerics and druids than they are for rangers. Such spells appear at the level appropriate to a cleric or druid (considered the default because rangers typically don't involve themselves in scribing scrolls).

If a divine spell is cast at different levels by clerics and druids, it appears at the level appropriate to a cleric (considered the default choice between clerics and druids).

Many spells are either arcane or divine, depending on the class of the caster. Such spells appear on both lists at the level appropriate to the class of the arcane or divine caster. 
\subsectionA{Staffs}
A staff is a long shaft of wood that stores several spells. Unlike wands, which can contain a wide variety of spells, each staff is of a certain kind and holds specific spells. A staff has 50 charges when created.

\textbf{Physical Description:} A typical staff is 1.2 meter to 2.1 meters long and 5 centimeters to 7 centimeters thick, weighing about 2.5 kilograms. Most staffs are wood, but a rare few are bone, metal, or even glass. (These are extremely exotic.) Staffs often have a gem or some device at their tip or are shod in metal at one or both ends. Staffs are often decorated with carvings or runes. A typical staff is like a walking stick, quarterstaff, or cudgel. It has AC 7, 10 hit points, hardness 5, and a break DC of 24.

\textbf{Activation:} Staffs use the spell trigger activation method, so casting a spell from a staff is usually a standard action that doesn't provoke attacks of opportunity. (If the spell being cast, however, has a longer casting time than 1 standard action, it takes that long to cast the spell from a staff.) To activate a staff, a character must hold it forth in at least one hand (or whatever passes for a hand, for nonhumanoid creatures).

\Table{Staffs}{Xr{15mm}}{
\tableheader Staff & \tableheader Market Price\\
Charming         & 16,500 cp \\
Fire             & 28,500 cp \\
Swarming insects & 24,750 cp \\
Healing          & 27,750 cp \\
Size alteration  & 29,000 cp \\
Trickster        & 33,200 cp \\
Concurrence      & 39,000 cp \\
Illumination     & 48,250 cp \\
Desert Travel    & 52,000 cp \\
Pain             & 54,750 cp \\
Frost            & 56,250 cp \\
Defense          & 58,250 cp \\
Abjuration       & 65,000 cp \\
Conjuration      & 65,000 cp \\
Enchantment      & 65,000 cp \\
Evocation        & 65,000 cp \\
Illusion         & 65,000 cp \\
Necromancy       & 65,000 cp \\
Transmutation    & 65,000 cp \\
Divination       & 73,500 cp \\
Earth and stone  & 80,500 cp \\
Woodlands        & 101,250 cp \\
Life             & 155,750 cp \\
Passage          & 170,500 cp \\
Power            & 211,000 cp \\
}

\subsubsection{Staff Descriptions}
Staffs use the wielder's ability score and relevant feats to set the DC for saves against their spells. Unlike with other sorts of magic items, the wielder can use his caster level when activating the power of a staff if it's higher than the caster level of the staff.

This means that staffs are far more potent in the hands of a powerful spellcaster. Because they use the wielder's ability score to set the save DC for the spell, spells from a staff are often harder to resist than ones from other magic items, which use the minimum ability score required to cast the spell. Not only are aspects of the spell dependant on caster level (range, duration, and so on) potentially higher, but spells from a staff are harder to dispel and have a better chance of overcoming a target's spell resistance.

Furthermore, a staff can hold a spell of any level, unlike a wand, which is limited to spells of 4th level or lower. The minimum caster level of a staff is 8th. Standard staffs are described below.

\textbf{Abjuration:} Usually carved from the heartwood of an ancient oak or other large tree, this staff allows use of the following spells:
\begin{itemize*}
\item \spell{shield} (1 charge)
\item \spell{resist energy} (1 charge)
\item \spell{dispel magic} (1 charge)
\item \spell{lesser globe of invulnerability} (2 charges)
\item \spell{dismissal} (2 charges)
\item \spell{repulsion} (3 charges)
\end{itemize*}

Strong abjuration; CL 13th; \feat{Craft Staff}, \spell{dismissal}, \spell{dispel magic}, \spell{lesser globe of invulnerability}, \spell{resist energy}, \spell{repulsion}, \spell{shield}; Price 65,000 cp.

\textbf{Charming:} Made of twisting wood ornately shaped and carved, this staff allows use of the following spells:
\begin{itemize*}
\item \spell{charm person} (1 charge)
\item \spell{charm monster} (2 charges)
\end{itemize*}

Moderate enchantment; CL 8th; \feat{Craft Staff}, \spell{charm person}, \spell{charm monster}; Price 16,500 cp.

\textbf{Concurrence:} Made from the fine grained, straight wood of a kaor tree that has been blessed by a druid, this staff allows the use of the following spells:
\begin{itemize*}
\item \spell{sleep} (1 charge)
\item \spell{hold person} (1 charge)
\item \spell{tiny hut} (1 charge)
\item \spell{resilient sphere} (2 charges)
\item \spell{polymorph} (2 charges)
\end{itemize*}

Moderate varied; CL 8th; \feat{Craft Staff}, \spell{hold person}, \spell{polymorph}, \spell{resilient sphere}, \spell{sleep}, \spell{tiny hut}; Price 39,000 cp.

\textbf{Conjuration:} This staff is usually made of ash or walnut and bears ornate carvings of many different kinds of creatures. It allows use of the following spells:
\begin{itemize*}
\item \spell{unseen servant} (1 charge)
\item \spell{summon swarm} (1 charge)
\item \spell{stinking cloud} (1 charge)
\item \spell{minor creation} (2 charges)
\item \spell{cloudkill} (2 charges)
\item \spell{summon monster VI} (3 charges)
\end{itemize*}

Strong conjuration; CL 13th; \feat{Craft Staff}, \spell{cloudkill}, \spell{stinking cloud}, \spell{summon monster VI}, \spell{summon swarm}, \spell{unseen servant}; Price 65,000 cp.

\textbf{Defense:} The staff of defense is a simple-looking staff that throbs with power when held defensively. It allows use of the following spells:
\begin{itemize*}
\item \spell{shield} (1 charge)
\item \spell{shield of faith} (1 charge)
\item \spell{shield other} (1 charge)
\item \spell{shield of law} (3 charges)
\end{itemize*}

Strong abjuration; CL 15th; \feat{Craft Staff}, \spell{shield}, \spell{shield of faith}, \spell{shield of law}, \spell{shield other}, creator must be lawful; Price 58,250 cp.

\textbf{Desert Travel:} Crafted from the wood of a wanderer's staff, a straight and tall species of trees that grows in the Ringing Mountains, this staff allows use of the following spells:
\begin{itemize*}
\item \spell{cooling canopy} (1 charge)
\item \spell{create element} (water) (1 charge)
\item \spell{purify food and drink} (1 charge)
\item \spell{summon nature's ally VI} (water elemental or rain paraelemental beast only) (2 charges)
\end{itemize*}

The staff may be used as a weapon, functioning as a \emph{+2 quarterstaff}. The \emph{staff of desert travel} also allows its wielder to \spell{detect animals or plants} and \spell{know direction} at will, with no charge cost. These two attributes continue to function even after all the staff's charges are expended.

Moderate varied; CL 13th; \feat{Craft Staff}, \feat{Craft Magic Arms and Armor}, \spell{create element}, \spell{cooling canopy}, \spell{detect animals or plants}, \spell{know direction}, \spell{purify food and drink}, \spell{summon nature's ally VI}; Price 52,000 cp.

\textbf{Divination:} Made from a supple length of willow, often with a forked tip, this staff allows use of the following spells:
\begin{itemize*}
\item \spell{detect secret doors} (1 charge)
\item \spell{locate object} (1 charge)
\item \spell{tongues} (1 charge)
\item \spell{locate creature} (2 charges)
\item \spell{prying eyes} (2 charges)
\item \spell{true seeing} (3 charges)
\end{itemize*}

Strong divination; CL 13th; \feat{Craft Staff}, \spell{detect secret doors}, \spell{locate creature}, \spell{locate object}, \spell{prying eyes}, \spell{tongues}, \spell{true seeing}; Price 73,500 cp.

\textbf{Earth and Stone:} This staff is topped with a fist-sized emerald that gleams with smoldering power. It allows the use of the following spells:
\begin{itemize*}
\item \spell{passwall} (1 charge)
\item \spell{move earth} (1 charge)
\end{itemize*}

Moderate transmutation; CL 11th; \feat{Craft Staff}, \spell{move earth}, \spell{passwall}; Price 80,500 cp.

\textbf{Enchantment:} Often made from applewood and topped with a clear crystal, this staff allows use of the following spells:
\begin{itemize*}
\item \spell{sleep} (1 charge)
\item \spell{hideous laughter} (1 charge)
\item \spell{suggestion} (1 charge)
\item \spell{crushing despair} (2 charges)
\item \spell{mind fog} (2 charges)
\item \spell{mass suggestion} (3 charges)
\end{itemize*}

Strong enchantment; CL 13th; \feat{Craft Staff}, \spell{crushing despair}, \spell{mass suggestion}, \spell{mind fog}, \spell{sleep}, \spell{suggestion}, \spell{hideous laughter}; Price 65,000 cp.

\textbf{Evocation:} Usually very smooth and carved from hickory, willow, or yew, this staff allows use of the following spells:
\begin{itemize*}
\item \spell{magic missile} (1 charge)
\item \spell{shatter} (1 charge)
\item \spell{fireball} (1 charge)
\item \spell{ice storm} (2 charges)
\item \spell{wall of force} (2 charges)
\item \spell{chain lightning} (3 charges)
\end{itemize*}

Strong evocation; CL 13th; \feat{Craft Staff}, \spell{chain lightning}, \spell{fireball}, \spell{ice storm}, \spell{magic missile}, \spell{shatter}, \spell{wall of force}; Price 65,000 cp.

\textbf{Fire:} Crafted from bronzewood with brass bindings, this staff allows use of the following spells:
\begin{itemize*}
\item \spell{burning hands} (1 charge)
\item \spell{fireball} (1 charge)
\item \spell{wall of fire} (2 charges)
\end{itemize*}

Moderate evocation; CL 8th; \feat{Craft Staff}, \spell{burning hands}, \spell{fireball}, \spell{wall of fire}; Price 28,500 cp.

\textbf{Frost:} Tipped on either end with a glistening diamond, this rune-covered staff allows use of the following spells:
\begin{itemize*}
\item \spell{ice storm} (1 charge)
\item \spell{wall of ice} (1 charge)
\item \spell{cone of cold} (2 charge)
\end{itemize*}

Moderate evocation; CL 10th; \feat{Craft Staff}, \spell{cone of cold}, \spell{ice storm}, \spell{wall of ice}; Price 56,250 cp.

\textbf{Healing:} This white ash staff, with inlaid silver runes, allows use of the following spells:
\begin{itemize*}
\item \spell{lesser restoration} (1 charge)
\item \spell{cure serious wounds} (1 charge)
\item \spell{remove blindness/deafness} (2 charges)
\item \spell{remove disease} (3 charges)
\end{itemize*}

Moderate conjuration; CL 8th; \feat{Craft Staff}, \spell{cure serious wounds}, \spell{lesser restoration}, \spell{remove blindness/deafness}, \spell{remove disease}; Price 27,750 cp.

\textbf{Illusion:} This staff is made from ebony or other dark wood and carved into an intricately twisted, fluted, or spiral shape. It allows use of the following spells:
\begin{itemize*}
\item \spell{disguise self} (1 charge)
\item \spell{mirror image} (1 charge)
\item \spell{major image} (1 charge)
\item \spell{rainbow pattern} (2 charges)
\item \spell{persistent image} (2 charges)
\item \spell{mislead} (3 charges)
\end{itemize*}

Strong illusion; CL 13th; \feat{Craft Staff}, \spell{disguise self}, \spell{major image}, \spell{mirror image}, \spell{persistent image}, \spell{project image}, \spell{rainbow pattern}; Price 65,000 cp.

\textbf{Illumination:} This staff is usually sheathed in silver and decorated with sunbursts. It allows use of the following spells:
\begin{itemize*}
\item \spell{dancing lights} (1 charge)
\item \spell{flare} (1 charge)
\item \spell{daylight} (2 charges)
\item \spell{sunburst} (3 charges)
\end{itemize*}

Strong evocation; CL 15th; \feat{Craft Staff}, \spell{dancing lights}, \spell{daylight}, \spell{flare}, \spell{sunburst}; Price 48,250 cp.

\textbf{Life:} Made of thick oak shod in gold, this staff allows use of the following spells:
\begin{itemize*}
\item \spell{heal} (1 charge)
\item \spell{resurrection} (5 charges)
\end{itemize*}

Moderate conjuration; CL 13th; \feat{Craft Staff}, \spell{heal}, \spell{resurrection}; Price 155,750 cp.

\textbf{Necromancy:} This staff is made from ebony or other dark wood and carved with the images of bones and skulls. It allows use of the following spells:
\begin{itemize*}
\item \spell{cause fear} (1 charge)
\item \spell{ghoul touch} (1 charge)
\item \spell{halt undead} (1 charge)
\item \spell{enervation} (2 charges)
\item \spell{waves of fatigue} (2 charges)
\item \spell{circle of death} (3 charges)
\end{itemize*}

Strong necromancy; CL 13th; \feat{Craft Staff}, \spell{cause fear}, \spell{circle of death}, \spell{enervation}, \spell{ghoul touch}, \spell{halt undead}, \spell{waves of fatigue}; Price 65,000 cp.

\textbf{Pain:} This weapon is typically used by the templars of Athas to motivate slaves under their control by inflict pain upon them. It allows use of the following spells:
\begin{itemize*}
\item \spell{hold person} (1 charge)
\item \spell{image of the sorcerer-king} (1 charge)
\item \spell{psychic turmoil} (1 charge)
\item \spell{greater psychic turmoil} (2 charges)
\end{itemize*}

The wielder of a \emph{staff of pain} can channel power points to it to inflict pain to any living creature he touches with the staff (by making a melee touch attack). This attack deals 2d6 points of nonlethal damage for each 2 power points spent, up to a maximum of 14d6 points of nonlethal damage by spending 14 power points. The wielder cannot spend more power points than his manifester level. Each use of this touch consumes 3 charges of the staff.

Strong varied; CL 14th; \feat{Craft Staff}, \spell{dweomer of transference}, \spell{greater psychic turmoil}, \spell{hold person}, \spell{image of the sorcerer-king}, \spell{inflict critical wounds}, \spell{psychic turmoil}; Price 54,750 cp.

\textbf{Passage:} This potent item allows use of the following spells:
\begin{itemize*}
\item \spell{dimension door} (1 charge)
\item \spell{passwall} (1 charge)
\item \spell{phase door} (2 charges)
\item \spell{greater teleport} (2 charges)
\item \spell{astral projection} (2 charges)
\end{itemize*}

Strong varied; CL 17th; \feat{Craft Staff}, \spell{astral projection}, \spell{dimension door}, \spell{greater teleport}, \spell{passwall}, \spell{phase door}; Price 170,500 cp.

\textbf{Power:} The \emph{staff of power} is a very potent magic item, with offensive and defensive abilities. It is usually topped with a glistening gem, its shaft straight and smooth. It has the following powers:
\begin{itemize*}
\item \spell{magic missile} (1 charge)
\item \spell{ray of enfeeblement} (heightened to 5th level) (1 charge)
\item \spell{continual flame} (1 charge)
\item \spell{levitate} (1 charge)
\item \spell{lightning bolt} (heightened to 5th level) (1 charge)
\item \spell{fireball} (heightened to 5th level) (1 charge)
\item \spell{cone of cold} (2 charges)
\item \spell{hold monster} (2 charges)
\item \spell{wall of force} (in a 3-m-diameter hemisphere around the caster only) (2 charges)
\item \spell{globe of invulnerability} (2 charges)
\end{itemize*}

The wielder of a \emph{staff of power} gains a +2 luck bonus to AC and saving throws. The staff is also a \emph{+2 quarterstaff}, and its wielder may use it to smite opponents. If 1 charge is expended (as a free action), the staff causes double damage ($\times$3 on a critical hit) for 1 round.

A \emph{staff of power} can be used for a retributive strike, requiring it to be broken by its wielder. (If this breaking of the staff is purposeful and declared by the wielder, it can be performed as a standard action that does not require the wielder to make a Strength check.) All charges currently in the staff are instantly released in a 9-meter radius. All within 2 squares of the broken staff take points of damage equal to 8 $\times$ the number of charges in the staff, those 3 or 4 squares away take 6 $\times$ the number of charges in damage, and those 5 or 6 squares distant take 4 $\times$ the number of charges in damage. All those affected can make DC 17 Reflex saves to reduce the damage by half.

The character breaking the staff has a 50\% chance of traveling to another plane of existence, but if he does not, the explosive release of spell energy destroys him. Only certain items, including the staff of the magi and the \emph{staff of power}, are capable of being used for a retributive strike.

After all charges are used up from the staff, it remains a \emph{+2 quarterstaff}. (Once empty of charges, it cannot be used for a retributive strike.)

Strong varied; CL 15th; \feat{Craft Staff}, \feat{Craft Magic Arms and Armor}, \spell{magic missile}, heightened \spell{ray of enfeeblement}, \spell{continual flame}, \spell{levitate}, heightened \spell{fireball}, heightened \spell{lightning bolt}, \spell{cone of cold}, \spell{hold monster}, \spell{wall of force}, \spell{globe of invulnerability}; Price 211,000 cp.

\textbf{Size Alteration:} Stout and sturdy, this staff of dark wood allows use of the following spells:
\begin{itemize*}
\item \spell{enlarge person} (1 charge)
\item \spell{reduce person} (1 charge)
\item \spell{shrink item} (1 charge)
\item \spell{mass enlarge person} (1 charge)
\item \spell{mass reduce person} (1 charge)
\end{itemize*}

Faint conjuration; CL 8th; \feat{Craft Staff}, \spell{enlarge person}, \spell{mass enlarge person}, \spell{reduce person}, \spell{mass reduce person}, \spell{shrink item}; Price 29,000 cp.

\textbf{Swarming Insects:} Made of twisted dark wood with dark spots resembling crawling insects (which occasionally seem to move), this staff allows use of the following spells:
\begin{itemize*}
\item \spell{summon swarm} (1 charge)
\item \spell{insect plague} (3 charges)
\end{itemize*}

Moderate conjuration; CL 9th; \feat{Craft Staff}, \spell{insect plague}, \spell{summon swarm}; Price 24,750 cp.

\textbf{Transmutation:} This staff is generally carved from or decorated with petrified wood and allows use of the following spells:
\begin{itemize*}
\item \spell{expeditious retreat} (1 charge)
\item \spell{alter self} (1 charge)
\item \spell{blink} (1 charge)
\item \spell{polymorph} (2 charges)
\item \spell{baleful polymorph} (2 charges)
\item \spell{disintegrate} (3 charges)
\end{itemize*}

Strong transmutation; CL 13th; \feat{Craft Staff}, \spell{alter self}, \spell{baleful polymorph}, \spell{blink}, \spell{disintegrate}, \spell{expeditious retreat}, \spell{polymorph}; Price 65,000 cp.

\textbf{Trickster:} Mostly seen in the hands of elven wizards, this one-meter long bone staff is crafted from drake ivory and covered with elaborate carvings depicting laughing and smiling male and female elven faces intertwined with smaller, full-body reliefs of elves dancing, running, or embracing in positions of physical pleasure. The staff is adorned with small, embedded obsidian spheres that spiral around its haft. Legend has it that the first of these items was crafted by the elven defiler and trickster Daaku, renowned for his audacious and daring schemes and exploits, sometime before his mysterious disappearance after making enemies of the Shadows tribe.

This staff allows use of the following spells:
\begin{itemize*}
\item \spell{magic aura} (1 charge)
\item \spell{disguise self} (1 charge)
\item \spell{expeditious retreat} (1 charge)
\item \spell{greater invisibility} (2 charges)
\item \spell{dimension door} (2 charges)
\end{itemize*}

This staff also grants its wielder a +5 bonus to \skill{Bluff} checks and the ability to use \spell{ghost sound}, as a spell-like ability at caster level 9th, at will. These two attributes continue to function even after all the charges have been expended.

Moderate varied; CL 9th; \feat{Craft Staff}, \spell{dimension door}, \spell{disguise self}, \spell{expeditious retreat}, \spell{ghost sound}, \spell{greater invisibility}, \spell{magic aura}; Price 33,200 cp.

\textbf{Woodlands:} Appearing to have grown naturally into its shape, this oak, ash, or yew staff allows use of the following spells:
\begin{itemize*}
\item \spell{charm animal} (1 charge)
\item \spell{speak with animals} (1 charge)
\item \spell{barkskin} (2 charges)
\item \spell{wall of thorns} (3 charges)
\item \spell{summon nature's ally VI} (3 charges)
\item \spell{animate plants} (4 charges)
\end{itemize*}

The staff may be used as a weapon, functioning as a \emph{+2 quarterstaff}. The staff of the woodlands also allows its wielder to \spell{pass without trace} at will, with no charge cost. These two attributes continue to function after all the charges are expended.

Moderate varied; CL 13th; \feat{Craft Staff}, \feat{Craft Magic Arms and Armor}, \spell{animate plants}, \spell{barkskin}, \spell{charm animal}, \spell{pass without trace}, \spell{speak with animals}, \spell{summon nature's ally VI}, \spell{wall of thorns}; Price 101,250 cp. 


\Table{Wands I}{Xr{15mm}}{
\tableheader Wand & \tableheader Market Price \\
\spell{defiler scent}                                     &    375 cp \\
\spell{detect magic}                                      &    375 cp \\
\spell{light}                                             &    375 cp \\
\spell{burning hands}                                     &    750 cp \\
\spell{charm animal}                                      &    750 cp \\
\spell{charm person}                                      &    750 cp \\
\spell{color spray}                                       &    750 cp \\
\spell{cooling canopy}                                    &    750 cp \\
\spell{cure light wounds}                                 &    750 cp \\
\spell{detect secret doors}                               &    750 cp \\
\spell{enlarge person}                                    &    750 cp \\
\spell{image of the sorcerer-king}                        &    750 cp \\
\spell{magic missile} (1st)                               &    750 cp \\
\spell{shocking grasp}                                    &    750 cp \\
\spell{summon monster I}                                  &    750 cp \\
\spell{magic missile} (3rd)                               &  2,250 cp \\
\spell{magic missile} (5th)                               &  3,750 cp \\
\spell{acid arrow}                                        &  4,500 cp \\
\spell{battlefield healing}                               &  4,500 cp \\
\spell{bear's endurance}                                  &  4,500 cp \\
\spell{bull's strength}                                   &  4,500 cp \\
\spell{cat's grace}                                       &  4,500 cp \\
\spell{cure moderate wounds}                              &  4,500 cp \\
\spell{darkness}                                          &  4,500 cp \\
\spell{daze monster}                                      &  4,500 cp \\
\spell{delay poison}                                      &  4,500 cp \\
\spell{eagle's splendor}                                  &  4,500 cp \\
\spell{false life}                                        &  4,500 cp \\
\spell{fox's cunning}                                     &  4,500 cp \\
\spell{ghoul touch}                                       &  4,500 cp \\
\spell{hold person}                                       &  4,500 cp \\
\spell{invisibility}                                      &  4,500 cp \\
\spell{knock}                                             &  4,500 cp \\
\spell{levitate}                                          &  4,500 cp \\
\spell{mirror image}                                      &  4,500 cp \\
\spell{owl's wisdom}                                      &  4,500 cp \\
\spell{shatter}                                           &  4,500 cp \\
\spell{silence}                                           &  4,500 cp \\
\spell{summon monster II}                                 &  4,500 cp \\
\spell{web}                                               &  4,500 cp \\
\spell{magic missile} (7th)                               &  5,250 cp \\
\spell{magic missile} (9th)                               &  6,750 cp \\
}

\Table{Wands II}{Xr{15mm}}{
\tableheader Wand & \tableheader Market Price \\
\spell{call lightning} (5th)                              & 11,250 cp \\
\spell{charm person}, heightened (3rd-level spell)        & 11,250 cp \\
\spell{contagion}                                         & 11,250 cp \\
\spell{cure serious wounds}                               & 11,250 cp \\
\spell{dispel magic}                                      & 11,250 cp \\
\spell{fireball} (5th)                                    & 11,250 cp \\
\spell{image of the sorcerer-king}                        & 11,250 cp \\
\spell{keen edge}                                         & 11,250 cp \\
\spell{lightning bolt} (5th)                              & 11,250 cp \\
\spell{major image}                                       & 11,250 cp \\
\spell{slow}                                              & 11,250 cp \\
\spell{suggestion}                                        & 11,250 cp \\
\spell{summon monster III}                                & 11,250 cp \\
\spell{surface walk}                                      & 11,250 cp \\
\spell{fireball} (6th)                                    & 13,500 cp \\
\spell{lightning bolt} (6th)                              & 13,500 cp \\
\spell{searing light} (6th)                               & 13,500 cp \\
\spell{call lightning} (8th)                              & 18,000 cp \\
\spell{fireball} (8th)                                    & 18,000 cp \\
\spell{lightning bolt} (8th)                              & 18,000 cp \\
\spell{charm monster}                                     & 21,000 cp \\
\spell{cure critical wounds}                              & 21,000 cp \\
\spell{dimensional anchor}                                & 21,000 cp \\
\spell{fear}                                              & 21,000 cp \\
\spell{greater invisibility}                              & 21,000 cp \\
\spell{hold person}, heightened (4th-level spell)         & 21,000 cp \\
\spell{ice storm}                                         & 21,000 cp \\
\spell{inflict critical wounds}                           & 21,000 cp \\
\spell{mage seeker}                                       & 21,000 cp \\
\spell{neutralize poison}                                 & 21,000 cp \\
\spell{poison}                                            & 21,000 cp \\
\spell{polymorph}                                         & 21,000 cp \\
\spell{ray of enfeeblement}, heightened (4th-level spell) & 21,000 cp \\
\spell{suggestion}, heightened (4th-level)                & 21,000 cp \\
\spell{summon monster IV}                                 & 21,000 cp \\
\spell{wall of fire}                                      & 21,000 cp \\
\spell{wall of ice}                                       & 21,000 cp \\
\spell{wrath of the sorcerer-king}                        & 21,000 cp \\
\spell{dispel magic} (10th)                               & 22,500 cp \\
\spell{fireball} (10th)                                   & 22,500 cp \\
\spell{lightning bolt} (10th)                             & 22,500 cp \\
\spell{chaos hammer} (8th)                                & 24,000 cp \\
\spell{holy smite} (8th)                                  & 24,000 cp \\
\spell{order's wrath} (8th)                               & 24,000 cp \\
\spell{unholy blight} (8th)                               & 24,000 cp \\
\spell{restoration}\footnotemark[1]                       & 26,000 cp \\
\spell{stoneskin}\footnotemark[2]                         & 33,500 cp \\
\TableNote{2}{1 The cost to create a wand of \spell{restoration} is 10,500 cp, 840 XP, plus 5,000 cp for the material components.}\\
\TableNote{2}{1 The cost to create a wand of \spell{stoneskin} is 10,500 cp, 840 XP, plus 12,500 cp for the material components.}\\
}

\subsectionA{Wands}
A wand is a thin baton that contains a single spell of 4th level or lower. Each wand has 50 charges when created, and each charge expended allows the user to use the wand's spell one time. A wand that runs out of charges is just a stick.

\textbf{Physical Description:} A typical wand is 15 centimeters to 30 centimeters long and about 60 milimeters thick, and often weighs no more than 30 grams. Most wands are wood, but some are bone. A rare few are metal, glass, or even ceramic, but these are quite exotic. Occasionally, a wand has a gem or some device at its tip, and most are decorated with carvings or runes. A typical wand has AC 7, 5 hit points, hardness 5, and a break DC of 16.

\textbf{Activation:} Wands use the spell trigger activation method, so casting a spell from a wand is usually a standard action that doesn't provoke attacks of opportunity. (If the spell being cast, however, has a longer casting time than 1 standard action, it takes that long to cast the spell from a wand.) To activate a wand, a character must hold it in hand (or whatever passes for a hand, for nonhumanoid creatures) and point it in the general direction of the target or area. A wand may be used while grappling or while swallowed whole.

\subsubsection{Wand Descriptions}
All wands are simply storage devices for spells and thus have no special descriptions. Refer to the spell descriptions for all pertinent details. 

\subsectionA{Wondrous Items}
This is a catch-all category for anything that doesn’t fall into the other groups. Anyone can use a wondrous item (unless specified otherwise in the description).

\textbf{Physical Description:} Varies.

\textbf{Activation:} Usually use-activated or command word, but details vary from item to item.

\textbf{Special Qualities:} Roll d\%. An 01 result indicates the wondrous item is intelligent, 02--31 indicates that something (a design, inscription, or the like) provides a clue to its function, and 32--100 indicates no special qualities. Intelligent items have extra abilities and sometimes extraordinary powers and special purposes.

Wondrous items with charges can never be intelligent.

\subsubsection{Neck}

\textbf{Amulet of Health:} This amulet is a golden disk on a chain. It usually bears the image of a lion or other powerful animal. The wearer's Constitution score is 19 while they wear this amulet. It has no effect if their Constitution is already 19 or higher.

Moderate transmutation; CL 8th; \feat{Craft Wondrous Item}, \spell{bear's endurance}; Price 16,000 cp.


\textbf{Amulet of Mighty Fists:} This amulet grants an enhancement bonus of +1 to +5 on attack and damage rolls with unarmed attacks and natural weapons.

Faint evocation; CL 5th; \feat{Craft Wondrous Item}, \spell{greater magic fang}, creator's caster level must be at least three times the amulet's bonus; Price 6,000 cp (+1), 24,000 cp (+2), 54,000 cp (+3), 96,000 cp (+4), 150,000 cp (+5).


\textbf{Amulet of Natural Armor:} This amulet, usually crafted from bone or beast scales, toughens the wearer's body and flesh, giving him an enhancement bonus to his natural armor bonus of from +1 to +5, depending on the kind of amulet.

Faint transmutation; CL 5th; \feat{Craft Wondrous Item}, \spell{barkskin}, creator's caster level must be at least three times the amulet's bonus; Price 2,000 cp (+1), 8,000 cp (+2), 18,000 cp (+3), 32,000 cp (+4), or 50,000 cp (+5).


\textbf{Amulet of the Planes:} This device usually appears to be a black circular amulet, although any character looking closely at it sees a dark, moving swirl of color. The amulet allows its wearer to utilize plane shift. However, this is a difficult item to master. The user must make a DC 15 Intelligence check in order to get the amulet to take her to the plane (and the specific location on that plane) that she wants. If she fails, the amulet transports her and all those traveling with her to a random location on that plane (01--60 on d\d) or to a random plane (61--100).

Strong conjuration; CL 15th; \feat{Craft Wondrous Item}, \spell{plane shift}; Price 120,000 cp.


\textbf{Amulet of Proof against Detection and Location:} This silver amulet protects the wearer from scrying and magical location just as a \spell{nondetection} spell does. If a divination spell is attempted against the wearer, the caster of the divination must succeed on a caster level check (1d20 + caster level) against a DC of 19 (as if the caster had cast \spell{nondetection} on herself).

Moderate abjuration; CL 8th; \feat{Craft Wondrous Item}, \spell{nondetection}; Price 35,000 cp.


\textbf{Hand of Glory:} This mummified human hand hangs by a leather cord around a character's neck (taking up space as a magic necklace would). If a magic ring is placed on one of the fingers of the hand, the wearer benefits from the ring as if wearing it herself, and it does not count against her two-ring limit. The hand can wear only one ring at a time. Even without a ring, the hand itself allows its wearer to use \spell{daylight} and \spell{see invisibility} each once per day.

Faint varied; CL 5th; \feat{Craft Wondrous Item}, \spell{animate dead}, \spell{daylight}, \spell{see invisibility}; Price 8,000 cp; Weight 1 kg.


\textbf{Hand of the Mage:} This mummified elf hand hangs by a golden chain around a character's neck (taking up space as a magic necklace would). It allows the wearer to utilize the spell \spell{mage hand} at will.

Faint transmutation; CL 2nd; \feat{Craft Wondrous Item}, \spell{mage hand}; Price 900 cp; Weight 1 kg.


\textbf{Medallion of Thoughts:} This appears to be a normal pendant disk hung from a neck chain. Usually fashioned from bronze, copper, or nickel-silver, the medallion allows the wearer to read the thoughts of others, as with the spell \spell{detect thoughts}.

Faint divination; CL 5th; \feat{Craft Wondrous Item}, \spell{detect thoughts}; Price 12,000 cp.


\textbf{Necklace of Adaptation:} This necklace is a heavy chain with a platinum medallion. The magic of the necklace wraps the wearer in a shell of fresh air, making him immune to all harmful vapors and gases (such as cloudkill and stinking cloud effects, as well as inhaled poisons) and allowing him to breathe, even underwater or in a vacuum.

Moderate transmutation; CL 7th; \feat{Craft Wondrous Item}, \spell{alter self}; Price 9,000 cp.


\textbf{Necklace of Fireballs:} This device appears to be nothing but beads on a string, sometimes with the ends tied together to form a necklace. (It does not count as an item worn around the neck for the purpose of determining which of a character's worn magic items is effective.) If a character holds it, however, all can see the strand as it really is---a golden chain from which hang a number of golden spheres. The spheres are detachable by the wearer (and only by the wearer), who can easily hurl one of them up to 70 feet. When a sphere arrives at the end of its trajectory, it detonates as a \spell{fireball} spell (Reflex DC 14 half).

Spheres come in different strengths, ranging from those that deal 2d6 points of fire damage to those that deal 10d6. The market price of a sphere is 150 cp for each die of damage it deals.

Each \emph{necklace of fireballs} contains a combination of spheres of various strengths. Some traditional combinations, designated types I through VII, are detailed below.

\Table{}{l *9c R}{
  \tableheader Necklace
& \tableheader 10d6
& \tableheader 9d6
& \tableheader 8d6
& \tableheader 7d6
& \tableheader 6d6
& \tableheader 5d6
& \tableheader 4d6
& \tableheader 3d6
& \tableheader 2d6
& \tableheader Market Price \\
Type I   &   &   &   &   &   & 1 &   & 2 &   & 1,650 cp \\
Type II  &   &   &   &   & 1 &   & 2 &   & 2 & 2,700 cp \\
Type III &   &   &   & 1 &   & 2 &   & 4 &   & 4,350 cp \\
Type IV  &   &   & 1 &   & 2 &   & 2 &   & 4 & 5,400 cp \\
Type V   &   & 1 &   & 2 &   & 2 &   & 2 &   & 5,850 cp \\
Type VI  & 1 &   & 2 &   & 2 &   & 4 &   &   & 8,100 cp \\
Type VII & 1 & 2 &   & 2 &   & 2 &   & 2 &   & 8,700 cp \\
}

If the necklace is being worn or carried by a character who fails her saving throw against a magical fire attack, the item must make a saving throw as well (with a save bonus of +7). If the necklace fails to save, all its remaining spheres detonate simultaneously, often with regrettable consequences for the wearer.

Moderate evocation; CL 10th; \feat{Craft Wondrous Item}, \spell{fireball}.


\textbf{Periapt of Health:} The wearer of this blue gem on a silver chain is immune to disease, including supernatural diseases.

Faint conjuration; CL 5th; \feat{Craft Wondrous Item}, \spell{remove disease}; Price 7,400 cp.


\textbf{Periapt of Proof against Poison:} This item is a brilliant-cut black gem on a delicate silver chain. The wearer is immune to poison, although poisons still active when the periapt is first donned still run their course.

Faint conjuration; CL 5th; \feat{Craft Wondrous Item}, \spell{neutralize poison}; Price 27,000 cp.


\textbf{Periapt of Wisdom:} Although it appears to be a normal pearl on a light chain, a \emph{periapt of wisdom} actually changes increases the possessor's Wisdom score to 19. It has no effect if their Wisdom is already 19 or higher.

Moderate transmutation; CL 8th; \feat{Craft Wondrous Item}, \spell{owl's wisdom}; Price 16,000 cp.


\textbf{Periapt of Wound Closure:} This stone is bright red and dangles on a gold chain. The wearer of this periapt automatically becomes stable if his hit points drop to between $-1$ and $-9$ inclusive. The periapt doubles the wearer's normal rate of healing or allows normal healing of wounds that would not do so normally. Hit point damage that involves bleeding is negated for the wearer of the periapt, but he is still susceptible to damage from bleeding that causes Constitution loss, such as that dealt by a wounding weapon.

Moderate conjuration; CL 10th; \feat{Craft Wondrous Item}, \spell{heal}; Price 15,000 cp.


\textbf{Scarab of Protection:} This device appears to be a silver medallion in the shape of a beetle. If it is held for 1 round, an inscription appears on its surface letting the holder know that it is a protective device.

The scarab's possessor gains spell resistance 20. The scarab can also absorb energy-draining attacks, death effects, and negative energy effects. Upon absorbing twelve such attacks, the scarab turns to powder and is destroyed.

Strong abjuration and necromancy; CL 18th; \feat{Craft Wondrous Item}, \spell{death ward}, \spell{spell resistance}; Price 38,000 cp.


\textbf{Scarab, Golembane:} This beetle-shaped pin enables its wearer to detect any golem within 18 meters, although he must concentrate (a standard action) in order for the detection to take place. A scarab enables its possessor to combat golems with weapons, unarmed attacks, or natural weapons as if those golems had no damage reduction.

Moderate divination; CL 8th; \feat{Craft Wondrous Item}, \spell{detect magic}, creator must be at least 10th level; Price 2,500 cp.

\subsubsection{Feet}

\textbf{Boots of Elvenkind:} These soft boots enable the wearer to move quietly in virtually any surroundings, granting a +5 competence bonus on Move Silently checks.

Faint transmutation; CL 5th; \feat{Craft Wondrous Item}, creator must be an elf; Price 2,500 cp; Weight 1 lb.

\textbf{Boots of Levitation:} On command, these leather boots allow the wearer to levitate as if she had cast levitate on herself.

Faint transmutation; CL 3rd; \feat{Craft Wondrous Item}, \spell{levitate}; Price 7,500 cp; Weight 1 lb.

\textbf{Boots of Speed:} As a free action, the wearer can click her boot heels together, enabling her to act as though affected by a haste spell for up to 10 rounds each day. The duration of the haste effect need not be consecutive rounds.

Moderate transmutation; CL 10th; \feat{Craft Wondrous Item}, \spell{haste}; Price 12,000 cp; Weight 1 lb.

\textbf{Boots of Striding and Springing:} These boots increase the wearer's base land speed by 10 feet. In addition to this striding ability (considered an enhancement bonus), these boots allow the wearer to make great leaps. She can jump with a +5 competence bonus on Jump checks.

Faint transmutation; CL 3rd; \feat{Craft Wondrous Item}, longstrider, creator must have 5 ranks in the Jump skill; Price 5,500 cp; Weight 1 lb.

\textbf{Boots of Teleportation:} Any character wearing this footwear may teleport three times per day, exactly as if he had cast the spell of the same name.

Moderate conjuration; CL 9th; \feat{Craft Wondrous Item}, \spell{teleport}; Price 49,000 cp; Weight 3 lb.

\textbf{Boots of the Winterlands:} This footgear bestows many powers upon the wearer. First, he is able to travel across snow at his normal speed, leaving no tracks. The boots also enable him to travel at normal speed across the most slippery ice (horizontal surfaces only, not vertical or sharply slanted ones) without falling or slipping. Finally, boots of the winterlands warm the wearer, as if he were affected by an endure elements spell.

Faint abjuration and transmutation; CL 5th; \feat{Craft Wondrous Item}, \spell{cat's grace}, \spell{endure elements}, \spell{pass without trace}; Price 2,500 cp; Weight 1 lb.

\textbf{Boots, Winged:} These boots appear to be ordinary footgear. On command, the boots sprout wings at the heel and let the wearer fly, without having to maintain concentration, as if affected by a fly spell. He can fly three times per day for up to 5 minutes per flight.

Faint transmutation; CL 5th; \feat{Craft Wondrous Item}, \spell{fly}; Price 16,000 cp; Weight 1 lb.

\textbf{Horseshoes of Speed:} These iron shoes come in sets of four like ordinary horseshoes. When affixed to an animal's hooves, they increase the animal's base land speed by 30 feet; this counts as an enhancement bonus. As with other effects that increase speed, jumping distances increase proportionally. All four shoes must be worn by the same animal for the magic to be effective.

Faint transmutation; CL 3rd; \feat{Craft Wondrous Item}, \spell{haste}; Price 3,000 cp; Weight 12 lb. (for four).

\textbf{Horseshoes of a Zephyr:} These four iron shoes are affixed like normal horseshoes. They allow a horse to travel without actually touching the ground. The horse must still run above (always around 4 inches above) a roughly horizontal surface. This means that nonsolid or unstable surfaces can be crossed, and that movement is possible without leaving tracks on any sort of ground. The horse moves at its normal base land speed. All four shoes must be worn by the same animal for the magic to be effective.

Faint transmutation; CL 3rd; \feat{Craft Wondrous Item}, \spell{levitate}; Price 6,000 cp; Weight 4 lb. (for four).

\subsectionA{Living Magical Items}
Preservers of a bygone age created means to help their cause with magical trees that could store and channel powers. The most common and fragile of them is a \emph{potion tree}, which can be planted by a very skillful gardener. The other one is a \emph{tree of life}, which requires a powerful spellcaster to create and may live forever.

\subsubsection{Potion Trees}
In Athas, \emph{potion fruits} can be planted in order to grow more \emph{potion fruits}. Once planted, they need to be tended every day for 1d6 weeks (\skill{Profession} (gardener) DC 25, each day). After that time, the \emph{potion tree} grows 1d3 new \emph{potion fruits} among all nonmagical fruits. Those are all the \emph{potion fruits} tree will normally give.

If a \spell{permanency} spell is cast on it, the tree continues to grow \emph{potion fruits}. A \emph{permanent potion tree} can hold only one \emph{potion fruit}, but once it is picked a new one grows in 1d6 days.

\emph{Potion trees} are very sensitive to its environment. Any severe change in the weather, such as a drought or freeze, will ruin the tree and it will not bear any fruits. Any defiling that happens near the tree will kill it and its fruits.

\Figure{t}{images/tree-of-life.png}
\subsubsection{Trees of Life}
A \emph{tree of life} is a powerful magical tree, that bolsters the capabilities of elemental clerics and druids. Initially created during the Preserver Jihad, this type of living magical item was designed to withstand the destruction defilers were causing on vegetation. Since they are virtually eternal, most of those existing in the present predate the villages that have grown around them.

A \emph{tree of life} can endure any climate or terrain---they will flourish in the middle of the desert, or on a rocky mountain, through drought, lightning, and even earthquakes. Nothing in the natural world can affect a \emph{tree}, as when chopped down it will regrow back to full size after a month. They can sustain even the effects of the defiling radius.

This ability to sustain destruction is why defilers seek those trees. Sorcerer-monarchs have groves of \emph{trees of life}, where they and their royal defilers can exercise magic without decimating their cities.

A \emph{tree of life} is always created from a living sapling, no older than one year. After a week, it grows to its full size. It has all its abilities at time of creation.

\textbf{Granted Abilities:} Elemental clerics and druids in contact with a \emph{tree of life} gain the following spell-like abilities: 1/day---\spell{augury}, \spell{divination}, \spell{heal}, and \spell{scrying}.

\textbf{Vital Force:} A \emph{tree of life} consists of two parts, its physical form and its vital force. It can only be destroyed if both parts die. The physical form can be destroyed as a normal tree, as it is not magical. A \emph{tree of life}'s vital force has a total of 100 hit points, divided in 10 layers of 10 hit points each. It regenerates one layer per hour, and it can be affected in the following ways:
\begin{itemize*}
    \item Each negative level removes one layer;
    \item A necromancy death spell that can target a living tree removes three layers;
    \item A \emph{restoration} spell restores one layer;
    \item A \emph{raise dead} or \emph{resurrection} spell restore three layers;
    \item A \emph{greater restoration} or \emph{true resurrection} spell restore all layers;
    \item Only \spell{wish} or \spell{miracle} can outright slay a \emph{tree of life}.
\end{itemize*}

Defiling will damage its vital force as well. Whenever a wizard defiles within 90 meters of a \emph{tree of life}, it loses a layer for each spell level cast. This damage replaces the normal defiling radius, protecting surrounding vegetation.

Strong transmutation; CL 15th; \feat{Craft Wondrous Item}, creator must have 15 ranks in the \skill{Knowledge} (nature) skill; Price 60,000 cp.

