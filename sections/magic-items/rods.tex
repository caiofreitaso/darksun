\subsectionA{Rods}
Rods are scepterlike devices that have unique magical powers and do not usually have charges. Anyone can use a rod.
Physical Description

Rods weigh approximately 2.5 kilograms.

They range from 50 centimeters to 1 meter long and are usually made of iron or some other metal. (Many, as noted in their descriptions, can function as light maces or clubs due to their sturdy construction.)

These sturdy items have AC 9, 10 hit points, hardness 10, and a break DC of 27.

\textbf{Activation:} Details relating to rod use vary from item to item. See the individual descriptions for specifics.

\Table{Rods}{Xr{15mm}}{
\tableheader Rod & \tableheader Market Price \\
Metamagic, Enlarge, lesser   & 3,000 cp \\
Metamagic, Extend, lesser    & 3,000 cp \\
Metamagic, Silent, lesser    & 3,000 cp \\
Immovable                    & 5,000 cp \\
Metamagic, Empower, lesser   & 9,000 cp \\
Metal and mineral detection  & 10,500 cp \\
Cancellation                 & 11,000 cp \\
Elemental                    & 11,000 cp \\
Metamagic, Enlarge           & 11,000 cp \\
Metamagic, Extend            & 11,000 cp \\
Metamagic, Silent            & 11,000 cp \\
Wonder                       & 12,000 cp \\
Python                       & 13,000 cp \\
Metamagic, Maximize, lesser  & 14,000 cp \\
Flame extinguishing          & 15,000 cp \\
Viper                        & 19,000 cp \\
Enemy detection              & 23,500 cp \\
Metamagic, Enlarge, greater  & 24,500 cp \\
Metamagic, Extend, greater   & 24,500 cp \\
Metamagic, Silent, greater   & 24,500 cp \\
Splendor                     & 25,000 cp \\
Withering                    & 25,000 cp \\
Metamagic, Empower           & 32,500 cp \\
Thunder and lightning        & 33,000 cp \\
Metamagic, Quicken, lesser   & 35,000 cp \\
Negation                     & 37,000 cp \\
Absorption                   & 50,000 cp \\
Flailing                     & 50,000 cp \\
Metamagic, Maximize          & 54,000 cp \\
Desiccating                  & 60,000 cp \\
Rulership                    & 60,000 cp \\
Security                     & 61,000 cp \\
% Lordly might                 & 70,000 cp \\
Metamagic, Empower, greater  & 73,000 cp \\
Guardianship                 & 75,000 cp \\
Metamagic, Quicken           & 75,500 cp \\
Alertness                    & 85,000 cp \\
Metamagic, Maximize, greater & 121,500 cp \\
Metamagic, Quicken, greater  & 170,000 cp \\
}

\subsubsection{Rod Descriptions}
Although all rods are generally scepterlike, their configurations and abilities run the magical gamut. Standard rods are described below.

\textbf{Absorption:} This rod acts as a magnet, drawing spells or spell-like abilities into itself. The magic absorbed must be a single-target spell or a ray directed at either the character possessing the rod or her gear. The rod then nullifies the spell's effect and stores its potential until the wielder releases this energy in the form of spells of her own. She can instantly detect a spell's level as the rod absorbs that spell's energy. Absorption requires no action on the part of the user if the rod is in hand at the time.

A running total of absorbed (and used) spell levels should be kept. The wielder of the rod can use captured spell energy to cast any spell she has prepared, without expending the preparation itself. The only restrictions are that the levels of spell energy stored in the rod must be equal to or greater than the level of the spell the wielder wants to cast, that any material components required for the spell be present, and that the rod be in hand when casting. For casters such as templars who do not prepare spells, the rod's energy can be used to cast any spell of the appropriate level or levels that they know.

A \emph{rod of absorption} absorbs a maximum of fifty spell levels and can thereafter only discharge any remaining potential it might have. The rod cannot be recharged. The wielder knows the rod's remaining absorbing potential and current amount of stored energy.

To determine the absorption potential remaining in a newly found rod, roll d\% and divide the result by 2. Then roll d\% again: On a result of 71-100, half the levels already absorbed by the rod are still stored within.

Strong abjuration; CL 15th; \feat{Craft Rod}, \spell{spell turning}; Price 50,000 cp.

\textbf{Alertness:} This rod is indistinguishable from a \emph{+1 light mace}. It has eight flanges on its macelike head. The rod bestows a +1 insight bonus on initiative checks. If grasped firmly, the rod enables the holder to use \spell{detect evil}, \spell{detect good}, \spell{detect chaos}, \spell{detect law}, \spell{detect magic}, \spell{discern lies}, \spell{light}, or \spell{see invisibility}. Each different use is a standard action.

If the head of a \emph{rod of alertness} is planted in the ground, and the possessor wills it to alertness (a standard action), the rod senses any creature within 36 meters who intends to harm the possessor. At the same time, the rod creates the effect of a prayer spell upon all creatures friendly to the possessor in a 6-meter radius. Immediately thereafter, the rod sends forth a mental alert to these friendly creatures, warning them of possible danger from the unfriendly creature or creatures within the 36-meter radius. These effects last for 10 minutes, and the rod can perform this function once per day. Last, the rod can be used to simulate the casting of an animate objects spell, utilizing any eleven (or fewer) Small objects located roughly around the perimeter of a 1.5-meter-radius circle centered on the rod when planted in the ground. Objects remain animated for 11 rounds. The rod can perform this function once per day.

Moderate abjuration, divination, enchantment, and evocation; CL 11th; \feat{Craft Rod}, \spell{alarm}, \spell{detect chaos}, \spell{detect evil}, \spell{detect good}, \spell{detect law}, \spell{detect magic}, \spell{discern lies}, \spell{light}, \spell{see invisibility}, \spell{prayer}, \spell{animate objects}; Price 85,000 cp.

\textbf{Cancellation:} This dreaded rod is a bane to magic items, for its touch drains an item of all magical properties. The item touched must make a DC 23 Will save to prevent the rod from draining it. If a creature is holding it at the time, then the item can use the holder's Will save bonus in place of its own if the holder's is better. In such cases, contact is made by making a melee touch attack roll. Upon draining an item, the rod itself becomes brittle and cannot be used again. Drained items are only restorable by \spell{wish} or \spell{miracle}. (If a \emph{sphere of annihilation} and a \emph{rod of cancellation} negate each other, nothing can restore either of them.)

Strong abjuration; CL 17th; \feat{Craft Rod}, \spell{mage's disjunction}; Price 11,000 cp.

\textbf{Desiccating:} This rod is crafted from the thigh-bone of a thrax, with a series of gaunt, dehydrated faces carved into it in a spiral. The \emph{desiccating rod} is often used by silt and sun clerics to drain water from living creatures. This rod is wielded as a \emph{+1 club} that deals 1d6+5 points of temporary Constitution damage to any creature touched (by making a melee touch attack) instead of the usual hit point damage. Oozes, plants and creatures with the aquatic subtype are more susceptible to this attack and instead take 1d8+5 points of Constitution damage. In either case, the defender negates the effect with a DC 22 Fortitude save.

Against creatures immune to ability score damage or who have no Constitution score, this rod causes damage as normal for a \emph{+1 club}.

Strong necromancy; CL 15th; \feat{Craft Rod}, \feat{Craft Magic Arms and Armor}, \spell{horrid wilting}; Price 60,000 cp.

\textbf{Elemental:} These widely varied looking rods are crafted by the clerics of all elemental affiliations to help them in their quest to restore or augment the power of their patron element. Although these rods are made specifically by and for elemental clerics, any abilties confered by a rod and unrelated to spellcasting are accessible to anybody possessing one, unless otherwise noted. One specific rod exists for every domain available to Athasian clerics.

The possessor of the rod gains a +4 bonus to the class skill added by the possession of the corresponding domain.

% An elemental rod allows the casting, once per day, of any one of the domain spells on the corresponding domain list, up to the maximum current level achieved by the wielder, but only if the wielder has access to said domain.

Finally, once per day, the rod can be used as a divine focus for a spell on the domain spell list. While used so, the spell cast by the wearer has its the save DC increased by +1. If no saving throw applies or is allowed, it instead adds +1 to the effective caster level of the effect. This ability can be used by any caster provided he has the domain spell on his spell list.

Strong evocation; CL 12th; \feat{Craft Rod}, \spell{imbue with spell ability}, access to the associated domain; Price 11,000 cp.

\textbf{Enemy Detection:} This device pulses in the wielder's hand and points in the direction of any creature or creatures hostile to the bearer of the device (nearest ones first). These creatures can be invisible, ethereal, hidden, disguised, or in plain sight. Detection range is 18 meters. If the bearer of the rod concentrates for a full round, the rod pinpoints the location of the nearest enemy and indicates how many enemies are within range. The rod can be used three times each day, each use lasting up to 10 minutes. Activating the rod is a standard action.

Moderate divination; CL 10th; \feat{Craft Rod}, \spell{true seeing}; Price 23,500 cp.

\textbf{Flailing:} Upon the command of its possessor, the rod activates, changing from a normal-seeming rod to a \emph{+3 dire flail}. The dire flail is a double weapon, which means that each of the weapon's heads can be used to attack. The wielder can gain an extra attack (with the second head) at the cost of making all attacks at a $-2$ penalty (as if she had the \feat{Two-Weapon Fighting} feat).

Once per day the wielder can use a free action to cause the rod to grant her a +4 deflection bonus to Armor Class and a +4 resistance bonus on saving throws for 10 minutes. The rod need not be in weapon form to grant this benefit.

Transforming it into a weapon or back into a rod is a move action.

Moderate enchantment; CL 9th; \feat{Craft Rod}, \feat{Craft Magic Arms and Armor}, \spell{bless}; Price 50,000 cp.

\textbf{Flame Extinguishing:} This rod can extinguish Medium or smaller nonmagical fires with simply a touch (a standard action). For the rod to be effective against other sorts of fires, the wielder must expend 1 or more of the rod's charges.

Extinguishing a Large or larger nonmagical fire, or a magic fire of Medium or smaller (such as that of a flaming weapon or a \spell{burning hands} spell), expends 1 charge. Continual magic flames, such as those of a weapon or a fire creature, are suppressed for 6 rounds and flare up again after that time. To extinguish an instantaneous fire spell, the rod must be within the area of the effect and the wielder must have used a ready action, effectively countering the entire spell.

When applied to Large or larger magic fires, such as those caused by \spell{fireball}, \spell{flame strike}, or \spell{wall of fire}, extinguishing the flames expends 2 charges from the rod.

If the device is used upon a fire creature (a melee touch attack), it deals 6d6 points of damage to the creature. This use requires 3 charges.

A \emph{rod of flame extinguishing} has 10 charges when found. Spent charges are renewed every day, so that a wielder can expend up to 10 charges in any 24-hour period.

Strong transmutation; CL 12th; \feat{Craft Rod}, \spell{pyrotechnics}; Price 15,000 cp.

\textbf{Guardianship:} Made from the gnarled root of an ancient \emph{tree of life}, this 1-meter long rod appears to pulse with life energy and is used to prevent defiling from further destroying the ecology of Athas. The wood that composes the rod must be taken from a living \emph{tree of life} as its ability to combat defiling is tied to the tree, and only functions while the tree still lives. As such, if the \emph{tree of life} from which this rod is crafted is ever destroyed, the rod loses all its powers.

Simply holding the rod while within the defiling radius of a wizard dampens the gathering of plant life energy, effectively making the terrain type for the wizard one step worse. As such, wizards cannot defile on desolate terrain. Furthermore, the rod prevents affected wizards from using Raze feats or extending the casting time of their spells for the purpose of increasing their effective caster level.

Three times per day, as an immediate action, the rod holder---when within the defiling radius of a casting wizard---can reduce the defiling radius to a distance equal to that between his location and that of the defiler, effectively limiting the defiler from using his highest level spells by reducing the maximum radius from which he can summon plant life energy. For example, a character holding a \emph{rod of guardianship} and standing 9 meters from a wizard could limit him to only casting 6th-level spells, as they defile a circular area 9 meters in radius; a 9th-level spell, with its 13.5 meters radius, would ``pass'' the rod wielder, and thus could be blocked. When this ability is used, a wizard attempting to cast a spell with a defiling radius that extends past the rod wielder simply fails in his action, losing the spell.

Finally, once per day as an immediate action, the rod holder---when within the defiling radius of a casting wizard---can completely negate the energy gathering process, disrupting the casting and causing the wizard to lose the spell.

Strong abjuration; CL 15th; \feat{Craft Rod}, \spell{conversion}; Price 75,000 cp.

\textbf{Immovable Rod:} This rod is a flat iron bar with a small button on one end. When the button is pushed (a move action), the rod does not move from where it is, even if staying in place defies gravity. Thus, the owner can lift or place the rod wherever he wishes, push the button, and let go. Several \emph{immovable rods} can even make a ladder when used together (although only two are needed). An \emph{immovable rod} can support up to 4,000 kilograms before falling to the ground. If a creature pushes against an \emph{immovable rod}, it must make a DC 30 Strength check to move the rod up to 3 meters in a single round.

Moderate transmutation; CL 10th; \feat{Craft Rod}, \spell{levitate}; Price 5,000 cp.

% \textbf{Lordly Might:} This rod has functions that are spell-like, and it can also be used as a magic weapon of various sorts. It also has several more mundane uses. The \emph{rod of lordly might} is metal, thicker than other rods, with a flanged ball at one end and six studlike buttons along its length. (Pushing any of the rod's buttons is equivalent to drawing a weapon.) It weighs 5 kilograms.

% The following spell-like functions of the rod can each be used once per day.

% \begin{itemize*}
% \item \spell{Hold Person} upon touch, if the wielder so commands (Will DC 14 negates). The wielder must choose to use this power and then succeed on a melee touch attack to activate the power. If the attack fails, the effect is lost.
% \item \spell{Fear} upon all enemies viewing it, if the wielder so desires (3-meter maximum range, Will DC 16 partial). Invoking this power is a standard action.
% \item Deal 2d4 hit points of damage to an opponent on a successful touch attack (Will DC 17 half) and cure the wielder of a like amount of damage. The wielder must choose to use this power before attacking, as with \spell{hold person}.
% \end{itemize*}

% The following weapon functions of the rod have no limit on the number of times they can be employed.

% \begin{itemize*}
% \item In its normal form, the rod can be used as a \emph{+2 light mace}.
% \item When button 1 is pushed, the rod becomes a \emph{+1 flaming longsword}. A blade springs from the ball, with the ball itself becoming the sword's hilt. The weapon lengthens to an overall length of 1.2 meter.
% \item When button 2 is pushed, the rod becomes a \emph{+4 battleaxe}. A wide blade springs forth at the ball, and the whole lengthens to 1.2 meter.
% \item When button 3 is pushed, the rod becomes a \emph{+3 shortspear} or \emph{+3 longspear}. The spear blade springs forth, and the handle can be lengthened up to 3.6 meters (wielder's choice), for an overall length of from 1.8 meter to 4.5 meters. At its 4.5-meter length, the rod is suitable for use as a lance.
% \end{itemize*}

% The following other functions of the rod also have no limit on the number of times they can be employed.

% \begin{itemize*}
% \item Climbing pole/ladder. When button 4 is pushed, a spike that can anchor in granite is extruded from the ball, while the other end sprouts three sharp hooks. The rod lengthens to anywhere between 1.5 and 15 meters in a single round, stopping when button 4 is pushed again. Horizontal bars 8 centimeters long fold out from the sides, 30 centimeters apart, in staggered progression. The rod is firmly held by the spike and hooks and can bear up to 2,000 kilograms. The wielder can retract the pole by pushing button 5.
% \item The ladder function can be used to force open doors. The wielder plants the rod's base 9 meters or less from the portal to be forced and in line with it, then pushes button 4. The force exerted has a Strength modifier of +12.
% \item When button 6 is pushed, the rod indicates magnetic north and gives the wielder a knowledge of his approximate depth beneath the surface or height above it.
% \end{itemize*}

% Strong enchantment, evocation, necromancy, and transmutation; CL 19th; \feat{Craft Rod}, \feat{Craft Magic Arms and Armor}, \spell{inflict light wounds}, \spell{bull's strength}, \spell{flame blade}, \spell{hold person}, \spell{fear}; Price 70,000 cp.

\textbf{Metal and Mineral Detection:} This rod pulses in the wielder's hand and points to the largest mass of metal within 9 meters. However, the wielder can concentrate on a specific metal or mineral. If the specific mineral is within 9 meters, the rod points to any places it is located, and the rod wielder knows the approximate quantity as well. If more than one deposit of the specified metal or mineral is within range, the rod points to the largest cache first. Each operation requires a full-round action.

Moderate divination; CL 9th; \feat{Craft Rod}, \spell{locate object}; Price 10,500 cp.

\textbf{Metamagic Rods:} \emph{Metamagic rods} hold the essence of a metamagic feat but do not change the spell slot of the altered spell. All the rods described here are use-activated (but casting spells in a threatened area still draws an attack of opportunity). A caster may only use one \emph{metamagic rod} on any given spell, but it is permissible to combine a rod with metamagic feats possessed by the rod's wielder. In this case, only the feats possessed by the wielder adjust the spell slot of the spell being cast.

Possession of a \emph{metamagic rod} does not confer the associated feat on the owner, only the ability to use the given feat a specified number of times per day. A templar still must take a full-round action when using a metamagic rod, just as if using a metamagic feat he possesses.

\textit{Lesser and Greater Metamagic Rods:} Normal metamagic rods can be used with spells of 6th level or lower. Lesser rods can be used with spells of 3rd level or lower, while greater rods can be used with spells of 9th level or lower.

% \begin{itemize*}
\textbf{Metamagic, Empower:} The wielder can cast up to three spells per day that are empowered as though using the \feat{Empower Spell} feat.

Strong (no school); CL 17th; \feat{Craft Rod}, \feat{Empower Spell}; Price 9,000 cp (lesser), 32,500 cp (normal), 73,000 cp (greater).

\textbf{Metamagic, Enlarge:} The wielder can cast up to three spells per day that are enlarged as though using the \feat{Enlarge Spell} feat.

Strong (no school); CL 17th; \feat{Craft Rod}, \feat{Enlarge Spell}; Price 3,000 cp (lesser), 11,000 cp (normal), 24,500 cp (greater).

\textbf{Metamagic, Extend:} The wielder can cast up to three spells per day that are extended as though using the \feat{Extend Spell} feat.

Strong (no school); CL 17th; \feat{Craft Rod}, \feat{Extend Spell}; Price 3,000 cp (lesser), 11,000 cp (normal), 24,500 cp (greater).

\textbf{Metamagic, Maximize:} The wielder can cast up to three spells per day that are maximized as though using the \feat{Maximize Spell} feat.

Strong (no school); CL 17th; \feat{Craft Rod}, \feat{Maximize Spell} feat; Price 14,000 cp (lesser), 54,000 cp (normal), 121,500 cp (greater).

\textbf{Metamagic, Quicken:} The wielder can cast up to three spells per day that are quickened as though using the \feat{Quicken Spell} feat.

Strong (no school); CL 17th; \feat{Craft Rod}, \feat{Quicken Spell}; Price 35,000 cp (lesser), 75,500 cp (normal), 170,000 cp (greater).

\textbf{Metamagic, Silent:} The wielder can cast up to three spells per day without verbal components as though using the \feat{Silent Spell} feat.

Strong (no school); CL 17th; \feat{Craft Rod}, \feat{Silent Spell}; Price 3,000 cp (lesser), 11,000 cp (normal), 24,500 cp (greater).
% \end{itemize*}

\textbf{Negation:} This device negates the spell or spell-like function or functions of magic items. The wielder points the rod at the magic item, and a pale gray beam shoots forth to touch the target device, attacking as a ray (a ranged touch attack). The ray functions as a \spell{greater dispel magic} spell, except it only affects magic items. To negate instantaneous effects from an item, the rod wielder needs to have used a ready action. The dispel check uses the rod's caster level (15th). The target item gets no saving throw, although the rod can't negate artifacts (even minor artifacts). The rod can function three times per day.

Strong varied; CL 15th; \feat{Craft Rod}, \spell{dispel magic}, and \spell{limited wish} or \spell{miracle}; Price 37,000 cp.

\textbf{Python:} This rod is longer than normal rods. It is about 1.2 meter long and weighs 5 kilograms. It strikes as a \emph{+1/+1 quarterstaff}. If the user throws the rod to the ground (a standard action), it grows to become a giant constrictor snake by the end of the round. The python obeys all commands of the owner. (In animal form, it retains the +1 enhancement bonus on attacks and damage possessed by the rod form.) The serpent returns to rod form (a full-round action) whenever the wielder desires, or whenever it moves farther than 30 meters from the owner. If the snake form is slain, it returns to rod form and cannot be activated again for three days. A \emph{python rod} only functions if the possessor is good.

Moderate transmutation; CL 10th; \feat{Craft Rod}, \feat{Craft Magic Arms and Armor}, \spell{baleful polymorph}, creator must be good; Price 13,000 cp.

\textbf{Security:} This item creates a nondimensional space, a pocket paradise. There the rod's possessor and as many as 199 other creatures can stay in complete safety for a period of time, up to 200 days divided by the number of creatures affected. All fractions are rounded down.

In this pocket paradise, creatures don't age, and natural healing take place at twice the normal rate. Fresh water and food (fruits and vegetables only) are in abundance. The climate is comfortable for all creatures involved.

Activating the rod (a standard action) causes the wielder and all creatures touching the rod to be transported instantaneously to the paradise. Members of large groups can hold hands or otherwise maintain physical contact, allowing all connected creatures in a circle or a chain to be affected by the rod. Unwilling creatures get a DC 17 Will save to negate the effect. If such a creature succeeds on its save, other creatures beyond that point in a chain can still be affected by the rod.

When the rod's effect expires or is dispelled, all the affected creatures instantly reappear in the location they occupied when the rod was activated. If something else occupies the space that a traveler would be returning to, then his body is displaced a sufficient distance to provide the space required for reentry. The rod's possessor can dismiss the effect whenever he wishes before the maximum time period expires, but the rod can only be activated once per week.

Strong conjuuration; CL 20th; \feat{Craft Rod}, \spell{gate}; Price 61,000 cp.

\textbf{Splendor:} The possessor of this rod gains a +4 enhancement bonus to her Charisma score for as long as she holds or carries the item. Once per day, the rod creates and garbs her in clothing of the finest fabrics, plus adornments of furs and jewels.

Apparel created by the magic of the rod remains in existence for 12 hours. However, if the possessor attempts to sell or give away any part of it, to use it for a spell component, or the like, all the apparel immediately disappears. The same applies if any of it is forcibly taken from her.

The value of noble garb created by the rod ranges from 7,000 to 10,000 cp (1d4+6 $\times$ 1,000 cp)---1,000 cp for the fabric alone, 5,000 cp for the furs, and the rest for the jewel trim (maximum of twenty gems, maximum value 200 cp each).

In addition, the rod has a second special power, usable once per week. Upon command, it creates a palatial tent---a huge pavilion of silk 18 meters across. Inside the tent are temporary furnishings and food suitable to the splendor of the pavilion and sufficient to entertain as many as one hundred persons. The tent and its trappings last for one day. At the end of that time, the tent and all objects associated with it (including any items that were taken out of the tent) disappear.

Strong conjuration and transmutation; CL 12th; \feat{Craft Rod}, \spell{eagle's splendor}, \spell{fabricate}, \spell{major creation}; Price 25,000 cp.

\textbf{Thunder and Lightning:} Constructed of iron set with silver rivets, this rod has the properties of a \emph{+2 light mace}. Its other magical powers are as follows.
\begin{itemize*}
\item \textit{Thunder:} Once per day, the rod can strike as a \emph{+3 light mace}, and the opponent struck is stunned from the noise of the rod's impact (Fortitude DC 16 negates). Activating this power counts as a free action, and it works if the wielder strikes an opponent within 1 round.
\item \textit{Lightning:} Once per day, when the wielder desires, a short spark of electricity can leap forth when the rod strikes an opponent to deal the normal damage for a \emph{+2 light mace} (1d6+2) and an extra 2d6 points of electricity damage. Even when the rod might not score a normal hit in combat, if the roll was good enough to count as a successful melee touch attack hit, then the 2d6 points of electricity damage still applies. The wielder activates this power as a free action, and it works if he strikes an opponent within 1 round.
\item \textit{Thunderclap:} Once per day as a standard action, the wielder can cause the rod to give out a deafening noise, just as a \spell{shout} spell (Fortitude DC 16 partial, 2d6 points of sonic damage, target deafened for 2d6 rounds).
\item \textit{Lightning Stroke:} Once per day as a standard action, the wielder can cause the rod to shoot out a 1.5-meter-wide \spell{lightning bolt} (9d6 points of electricity damage, Reflex DC 16 half) to a range of 60 meters.
\item \textit{Thunder and Lightning:} Once per week as a standard action, the wielder of the rod can combine the thunderclap described above with a \spell{lightning bolt}, as in the lightning stroke. The thunderclap affects all within 3 meters of the bolt. The lightning stroke deals 9d6 points of electricity damage (count rolls of 1 or 2 as rolls of 3, for a range of 27 to 54 points), and the thunderclap deals 2d6 points of sonic damage. A single DC 16 Reflex save applies for both effects.
\end{itemize*}

Moderate evocation; CL 9th; \feat{Craft Rod}, \feat{Craft Magic Arms and Armor}, \spell{lightning bolt}, \spell{shout}; Price 33,000 cp.

\textbf{Viper:} This rod strikes as a \emph{+2 heavy mace}. Once per day, upon command, the head of the rod becomes that of an actual serpent for 10 minutes. During this period, any successful strike with the rod deals its usual damage and also poisons the creature hit. The poison deals 1d10 points of Constitution damage immediately (Fortitude DC 14 negates) and another 1d10 points of Constitution damage 1 minute later (Fortitude DC 14 negates). The rod only functions if its possessor is evil.

Moderate necromancy; CL 10th; \feat{Craft Rod}, \feat{Craft Magic Arms and Armor}, \spell{poison}, creator must be evil; Price 19,000 cp.

\textbf{Withering:} A \emph{rod of withering} acts as a \emph{+1 light mace} that deals no hit point damage. Instead, the wielder deals 1d4 points of Strength damage and 1d4 points of Constitution damage to any creature she touches with the rod (by making a melee touch attack). If she scores a critical hit, the damage from that hit is permanent ability drain. In either case, the defender negates the effect with a DC 17 Fortitude save.

Strong necromancy; CL 13th; \feat{Craft Rod}, \feat{Craft Magic Arms and Armor}, contagion; Price 25,000 cp.

\textbf{Wonder:} A \emph{rod of wonder} is a strange and unpredictable device that randomly generates any number of weird effects each time it is used. (Activating the rod is a standard action.) Typical powers of the rod include the following.

\Table{}{cX}{
\tableheader d\% & \tableheader Wondrous Effect\\
01--05  & Slow creature pointed at for 10 rounds (Will DC 15 negates). \\
06--10  & \spell{Faerie Fire} surrounds the target. \\
11--15  & Deludes wielder for 1 round into believing the rod functions as indicated by a second die roll (no save). \\
16--20  & \spell{Gust of Wind}, but at windstorm force (Fortitude DC 14 negates). \\
21--25  & Wielder learns target's surface thoughts (as with \spell{detect thoughts}) for 1d4 rounds (no save). \\
26--30  & \spell{Stinking Cloud} at 9-m range (Fortitude DC 15 negates). \\
31--33  & Heavy rain falls for 1 round in 18-m radius centered on rod wielder. \\
34--36  & Summon an animal---a rhino (01--25 on d\%), elephant (26--50), or mouse (51--100). \\
37--46  & \spell{Lightning Bolt} (21 m long, 1.5 m wide), 6d6 damage (Reflex DC 15 half). \\
47--49  & Stream of 600 large butterflies pours forth and flutters around for 2 rounds, blinding everyone (including wielder) within 7.5 m (Reflex DC 14 negates). \\
50--53  & \spell{Enlarge Person} if within 18 m of rod (Fortitude DC 13 negates). \\
54--58  & \spell{Darkness}, 9-m-diameter hemisphere, centered 30 ft. away from rod. \\
59--62  & Grass grows in 160-sq.-ft. area before the rod, or grass existing there grows to ten times normal size. \\
63--65  & Turn ethereal any nonliving object of up to 500 kg mass and up to 0.85 m$^3$ in size. \\
66--69  & Reduce wielder to 1/12 height (no save). \\
70--79  & \spell{Fireball} at target or 30 m straight ahead, 6d6 damage (Reflex DC 15 half). \\
80--84  & \spell{Invisibility} covers rod wielder. \\
85--87  & Leaves grow from target if within 18 m of rod. These last 24 hours. \\
88--90  & 10-40 gems, value 1 cp each, shoot forth in a 9-m-long stream. Each gem deals 1 point of damage to any creature in its path: Roll 5d4 for the number of hits and divide them among the available targets. \\
91--95  & Shimmering colors dance and play over a 40-ft.-by-9-m area in front of rod. Creatures therein are blinded for 1d6 rounds (Fortitude DC 15 negates). \\
96--97  & Wielder (50\% chance) or target (50\% chance) turns permanently blue, green, or purple (no save). \\
98--100 & \spell{Flesh to Stone} (or \spell{stone to flesh} if target is stone already) if target is within 18 m (Fortitude DC 18 negates). \\
}

Moderate enchantment; CL 10th; \feat{Craft Rod}, \spell{confusion}, creator must be chaotic; Price 12,000 cp.
