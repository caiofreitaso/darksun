\section{Handling Magic Items}

\subsection{Identifying Items}
\textbf{Trial and Error:} Close study of an item might provide some information. A DC 15 (or maybe 20) \skill{Search} check should reveal a clue for how the item works, such as a tiny command word etched inside a ring or feather motifs that hint the ability to fly.

You might also permit a character to attempt a DC 30 \skill{Spellcraft} check or \skill{Knowledge} (arcana) check to determine if she can attune herself with the item's power or if she remembers reading of it once in her studies.

\textbf{Spells:} The easiest way to discern whether an object is magical is to use \spell{detect magic}. A character can focus on an item to determine the school of the spells embedded within it and the strength of the aura it gives off. The \spell{identify} and \spell{analyze dweomer} spells provide much more information.

\subsubsection{Magic Items and \emph{Detect Magic}}
When \spell{detect magic} identifies a magic item's school of magic, this information refers to the school of the spell placed within the potion, scroll, or wand, or the prerequisite given for the item. The description of each item provides its aura strength and the school it belongs to.

If more than one spell is given as a prerequisite, use the highest-level spell. If no spells are included in the prerequisites, use the following default guidelines.

\Table{}{lX}{
\tableheader Item Nature & \tableheader School \\
Armor and protection items & Abjuration \\
Weapons or offensive items & Evocation \\
Bonus to ability score, on skill check, etc. & Transmutation \\
}

\subsubsection{Psionic Items and \emph{Detect Psionics}}
When \psionic{detect psionics} identifies a magic item's school of magic, this information refers to the school of the spell placed within the potion, scroll, or wand, or the prerequisite given for the item. The description of each item provides its aura strength and the school it belongs to.

If more than one spell is given as a prerequisite, use the highest-level spell. If no spells are included in the prerequisites, use the following default guidelines.

\Table{}{lX}{
\tableheader Item Nature & \tableheader School \\
Armor and protection items & Abjuration \\
Weapons or offensive items & Evocation \\
Bonus to ability score, on skill check, etc. & Transmutation \\
}

\subsection{Using Items}
To use a item of power, it must be activated, although sometimes activation simply means putting a ring on your finger. Some items, once donned, function constantly. In most cases, using an item requires a standard action that does not provoke attacks of opportunity. By contrast, spell completion items are treated like spells in combat and do provoke attacks of opportunity.

Activating a item of power is a standard action unless the item description indicates otherwise. However, the casting time of a spell is the time required to activate the same power in an item, regardless of the type of item of power, unless the item description specifically states otherwise.

The five ways to activate items of power are described below.

\textbf{Spell Completion:} This is the activation method for scrolls. A scroll is a spell that is mostly finished. The preparation is done for the caster, so no preparation time is needed beforehand as with normal spellcasting. All that's left to do is perform the finishing parts of the spellcasting (the final gestures, words, and so on). To use a spell completion item safely, a character must be of high enough level in the right class to cast the spell already. If he can't already cast the spell, there's a chance he'll make a mistake. Activating a spell completion item is a standard action and provokes attacks of opportunity exactly as casting a spell does.

\textbf{Spell Trigger:} Spell trigger activation is similar to spell completion, but it's even simpler. No gestures or spell finishing is needed, just a special knowledge of spellcasting that an appropriate character would know, and a single word that must be spoken. Anyone with a spell on his or her spell list knows how to use a spell trigger item that stores that spell. (This is the case even for a character who can't actually cast spells, such as a 3rd-level ranger.) The user must still determine what spell is stored in the item before she can activate it. Activating a spell trigger item is a standard action and does not provoke attacks of opportunity.

\textbf{Command Word:} If no activation method is suggested either in the magic item description or by the nature of the item, assume that a command word is needed to activate it. Command word activation means that a character speaks the word and the item activates. No other special knowledge is needed.

A command word can be a real word, but when this is the case, the holder of the item runs the risk of activating the item accidentally by speaking the word in normal conversation. More often, the command word is some seemingly nonsensical word, or a word or phrase from an ancient language no longer in common use. Activating a command word magic item is a standard action and does not provoke attacks of opportunity. 

Sometimes the command word to activate an item is written right on the item. Occasionally, it might be hidden within a pattern or design engraved on, carved into, or built into the item, or the item might bear a clue to the command word.

The \skill{Knowledge} (arcana) and \skill{Knowledge} (history) skills might be useful in helping to identify command words or deciphering clues regarding them. A successful check against DC 30 is needed to come up with the word itself. If that check is failed, succeeding on a second check (DC 25) might provide some insight into a clue.

The spells \spell{identify} and \spell{analyze dweomer} both reveal command words.

\textbf{Command Thought:} If no activation method is suggested either in the psionic item description or by the nature of the item, assume that a command thought is needed to activate it. Command thought activation means that a character mentally projects a thought, and the item activates. No other special knowledge is needed. Activating a command thought psionic item is a standard action that does not provoke attacks of opportunity.

Sometimes the command thought to activate an item is mentally imprinted within it and is whispered into the mind of a creature who picks it up. Other items are silent, but a \skill{Knowledge} (psionics) or \skill{Knowledge} (history) check might be useful in helping to identify command thoughts. A successful DC 30 check is needed to come up with the command thought in this case. The power \psionic{psionic identify} reveals command thoughts.

Powers stored in command thought items are usually not augmented, because the manifester level of such an item is assumed to be the minimum possible to manifest the stored power.

\textbf{Use-Activated:} This type of item simply has to be used in order to activate it. A character has to drink a potion, swing a sword, interpose a shield to deflect a blow in combat, look through a lens, sprinkle dust, wear a ring, or don a hat. Use activation is generally straightforward and self-explanatory.

Many use-activated items are objects that a character wears. Continually functioning items are practically always items that one wears. A few must simply be in the character's possession (on his person). However, some items made for wearing must still be activated. Although this activation sometimes requires a command word, usually it means mentally willing the activation to happen. The description of an item states whether a command word is needed in such a case.

Unless stated otherwise, activating a use-activated item of power is either a standard action or not an action at all and does not provoke attacks of opportunity, unless the use involves performing an action that provokes an attack of opportunity in itself. If the use of the item takes time before a magical effect occurs, then use activation is a standard action. If the item's activation is subsumed in its use and takes no extra time use activation is not an action at all.

Use activation doesn't mean that if you use an item, you automatically know what it can do. You must know (or at least guess) what the item can do and then use the item in order to activate it, unless the benefit of the item comes automatically, such from drinking a potion or swinging a sword.
\subsection{Size And Items of Power}
When an article of empowered clothing or jewelry is discovered, most of the time size shouldn't be an issue. Many garments of power are made to be easily adjustable, or they adjust themselves magically to the wearer. Size should not keep characters of various kinds from using items of power.

There may be rare exceptions, especially with racial specific items.

\textbf{Armor and Weapon Sizes:} Armor and weapons that are found at random have a 30\% chance of being Small (01--30), a 60\% chance of being Medium (31--90), and a 10\% chance of being any other size (91--100). 
\subsection{Items of Power On The Body}
Many items of power need to be donned by a character who wants to employ them or benefit from their abilities. It's possible for a creature with a humanoid-shaped body to wear as many as twelve items of power at the same time. However, each of those items must be worn on (or over) a particular part of the body.

A humanoid-shaped body can be decked out in magic gear consisting of one item from each of the following groups, keyed to which place on the body the item is worn.

\begin{itemize*}
\item One headband, hat, helmet, or phylactery on the head
\item One pair of eye lenses or goggles on or over the eyes
\item One amulet, brooch, medallion, necklace, periapt, or scarab around the neck
\item One vest, vestment, or shirt on the torso
\item One robe or suit of armor on the body (over a vest, vestment, or shirt)
\item One belt around the waist (over a robe or suit of armor)
\item One cloak, cape, or mantle around the shoulders (over a robe or suit of armor)
\item One pair of bracers or bracelets on the arms or wrists
\item One glove, pair of gloves, or pair of gauntlets on the hands
\item One ring on each hand (or two rings on one hand)
\item One pair of boots or shoes on the feet
\end{itemize*}

Of course, a character may carry or possess as many items of the same type as he wishes. However, additional items beyond those listed above have no effect.

Some items can be worn or carried without taking up space on a character's body. The description of an item indicates when an item has this property. 

\subsection{Saving Throws Against Magic Item Powers}
Magic items produce spells or spell-like effects. For a saving throw against a spell or spell-like effect from a magic item, the DC is 10 + the level of the spell or effect + the ability modifier of the minimum ability score needed to cast that level of spell. Another way to figure the same number is to multiply the power's level by 1\onehalf and add 10 to the result.

Staffs are an exception to the rule. Treat the saving throw as if the wielder cast the spell, including caster level and all modifiers to save DC.

Most item descriptions give saving throw DCs for various effects, particularly when the effect has no exact spell equivalent (making its level otherwise difficult to determine quickly).

\subsection{Damaging Items of Power}
An item of power doesn't need to make a saving throw unless it is unattended, it is specifically targeted by the effect, or its wielder rolls a natural 1 on his save. Items of power should always get a saving throw against spells that might deal damage to them---even against attacks from which a mundane item would normally get no chance to save. Items of power use the same saving throw bonus for all saves, no matter what the type (Fortitude, Reflex, or Will). An item of power's saving throw bonus equals 2 + \onehalf its caster/manifester level (round down). The only exceptions to this are intelligent items, which make Will saves based on their own Wisdom scores.

Items of power, unless otherwise noted, take damage as mundane items of the same sort. A damaged magic item continues to function, but if it is destroyed, all its power is lost.

\subsection{Repairing Items of Power}
Some items of power take damage over the course of an adventure. It costs no more to repair a item of power with the \skill{Craft} skill than it does to repair its nonmagical counterpart. The \spell{make whole} spell also repairs a damaged---but not completely broken---item of power.

% \subsection{Intelligent Items}
% Some magic items, particularly weapons, have an intelligence all their own. Only permanent magic items (as opposed to those with a single use or those with charges) can be intelligent. (This means that potions, scrolls, and wands, among other items, are never intelligent.)

% In general, less than 1\% of magic items have intelligence.

\subsection{Cursed Items}
Some items are cursed---incorrectly made, or corrupted by outside forces. Cursed items might be particularly dangerous to the user, or they might be normal items with a minor flaw, an inconvenient requirement, or an unpredictable nature. Randomly generated items are cursed 5\% of the time.

\subsection{Charges, Doses, And Multiple Uses}
Many items, particularly wands and staffs, are limited in power by the number of charges they hold. Normally, charged items have 50 charges at most. If such an item is found as a random part of a treasure, roll d\% and divide by 2 to determine the number of charges left (round down, minimum 1). If the item has a maximum number of charges other than 50, roll randomly to determine how many charges are left.

Prices listed are always for fully charged items. (When an item is created, it is fully charged.) For an item that's worthless when its charges run out (which is the case for almost all charged items), the value of the partially used item is proportional to the number of charges left. For an item that has usefulness in addition to its charges, only part of the item's value is based on the number of charges left.