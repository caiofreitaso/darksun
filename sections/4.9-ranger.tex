\section{Ranger}
\Quote{What you call monsters and beasts are simply other beings trying to survive in the wastelands. Some of them are just as desperate, lost, and confused as you are.}{Sudatu, elven scout}

The wastes of Athas are home to fierce and cunning creatures, from the bloodthirsty tembo to the malicious gaj. Because of that, Athasians have long learned how to adapt and survive even in the most inhospitable and savage environments.

One of the most cunning and powerful creatures of the wastes is the ranger, a skilled hunter and stalker. He knows his lands as if they were his home (as indeed they are); he knows his prey in deadly detail.

\subsection{Making a Ranger}

Rangers are capable in combat, although less so in open melee than the fighter, gladiator, or barbarian. His skills allow him to survive in the wilderness, to find his prey and to avoid detection. The ranger has the ability to gain special knowledge of certain types of creatures or lands. Knowledge of his enemies makes him more capable of finding and defeating those foes. Knowledge of terrain types or of specific favored lands makes it easier for him to live off the land, and makes it easier for him to take advantage of less knowledgeable foes. Rangers eventually learn to use the lesser spirits that inhabit Athas in order to produce spell‐like effects. These lesser spirits inhabit small features of the land – rocks, trees, cacti and the like.

These spirits are relatively powerless, and cannot manifest themselves. Their awareness is low, and their instincts are of the most primitive sort. The relationship between these lesser spirits and the creatures known as Spirits of the Land is unknown.

\textbf{Races:} As the race that carries the most fear and hatred of other races, and as the people with the richest land to protect, Halflings become rangers more commonly than any other race except for half‐elves. Halflings are at home in their terrain (typically Forest Ridge or the Jagged Cliffs) and the ranger class teaches them the grace to move without detection, often to deadly effect. Their practice of cannibalism to emphasize their superiority over other sentient beings puts the ranger’s tracking abilities to deadly use. Halfling rangers tend to take favored lands primarily, followed by favored enemy benefits. In the Forest Ridge, halfling rangers tend to pick pterrans and other neighboring races as favored enemies; rangers of the Jagged Cliffs tend to focus on bvanen, and kreen.

Elves frequently become rangers, serving as scouts and hunters for their tribes, but elves are not as naturally drawn to the wilderness as they are to magic. Half‐elves are the race most compellingly drawn to the ranger class, since their isolation and natural gift with animals gives them a head start above rangers of other races. Half‐Elven rangers sometimes seek to impress their Elven cousins with their desert skills, and when they are rejected, the wilderness often becomes the half‐elf’s only solace. A few half‐elves turn to bitter hatred of the parent races that rejected them, and become merciless slave–hunters.

Although ranger skills do not come to naturally humans, their famous adaptability wins out in the end, and many humans make fine rangers. A few muls take up the ranger class while surviving in the wilderness after escaping slavery. Dwarves who become rangers find that their focus ability combines powerfully with the abilities of favored enemy and favored lands, but such characters rarely become adventurers since they tend to master wilderness skills in order to guard Dwarven communities.

Pterran rangers are common since rangers get along so well with the druidic and psionic leaders of the pterran villages. Aarakocra are similarly drawn to the ranger class to protect their villages from predators and enemies. Rangers are not unusual among the most hated humanoid races of Athas, such as gith, belgoi, and braxat. Among the various and dwindling communities of the wastes rangers are the most common character class.

\textbf{Alignment:} Rangers can be of any alignment, although they tend not to be lawful, preferring nature to civilization, silence to casual conversation, and ambush to meeting a foe boldly on the battlefield. Good rangers often serve as protectors of a village or of a wild area. In this capacity, rangers try to exterminate or drive off evil creatures that threaten the rangers’ lands. Good rangers sometimes protect those who travel through the wilderness, serving sometimes as paid guides, but sometimes as unseen guardians. Neutral rangers tend to be wanderers and mercenaries, rarely tying themselves down to favored lands. The tracking and animal skills of rangers are well known in the World; virtually every trade caravan has at least one ranger scout or mekillot handler. Sometimes they stalk the land for vengeance, either for themselves or for an employer. Generally only evil rangers ply their skills in the slave trade. Other evil rangers seek to emulate nature’s most fearsome predators, and take pride and pleasure in the terror that strangers take in their names.

\subsection{Game Rule Information}

\textbf{Hit Die:} d8.

\subsection{Class Skills}

\textbf{Class Skills:} Climb (Str), Concentration (Con), Craft (Int), Handle Animal (Cha), Heal (Wis), Hide (Dex), Jump (Str), Knowledge (dungeoneering) (Int), Knowledge (geography) (Int), Knowledge (nature) (Int), Listen (Wis), Move Silently (Dex), Profession (Wis), Ride (Dex), Search (Int), Spot (Wis), Survival (Wis), Swim (Str), and Use Rope (Dex).

\textbf{Skill Points per Level:} 6 + Int modifier ($\times 4$ at 1st level).

\subsection{Class Features}

\textbf{Weapon and Armor Proficiency:} A ranger is proficient with all simple and martial weapons, and with light armor and shields (except tower shields).

\textbf{Favored Enemy (Ex):} At 1st level, a ranger may select a type of creature from among those given on Table: Ranger Favored Enemies. The ranger gains a +2 bonus on Bluff, Listen, Sense Motive, Spot, and Survival checks when using these skills against creatures of this type. Likewise, he gets a +2 bonus on weapon damage rolls against such creatures.

At 5th level and every five levels thereafter (10th, 15th, and 20th level), the ranger may select an additional favored enemy from those given on the table. In addition, at each such interval, the bonus against any one favored enemy (including the one just selected, if so desired) increases by 2.

If the ranger chooses humanoids or outsiders as a favored enemy, he must also choose an associated subtype, as indicated on the table. If a specific creature falls into more than one category of favored enemy, the ranger’s bonuses do not stack; he simply uses whichever bonus is higher.

\textbf{Track:} A ranger gains Track as a bonus feat.

\textbf{Wild Empathy (Ex):} A ranger can improve the attitude of an animal. This ability functions just like a Diplomacy check to improve the attitude of a person. The ranger rolls 1d20 and adds his ranger level and his Charisma modifier to determine the wild empathy check result. The typical domestic animal has a starting attitude of indifferent, while wild animals are usually unfriendly.

To use wild empathy, the ranger and the animal must be able to study each other, which means that they must be within 30 feet of one another under normal visibility conditions. Generally, influencing an animal in this way takes 1 minute, but, as with influencing people, it might take more or less time.

The ranger can also use this ability to influence a magical beast with an Intelligence score of 1 or 2, but he takes a -4 penalty on the check.

\textbf{Combat Style (Ex):} At 2nd level, a ranger must select one of two combat styles to pursue: archery or two-weapon combat. This choice affects the character’s class features but does not restrict his selection of feats or special abilities in any way.

If the ranger selects archery, he is treated as having the Rapid Shot feat, even if he does not have the normal prerequisites for that feat.

If the ranger selects two-weapon combat, he is treated as having the Two-Weapon Fighting feat, even if he does not have the normal prerequisites for that feat.

The benefits of the ranger’s chosen style apply only when he wears light or no armor. He loses all benefits of his combat style when wearing medium or heavy armor.

\textbf{Endurance:} A ranger gains Endurance as a bonus feat at 3rd level.

\textbf{Animal Companion (Ex):} At 4th level, a ranger gains an animal companion selected from the following list: badger, camel, dire rat, dog, riding dog, eagle, hawk, horse (light or heavy), owl, pony, snake (Small or Medium viper), or wolf. If the campaign takes place wholly or partly in an aquatic environment, the following creatures may be added to the ranger’s list of options: manta ray, porpoise, Medium shark, and squid. This animal is a loyal companion that accompanies the ranger on his adventures as appropriate for its kind.

This ability functions like the druid ability of the same name, except that the ranger’s effective druid level is one-half his ranger level. A ranger may select from the alternative lists of animal companions just as a druid can, though again his effective druid level is half his ranger level. Like a druid, a ranger cannot select an alternative animal if the choice would reduce his effective druid level below 1st.

\textbf{Spells:} Beginning at 4th level, a ranger gains the ability to cast a small number of divine spells, which are drawn from the ranger spell list. A ranger must choose and prepare his spells in advance (see below).

To prepare or cast a spell, a ranger must have a Wisdom score equal to at least 10 + the spell level. The Difficulty Class for a saving throw against a ranger’s spell is 10 + the spell level + the ranger’s Wisdom modifier.

Like other spellcasters, a ranger can cast only a certain number of spells of each spell level per day. His base daily spell allotment is given on Table: The Ranger. In addition, he receives bonus spells per day if he has a high Wisdom score. When Table: The Ranger indicates that the ranger gets 0 spells per day of a given spell level, he gains only the bonus spells he would be entitled to based on his Wisdom score for that spell level. The ranger does not have access to any domain spells or granted powers, as a cleric does.

A ranger prepares and casts spells the way a cleric does, though he cannot lose a prepared spell to cast a cure spell in its place. A ranger may prepare and cast any spell on the ranger spell list, provided that he can cast spells of that level, but he must choose which spells to prepare during his daily meditation.

Through 3rd level, a ranger has no caster level. At 4th level and higher, his caster level is one-half his ranger level.

\textbf{Improved Combat Style (Ex):} At 6th level, a ranger’s aptitude in his chosen combat style (archery or two-weapon combat) improves. If he selected archery at 2nd level, he is treated as having the Manyshot feat, even if he does not have the normal prerequisites for that feat.

If the ranger selected two-weapon combat at 2nd level, he is treated as having the Improved Two-Weapon Fighting feat, even if he does not have the normal prerequisites for that feat.

As before, the benefits of the ranger’s chosen style apply only when he wears light or no armor. He loses all benefits of his combat style when wearing medium or heavy armor.

\textbf{Woodland Stride (Ex):} Starting at 7th level, a ranger may move through any sort of undergrowth (such as natural thorns, briars, overgrown areas, and similar terrain) at his normal speed and without taking damage or suffering any other impairment.

However, thorns, briars, and overgrown areas that are enchanted or magically manipulated to impede motion still affect him.

\textbf{Swift Tracker (Ex):} Beginning at 8th level, a ranger can move at his normal speed while following tracks without taking the normal -5 penalty. He takes only a -10 penalty (instead of the normal -20) when moving at up to twice normal speed while tracking.

\textbf{Evasion (Ex):} At 9th level, a ranger can avoid even magical and unusual attacks with great agility. If he makes a successful Reflex saving throw against an attack that normally deals half damage on a successful save, he instead takes no damage. Evasion can be used only if the ranger is wearing light armor or no armor. A helpless ranger does not gain the benefit of evasion.

\textbf{Combat Style Mastery (Ex):} At 11th level, a ranger’s aptitude in his chosen combat style (archery or two-weapon combat) improves again. If he selected archery at 2nd level, he is treated as having the Improved Precise Shot feat, even if he does not have the normal prerequisites for that feat.

If the ranger selected two-weapon combat at 2nd level, he is treated as having the Greater Two-Weapon Fighting feat, even if he does not have the normal prerequisites for that feat.

As before, the benefits of the ranger’s chosen style apply only when he wears light or no armor. He loses all benefits of his combat style when wearing medium or heavy armor.

\textbf{Camouflage (Ex):} A ranger of 13th level or higher can use the Hide skill in any sort of natural terrain, even if the terrain doesn’t grant cover or concealment.

\textbf{Hide in Plain Sight (Ex):} While in any sort of natural terrain, a ranger of 17th level or higher can use the Hide skill even while being observed.