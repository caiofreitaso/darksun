\section{The Dragon Age}
\textbf{FY $-2,024$ to FY $-1$}

\Quote{What monstrosity is this? My scouts are turned to ash, and a mekillot is torn in two! Where are the Armies of the Champions to protect us from this foul beast?}
{The final words of Hyland Gothril, Merchant of Walis while on route to Tyr.}

\subsection{A Brief Overview}
The Dragon Age began during a time of Terror, when Borys of Ebe became the Dragon of Tyr, and scoured the planet of Athas far and wide. Anything that was in his path was consumed, and many tribes and cities fell to his destructive presence. Those who could find safety in the cities did so, as all of Athas lived in fear of the Dragon and his wrath. The damage done by defiling during the Cleansing Wars was significant, but it was the Dragon that completed the transformation of Athas from a land of green to a desert stained red. After one hundred years, the terror that gripped Athas ended as Borys recovered from his madness. With his sanity returned, he found Rajaat's prison was nearly broken, so he demanded that the others pay their due and give him what was needed to keep Rajaat locked away. He would visit the city-states once a year, and collect tribute in the form of slaves, wealth and live stock, using these to build fortifications for the prison, and using the life-force of those slaves as fuel for his mighty spells.

The Champions of Rajaat took unclaimed cities to settle in and save themselves. The cities would make for good power bases, and would provide a buffer from the Dragon, and allow them to find ways to improve themselves. First on the list of things to accomplish for the Champions was to distance themselves from their pasts. With the destruction of the Cleansing Wars creating a rift between the races, the Champions dubbed themselves sorcerer-kings and settled into cities. Once in power, they quickly made the cities their own personal domains. The discovery of their connection to the elemental planes allowed them to grant spells to their templars. Some set themselves up as gods, others rigged elections, while others attempted to rule as the emissary of a higher power. Regardless of the methods, all the sorcerer-kings worked to reshape history to avoid connecting themselves to the past. Reading and writing were forbidden in the cities. Arcane magic was outlawed, and only those approved by the city-state, royal court defilers who were instructed by the god-kings personally, could practice.

There was one source of information that the new god-kings needed to get rid of: druids. Because of the connection between druids and the Spirits of the Land, druids had access to memories from the previous ages. The Spirits could tell of the atrocities that were unleashed during the Cleansing Wars, and point the finger at who was responsible. Knowing that killing a Spirit of the Land was practically impossible, for it would simply change based on how its inhabited features changed; for example, many forest Spirits were now desert Spirits. The former Champions lead a brutal campaign against the druids that became known as the Eradication; thousands of druids were killed, and the remaining few settled into the Forest Ridge, beyond the Tablelands, and hid for centuries, passing on knowledge as they could.

In the defense of the cities against the Dragon and each other, the sorcerer kings worked to create the perfect soldiers. Two such creations occurred. The first were the muls, which were a natural breeding of dwarves and humans. The birthing process was difficult for the mother, and the children were sterile, and unable to reproduce, but the resulting creation was strong and had incredible endurance. The others were the half-giants. Originally developed by Abalach-Re of Raam, half-giants were a magical creation, and the race bred true. They were powerful soldiers, and strong workers, but they lacked the intelligence or will power to be leaders or generals. They followed orders well, and, lacking their own culture, sought to mimic those around them who were seen as socially important. This proved useful in the creation of armies, as it made them into effective units.

Besides working to appease the Dragon, sorcerer-kings fought amongst themselves. Early in the Dragon Age, Abalach-Re lead a strike against Dregoth, in order to kill him before the madness that gripped Borys caused Dregoth to become another enraged beast. Dregoth had advanced far in his transformation into a dragon, and the remaining sorcerer-kings felt they needed to protect Athas from yet another source of possible destruction. Nibenay and the Oba of Gulg have been locked in countless wars back and forth, with neither side winning a decisive victory or destroying the other. Urik has been victorious numerous times. Hamanu destroyed the city of Yaramuke, killing the sorcerer-queen Sielba.

In the conflicts between city-states, the god-kings sought power and how to advance their metamorphosis. In Kalidnay, Kalid-Ma attempted to skip several steps, moving to a much higher stage. The process, like that Borys went through, proved to be too much, and his mind broke as well. The city was left in ruins, and it took Hamanu, Borys and Kalak to track down the beast and kill it.

In the Northlands, Daskinor sought to defend himself against the Dragon's levy. Daskinor's attempt was successful, and the Dragon left. However, this broke Daskinor's already fragile mind and he retreated into his city, slowly turning it into the prison state it is today. The Dragon stopped coming north, and left both Eldaarich and Kurn alone. Oronis, seeing this as a blessing, and steeped with regret, sought to undo the damage that he had done.

During this time, psionics was formalized in the city-states into the current method of teaching and terminology by Tarandas the Gray. After her systemization, she opened schools across the Tablelands, until she mysteriously disappeared.

Eventually, after a few hundred years, things settled down in the Dragon Age. The sorcerer-kings ruled with fists of iron, and enforced strict rigid rules on their subjects. Life continued, with the battles between city-states being little more than diversions to keep the attention of the sorcerer-kings while they slowly sought methods of advancement that would not end in their death or the madness that they had seen in Borys and Kalid-Ma. Life continued, scratched out of dry dirt, in this miserable way, until the end of the Age.

The signal of the end of the Age was not a massive world changing event, like the transitions of the previous ages were, but a small rebellion that was for the first time, in many centuries successful. A small group of citizens in Tyr attacked Kalak during the gladiatorial games to commemorate the completion of his ziggurat. With his death, the first city-state free from the rule of Rajaat's former Champions saw its first dawn in thousands of years. The Dragon Age was ended, and the Age of Heroes had begun.

\subsection{Playing in the Dragon Age}
The Dragon Age is one of the two typical ages to play in for a Dark Sun campaign. There are three things that stand out for a Dragon Age campaign.

First is stability. The god-kings rule with absolute authority. Their will is law, and those who deviate from it are punished severely, and then killed. The stability this creates makes life in the walls of the city-states predictable if not a happy existence. Resistance to the power of the sorcerer-kings can provide many fun adventures.

Second is war. The former Champions fought constantly against each other, and players can get caught up in this and become agents of the sorcerer-kings in these conflicts.

Brutality is the last aspect of playing in the Dragon Age. Life is short and brutal. Characters will die. Often. When playing in the Dragon Age, it is important to remember this. Sometimes the best option is to run or surrender to become a slave, hoping to escape another day. The world of Athas has always been brutal, and the Dragon Age is no exception.

\subsection{Character Races}
Any of the races in this book are available during the Dragon Age. Half-giants did not appear until this era, so this is the first time that they would be available. Muls are much more common during this time period as well. Humans are the dominant race, with the vast majority of city inhabitance being human. Elf tribes and dwarven villages dot the landscape, as do slave tribes of mixed races. Pterrans exist in the Hinterlands, and aarakocra in the high mountains, although still very isolated from the Tyr Region. New races are appearing all the time, as tribes, merchants, and lost souls travel too close to the Pristine Tower and find themselves corrupted by its influence.

\subsection{Character Classes}
Templars are now a playable class during this period when the Champions of Rajaat became the sorcerer-kings. Gladiatorial matches became much more common, increasing the likelihood of playing this class. Mages of all types are rare; citizens of the city-states and the villages of the waste hate and fear arcane magic, and will kill its users on sight. Defilers in the employ of the city-states, and perseveres who have joined the Veiled Alliance are the few mages who can expect support.

Rangers and druids are important classes for tribes of the wastes, as they provide protection and can give aid to the tribe in the barren wasted landscape. Psionics is still an important practice, with the academies of the Will and the Way in every city-state. Elemental temples also exist in some of the city-states, villages, and tribes of Athas, offering support to their communities in return for devotion to their element.

All the classes in this book are playable during the Dragon Age.
