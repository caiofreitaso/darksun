\section{The Blood Age}
\textbf{FY $-3,529$ to FY $-2,024$}

\Quote{For thousands of years they have lorded their wealth over us, treating us like vermin, like dogs. Well, no longer. They killed my brother, they killed your kin. They have had it too good for too long. I learned their ways long ago, while I studied in Hogalay, and their arrogance knows no bounds. We will not rest until every last dwarf is dead. We will not rest until we have paid them back in full for their treachery. My brother's death will not go unavenged, and he will ride into the afterlife on an ocean of dwarven blood.}
{Egendo of Carsys to his troops before the Siege of Navargos}

\subsection{A Brief Overview}
The Cleansing Wars were perhaps the most dynamic and horrific time of all for Athas. Races that once thrived were wiped out, and others brought right to the brink of destruction. The Wars themselves did not start out of nowhere, no war does. The build up to the Cleansing Wars began during the Time of Magic, when Rajaat began teaching some of his students the secrets of defiling. Those who also could command the Way were brought into his inner circle of trusted disciples.

\subsubsection{Preserver Jihad}
After realizing it was humans who best suited his plans, Rajaat collected his loyal servants and his best students to him and spun his tale of what was becoming of Athas and what they needed to do about it. Those whom he had taught magic to were turning on them. The preservers had betrayed him, and had been using their magic to gain power over others, and further the cause of small interests, whether personal or even in the service of the larger nations or empire. Rajaat had taught magic to the races for a grander purpose, for the whole of Athas, to help everyone. Rajaat told his assembled followers how these traitor would misuse their power until all of Athas was under their control, not unlike how it was before the Time of Magic, when a few practitioners of the Way controlled the lives of so many. What was needed, was to set things right. With cheers and promises of violence, fueled by lies and half truths, Rajaat began the Preserver Jihad.

Rajaat initially gave his disciples a specific list of influential preservers for covert assassination: court advisers, wizards that had opened their own schools, and renowned mages who had taken nobles as apprentices. Rajaat's followers initially managed to remain in the shadows, sowing suspicion between preservers. Even when some of the more powerful preservers began to accuse Rajaat and to try to rally other preservers to their defense, many refused to accept that Rajaat was responsible. Any who had influence enough to raise the alarm were taken out early to avoid knowledge of the conflict getting out. However, opposition formed, and secret strikes escalated to open war. Soon, any city or town who sheltered preservers became targets of Rajaat's wrath.

What few realized, on either side of the conflict, was that this was to begin a much bigger conflict. The slaughter of so many preservers resulted in preserving magic being effectively lost to the planet. With preservers pushed underground and into hiding, their knowledge was not passed on, and as defilers learned their new way, they became the dominant arcane spell casters. The preserver methods used exist in isolated places, known to a few who would pass it on only to a chosen student.

Eventually, Rajaat called for an end to the fighting. Much destruction was done, and some defilers had moved on and renounced their ways, serving the very people they had once fought against.

\subsubsection{Cleansing Wars}
Rajaat brought his inner circle to him and told them whatever they needed to hear to convince them to agree to the changes that were to be made both to them and the world. Some were told of lost loved ones who could be avenged, others were promised power, still others did so simply out of loyalty to Rajaat. Whatever their reasons, not all knew what it was they were agreeing to.

Rajaat brought the Champions into the Pristine Tower in groups of four, accompanied by his strange halfling followers who never spoke. They emerged from the Steeple of Crystals changed, and hungry. Athas had changed as well. The yellow sun had grown and turned crimson. The magnitude of this new crusade sunk in as Rajaat explained that the power that coursed through their veins had required the sun to give up much of its life and light. At the base of the Pristine Tower were thousands of men, many of them veterans of the Preserver Jihad, gathered together for the first time. Rajaat divided the men among the Champions, and declared them the nucleus of armies they were to lead.

Rajaat, speaking into the minds of those present, told them of the Wars to come. He told the human armies that the world was ruined by the mixing of races. He talked of the greed of the dwarves and the gnomes, the savage brutality of the wemics, the arrogance of the aarakocra and the elves. He spoke to the hurt and the pain that the armies had seen, how orcs had slain their kin, and how pixies had killed their neighbors. There was only one thing to do, Rajaat told them: wipe them out. With this host and his fifteen Champions, Rajaat marched the armies to the south into Ulyan, deep to the city of Nagarvos. Once there, Rajaat demanded the surrender of the rogue preserver Pandruj. The leaders of the city, seeing the vast host assembled, try to negotiate. Rajaat took the cities emissaries in, and kept them in deliberations for months. With mighty magic, Rajaat had Gretch block the transmissions of the city's guardians and a psionic gem that were in contact with others in the region until they could be destroyed, and began the siege. Though the city leaders worked hard to keep the forces at bay, the city was in ruins within three days of the attack order.

As the black billows of Nagarvos' destructions faded to brown wisps from the ruins, Rajaat summoned his Champions to a final council; even Keltis was recalled from his campaign in the swamps of Sagramog around the city to join them. Rumors were spreading, but the fate of Nagarvos' was still a secret known only to the Champions and their armies: the destruction of the psionic gem of Nagarvos', the cordon of pickets that Gallard had ordered to hunt down any messengers from the doomed city, and various magical and psionic wards that Rajaat created, had prevented any reliable news from escaping. To take advantage of the confusion and strike before any of the neighboring kingdoms could prepare their defenses, Rajaat ordered his Champions to disperse on their cleansing missions immediately.

Wyan marched west, past Gretch's Grey Tower, bound for Small Home. Gallard marched with him as far as the gnomish city of Arludas, where his men began their work. Sacha and Daskinor struck due west, raiding the kobold and goblin warrens in the hills. Abalach-Re and Dregoth marched together to the Winding Way, and up it, battling the defenders of Fort Tru'ezarr and continuing onward to Celik. Myron Troll-Scorcher led his army north, striking the trollish kingdom of the Sagocracy of far northern Ulyan; the Dwarf-Butcher marched with him, and then continued west to Toganay. When the dwarven hold there was wrecked, he followed the others up the Winding Way and departed Ulyan for northern lands.

Tectuktitlay refused to wait for the others' armies to pass up the Winding Way; no wemics lived in Ulyan, and the Wemic Annihilator was eager to smite his foes. Tectuktitlay forced his men to climb the impassable cliffs of northern Ulyan, hacking a narrow way up the cliffs at the cost of thousands of lives. Tectuktitlay's Stair remains there to this day, though few now remember it. Keltis marched east, finishing the lizardmen of Sagramog and then taking the narrower but more direct route east up toward the Sunrise Sea. Albeorn followed him, striking at the elven settlements along the route to Arkhold.

The Champions went their separate ways, attacking those that they could. For many, it was easy to find humans who would fight for the cause. Attacking a dwarven outpost would cause dwarves to retaliate against the humans in a nearby village, adding fresh recruits who were hungry for vengeance. Working with northern herders to fight against wemics who threatened their flocks increased the numbers of Tectuktitlay's army. Once word of the attacks became common knowledge, racial tensions increased. Few knew who to trust.

Each Champion fought his war of genocide in their own way in order to attack the weakness of the race. Pixies were natural druids, connected with the forests in a symbiotic way. Male pixies were short lived, but gained second life by being ``reborn'' as trees. By defiling the forests, Pixie's Blight was able to kill off half the race and prevent any new pixies from being born.

Champions tended to focus their efforts on massive epic spells only occasionally. Besides the enormous cost in resources and time, these powerful spells could be countered by elemental priests or by defilers and preservers within the racial strongholds and fastnesses that held out against the Champions. The Champion's distrusted any wizard not under their control, so new mages were rarely taught, limiting their number.

Believing that humans would one day live in the cities that were under siege, the Champions were initially careful about the amount of defiling the permitted. Their cabal of defiler attendants needed permission for most spells. Spells that could shatter mountains or sink islands were used, but they were used rarely, while other methods of engagement were used.

An early instance that proved particularly destructive was a spell cast by Keltis to obliterate a lizardmen settlement in the Sunrise Sea. After months of preparations, the spell was complete and the water started to turn to silt, polluting the settlement and killing many. The spell was more powerful than anticipated. The silt continued to grow until it became the Silt Sea that exists today.

Green Age technology, like guardians, was used as often as possible. As many of the cities had these posted on their walls and in vast networks outside their cities and strongholds, many were destroyed in the fighting. The elves were the first to discover a magic that would shatter obsidian which proved useful against the Champions and their Green Age weapons. However, after Albeorn struck the forest outpost of Sylibar, the secret was taken from them and passed to the other Champions. Guardians did not last long under these conditions, meaning that direct conflict, sword to sword, spear to spear, and claw to fist became much more the norm.

The various races did not simply fad away, however. Some cities with mixed race citizens rallied together, and worked against the Champions. The ogres used dark magics to mutate their greatest warriors into nightmarish anti-siege monstrosities. As the ogre encampments fell to Kalak, he was able to capture some of these creations, and transfer into them the minds of some of his trusted captains. Turning these creatures on their owners proved useful, especially after warping them by the corrupting influence of defiling magic and the Pristine Tower. After the wars, the living siege engines were abandoned by their master, and set up domains of their own, using their powers to sow nightmares to attack wandering caravans to this day.

Elves took to a nomadic existence, abandoning their cities and outpost, and scattered to the far edges of the world. Dwarves dug their settlements in deep, resisting the attacks of Egendo. In the northern city of Hogalay, the dwarves did what no other race had: captured and eliminated a Champion. After this loss, Rajaat elevated one of Egendo's captains, Borys of Ebe, to replace him.

One of the most important forces in the defense of the races were the pyreen. Gaining the name peace-bringers in the process, the pyreen used various methods to try to save the races. Some were called Great Ones by their followers, fought side by side within the cities and took forms that made them stand out as leaders. Halflings with wings, orcs twice the size of their kin, and all other manner of fierce and protective images were used by the pyreen in their defense of the Rebirth races.

One such pyreen took his followers to find the source of Rajaat's magic. His personal disciples, along with a tribe of elves found their way to the Swamp where Rajaat first discovered arcane magic. Along the way, some of the elves mutated from the effects of the swamp, and became the reggelid that occupy the swamp today. Those who survived founded a citadel in the north end of the swamp,
where the magical effects of Rajaat's experiments made energy from the Gray seep onto Athas in places. There they worked to discover the secrets to undo Rajaat.

The resistance was as varied as the races, with air strikes by aarakocra upon supply lines, and gnomish traps and magical and psionic horrors unleashed upon human populations. The constant back and forth caused deep divisions between races, as armies of the Champions would attack races under powerful illusions making them appear to be non-human soldiers. This eventually led most races to distrust not only humans but other races as well. Trolls, forced to scavenge and pillage to survive, would loot dwarven settlements as easily as they would human ones. Animosity became great as the wars moved on, and the world became dominated by humans. Many races that were common and a vital part of the life of the Green Age were wiped out during the Cleansing Wars.

\subsubsection{Aftermath}
During the conflicts, Rajaat had a group of researchers busy working to figure out the connection that elemental priests had with the planes so that this could be a source of power that was used in the Restoration. Working in the ruined heart of Nagarvos in a research facility named the Navel, these researchers found ways to summon beings from the elemental planes, but were unsuccessful in discovering a means of taping the energy of the planes directly. When some survivors of the massacres of the southlands amassed a large force of meorties and others and attacked the research facility, one of the researchers miscast a \spell{gate} spell and opened a portal to the Paraelemental Plane of Magma. This portal caused magma to spew out, stretching out for miles and miles, until it covered the land in a thick layer of obsidian. These planes eventually became known as the Deadlands, as everything living on the vast expanse of obsidian woke to find itself undead and in a strange and terrible new world. How the portal was closed is unknown, and those who dwell on the plane assume that all of Athas is now a barren wasteland.

Eventually, some of the Champions were successful and started to settle their armies in various cities, waiting for the final victory that was promised to them. They took control of the cities in different ways. Some abandoned their armies, and took up residence in a city as a citizen who then rose to power. Some moved their existing armies into a conquered city, turning the citizens into slaves and the soldiers into nobility. Still others marched their armies to the gates of the city, only to take a new identity and rally the city against the outside aggressors, and win the hearts of the city's inhabitance. A few Champions returned to the Pristine Tower, abandoning their armies to be in service to their master, the Warbringer. During this time, Myron the Troll Scorcher, came to the attention of Rajaat for some failing. One of his soldiers, Manu of Deche, was chosen as his replacement. Manu took the name of Hamanu, and was victorious in eliminating the trolls. As the Champions took to settling down in cities, they worked at starting new lives, and preparing their people for the paradise to come, where all the land was set for humanity.

\subsubsection{Rebellion}
At some point, Borys of Ebe came to the Champions and told them that he had discovered the truth behind Rajaat's plans. Though they all must have known that Rajaat was mad, they somehow discovered that he did not intend to leave humanity intact either. They discovered that Rajaat wanted to return the world to the Blue Age, and to reverse the Rebirth. Seeing that their master would betray them, they decided to save themselves.

The Rebellion happened quickly, with the Champions striking their master without warning. Battering him with fierce magic and the Way, as well as with powerful weapons, the Champions came and trapped their lord. However, during their attack, they discovered that their master would not be easy to vanquish. Two of the Champions stayed loyal to their master. Sacha and Wyan worked to help the First Sorcerer escape. A powerful attack by the halfling servants of Rajaat was aided by the two traitors. During this attempt they were discovered, and beheaded for their treachery, with the halflings thrown into the black and imprisoned with their master. While this infighting was occurring, the dwarves Jo'orsh and Sa'ram stole into the Steeple of Crystals and took the Dark Lens. By the time the Champions found it gone, wards were placed upon the Dark Lens that made it undetectable to them. The Champions began to research a way to keep their master imprisoned without the aid of the Lens.

Eventually, Gallard Bane of Gnomes told of a ritual that he had developed to keep Rajaat lock away. Casting this spell would allow the Champions to place Rajaat beyond the realm of Athas, underneath the plane of the Black in a place named the Hollow. There, he would be unable to attack them and seek retribution. However, without the Dark Lens, there would need some way to keep Rajaat in his prison. Borys, the self-declared leader of the rebellion, said that he would take the role of prison warden. Dregoth and Borys had talked and Dregoth told him of the spell he had been working on to transform himself into a being of legend, towards the level of power and godhood that Rajaat had promised. Dregoth offered this to Borys, as this would allow him to become powerful enough to cast the spell on his own. In order to accomplish this, however, they all needed to start the process. Guided by Dregoth, the Champions cast the spells of transformation upon themselves, taking the first step towards dragonhood. The powerful energies that were unleashed at the time attracted the attention of strange beings that connected the elemental planes to Athas, joining with the Champions and allowing them to grant others access to the elemental planes. The Champions became sorcerer-kings that day, though it would be many years before the secrets of this new reality were to become known.

The Champions finalized their pact. They would complete the spells, turning Borys into a full dragon, and they would sacrifice together to keep Rajaat locked away. The Cleansing Wars would be over, and the Champions would settle the land as they could. Some already had cities, while others would take them where they could. They joined together, and cast the spells to change Borys into the Dragon. However, not everything went as planned. The process was too much, even for Borys, and his mind broke. He became mad with fury and pain, and an incredible hunger. For one hundred years, he scoured the planet, sucking up life force and destroying everything that was in his path. The Champions fled, and fortified their cities quickly, in order to defend themselves against this new threat that they had made. The Cleansing Wars ended, not in the victory of paradise, but in a new hellish landscape. The Athas of today was shaped by this time. Metal became scarce, as eons of war depleted mines and resources, and existing material was destroyed by spell and by claw. The Dragon drained the land of life, making Athas the desert that it is, and turning the blood soak earth to ash and sand.

\subsection{Playing in the Blood Age}
Playing in the Blood Age presents some interesting challenges and rewards. It does require some thought however, as any campaign set in this time will have a natural progression towards epic play, given the intense nature of the Age.

The Preserver Jihad allows for players to try figuring out Rajaat's plans. Because they began in secret, discovering that this conflict started is an interesting arc that could be developed. Or, players could be warriors for Rajaat, leading strikes against rogue mages and bringing people to ``justice.''

The Clensing Wars allow for players to take on any role they want. They could be soldiers in a Champion's army, members of a race fighting for survival, or even play the Champions themselves. Just how epic you want to go depends on the DM.

To be certain, playing in this time is a brutal prospect. The Champions themselves refer to this time as a time of horrors, and they are the ones who perpetrated it. Ruthlessness and callous acceptance are necessary in this Age. Trust is non-existent. Playing in this age should highlight the terrible price that was forced on the cleansed races, as the Champions did all that they could to wipe out their foes.

The Rebellion could also be an interesting campaign, playing either as the Champions against Rajaat, or as member of the dwarven race intent on stealing the Dark Lens.

Campaigns set in this Age might not be epic in game level terms, but they will all be so in scope. It is by far the most dynamic time of Athas.

\subsection{Character Races}
All of the races of the Green Age are available for play in the Blood Age period. However, as time moves forward, certain races were cleansed from the planet, and are no longer available for play. Playing as a soldier in the armies of the Champions, or as survivors from different races will allow Characters to play throughout the time period. Pay attention to the intentions of the player, and make sure that they understand that playing an orc or a gnome does not bode well for that character's survival. Check with your DM before creating a character from one of those races.

\subsection{Character Classes}
At this point in the history of Athas, all classes are available to characters with the exception of templars. It was not until the very end of the Blood Age that Champions could grant spells, so they did not exist until that time. All other classes existed, with rangers and druids being common among the races seeking survival, and defilers being a sought after class by those in the armies of the Champions. Fighters and barbarians also existed in high numbers, though the latter tended to be grouped into shock troop units, to better take advantage of their ferocity.

Psionics was still common, and many people had access to training that was necessary to become psions. These would be valuable forces to have in the conflicts of the age, especially after the loss of so many guardians early in the war. Clerical magic was important as well. The healing that clerics could bring was seen as valuable to the forces on all sides of the Wars, and they were in high demand.
