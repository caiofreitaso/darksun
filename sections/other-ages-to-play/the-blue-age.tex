\section{The Blue Age}
\textbf{FY $-14,577$ to FY $-14,025$}

\Quote{The water near the edge of the walkway seemed to bubble for an eternity. Terror ran through the streets as giant monstrous creatures poured into the city .It was too late, the Benders were here. Those few of us that escaped now know the horror that has been unleashed upon us by the Benders' disregard for life. Our only hope is that the Nature Masters can fight off these vile mutants and destroy the Benders before it is too late. I will never forgive them for what they did to my home.}
{Journal of Bur-han Vut-luc, resident of Tyr'agi recovered by a Tyrian explorer after the Great Earthquake.}

\subsection{A Brief Overview}
The Blue Age was a time of wonder, when Athas was covered by vast oceans dotted with small island chains that were inhabited by the rhulisti. They ruled the land  and sea through the power of the age: Life-shaping.

Drawing on their ability to alter the basic building blocks of life itself, the rhulisti lived in harmony with their environment and were able to prosper and grow across the face of Athas. During this time, the few conflicts that broke out amongst different settlements were settled with wisdom and foresight by the lords of the age, the naturemasters. The nature masters were the elite of the rhulisti, and their knowledge allowed them to shape not only the life around them, but the political climate and society as well. Contests for various offices were fierce, and not every rhulisti settlement had the same style of governance or even used the same life-shaped items.

A small group of nature masters who sought to change creatures in questionable ways started experimenting and gaining  some influence. This group became known as the nature benders. Eventually, ethical differences erupted into a full scale war with both sides using their life-shaped creations against each other. Elemental clerics first started to appear at this time as did wild talents and minor users of psionics, and both were seen as abominations as they used their new powers with the nature benders in the First War.

The nature masters eventually were victorious, but their victory proved to be short lived. While seeking to expand their empire, the nature masters tried to double the output of the seas, and instead a Brown Tide swept the oceans, slowly covering Athas in a deadly plague that killed what all in its path. Eventually, the nature masters constructed the Pristine Tower, and used the energy of the sun to destroy the Brown Tide. Doing so caused the seas to recede and the sun to change from a blue to a yellow sun. The Rebirth followed, where those who survived the Brown Tide changed themselves into different races, and set about populating this new world with new peoples. The Blue Age ended, and the Green Age began.

\subsection{Playing in the Blue Age}
There are many different opportunities for campaigning in the Blue Age. One can take part in any one of the three major events of the Age, or find other interesting things to do based on where the DM would want to direct the campaign.

First on the list of interesting campaign hooks would be the discovery of the nature benders. This group began to twist their creations and use them against others in order to dominate the rhulisti. A DM could have the characters play both detective and strike force for the nature masters, by figuring out what was happening and trying to stop it before it went to a full scale war. Or, the PCs could be warriors during the First War and fight for either side, striking at the leadership of the other in an attempt to win for their side. Either way, this style of campaign can take place at any level of play, from beginning characters to epic movers and shakers.

Finally, the Brown Tide could be an interesting arc, especially for epic characters looking to fix things. Having the PCs create the Pristine Tower and enact the Rebirth could be an interesting and fun campaign for both the players and the DM. Also of interest in this time would be helping the Rebirth races discover each other and struggle to survive the beginnings of their culture. This style of campaign would rely more on political intrigue and other types of social setting interactions than on pure combat.

Regardless, campaigns set in this Age should highlight the power of the Age. Play up the wonder and basic utility of life-shaped items. Things like illness and early deaths were rare, and even regular halflings had no real need to worry about hard physical labor, as life-shaped items were there to serve the needs of the people. It was an age of comfort and plenty, even with the battles between the nature masters and the benders. In fact, the relative peace and prosperity of the age was one of the things that made the benders so horrible in the eyes of the nature masters. Encourage players to take levels in the life-shaper (LSH 29) and nature-master prestige classes (LSH 33) in order to better understand what was possible at the time. Look to the Life-Shaping Handbook for more info on life-shaped items and their applications and uses. Feel free to develop your own items and take things as far as you want. Those who describe the Blue Age talk about it with awe; it should be played as something totally alien to current day Athas. If your players are unaware of other Ages, then they may never know that the Blue Age and life shaping is Athas at all.

\subsection{Character Races}
There is only one race that really is acceptable for playing in the Blue Age: rhulisti. No other races existed during this age, and the thri-kreen's ancestors had yet to gain sentience. Regardless, the rhulisti were the masters of the world during this age, and should be the focus of any campaign set in this time.

One other interesting possibility would be to play intelligent life-shaped creations that existed during the First Wars. Given that these creations could be grown any number of ways, the only limit on this sort of PC would be your own creativity and campaign. If one were to play out the beginning of the Rebirth, any number of races could be allowed. These races shouldn't be limited to just the ones listed in Rajaat's list of Rebirth races. Any creature could be used as a sentient being at this time. They may have died off because of a lack of ability to continue breeding, being killed by many of the new creatures, or may even exist still on some distant part of Athas in the current age. Regardless, if a DM is playing the Rebirth, anything goes as the rhulisti would not have known what would and wouldn't work during the Green Age.

\subsection{Character Classes}
Character classes available exclude the following: gladiator, psion, templar and wizard.

Psionics was rare, and did not find common enough usage for people to train in this area. Arcane magic did not exist during the Blue Age, and violence for the pleasure of others was not a norm for this time. Clerics to the elements were rare, and druids were almost unheard of. The connections of Athas to the elemental planes were not well known at this time, and the focus of the Age was life shaping. Bards make the best life shapers, as their skill focus can go into crafting, as well as their natural abilities with political and social interaction. Fighters, scouts and thieves would have been important at the time, as they would have been the soldiers and spies of the era, as well as explorers and the rare adventurer. Druids, clerics, and rangers whose focus was water would have had the most influence among others, while fire would have been seen as dangerous. Paraelemental clerics and templars did not exist at all during this Age.
