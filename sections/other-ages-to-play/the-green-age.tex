\section{The Green Age}
\textbf{FY $-14,024$ to FY $-3,530$}

\Quote{We teleported into the heart of the city with our strike force just a few dozen in number to begin with. It appeared that our presence was undetected, though this would only last a few more moments. My team knew what to do, and I ordered our combat guardians to summon their physical harbingers. The seekers found the target three heartbeats after my orders were given, and our handful of artillery guardians had their default orders confirmed. Our team had tripled in size within five heartbeats and we stormed the keep, laying waste to all in side. I secured the head of the noblewoman Var'el Rictus as you asked and have included it as requested. Within ten heartbeats of our arrival, we were gone. I will be expecting payment as we agreed upon within the week.}
{a message by Ryl'in Zycart'es of the Astral Company imbedded into a crystal shard found outside of Gulg during a Red Moon Hunt.}

\subsection{A Brief Overview}
The Rebirth took a few centuries to stabilize into the Green Age. Races banded together, fought amongst each other, rose and fell, found their footing, and eventually established their own cultures and traditions during the first centuries of the Green Age.

\subsubsection{Psionics}
Psionics was the power of the Age. After the first few centuries, people developed and categorized psionics in a way that made it the dominant force of the Age. While life-shaping shaped life, psionics allowed the Will to be used in a Way that made thoughts real. Psionics has always been the most elegant power source on Athas, and it found its most impressive expression during the Green Age. As societies emerged, those who knew psionics became the nobles and rulers of the cities and nations that developed. They taught their powers to those who could afford them, and ensured that they could stay in power. Power plays between rivals became legendary tales among the common folk. Occasionally, a rare student would gain admittance into a psionic academy based solely on their determination and raw talent, but those in power would only let those few in after detailed and intense screenings.

Psionics allowed for an age of wonder to unfold. It powered massive building projects and transportation. Every city-state, nation-state and empire had different methods that were preferred. Some, like Giustenal, relied on massive moving platforms. Others, like Carsys, used ships that could sail in both the Sunrise Sea and the sky. Elven citadels were constructed in impossible positions, ogre fortresses floated above the sky, and gnomish factories produced psionic items that were the envy of every other race.

Psionics quickened the development of public works, from water to mass transit within nations to farms powered by projections of pure mind. Peace was not, however, the norm during this time. War was near constant, as the Reman people and the Tanysh people fought amongst themselves, as well as some of the more independent isolationist races, such as orcs and wemics. Psionics powered war as well as peace. Almost all the militaries used teleportation for both troops and for other nefarious uses. Psionic masters who specialized in warfare would teleport enormous boulders high above a city, and watch it crush what it landed on. Those living close to the Sunrise Sea would attempt to drown their opponents with vast quantities of water teleported into fortresses and contained with powerful walls of psionic force. Units would teleport behind enemy lines and increase their number with physical harbingers. Sometimes, battles would be solely for the minds of soldiers, as master telepaths of both sides wrestled to control the thoughts of those on the field. In the Green Age, thoughts became reality for those who practiced the Will and the Way.

\subsubsection{Guardians}
The work horses of the age were the guardians, former living beings striped of personality to ensure their loyalty and placed within obsidian orbs. While the life-shaped items of the Blue Age were able to produce what was needed, from food to protection to housing, they did so in a way that was grown. Guardians were able to turn the thoughts of their masters into reality. The floating platforms or flying ships of the various city-states were powered by guardians more often than not. These psionic slaves could be used to perform any number of tasks. Seekers sought information and kept a watch on those who were designated. Warriors were developed to fight alongside living breathing soldiers. Artillery batteries used psionic force and manipulated energy to blast away at troop formations and enemy fortifications. Workers would plow fields, build public works and private homes, operate gates, and levitation platforms. House servants would clean and cook, send a worker to collect supplies and food, and stay in contact with others. Guardians were the wonder of the age, with enough freedom and decision making to operate specific tasks, yet without the ability to resist the orders of their masters.

The creation of guardians was a closely kept secret by the masters of various academies. They kept it secret, for fear of it being used on themselves. Often results of covert or overt conflict between noble houses of a city resulted in the victors turning their fallen foes into Guardians, to forever serve the victor. Convicts and slaves were sometimes placed under a process to awaken their psionic potential, and then turned into guardians. Prisoners of war were almost always turned into warrior guardians, to forever serve those who vanquished them. The Green Age was an age of wonders, but strife, struggle and savagery were still the norm.

\subsubsection{Priests}
During the Green Age, clerical magic also came to be a source of power for people. While psionics was the dominant power and the source favored by the ruling elite, the elemental planes opened up their benefits to those who would make a pact with their patrons without concern for that person's station or nobility. Some elemental patrons took on personalities in the eyes of followers, and so a god of the forge and war was a patron from the elemental plane of fire, and the god of the harvest and fertility was connected to either earth or water. While rituals and themes varied from patron to patron and community to community, the source of these powers, the elemental planes, remained the same. Each city-state or tribe had its own views, though eventually the views of the Remans people took precedence. They had a Great Pantheon of gods, with elemental patrons taking on the identities of various figures for these groups. The Tanysh, in contrast, were concerned more with power and control, and didn't develop the same theological vision that the Remans did.

\subsubsection{Empires}
During this time, two different styles of empires fought back and forth. To the south lived the Remans, who populated what is now the Tablelands with a loose confederation of city-states. They would occasionally war with each other, but psionics flourished in these lands, and the centers for research and learning were found here. The wonders of Guestinel, the majesty of Bodach, the marvels of Celik and the beauty of Tyr were legendary. These were the jewels of the land, sparkling with knowledge and power. Each city-state had its own culture and leadership style, but as a group they were interested in prosperity and growth, and worked hard to develop their confederation into a land of plenty that was known as the Heartlands.

To the south of the Reman Confederation extending south to the icy Hoarwall, were the forests of Vanarra where the pixies and sprites and gnomes lived. Ulyan was also located here in the deep south, and was a place of trade and knowledge where all races met and mingled. The city of Nagarvos was a city ruled by mixed races. Nagarvos took after the settlements of the Remans, but did so in a way that made them look pedestrian. The people of Nagarvos were cosmopolitan, and enjoyed diversity and learning, as well as cultural racial diversity in a way unmatched anywhere else in the lands the Rebirth races settled.

The Tanysh empires were to the north, in places like Carsys, and on islands to the east, like Ebe and Draxa. They were highly aggressive, and fought amongst each other as much as against the Reman people. These peoples would sometimes conquer each other, only to have a vassal rise up and conquer its neighbors again. The rulers here would also send their sons and daughters to the centers of learning in the Heartlands, to better prepare them to lead the cities of the North. However, these children would often fight amongst themselves as often as against their rival nations, so progress in the Tanysh lands was often in rapid moves both forward and backwards. Cities were destroyed, only to be rebuilt later and become conquerors of others. This back and forth allowed for a culture of harsh dominance to develop, where leaders were willing to sacrifice their followers to reach their goals. Power was important beyond all else. It was like sand in their hands, constantly flowing in and out again.

Some places existed that were city-states unaffiliated with either the Reman or the Tanysh. Places like Saragar on the banks of the Marnita Sea were home to psionic academies that were rivals of Bodach and Celik. Eventually, Saragar became ruled by the Mind Lords, psionic users of such power that they became like gods.

Other places like Hogalay were centers that developed around one specific race. The dwarves in Hogalay were aligned with the elemental powers of Earth, and were an important place for the dwarven race, a northern outpost as prosperous as the capital of Kemalok was in the Heartlands.

During the Green Age, a new race appeared; migrating from the west from a place they called the Crimson Savanna. They did not say what it was that they were leaving, but the thri-kreen were both hard working and wise, even though their short life spans and insect mindset made interactions strained at first. After a time, their migration stopped, and they settled into a nomadic existence, hunting prey across the plains of the Tablelands and beyond.

\subsubsection{Rajaat}
Near the beginning of the Green Age a pyreen was born close to the Pristine Tower. The influence of the Tower on his body was extreme. Unlike other pyreen, he represented the worst features of the Rebirth races. Very deformed and ugly, he was, however, blessed with an incredible intellect. For millennia, he learned of the Will and the Way, traveled and spoke with the people of the Age. Despite his interactions with those around him, he could not accept himself as he was, and research into the past revealed to him that he was nothing more than a misshapen deformed accident.

His research into the past touched on the life-shaping of the nature-masters, and he sought to learn as much as possible. He spent time among the halflings of the Jagged Cliffs, learning from the one group that still had some semblance of understanding of the nature of life on Athas. With a few followers from the so called rhul-thaun, he traveled to a vast forest at the base of the cliffs to research a new way of manipulating life energy. After two centuries, he discovered two distinct ways of using the natural life force of living things called arcane magic: preserving and defiling. His desire for a cure to his condition caused him to attempt to use the life force of Athas itself as a power source. The attempt almost killed him and caused the entire forest to become a wasted swamp, with magical mutational energies that rival that of the Pristine Tower. He left with his followers for the site of his tragic birth, and the location that caused the Rebirth. There he refined the two types of magic for the next three thousand years, and altering the Tower to fit his ultimate needs. He made many plans for his final solution to the problem of the Rebirth until he found the one that pleased him.

When he made his presence known to the people of Athas, Rajaat taught magic to anyone who showed promise. Whereas the rules of the Age were that nobles were taught psionics, so as to keep them in power, the normal social structures did not apply to his students. Having been an outcast and reject by others, Rajaat welcomed those who could not gain entrance into the psionic academies, and also welcomed those who did. Magic was never as elegant as psionics. Its edge was in its simplicity and its power. He openly taught preserving magic to his students, being observant of which races did best with the studies and the methods involved. He set up schools and academies, sending trusted disciples to run them in far of city-states, and visiting each to make sure that the new novices learned directly from the one the people called the First Sorcerer. His calls for students went in waves with decades between accepting students. Those who answered underwent numerous difficult exams to prove their readiness and ability. Entrance into the schools allowed them to enter into a new life.

Because the power Rajaat granted to his students was not part of the social fabric of the time, it caused upheavals in some places. Students from marginalized groups banded together when they returned, and demanded recognition and a place in the decision making of the city-states. Mages, learned in more than just arcane magic, built towers within the cities and worked to help those around them. While it was initially a period of tension and adjustment, magic helped make the Time of Magic a wonderland that rivaled the rest of the Green Age. Cabals of wizards formed, working for good. One such group, the Wind Mages, worked to both find ways to improve the world, and also to improve their own position. These expansive fraternities often feuded with each other, but direct conflict was forbidden by Rajaat. Mages were to learn from each other, and rivalries should not turn to bloodshed. The First Sorcerer told all that he wanted his students to be better than the petty squabbling nobles with their psionic schools.

In secret, Rajaat taught students of questionable morality another way, defiling. Keeping these students close at hand and isolated from those who might discover his ultimate plan, Rajaat learned which races would take most quickly to defiling, as these would be the ones who could return Athas to the way it was supposed to be. Eventually, he learned what he needed to, and Rajaat stopped calling for new students. Rajaat told the world that they were ready to stand on their own now, and that he would stay in his Tower, happy to see the world had move closer to the harmony he had always wanted. Few knew the reality of what that harmony entailed.

\subsection{Playing in the Green Age}
Playing during the Green Age provides numerous examples of high adventure and rich campaigns. Of all the times that are able to be played, the Green Age is the most like a standard fantasy world, with psionics replacing arcane magic. This time period spanned thousands of years. Huge empires and city-states rose and fell. Campaigns around this, conflicts between city-states, nations and empires, or even internal politics, could be fruitful for players and DMs alike.

During this time, metal was more common than it currently is on Athas, and items and weapons made of metal were as common as bone and obsidian are now. Magic items existed, though they were either psionic in nature, or created by the priests who worked for elemental patrons. Games set in this time should highlight the power that is psionics, and it should rightly be a time of wonder and plenty.

During the Time of Magic, campaigns can introduce this power, and play through the introduction of magic and its effects on the social standing of different groups. Campaigns around discovering the origins of Rajaat, or his motives could be very interesting, perhaps even leading to epic level games. Near the end of the Time of Magic, games that involve the discovery of defilers and the plans of Rajaat could also be epic in nature, with as much detective work as combat.

The Green Age provides a rich world of adventure and exploration that should allow for anyone playing in this time to enjoy themselves.

\subsection{Character Races}
Almost any race can be played in the Green Age. Thri-kreen had at best limited interaction with the people of the Rebirth, and so this race would most likely not be playable. Half-giants had not been created, as they were the later work of the sorcerer-kings, who had yet to need such a large-bodied soldier. Muls could occur anywhere that humans and dwarves meet, though they would still be a rarity. Any other race can be played, however. One should use the standard races available to include orcs, goblins, gnomes, wemics, sprites and the other PC races. Any trait that they have that is a spell-like power should be replaced with its psionic equivalent, even if played during the Time of Magic. Races that aren't standard could also certainly exist, as they could have fallen by the wayside. The Pristine Tower is just as likely to mutate those who pass too close then as it is in the current age.

\subsection{Character Classes}
Almost any class is available during the Green Age. Gladiators existed in certain locales, but templars did not exist yet. Wizards did not exist until the Time of Magic. Necromancers, and shadow wizards did not exist until the start of the Cleansing Wars, and are not playable at this time. Templars did not exist as they do now, but tales abound of followers of the Great Ones being able to cast spells for their devotion.

Of all the classes, psions and psychic warriors were the most common. Wilders were also common amongst tribes and those who did not have access to the academies of the cities. Psionics is the tool of the age, even after the advent of arcane magic, and so psionic using characters will naturally have the most influence and prestige. Elemental clerics were also important during this time, and helped to bring stability to the different regions.

Arcane magic, once it came to be taught, would be seen as something exotic and also something that was able to change the social standing of those who practiced it. Those who learned did so under the gaze of Rajaat, whether directly or indirectly, and so this knowledge was both secretive and yet openly displayed for others to see. Playing a preserver in this time period would be something of an enigma, with the common people supporting their actions. Playing a defiler is a risk during this time, as only those closest to Rajaat would learn the secrets of defiling. Those preservers who existed would still not know of defiling, and would see the act as a failure of control and of willpower.

