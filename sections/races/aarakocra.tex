\section{Aarakocra}
\Quote{You are all slaves. You all suffer from the tyranny of the ground. Only in the company of clouds will you find the true meaning of freedom.}{Kekko Cloud-Brother, aarakocra cleric}

Aarakocra are the most commonly encountered bird-people of the Tablelands. Some are from Winter Nest in the White Mountains near Kurn, while others are from smaller tribes scattered in the Ringing Mountains and elsewhere. These freedom-loving creatures rarely leave their homes high in the mountains, but sometimes, either as young wanderers or cautious adventurers, they venture into the inhabited regions of the Tablelands.

\textbf{Personality:} These bird-people can spend hours riding the wind currents of the mountains, soaring in the olive-tinged Athasian sky. While traveling, aarakocra prefer to fly high above to get a good view all around their location and detect any threats well in advance. When they stop to rest, they tend to perch on high peaks or tall buildings. Enclosed spaces threaten the aarakocra, who have a racial fear of being anywhere they cannot stretch their wings. This claustrophobia affects their behavior. Unless it is absolutely necessary, no aarakocra will enter a cave or enclosed building, or even a narrow canyon.

\textbf{Physical Description:} Aarakocra stand 2 to 2.4 meters tall, with a wingspan of about 6 meters. They have black eyes, gray beaks, and from a distance they resemble lanky disheveled vultures. Aarakocran plumage ranges from silver white to brown, even pale blue. Male aarakocra weigh around 50 kg, while females average 42.5 kg. An aarakocra's beak comprises much of its head, and it can be used in combat. At the center of their wings, aarakocra have three-fingered hands with an opposable thumb, and the talons of their feet are just as dexterous. While flying, aarakocra can use their feet as hands, but while walking, they use their wing-hands to carry weapons or equipment. Aarakocra have a bony plate in their chest (the breastbone), which provides protection from blows. However, most of their bones are hollow and brittle and break more easily than most humanoids. The aarakocra's unusual build means they have difficulty finding armor, unless it has been specifically made for aarakocra. Aarakocra usually live between 30 and 40 years.

\textbf{Relations:} Aarakocra zealously defend their homeland. They are distrustful of strangers that venture onto their lands. Many of the southern tribes exact tolls on all caravans passing through their lands, sometimes kidnapping scouts or lone riders until tribute is paid. Tribute can take the form of livestock or shiny objects, which aarakocra covet. Some evil tribes may attack caravans without provocation. Aarakocra have great confidence and pride in their ability to fly, but have little empathy for land-bound races.

\textbf{Alignment:} Aarakocra tend towards neutrality with regard to law or chaos. With respect to good and evil, Aarakocran tribes usually follow the alignment of their leader. A tribe whose leader is neutral good will contain lawful good, neutral good, chaotic good and neutral members, with most members being neutral good. Aarakocra, even good ones, rarely help out strangers.

\textbf{Aarakocran Lands:} Most Aarakocran communities are small nomadic tribes. Some prey on caravans, while others or build isolated aeries high in the mountains. The least xenophobic aarakocra generally come from Winter Nest, in the White Mountains, a tribe allied with the city-state of Kurn. Of all the human communities, only Kurn builds perches especially made for aarakocra to rest and do business. In contrast, king Daskinor of Eldaarich has ordered the capture and extermination of all aarakocra. Other human communities tolerate Aarakocran characters but do not welcome them. Merchants will do business with aarakocra as long as they remain on foot. Most land-bound creatures are suspicious of strange creatures that fly over their herds or lands unannounced, and templars, even in Kurn, have standing orders to attack creatures that fly over the city walls without permission.

\textbf{Magic:} Most Aarakocran tribes shun wizardly magic, but a few evil tribes have defilers, and one prominent good-aligned tribe, Winter's Nest, has several preservers.

\textbf{Psionics:} Aarakocra are as familiar with psionics as other races of the tablelands. They particularly excel in the psychoportation discipline. In spite of their low strength and constitutions, they excel as psychic warriors, often using ranged touch powers from above to terrifying effect.

\textbf{Religion:} Aarakocran shamans are usually air clerics, sometimes sun clerics, and occasionally druids. Most rituals of Aarakocran society involve the summoning of an air elemental, or Hraak'thunn in Auran (although an aarakocra would call their language Silvaarak, and not Auran). Summoned air elementals are often used in an important ritual, the Hunt. The Aarakocran coming of age ceremony involves hunting the great beasts found in the Silt Sea.

\textbf{Language:} Athasian aarakocra speak Auran. Aarakocra have no written language of their own, though some of the more sophisticated tribes have borrowed alphabets from their land-bound neighbors. Regardless of the language spoken, aarakocra do not possess lips, and therefore cannot even approximate the 'm', 'b' or 'p' sounds. They have difficulty also with their 'f's and 'v's, and tend to pronounce these as 'th' sounds.

\textbf{Male Names:} Akthag, Awnunaak, Cawthra, Driikaak, Gazziija, Kraah, Krekkekelar, Nakaaka, Thraka.

\textbf{Female Names:} Arraako, Kariko, Kekko, Lisako, Troho.

\textbf{Tribal Names:} Cloud Gliders, Sky Divers, Peak Masters, Far Eyes, Brothers of the Sun.

\textbf{Adventurers:} Adventuring aarakocra are usually young adults with a taste for the unknown. They are usually curious, strong-minded individuals that wish to experience the lives of the land-bound peoples. Good tribes see these young ones as undisciplined individuals, but can tolerate this behavior. Evil tribes may view this sort of adventurous behavior as treacherous, and may even hunt down the rogue member.

\subsection{Aarakocra Society}
The aarakocra have a tribal society. The civilized tribes of Winter Nest form the largest known community of aarakocra in the Tyr region. Though their communities are lead by a chieftain, the aarakocra have a great love of personal freedom. So while the chieftain makes all major decisions for the community, unless she consults with the tribal elders and builds a strong consensus within the tribe first, her decisions may be ignored.

Air and sun shamans play an important role in aarakocra societies. Aarakocra worship the sun because it provides them with the thermals they need to soar. The air shamans of Winter Nest lead their community in daily worship of the air spirits.

Aarakocra of Winter Nest have a deep and abiding respect for the gifts of nature and little patience for those who abuse those gifts. They look after the natural resources of the White Mountains and have been known to punish those who despoil or abuse them.

In more primitive societies, female aarakocra rarely travel far from the safety of the nest, and focus solely on raising the young. In Winter Nest, both sexes participate in all aspects of society, with females more often elected by the elders to be chieftains.

Aarakocra believe that their ability to fly makes them superior to all other races and thus they have great confidence and pride in themselves. Though they often express sympathy for people unable to fly, this more often comes across as condescending.

Aarakocra are carnivores, but do not eat intelligent prey.
\subsection{Roleplaying Suggestions}
Loneliness doesn't bother you like it bothers people of other races. You loathe the heat and stink of the cities, and long for cold, clean mountain air. The spectacle and movement of so many sentient beings fascinates you, but watching them from above satisfies your curiosity. The very thought of being caught in a crowd of creatures, pinned so tight that you can't move your own wings, fills you with terror.

You are friendly enough with people of other races, provided they respect your physical distance, and are willing to be the ones that approach you. You form relationships with individuals, but don't involve yourself in the politics of other racial communities - in such matters you prefer to watch from above and to keep your opinions to yourself unless asked.

You prefer to enter buildings through a window rather than through a door. Your instincts are to keep several scattered, hidden, nests throughout the areas that you travel regularly: one never knows when one might need a high place to rest. Remember your love of heights and claustrophobia, and rely on Aarakocran skills and tactics (dive-bombing). Take advantage of your flying ability to scout out the area and keep a ``bird's eye view'' of every situation.

\subsection{Aarakocra Racial Traits}
\begin{itemize*}
    \item $-2$ Strength, +4 Dexterity, $-2$ Constitution: Aarakocra have keen reflexes, but their lightweight bones are fragile.
    \item Monstrous Humanoid: Aarakocra are not subject to spells or effects that affect humanoids only, such as charm person or dominate person.
    \item Medium: As Medium creatures, aarakocra have no special bonuses or penalties due to size.
    \item Low-light vision: Aarakocra can see twice as far as a human in moonlight and similar conditions of poor illumination, retaining the ability to distinguish color and detail.
    \item Aarakocra base land speed is 6 meters, and can fly with a movement rate of 27 meters (average maneuverability).
    \item +6 racial bonus to \skill{Spot} checks in daylight. Aarakocra have excellent vision.
    \item Natural Armor: Aarakocra have +1 natural armor bonus due to their bone chest plate that provides some protection from blows.
    \item Natural Weaponry: An aarakocra can rake with its claws for 1d3 points of damage, and use its secondary bite attack for 1d2 points of damage.
    \item Claustrophobic: Aarakocra receive a $-2$ morale penalty on all rolls when in an enclosed space. Being underground or in enclosed buildings is extremely distressing for them.
    \item Aerial Dive: Aarakocra can make dive attacks. A dive attack works just like a charge, but the diving creature must move a minimum of 9 meters. If attacking with a lance, the aarakocra deals double damage on a successful attack. Optionally, the aarakocra can make a full attack with its natural weapons (two claws and one bite) at the end of the charge, dealing normal damage.
    \item Automatic Languages: Auran and Common. Bonus Languages: Elven, Gith, and Saurian. Aarakocra often learn the languages of their allies and enemies.
    \item Favored Class: Cleric. A multiclass aarakocra's cleric class does not count when determining whether he takes an experience point for multiclassing.
    \item Level Adjustment: +1. Aarakocra are slightly more powerful and gain levels more slowly than most of the humanoid races of the Tablelands.
\end{itemize*}