\section{Wizard}
\Quote{So what if the land becomes barren? It’s not like we’re going to stick around.}{Datuu Dawnchaser, elf defiler}

Athasian wizards drain energy from the surrounding soil. The method used labels the wizard as a defiler or a preserver. Preservers have the self-control to gather energy without destroying plants. Those who do not, or who feel no remorse about the damage caused, become Defilers. Defilers leave behind sterile soil and infertile ash when they cast spells. Because of this, most wastelanders blame wizards for the desert landscape that dominates the Tablelands today, and their hatred extends to defilers and preservers alike. In the seven cities, arcane magic is outlawed and feared.

Writing is also illegal in the Tablelands, thus wizards have to go to great lengths to conceal their spellbooks, and they have refined this art to the point where even fellow wizards can be hard pressed to identify a spell book. When found, they are precious resources, hoarded and studied by wizards thirsty for knowledge or power.

\subsection{Making a Wizard}
The wizard’s greatest strength is also his greatest liability. Often wizards will conceal their abilities, learning to mask their spellcasting behind other actions. For all but the most powerful wizards, secrecy is of prime importance, and some will not exercise their power in the presence of those that they do not feel they can trust. Because of this, and because of their generally frail nature, wizards can often be seen as a liability by those not aware of the power they hide.

\textbf{Races:} Elves and humans are the most likely to be wizards. Elves are more tolerant of the faults of magic, even at its worst, due to their nomadic nature. Defiling simply isn’t as much of a concern if the ruined land is fifty miles behind you by the end of the next day. The solitary life lead by most half‐elves makes it easier for them to conceal their wizardry, should they choose to follow that path. Some rare halflings and pterrans will take up the arts of wizardry, but these races are so closely tuned to flow of life on Athas that they will never willingly defile. Half‐giants, trusting and slow-witted, rarely become wizards, and those that do rarely survive for long. Dwarves rarely take to the magic arts, though their focus allows those that do to become exceptionally skilled. Thri‐kreen and muls almost never become wizards.

\textbf{Alignment:} Overall, most wizards display a tendency towards lawfulness. The self‐control and restraint necessary to keep oneself secret, as well as the disciplined need for long days of studying take their toll on many of the less careful wizards. Most wizards of good alignment have developed the skill and control necessary to master preserving, and only in the direst of situations would a good‐aligned wizard defile. Neutral or evil wizards, however, are more likely to become defilers, though evil preservers are not unheard of.

\subsection{Game Rule Information}

\textbf{Hit Die:} d8.

\subsubsection{Class Skills}
Bluff (Cha), Concentration (Con), Craft (Int), Decipher Script (Int), Disguise (Cha), Knowledge (all skills, taken individually) (Int), Literacy (N/A), Profession (Wis), and Spellcraft (Int).

\textbf{Skill Points per Level:} 2 + Int modifier ($\times4$ at 1st level).

\subsubsection{Class Features}
\textbf{Weapon and Armor Proficiency:} Wizards are proficient with the club, dagger, heavy crossbow, light crossbow, and quarterstaff, but not with any type of armor or shield. Armor of any type interferes with a wizard’s movements, which can cause her spells with somatic components to fail.

\textbf{Spells:} A wizard casts arcane spells which are drawn from the sorcerer/wizard spell list. A wizard must choose and prepare her spells ahead of time (see below).

To learn, prepare, or cast a spell, the wizard must have an Intelligence score equal to at least 10 + the spell level. The Difficulty Class for a saving throw against a wizard’s spell is 10 + the spell level + the wizard’s Intelligence modifier.

Like other spellcasters, a wizard can cast only a certain number of spells of each spell level per day. Her base daily spell allotment is given on \hyperref[tab:The Wizard]{Table: The Wizard}. In addition, she receives bonus spells per day if she has a high Intelligence score.

} wizard may know any number of spells. She must choose and prepare her spells ahead of time by getting a good night’s sleep and spending 1 hour studying her spellbook. While studying, the wizard decides which spells to prepare.

\textbf{Bonus Languages:} A wizard may substitute Draconic for one of the bonus languages available to the character because of her race.

\textbf{Familiar:} A wizard can obtain a familiar in exactly the same manner as a sorcerer can.

\textbf{Scribe Scroll:} At 1st level, a wizard gains Scribe Scroll as a bonus feat.

\textbf{Bonus Feats:} At 5th, 10th, 15th, and 20th level, a wizard gains a bonus feat. At each such opportunity, she can choose a metamagic feat, an item creation feat, or Spell Mastery. The wizard must still meet all prerequisites for a bonus feat, including caster level minimums.

These bonus feats are in addition to the feat that a character of any class gets from advancing levels. The wizard is not limited to the categories of item creation feats, metamagic feats, or Spell Mastery when choosing these feats.

\textbf{Spellbooks:} A wizard must study her spellbook each day to prepare her spells. She cannot prepare any spell not recorded in her spellbook, except for read magic, which all wizards can prepare from memory.

A wizard begins play with a spellbook containing all 0-level wizard spells (except those from her prohibited school or schools, if any; see School Specialization, below) plus three 1st-level spells of your choice. For each point of Intelligence bonus the wizard has, the spellbook holds one additional 1st-level spell of your choice. At each new wizard level, she gains two new spells of any spell level or levels that she can cast (based on her new wizard level) for her spellbook. At any time, a wizard can also add spells found in other wizards’ spellbooks to her own.


\SuperBlock{
School Specialization
A school is one of eight groupings of spells, each defined by a common theme. If desired, a wizard may specialize in one school of magic (see below). Specialization allows a wizard to cast extra spells from her chosen school, but she then never learns to cast spells from some other schools.

A specialist wizard can prepare one additional spell of her specialty school per spell level each day. She also gains a +2 bonus on Spellcraft checks to learn the spells of her chosen school.

The wizard must choose whether to specialize and, if she does so, choose her specialty at 1st level. At this time, she must also give up two other schools of magic (unless she chooses to specialize in divination; see below), which become her prohibited schools.

A wizard can never give up divination to fulfill this requirement.

Spells of the prohibited school or schools are not available to the wizard, and she can’t even cast such spells from scrolls or fire them from wands. She may not change either her specialization or her prohibited schools later.

The eight schools of arcane magic are abjuration, conjuration, divination, enchantment, evocation, illusion, necromancy, and transmutation.

Spells that do not fall into any of these schools are called universal spells.

\textbf{Abjuration:} Spells that protect, block, or banish. An abjuration specialist is called an abjurer.

\textbf{Conjuration:} Spells that bring creatures or materials to the caster. A conjuration specialist is called a conjurer.

\textbf{Divination:} Spells that reveal information. A divination specialist is called a diviner. Unlike the other specialists, a diviner must give up only one other school.

\textbf{Enchantment:} Spells that imbue the recipient with some property or grant the caster power over another being. An enchantment specialist is called an enchanter.

\textbf{Evocation:} Spells that manipulate energy or create something from nothing. An evocation specialist is called an evoker.

\textbf{Illusion:} Spells that alter perception or create false images. An illusion specialist is called an illusionist.

\textbf{Necromancy:} Spells that manipulate, create, or destroy life or life force. A necromancy specialist is called a necromancer.

\textbf{Transmutation:} Spells that transform the recipient physically or change its properties in a more subtle way. A transmutation specialist is called a transmuter.

\textbf{Universal:} Not a school, but a category for spells that all wizards can learn. A wizard cannot select universal as a specialty school or as a prohibited school. Only a limited number of spells fall into this category.
}