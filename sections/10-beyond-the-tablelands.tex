\section{Beyond the Tablelands}
Plenty of action and adventure can be found across Athas. More lands of wonder, mystery, and danger exist beyond the barrenness of the Tablelands. A quick tour of these other places follows, and more information about specific locations will be revealed in future products.

\SubCity{Eldaarich}
{21,000 (85\% humans, 8\% dwarves, 4\% half‐giants, 2\% muls, 1\% others).}
{Gold, silver.}
{Eldaarish, picts.}
{
	Eldaarich occupies a small island in the Sea of Silt, just off the mainland. Here, isolated and protected from the rest of Athas, the citizens huddle in the paranoid delusions of their mad sorcerer‐king. Daskinor, ruler of Eldaarich, believes that unknowable forces in the world are trying to destroy him.

	Every few years he puts a new name to these forces---the Order, the Veiled Alliance, Rajaat, pyreen, a merchant house, a lowly slave, or some other identifiable target becomes the imagined source of his fears for a time. Daskinor does his best to destroy these imagined enemies, and anyone who has even a passing resemblance to the target is persecuted until the next delusion grips him.

	Daskinor was never a stable ruler. From the beginning of his reign as sorcerer‐king of Eldaarich, he was tormented by unfounded fears and nameless terrors that preyed upon his mind. For the first few centuries of his reign, he was able to function more or less normally despite his growing paranoia. As time passed, genuine bouts of panic began to intrude upon his psyche. These bouts lasted longer and longer, paralyzing Daskinor for hours, days, sometimes even months at a time.

	Eldaarich was constructed to protect Daskinor from his fears. Fortified walls, a strong military, devoted templars, retractable bridges, and a series of keeps and forts ensured that the entire city‐state and surrounding area was secured against outsiders. Over time, it became less of a fort and more of a prison, locking king and citizens alike behind sturdy gates and high walls. Seven centuries ago, the sorcerer‐king's paranoia became acute. He completely sealed his city, cutting off all ties to the other city‐states. That was the way things remained until about FY 0 year, when limited trade was resumed with House Azeth of Kurn.

	Today, Eldaarich remains an isolated prison of a city. Daskinor's fears have become the fears of his citizenry, making everyone who lives under his rule as paranoid as he is. No one ever leaves Eldaarich, and no one ever enters its massive gates. It's a closed society---figuratively and literally.
}
{
	Every outsider wants to destroy their city‐state and their sorcerer‐king, and everyone who lives within the walls waits for an opportunity to betray you. That's what the people of Eldaarich believe, for that's what their leaders believe. Nowhere else in all of Athas is there such an underlying current of genuine, unattributable fear. It filters down from Daskinor himself, making citizen and slave alike tremble with uncontrollable paranoia.

	The citizenry is a subdued, cowering lot, given to unexpected bursts of violence once the fear inside them becomes too much to contain. In many cases, the ever‐crushing weight of terror and oppression keeps the masses down, but sometimes a delusional artisan will strike out at a templar or noble, causing the level of paranoia to rise even higher.

	The quality of life isn't good in Eldaarich. Because Daskinor doesn't trust anyone, he allows his templars to dispense only the barest essentials to the free citizens and slaves. With just enough food and water to sustain them and few personal possessions, the people of the city are a sad, pathetic lot. They have no hope of a better life and no concept that a better life exists outside the walls of Eldaarich. If anyone even suggests such a notion, the ingrained fear of the unknown kicks in and makes everyone else dismiss the idea.

	While the class structure of noble, free citizen and slave exists in Eldaarich, the truth is that everyone beneath the templars is a slave to Daskinor's all‐pervasive fear.

	The sorcerer‐king sees threats to his rule on every face and in every dark shadow. For this reason, he permits no freedoms of any sort, not even the token rights given to the citizens of other cities. Freedom, Daskinor believes, is just an opportunity to betray his trust. So he orders his templars to oppress the people of his city, to make their lives so miserable they don't have time or strength to contemplate treachery.

	The templars don't have it much better. They're kept in line by the high templars who, in turn, are subject to Daskinor's brutal whims.

	The majority of the population consists of humans, though there are also dwarves, half‐giants, and muls in significant numbers. There are also a few aarakocra wasting away in the slave pens. Daskinor has a particular hatred of the winged people and gives his templars special compensations for capturing aarakocra from the nearby White Mountains.

	If travelers were to find themselves in Eldaarich or one of its holdings (which isn't very likely), they'd feel the weight of oppression and smell the stench of mental illness that hangs in the hot, stifling air. Every year the darkness in Daskinor's soul grows deeper, his paranoia more acute. This mental deterioration is reflected in the city itself, as though each citizen were a part of the sorcerer‐king's diseased mind.
}
{
	The same model of government evident in the other city‐states exists in Eldaarich. The sorcerer‐king Daskinor (CE male stage II Champion of Rajaat dragon defiler 8/nomad 10/cerebremancer 10/Athasian dragon 2) stands atop the societal hierarchy, his troubled delusions coloring every aspect of life in the city‐state. His chaotic tendencies and often overwhelming paranoia infuse everyone he comes in contact with, making the city almost as wild and frenzied as Raam. The only thing that allows the city to function is that the citizens are a subdued lot, living in quiet fear instead of in rambunctious anarchy. Daskinor constantly watches over his shoulder for assassins that don't exist, and so do his templars and nobles. No one trusts anyone else in Eldaarich. This works out for the best, as the troubled atmosphere has fostered a society where the fear of murder and betrayal has encouraged the periodic use of such techniques by those who prefer to strike first.

	Templars and nobles regularly kill each other to keep the same from happening to them, or to gain power or position, or just because the tension of living behind heavy locks and being constantly on guard eventually drives even the most peaceful beings to violence. In Eldaarich, fears permeates everything---fear of the sorcerer‐king, fear of outsiders, fear of each other, and fear of the unknown. Because the society is closed off to the rest of the world, everything on the other side of its walls and locked gates is, by definition, unknown.

	If Eldaarich is a prison, Daskinor is its most prominent prisoner. The sorcerer‐king lives in a walled sub‐city and rarely ventures into other parts of his realm. His constant paranoia sometimes intensifies to such a fevered pitch that he ceases to function.

	In such a state, which may last as long as months at a time, Daskinor is cared for by his senior templars. At other times, his paranoia drives him to give a name to his fear. When this occurs, the entire city mobilizes to combat this supposed threat to the realm. Currently, the use of psionic abilities has been outlawed, as Daskinor believes that the Order has initiated a campaign against his rule. Even low‐powered psionicists and wild talents who openly display their abilities are subject to imprisonment or death because of the current edict. Only Daskinor, a psionicist of the highest caliber, is exempt from the terms of the edict.

	Daskinor's templars serve as administrators to the city, and also act as the sorcerer‐king's eyes and ears in all corners of the domain. They are charged with watching for signs of treachery among the masses‐and with dealing with such treachery before it gets out of hand. The templars are as paranoid and delusional as Daskinor, giving in to their fear whenever it overwhelms them. For this reason, Eldaarich has become a police state, and the templars are the police. They command the military. They oversee all records and the distribution of goods and services. They hold the power of life and death for the rest of the citizenry in their terrified hands.
}
{
\textbf{Kulag:} The Kulag Order controls Daskinor's silt fleet, which currently acts as the merchant house for the Dim Lands, a nearby archipelago. It is leaded by High Templar Kerillis (LE female human, templar 14). Sometimes they also resort to piracy in the nearby islands.

\textbf{Neshtap:} More commonly known as “red guards”, the Neshtap are the most feared, and the second‐most powerful of seven orders that Daskinor uses to maintain control of his city Eldaarich, and its client villages. They never speak, seemingly revere the element of fire, and are becoming increasingly powerful and independent from Daskinor.

\textbf{The Veiled Alliance:} Eldaarich has no Veiled Alliance. Daskinor rooted out the Alliance and destroyed it 400 years ago when the group of preservers became his imagined enemy of the moment. Some preservers still live in the city, but they remain hidden and are relatively weak due to a lack of adequate training. Preservers from Kurn sometimes sneak into the closed city to provide training and to see what the conditions are, but they don't do this very often. If they get caught, they're put to death, and if their city of origin is discovered, it could mean war between the two cities. No one, especially Oronis the Avangion of Kurn, wants a war to break out. He does, however, feel the pain that both Daskinor and his citizens project, and often contemplates finding a solution to Eldaarich's problems.
}
{}{}
{
	\item A silt schooner owned by House M'ke was attacked and captured by the navy of Eldaarich. The merchant house could hire the PCs to raid the harbor of Eldaarich and bring the schooner back.
	\item An aarakocra from Winter Nest was captured when she flew too close to Eldaarich. The PCs are asked to free her before she is executed by the templars of Eldaarich. Once freed from her cage, the aarakocra can easily fly back to Winter Nest on her own, but the PCs will have to sneak out of Eldaarich.
	\item Grehgatha is a Kurnan preserver who has snuck into Eldaarich many times to tutor young preservers. Since she has returned from her last attempt she is consumed with freeing an entire village from Daskinor and hires the PCs to help. The PCs must come up with a way to sneak 150 people past the templars of Eldaarich.
	\item The Red Guard has become jealous of the monopoly on trade held by the templars of the Kulag Order. In an attempt to disrupt the trade negotiations, the Red Guard mounts a surprise attack on Silt Side during a meeting between Corik Azeth and High Templar Kerillis. The PCs acting as guards for House Azeth may misinterpret the attack as directed by Corik or themselves.
	\item Concerned with a recent rise in the level of the silt sea around Eldaarich, the city has declared war on all silt clerics. Mercenaries are to be hired to help hunt down the silt clerics along the coast for a hundred miles north and south of Eldaarich. The PCs could become embroiled on either side.
	\item A major giant raid on the Huuros Islands has been repulsed by the Kulag Fleet, though many casualties were suffered. Templars assign the PCs to salvage what they can from the battle, equipment as well as the bodies of those who died. Most of the wreckage is just off shore in silt 15 to 20 feet deep.
}
\SubCity{Kurn}
{18.000 (65\% humans, 10\% elves, 6\% muls, 6\% aarakocra, 5\% dwarves, 4\% half‐elves, 3\% half‐giants, 1\% other).}
{livestock, magic items, medicines.}
{Elven, Kurnan.}
{
	Kurn is actually two city‐states: an ancient, public metropolis, and a utopian city hidden from the rest of the world. Old Kurn sits in a lush meadow on the eastern side of the White Mountains. The trade road running north out of Draj connects Kurn to the Tyr Region, and the city welcomes merchants from the south. New Kurn lies in a fertile valley hidden among the White Mountains themselves. A secluded road protected by a towering fortress keeps the valley safe from unwanted visitors---and New Kurn doesn't want any visitors.

	Old Kurn was a prosperous but relatively small city from the Green Age that suffered great devastation in the early days of the Cleansing Wars. Once situated in a vast forest that has long since faded from the landscape, the elf city of Kurn was destroyed by the Champion called Albeorn, Slayer of Elves. When the Champions finally turned against Rajaat and became the dragon kings, the one named Keltis decided to build his city‐state on the ruins of Old Kurn. He changed his name to Oronis, but decided to retain the name of the city he was building over.

	The ruins weren't in as bad a shape as Oronis originally thought. He was able to build upon many of the foundations, and a few whole structures were still fit for use. Within a decade, Oronis' Kurn was established. Within five decades, it was thriving. For five hundred years, Kurn followed the same course as the other sorcerer‐king domains. Throughout that time, Oronis was troubled by something few of his peers possessed---his conscience.

	When he was Keltis, Lizard Man Executioner, he succeeded at the task Rajaat handed to him. He eliminated the entire race from the face of Athas. As the years passed and Keltis the Champion became Oronis the sorcerer‐king, images of the atrocities he committed started to haunt him. After Oronis advanced to a second stage dragon king, his problems intensified. Now he had the deaths of his subjects on his head, for he had to use a specified amount of life force to power his transformation.

	He decided that none of this was what Rajaat originally promised him. Where was the restoration of the world? Athas hadn't gotten better because of the Cleansing Wars. It had gotten worse. What's more, the sorcerer‐kings were continuing the downward spiral, slowly killing the world by their actions. Oronis refused to be a part of that trend any longer. He renounced his defiling skills and his status as a dragon king and sought a different path.

	That was when Kurn broke off relations with the other city‐states. Mercantile activities continued, of course, but at a reduced rate. After a time, Kurn became one of the forgotten cities---just as Oronis had hoped. In the meantime, he set the next part of his plan for redemption in motion. Oronis wanted to make amends for the horrors of his past.

	The first step was to change the rules of society in Kurn. Though the city had to maintain an illusion of normalcy to keep the other sorcerer‐kings from detecting treachery or weakness, Oronis secretly freed all slaves and instituted fair and just practices at all levels of society. He swore his citizens to secrecy, for if word got out he was sure his one‐time peers would flock to Kurn like gith to a dying braxat. The second step was to begin construction on the utopia he envisioned. Like all ex‐Champions, Oronis originally wanted to return Athas to the glory of the Blue Age.

	He decided to once more strive for that goal. In a hidden valley among the peaks of the White Mountains, the foundation stones of New Kurn were laid. As his templars and citizens worked to build New Kurn, Oronis went in search of a better path to power. Using the techniques and practices of preserving magic, Oronis looked for a way to combine magic with psionics in a more positive way than through dragon magic. It took nearly 1,000 years of study and experimentation for Oronis to develop the preserver metamorphosis spell. With it, the reformed sorcerer‐king could become an advanced being aligned to goodness instead of another force for evil.

	Today, the twin cities of Kurn continue along their parallel courses. Old Kurn displays a typical sorcerer‐king's domain to the other inhabitants of the region, at least on the surface, while New Kurn works to complete Oronis' experiment in regressing a small portion of Athas back through time. Between the two cities, Kurn has a total population of 18,000 people. The majority live in the new city, as each year more citizens are moved from the old city to the new. Old Kurn has such a small number of residents that it appears to be almost a ghost town, and one day Oronis plans to completely abandon it in favor of his secluded valley.
}
{
	The state of life in Kurn depends on which of the twin cities is being considered. Old Kurn, on the surface, appears to be much like any city in the Tyr Region still ruled by a sorcerer‐king. Surface appearances, however, can be deceiving. Travelers who stay for any length of time might notice a few oddities. For example, the slaves seem to have a sparkle in their eyes and a bounce in their step that isn't seen in the other city‐states, and templars aren't given as wide a berth as their counterparts in Urik or Nibenay. Additionally, while the merchant and tradesmen districts are always crowded, the rest of the city is as empty and desolate as the ruins of Giustenal.

	Old Kurn maintains its illusion of business‐as‐usual through the cooperation of its citizens and the advanced powers of its sorcerer‐king. If visitors notice that the noble and templar quarters of the city are practically deserted, they usually attribute it to the rumors that Kurn is slowly dying. Dying or not, the city is far from defenseless. More than one raiding tribe has attempted to take advantage of the “dying” city only to discover that its defenders were more than capable of driving them off.

	Through the efforts of House Azeth and the commerce provided by other traders, Kurn maintains a modest economy. While most of the inhabitants of the Tyr Region have forgotten that this northern city exists, Kurn interacts with its closest neighbors on a regular basis. It has good relations with the aarakocra of Winter Nest, the merchants of Draj's House Tsalaxa, and the elves of a few of the local tribes. Except for the contact between House Azeth and the trade templars of Eldaarich, Kurn has little interaction with its neighboring city‐state. On the other hand, Kurn sometimes has trouble with raiders from the Bandit States. The raiders don't come to the gates of the city (at least not very often), but they do attack travelers on the trade road and even plunder the client villages on rare occasions.

	New Kurn is a different matter. The high, sturdy walls of Fort Protector block the eastern entrance to the hidden valley, while the tall, steep peaks of the White Mountains make the other directions inaccessible. The only approach that might be open is by air, though flying creatures loyal to Oronis nest in the vertical peaks.

	Within the valley, Oronis' restoration project is in full swing. He has turned the valley into a place from the past, recreating the conditions of the Green Age in its sheltered space. A thick forest surrounds a lush clearing where the city of New Kurn has been built beside a small, clean lake. Oronis hopes to eventually regress the valley to conditions as they were in the Blue Age, but that's still many years away.

	The new city resembles Oronis' vision of utopia. Airy buildings with tall, elegant spires grace wide, open streets paved with white stone. Here, the people govern themselves through a system of fair laws and majority rule. Everyone has a say in the workings of the city, from the poorest laborer to the highest elected official. And if someone doesn't like the way things are going, they're free to run for a position when the current terms of office expire.

	Thanks to the fertile valley and the lush forest, no one goes hungry or thirsty in New Kurn. No creatures are hunted out of existence and no plants are plucked completely from a given area. The templars monitor the forest on a daily basis to make sure the delicate balance is maintained. For this reason, no defilers are permitted within the ranks of the templars or anywhere in the twin cities. It is strictly against the laws of Kurn to practice defiling magic.

	Oronis continues to advance as an avangion, and he tries to instill the same serene, peaceful, life‐giving properties of his new form in the city and people who follow him. Where once there was a man of evil, now Oronis is a force for good in the world. His templars work to promote his plans and prepare to someday strike out from the valley with the knowledge of how to restore all of Athas. Until then, they'll work to finish the restoration of the valley and to perfect the society that Oronis has inspired.
}
{
	Oronis the Avangion (LG male Champion of Rajaat stage IV avangion, preserver 5/shaper 5/cerebremancer 10/loremaster 3/avangion 5) guides the paths of the twin cities. Oronis spent centuries redeeming himself, going so far as to change his very nature from evil to good, though he still feels he has a long way to go to make up for his acts as a Champion of Rajaat and a sorcerer‐king. For this reason, he has dedicated himself and his citizens to working toward the eventual restoration of all Athas.

	While in Old Kurn, Oronis wears the guise of a normal human. In this psionically and magically induced disguise, he appears as a tall, lanky, middle‐aged man with short golden hair, pale‐blue eyes, and a close‐cropped blond beard. He covers himself in the trappings of a sorcerer‐king, wearing a golden circlet on the crown of his head and carrying an obsidian‐topped walking staff. In New Kurn, however, such disguises aren't called for. There he openly displays his true avangion form---a tall, thin, hairless humanoid with golden skin, silver eyes, and gossamer wings.

	Though Old Kurn appears to run like any other city-state, Oronis long ago abandoned a monarchical form of government. He allows his subjects to govern themselves via a democratic system he developed. In this system, nobles and all citizens except templars may hold public office. Elections are held at regular intervals and term limits are set. The highest elected official is called the Presider, who sits at the head of a body called the Tribunal. Members of the Tribunal are referred to as Tribunes. Together, the Presider and the Tribunes draft the laws that keep the city‐state running smoothly. The current Presider is Ulali of Prusicles (LG female half‐elf, preserver 8), now in the second year of a five‐year term.

	Oronis refuses to hold an official position, though he does pretend to be sorcerer‐king in the old city. He acts as an adviser when the Presider or Tribunal requests his presence, but otherwise, he's more concerned with advancing as an avangion and keeping the valley restoration project on track. Oronis' templars don't serve as administrators in Kurn, either. Instead, they are the keepers and dispensers of knowledge, serving as teachers and advisers to local officials and businesses. It's also their job to oversee and handle the restoration process, under Oronis' supervision.
}
{
	\textbf{Black Brethren:} Oronis' Black Brethren are Kurn's elite army, charged with patrolling Kurn and making sure Kurn is safe and secret.

	\textbf{School of Spies:} Kurn's School of Spies is an organization of Kurnan spies, mostly female, that studies non‐Kurnan societies, and brings back information to defend Kurn and improve its way of life. They have managed to infiltrate into Merchant Houses and even the templarate of every city‐state in the Tablelands.

	\textbf{The Veiled Alliance:} Kurn has no Veiled Alliance. Preservers are a welcome and significant part of the society, so there's no reason for them to hide behind a veil of secrecy. In fact, preservers from other Alliance factions sometimes come to Kurn to study with Oronis. One preserver, Korgunard of Urik, even learned the steps to become an avangion and followed the path forged by Oronis. It's conceivable that more avangions will appear in the future, though when and how many is hard to say. While preservers are accepted and integral to Kurn society, defilers are considered enemies of everything Oronis stands for. The avangion is reluctant to allow his followers to make defiling magic punishable by death, as he himself was once a defiler of the highest order. However, he knows that in most cases defilers can't make the mental and spiritual changes necessary to reject that path, so he has agreed that known defilers must be banished from the society.
}
{
	\textbf{Azeth's Rest (Village, 900):} This fortified oasis and trade village lies on the trade road, reaching north from Draj to Kurn. It has remained in the hands of House Azeth ever since the trade village was founded. Fifty tough mercenaries protect it and the nearby road, manning the ballistae and fixed crossbows atop its great walls. Azeth's Rest welcomes all traders, provided they can pay the fees for using its services.

	\textbf{Silt Side (village, varies):} Silt Side is an open village on the coast of the Silt Sea. Silt Side handles trade with Eldaarich; in fact, this village is the only connection with the outside world that Eldaarich maintains. Silt Side is a seasonal village, populating and emptying for a few weeks three times every year when House Azeth members meet to trade.
}
{}
{
	\item Oronis needs many unusual spell components for his studies. Often times he does not have the time to gather all of them himself, so he hires the PCs to collect some of the rarer spell components he needs, such as roc eggshells, leather from a dune reaper matron, the bark of a zhackal, or silt eel tongues.
	\item The last time the PCs were in Kurn, they were befriended by Aloth, a friendly merchant. But now that they have returned to Kurn, Aloth has disappeared and his shop is being used by another merchant. No one claims to have heard of Aloth when the PCs ask. Has Aloth been secretly granted citizenship in New Kurn, or has something more sinister happened to him?
	\item The residents of New Kurn are up in arms when a patch of defiled ground is discovered. Suspicion falls quickly on the newest members of the community, the PCs. Actually, there is no defiler in New Kurn. The defiled ground was caused by a magical object that uses a defiling effect to power its magic. One of the preservers of New Kurn recently acquired the item and tested it, not realizing what it would do. Now he is horrified that he will be blamed for the defilement, anger Oronis, and be forbidden to practice magic, so he remains silent.
	\item The PCs have to figure out who murdered a merchant in Kurn. But the investigation is hampered when many of the witnesses and suspects disappear. Are they being relocated to New Kurn or is something more sinister happening?
	\item New Kurn needs a cistern fiend to purify its water supply. The Tribunal will greatly reward adventurers that can find and transport a cistern fiend to New Kurn.
	\item The bee keepers of Kurn are concerned. Their bees have been disappearing. Every morning the bees leave the hives and every afternoon less of the bees return. Is this some new threat from the sorcerer‐king of Eldaarich or is someone or something gathering the bees in the desert?
}

% Pterran Vale
% the other side of the Ringing Mountains. Trade routes are
% being established with Tyr and areas beyond.
% Government and Politics
% As in all pterran communities, Pterran Vale is lead by
% a Triumvirate. The Triumvirate is made up of the eldest
% member from each of the three primary Life Paths. The
% Triumvirate has the power to make all decisions for the
% community. However, before important decisions are
% made the entire community gathers to debate the question
% in front of the Triumvirate. Only after all pterrans have
% had their say does the Triumvirate make their decision.
% Power Groups
% Traders: Recently, the prestige of the merchants of
% Pterran Vale, lead by Ptellac Goldeye, has been growing.
% Since they are at the forefront of the pterrans’ new push to
% make contact with civilizations outside of their vale, the
% traders have become well respected. In addition, the new
% trade routes they have developed to the cities of the
% Tablelands have brought them increased wealth.
% The traders are not an organized group as yet.
% Thought the different merchants are business rivals some
% have begun to recognize their new found status and how
% if they united, they could exert significant influence over
% the pterran society of the Hinterlands.
% Major Settlements
% Pterran Vale is the largest community of civilized
% pterrans in the Hinterlands. The buildings are lodges
% constructed from the bones and hides of large creatures,
% such as mekillots, built over hollowed out pits. Each
% building has steps leading down into the interior. Lost Scale (Small Town, 2,000): Centuries ago, a
% religious dispute resulted in a schism in Pterran Vale. One
% group found itself in the minority and chose to leave
% Pterran Vale and establish their own community, and
% founding Lost Scale. Today the disagreement has long
% been settled and the two communities work together.
% Lost Scale is noted for its legion of expert pterrax
% riders. Each of these warriors searches rocky badlands
% and canyons for a pterrax egg. The baby pterrax that
% hatches is raised and trained from birth by its rider.
% Life and Society Adventure Ideas
% Population: 4,000 (99% pterran, 1% other)
% Exports: bones tools, livestock
% Languages: Pterran
% The pterrans of Pterran Vale survive by hunting,
% farming, and herding. In addition to using bone in the
% construction of their buildings, they make fine bone
% weapons and tools.
% Each pterran must choose a life path when they come
% of age. There are three main Life Paths, the Path of the
% Warrior, the Path of the Druid, and the Path of the Psion.
% However, other lesser Life Paths, farmer, crafter, traders,
% and herders, also exist. Those pterrans following one of
% the lesser Life Paths are treated with respect but the three
% primary Life Paths are more prestigious. All leaders are in
% pterran society are selected from the primary Life Paths.
% The pterrans revere Athas as the Earth Mother, and
% their religious ceremonies and celebrations are devoted to
% her. After the Great Earthquake, the pterrans became
% convinced that the earthquake was a call from the Earth
% Mother to them, directing them to become more involved
% with the affairs of others. In response, explorers have been
% sent out resulting in contact with the city‐state of Tyr on
% ―After the success of Ptellac Goldeye’s effort to make
% contact with the cities of the Tablelands, the leaders of
% Pterran Vale have decided to send emissaries to the south,
% where rumors state long lost communities of pterrans
% exist. While the rumors are indeed true, the pterrans of
% the south have long ago degenerated into barbarism and
% cannibalism, making the expedition fraught with peril.
% ―Pterran scouts have witnessed thri‐kreen with
% unusual coloration herding trin packs into a large
% enclosure in the desert. Concerned as to who these
% strange thri‐kreen are and why they are herding trin, the
% leaders of Pterran Vale send the PCs to investigate.
% ―The crops in the fields around Pterran Vale are not
% growing as healthily as they normal do. The farmers
% quickly discovered why. Blood grass has sprouted up
% throughout the fields, stealing nutrients from the crops
% and attacking farmers who approach too close.
% Adventurers are needed to clear the blood grass from the
% fields, as well as discover how the blood grass came to be
% there in the first place.
% ―The pterrax riders of Lost Scale are having a
% problem locating pterrax eggs. It seems the pterrax have
% been driven from their normal nesting areas by an
% unusually large concentration of giant hornets. The
% pterrans hope the PCs can kill or drive off enough giant
% hornets to reduce the number to a balanced level so that
% the pterrax could return to their nesting area.
% ―The Dark One is a pterran outcast from Pterran
% Vale. An earth cleric, the Dark One was exiled for
% claiming the Great Earthquake and the aftershocks were
% calls from the Earth Mother demanding sacrifices of
% young pterrans. Now a hermit in the wilderness, the Dark
% One believes he has developed direct communication
% with the Earth Mother through a large hole in the ground.
% At the direction of the voice from this hole, the Dark One
% makes sacrifices to the Earth Mother by kidnapping and
% throwing pterrans into the hole. However, the Dark One
% is being manipulated by an earth drake, who poses as the
% Earth Mother to have the Dark One drop food into its lair.
% ―A pack of dune freaks have migrated to an oasis
% near Pterran Vale, posing a hazard to travelers going
% north. The PCs could be asked to clean out the dune
% freaks; however, they are only a lesser evil. The dune
% freaks were forced out of their normal hunting grounds
% by the increased patrols of zik‐trin coming from the Great
% Rift.
% Saragar
% Population: 30,000 (85% humans, 6% elves, 6%
% dwarves, 2% other)
% Exports: Metal weapons, puddingfish cloth, fresh
% water
% Languages: Saragarian
% Separated from the rest of the region by the Burning
% Plains and the Thunder Mountains, the city of Saragar sits
% on the shores of the Last Sea, called Marnita by
% Saragarians. Visitors from the Tablelands would consider
% Saragar a miracle. All of the drudge work performed by
% slaves in other cities is taken care of by the minds of
% ancient criminals trapped forever in obsidian spheres. The
% streets are cleaned, cattle herded, crops tended, garbage
% removed, and water purified by these psionic powered
% spheres.
% The only price the citizens must pay to have all of
% their needs looked after in this way is that they must
% remain happy. The primary law of Saragar is, “Happiness
% must be maintained!”
% Life and Society
% For the most part Saragar maintains a closed self‐
% sufficient society. To visit Saragar is to step back into the
% Green age. People dress in tabards and gowns befitting a
% less savage age. The relatively cooler climate in the vale
% makes such clothing practical and comfortable. There is
% an abundance of metal in Saragar compared to the
% Tablelands, though most of it is ancient and shows signs
% of wear. Some new sources of ore exist in the surrounding
% mountains, but few citizens of Saragar still know how to
% extract it, let alone forge it into new items.
% Someone from the city‐states of the Tyr region might
% consider Saragar to be a paradise. That certainly is the
% perception the Triune Mind Lords try to propagate. They
% generate laws to bolster the illusion of happiness and
% serenity but do nothing to truly address those concerns.
% The lawkeepers enforce these rules. For this reason the
% Saragar dwellers have learned to constantly display
% serene attitudes.
% There are no wizards of any sort in Saragar. Wizardry
% is considered evil and most citizens in Saragar who
% witness it don’t have any idea what they are seeing.
% Psionics are the true power of the domain.
% Government and Politics
% The basic form of Saragar’s government is a triune of
% lawmakers who write the city’s laws, an army of
% lawkeepers to enforce the laws, and a bureaucracy of
% lawtenders to perform the administrative function.
% The trio that make up Saragar’s Triune Mind Lords
% are powerful, ageless masters of psionics. They are Thesik
% (LE male mindlord human, kineticist 29), Barani (NE
% female mindlord human, telepath 28), and Kosveret (CE
% male mindlord elf, nomad 27). The citizens of Saragar
% consider the Mind Lords gods and treat them as such;
% thought there is little interaction between the Mind Lords
% and the populous.
% Senior Lawkeeper Efkenu (LN male human, psychic
% warrior 17) is the only person to have regular contact with
% the Mind Lords. He passes on their edicts and as head of
% the lawkeepers sees that their laws are enforced. Though a
% fair man, Efkenu makes no distinctions between the types
% of offenses and all criminal acts are punished in the same
% manner. The accused is taken to the harmonizers. The
% harmonizers are psions who reach into a subject’s mind to
% sift and shape thoughts back to the track the Mind Lords
% have dictated.
% The lawkeepers are as corrupt as any templar. They
% enforce the laws arbitrarily and to suit their own desires.
% Supervisors rarely leave their offices to check on their
% subordinates and only rebuke subordinates for their
% behavior if it interferes with their own plans.
% The lawtenders perform all of the administrative
% work. They tend to be the most optimistic of people,
% determined that there is no problem that cannot be solved
% with a little determination and positive thinking. While
% they are not corrupt like the lawkeepers, the lawtenders
% are not very good administrators. They insist on only
% performing their duties by the book, and refuse to
% delineate from their guidelines no matter how inefficient
% or incorrect those guidelines are.
% Power Groups
% The Underground: Despite the relative pleasantness
% of Saragar, there are some people who recognize that they
% are living in a society in decay; one that relies on powerful
% immortals for every aspect of their lives. These people
% make up the Underground which has been growing in
% Saragar for the past few hundred years.
% Most members of the Underground are just upset that
% their lives have become more inconvenient as some of the
% obsidian orbs have begun to fail. Others just like having
% someone to complain to without being arrested by the
% lawkeepers.
% A smaller group, who consider themselves the real
% Underground, speak out on street corners against the
% Mind Lords. They are always working on crazed schemes
% such as assassinating the Mind Lords, or destroying all of
% the obsidian orbs, but they lack the power to implement
% any of these plans.
% Important Sites
% The Distillery and the Water Tower: The distillery is
% a psionically powered factory used to transform salt water
% from the Last Sea into fresh water. The water is pumped
% from the distillery into the water tower which is
% connected to a citywide plumbing network that pipes
% fresh water into every building in Saragar.
% The Palace: A massive palace overlooks the city of
% Saragar from a hill east of the city. Unlike the palaces of
% the sorcerer‐kings, the palace of the Mind Lords was built
% more for awe‐inspiring beauty than for defense. The
% security provided by the lawkeepers is lax around the
% palace, as the Mind Lords are confident they could handle
% any intruder.
% Statues of the three Mind Lords stand on a circular
% base at the highest point of the palace. The base slowly
% rotates throughout the day powered by an obsidian
% sphere. The people of Saragar use the statues to tell time,
% as the statues complete a full rotation every hour.
% Major Settlements
% Blufftown (Thorp, 50): This small settlement sits on
% the side of a bluff on an isolated butte in the middle of the
% Last Sea called the Lonely Butte. The lawkeepers
% generally refuse to set foot on the Lonely Butte unless
% directly ordered to do so by the Mind Lords, which makes
% Blufftown a perfect safe haven for the Underground and
% other fugitives from the Mind Lords’ rule.
% The community is little more than a couple of inns
% sitting inside a cave in the side of a cliff. The only way to
% enter the village is to be hauled up in a device consisting
% of a large wicker basket and a series of ropes and pulleys
% powered by an obsidian orb.
% Cubarto (Small Town, 1,500): Cubarto is located on
% the opposite of side of Marnita from Saragar. The people
% of Cubarto are loud and lusty and would not fit in
% Saragar. With the lack of a presence of the lawkeepers,
% most people in Cubarto support the Underground,
% though discreetly. The villagers make their living off of
% fishing and trade coming into their port on its way to
% Kharzden or Sylvandretta. The village is known
% throughout the valley of the Last Sea for throwing a large
% party at the end of the year at which a large public feast is
% held.
% Kharzden (Large Town, 2,000): Kharzden is a
% Dwarven colony scattered through ancient mining shafts
% in the Thunder Mountains. Most of the veins of ore were
% mined out long ago, and most of the metal items the
% dwarves have are ancient. The Dwarven society is
% matriarchal and is lead by Queen Elakta. Her word is law
% and is to be obeyed by all. Queen Elakta refused to have
% much to do with the lawkeepers, and maintains the
% tradition in Kharzden of not calling for help from the
% lawkeepers. The dwarves live underground and grow
% subterranean crops in massive chambers underneath the
% mountains. The dwarves have always doubted the power
% of the Mind Lords to keep the rest of the world at bay,
% and tried to make their community as self sufficient as
% possible.
% Shallat (Hamlet, 300): Shallat is one of a number of
% small fishing villagers on the shores of Marnita. What
% makes the village stand out is the Shallat family who rules
% the villages. Each member of the Shallat family is a skilled
% physician and many are also water clerics. The Shallat
% healers provide their services to anyone in need, no
% matter who they are. The villagers of Shallat are fun‐
% loving people and are generally treated well by everyone
% living on the shores of the Last Sea. Even brigands and
% pirates do not harass the village, as potentially they might
% need the skills of the Shallat healers.
% Sylvandretta (Small Village, 500): The elves of
% Sylvandretta are called “ghost elves” by the people of
% Saragar because of their fair skin and their cold and aloof
% nature. The ghost elves believe that the purity of their
% bloodline must be preserved above all other concerns, and
% isolate themselves from the other races of the Last Sea
% region.
% The secluded settlement of Sylvandretta is located in
% the Spirit Forest nestled within a grove of trees of life. The
% community is run by a council of seven elders, elected by
% the general Elven population.
% Adventure Ideas
% ―Vikus and Mylandus are two merchants who run a
% successful business in Saragar until Mylandus disappears
% with most of the funds from the business. The PCs are
% hired by Vikus to track down his partner. Mylandus has
% discovered some secret that has scared him greatly
% enough that he has fled the city and is trying to leave the
% Last Sea area completely. Unfortunately for him,
% Mylandus has no idea how to survive in the devastated
% environment of the rest of Athas and will not survive long
% if he is able to find a way out of the area of the Last Sea.
% ―Jarsius, a tavern owner in Saragar, has begun to
% have disturbing visions, in which he sees himself
% behaving in random acts of violence. In actuality, the
% visions are memories. Jarsius was an active leader of the
% Underground until he was captured three years ago, but
% his memories were erased. The effect was not perfect and
% now some of his suppressed memories are returning.
% Members of the Underground still watch Jarsius, to see if
% he remembers what happened to him. For Jarsius’s mind
% was not destroyed by the lawkeepers but by members of
% the Underground who acted to mindwipe him to protect
% their identities.
% ―The lawkeepers based at South Pass discover the
% tracks of a large beast they have never encountered
% before. The PCs, as outlanders who may have seen such a
% beast before, are drafted to help track down the beast.
% ―A wealthy Saragarian wants to see what the world
% is like outside of the Last Sea. He hires the PCs to get him
% through the Border of Guardians.
% ―Because of the ragged appearance of most
% outlanders, the PCs are mistaken for druids by a small
% fishing community. The villagers ask the PCs for help
% with a school of sharks that is making fishing difficult in
% the area.
% ―A man is found beaten to death. His face was so
% badly beaten that the only way to identify him was by a
% letter found in his pocket. The letter was addressed from
% one of the PCs, and the lawkeepers wish to talk to the PC
% to see how he was involved with the murdered man. The
% PC has never heard of the man and has no idea why the
% dead man had the PCs name in his pocket.
% Thamasku
% Population: 12,000 (99% rhul‐thaun, 1% other)
% Exports: Life‐shapes, fish
% Languages: Rhul‐thaun
% The ancient rhul‐thaun city of Thamasku sits next to
% Ghavin Lake at the top of the Jagged Cliffs. The city is
% surrounded by a forest of hardwood trees. Like all rhul‐
% thaun communities, the buildings are constructed of
% organically grown material. The architecture focuses on
% the vertical, with most buildings having many storeys.
% There is one difference from other rhul‐thaun settlements.
% Because the city is not cramped onto a ledge of the Jagged
% Cliffs, the buildings of Thamasku are not crowded
% together allowing for wider streets and a more open feel
% for the city.
% Life and Society
% Rhul‐thaun society is highly ritualized. Each aspect of
% their lives has a ritual attached to it, and throughout the
% day the rhul‐thaun perform various rituals. Simple rituals
% from the greeting ritual to the payment ritual, to the
% before meal ritual last less than a minute, while more
% complex rituals such as those for legal procedures may
% last hours. Often the ritual is just as or more important
% than the associated action it is attached to, and if one of
% the participants makes a mistake the entire ritual is begun
% again. The rituals are more akin to superstitions than to a
% religious devotion, and allow a rhul‐thaun to feel he has
% some control over the chaotic forces that rule his life.
% As the center of the rhul‐thaun society, Thamasku has
% a diverse population. The wealthiest rhul‐thaun live side
% by side with the poorest of the cliff dwellers. The citizens
% of Thamasku are some of the few individuals on Athas
% who do not have to struggle daily to survive. Life‐shaped
% devices provide a vast array of conveniences and basic
% needs, from nourishment to waste disposal. Most homes
% have indoor plumbing, operated by life‐shaped engines
% that pump water from the lake.
% Because they do not struggle daily for survival, the
% rhul‐thaun of Thamasku have developed a rich culture of
% the arts and entertainment. Dance halls, theaters, art
% galleries, and auditoriums are numerous throughout the
% city, with many located in the Art Quarter on the city’s
% eastern side.
% Government and Politics
% The rhul‐thaun of Thamasku are divided into 28
% different clans. The clan leaders are called Har‐etuil. The
% Har‐etuil act as judges for matters within their clans.
% Disputes between clans are settled by a council of Har‐
% etuil. The collective of Har‐etuil appoints the city
% administrator.
% Currently, Vher‐asach (LN female rhul‐thaun, rogue
% 10) holds the title of city administrator, since she inherited
% the position from her mother. She has proved herself a
% capable administrator and most expect she will remain in
% the position for the time being.
% Power Groups
% Ban‐ghesh: A guild of thieves, assassins, and hired
% thugs, the ban‐ghesh runs the criminal activities in
% Thamasku. Extortion is the guild’s main source of income,
% though their activities include burglary, smuggling, and
% gambling. There is little to challenge the ban‐ghesh as
% they have a network of corrupt lawkeepers and
% administrative officials protecting their organization. The
% ban‐ghesh also enters into legitimate business with
% merchants, providing financial support in exchange for a
% percentage of the merchant’s profits.
% Chahn: The Chahn is a revolutionary organization
% which does not hesitate to use violence to achieve their
% goals. Their goal is the complete overthrow of rhul‐thaun
% society. The Chahn are against almost every tradition in
% rhul‐thaun society from clan‐rule, the mastery of the life‐
% shapers, to the daily rituals that dominate rhul‐thaun life.
% The lawkeepers have branded them a terrorist group and
% most rhul‐thaun live in fear of them.
% Life‐Shapers: The life‐shapers are a secret society that
% holds the knowledge of life‐shaped creations. They hold a
% place of reverence by the rhul‐thaun as the entire society
% is based on their works. Through the study of life‐shaping
% the life‐shapers feel a strong connection with the past back
% to the rhulisti, the inventors of life‐shaping. The life‐
% shapers feel they are superior to the rest of the rhul‐thaun,
% because of this connection.
% The life‐shapers guard their knowledge protectively,
% letting no one outside of their order learn their secrets.
% Because they control the creation of all life‐shaped items
% the life‐shapers can exert control over the all of the rhul‐
% thaun, forcing the Har‐etuil to listen to the life‐shapers’
% opinions strongly.
% The life‐shapers are led by Loi Far‐oneth (LG male
% rhul‐thaun, bard 7/life‐shaper 5) and his chief lieutenant,
% Gil‐ogres (LE male rhul‐thaun, rogue 5/graftwarrior 7),
% who reside at the Sanctuary in Thamasku. Each rhul‐
% thaun settlement has a life‐shaper sanctuary with a head
% life‐shaper who reports directly to Loi Far‐oneth and his
% lieutenant.
% Windriders: Windrider is the name given to the rhul‐
% thaun who dare to fly on the backs of various life‐shaped
% creatures through the high winds and mist that plague the
% Jagged Cliff. Traveling between the rhul‐thaun
% communities, they transport messages and merchant
% goods, allowing trade and communication between all
% their settlements. Windrider is the most glamorous
% position in rhul‐thaun society. Though there is a
% windriders guild, it maintains a loose organization with
% little structure or hierarchy. The windriders typically
% work in small independent groups of 2 to 8 windriders.
% Important Sites
% Air Temple: The air temple is located in one of the
% tallest of the city’s spires. A dozen clerics of air staff the
% temple, but they have little interaction with most of the
% city’s population. There are few devout followers in the
% city, and the clerics take no interest in politics. The air
% clerics do interact with the windriders regularly. In fact
% they have turned their temple into a safehome for
% windriders, where they can receive free food, a free room
% and lodging for their mounts. The clerics look on
% windriding as the ideal way to commune with the air
% spirits and treat the windriders as holy men.
% Aviary: The tall tower known as the aviary is home to
% hundreds of birds that fly about the city. The eclectic rhul‐
% thaun known only as the Birdmaster cares for and
% watches over the birds. The tower is large enough for
% large flying creatures to roost there, and the Birdmaster
% allows windriders to stable their mounts at the aviary for
% free when in the city.
% Conclave: The Conclave is the meeting hall of the Har‐
% etuil. It is a grand structure that sees little use when the
% Har‐etuil council is not in session.
% The Sanctuary: The Sanctuary is the headquarters and
% workplace of the life‐shapers in Thamasku. The
% mushroom shaped structure is located some 300 feet
% below the surface of Ghavin Lake. This masterpiece of
% life‐shaping technology maintains fresh air within the
% structure by extracting it from the surrounding water
% through a complex gill‐system. Over 150 life‐shapers
% work at the Sanctuary creating and maintaining the life‐
% shaped items that are used throughout Thamasku.
% Major Settlements
% Sol‐fehn (Hamlet, 300): Sol‐fehn is a small village
% located at the top of a waterfall created by a river flowing
% from Ghavin Lake. The village serves as a hub for goods
% and rhul‐thaun leaving Thamasku for the rest of the
% settlements scattered across the Jagged Cliffs. Almost all
% of the villagers make their living through transportation.
% The villagers are members of two clans that are centered
% in the city of Thamasku, so there is no Har‐etuil in the
% village. An administrator appointed by the administrator
% of Thamasku runs the city. The current administrator is
% Rath‐omak (LN male rhul‐thaun, fighter 5).
% Adventure Ideas
% ―No one has seen the Birdmaster of Thamasku for
% many weeks, but because of his reclusive nature, few
% citizens have realized this. So why are some of his birds
% following the PCs everywhere they go? Are the birds
% trying to send the PCs a message?
% ―The Ban‐ghesh claim they have managed to
% infiltrate the Sanctuary of Thamasku and steal a
% wonderful new type of life‐shaped item. There are rumors
% that the Ban‐ghesh plan to sell this new item to the Chahn.
% The life‐masters hire the PCs to recover the stolen item.
% The life‐masters are unconcerned with the item falling
% into the hands of the Chahn, because there is really
% nothing extraordinary about the item. It is simply an
% unfinished common life‐shaped tool. The life‐masters are
% more concerned about the slight of someone stealing from
% the Sanctuary.
% ―A high‐level reggelid believes that the secrets of
% Rajaat may be hidden in Thamasku. He has charmed a
% rhul‐thaun climber named Bal‐orean, and sent him to the
% rhul‐thaun capital. The reggelid seeks to create a web of
% charmed agents throughout the city, and so has Bal‐orean
% lure other rhul‐thaun to a secluded part of the Jagged
% Cliffs where the reggelid has its lair. Once there the victim
% is charmed and sent back to Thamasku, as the reggelid
% attempts to create a spy network in Thamasku to root out
% the location of Rajaat’s secrets.
% ―A wealthy benefactor hires the PCs to accompany
% him on a journey down the Jagged Cliffs to the Crimson
% Savanna to recover a long lost life‐shaped artifact hidden
% on the Savanna. Their employer is not who he claims to be
% however. Taen‐ofuth is really the high priest of the
% forbidden temple of fire in Thamasku. Frustrated with his
% lack of opportunity to demonstrate his devotion to his
% element, Taen‐ofuth desires to travel to the Crimson
% Savanna to set a massive brush fire. He seeks to revel in
% the fire’s destruction but also hopes to impress other rhul‐
% thaun into seeing the benefits of devotion to the element
% of fire.
% ―Two weeks ago, a windrider arrived at the air
% temple and went immediately into a private meeting with
% Thim‐obec, high priest of the temple. Two days later,
% Thim‐obec left Thamasku with the windrider and a high
% ranking life‐master on the windrider’s gon‐evauth. They
% have not been seen since, and the temple of air is seeking
% adventurers to find the high priest.
% ―Ghoun‐awir is famous windancer, known
% throughout Thamasku for her daring performances. After
% a recent performance, a wealthy citizen claimed that
% Ghoun‐awir is a thief and burglarized his home during
% her performance. The citizen’s home was used as part of
% the performance. Ghoun‐awir proclaimed her innocence
% before the lawkeepers arrived, but when they tried to
% arrest her she escaped. Since windancers wear face paint
% and costumes, no one is sure what Ghoun‐awir really
% looks like. PCs could be hired to track Ghoun‐awir down,
% or Ghoun‐awir could approach the PCs to help prove her
% innocence by finding the true criminal.
% Winter Nest
% Population: 650 (100% aarakocra)
% Exports: Ice, feathers
% Languages: Auran, Kurnan
% The village of Winter Nest is located in the frozen
% peaks of the White Mountains. It is the home of a civilized
% tribe of aarakocra.
% The unusual buildings of Winter Nest are formed
% from a mixture of ice, stone, and shaped bricks. To new
% visitors the village looks like a cluster of towers, giving
% the appearance that the mountain peak has a crown.
% There are no roads in Winter Nest and very few
% connecting walkways between the buildings, as the
% aarakocra fly rather than walk. Doorways appear all along
% the face of the buildings, though most are clustered near
% the top of each tower. Landing platforms and resting
% perches decorate the outsides of most building. Each
% tower is topped with a large rounded structure. Most of
% these sphere‐shaped constructs are communal areas,
% though the highest are the personal quarters of the leaders
% of Winter Nest.
% Life and Society
% The aarakocra of Winter Nest called themselves
% “silvaarak,” which means “people of the silver wing.”
% They are perceptive, and have great confidence and pride
% in themselves. This translates into arrogance at times,
% because the silvaarak believe that their ability to fly makes
% them superior to all other races. Though they often
% express sympathy for people unable to fly, this more often
% comes across as condescending.
% The aarakocra have had a difficult time forming
% friendly relations with others over the years. Only in Kurn
% have they made dedicated friends. Traders from Winter
% Nest visit the city‐state of Kurn a few times each year for
% trade. Other attempts to make contact with other
% communities have meet with failure. Either due to the
% hostility of the natives such as in Eldaarich and the Bandit
% States, or the silvaarak’s condescending nature towards
% other races.
% Government and Politics
% Winter Nest is lead by Traaka (LG female aarakocra,
% air cleric 5/elementalist 2) a female aarakocra of many
% years. Traditionally, the aarakocra are isolationists, and
% Traaka supports this policy. The isolationist policy was
% adopted years ago after bad experiences with Eldaarich
% and later with the peoples of the Bandit States. The policy
% has kept the village safe over the years and most of the
% silvaarak want to see it continue.
% However, many of the younger generation of bird‐
% people desire to explore the world beyond the White
% Mountains. They have been vocal in their wish to explore
% and make contact with other civilizations, believing they
% will not experience such bad receptions as those the
% aarakocra received in Eldaarich or the Bandit States.
% Pointing to Kurn, these young bloods believe there is
% opportunity for the silvaarak in positive relationships
% with outsiders.
% Traaka understands the young aarakocra’s desires, but
% wishes to maintain the status quo for the protection of the
% village. She is trying to develop a middle path that would
% allow some exploration without making the location of
% the village well known to its enemies.
% Power Groups
% Air Clerics: Winter Nest is ruled by clerics of Air and
% Ice drawn from the leading aarakocra families. The clerics
% meet in a large hall in Winter Nest to discuss community
% issues; when there is a particularly contentious debate, the
% priests adjourn to the very summit of a nearby mountain
% overlooking the village. There, perched on the ice and
% surrounded by the sky, the priests of the two faiths pray
% for guidance together.
% Important Sites
% Air Temple: The Air Temple is the grandest structure
% in the village. The temple is built like a huge brazier, with
% four legs made of massive evergreen tree trunks dragged
% up from the foothills centuries ago. These tree boles, each
% more than 100 feet long, are set in the icy ground and
% canted to nearly join at the tops. There is a concave plate
% of ice, 20 feet in diameter, held up between the four posts
% with a hole 8 feet in diameter cut in its center. Priests of
% Air preach from the center of the bowl, while congregants
% gather on the rim of the bowl and on the perches placed at
% intervals along the legs.
% Ice Temple: Smaller only to the Air Temple, the Ice
% Temple (which is basically another word for water at such
% high altitudes most of the year) is built of large sheets of
% translucent white and blue ice, layered upon one another
% to create a five‐sided pyramid more than 40 feet tall. The
% interior is sunken below ground level dug into the glacier
% so all the worshippers are surrounded by primordial ice
% throughout the services. Fresh plates of ice are added to
% the temple throughout the High Sun.
% Adventure Ideas
% ―Few in Winter Nest took much notice when a roc
% landed on a perch overlooking the village. Two days later
% the roc has been joined by a dozen more of his kind. The
% large birds rarely move from their perches, but their
% menacing presence is unnerving the aarakocra of Winter
% Nest.
% ―The wind patterns around Winter Nest have
% changed drastically. A dangerous downdraft has
% developed making any attempt at flying from the village
% fraught with peril. Town elders are puzzled by this
% sudden change, and have forbidden all but the strongest,
% most agile fliers from leaving Winter Nest. Air clerics are
% calling for a sacrifice to appease the air spirits, but the
% town elders want to understand what is going on before
% they decide, and the local druid who communes with the
% spirit of the land has disappeared.
% ―The aarakocra of Winter Nest tell tales of a wise old
% aviarag named Vocia that lives in a cave near the base of
% the White Mountains. The noble beast has not been seen
% for three years. Templars from Eldaarich, intent on
% plundering Vocia’s lair, have been spotted approaching
% the cave. Traaka needs volunteers to warn Vocia.
% Unfortunately Vocia has passed away due to old age,
% leaving the PCs to defend her cave, as well as her remains,
% which the templars wish to plunder.
% ―A defiler has polymorphed himself into an
% aarakocra and infiltrated Winter Nest, seeking to gain
% some of the knowledge from the preservers of Winter
% Nest. His defiling is having an adverse affect on the ice
% sculpted portions of Winter Nest’s buildings. If he is not
% unmasked soon, one or more buildings in the community
% may collapse.
% ―Some of the more adventuresome young aarakocra
% enjoy a deadly challenge. They know of a lair of an air
% drake on one of the other mountains in the White
% Mountain range. To show their bravery they occasionally
% sneak into the lair and come back with a scale or other
% souvenir. The act is not as dangerous as it sounds, since
% the aarakocra know the air drake’s migration pattern and
% typically know when it is not in this particular lair. Some
% of these youths could challenge the PCs to try this stunt,
% but unfortunately for the PCs the air drake has returned
% to the lair earlier than expected in order to lay eggs.
% ―A heavily armed Tsalaxan caravan has arrived at
% the foot of the White Mountains from Draj. The Tsalaxans
% seem to be trying to reach Winter Nest but the steep
% mountain sloop prevents them from approaching from
% below. They have not given up and continue to search for
% some path up the mountain to Winter Nest. The aarakocra
% believe the Tsalaxans are raiders and wish to avoid them.
% The Winter Nesters know their village cannot be reached
% except through the air, and are not concerned that the
% Tsalaxans will be able to reach the village. However,
% Traaka wishes to determine the caravan master’s true
% intentions in case the aarakocra are mistaken. PC allies of
% the aarakocra could infiltrate the caravan while not
% obviously tying directly back to the aarakocra.