\section{Psionic Feats}

\Feat[Psionic]{Aligned Attack}
{Your melee or ranged attack overcomes your opponent's alignment-based damage reduction and deals additional damage.}
{Base attack bonus +6.}
{When you take this feat, choose either chaos, good, evil or law. Your choice must match one of your alignment components. Once you've made this alignment choice, it cannot be changed.

To use this feat, you must expend your psionic focus. When you make a successful melee or ranged attack, you deal an extra 1d6 points of damage, and your attack is treated as either a good, evil, chaotic, or lawful attack (depending on your original choice) for the purpose of overcoming damage reduction.

You must decide whether or not to use this feat prior to making an attack. If your attack misses, you still expend your psionic focus.}
{}{}

\Feat[Psionic]{Body Fuel}
{You can expand your power point total at the expense of your health.}
{}
{You can recover 2 power points by taking 1 point of ability burn damage to each of your three ability scores: Strength, Dexterity, and Constitution.

You can recover additional power points for a proportional cost to Strength, Dexterity, and Constitution. These recovered points are added to your power point reserve as if you had gained them by resting overnight.}
{}
{Only living creatures can use this feat. You can take advantage of this feat only while in your own body.}

\Feat[Psionic]{Boost Construct}
{Your astral constructs have more abilities.}
{}
{When you create an astral construct, you can give it one additional special ability from any menu that the construct currently has an ability from.}
{}{}

\Feat[Psionic]{Combat Manifestation}
{You are adept at manifesting powers in combat.}
{}
{You get a +4 bonus on \skill{Concentration} checks made to manifest a power or use a psi-like ability while on the defensive or while you are grappling or pinned.}
{}{}

\Feat[Psionic]{Deep Impact}
{You can strike your foe with a melee weapon as if making a touch attack.}
{Str 13, \feat{Psionic Weapon}, base attack bonus +5.}
{To use this feat, you must expend your psionic focus. You can resolve your attack with a melee weapon as a touch attack. You must decide whether or not to use this feat prior to making an attack. If your attack misses, you still expend your psionic focus.}
{}{}

\Feat[Psionic]{Elemental Manifestation}
{Your patron element aids you in your energy manifestations.}
{Access to domain spells, manifester level 3rd.}
{To use this feat, you must expend your psionic focus. You add 2 to the save DC of a power you manifest if that power has the same descriptor as your patron element.}
{}{}

\Feat[Psionic]{Expanded Knowledge}
{You learn another power.}
{Manifester level 3rd.}
{Add to your powers known one additional power of any level up to one level lower than the highest-level power you can manifest. You can choose any power, including powers from another discipline's list or even from another class's list.}
{}
{You can gain this feat multiple times. Each time, you learn one new power at any level up to one less than the highest-level power you can manifest.}

\Feat[Psionic]{Fell Shot}
{You can strike your foe with a ranged weapon as if making a touch attack.}
{Dex 13, \feat{Point Blank Shot}, \feat{Psionic Shot}, base attack bonus +5.}
{To use this feat, you must expend your psionic focus. You can resolve your ranged attack as a ranged touch attack.

You must decide whether or not to use this feat prior to making an attack. If your attack misses, you still expend your psionic focus.}
{}{}

\Feat[Psionic]{Focused Mind}
{Your meditations strengthen your reasoning.}
{Int 13.}
{As long as you are psionically focused, you receive a +2 bonus on \skill{Appraise}, \skill{Decipher Script} and \skill{Search} checks.}
{}{}

\Feat[Psionic]{Focused Sunder}
{You can sense the stress points on others' weapons.}
{Str 13, \feat{Power Attack}, \feat{Improved Sunder}.}
{To use this feat, you must expend your psionic focus.

When you strike at an opponent's weapon, you ignore half of the weapon's total hardness (round down). Total hardness includes any magical or psionic enhancements possessed by the weapon that increase its hardness.}
{}
{You can also sense the stress points in any hard construction, such as wooden doors or stone walls, and can ignore half of the object's total hardness (round down) when attacking that object.}

\Feat[Psionic]{Ghost Attack}
{Your deadly strikes against incorporeal foes always find their mark.}
{Base attack bonus +3.}
{You must be psionically focused to use this feat. When you make a melee attack or a ranged attack against an incorporeal creature, you can make two rolls to check for the miss chance. If either is successful, the attack is treated as if it were made with a ghost touch weapon for the purpose of affecting the creature. Your weapon or natural weapon actually appears to become briefly incorporeal as the attack is made.}
{}{}

\Feat[Psionic]{Greater Power Penetration}
{Your powers are especially potent at breaking through power resistance.}
{\feat{Power Penetration}.}
{To use this feat, you must expend your psionic focus. You get a +8 bonus on manifester level checks to overcome a creature's power resistance. This bonus overlaps with the bonus from \feat{Power Penetration}.}
{}{}

\Feat[Psionic]{Greater Power Specialization}
{You deal more damage with your powers.}
{\feat{Power Specialization}, \feat{Weapon Focus} (ray), manifester level 12th.}
{Your powers that deal damage deal an extra 2 points of damage. This damage stacks with other bonuses on damage rolls to powers, including the one from \feat{Power Specialization}. The damage bonus applies only if the target or targets are within 30 feet.}
{}{}

\Feat[Psionic]{Greater Psionic Endowment}
{You can use meditation to focus your powers.}
{\feat{Psionic Endowment}.}
{When you use the \feat{Psionic Endowment} feat, you add +2 to the save DC of a power you manifest instead of +1.}
{}{}

\Feat[Psionic]{Greater Psionic Fist}
{You can charge your unarmed strike or natural weapon with additional damage potential.}
{Str 13, \feat{Psionic Fist}, base attack bonus +5.}
{When you use the \feat{Psionic Fist} feat, your unarmed attack or attack with a natural weapon deals an extra 4d6 points of damage instead of an extra 2d6 points.}
{}{}

\Feat[Psionic]{Greater Psionic Shot}
{You can charge your ranged attacks with additional damage potential.}
{\feat{Point Blank Shot}, \feat{Psionic Shot}, base attack bonus +5.}
{When you use the \feat{Psionic Shot} feat, your ranged attack deals an extra 4d6 points of damage instead of an extra 2d6 points.}
{}{}

\Feat[Psionic]{Greater Psionic Weapon}
{You can charge your melee weapon with additional damage potential.}
{Str 13, \feat{Psionic Weapon}, base attack bonus +5.}
{When you use the \feat{Psionic Weapon} feat, your attack with a melee weapon deals an extra 4d6 points of damage instead of an extra 2d6 points.}
{}{}

\Feat[Psionic]{Improved Dwarven Focus}
{You can use the Way to help fulfill your focus.}
{Dwarf.}
{You must be psionically focused to use this feat. While actively pursuing your dwarven focus, you receive a +2 morale bonus on all checks related to your focus.}
{You receive a +1 morale bonus on checks related to completing your focus.}
{}

\Feat[Psionic]{Improved Elf Run}
{You can use the Way to run faster.}
{Elf.}
{You must be psionically focused to use this feat. While in an elf run state, you gain an insight bonus to your speed of 15 feet.}
{}{}

\Feat[Psionic]{Improved Psicrystal}
{You can upgrade your psicrystal.}
{\feat{Psicrystal Affinity}.}
{You can implant another personality fragment in your psicrystal. You gain the benefits of both psicrystal personalities. Your psicrystal's personality adjusts and becomes a blend between all implanted personality fragments. From now on, when determining the abilities of your psicrystal, treat your manifester level as one higher than your normal manifester level.}
{}
{You can gain this feat multiple times. Each time, you implant a new personality fragment in your psicrystal, from which you derive the noted benefits, and you treat your level as one higher for the purpose of determining your psicrystal's abilities.}

\Feat[Psionic]{Inquisitor}
{You know when others lie.}
{Wis 13.}
{To use this feat, you must expend your psionic focus.

You gain a +10 bonus on a \skill{Sense Motive} check to oppose a \skill{Bluff} check.

You must decide whether or not to use this feat prior to making a \skill{Sense Motive} check. If your check fails, or if the opponent isn't lying, you still expend your psionic focus.}
{}{}

\Feat[Psionic]{Jump Charge}
{You can charge an opponent by jumping at them, hitting the enemy with a powerful attack.}
{\feat{Psionic Fist} or \feat{Psionic Weapon}, \skill{Jump} 8 ranks.}
{To use this feat, you must expend your psionic focus. When charging an opponent, you may jump at them as part of the movement. Make a \skill{Jump} check. If your horizontal jump is at least 10 feet and you end your jump in a square in which you may threaten the opponent, you may increase by one-half the damage dealt by your \feat{Psionic Weapon} or \feat{Psionic Fist}. While using a two-handed weapon, damage is doubled instead. This attack must follow all the rules for charging and the \skill{Jump} skill, with the exception that you ignore the ground terrain in any spaces you jump over.}
{}{}

\Feat[Psionic]{Mental Leap}
{You can make amazing jumps.}
{Str 13, \skill{Jump} 5 ranks.}
{To use this feat, you must expend your psionic focus. You gain a +10 bonus on a \skill{Jump} check.}
{}{}

\Feat[Psionic]{Metamorphic Transfer}
{You can gain a supernatural ability of a metamorphed form.}
{Wis 13, manifester level 5th.}
{Each time you change your form, such as through the \psionic{metamorphosis} power, you gain one of the new form's supernatural abilities, if it has any.

You gain only three uses of the metamorphic ability per day, even if the creature into which you metamorph has a higher limit on uses (you are still subject to other restrictions on the use of the ability.) The save DC to resist a supernatural ability gained through Metamorphic Transfer (if it is an attack) is 10 + your Cha modifier + \onehalf your Hit Dice. No matter how many times you manifest the \psionic{metamorphosis} power on a given day, you can gain only a total of three supernatural ability transfers per day.}
{You cannot use the supernatural abilities of creatures whose form you assume.}
{You can gain this feat multiple times. Each time, you can gain one additional supernatural ability.}

\Feat[Psionic]{Narrow Mind}
{Your ability to concentrate is as keen as an arrowhead, allowing you to gain your psionic focus even in the most turbulent situations.}
{Wis 13.}
{You gain a +4 bonus on \skill{Concentration} checks you make to become psionically focused.}
{}{}

\Feat[Psionic]{Overchannel}
{You burn your life force to strengthen your powers.}
{}
{While manifesting a power, you can increase your effective manifester level by one, but in so doing you take 1d8 points of damage. At 8th level, you can choose to increase your effective manifester level by two, but you take 3d8 points of damage. At 15th level, you can increase your effective manifester level by three, but you take 5d8 points of damage.

The effective increase in manifester level increases the number of power points you can expend on a single power manifestation, as well as increasing all manifester level-dependent effects, such as range, duration, and overcoming power resistance.}
{Your manifester level is equal to your total levels in classes that manifest powers.}
{}

\Feat[Psionic]{Power Penetration}
{Your powers are especially potent, breaking through power resistance more readily than normal.}
{}
{To use this feat, you must expend your psionic focus. You get a +4 bonus on manifester level checks made to overcome a creature's power resistance.}
{}{}

\Feat[Psionic]{Power Specialization}
{You deal more damage with your powers.}
{\feat{Weapon Focus} (ray), manifester level 4th.}
{With rays and ranged touch attack powers that deal damage, you deal an extra 2 points of damage. If you expend your psionic focus when you manifest a ray or a ranged touch attack power that deals damage, you add your key ability bonus to the damage (instead of adding 2).}
{}{}

\Feat[Psionic]{Psicrystal Affinity}
{You have created a psicrystal.}
{Manifester level 1st.}
{This feat allows you to gain a psicrystal.}
{}{}

\Feat[Psionic]{Psicrystal Containment}
{Your psicrystal has advanced enough that it can hold a psionic focus that you store within it.}
{\feat{Psicrystal Affinity}, manifester level 3rd.}
{You can spend a full-round action attempting to psionically focus your psicrystal. At any time when you need to expend your psionic focus, you can expend your psicrystal's psionic focus instead, as long as the crystal is within 5 feet of you. Psionically focusing your psicrystal works just like focusing yourself. The psicrystal cannot focus itself---only the owner can spend the time to focus the crystal.}
{}{}

\Feat[Psionic]{Psionic Body}
{Your mind reinforces your body.}
{}
{When you take this feat, you gain 2 hit points for each psionic feat you have (including this one). Whenever you take a new psionic feat, you gain 2 more hit points.}
{}{}

\Feat[Psionic]{Psionic Charge}
{You can charge in a crooked line.}
{Dex 13, \feat{Speed of Thought}.}
{To use this feat, you must expend your psionic focus. When you charge, you can make one turn of up to 90 degrees during your movement. All other restrictions on charges still apply; for instance, you cannot pass through a square that blocks or slows movement, or that contains a creature. You must have line of sight to the opponent at the start of your turn.}
{}{}

\Feat[Psionic]{Psionic Dodge}
{You are proficient at dodging blows.}
{Dex 13, \feat{Dodge}.}
{You must be psionically focused to use this feat. You receive a +1 dodge bonus to your Armor Class. This bonus stacks with the bonus from the \feat{Dodge} feat (but only applies on attacks made by the opponent you have designated).}
{}{}

\Feat[Psionic]{Psionic Endowment}
{You can endow your manifestations with more concentrated focus.}
{}
{To use this feat, you must expend your psionic focus. You add 1 to the save DC of a power you manifest.}
{}{}

\Feat[Psionic]{Psionic Fist}
{You can charge your unarmed strike or natural weapon with additional damage potential.}
{Str 13.}
{To use this feat, you must expend your psionic focus. Your unarmed strike or attack with a natural weapon deals an extra 2d6 points of damage.

You must decide whether or not to use this feat prior to making an attack. If your attack misses, you still expend your psionic focus.}
{}{}

\Feat[Psionic]{Psionic Meditation}
{You can focus your mind faster than normal, even under duress.}
{Wis 13, \skill{Concentration} 7 ranks.}
{You can take a move action to become psionically focused.}
{A character without this feat must take a full-round action to become psionically focused.}
{}

\Feat[Psionic]{Psionic Shot}
{You can charge your ranged attacks with additional damage potential.}
{\feat{Point Blank Shot}.}
{To use this feat, you must expend your psionic focus. Your ranged attack deals +2d6 points of damage. You must decide whether or not to use this feat prior to making an attack. If your attack misses, you still expend your psionic focus.}
{}{}

\Feat[Psionic]{Psionic Talent}
{You gain additional power points to supplement those you already had.}
{Having a power point reserve.}
{When you take this feat for the first time, you gain 2 power points.}
{}
{You can take this feat multiple times. Each time you take the feat after the first time, the number of power points you gain increases by 1.}

\Feat[Psionic]{Psionic Weapon}
{You can charge your melee weapon with additional damage potential.}
{Str 13.}
{To use this feat, you must expend your psionic focus.

Your attack with a melee weapon deals an extra 2d6 points of damage. You must decide whether or not to use this feat prior to making an attack. If your attack misses, you still expend your psionic focus.}
{}{}

\Feat[Psionic]{Pterran Telepathy}
{You can leverage your missive psi-like ability to communicate with other creatures.}
{Pterran, missive psi-like ability.}
{You can use your missive ability to communicate with all humanoid creatures in addition to reptiles. Your manifester level for this effect is equal to \onehalf your Hit Dice (minimum 1st).}
{}{}

\Feat[Psionic]{Return Shot}
{You can return incoming arrows, as well as crossbow bolts, spears, and other projectile or thrown weapons.}
{\feat{Point Blank Shot}, \feat{Psionic Shot}, \feat{Fell Shot}, base attack bonus +3.}
{To use this feat, you must expend your psionic focus and have at least one hand free. Once per round when you would normally be hit by a projectile or a thrown weapon no more than one size category larger than your size, you can deflect the attack so that you take no damage from it. The attack is deflected back at your attacker, using the attack bonus of the original attack on you. You must be aware of the attack and not flat-footed. Attempting to return a shot is a free action.}
{}
{If you also have the \feat{Deflect Arrows} feat, the deflected attack is made with the original attack bonus plus your Dexterity bonus.}

\Feat[Psionic]{Speed of Thought}
{The energy of your mind energizes the alacrity of your body.}
{Wis 13.}
{As long as you are psionically focused and not wearing heavy armor, you gain an insight bonus to your speed of 10 feet.}
{}{}

\Feat[Psionic]{Talented}
{You can overchannel powers with less cost to yourself.}
{\feat{Overchannel}.}
{To use this feat, you must expend your psionic focus. When manifesting a power of 3rd level or lower, you do not take damage from overchanneling.}
{}{}

\Feat[Psionic]{Unavoidable Strike}
{You can make an unarmed strike or use a natural weapon against your foe as if delivering a touch attack.}
{Str 13, \feat{Psionic Fist}, base attack bonus +5.}
{To use this feat, you must expend your psionic focus. You can resolve your unarmed strike or attack with a natural weapon as a touch attack.

You must decide whether or not to use this feat prior to making an attack. If your attack misses, you still expend your psionic focus.}
{}{}

\Feat[Psionic]{Up The Walls}
{You can run on walls for brief distances.}
{Wis 13.}
{While you are psionically focused, you can take part of one of your move actions to traverse a wall or other relatively smooth vertical surface if you begin and end your move on a horizontal surface. The height you can achieve on the wall is limited only by this movement restriction. If you do not end your move on a horizontal surface, you fall prone, taking falling damage as appropriate for your distance above the ground. Treat the wall as a normal floor for the purpose of measuring your movement. Passing from floor to wall or wall to floor costs no movement; you can change surfaces freely. Opponents on the ground can make attacks of opportunity as you move up the wall.}
{}
{You can take other move actions in conjunction with moving along a wall. For instance, the \feat{Spring Attack} feat allows you to make an attack from the wall against a foe standing on the ground who is within the area you threaten; however, if you are somehow prevented from completing your move, you fall. Likewise, you could tumble along the wall to avoid attacks of opportunity.}

\Feat[Psionic]{Wounding Attack}
{Your vicious attacks wound your foe.}
{Base attack bonus +8.}
{To use this feat, you must expend your psionic focus. You can make an attack with such vicious force that you wound your opponent. A wound deals 1 point of Constitution damage to your foe in addition to the usual damage dealt.

You must decide whether or not to use this feat prior to making an attack. If your attack misses, you still expend your psionic focus.}
{}{}
