\section{Using Skills}
When your character uses a skill, you make a skill check to see how well he or she does. The higher the result of the skill check, the better. Based on the circumstances, your result must match or beat a particular number (a DC or the result of an opposed skill check) for the check to be successful. The harder the task, the higher the number you need to roll.

Circumstances can affect your check. A character who is free to work without distractions can make a careful attempt and avoid simple mistakes. A character who has lots of time can try over and over again, thereby assuring the best outcome. If others help, the character may succeed where otherwise he or she would fail.

\subsection{Skill Checks}
A skill check takes into account a character’s training (skill rank), natural talent (ability modifier), and luck (the die roll). It may also take into account his or her race’s knack for doing certain things (racial bonus) or what armor he or she is wearing (armor check penalty), or a certain feat the character possesses, among other things.

To make a skill check, roll 1d20 and add your character’s skill modifier for that skill. The skill modifier incorporates the character’s ranks in that skill and the ability modifier for that skill’s key ability, plus any other miscellaneous modifiers that may apply, including racial bonuses and armor check penalties. The higher the result, the better. Unlike with attack rolls and saving throws, a natural roll of 20 on the d20 is not an automatic success, and a natural roll of 1 is not an automatic failure.

\subsubsection{Difficulty Class}
Some checks are made against a Difficulty Class (DC). The DC is a number (set using the skill rules as a guideline) that you must score as a result on your skill check in order to succeed.

\Table{Difficulty Class Examples}{l X}{
\tableheader Difficulty (DC) & \tableheader Example (Skill Used)\\
Very easy (0) & Notice something large in plain sight (Spot)\\
Easy (5) & Climb a knotted rope (Climb)\\
Average (10) & Hear an approaching guard (Listen)\\
Tough (15) & Rig a wagon wheel to fall off (Disable Device)\\
Challenging (20) & Swim in stormy water (Swim)\\
Formidable (25) & Open an average lock (Open Lock)\\
Heroic (30) & Leap across a 30-foot chasm (Jump)\\
Nearly impossible (40) & Track a squad of orcs across hard ground after 24 hours of rainfall (Survival)
}

\subsubsection{Opposed Checks}
An opposed check is a check whose success or failure is determined by comparing the check result to another character’s check result. In an opposed check, the higher result succeeds, while the lower result fails. In case of a tie, the higher skill modifier wins. If these scores are the same, roll again to break the tie.

\Table{Example Opposed Checks}{X b{2.67cm} b{2.5cm}}{
\tableheader Task & \tableheader Skill (Key Ability) & \tableheader Opposing Skill (Key Ability)\\
Con someone & Bluff (Cha) & Sense Motive (Wis)\\
Pretend to be someone else & Disguise (Cha) & Spot (Wis)\\
Create a false map & Forgery (Int) & Forgery (Int)\\
Hide from someone & Hide (Dex) & Spot (Wis)\\
Sneak up on someone & Move Silently (Dex) & Listen (Wis)\\
Steal a coin pouch & Sleight of Hand (Dex) & Spot (Wis)\\
Tie a prisoner securely & Use Rope (Dex) & Escape Artist (Dex)
}

\subsubsection{Trying Again}
In general, you can try a skill check again if you fail, and you can keep trying indefinitely. Some skills, however, have consequences of failure that must be taken into account. A few skills are virtually useless once a check has failed on an attempt to accomplish a particular task. For most skills, when a character has succeeded once at a given task, additional successes are meaningless.

\subsubsection{Untrained Skill Checks}
Generally, if your character attempts to use a skill he or she does not possess, you make a skill check as normal. The skill modifier doesn’t have a skill rank added in because the character has no ranks in the skill. Any other applicable modifiers, such as the modifier for the skill’s key ability, are applied to the check.

Many skills can be used only by someone who is trained in them.

\subsubsection{Favorable And Unfavorable Conditions}
Some situations may make a skill easier or harder to use, resulting in a bonus or penalty to the skill modifier for a skill check or a change to the DC of the skill check.

The chance of success can be altered in four ways to take into account exceptional circumstances.
\begin{enumerate*}
\item Give the skill user a +2 circumstance bonus to represent conditions that improve performance, such as having the perfect tool for the job, getting help from another character (see Combining Skill Attempts), or possessing unusually accurate information.
\item Give the skill user a -2 circumstance penalty to represent conditions that hamper performance, such as being forced to use improvised tools or having misleading information.
\item Reduce the DC by 2 to represent circumstances that make the task easier, such as having a friendly audience or doing work that can be subpar.
\item Increase the DC by 2 to represent circumstances that make the task harder, such as having an uncooperative audience or doing work that must be flawless.
\end{enumerate*}
Conditions that affect your character’s ability to perform the skill change the skill modifier. Conditions that modify how well the character has to perform the skill to succeed change the DC. A bonus to the skill modifier and a reduction in the check’s DC have the same result: They create a better chance of success. But they represent different circumstances, and sometimes that difference is important.

\subsubsection{Time And Skill Checks}
Using a skill might take a round, take no time, or take several rounds or even longer. Most skill uses are standard actions, move actions, or full-round actions. Types of actions define how long activities take to perform within the framework of a combat round (6 seconds) and how movement is treated with respect to the activity. Some skill checks are instant and represent reactions to an event, or are included as part of an action.

These skill checks are not actions. Other skill checks represent part of movement.

\subsubsection{Checks Without Rolls}
A skill check represents an attempt to accomplish some goal, usually while under some sort of time pressure or distraction. Sometimes, though, a character can use a skill under more favorable conditions and eliminate the luck factor.

\textbf{Taking 10}: When your character is not being threatened or distracted, you may choose to take 10. Instead of rolling 1d20 for the skill check, calculate your result as if you had rolled a 10. For many routine tasks, taking 10 makes them automatically successful. Distractions or threats (such as combat) make it impossible for a character to take 10. In most cases, taking 10 is purely a safety measure —you know (or expect) that an average roll will succeed but fear that a poor roll might fail, so you elect to settle for the average roll (a 10). Taking 10 is especially useful in situations where a particularly high roll wouldn’t help.

\textbf{Taking 20:} When you have plenty of time (generally 2 minutes for a skill that can normally be checked in 1 round, one full-round action, or one standard action), you are faced with no threats or distractions, and the skill being attempted carries no penalties for failure, you can take 20. In other words, eventually you will get a 20 on 1d20 if you roll enough times. Instead of rolling 1d20 for the skill check, just calculate your result as if you had rolled a 20.

Taking 20 means you are trying until you get it right, and it assumes that you fail many times before succeeding. Taking 20 takes twenty times as long as making a single check would take.

Since taking 20 assumes that the character will fail many times before succeeding, if you did attempt to take 20 on a skill that carries penalties for failure, your character would automatically incur those penalties before he or she could complete the task. Common “take 20” skills include Escape Artist, Open Lock, and Search.

\textbf{Ability Checks and Caster Level Checks:} The normal take 10 and take 20 rules apply for ability checks. Neither rule applies to caster level checks.

\subsection{Combining Skill Attempts}
When more than one character tries the same skill at the same time and for the same purpose, their efforts may overlap.

\subsubsection{Individual Events}
Often, several characters attempt some action and each succeeds or fails independently. The result of one character’s Climb check does not influence the results of other characters Climb check.

\subsubsection{Aid Another}
You can help another character achieve success on his or her skill check by making the same kind of skill check in a cooperative effort. If you roll a 10 or higher on your check, the character you are helping gets a +2 bonus to his or her check, as per the rule for favorable conditions. (You can’t take 10 on a skill check to aid another.) In many cases, a character’s help won’t be beneficial, or only a limited number of characters can help at once.

In cases where the skill restricts who can achieve certain results you can’t aid another to grant a bonus to a task that your character couldn’t achieve alone.

\subsubsection{Skill Synergy}
It’s possible for a character to have two skills that work well together. In general, having 5 or more ranks in one skill gives the character a +2 bonus on skill checks with each of its synergistic skills, as noted in the skill description. In some cases, this bonus applies only to specific uses of the skill in question, and not to all checks. Some skills provide benefits on other checks made by a character, such as those checks required to use certain class features.

\Table{Skill Synergies}{X p{4.8cm}}{
\tableheader 5 or more ranks in… & \tableheader Gives a +2 bonus on…\\
Autohypnosis & Knowledge (psionics) checks\\
Bluff & Diplomacy checks\\
Bluff & Disguise checks to act in character\\
Bluff & Intimidate checks\\
Bluff & Sleight Of Hand checks\\
Concentration & Autohypnosis checks\\
Craft & related Appraise checks\\
Decipher Script & Use Magic Device checks involving scrolls\\
Escape Artist & Use Rope checks involving bindings\\
Handle Animal & Ride checks\\
Handle Animal & wild empathy checks\\
Jump & Tumble checks\\
Knowledge (arcana) & Spellcraft checks\\
~ (architecture and engineering) & Search checks involving secret doors and similar compartments\\
~ (dungeoneering) & Survival checks when underground\\
~ (geography) & Survival checks to keep from getting lost or for avoiding hazards\\
~ (history) & bardic knowledge checks\\
~ (local) & Gather Information checks\\
~ (nature) & Survival checks in aboveground natural environments\\
~ (nobility and royalty) & Diplomacy checks\\
~ (psionics) & Psicraft\\
~ (religion) & checks to turn or rebuke undead\\
~ (the planes) & Survival checks when on other planes\\
Psicraft & Use Psionic Device checks involving power stones\\
Search & Survival checks when following tracks\\
Sense Motive & Diplomacy checks\\
Spellcraft & Use Magic Device involving scrolls\\
Survival & Knowledge (nature) checks\\
Tumble & Balance checks\\
Tumble & Jump checks\\
Use Magic Device & Spellcraft checks to decipher scrolls\\
Use Psionic Device & Psicraft checks to address power stones\\
Use Rope & Climb checks involving climbing ropes\\
Use Rope & Escape Artist checks involving ropes
}

\subsection{Ability Checks}
Sometimes a character tries to do something to which no specific skill really applies. In these cases, you make an ability check. An ability check is a roll of 1d20 plus the appropriate ability modifier. Essentially, you’re making an untrained skill check.

In some cases, an action is a straight test of one’s ability with no luck involved. Just as you wouldn’t make a height check to see who is taller, you don’t make a Strength check to see who is stronger.