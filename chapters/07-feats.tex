\Chapter{Feats}
{Turning from political histories to folklore, who has not heard a bard's sonorous voice sing the marvels of the world before ours? The lyrics speak of a land of plenty, with grass on every hill and water in every draw. Fields of barley and whey stretched for miles, and there were so many sheep that the herds could not be counted. Proud forests of oak and maple covered the wild lands, and men were the masters of the beasts.

These ballads sing the praises of warriors who fought not for food or entertainment, but for honor, glory, and lady love. The kings in these songs were noble warriors who fought terrible beasts and waged righteous wars in defense of their subjects. Clearly, they were men who placed the needs of their domains above their own desires and cravings.}
{The Wanderer's Journal}

\section{Prerequisites}
Some feats have prerequisites. Your character must have the indicated ability score, class feature, feat, skill, base attack bonus, or other quality designated in order to select or use that feat. A character can gain a feat at the same level at which he or she gains the prerequisite.

A character can't use a feat if he or she has lost a prerequisite.

\section{Types Of Feats}
Some feats are general, meaning that no special rules govern them as a group. Others are item creation feats, which allow spellcasters to create magic items of all sorts. A metamagic feat lets a spellcaster prepare and cast a spell with greater effect, albeit as if the spell were a higher spell level than it actually is.

\subsection{Fighter Bonus Feats}
Any feat designated as a fighter feat can be selected as a fighter's bonus feat. This designation does not restrict characters of other classes from selecting these feats, assuming that they meet any prerequisites.

\subsection{Racial Feats}
Feats designated as racial feats require the character to be of a specific race in order to select the feat. These feats share no other special properties and are considered to be a subset of a larger category of feats (e.g., general, psionic, or tactical).

\subsection{Regional Feats}
Regional feats are feats that require that a character belongs to a certain culture, either a race or a specified area. These feats can only be selected at 1st level. A character can only take one single regional feat. This is because most regional feats are more powerful when compared to other feats.

\subsection{Psionic Feats}
Psionic feats are available only to characters and creatures with the ability to manifest powers. (In other words, they either have a power point reserve or have psi-like abilities.)

Because psionic feats are supernatural abilities---a departure from the general rule that feats do not grant supernatural abilities---they cannot be disrupted in combat (as powers can be) and generally do not provoke attacks of opportunity (except as noted in their descriptions). Supernatural abilities are not subject to power resistance and cannot be dispelled; however, they do not function in areas where psionics is suppressed, such as a null psionics field. Leaving such an area immediately allows psionic feats to be used.

Many psionic feats can be used only when you are psionically focused; others require you to expend your psionic focus to gain their benefit. Expending your psionic focus does not require an action; it is part of another action (such as using a feat). When you expend your psionic focus, it applies only to the action for which you expended it.

\subsection{Divine Feats}
Divine feats are available only to characters with the ability to turn or rebuke undead. Using a divine feat costs at least one turn attempt from the character's number of attempts each day. If you don't have any turn or rebuke attempts left, you can't use a divine feat.

As psionic feats, divine feats are also supernatural abilities. Activating a divine feat does not provoke attacks of opportunity, unless otherwise noted. You can activate only one divine feat (or turn/rebuke once) per round.

\subsection{Raze Feats}
Raze feats are feats that require an arcane spellcaster to be a defiler. They can only be applied when defiling.

Multiple raze feats can be applied at the same time. For example, a defiler who has \feat{Distance Raze}, \feat{Destructive Raze} and \feat{Fast Raze} can benefit from all of them when casting a single spell. A wizard's bonus feats can be used to acquire Raze feats if the wizard fulfills the feat prerequisites.

\subsection{Metamagic Feats}
As a spellcaster's knowledge of magic grows, she can learn to cast spells in ways slightly different from the ways in which the spells were originally designed or learned. Preparing and casting a spell in such a way is harder than normal but, thanks to metamagic feats, at least it is possible. Spells modified by a metamagic feat use a spell slot higher than normal. This does not change the level of the spell, so the DC for saving throws against it does not go up.

\textbf{Wizards and Divine Spellcasters:} Wizards and divine spellcasters must prepare their spells in advance. During preparation, the character chooses which spells to prepare with metamagic feats (and thus which ones take up higher-level spell slots than normal).

\textbf{Templars:} Templars choose spells as they cast them. They can choose when they cast their spells whether to apply their metamagic feats to improve them. As with other spellcasters, the improved spell uses up a higher-level spell slot. But because the templar has not prepared the spell in a metamagic form in advance, he must apply the metamagic feat on the spot. Therefore, such a character must also take more time to cast a metamagic spell (one enhanced by a metamagic feat) than he does to cast a regular spell. If the spell's normal casting time is 1 standard action, casting a metamagic version is a full-round action for a templar. (This isn't the same as a 1-round casting time.)

For a spell with a longer casting time, it takes an extra full-round action to cast the spell.

\textbf{Spontaneous Casting and Metamagic Feats:} A cleric spontaneously casting a \spellref{cure light wounds}{cure} or \spellref{inflict light wounds}{inflict} spell can cast a metamagic version of it instead. Extra time is also required in this case. Casting a 1-action metamagic spell spontaneously is a full-round action, and a spell with a longer casting time takes an extra full-round action to cast.

\textbf{Effects of Metamagic Feats on a Spell:} In all ways, a metamagic spell operates at its original spell level, even though it is prepared and cast as a higher-level spell. Saving throw modifications are not changed unless stated otherwise in the feat description.

The modifications made by these feats only apply to spells cast directly by the feat user. A spellcaster can't use a metamagic feat to alter a spell being cast from a wand, scroll, or other device.

Metamagic feats that eliminate components of a spell don't eliminate the attack of opportunity provoked by casting a spell while threatened. However, casting a spell modified by \feat{Quicken Spell} does not provoke an attack of opportunity.

Metamagic feats cannot be used with all spells. See the specific feat descriptions for the spells that a particular feat can't modify.

\textbf{Multiple Metamagic Feats on a Spell:} A spellcaster can apply multiple metamagic feats to a single spell. Changes to its level are cumulative. You can't apply the same metamagic feat more than once to a single spell.

\textbf{Magic Items and Metamagic Spells:} With the right item creation feat, you can store a metamagic version of a spell in a scroll, potion, or wand. Level limits for potions and wands apply to the spell's higher spell level (after the application of the metamagic feat). A character doesn't need the metamagic feat to activate an item storing a metamagic version of a spell.

\textbf{Counterspelling Metamagic Spells:} Whether or not a spell has been enhanced by a metamagic feat does not affect its vulnerability to counterspelling or its ability to counterspell another spell.

\subsection{Metapsionic Feats}
As a manifester's knowledge of psionics grows, he can learn to manifest powers in ways slightly different from how the powers were originally designed or learned. Of course, manifesting a power while using a metapsionic feat is more expensive than manifesting the power normally.

\textbf{Manifesting Time:} Powers manifested using metapsionic feats take the same time as manifesting the powers normally unless the feat description specifically says otherwise.

\textbf{Manifestation Cost:} To use a metapsionic feat, a psionic character must both expend his psionic focus (see the \skill{Concentration} skill description) and pay an increased power point cost as given in the feat description.

\textbf{Limits on Use:} As with all powers, you cannot spend more power points on a power than your manifester level. Metapsionic feats merely let you manifest powers in different ways; they do not let you violate this rule.

\textbf{Effects of Metapsionic Feats on a Power:} In all ways, a metapsionic power operates at its original power level, even though it costs additional power points. The modifications to a power made by a metapsionic feat have only their noted effect on the power. A manifester can't use a metapsionic feat to alter a power being cast from a power stone, dorje, or other device.

Manifesting a power modified by the \feat{Quicken Power} feat does not provoke attacks of opportunity.

Some metapsionic feats apply only to certain powers, as described in each specific feat entry.

\textbf{Psionic Items and Metapsionic Powers:} With the right psionic item creation feat, you can store a metapsionic power in a power stone, psionic tattoo, or dorje. Level limits for psionic tattoos apply to the power's higher metapsionic level.

A character doesn't need the appropriate metapsionic feat to activate an item in which a metapsionic power is stored, but does need the metapsionic feat to create such an item.

\subsection{Item Creation Feats}
An item creation feat lets a spellcaster create a magic item of a certain type. Regardless of the type of items they involve, the various item creation feats all have certain features in common.

\textbf{XP Cost:} Experience that the spellcaster would normally keep is expended when making a magic item. The XP cost equals 1/25 of the cost of the item in gold pieces. A character cannot spend so much XP on an item that he or she loses a level. However, upon gaining enough XP to attain a new level, he or she can immediately expend XP on creating an item rather than keeping the XP to advance a level.

\textbf{Raw Materials Cost:} The cost of creating a magic item equals one-half the sale cost of the item.

Using an item creation feat also requires access to a laboratory or magical workshop, special tools, and so on. A character generally has access to what he or she needs unless unusual circumstances apply.

\textbf{Time:} The time to create a magic item depends on the feat and the cost of the item. The minimum time is one day.

\textbf{Item Cost:} \feat{Brew Potion}, \feat{Craft Wand}, and \feat{Scribe Scroll} create items that directly reproduce spell effects, and the power of these items depends on their caster level---that is, a spell from such an item has the power it would have if cast by a spellcaster of that level. The price of these items (and thus the XP cost and the cost of the raw materials) also depends on the caster level. The caster level must be high enough that the spellcaster creating the item can cast the spell at that level. To find the final price in each case, multiply the caster level by the spell level, then multiply the result by a constant, as shown below:

\textit{Scrolls:} Base price = spell level $\times$ caster level $\times$ 25 cp.

\textit{Potions:} Base price = spell level $\times$ caster level $\times$ 50 cp.

\textit{Wands:} Base price = spell level $\times$ caster level $\times$ 750 cp.

A 0-level spell is considered to have a spell level of \onehalf for the purpose of this calculation.

\textbf{Extra Costs:} Any potion, scroll, or wand that stores a spell with a costly material component or an XP cost also carries a commensurate cost. For potions and scrolls, the creator must expend the material component or pay the XP cost when creating the item.

For a wand, the creator must expend fifty copies of the material component or pay fifty times the XP cost.

Some magic items similarly incur extra costs in material components or XP, as noted in their descriptions.


\subsection{Psionic Item Creation Feats}
Manifesters can use their personal power to create lasting psionic items. Doing so, however, is draining. A manifester must put a little of himself or herself into every psionic item he or she creates.

A psionic item creation feat lets a manifester create a psionic item of a certain type. Regardless of the type of items they involve, the various item creation feats all have certain features in common.

\textbf{XP Cost:} Power and energy that the manifester would normally keep is expended when making a psionic item. The experience point cost of using a psionic item creation feat equals 1/25 the cost of the item in gold pieces. A character cannot spend so much XP on an item that he or she loses a level. However, upon gaining enough XP to attain a new level, he or she can immediately expend XP on creating an item rather than keeping the XP to advance a level.

\textbf{Raw Materials Cost:} Creating a psionic item requires costly components, most of which are consumed in the process. The cost of these materials equals onehalf the cost of the item.

Using a psionic item creation feat also requires access to a laboratory or psionic workshop, special tools, and other equipment. A character generally has access to what he or she needs unless unusual circumstances apply (such as if he's traveling far from home).

\textbf{Time:} The time to create a psionic item depends on the feat and the cost of the item. The minimum time is one day.

% \textbf{Item Cost:} Craft Dorje, Imprint Stone, and Scribe Tattoo create items that directly reproduce the effects of powers, and the strength of these items depends on their manifester level---that is, a power from such an item has the strength it would have if manifested by a manifester of that level. Often, that is the minimum manifester level necessary to manifest the power. (Randomly discovered items usually follow this rule.) However, when making such an item, the item's strength can be set higher than the minimum. Any time a character creates an item using a power augmented by spending additional power points, the character's effective manifester level for the purpose of calculating the item's cost increases by 1 for each 1 additional power point spent. (Augmentation is a feature of many powers that allows the power to be amplified in various ways if additional power points are spent.) All other level-dependent parameters of the power forged into the item are set according to the effective manifester level.

% The price of psionic items (and thus the XP cost and the cost of the raw materials) depends on the level of the power and a character's manifester level. The character's manifester level must be high enough that the item creator can manifest the power at the chosen level. To find the final price in each case, multiply the character's manifester level by the power level, then multiply the result by a constant, as shown below.

\textbf{Item Cost:} \feat{Scribe Tattoo} create items that directly reproduce the effects of powers, and the strength of these items depends on the effective level of the power: power level + the power check DC modifier. Often, that is the minimum power check modifier necessary to manifest the power. (Randomly discovered items usually follow this rule.) However, when making such an item, the item's strength can be set higher than the minimum. Any time a character creates an item using a power augmented by spending additional power points, the power check DC modifier may increase depending on the psionic power. (Augmentation is a feature of many powers that allows the power to be amplified in various ways if additional power points are spent.) All other level-dependent parameters of the power forged into the item are set according to the effective manifester level.

The price of psionic items (and thus the XP cost and the cost of the raw materials) depends on the effective level of the power and a character's manifester level. The character's manifester level must be high enough that the item creator can manifest the power. To find the final price in each case, multiply the character's manifester level by the effective power level, then multiply the result by a constant, as shown below.

\textit{Psionic Tattoos:} Base price = (power level + power check DC modifier) $\times$ manifester level $\times$ 4,000 cp.

\textbf{Extra Costs:} Any psionic tattoo that stores a power with an XP cost also carries a commensurate cost, the creator must pay the XP cost when creating the item.

Some psionic items similarly incur extra costs in XP, as noted in their descriptions.


\section{Feat Descriptions}
Here is the format for feat descriptions.

\subsubsection{Feat Name {\large[Type Of Feat]}}
\textbf{Prerequisite:} A minimum ability score, another feat or feats, a minimum base attack bonus, a minimum number of ranks in one or more skills, or a class level that a character must have in order to acquire this feat. This entry is absent if a feat has no prerequisite. A feat may have more than one prerequisite.

\textbf{Benefit:} What the feat enables the character (``you'' in the feat description) to do. If a character has the same feat more than once, its benefits do not stack unless indicated otherwise in the description.

In general, having a feat twice is the same as having it once.

\textbf{Normal:} What a character who does not have this feat is limited to or restricted from doing. If not having the feat causes no particular drawback, this entry is absent.

\textbf{Special:} Additional facts about the feat that may be helpful when you decide whether to acquire the feat.

\FeatTable{General}{
	\feat{Ability Focus} & Special attack & +2 DC to chosen special attack \\
	\feat{Ancestral Knowledge} & Int 13, \skill{Knowledge} (history) 10 ranks & +10 bonus on \skill{Knowledge} (history) checks or bardic knowledge checks about a time period\\
	\feat{Antipsionic Magic} & \skill{Spellcraft} 5 ranks & +2 bonus on caster level check to defeat power resistence\\
	\feat{Arena Clamor} & Cha 13, \feat{Improved Critical}, Perform 5 ranks & +2 bonus on attack rolls after a critical hit\\
	\feat{Armor Proficiency (Light)} && No armor check penalty on attack rolls\\
	~\feat{Armor Proficiency (Medium)} & Armor Proficiency (Light) & No armor check penalty on attack rolls\\
	~ ~\feat{Armor Proficiency (Heavy)} & Armor Proficiency (Medium) & No armor check penalty on attack rolls\\
	\feat{Augment Summoning} & \feat{Spell Focus} (conjuration) & Summoned creatures gain +4 Str, +4 Con\\
	\feat{Brutal Attack} & Cha 13, \feat{Improved Critical}, \skill{Perform} 5 ranks & Enemies become shaken after a critical hit\\
	\feat{Bug Trainer} & \skill{Handle Animal} 5 ranks, \skill{Knowledge} (nature) 5 ranks & Use \skill{Handle Animal} on vermin\\
	\feat{Chaotic Mind} & Chaotic alignment, Cha 15 & Creatures do not gain insight bonus against you\\
	\feat{Cloak Dance} & \skill{Hide} 10 ranks, \skill{Perform} (dance) 2 ranks & Gain concealment for one turn\\
	\feat{Closed Mind} && +2 bonus on saves to resist powers\\
	\feat{Combat Casting} && +4 bonus on checks for defensive casting\\
	\feat{Cornered Fighter} & Base attack bonus +5 & +2 on attack rolls and AC when flanked \\
	\feat{Deadly Precision} & Dex 15, base attack bonus +5 & Reroll 1 on sneak attack's dice\\
	\feat{Defender of the Land} & Wild shape class feature & Bonus damage in spells cast against defilers\\
	\feat{Dissimulated} & Int 13, Cha 13, \skill{Bluff} 5 ranks & Add Int modifier to \skill{Bluff} checks\\
	\feat{Drake's Child} & Str 13, Wis 13 & +1 bonus to Will and Fortitude saves\\
	\feat{Elemental Cleansing} & Ability to turn or rebuke undead & Turn undead deals 2d6 energy damage\\
	\feat{Empower Spell-Like Ability} & Spell-like ability at caster level 6th or higher & +50\% damage to chosen spell-like ability\\
	\feat{Endurance} && +4 bonus on checks to resist nonlethal damage\\
	~ \feat{Diehard} & \feat{Endurance} & Become stable between $-1$ and $-9$ hit points\\
	\feat{Eschew Materials} && Cast spells without material components\\
	\feat{Extra Turning} & Ability to turn or rebuke creatures & Can use turn or rebuke 4 more times per day\\
	\feat{Faithful Follower} && +5 bonus on saves against fear if near an ally with \feat{Leadership} feat\\
	\feat{Favorite} & \feat{Secular Authority}, \skill{Diplomacy} 10 ranks & Can use \feat{Secular Autority} 4 more times per day\\
	\feat{Fearsome} & Str 15 & Use Str on \skill{Intimidade} checks\\
	\feat{Flyby Attack} & Fly speed & Take a standard action at any point during flight\\
	\feat{Gladiatorial Entertainer} & Gladiatorial performance class feature & Can use gladiatorial performance 4 more times per day\\
	\feat{Greasing the Wheels} & Cha 13, \skill{Diplomacy} 7 ranks, \skill{Knowledge} (local) 5 ranks & Use \skill{Diplomacy} to bribe a character\\
	\feat{Great Fortitude} && +2 bonus on Fortitude saves\\
	\feat{Grovel} & Eldaarich, \skill{Perform} 1 rank & +3 on \skill{Diplomacy} and \skill{Bluff} checks\\
	\feat{Hard as Rock} & Con 15, \feat{Diehard}, \feat{Great Fortitude} & Become immune to death from massive damage\\
	\feat{Hover} & Fly speed & Stay in place while flying\\
	\feat{Hostile Mind} & Cha 15 & Deal damage to telepathic manifesters\\
	\feat{Improved Counterspell} && Use spell from the same school to counterspell\\
	\feat{Improved Familiar} & Ability to acquire a new familiar & Acquire nonstandard familiar\\
	\feat{Improved Natural Armor} & Natural armor, Con 13 & +1 natural armor\\
	\feat{Improved Natural Attack} & Natural attack, base attack bonus +4 & Increase dice damage of chosen natural attack\\
	\feat{Improved Sigil} & Sigil ability, \skill{Diplomacy} 9 ranks & Use two 1st-level spells as spell-like ability\\
	\feat{Improved Turning} & Ability to turn or rebuke creatures & +1 level on turning checks\\
	\feat{Improviser} & Wis 13, base attack bonus +3 & Reduce penalty of improvised weapons\\
	\feat{Innate Hunter} & \feat{Track}, \skill{Survival} 5 ranks & +4 bonus on \skill{Survival} for hunting, +1 bonus on attack rolls versus animals\\
	\feat{Iron Will} && +2 bonus on Will saves\\
	~ \feat{Force of Will} & \feat{Iron Will} & Can make a Will save instead of a Fort or Ref save against psionics powers\\
	\feat{Jaguar Roar} & Cha 13, Draj, \skill{Intimidate} 9 ranks & Affected creatures become shaken for 2d4 rounds\\
	\feat{Kiltektet} && All \skill{Knowledge} skills are class skills\\
	\feat{Leadership} & Character level 6th & Gain followers and cohort\\
	\feat{Lightning Reflexes} && +2 bonus on Reflex saves\\
	\feat{Linguist} && Gain two spoken languages\\
	\feat{Martial Weapon Proficiency} && No penalties attacking with specific weapon\\
	\feat{Mastyrial Blood} & Con 13 & +4 bonus on saves against poison\\
	\feat{Mental Resistance} & Base Will save bonus +2 & 3/--- against psionic attacks\\
	\feat{Mind Over Body} & Con 13 & Heal ability damage more quickly\\
	\feat{Multiattack} & Three or more natural attacks & Reduce secondary attack's penalty\\
	\feat{Multiweapon Fighting} & Dex 13, three or more hands & Reduce penalty for fighting with multiple weapons\\
	\feat{Natural Spell} & Wis 13, wild shape ability & Cast spells while in wild shape\\
	\feat{Open Minded} && +5 skill points\\
	\feat{Path Dexter} & Preserver & +1 caster level for chosen abjuration/divination spells\\
	\feat{Path Sinister} & Defiler & +1 caster level for chosen evocation/necromancy spells\\
	\feat{Protective} && +4 bonus to saves to items\\
	\feat{Psionic Hole} & Con 15 & Foes lose psionic focus and power points\\
	\feat{Psionic Mimicry} & \skill{Bluff} 8 ranks, \skill{Knowledge} (psionics) 4 ranks, \skill{Psicraft} 4 ranks & Use \skill{Bluff} to disguise spell as psionics\\
	\feat{Psionic Schooling} && One psionic class becomes an additional favored class\\
}

\FeatTable{General}{
	\feat{Quicken Spell-Like Ability} & Spell-like ability at caster level 10th or higher & Use chosen spell-like ability as swift action\\
	\feat{Raised by Beasts} && Wild empathy for a kind of animal\\
	\feat{Rapid Metabolism} & Con 13 & Heal hit points more quickly\\
	\feat{Reckless Offense} & Base attack bonus +1 & Take $-4$ penalty to AC to gain +2 on melee attack rolls\\
	\feat{Reign of Terror} & \skill{Intimidate} 5 ranks, member of Takrits, Savak, or Neshtap Order & +4 bonus on \feat{Secular Authority} checks\\
	\feat{Run} && Move 5$\times$ normal speed, +4 bonus on \skill{Jump} checks\\
	\feat{Secular Authority} & Cha 13, \skill{Diplomacy} 6 ranks, \feat{Negotiator}, accepted into city-state's templarate & New uses for \skill{Diplomacy}\\
	\feat{Shield Proficiency} && No shield penalty on attack rolls\\
	~ \feat{Tower Shield Proficiency} && No shield penalty on attack rolls using tower shield\\
	\feat{Sidestep Charge} & Dex 13, \feat{Dodge} & +4 bonus to AC against charge attacks\\
	\feat{Simple Weapon Proficiency} && No $-4$ penalty on attack rolls for simple weapons\\
	\feat{Sniper} & Dex 13, \skill{Hide} 1 rank & +5 bonus on \skill{Hide} checks to stay hidden\\
	\feat{Skill Focus} && +3 bonus on checks of chosen skill\\
	\feat{Spell Focus} && +1 bonus to DC on spells of chosen school\\
	~ \feat{Greater Spell Focus} && +1 bonus to DC on spells of chosen school\\
	\feat{Spell Mastery} & Wizard level 1st & Can prepare some spells without spellbook\\
	\feat{Spell Penetration} && +2 bonus on caster level check to defeat spell resistence\\
	~ \feat{Greater Spell Penetration} & \feat{Spell Penetration} & +2 bonus on caster level check to defeat spell resistence\\
	\feat{Stand Still} & Str 13 & Use attack of opportunity to stop moving foe\\
	\feat{Toughness} && +3 hit points\\
	\feat{Track} && Use \skill{Survival} to track\\
	\feat{Wastelander} && +1 bonus on Fortitude saves, +2 bonus on \skill{Survival} checks\\
	\feat{Wild Talent} && Gain psionic powers and 2 power points\\
	\feat{Wingover} & Fly speed & Change flight direction using 3 m of movement\\
}

\FeatTable[p{3cm}]{Skill}{
	\feat{Acrobatic} && +2 bonus on \skill{Jump} and \skill{Tumble} checks \\
	\feat{Agile} && +2 bonus on \skill{Balance} and \skill{Escape Artist} checks \\
	\feat{Alertness} && +2 bonus on \skill{Listen} and \skill{Spot} checks \\
	\feat{Animal Affinity} && +2 bonus on \skill{Handle Animal} and \skill{Ride} checks \\
	\feat{Athletic} && +2 bonus on \skill{Climb} and \skill{Swim} checks \\
	\feat{Autonomous} && +2 bonus on \skill{Autohypnosis} and \skill{Knowledge} (psionics) checks \\
	\feat{Deceitful} && +2 bonus on \skill{Disguise} and \skill{Forgery} checks \\
	\feat{Deft Hands} && +2 bonus on \skill{Sleight of Hand} and \skill{Use Rope} checks \\
	\feat{Diligent} && +2 bonus on \skill{Appraise} and \skill{Decipher Script} checks \\
	\feat{Field Officer} && +2 bonus on \skill{Diplomacy} and \skill{Knowledge} (warcraft) checks \\
	\feat{Investigator} && +2 bonus on \skill{Gather Information} and \skill{Search} checks \\
	\feat{Magical Aptitude} && +2 bonus on \skill{Spellcraft} and \skill{Use Magic Device} checks \\
	\feat{Negotiator} && +2 bonus on \skill{Diplomacy} and \skill{Sense Motive} checks \\
	\feat{Nimble Fingers} && +2 bonus on \skill{Disable Device} and \skill{Open Lock} checks \\
	\feat{Persuasive} && +2 bonus on \skill{Bluff} and \skill{Intimidate} checks \\
	\feat{Psionic Affinity} && +2 bonus on \skill{Psicraft} and \skill{Use Psionic Device} checks \\
	\feat{Self-Sufficient} && +2 bonus on \skill{Heal} and \skill{Survival} checks \\
	\feat{Stealthy} && +2 bonus on \skill{Hide} and \skill{Move Silently} checks \\
	\feat{Trader} && +2 bonus on \skill{Appraise} and \skill{Bluff} checks
}

\FeatTable[p{3cm}]{Racial}{
	\feat{Active Glands} & Thri-kreen & Use poison two additional times per day\\
	\feat{Advanced Antennae} & Thri-kreen & Gain scent ability\\
	\feat{Blend} & Thri-kreen & +3 on \skill{Hide} checks in sandy or arid terrain\\
	\feat{Blessed by the Ancestors} & Thri-kreen & +1 bonus on all saves\\
	\feat{Cannibalism Ritual} & Wis 13, halfling & Gain ability bonus for 1 day after devouring slain foe\\
	\feat{Dwarven Vision} & Mul & Gain darkvision 18 m\\
	\feat{Elfeater} & Thri-kreen & +1 on attack rolls and +2 on some skill checks against elves\\
	\feat{Improved Gyth'sa} & Thri-kreen, Con 13 & Recover double hit points after a night's rest\\
	\feat{Longshanks} & Half-elf, both parents must be half-elves & +3 m speed\\
	\feat{Tikchak} & Thri-kreen, character level 5th & Add Wis to \skill{Survival} checks, gain proficiency with chatkcha\\
	\feat{Tokchak} & Thri-kreen & Adjacent allies gain +1 bonus on Ref saves
}

\FeatTable[p{3cm}]{Regional}{
	\feat{Artisan} & {Nibenay, Raam, Urik} & +3 on \skill{Concentration} and \skill{Craft} checks\\
	\feat{Astrologer} & {Draj, Nibenay} & +3 on \skill{Knowledge} (nature) checks, +5 to avoid getting lost\\
	\feat{Companion} & {Kurn, Tyr} & Your aid another action grants +3 bonus\\
	\feat{Disciplined} & {Dwarf, Urik} & +1 to Will saves, +3 on \skill{Concentration} checks\\
	\feat{Elfish Eloy} & Half-elf, both parents must be half-elves & +3 on \skill{Hide} while aboveground\\
	\feat{Freedom} & {Tyr} & Take extra actions per day\\
	\feat{Giant Killer} & {Sea of Silt} & +4 to confirm criticals and +2 to AC against giants\\
	\feat{Jungle Fighter} & {Forest Ridge, Gulg} & +2 dodge bonus to AC in forests\\
	\feat{Legerdemain} & {Elf, Salt View} & +3 bonus on \skill{Open Lock} and \skill{Sleight of Hand} checks\\
	\feat{Mansabdar} & {Raam} & +1 to Fort saves, +3 bonus on \skill{Intimidate} checks\\
}
\FeatTable[p{3cm}]{Regional}{
	\feat{Mekillothead} & {Draj, mul} & +1 to Will saves, +3 bonus on \skill{Intimidate} checks\\
	\feat{Metalsmith} & {Dwarf, Tyr} & You do not suffer $-5$ penalty to forge metal items\\
	\feat{Nature's Child} & {Gulg, halfling} & +3 on \skill{Knowledge} (nature) and \skill{Survival} checks\\
	\feat{Paranoid} & {Eldaarich} &  +1 to Ref saves, +3 bonus on \skill{Sense 
	motive} checks\\
	\feat{Performance Artist} & {Balic, Nibenay, Salt View} & +3 bonus on \skill{Knowledge} (local) and \skill{Perform} checks\\
	\feat{Tarandan Method} & {Raam} & +2 DC to powers from chosen discipline
}

\vskip3cm
\subsectionA{General Feats}

\Feat{Ability Focus}
{Choose one of the creature's special attacks.}
{Special attack.}
{Add +2 to the DC for all saving throws against the special attack on which the creature focuses.}
{}
{A creature can gain this feat multiple times. Its effects do not stack. Each time the creature takes the feat it applies to a different special attack.}

\Feat{Ancestral Knowledge}
{You know legends and facts about long past events that have been shrouded by the sands of time.}
{Int 13, \skill{Knowledge} (history) 10 ranks.}
{Choose one of the following time periods: Blue Age, Green Age, or Cleansing Wars. You gain a +10 bonus on \skill{Knowledge} (history) checks or bardic knowledge checks to gain information about the chosen category.}{}
{You can take this feat more than once, but the bonus doesn't stack. Each time you take this feat, you choose another time period.}

% \Feat{Antipsionic Magic}
% {Your spells are more potent when used against psionic characters and creatures.}
% {\skill{Spellcraft} 5 ranks.}
% {You get a get a +2 bonus on caster level checks made to overcome a psionic creature's power resistance.

% This bonus stacks with the bonus conferred by Spell Penetration and Greater Spell Penetration. Moreover, whenever a psionic creature attempts to dispel a spell you cast, it makes its manifester level check against a DC of 13 + its manifester level.

% The benefits of this feat apply only to power resistance.

% % The bonus does not apply to spell resistance. This is an exception to the psionics-magic transparency rule.
% }{}
% {You cannot take or use this feat if you have the ability to use powers (if you have a power point reserve or psi-like abilities).}

\Feat{Arena Clamor}
{With your savage blows, you can make your companions give their best.}
{Cha 13, \feat{Improved Critical}, \skill{Perform} (arena fighting) 5 ranks.}
{Whenever you confirm a critical hit, all allies within a 18-meter radius who have line of sight on you receive a +2 morale bonus on attack rolls for 1 round. This is a mind-affecting effect. This effect is not cumulative. Characters cannot be affected more than once in this way in the same combat.}{}{}

\GFeat{Arena Fighter}
{Martial prowess, stance (arena guile).}
{
Your gladiator and fighter levels stack for the purpose of determining your bonus from any martial prowess technique. For example, a 5th-level gladiator/7th-level fighter with bravery and tortoise style would gain +6 on Will saves against mind-affecting abilities and would improve the shield bonus by 3, as if he was a 12th-level fighter.

Your gladiator and fighter levels also stack for the purpose of determining which stances you have access to. For example, a 5th-level gladiator/7th-level fighter would have access to keen eye and scoundrel pose, as if he was a 12th-level gladiator.
}

\GFeat{Arena Performer}
{Bardic music, gladiatorial performance.}
{
Your bard and gladiator levels stack for the purpose of determining your bonus from inspire courage. For example, a 3rd-level bard/5th-level gladiator would give +2 morale bonus with inspire courage, as if she were an 8th-level bard.

Your bard and gladiator levels also stack for the purpose of determining the bonus and penalties from dirty trick. For example, a 3rd-level bard/5th-level gladiator could give +2 morale bonus or -2 morale penalty with dirty trick, as if she were an 8th-level gladiator.

Additionally, you may expend a bardic music daily use to start a gladiatorial performance, and you may use \skill{Perform} (arena fighting) to start a bardic music.
}

% \GFeat{Arena Psionicist}
% {Gladiatorial performance, psychic warrior level 1st.}
% {
% Your gladiator and psychic warrior levels stack for the purpose of determining which gladiatorial performances you have access. For example, a 8th-level gladiator/1st-level psychic warrior would have access to shake off, as if he was a 9th-level gladiator.

% Your gladiator and psychic warrior levels also stack for the purpose of determining your manifester level. For example, a 8th-level gladiator/1st-level psychic warrior with a Wisdom score of 16 would have 13 additional power points from his ability score and could spend 9 power points in a single psionic power, as if he was a 9th-level psychic warrior.
% }

\Feat{Armor Proficiency (Heavy)}
{}
{Armor Proficiency (light), Armor Proficiency (medium).}
{See Armor Proficiency (light).}
{See Armor Proficiency (light).}
{Fighters, psychic warriors, and clerics automatically have Armor Proficiency (heavy) as a bonus feat. They need not select it.}

\Feat{Armor Proficiency (Light)}
{}{}
{When you wear a type of armor with which you are proficient, the armor check penalty for that armor applies only to \skill{Balance}, \skill{Climb}, \skill{Escape Artist}, \skill{Hide}, \skill{Jump}, \skill{Move Silently}, \skill{Sleight of Hand}, and \skill{Tumble} checks.}
{A character who is wearing armor with which she is not proficient applies its armor check penalty to attack rolls and to all skill checks that involve moving, including Ride.}
{All characters except wizards, and psions automatically have Armor Proficiency (light) as a bonus feat. They need not select it.}

\Feat{Armor Proficiency (Medium)}
{}
{Armor Proficiency (light).}
{See Armor Proficiency (light).}
{See Armor Proficiency (light).}
{Fighters, barbarians, gladiators, psychic warriors, clerics, druids, and templars automatically have Armor Proficiency (medium) as a bonus feat. They need not select it.}

\Feat{Armored Stealth}
{}
{Dex 13, \skill{Hide} 4 ranks, \skill{Move Silently} 4 ranks, \feat{Armor Proficiency (Light)}, \feat{Stealthy}.}
{You do not apply armor check penalty to \skill{Hide} and \skill{Move Silently} checks, whenever you are wearing a light armor.}
{You apply the armor check penalty to all skills with the notation (see \chapref{Skills}), including \skill{Hide} and \skill{Move Silently}.}
{}

\GFeat{Augment Summoning}
{\feat{Spell Focus} (conjuration).}
{Each creature you conjure with any summon spell gains a +4 enhancement bonus to Strength and Constitution for the duration of the spell that summoned it.}

\Feat{Brutal Attack}
{Your decisive attacks are especially frightening for those who watch.}
{Cha 13, \feat{Improved Critical}, \skill{Perform} (arena fighting) 5 ranks.}
{Whenever you confirm a critical hit, all enemies within a 3-meter radius who have line of sight on you must make a Will save (DC 10 + \onehalf your character level + your Cha modifier) or become shaken for a number of rounds equal to your Cha modifier. This is a mind-affecting fear effect.

Whether or not the save is successful, that creature cannot be affected again by the same character's brutal attack for 24 hours.}{}{}

\Feat{Bug Trainer}
{You can train vermin creatures, such as kanks and cilops.}
{\skill{Handle Animal} 5 ranks, \skill{Knowledge} (nature) 5 ranks.}
{You can use the \skill{Handle Animal} skill for vermin as though they were animals with an Intelligence score of 1.}
{You can use the \skill{Handle Animal} skill only on creatures with an Intelligence score of 1 or 2.}
{}

\Feat{Chaotic Mind}
{The turbulence of your thoughts prevents others from gaining insight into your actions.}
{Chaotic alignment, Cha 15.}
{Creatures and characters who have an insight bonus on their attack rolls, an insight bonus to their Armor Class, or an insight bonus on skill checks or ability checks do not gain those bonuses against you.

The benefit of this feat applies only to insight bonuses gained from psionic powers and psi-like abilities. %This is an exception to the psionics-magic transparency rule.
}{}
{You cannot take or use this feat if you have the ability to use powers (if you have a power point reserve or psi-like abilities).}

\Feat{Cloak Dance}
{You are skilled at using optical tricks to make yourself seem to be where you are not.}
{\skill{Hide} 10 ranks, \skill{Perform} (dance) 2 ranks.}
{You can take a move action to obscure your exact position. Until your next turn, you have concealment. Alternatively, you can take a full-round action to entirely obscure your exact position. Until your next turn, you have total concealment.}{}{}

\Feat{Closed Mind}
{Your mind is better able to resist psionics than normal.}{}
{You get a +2 bonus on all saving throws to resist powers.

The benefit of this feat applies only to psionic powers and psi-like abilities. %This is an exception to the psionics-magic transparency rule.
}{}
{You cannot take or use this feat if you have the ability to use powers (if you have a power point reserve or psi-like abilities).}

\GFeat{Combat Casting}{}
{You get a +4 bonus on Concentration checks made to cast a spell or use a spell-like ability while on the defensive or while you are grappling or pinned.}

\Feat{Cornered Fighter}
{You fight better when you freedom is put at risk.}
{Base attack bonus +5.}
{You receive a +2 bonus on attack rolls and a +2 bonus to AC when fighting against opponents who flank you.}
{}{}

\Feat{Deadly Precision}
{You empty your mind of all distracting emotion, becoming an instrument of deadly precision.}
{Dex 15, base attack bonus +5.}
{You have deadly accuracy with your sneak attacks. You can reroll any result of 1 on your sneak attack's extra damage dice. You must keep the result of the reroll, even if it is another 1.}{}{}

\Feat{Defender of the Land}
{You share power with the spirit from your guarded land, to nurture and protect the land to which the spirit is tied.}
{Wild shape class feature.}
{You receive a +1 caster level on spells you cast against defilers and your spells damage is increased by 1 per die against defilers.}
{}{}

\Feat{Diehard}
{}
{\feat{Endurance}.}
{When reduced to between $-1$ and $-9$ hit points, you automatically become stable. You don't have to roll d\% to see if you lose 1 hit point each round.

When reduced to negative hit points, you may choose to act as if you were disabled, rather than dying. You must make this decision as soon as you are reduced to negative hit points (even if it isn't your turn). If you do not choose to act as if you were disabled, you immediately fall unconscious.

When using this feat, you can take either a single move or standard action each turn, but not both, and you cannot take a full round action. You can take a move action without further injuring yourself, but if you perform any standard action (or any other action deemed as strenuous, including some free actions, swift actions, or immediate actions, such as casting a quickened spell) you take 1 point of damage after completing the act. If you reach $-10$ hit points, you immediately die.}
{A character without this feat who is reduced to between $-1$ and $-9$ hit points is unconscious and dying.}
{}

\Feat{Dissimulated}
{Your ability to speak what others want to hear increases the credibility of your words.}
{Int 13, Cha 13, \skill{Bluff} 5 ranks.}
{In addition to your Charisma modifier, you can add your Intelligence modifier to your \skill{Bluff} checks.}
{}{}

\Feat{Drake's Child}
{You are what is known as a drake's child, an individual who shows both exceptional strength and wisdom.}
{Str 13, Wis 13.}
{You get a +1 bonus to Will saves and a +1 bonus to Fortitude saves. You gain an additional +1 bonus to saving throws against ability drain, ability damage, energy drain, and death effects.}
{}{}

\GFeat{Druidic Hunter}
{Animal companion, wild shape.}
{
Your druid and ranger levels stack for the purpose of determining your effective ranger level for the purpose of determining the bonus Hit Dice, extra tricks, special abilities, and other bonuses that your animal companion receives. For example, a 5th-level druid/4th-level ranger's animal companion would gain +4 HD and have a total of 3 bonus tricks, as if he was a 9th-level ranger.
}


\Feat{Elemental Cleansing}
{Undead you turn or rebuke suffer elemental damage.}
{Ability to turn or rebuke undead.}
{Any undead that you successfully turn or rebuke takes 2d6 points of energy damage in addition to the normal turning or rebuking effect. The type of damage dealt is the one associated with your patron element.}
{}{}

\Feat{Empower Spell-Like Ability}
{}
{Spell-like ability at caster level 6th or higher.}
{Choose one of the creature's spell-like abilities, subject to the restrictions below. The creature can use that ability as an empowered spell-like ability three times per day (or less, if the ability is normally usable only once or twice per day).

When a creature uses an empowered spell-like ability, all variable, numeric effects of the spell-like ability are increased by one half. Saving throws and opposed rolls are not affected. Spell-like abilities without random variables are not affected.

The creature can only select a spell-like ability duplicating a spell with a level less than or equal to half its caster level (round down) $-2$. For a summary, see the table in the description of the \feat{Quicken Spell-Like Ability} feat.}
{}
{This feat can be taken multiple times. Each time it is taken, the creature can apply it to a different one of its spell-like abilities.}

\Feat{Endurance}
{}
{You gain a +4 bonus on the following checks and saves: Swim checks made to resist nonlethal damage, Constitution checks made to continue running, Constitution checks made to avoid nonlethal damage from a forced march, Constitution checks made to hold your breath, Constitution checks made to avoid nonlethal damage from starvation or thirst, Fortitude saves made to avoid nonlethal damage from hot or cold environments, and Fortitude saves made to resist damage from suffocation. Also, you may sleep in light or medium armor without becoming fatigued.}
{A character without this feat who sleeps in medium or heavier armor is automatically fatigued the next day.}{}
{A ranger automatically gains Endurance as a bonus feat at 3rd level. He need not select it.}

\GFeat{Eschew Materials}{}
{You can cast any spell that has a material component costing 1 cp or less without needing that component. (The casting of the spell still provokes attacks of opportunity as normal.) If the spell requires a material component that costs more than 1 cp, you must have the material component at hand to cast the spell, just as normal.}

\Feat{Extra Music}
{}
{Bardic music ability.}
{Each time you take this feat, you can use your bardic music ability four more times per day than normal.}
{}
{You can gain Extra Music multiple times. Its effects stack. Each time you take the feat, you can use bardic music four additional times per day.}

\Feat{Extra Performance}
{You can make gladiatorial performances more often than normal.}
{Gladiatorial performance ability.}
{Each time you take this feat, you can use your gladiatorial performance ability four more times per day than normal.}
{}
{You can gain Extra Performance multiple times. Its effects stack. Each time you take the feat, you can use gladiatorial performance four additional times per day.}

\Feat{Extra Turning}
{}
{Ability to turn or rebuke creatures.}
{Each time you take this feat, you can use your ability to turn or rebuke creatures four more times per day than normal.

If you have the ability to turn or rebuke more than one kind of creature each of your turning or rebuking abilities gains four additional uses per day.}
{Without this feat, a character can typically turn or rebuke undead (or other creatures) a number of times per day equal to 3 + his or her Charisma modifier.}
{You can gain Extra Turning multiple times. Its effects stack. Each time you take the feat, you can use each of your turning or rebuking abilities four additional times per day.}

\Feat{Faithful Follower}
{You overcome your fears while being led.}{}
{You receive a +5 morale bonus on saving throws against fear effects whenever you are within 6 meters of an ally with the \feat{Leadership} feat.}{}{}

\Feat{Favorite}
{You have gained the graces of your sorcerer-monarch, receiving extra benefits.}
{\feat{Secular Authority}, \skill{Diplomacy} 10 ranks.}
{You can use your secular authority ability four more times per day than normal. Furthermore, whenever you contest or are contested in the use of secular authority, you receive a +2 bonus on your opposed \skill{Diplomacy} check.}
{Without this feat, a templar can typically use secular authority only once per day per templar level.}
{You can gain Favorite multiple times. Its effects stack. Each time you take the feat, you can use secular authority four additional times per day.}

\Feat{Fearsome}
{Your might frightens your foes.}
{Str 15.}
{You can use your Strength modifier instead of your Charisma modifier on \skill{Intimidate} checks. Additionally, you receive a +2 bonus on \skill{Intimidate} checks.}{}{}

\Feat{Flyby Attack}
{}
{Fly speed.}
{When flying, the creature can take a move action (including a dive) and another standard action at any point during the move. The creature cannot take a second move action during a round when it makes a flyby attack.}
{Without this feat, the creature takes a standard action either before or after its move.}
{}

\Feat{Force of Will}
{You are able to resist psionic attacks with extreme force of will.}
{\feat{Iron Will}.}
{Once per round, when targeted by a psionic effect that allows a Reflex save or a Fortitude save, you can instead make a Will saving throw to avoid the effect.

The benefit of this feat applies only to psionic powers and psi-like abilities. %This is an exception to the psionics-magic transparency rule.
}{}
{You cannot take or use this feat if you have the ability to use powers (if you have a power point reserve or psi-like abilities).}

\Feat{Greasing the Wheels}
{You can circumvent various official obstacles when a person in a position of trust or authority is willing to accept ``presents.''}
{Cha 13, \skill{Diplomacy} 7 ranks, \skill{Knowledge} (local) 5 ranks.}
{This feat grants a new use for the \skill{Diplomacy} skill. You must share a language with a creature in order to use this option. This option cannot be used during combat.

% \textit{Bribery Etiquette:} You can discern the timing of the offer, the amount that will most likely garner the wanted reaction, and the best way to disguise the bribe so that it doesn't draw attention from unwanted witnesses. An insulted character will have his attitude changed one step for the worse and might report you to the proper authorities (this can be negated by a successful \skill{Diplomacy} check, albeit with a $-10$ penalty).

% To bribe a character, you must give him a number of ceramic pieces (in coins, items or other valuables) as shown below.

\textit{Bribe Official:} You can discern the timing of the offer, the amount that will most likely garner the wanted reaction, and the best way to disguise the bribe so that it doesn't draw attention from unwanted witnesses.

You can help a \skill{Diplomacy} check to improve an NPC attitude by bribing. To determine the amount of ceramic pieces, roll the base bribe described in the table below and multiply it by the target DC (see \tabref{Diplomacy DCs}).

\Table{}{l R}{
\tableheader NPC station & \tableheader Base Bribe\\
Peasant or slave    &  2d4 cp\\
Freeman or soldier  &  3d8 cp\\
Merchant or officer & 5d10 cp\\
Noble or general    & 5d10 $\times$ 10 cp\\
% Templar or guard, low-level (1--4) & 12 cp\\
% Templar or guard, mid-level (5--8) & 20 cp\\
% Templar or guard, high-level (9+) & 30 cp
}

Several factors can affect the amount needed to bribe a character, but the DM may modify these values as he sees fit.

\Table{}{X c}{
\tableheader PC Background & \tableheader Bribe Modifier\\
Renown             & $-2$\\
Illegal profession & +5\\
PC is wanted       & $\times2$\\
Slave              & +2\\
Noble              & $-2$\\
Templar            & $\times$\onehalf\\
}
}{}{}

\GFeat{Great Fortitude}{}
{You get a +2 bonus on all Fortitude saving throws.}

\Feat{Greater Spell Focus}
{Choose a school of magic to which you already have applied the \feat{Spell Focus} feat.}{}
{Add +1 to the Difficulty Class for all saving throws against spells from the school of magic you select. This bonus stacks with the bonus from \feat{Spell Focus}.}{}
{You can gain this feat multiple times. Its effects do not stack. Each time you take the feat, it applies to a new school of magic to which you already have applied the \feat{Spell Focus} feat.}{}

\Feat{Greater Counterspell}
{}
{\skill{Spellcraft} 8 ranks, \feat{Improved Counterspell}.}
{You may counterspell using an immediate action, instead of readying action.}
{Without this feat, you may counter a spell only by choosing to ready action.}
{}

\GFeat{Greater Spell Penetration}
{\feat{Spell Penetration}.}
{You get a +2 bonus on caster level checks (1d20 + caster level) made to overcome a creature's spell resistance. This bonus stacks with the one from Spell Penetration.}

\GFeat{Grovel}
{Eldaarich, \skill{Perform} 1 rank.}
{By dramatically throwing yourself prone and helpless on the ground, you get a +3 bonus to all \skill{Diplomacy} checks and \skill{Bluff} checks.}

\Feat{Hard as Rock}
{You are resolute while fighting and particularly tough to kill.}
{Con 15, \feat{Diehard}, \feat{Endurance}.}
{You are immune to death from massive damage.

You can use your Constitution modifier in place of your Dexterity modifier on Reflex saves.}
{}{}

\Feat{Hostile Mind}
{Your mind recoils violently against those who use psionics against you.}
{Cha 15.}
{Whenever you are subject to a power from the telepathy discipline (regardless of whether the power is harmful or beneficial to you), the manifester must make a Will saving throw against a DC of 10 + \onehalf your character level + your Charisma bonus or take 2d6 points of damage.

The benefit of this feat applies only to psionic powers and psi-like abilities. %This is an exception to the psionics-magic transparency rule.
}{}
{You cannot take or use this feat if you have the ability to use powers (if you have a power point reserve or psi-like abilities).}

\Feat{Hover}
{}
{Fly speed.}
{When flying, the creature can halt its forward motion and hover in place as a move action. It can then fly in any direction, including straight down or straight up, at half speed, regardless of its maneuverability.

If a creature begins its turn hovering, it can hover in place for the turn and take a full-round action. A hovering creature cannot make wing attacks, but it can attack with all other limbs and appendages it could use in a full attack. The creature can instead use a breath weapon or cast a spell instead of making physical attacks, if it could normally do so.

If a creature of Large size or larger hovers within 6 meters of the ground in an area with lots of loose debris, the draft from its wings creates a hemispherical cloud with a radius of 18 meters. The winds so generated can snuff torches, small campfires, exposed lanterns, and other small, open flames of non-magical origin. Clear vision within the cloud is limited to 3 meters. Creatures have concealment at 4.5 to 6 meters (20\% miss chance). At 7.5 meters or more, creatures have total concealment (50\% miss chance, and opponents cannot use sight to locate the creature).

Those caught in the cloud must succeed on a \skill{Concentration} check (DC 10 + onehalf creature's HD) to cast a spell.}
{Without this feat, a creature must keep moving while flying unless it has perfect maneuverability.}
{}

\Feat{Improved Counterspell}
{}
{\skill{Concentration} 4 ranks, \skill{Knowledge} (arcana) 2 ranks, \skill{Spellcraft} 4 ranks.}
{When counterspelling, you may use a spell of the same school that is one or more spell levels higher than the target spell.}
{Without this feat, you may counter a spell only with the same spell or with a spell specifically designated as countering the target spell.}{}

\Feat{Improved Familiar}
{This feat allows spellcasters to acquire a new familiar from a nonstandard list, but only when they could normally acquire a new familiar.}
{Ability to acquire a new familiar, sufficiently high level (see below).}
{When choosing a familiar, the creatures listed below are also available to the spellcaster. The spellcaster may choose a familiar with an alignment up to one step away on each of the alignment axes (lawful through chaotic, good through evil).

\Table{Improved Familiars}{p{2cm} X Z{1.2cm}}{
\tableheader Familiar & \tableheader Condition & \tableheader Arcane Spellcaster Level\\
Black/Gray Touched & Ability to channel energy from the Black or the Grey & 3rd\\
Boneclaw, lesser &  Neutral Alignment & 3rd\\
Pterrax & Reptilian subtype or ability to manifest psionic powers & 3rd\\
Element-touched\footnotemark[1] & Matching subtype or patron element, or preserver & 5th\\
Paraelement-touched\footnotemark[1] & Matching subtype or patron element, or defiler & 5th\\
Tagster & Preserver & 5th\\
Dagorran & Neutral alignment & 5th\\
Elemental, Small & Matching subtype or patron element, or preserver & 5th\\
Paraelemental, Small & Matching subtype or patron element, or defiler & 5th\\
Boneclaw, greater & Neutral alignment or ability to manifest psionic powers & 7th\\
Tigone & Neutral alignment or preserver & 7th\\
Tembo & Defiler or ability to manifest psionic powers & 7th\\
Wall walker & Neutral alignment or defiler & 7th\\
Psionocus & Ability to manifest psionic powers (The master must first create the psionocus.) & 7th\\

\TableNote{3}{1 Apply the template to a familiar from the standard list.}\\
}



Improved familiars otherwise use the rules for regular familiars, with two exceptions: If the creature's type is something other than animal, its type does not change; and improved familiars do not gain the ability to speak with other creatures of their kind (although many of them already have the ability to communicate).}{}{}

\Feat{Improved Natural Armor}
{}
{Natural armor, Con 13.}
{Your natural armor bonus increases by 1.}
{}
{You can gain this feat multiple times. Its effects stack. Each time you take this feat, your natural armor bonus increases by another point.}

\Feat{Improved Natural Attack}
{}
{Natural weapon, base attack bonus +4.}
{Choose one of the your natural attack forms. The damage for this natural weapon increases by one step, as if the your size had increased by one category: 1d2, 1d3, 1d4, 1d6, 1d8, 2d6, 3d6, 4d6, 6d6, 8d6, 12d6.

A weapon or attack that deals 1d10 points of damage increases as follows: 1d10, 2d8, 3d8, 4d8, 6d8, 8d8, 12d8.}
{}
{You can gain this feat multiple times. Its effects do not stack. Each time you take the feat, it applies to a different natural attack.

You may choose your unarmed strike as natural attack.}

\Feat{Improved Sigil}
{You templar sigil has been imbued with special powers.}
{Sigil ability, \skill{Diplomacy} 9 ranks.}
{Choose two 1st-level divine spells on your spell list. You can use them once per day as a spell-like ability. You must grasp and hold your sigil to use this ability. The save DC for these spell-like abilities is 11 + your Charisma modifier.}{}{}

\GFeat{Improved Turning}
{Ability to turn or rebuke creatures.}
{You turn or rebuke creatures as if you were one level higher than you are in the class that grants you the ability.}

\Feat{Improviser}
{You are adept at using makeshift weapons.}
{Wis 13, base attack bonus +3.}
{Whenever using improvised weapons in combat, you suffer a $-1$ penalty on attack rolls made with them.}
{Whenever using improvised weapons in combat, you suffer a $-4$ penalty on attack rolls made with them.}{}

\Feat{Innate Hunter}
{You are an excellent hunter, capable to find sustenance even in the most desolate areas.}
{\feat{Track}, \skill{Survival} 5 ranks.}
{You receive a +4 insight bonus on \skill{Survival} checks involving hunting. You also receive a +1 insight bonus on attack rolls when fighting with creatures with the animal type.}{}{}

\GFeat{Iron Will}{}
{You get a +2 bonus on all Will saving throws.}

% \GFeat{Jaguar Roar}
% {Cha 13, Draj, \skill{Intimidate} 9 ranks.}
% {Making a jaguar roar is a swift action. All intelligent creatures who can hear you and who are within 9 meters may become shaken for 2d4 rounds. A creature in the affected area can resist the effect with a successful Will save (DC 10 + your level + Charisma modifier). Any creature that successfully resists the effect cannot be affected again by the same character's jaguar roar for 24 hours.}

\Feat{Kiltektet}
{The Kiltektet is a group consisting mostly, but not solely, of kreen dedicated to hunting for knowledge and spreading it.}{}
{All Knowledge skills are class skills for you.}{}{}

\GFeat{Leadership}
{Character level 6th.}
{Having this feat enables the character to attract loyal companions and devoted followers, subordinates who assist her. See the table below for what sort of cohort and how many followers the character can recruit.

\Table{Leadership}{p{2cm} C *{6}{Z{.4cm}}}{
\rowcolor{white}
\multirow{2}{2cm}{\tableheader Leadership Score} & \multirow{2}{*}{\tableheader Cohort Level} & \multicolumn{6}{c}{\tableheader Number of Followers by Level}\\
\cmidrule[0.5pt]{3-8}
             &      & 1st & 2nd & 3rd & 4th & 5th & 6th \\
1 or lower   &      &     &     &     &     &     &     \\
2            &  1st &     &     &     &     &     &     \\
3            &  2nd &     &     &     &     &     &     \\
4            &  3rd &     &     &     &     &     &     \\
5            &  3rd &     &     &     &     &     &     \\
6            &  4th &     &     &     &     &     &     \\
7            &  5th &     &     &     &     &     &     \\
8            &  5th &     &     &     &     &     &     \\
9            &  6th &     &     &     &     &     &     \\
10           &  7th &   5 &     &     &     &     &     \\
11           &  7th &   6 &     &     &     &     &     \\
12           &  8th &   8 &     &     &     &     &     \\
13           &  9th &  10 &   1 &     &     &     &     \\
14           & 10th &  15 &   1 &     &     &     &     \\
15           & 10th &  20 &   2 &  1  &     &     &     \\
16           & 11th &  25 &   2 &  1  &     &     &     \\
17           & 12th &  30 &   3 &  1  &  1  &     &     \\
18           & 12th &  35 &   3 &  1  &  1  &     &     \\
19           & 13th &  40 &   4 &  2  &  1  &  1  &     \\
20           & 14th &  50 &   5 &  3  &  2  &  1  &     \\
21           & 15th &  60 &   6 &  3  &  2  &  1  &  1  \\
22           & 15th &  75 &   7 &  4  &  2  &  2  &  1  \\
23           & 16th &  90 &   9 &  5  &  3  &  2  &  1  \\
24           & 17th & 110 &  11 &  6  &  3  &  2  &  1  \\
25 or higher & 17th & 135 &  13 &  7  &  4  &  2  &  2  \\
}

\textit{Leadership Score:} A character's base Leadership score equals his level plus any Charisma modifier. In order to take into account negative Charisma modifiers, this table allows for very low Leadership scores, but the character must still be 6th level or higher in order to gain the Leadership feat. Outside factors can affect a character's Leadership score, as detailed above.

\textit{Cohort Level:} The character can attract a cohort of up to this level. Regardless of a character's Leadership score, he can only recruit a cohort who is two or more levels lower than himself. The cohort should be equipped with gear appropriate for its level. A character can try to attract a cohort of a particular race, class, and alignment. The cohort's alignment may not be opposed to the leader's alignment on either the law-vs-chaos or good-vs-evil axis, and the leader takes a Leadership penalty if he recruits a cohort of an alignment different from his own.

Cohorts earn XP as follows:

\begin{enumerate*}
\item The cohort does not count as a party member when determining the party's XP.
\item Divide the cohort's level by the level of the PC with whom he or she is associated (the character with the Leadership feat who attracted the cohort).
\item Multiply this result by the total XP awarded to the PC and add that number of experience points to the cohort's total.
\end{enumerate*}

If a cohort gains enough XP to bring it to a level one lower than the associated PC's character level, the cohort does not gain the new level---its new XP total is 1 less than the amount needed attain the next level.

\textit{Number of Followers by Level:} The character can lead up to the indicated number of characters of each level. Followers are similar to cohorts, except they're generally low-level NPCs. Because they're generally five or more levels behind the character they follow, they're rarely effective in combat.

Followers don't earn experience and thus don't gain levels. However, when a character with Leadership attains a new level, the player consults the table above to determine if she has acquired more followers, some of which may be higher level than the existing followers. (You don't consult the table to see if your cohort gains levels, however, because cohorts earn experience on their own.)

\textit{Leadership Modifiers:} Several factors can affect a character's Leadership score, causing it to vary from the base score (character level + Cha modifier). A character's reputation (from the point of view of the cohort or follower he is trying to attract) raises or lowers his Leadership score, see \tabref{Reputation}.

\Table{Reputation}{X c}{
\tableheader Leader's Reputation & \tableheader Modifier\\
Great renown & +2\\
Fairness and generosity & +1\\
Special power & +1\\
Failure & $-1$\\
Aloofness & $-1$\\
Cruelty & $-2$
}

Other modifiers may apply when the character tries to attract a cohort, see \tabref{Attracting Cohorts}.

\Table{Attracting Cohorts}{X c}{
\tableheader The Leader... & \tableheader Modifier\\
Has a familiar, special mount, or animal companion & $-2$\\
Recruits a cohort of a different alignment & $-1$\\
Caused the death of a cohort & $-2$ per cohort killed
}

Followers have different priorities from cohorts. When the character tries to attract a new follower, use any of the modifiers that apply on \tabref{Attracting Followers}.

\Table{Attracting Followers}{X c}{
\tableheader The Leader... & \tableheader Modifier\\
Has a stronghold, base of operations, guildhouse, or the like & +2\\
Moves around a lot & $-1$\\
Caused the death of other followers & $-1$
}}

\GFeat{Lightning Reflexes}{}
{You get a +2 bonus on all Reflex saving throws.}

\Feat{Linguist}
{You have an ear for language.}{}
{Speak Language is a class skill to you. You can also speak 2 additional languages.}{}
{This feat must be selected at 1st level.}

\GFeat{Martial Performer}
{Bardic music, martial prowess.}
{
Your bard and fighter levels stack for the purpose of determining your bonus from any martial prowess technique. For example, a 4th-level bard/8th-level fighter with bravery and tortoise style would gain +6 on Will saves against mind-affecting abilities and would improve the shield bonus by 3, as if he was a 12th-level fighter.

Your bard and fighter levels also stack for the purpose of determining the bonuses given by inspire courage. For example, a 4th-level bard/8th-level fighter would give +2 morale bonus with inspire courage, as if he was a 12th-level bard.

Additionally, you may forgo your attack with the lowest base attack bonus in a total attack to start or concentrate on a bardic music.
}

% \GFeat{Martial Psionicist}
% {Martial prowess, psychic warrior level 1st.}
% {
% Your fighter and psychic warrior levels stack for the purpose of determining your bonus from any martial prowess technique. For example, a 5th-level fighter/1st-level psychic warrior with tortoise style would improve the shield bonus by 3, as if he was a 6th-level fighter.

% Your fighter and psychic warrior levels also stack for the purpose of determining your manifester level. For example, a 5th-level fighter/1st-level psychic warrior with a Wisdom score of 16 would have 9 additional power points from his ability score and could spend 6 power points in a single psionic power, as if he was a 6th-level psychic warrior.
% }

\Feat{Martial Weapon Proficiency}
{Choose a type of martial weapon. You understand how to use that type of martial weapon in combat.}
{}
{You make attack rolls with the selected weapon normally.}
{When using a weapon with which you are not proficient, you take a $-4$ penalty on attack rolls.}
{Barbarians, fighters, gladiators, psychic warriors, and rangers are proficient with all martial weapons. They need not select this feat.

You can gain Martial Weapon Proficiency multiple times. Each time you take the feat, it applies to a new type of weapon.

A Hamanu's templar, because of the the War domain, automatically gains the Martial Weapon Proficiency feat related to his sorcerer-monarchs's favored weapon as a bonus feat, the longsword. He need not select it.}

\Feat{Mastyrial Blood}
{You have an uncanny resistance against toxic substances.}
{Con 13.}
{You receive a +4 bonus on saving throws against poison.}{}
{This feat must be selected at 1st level.}

\Feat{Mental Resistance}
{Your mind is armored against mental intrusion.}
{Base Will save bonus +2.}
{Against psionic attacks that do not employ an energy type to deal damage you gain damage reduction 3/--. In addition, when you are hit with ability damage (but not ability drain or ability burn damage) from a psionic attack, you take 3 points less than you would normally take.

The benefit of this feat applies only to psionic powers and psi-like abilities. %This is an exception to the psionics-magic transparency rule.
}{}
{You cannot take or use this feat if you have the ability to use powers (if you have a power point reserve or psi-like abilities).}

\Feat{Mind Over Body}
{Your ability damage heals more rapidly.}
{Con 13.}
{You heal ability damage and ability burn damage more quickly than normal. You heal a number of ability points per day equal to 1 + your Constitution bonus.}
{You heal ability damage and ability burn damage at a rate of 1 point per day.}{}

\Feat{Multiattack}
{}
{Three or more natural attacks.}
{The creature's secondary attacks with natural weapons take only a $-2$ penalty.}
{Without this feat, the creature's secondary attacks with natural weapons take a $-5$ penalty.}
{}

\Feat{Multiweapon Fighting}
{}
{Dex 13, three or more hands.}
{Penalties for fighting with multiple weapons are reduced by 2 with the primary hand and reduced by 6 with off hands.}
{A creature without this feat takes a $-6$ penalty on attacks made with its primary hand and a $-10$ penalty on attacks made with its off hands. (It has one primary hand, and all the others are off hands.) See \feat{Two-Weapon Fighting}.}
{This feat replaces the \feat{Two-Weapon Fighting} feat for creatures with more than two arms.}

\GFeat{Natural Spell}
{Wis 13, wild shape ability.}
{You can complete the verbal and somatic components of spells while in a wild shape. You substitute various noises and gestures for the normal verbal and somatic components of a spell.

You can also use any material components or focuses you possess, even if such items are melded within your current form. This feat does not permit the use of magic items while you are in a form that could not ordinarily use them, and you do not gain the ability to speak while in a wild shape.}

\Feat{Open Minded}
{You are naturally able to reroute your memory, mind, and skill expertise.}{}
{You immediately gain an extra 5 skill points. You spend these skill points as normal. If you spend them on a cross-class skills they count as onehalf ranks. You cannot exceed the normal maximum ranks for your level in any skill.}{}
{You can gain this feat multiple times. Each time, you immediately gain another 5 skill points.}

\Feat{Protective}
{You know that your gear could save your life, and you will do anything to protect it.}{}
{Gear on your person gains a +4 bonus to saving throws. If an item takes damage while you're holding it in your hands, you may make a Reflex save DC 10 + the amount of damage the item takes (after subtracting hardness) to transfer the damage to yourself.}{}{}

\Feat{Psionic Hole}
{You are anathema to psionic creatures and characters.}
{Con 15.}
{When a foe strikes you in melee combat, the foe immediately loses its psionic focus, if any. Also, if you are the target of a power, the manifester of the power must spend an additional number of power points equal to your Wisdom bonus, or the power fails (all the power points spent on the power are still lost). This extra cost does not count toward the maximum power points a manifester can spend on a single power.}{}
{You cannot take or use this feat if you have the ability to use powers (if you have a power point reserve or psi-like abilities).}

\Feat{Psionic Mimicry}
{Due to your study of psionic powers, you can pass off your spells as such.}
{\skill{Bluff} 8 ranks, \skill{Knowledge} (psionics) 4 ranks, \skill{Psicraft} 4 ranks.}
{You can disguise your spells as psionic powers by making a successful \skill{Bluff} check (DC 10 + spell level). An onlooker suspecting the nature of your spellcasting can attempt to identify a spell being cast using the \skill{Spellcraft} skill, but your check DC increases by 2.}{}{}

\Feat{Psionic Schooling}
{In your homeland, all who show some skill in the Way may receive training as a psion.}{}
{Psion, psychic warrior, or wilder is now a favored class for you (pick one), in addition to any other favored class you already possess. It does not count when determining multiclass XP penalties.}
{A character can have one favored class.}
{This feat must be selected at 1st level.}

% \GFeat{Psychic Hunter}
% {Favored enemy, psychic warrior level 1st.}
% {
% Your ranger and psychic warrior levels stack for the purpose of determining your manifester level. For example, a 5th-level ranger/1st-level psychic warrior with a Wisdom score of 16 would have 9 additional power points from his ability score and could spend 6 power points in a single psionic power, as if he was a 6th-level psychic warrior.

% As a swift action, you can sacrifice any number of power points (up to a maximum equal to your manifester level) to add a bonus to your attack and damage rolls against a favored enemy. This bonus is equal to half that number.
% }

\Feat{Quicken Spell-Like Ability}
{}
{Spell-like ability at caster level 10th or higher.}
{Choose one of the creature's spell-like abilities, subject to the restrictions described below. The creature can use that ability as a quickened spell-like ability three times per day (or less, if the ability is normally usable only once or twice per day).

Using a quickened spell-like ability is a swift action that does not provoke an attack of opportunity. The creature can perform another action---including the use of another spell-like ability---in the same round that it uses a quickened spell-like ability. The creature may use only one quickened spell-like ability per round.

The creature can only select a spell-like ability duplicating a spell with a level less than or equal to half its caster level (round down) $-4$. For a summary, see the associated table.

In addition, a spell-like ability that duplicates a spell with a casting time greater than 1 full round cannot be quickened.}
{Normally the use of a spell-like ability requires a standard action and provokes an attack of opportunity unless noted otherwise.}
{This feat can be taken multiple times. Each time it is taken, the creature can apply it to a different one of its spell-like abilities.}

\Feat{Raised by Beasts}
{Abandoned when you were very young, you were raised by wild animals.}{}
{Choose a kind of animal (amphibian, avian, mammal, fish, or reptile). You receive the wild empathy ability with animals of that kind. You also receive a +2 insight bonus on all \skill{Handle Animal} checks with animals of that kind.}{}
{This feat must be selected at 1st level.}

\Feat{Rapid Metabolism}
{Your wounds heal rapidly.}
{Con 13.}
{You naturally heal a number of hit points per day equal to the standard healing rate + double your Constitution bonus. You heal even if you do not rest. This healing replaces your normal natural healing. If you are tended successfully by someone with the Heal skill, you instead regain double the normal amount of hit points + double your Constitution bonus.}{}{}

\Feat{Reckless Offense}
{You can shift your focus from defense to offense.}
{Base attack bonus +1.}
{When you use the attack action or full attack action in melee, you can take a penalty of $-4$ to your Armor Class and add a +2 bonus on your melee attack roll. The bonus on attack rolls and penalty to Armor Class last until the beginning of your next turn.}{}{}

\GFeat{Reign of Terror}
{Raam, \skill{Intimidate} 5 ranks.}
{You gain a +4 bonus on \feat{Secular Authority} checks.}

\GFeat{Rogue Performer}
{Bardic music, sneak attack +2d6.}
{
Your bard and rogue levels stack for the purpose of determining the number of times per day that you can use your bardic music. For example, a 3rd-level bard/5th-level rogue could use her bardic music 8 times per day, as if she were an 8th-level bard.

Your bard and rogue levels stack for the purpose of determining your sneak attack bonus damage. For example, a 3rd-level bard/5th-level rogue would deal an extra 4d6 points of damage with her sneak attack, as if she were an 8th-level rogue.
}

\Feat{Run}
{}{}
{When running, you move five times your normal speed (if wearing medium, light, or no armor and carrying no more than a medium load) or four times your speed (if wearing heavy armor or carrying a heavy load). If you make a jump after a running start (see the \skill{Jump} skill description), you gain a +4 bonus on your \skill{Jump} check. While running, you retain your Dexterity bonus to AC.}
{You move four times your speed while running (if wearing medium, light, or no armor and carrying no more than a medium load) or three times your speed (if wearing heavy armor or carrying a heavy load), and you lose your Dexterity bonus to AC.}{}

\Feat{Secular Authority}
{You can use your authority within your city-state to order slaves to do your bidding, requisition troops, enter the homes of freemen and nobles, and have them arrested.}
{Cha 13, \skill{Diplomacy} 6 ranks, \feat{Negotiator}, accepted into city-state's templarate.}
{This feat grants four new uses for the \skill{Diplomacy} skill. None of them functions during combat.

\textit{Requisition:} You can draw upon the resources of your city, gaining the use of any slave, overriding the wishes of its owner.

\textit{Intrude:} You can, at any time, search the home, person or possessions of a slave. You may search and impound any evidence of wrongdoing, if found. Your authority does not extend to confiscating items for personal use.

\textit{Accuse:} You may have a slave imprisoned indefinitely, awaiting the gathering of evidence against him. You may only imprison one suspect in such a manner.

\textit{Judge:} You may pass judgment on a slave. This includes setting fines, prison sentences, death sentences or anything else you wish, within the laws of your city-state.

As you gain more ranks in the \skill{Diplomacy} ranks, you gain the authority to take these actions against progressively higher social rankings, as described on the table below.

\Table{Secular Authority abilities}{c X}{
\tableheader Ranks & \tableheader Ability\\
2 & Requisition slave\\
3 & Intrude on slave\\
4 & Accuse slave\\
5 & Requisition troops\\
6 & Intrude on freeman\\
7 & Judge slave\\
8 & Accuse freeman\\
9 & Requisition gear\\
10 & Intrude on noble\\
11 & Judge freeman\\
12 & Accuse noble\\
13 & Requisition spellcaster/manifester\\
14 & Intrude on templar\\
15 & Judge noble\\
16 & Accuse templar\\
17 & Requisition property\\
18+ & Judge templar
}

Failure to comply with these demands is usually sanctioned with fines, imprisonment, outlaw status, and possibly execution. Any of this ability can be contested by another person with the Secular Authority feat, and move to have the action reversed with an opposed \skill{Diplomacy} check. If the challenger wins the opposed roll, the defending templar's action is reversed (for example an imprisoned freeman is set free). If the defender wins the opposed roll nothing happens. Secular Authority can be contested in a particular case only once. A defending character who loses the opposed roll may not contest the result. Nor can he use Secular Authority to repeat the action that was contested against the same target.

You may use Secular Authority once per day for every four levels you have attained (but see Special), but only
within your city-state.}
{}
{A templar automatically gains Secular Authority as a bonus feat. He need not select it. A templar may use Secular Authority a number of times per day equal to his templar level, plus one more time per day for every four levels he has in classes other than templar.}

\Feat{Shield Proficiency}
{}{}
{You can use a shield and take only the standard penalties.}
{When you are using a shield with which you are not proficient, you take the shield's armor check penalty on attack rolls and on all skill checks that involve moving, including \skill{Ride} checks.}
{Barbarians, clerics, druids, fighters, gladiators, psychic warriors, rangers, and templars automatically have Shield Proficiency as a bonus feat. They need not select it.}

\Feat{Sidestep Charge}
{You are skilled at dodging past charging opponents and taking advantage when they miss.}
{Dex 13, \feat{Dodge}.}
{You get a +4 dodge bonus to Armor Class against charge attacks. If a charging opponent fails to make a successful attack against you, you gain an immediate attack of opportunity. This feat does not grant you more attacks of opportunity than you are normally allowed in a round. If you are flat-footed or otherwise denied your Dexterity bonus to Armor Class, you do not gain the benefit of this feat.}{}{}

\Feat{Simple Weapon Proficiency}
{}{}
{You make attack rolls with simple weapons normally.}
{When using a weapon with which you are not proficient, you take a $-4$ penalty on attack rolls.}
{All characters except for druids, psions, and wizards are automatically proficient with all simple weapons. They need not select this feat.}

\Feat{Sniper}
{You are better at hiding when firing missile weapons and trying to stay hidden.}
{Dex 13, \skill{Hide} 1 rank.}
{You receive a +5 competence bonus to \skill{Hide} checks when firing missiles while trying to stay hidden.}{}{}

\Feat{Skill Focus}
{Choose a skill.}{}
{You get a +3 bonus on all checks involving that skill. This skill is treated as a class skill in all respects for all classes you have levels in, both current and future.}{}
{You can gain this feat multiple times. Its effects do not stack. Each time you take the feat, it applies to a new skill.}

\Feat{Spell Focus}
{Choose a school of magic.}{}
{Add +1 to the Difficulty Class for all saving throws against spells from the school of magic you select.}{}
{You can gain this feat multiple times. Its effects do not stack. Each time you take the feat, it applies to a new school of magic.}

\Feat[Special]{Spell Mastery}{}
{Wizard level 1st.}
{Each time you take this feat, choose a number of spells equal to your Intelligence modifier that you already know. From that point on, you can prepare these spells without referring to a spellbook.}
{Without this feat, you must use a spellbook to prepare all your spells, except read magic.}{}

\GFeat{Spell Penetration}{}
{You get a +2 bonus on caster level checks (1d20 + caster level) made to overcome a creature's spell resistance.}

\Feat{Stand Still}
{You can prevent foes from fleeing or closing.}
{Str 13.}
{When a foe's movement out of a square you threaten grants you an attack of opportunity, you can give up that attack and instead attempt to stop your foe in his tracks. Make your attack of opportunity normally. If you hit your foe, he must succeed on a Reflex save against a DC of 10 + your damage roll (the opponent does not actually take damage), or immediately halt as if he had used up his move actions for the round.

Since you use the Stand Still feat in place of your attack of opportunity, you can do so only a number of times per round equal to the number of times per round you could make an attack of opportunity (normally just one).}
{Attacks of opportunity cannot halt your foes in their tracks.}{}

\Feat{Toughness}
{}{}
{You gain +3 hit points.}{}
{A character may gain this feat multiple times. Its effects stack.}

\Feat{Tower Shield Proficiency}
{}
{\feat{Shield Proficiency}.}
{You can use a tower shield and suffer only the standard penalties.}
{A character who is using a shield with which he or she is not proficient takes the shield's armor check penalty on attack rolls and on all skill checks that involve moving, including Ride.}
{Fighters automatically have Tower Shield Proficiency as a bonus feat. They need not select it.}

\Feat{Track}
{}{}
{To find tracks or to follow them for 1.5 kilometer requires a successful \skill{Survival} check. You must make another \skill{Survival} check every time the tracks become difficult to follow.

You move at half your normal speed (or at your normal speed with a $-5$ penalty on the check, or at up to twice your normal speed with a $-20$ penalty on the check). The DC depends on the surface and the prevailing conditions, as given on \tabref{Track DC}.


\Table{Track DC}{X Z{1.4cm} X Z{1.4cm}}{
\tableheader Surface & \tableheader \skill{Survival} DC & \tableheader Surface & \tableheader \skill{Survival} DC\\
Very soft ground & 5 & Firm ground & 15\\
Soft ground & 10 & Hard ground & 20
}

\textit{Very Soft Ground:} Any surface (fresh snow, thick dust, wet mud) that holds deep, clear impressions of footprints.

\textit{Soft Ground:} Any surface soft enough to yield to pressure, but firmer than wet mud or fresh snow, in which a creature leaves frequent but shallow footprints.

\textit{Firm Ground:} Most normal outdoor surfaces (such as lawns, fields, woods, and the like) or exceptionally soft or dirty indoor surfaces (thick rugs and very dirty or dusty floors). The creature might leave some traces (broken branches or tufts of hair), but it leaves only occasional or partial footprints.

\textit{Hard Ground:} Any surface that doesn't hold footprints at all, such as bare rock or an indoor floor. Most streambeds fall into this category, since any footprints left behind are obscured or washed away. The creature leaves only traces (scuff marks or displaced pebbles).

Several modifiers may apply to the \skill{Survival} check, as given on \tabref{Track DC Modifiers}.

\Table{Track DC Modifiers}{X Z{1.4cm}}{
\tableheader Condition & \tableheader \skill{Survival} DC Modifier\\
Every three creatures in the group being tracked & $-1$\\
Size of creature or creatures being tracked & \\
~ Fine & +8\\
~ Diminutive & +4\\
~ Tiny & +2\\
~ Small & +1\\
~ Medium & 0\\
~ Large & $-1$\\
~ Huge & $-2$\\
~ Gargantuan & $-4$\\
~ Colossal & $-8$\\
Every 24 hours since the trail was made & +1\\
Every hour of rain since the trail was made & +1\\
Fresh snow cover since the trail was made & +10\\
Poor visibility (Apply only the largest modifier from this category.) & \\
~ Overcast or moonless night & +6\\
~ Moonlight & +3\\
~ Fog or precipitation & +3\\
Tracked party hides trail (and moves at half speed) & +5
}

For a group of mixed sizes, apply only the modifier for the largest size category.

If you fail a \skill{Survival} check, you can retry after 1 hour (outdoors) or 10 minutes (indoors) of searching.}
{Without this feat, you can use the \skill{Survival} skill to find tracks, but you can follow them only if the DC for the task is 10 or lower. Alternatively, you can use the \skill{Search} skill to find a footprint or similar sign of a creature's passage using the DCs given above, but you can't use Search to follow tracks, even if someone else has already found them.}
{A ranger automatically has Track as a bonus feat. He need not select it.

This feat does not allow you to find or follow the tracks made by a subject of a pass without trace spell.}

\Feat{Urban Tracking}
{You can track down the location of missing persons or wanted individuals within communities.}{}
{
To find the trail of an individual or to follow it for 1 hour requires a \skill{Gather Information} check. You must make another \skill{Gather Information} check every hour of the search, as well as each time the trail becomes difficult to follow, such as when it moves to a different area of town.

\Table{Urban Tracking DC}{Xcl}{
\tableheader Community Size & \tableheader DC & \tableheader Checks Required\\
Thorp, hamlet, or village & 5 & 1d3\\
Small or large town & 10 & 1d4+1\\
Small or large city & 15 & 2d4\\
Metropolis & 20 & 2d4+2\\
}

The DC of the check, and the number of checks required to track down your quarry, depends on the community size and the conditions:

\Table{Urban Tracking Modifiers}{X Z{1.4cm}}{
\tableheader Conditions & \tableheader DC Modifier\\
Every three creatures in the group being sought & $-1$\\
Every 24 hours party has been missing/sought & +1\\
Tracked party ``lies low'' & +5\\
Tracked party matches community's primary racial demographic & +2\\
Tracked party does not match community's primary, or secondary racial demographic & -2\\
}
If you fail a \skill{Gather Information} check, you can retry after 1 hour of questioning. The DM should roll the number of checks required secretly, so that the player doesn't know exactly how long the task will require.

}
{A character without this feat can use \skill{Gather Information} to find out information about a particular individual, but each check takes 1d4+1 hours and doesn't allow effective trailing.}
{A character with 5 ranks in \skill{Knowledge} (local) gains a +2 bonus on the \skill{Gather Information} check to use this feat.

You can cut the time between \skill{Gather Information} checks in half (to 30 minutes per check rather than 1 hour), but you take a $-5$ penalty on the check.}

\Feat{Wastelander}
{You are an experienced survivor of the wastes.}{}
{You get a +1 bonus to Fortitude saves and a +2 bonus to \skill{Survival} checks.}{}{}

\Feat{Wild Talent}
{Your mind wakes to a previously unrealized talent for psionics.}
{}
{Your latent power of psionics flares to life, conferring upon you the designation of a psionic character. You learn one psionic power that does not have prerequistes. You must have the relevant key ability score 10 + the power's level (see the \class{Psion} class).

As a psionic character, you gain a reserve of power points equal to 5 $\times$ the power's level. If the chosen power can be maintained, you gain additional power points equal to 4 $\times$ the power's level.

You can take psionic feats, metapsionic feats, and psionic item creation feats.}
{}
{You cannot take or use this feat if you have the ability to use powers or cast spells.

This feat must be taken at 1st level.}
% {Your mind wakes to a previously unrealized talent for psionics.}{}
% {Your latent power of psionics flares to life, conferring upon you the designation of a psionic character. As a psionic character, you gain a reserve of 2 power points and can take psionic feats, metapsionic feats, and psionic item creation feats. You do not, however, gain the ability to manifest powers simply by virtue of having this feat.}{}{}

\GFeat{Wingover}
{Fly speed.}
{A flying creature with this feat can change direction quickly once each round as a free action. This feat allows it to turn up to 180 degrees regardless of its maneuverability, in addition to any other turns it is normally allowed. A creature cannot gain altitude during a round when it executes a wingover, but it can dive.

The change of direction consumes 3 meters of flying movement.}

\FeatTable[p{3cm}]{Skill}{
	\feat{Acrobatic} && +2 bonus on \skill{Jump} and \skill{Tumble} checks \\
	\feat{Agile} && +2 bonus on \skill{Balance} and \skill{Escape Artist} checks \\
	\feat{Alertness} && +2 bonus on \skill{Listen} and \skill{Spot} checks \\
	\feat{Animal Affinity} && +2 bonus on \skill{Handle Animal} and \skill{Ride} checks \\
	\feat{Athletic} && +2 bonus on \skill{Climb} and \skill{Swim} checks \\
	\feat{Autonomous} && +2 bonus on \skill{Autohypnosis} and \skill{Knowledge} (psionics) checks \\
	\feat{Deceitful} && +2 bonus on \skill{Disguise} and \skill{Forgery} checks \\
	\feat{Deft Hands} && +2 bonus on \skill{Sleight of Hand} and \skill{Use Rope} checks \\
	\feat{Diligent} && +2 bonus on \skill{Appraise} and \skill{Decipher Script} checks \\
	\feat{Field Officer} && +2 bonus on \skill{Diplomacy} and \skill{Knowledge} (warcraft) checks \\
	\feat{Investigator} && +2 bonus on \skill{Gather Information} and \skill{Search} checks \\
	\feat{Magical Aptitude} && +2 bonus on \skill{Spellcraft} and \skill{Use Magic Device} checks \\
	\feat{Negotiator} && +2 bonus on \skill{Diplomacy} and \skill{Sense Motive} checks \\
	\feat{Nimble Fingers} && +2 bonus on \skill{Disable Device} and \skill{Open Lock} checks \\
	\feat{Persuasive} && +2 bonus on \skill{Bluff} and \skill{Intimidate} checks \\
	\feat{Psionic Affinity} && +2 bonus on \skill{Psicraft} and \skill{Use Psionic Device} checks \\
	\feat{Self-Sufficient} && +2 bonus on \skill{Heal} and \skill{Survival} checks \\
	\feat{Stealthy} && +2 bonus on \skill{Hide} and \skill{Move Silently} checks \\
	\feat{Trader} && +2 bonus on \skill{Appraise} and \skill{Bluff} checks
}

\FeatTable[p{3cm}]{Racial}{
	\feat{Active Glands} & Thri-kreen & Use poison two additional times per day\\
	\feat{Advanced Antennae} & Thri-kreen & Gain scent ability\\
	\feat{Blend} & Thri-kreen & +3 on \skill{Hide} checks in sandy or arid terrain\\
	\feat{Blessed by the Ancestors} & Thri-kreen & +1 bonus on all saves\\
	\feat{Cannibalism Ritual} & Wis 13, halfling & Gain ability bonus for 1 day after devouring slain foe\\
	\feat{Dwarven Vision} & Mul & Gain darkvision 18 m\\
	\feat{Elfeater} & Thri-kreen & +1 on attack rolls and +2 on some skill checks against elves\\
	\feat{Improved Gyth'sa} & Thri-kreen, Con 13 & Recover double hit points after a night's rest\\
	\feat{Longshanks} & Half-elf, both parents must be half-elves & +3 m speed\\
	\feat{Tikchak} & Thri-kreen, character level 5th & Add Wis to \skill{Survival} checks, gain proficiency with chatkcha\\
	\feat{Tokchak} & Thri-kreen & Adjacent allies gain +1 bonus on Ref saves
}

% \FeatTable[p{3cm}]{Regional}{
% 	\feat{Artisan} & {Nibenay, Raam, Urik} & +3 on \skill{Concentration} and \skill{Craft} checks\\
% 	\feat{Astrologer} & {Draj, Nibenay} & +3 on \skill{Knowledge} (nature) checks, +5 to avoid getting lost\\
% 	\feat{Companion} & {Kurn, Tyr} & Your aid another action grants +3 bonus\\
% 	\feat{Disciplined} & {Dwarf, Urik} & +1 to Will saves, +3 on \skill{Concentration} checks\\
% 	\feat{Elfish Eloy} & Half-elf, both parents must be half-elves & +3 on \skill{Hide} while aboveground\\
% 	\feat{Freedom} & {Tyr} & Take extra actions per day\\
% 	\feat{Giant Killer} & {Sea of Silt} & +4 to confirm criticals and +2 to AC against giants\\
% 	\feat{Jungle Fighter} & {Forest Ridge, Gulg} & +2 dodge bonus to AC in forests\\
% 	\feat{Legerdemain} & {Elf, Salt View} & +3 bonus on \skill{Open Lock} and \skill{Sleight of Hand} checks\\
% 	\feat{Mansabdar} & {Raam} & +1 to Fort saves, +3 bonus on \skill{Intimidate} checks\\
% 	\feat{Mekillothead} & {Draj, mul} & +1 to Will saves, +3 bonus on \skill{Intimidate} checks\\
% 	\feat{Metalsmith} & {Dwarf, Tyr} & You do not suffer $-5$ penalty to forge metal items\\
% 	\feat{Nature's Child} & {Gulg, halfling} & +3 on \skill{Knowledge} (nature) and \skill{Survival} checks\\
% 	\feat{Paranoid} & {Eldaarich} &  +1 to Ref saves, +3 bonus on \skill{Sense 
% 	motive} checks\\
% 	\feat{Performance Artist} & {Balic, Nibenay, Salt View} & +3 bonus on \skill{Knowledge} (local) and \skill{Perform} checks\\
% 	\feat{Tarandan Method} & {Raam} & +2 DC to powers from chosen discipline
% }
\subsectionA{Skill Feats}

\GFeat[Skill]{Acrobatic}{}
{You get a +2 bonus on all \skill{Jump} checks and \skill{Tumble} checks.}

\GFeat[Skill]{Agile}{}
{You get a +2 bonus on all \skill{Balance} checks and \skill{Escape Artist} checks.}

\Feat[Skill]{Alertness}{}{}
{You get a +2 bonus on all \skill{Listen} checks and \skill{Spot} checks.}
{}
{The master of a familiar gains the benefit of the Alertness feat whenever the familiar is within arm's reach.}

\GFeat[Skill]{Animal Affinity}{}
{You get a +2 bonus on all \skill{Handle Animal} checks and \skill{Ride} checks.}

\GFeat[Skill]{Athletic}{}
{You get a +2 bonus on all \skill{Climb} checks and \skill{Swim} checks.}

\GFeat[Skill]{Autonomous}{}
{You get a +2 bonus on all \skill{Autohypnosis} checks and \skill{Knowledge} (psionics) checks.}

\GFeat[Skill]{Deceitful}{}
{You get a +2 bonus on all \skill{Disguise} checks and \skill{Forgery} checks.}

\GFeat[Skill]{Deft Hands}{}
{You get a +2 bonus on all \skill{Sleight of Hand} checks and \skill{Use Rope} checks.}

\GFeat[Skill]{Diligent}{}
{You get a +2 bonus on all \skill{Appraise} checks and Decipher \skill{Script} checks.}

\GFeat[Skill]{Field Officer}{}
{You get a +2 bonus on all \skill{Diplomacy} checks and \skill{Knowledge} (warcraft) checks.}

\GFeat[Skill]{Investigator}{}
{You get a +2 bonus on all \skill{Gather Information} checks and \skill{Search} checks.}

\GFeat[Skill]{Magical Aptitude}{}
{You get a +2 bonus on all \skill{Spellcraft} checks and \skill{Use Magic Device} checks.}

\GFeat[Skill]{Negotiator}{}
{You get a +2 bonus on all \skill{Diplomacy} checks and \skill{Sense Motive} checks.}

\GFeat[Skill]{Nimble Fingers}{}
{You get a +2 bonus on all \skill{Disable Device} checks and \skill{Open Lock} checks.}

\GFeat[Skill]{Persuasive}{}
{You get a +2 bonus on all \skill{Bluff} checks and \skill{Intimidate} checks.}

\GFeat[Skill]{Psionic Affinity}{}
{You get a +2 bonus on all \skill{Psicraft} checks and \skill{Use Psionic Device} checks.}

\GFeat[Skill]{Self-Sufficient}{}
{You get a +2 bonus on all \skill{Heal} checks and \skill{Survival} checks.}

\GFeat[Skill]{Stealthy}{}
{You get a +2 bonus on all \skill{Hide} checks and \skill{Move Silently} checks.}

\GFeat[Skill]{Trader}{}
{You get a +2 bonus on all \skill{Appraise} checks and \skill{Bluff} checks.}
\subsectionA{Racial Feats}

\Feat[Racial]{Active Glands}
{Your venom glands are particularly active.}
{Character level 5th, thri-kreen.}
{The Fortitude DC of your poison is 10 + \onehalf your class levels + Con modifier.}
{A thri-kreen poison Fortitude DC is fixed to 11 + Con modifier.}
{}

\Feat[Racial]{Advanced Antennae}
{Your antennae are more developed than those of your fellow thri-kreen, enabling you to detect both predators and prey near you.}
{Thri-kreen.}
{You gain the scent ability.}
{}{}

\Feat[Racial]{Blend}
{Your carapace meshes better with your surroundings.}
{Thri-kreen.}
{You gain advantage on \skill{Hide} checks in sandy or arid terrain.}
{}{}

\Feat[Racial]{Cannibalism Ritual}
{You know the secrets of ingesting a defeated opponent's flesh to improve your physical attributes.}
{Wis 17, halfling.}
{When you have slain a creature in combat with Hit Dice equal or superior to your character level you may devour its remains after performing a ritual alone that lasts two hours. If you do, you receive a +4 enhancement bonus to either Strength, Dexterity or Constitution for 24 hours. This effect is not cumulative.}
{}{}

\Feat[Racial]{Dwarven Vision}
{You are born with the full heritage of your dwarven parent's vision.}
{Mul.}
{You have darkvision up to 18 meters.}
{}{}

\Feat[Racial]{Hardened Body}
{You resist better to the Athasian weapons.}
{Con 15, mul.}
{You gain damage reduction 1/metal.}
{}{}

\Feat[Racial]{Improved Resilience}
{You are more resilient to magic than your brethren.}
{Con 17, dwarf or halfling.}
{You get spell resistance equal to 5 + your class levels.}
{}{}

\Feat[Racial]{Longshanks}
{You are as fast a full-blooded elf.}
{Dex 17, half-elf.}
{Your land speed is 3 meters faster.}
{}{}

\Feat[Racial]{Outlander}
{You have spent enough time in the wastelands to know your ways in the wild.}
{Half-elf.}
{You get advantage on \skill{Survival} checks made in aboveground natural environments.}
{}{}

\Feat[Racial]{Rock Throw}
{You have adopted the custom of hurling rocks as true giants.}
{Str 23, half-giant.}
{You can hurl rocks weighing 20 to 25 kilograms each (Small objects), with a range increment of 18 meters. These rocks deal 2d6 points of damage. As a thrown weapon, you can reach up to five range increments. You use only one hand to hurl rocks.}
{}
{}

% \section{Regional Feats}

\Feat[Regional]{Artisan}
{}
{Nibenay or Raam or Urik.}
{You get a +3 bonus on all \skill{Concentration} checks and one \skill{Craft} skill check.}
{}
{You may select this feat only at 1st level. You may only have one regional feat.}

\Feat[Regional]{Astrologer}
{}
{Draj or Nibenay.}
{You get a +3 bonus on all \skill{Knowledge} (nature) and a +5 bonus to \skill{Survival} checks made to avoid getting lost when able to see the sun, moon or stars.}
{}
{You may select this feat only at 1st level. You may only have one regional feat.}

\Feat[Regional]{Companion}
{}
{Kurn or Tyr.}
{When assisting on skill checks and using the aid another action you grant a +3 bonus.}
{When assisting on skill checks and using the Aid Another action you grant a +2 bonus.}
{You may select this feat only at 1st level. You may only have one regional feat.}

\Feat[Regional]{Disciplined}
{}
{Dwarf, Urik.}
{You get a +1 bonus to Will saves and a +3 bonus to \skill{Concentration} checks.}
{}
{You may select this feat only at 1st level. You may only have one regional feat.}

\Feat[Regional]{Elfish Eloy}
{}
{Half-elf, both parents must be half-elves}
{You receive the same natural resistance to extreme temperatures that regular elves have. In addition, you receive a +3 bonus to \skill{Hide} checks made in aboveground natural terrain.}
{}
{You may select this feat only at 1st level. You may only have one regional feat.}

\Feat[Regional]{Freedom}
{}
{Tyr.}
{You may take an extra move action or standard action in a round, which must be taken immediately (before you take any other actions). You may use this feat a number of times per day depending on your character level (as shown below), but never more than once per round.

\Table{}{C C}{
\tableheader Character Level & \tableheader Times per Day\\
1st--5th & 1 \\
6th--10th & 2 \\
11th--15th & 3 \\
16th--20th & 4
}}
{}
{You may select this feat only at 1st level. You may only have one regional feat.}

\Feat[Regional]{Giant Killer}
{}
{Sea of Silt.}
{You receive a +4 bonus on rolls to confirm criticals and a +2 dodge bonus to your AC when fighting against creatures with the giant type.}
{}
{You may select this feat only at 1st level. You may only have one regional feat.}

\Feat[Regional]{Jungle Fighter}
{}
{Forest Ridge or Gulg.}
{When fighting in forest terrain, you receive a +2 dodge bonus to AC.}
{}
{You may select this feat only at 1st level. You may only have one regional feat.}

\Feat[Regional]{Legerdemain}
{}
{Elf, Salt View.}
{You get a +3 bonus on all \skill{Open Lock} and \skill{Sleight of Hand} checks.}
{}
{You may select this feat only at 1st level. You may only have one regional feat.}

\Feat[Regional]{Mansabdar}
{}
{Raam.}
{You get a +3 bonus on all \skill{Intimidate} checks and a +1 bonus to Fortitude saves.}
{}
{You may select this feat only at 1st level. You may only have one regional feat.}

\Feat[Regional]{Mekillothead}
{}
{Draj, mul.}
{You get a +1 bonus to Will saves and a +3 bonus to Intimidate checks.}
{}
{You may select this feat only at 1st level. You may only have one regional feat.}

\Feat[Regional]{Metalsmith}
{}
{Dwarf, Tyr.}
{You suffer no penalty to \skill{Craft} checks when crafting items from metal. Put the item's price in silver pieces when calculating creation time.}
{You suffer a $-5$ penalty to \skill{Craft} checks when crafting items from metal. Put the item's price in ceramic pieces when calculating creation time.}
{You may select this feat only at 1st level. You may only have one regional feat.}

\Feat[Regional]{Nature's Child}
{}
{Gulg, halfling.}
{You get a +3 bonus on all \skill{Knowledge} (nature) and \skill{Survival} checks.}
{}
{You may select this feat only at 1st level. You may only have one regional feat.}

\Feat[Regional]{Paranoid}
{}
{Eldaarich.}
{You get a +3 bonus on all \skill{Sense Motive} checks and a +1 bonus to Reflex saves.}
{}
{You may select this feat only at 1st level. You may only have one regional feat.}

\Feat[Regional]{Performance Artist}
{}
{Balic or Nibenay or Salt View.}
{You get a +3 bonus to a specific type of \skill{Perform} checks and \skill{Knowledge} (local) checks for your region.}
{}
{You may select this feat only at 1st level. You may only have one regional feat.}

\Feat[Regional]{Tarandan Method}
{}
{Raam.}
{Add 2 to the save DC of powers from your chosen discipline.}
{}
{You may select this feat only at 1st level. You may only have one regional feat.}
\FeatTable[>{\raggedright}p{3.5cm}]{Fighter}{
	\feat{Blind-Fight} && Reduce penalties fighting invisible foes\\
	\feat{Combat Expertise} & Int 13 & Take penalty in attack rolls to increase AC\\
	~ \feat{Improved Disarm} & Int 13, \feat{Combat Expertise} & +4 on disarm rolls, no attack of opportunity\\
	~ \feat{Improved Feint} & Int 13, \feat{Combat Expertise} & Feint as a move action\\
	~ \feat{Improved Trip} & Int 13, \feat{Combat Expertise} & +4 on trip rolls, no attack of opportunity\\
	\feat{Combat Reflexes} && Make additional attacks of opportunity\\
	\feat{Commanding Presence} & \skill{Diplomacy} 7 ranks, \skill{Knowledge} (warcraft) 5 ranks & New use for \skill{Diplomacy}\\
	\feat{Concentrated Fire} & Base attack bonus +1 & +1 bonus on attack rolls when firing with a squad \\
	\feat{Dodge} & Dex 13 & +1 dodge bonus to AC against one opponent\\
	~ \feat{Mobility} & Dex 13, \feat{Dodge} & +4 dodge bonus against some attacks of opportunity\\
	~ ~ \feat{Shot On The Run} & Dex 13, \feat{Dodge}, \feat{Mobility}, \feat{Point Blank Shot}, base attack bonus +4 & Move before and after a ranged attack\\
	~ ~ \feat{Spring Attack} & Dex 13, \feat{Dodge}, \feat{Mobility}, base attack bonus +4 & Move before and after a melee attack\\
	~ ~ ~ \feat{Whirlwind Attack} & Dex 13, Int 13, \feat{Combat Expertise}, \feat{Dodge}, \feat{Mobility}, \feat{Spring Attack}, base attack bonus +4 & Attack all opponents within reach with one full-round attack\\
	\feat{Exotic Weapon Proficiency} & Base attack bonus +1 & No penalty when attacking with chosen weapon\\
	\feat{Implacable Defender} & Str 13, Base attack bonus +3 & +2 bonus to resist special attacks and improved defense\\
	\feat{Improved Critical} & Proficient with weapon, base attack bonus +8 & Double threat range of chosen weapon\\
	~ \feat{Greater Critical} & Proficient with weapon, base attack bonus +12, \feat{Improved Critical} with weapon & +1 to critical multiplier of chosen weapon and may score critical on immune foes\\
	\feat{Improved Initiative} && +4 bonus on initiative\\
	\feat{Improved Shield Bash} & \feat{Shield Proficiency} & Retain shield AC when attacking with it\\
	\feat{Improved Unarmed Strike} && Considered armed when unarmed\\
	~ \feat{Deflect Arrows} & Dex 13, \feat{Improved Unarmed Strike} & Deflect one ranged attack per round\\
	~ ~ \feat{Snatch Arrows} & Dex 15, \feat{Deflect Arrows}, \feat{Improved Unarmed Strike} & Catch ranged weapon instead of deflecting\\
	~ \feat{Improved Grapple} & Dex 13, \feat{Improved Unarmed Strike} & +4 bonus on grapple checks, no attack of opportunity\\
	~ \feat{Stunning Fist} & Dex 13, Wis 13, \feat{Improved Unarmed Strike}, base attack bonus +8 & Stuns foe with an unarmed attack\\
	\feat{Intimidating Presence} & Cha 13, \skill{Intimidate} 7 ranks & Demoralize more opponents per round\\
	\feat{Inspiring Presence} & Cha 13 & Allies within 3 meters gain bonus on Will saves\\
	\feat{Mounted Combat} & Ride 1 rank & Prevent one attack on your mount per round\\
	~ \feat{Mounted Archery} & Ride 1 rank, \feat{Mounted Combat} & Half penalties using ranged weapons while mounted\\
	~ \feat{Ride-By Attack} & Ride 1 rank, \feat{Mounted Combat} & Move before and after a mounted charge attack\\
	~ ~ \feat{Spirited Charge} & Ride 1 rank, \feat{Mounted Combat}, \feat{Ride-By Attack} & Double damage in a mounted charge\\
	~ \feat{Trample} & Ride 1 rank, \feat{Mounted Combat} & Mounted overrun cannot be avoided\\
}

\FeatTable[>{\raggedright}p{3.5cm}]{Fighter}{
	\feat{Point Blank Shot} && +1 bonus on attack and damage within 9 m\\
	~ \feat{Far Shot} & \feat{Point Blank Shot} & Increase range increment by 50\% or 100\%\\
	~ \feat{Precise Shot} & \feat{Point Blank Shot} & No penalty to shoot at engaged foe\\
	~ ~ \feat{Improved Precise Shot} & Dex 19, \feat{Point Blank Shot}, \feat{Precise Shot}, base attack bonus +11 & Ignore anything less than total cover or total concealment\\
	~ \feat{Rapid Shot} & Dex 13, \feat{Point Blank Shot} & Extra attack in full attack with $-2$ penalty\\
	~ ~ \feat{Manyshot} & Dex 17, \feat{Point Blank Shot}, \feat{Rapid Shot}, base attack bonus +6 & Shoot many arrows as standard action\\
	~ ~ ~ \feat{Greater Manyshot} & Dex 17, \feat{Manyshot}, \feat{Point Blank Shot}, \feat{Rapid Shot}, base attack bonus +6 & Use \feat{Manyshot} at different targets\\
	\feat{Power Attack} & Str 13 & Trade attack bonus for damage\\
	~ \feat{Cleave} & Str 13, \feat{Power Attack} & Extra attack after defeating foe, once per round\\
	~ ~ \feat{Great Cleave} & Str 13, \feat{Cleave}, \feat{Power Attack}, base attack bonus +4 & No limits for \feat{Cleave} per round\\
	~ \feat{Improved Bull Rush} & Str 13, \feat{Power Attack} & +4 bonus on bull rush checks, no attacks of opportunity\\
	~ ~ \feat{Awesome Blow} & Str 25, \feat{Power Attack}, \feat{Improved Bull Rush}, size Large or larger & Send foe flying 3 m as standard action\\
	~ \feat{Improved Overrun} & Str 13, \feat{Power Attack} & +4 bonus on overrun checks, it cannot be avoided\\
	~ \feat{Improved Sunder} & Str 13, \feat{Power Attack} & +4 bonus on sunder rolls, no attacks of opportunity\\
	\feat{Quick Draw} & Base attack bonus +1 & Draw weapon as free action\\
	\feat{Rapid Reload} & \feat{Weapon Proficiency} (crossbow) & Reduce reload time to free action or move action\\
	\feat{Rotate Lines} & Base attack bonus +3 & Swap places with an ally 1.5 m away\\
	\feat{Shield Evasion} & Dex 13, \feat{Shield Proficiency}, base attack bonus +3 & Add shield bonus to Reflex saves and gain evasion when fighting defensively\\
	\feat{Shield Wall} & \feat{Shield Proficiency}, base attack bonus +2 & +1 AC for each adjacent ally with a large shield\\
	\feat{Spear Wall} & Base attack bonus +1 & Double damage in ready attack against charging foe\\
	\feat{Tactical Expertise} & \skill{Knowledge} (warcraft) 7 ranks & Coordinate allies as standard action\\
	\feat{Teamwork} & Base attack bonus +1 & Aid another as move action\\
	\feat{Two-Weapon Fighting} & Dex 15 & Reduce penalties for fighting with two weapons\\
	~ \feat{Improved Two-Weapon Fighting} & Dex 17, \feat{Two-Weapon Fighting}, base attack bonus +6 & Second off-hand attack with $-5$ penalty\\
	~ ~ \feat{Greater Two-Weapon Fighting} & Dex 19, \feat{Improved Two-Weapon Fighting}, \feat{Two-Weapon Fighting}, base attack bonus +11 & Third off-hand attack with $-10$ penalty\\
	~ \feat{Two-Weapon Defense} & Dex 15, \feat{Two-Weapon Fighting} & +1 shield bonus to AC when wielding two weapons\\
	\feat{Weapon Finesse} & Base attack bonus +1 & Use Dex modifier instead of Str for some attack rolls\\
	\feat{Weapon Focus} & Proficiency with weapon, base attack bonus +1 & +1 bonus on attack rolls with chosen weapon\\
	~ \feat{Greater Weapon Focus} & Proficiency with weapon, \feat{Weapon Focus} with weapon, fighter level 8th & +1 bonus on attack rolls with chosen weapon\\
	~ ~ \feat{Greater Weapon Specialization} & Proficiency with weapon, \feat{Greater Weapon Focus} with weapon, \feat{Weapon Focus} with weapon, \feat{Weapon Specialization} with weapon, fighter level 12th& +2 bonus damage with chosen weapon\\
	~ \feat{Weapon Specialization} & Proficiency with weapon, \feat{Weapon Focus} with weapon, fighter level 4th & +2 bonus damage with chosen weapon\\
}
\subsectionA{Fighter Feats}

\GFeat[Fighter]{Awesome Blow}
{Str 25, \feat{Power Attack}, \feat{Improved Bull Rush}, size Large or larger.}
{As a standard action, the creature may choose to subtract 4 from its melee attack roll and deliver an awesome blow. If the creature hits a corporeal opponent smaller than itself with an awesome blow, its opponent must succeed on a Reflex save (DC = damage dealt) or be knocked flying 3 meters in a direction of the attacking creature's choice and fall prone. The attacking creature can only push the opponent in a straight line, and the opponent can't move closer to the attacking creature than the square it started in. If an obstacle prevents the completion of the opponent's move, the opponent and the obstacle each take 1d6 points of damage, and the opponent stops in the space adjacent to the obstacle.}

\Feat[Fighter]{Blind-Fight}
{}
{In melee, every time you miss because of concealment, you can reroll your miss chance percentile roll one time to see if you actually hit.

An invisible attacker gets no advantages related to hitting you in melee. That is, you don't lose your Dexterity bonus to Armor Class, and the attacker doesn't get the usual +2 bonus for being invisible. The invisible attacker's bonuses do still apply for ranged attacks, however.

You take only half the usual penalty to speed for being unable to see. Darkness and poor visibility in general reduces your speed to three-quarters normal, instead of one-half.}
{Regular attack roll modifiers for invisible attackers trying to hit you apply, and you lose your Dexterity bonus to AC. The speed reduction for darkness and poor visibility also applies.}
{The Blind-Fight feat is of no use against a character who is the subject of a \spell{blink} spell.}{}

\GFeat[Fighter]{Cleave}
{Str 13, \feat{Power Attack}.}
{If you deal a creature enough damage to make it drop (typically by dropping it to below 0 hit points or killing it), you get an immediate, extra melee attack against another creature within reach. You cannot take a 1.5-meter step before making this extra attack. The extra attack is with the same weapon and at the same bonus as the attack that dropped the previous creature. You can use this ability once per round.}

\Feat[Fighter]{Combat Expertise}
{}
{Int 13.}
{When you use the attack action or the full attack action in melee, you can take a penalty of as much as $-5$ on your attack roll and add the same number (+5 or less) as a dodge bonus to your Armor Class. This number may not exceed your base attack bonus. The changes to attack rolls and Armor Class last until your next action.}
{A character without the Combat Expertise feat can fight defensively while using the attack or full attack action to take a $-4$ penalty on attack rolls and gain a +2 dodge bonus to Armor Class.}{}

\Feat[Fighter]{Combat Reflexes}
{}
{You may make a number of additional attacks of opportunity equal to your Dexterity bonus.

With this feat, you may also make attacks of opportunity while flat-footed.}
{A character without this feat can make only one attack of opportunity per round and can't make attacks of opportunity while flat-footed.}
{The Combat Reflexes feat does not allow a rogue to use her opportunist ability more than once per round.}{}

\Feat[Fighter]{Commanding Presence}
{Your mere presence can enable your allies.}
{\skill{Diplomacy} 7 ranks, \skill{Knowledge} (warcraft) 5 ranks.}
{This feat grants a new use for the \skill{Diplomacy} skill.

\textit{Enabling an Ally:} You can remove harmful conditions from an ally as a move action by making a DC 20 \skill{Diplomacy} check. If the check succeeds, you can negate any one of the following conditions: cowering, dazed, fatigued, nauseated, panicked, shaken, or stunned.

You cannot use this ability on yourself.}{}{}

\Feat[Fighter]{Concentrated Fire}
{You are trained in formation archery and taking out specific targets through joint efforts.}
{Base attack bonus +1.}
{When readying and firing projectile weapons at a single target, you add a +1 bonus to your attack roll for every other participant with this feat who readies and fires at the same target on your initiative count. The total bonus cannot exceed +4.}{}{}

\GFeat[Fighter]{Deflect Arrows}
{Dex 13, \feat{Improved Unarmed Strike}.}
{You must have at least one hand free (holding nothing) to use this feat. Once per round when you would normally be hit with a ranged weapon, you may deflect it so that you take no damage from it. You must be aware of the attack and not flat-footed.

Attempting to deflect a ranged weapon doesn't count as an action. Unusually massive ranged weapons and ranged attacks generated by spell effects can't be deflected.}

\GFeat[Fighter]{Dodge}
{Dex 13.}
{During your action, you designate an opponent and receive a +1 dodge bonus to Armor Class against attacks from that opponent. You can select a new opponent on any action.

A condition that makes you lose your Dexterity bonus to Armor Class (if any) also makes you lose dodge bonuses. Also, dodge bonuses stack with each other, unlike most other types of bonuses.}

\Feat[Fighter]{Exotic Weapon Proficiency}
{Choose a type of exotic weapon. You understand how to use that type of exotic weapon in combat.}
{Base attack bonus +1 (plus Str 13 for bastard sword or dwarven waraxe).}
{You make attack rolls with the weapon normally.}
{A character who uses a weapon with which he or she is not proficient takes a $-4$ penalty on attack rolls.}
{You can gain Exotic Weapon Proficiency multiple times. Each time you take the feat, it applies to a new type of exotic weapon. Proficiency with the bastard sword or the dwarven waraxe has an additional prerequisite of Str 13.}

\GFeat[Fighter]{Far Shot}
{\feat{Point Blank Shot}.}
{When you use a projectile weapon, such as a bow, its range increment increases by one-half (multiply by 1\onehalf). When you use a thrown weapon, its range increment is doubled.}

\GFeat[Fighter]{Great Cleave}
{Str 13, \feat{Cleave}, \feat{Power Attack}, base attack bonus +4.}
{This feat works like \feat{Cleave}, except that there is no limit to the number of times you can use it per round.}

\Feat[Fighter]{Greater Manyshot}
{You are skilled at firing many arrows at once, even at different opponents.}
{Dex 17, \feat{Manyshot}, \feat{Point Blank Shot}, \feat{Rapid Shot}, base attack bonus +6.}
{When you use the \feat{Manyshot} feat, you can fire each arrow at a different target instead of firing all of them at the same target. You make a separate attack roll for each arrow, regardless of whether you fire them at separate targets or the same target. Your precision-based damage applies to each arrow fired, and, if you score a critical hit with more than one of the arrows, each critical hit deals critical damage.}{}{}

\Feat[Fighter]{Greater Two-Weapon Fighting}
{}
{Dex 19, \feat{Improved Two-Weapon Fighting}, \feat{Two-Weapon Fighting}, base attack bonus +11.}
{You get a third attack with your off-hand weapon, albeit at a $-10$ penalty. See the Two-Weapon Fighting special attack.}{}
{An 11th-level ranger who has chosen the two-weapon combat style is treated as having Greater Two-Weapon Fighting, even if he does not have the prerequisites for it, but only when he is wearing light or no armor.}

\Feat[Fighter]{Greater Weapon Focus}
{Choose one type of weapon for which you have already selected Weapon Focus. You can also choose unarmed strike or grapple as your weapon for purposes of this feat.}
{Proficiency with selected weapon, \feat{Weapon Focus} with selected weapon, fighter level 8th.}
{You gain a +1 bonus on all attack rolls you make using the selected weapon. This bonus stacks with other bonuses on attack rolls, including the one from Weapon Focus (see below).}{}
{You can gain Greater Weapon Focus multiple times. Its effects do not stack. Each time you take the feat, it applies to a new type of weapon.

A fighter must have Greater Weapon Focus with a given weapon to gain the \feat{Greater Weapon Specialization} feat for that weapon.}

\Feat[Fighter]{Greater Weapon Specialization}
{Choose one type of weapon for which you have already selected Weapon Specialization. You can also choose unarmed strike or grapple as your weapon for purposes of this feat.}
{Proficiency with selected weapon, \feat{Greater Weapon Focus} with selected weapon, \feat{Weapon Focus} with selected weapon, \feat{Weapon Specialization} with selected weapon, fighter level 12th.}
{You gain a +2 bonus on all damage rolls you make using the selected weapon. This bonus stacks with other bonuses on damage rolls, including the one from Weapon Specialization (see below).}{}
{You can gain Greater Weapon Specialization multiple times. Its effects do not stack. Each time you take the feat, it applies to a new type of weapon.}

\Feat[Fighter]{Implacable Defender}
{You have learned not to fall victim to certain types of attack.}
{Str 13, Base attack bonus +3.}
{You receive a +2 bonus on opposed Strength checks to resist bull rush, overrun, or trip attempts.}{}{}

\Feat[Fighter]{Improved Bull Rush}
{}
{Str 13, \feat{Power Attack}.}
{When you perform a bull rush you do not provoke an attack of opportunity from the defender. You also gain a +4 bonus on the opposed Strength check you make to push back the defender.}{}{}

\Feat[Fighter]{Improved Critical}
{Choose one type of weapon.}
{Proficient with weapon, base attack bonus +8.}
{When using the weapon you selected, your threat range is doubled.}{}
{You can gain Improved Critical multiple times. The effects do not stack. Each time you take the feat, it applies to a new type of weapon.

This effect doesn't stack with any other effect that expands the threat range of a weapon.}{}

\Feat[Fighter]{Improved Disarm}
{}
{Int 13, \feat{Combat Expertise}.}
{You do not provoke an attack of opportunity when you attempt to disarm an opponent, nor does the opponent have a chance to disarm you. You also gain a +4 bonus on the opposed attack roll you make to disarm your opponent.}
{See the normal disarm rules.}{}

\Feat[Fighter]{Improved Feint}
{}
{Int 13, \feat{Combat Expertise}.}
{You can make a \skill{Bluff} check to feint in combat as a move action.}
{Feinting in combat is a standard action.}{}

\Feat[Fighter]{Improved Grapple}
{}
{Dex 13, \feat{Improved Unarmed Strike}.}
{You do not provoke an attack of opportunity when you make a touch attack to start a grapple. You also gain a +4 bonus on all grapple checks, regardless of whether you started the grapple.}
{Without this feat, you provoke an attack of opportunity when you make a touch attack to start a grapple.}{}

\GFeat[Fighter]{Improved Initiative}
{}
{You get a +4 bonus on initiative checks.}

\Feat[Fighter]{Improved Overrun}
{}
{Str 13, \feat{Power Attack}.}
{When you attempt to overrun an opponent, the target may not choose to avoid you. You also gain a +4 bonus on your Strength check to knock down your opponent.}
{Without this feat, the target of an overrun can choose to avoid you or to block you.}{}

\Feat[Fighter]{Improved Precise Shot}
{}
{Dex 19, \feat{Point Blank Shot}, \feat{Precise Shot}, base attack bonus +11.}
{Your ranged attacks ignore the AC bonus granted to targets by anything less than total cover, and the miss chance granted to targets by anything less than total concealment. Total cover and total concealment provide their normal benefits against your ranged attacks.

In addition, when you shoot or throw ranged weapons at a grappling opponent, you automatically strike at the opponent you have chosen.}
{See the normal rules on the effects of cover and concealment. Without this feat, a character who shoots or throws a ranged weapon at a target involved in a grapple must roll randomly to see which grappling combatant the attack strikes.}
{An 11th-level ranger who has chosen the archery combat style is treated as having Improved Precise Shot, even if he does not have the prerequisites for it, but only when he is wearing light or no armor.}

\Feat[Fighter]{Improved Shield Bash}
{}
{\feat{Shield Proficiency}.}
{When you perform a shield bash, you may still apply the shield's shield bonus to your AC.}
{Without this feat, a character who performs a shield bash loses the shield's shield bonus to AC until his or her next turn.}{}

\Feat[Fighter]{Improved Sunder}
{}
{Str 13, \feat{Power Attack}.}
{When you strike at an object held or carried by an opponent (such as a weapon or shield), you do not provoke an attack of opportunity.

You also gain a +4 bonus on any attack roll made to attack an object held or carried by another character.}
{Without this feat, you provoke an attack of opportunity when you strike at an object held or carried by another character.}{}

\Feat[Fighter]{Improved Trip}
{}
{Int 13, \feat{Combat Expertise}.}
{You do not provoke an attack of opportunity when you attempt to trip an opponent while you are unarmed. You also gain a +4 bonus on your Strength check to trip your opponent.

If you trip an opponent in melee combat, you immediately get a melee attack against that opponent as if you hadn't used your attack for the trip attempt.}
{Without this feat, you provoke an attack of opportunity when you attempt to trip an opponent while you are unarmed.}{}

\Feat[Fighter]{Improved Two-Weapon Fighting}
{}
{Dex 17, \feat{Two-Weapon Fighting}, base attack bonus +6.}
{In addition to the standard single extra attack you get with an off-hand weapon, you get a second attack with it, albeit at a $-5$ penalty. See the Two-Weapon Fighting special attack.}
{Without this feat, you can only get a single extra attack with an off-hand weapon.}
{A 6th-level ranger who has chosen the two-weapon combat style is treated as having Improved Two-Weapon Fighting, even if he does not have the prerequisites for it, but only when he is wearing light or no armor.}

\Feat[Fighter]{Improved Unarmed Strike}{}
{}
{You are considered to be armed even when unarmed---that is, you do not provoke attacks or opportunity from armed opponents when you attack them while unarmed. However, you still get an attack of opportunity against any opponent who makes an unarmed attack on you.

In addition, your unarmed strikes can deal lethal or nonlethal damage, at your option.}
{Without this feat, you are considered unarmed when attacking with an unarmed strike, and you can deal only nonlethal damage with such an attack.}{}

\Feat[Fighter]{Intimidating Presence}
{Your mere presence can weaken a foe's resolve.}
{Cha 13, \skill{Intimidate} 7 ranks.}
{You can demoralize a number of opponents per round equal to your Charisma modifier.}
{You can demoralize only a single opponent per round.}{}

\Feat[Fighter]{Inspiring Presence}
{Your mere presence can strengthen your allies' resolve.}
{Cha 13.}
{Each ally within 3 meters of you gains a morale bonus on Will saves equal to your Charisma modifier.}{}{}

\Feat[Fighter]{Manyshot}
{}
{Dex 17, \feat{Point Blank Shot}, \feat{Rapid Shot}, base attack bonus +6}
{As a standard action, you may fire two arrows at a single opponent within 9 meters. Both arrows use the same attack roll (with a $-4$ penalty) to determine success and deal damage normally (but see Special).

For every five points of base attack bonus you have above +6, you may add one additional arrow to this attack, to a maximum of four arrows at a base attack bonus of +16. However, each arrow after the second adds a cumulative $-2$ penalty on the attack roll (for a total penalty of $-6$ for three arrows and $-8$ for four).

Damage reduction and other resistances apply separately against each arrow fired.}{}
{Regardless of the number of arrows you fire, you apply precision-based damage only once. If you score a critical hit, only the first arrow fired deals critical damage; all others deal regular damage.

A 6th-level ranger who has chosen the archery combat style is treated as having Manyshot even if he does not have the prerequisites for it, but only when he is wearing light or no armor.}

\GFeat[Fighter]{Mobility}
{Dex 13, \feat{Dodge}.}
{You get a +4 dodge bonus to Armor Class against attacks of opportunity caused when you move out of or within a threatened area. A condition that makes you lose your Dexterity bonus to Armor Class (if any) also makes you lose dodge bonuses.

Dodge bonuses stack with each other, unlike most types of bonuses.}

\GFeat[Fighter]{Mounted Archery}
{\skill{Ride} 1 rank, \feat{Mounted Combat}.}
{The penalty you take when using a ranged weapon while mounted is halved: $-2$ instead of $-4$ if your mount is taking a double move, and $-4$ instead of $-8$ if your mount is running.}

\GFeat[Fighter]{Mounted Combat}
{\skill{Ride} 1 rank.}
{Once per round when your mount is hit in combat, you may attempt a \skill{Ride} check (as a reaction) to negate the hit. The hit is negated if your \skill{Ride} check result is greater than the opponent's attack roll. (Essentially, the \skill{Ride} check result becomes the mount's Armor Class if it's higher than the mount's regular AC.)}

\GFeat[Fighter]{Point Blank Shot}
{}
{You get a +1 bonus on attack and damage rolls with ranged weapons at ranges of up to 9 meters.}

\Feat[Fighter]{Power Attack}
{}
{Str 13.}
{On your action, before making attack rolls for a round, you may choose to subtract a number from all melee attack rolls and add the same number to all melee damage rolls. This number may not exceed your base attack bonus. The penalty on attacks and bonus on damage apply until your next turn.}
{If you attack with a two-handed weapon, or with a one-handed weapon wielded in two hands, instead add twice the number subtracted from your attack rolls. You can't add the bonus from Power Attack to the damage dealt with a light weapon (except with unarmed strikes or natural weapon attacks), even though the penalty on attack rolls still applies. (Normally, you treat a double weapon as a one-handed weapon and a light weapon. If you choose to use a double weapon like a two-handed weapon, attacking with only one end of it in a round, you treat it as a two-handed weapon.)}{}

\GFeat[Fighter]{Precise Shot}
{\feat{Point Blank Shot}.}
{You can shoot or throw ranged weapons at an opponent engaged in melee without taking the standard $-4$ penalty on your attack roll.}

\Feat[Fighter]{Quick Draw}
{}
{Base attack bonus +1.}
{You can draw a weapon as a free action instead of as a move action. You can draw a hidden weapon (see the Sleight of Hand skill) as a move action.

A character who has selected this feat may throw weapons at his full normal rate of attacks (much like a character with a bow).}
{Without this feat, you may draw a weapon as a move action, or (if your base attack bonus is +1 or higher) as a free action as part of movement. Without this feat, you can draw a hidden weapon as a standard action.}{}

\Feat[Fighter]{Rapid Reload}
{Choose a type of crossbow (hand, light, or heavy).}
{\feat{Weapon Proficiency} (crossbow type chosen).}
{The time required for you to reload your chosen type of crossbow is reduced to a free action (for a hand or light crossbow) or a move action (for a heavy crossbow). Reloading a crossbow still provokes an attack of opportunity.

If you have selected this feat for hand crossbow or light crossbow, you may fire that weapon as many times in a full attack action as you could attack if you were using a bow.}
{A character without this feat needs a move action to reload a hand or light crossbow, or a full-round action to reload a heavy crossbow.}
{You can gain Rapid Reload multiple times. Each time you take the feat, it applies to a new type of crossbow.}

\Feat[Fighter]{Rapid Shot}
{}
{Dex 13, \feat{Point Blank Shot}.}
{You can get one extra attack per round with a ranged weapon. The attack is at your highest base attack bonus, but each attack you make in that round (the extra one and the normal ones) takes a $-2$ penalty. You must use the full attack action to use this feat.}{}
{A 2nd-level ranger who has chosen the archery combat style is treated as having Rapid Shot, even if he does not have the prerequisites for it, but only when he is wearing light or no armor.}

\GFeat[Fighter]{Ride-By Attack}
{\skill{Ride} 1 rank, \feat{Mounted Combat}.}
{When you are mounted and use the charge action, you may move and attack as if with a standard charge and then move again (continuing the straight line of the charge). Your total movement for the round can't exceed double your mounted speed. You and your mount do not provoke an attack of opportunity from the opponent that you attack.}

\Feat[Fighter]{Rotate Lines}
{In the heat of battle, weary and wounded soldiers retreat to be replaced by fresh, unwounded ones.}
{Base attack bonus +3.}
{You can swap positions with an ally within 1.5 m This is a move action that does not generate an attack of opportunity for you or your ally. You may not take a 1.5 m step in addition when rotating lines.}{}{}

\Feat[Fighter]{Shield Wall}
{You are trained in defensive infantry formation.}
{\feat{Shield Proficiency}, base attack bonus +2.}
{If using a large shield and forming a row with allies facing the same direction, you get a +1 circumstance bonus from each adjacent ally in the row also possessing a large shield and this feat, up to two (+2 AC bonus).}{}{}

\GFeat[Fighter]{Shot On The Run}
{Dex 13, \feat{Dodge}, \feat{Mobility}, \feat{Point Blank Shot}, base attack bonus +4.}
{When using the attack action with a ranged weapon, you can move both before and after the attack, provided that your total distance moved is not greater than your speed.}

\GFeat[Fighter]{Snatch Arrows}
{Dex 15, \feat{Deflect Arrows}, \feat{Improved Unarmed Strike}.}
{When using the \feat{Deflect Arrows} feat you may catch the weapon instead of just deflecting it. Thrown weapons can immediately be thrown back at the original attacker (even though it isn't your turn) or kept for later use.

You must have at least one hand free (holding nothing) to use this feat.}

\Feat[Fighter]{Spear Wall}
{You are trained in inflicting as much damage as possible on a charging opponent.}
{Base attack bonus +1.}
{When readying a spear or other weapon that would inflict double damage against a charging opponent, you instead inflict triple damage on a hit.}{}{}

\GFeat[Fighter]{Spirited Charge}
{\skill{Ride} 1 rank, \feat{Mounted Combat}, \feat{Ride-By Attack}.}
{When mounted and using the charge action, you deal double damage with a melee weapon (or triple damage with a lance).}

\GFeat[Fighter]{Spring Attack}
{Dex 13, \feat{Dodge}, \feat{Mobility}, base attack bonus +4.}
{When using the attack action with a melee weapon, you can move both before and after the attack, provided that your total distance moved is not greater than your speed. Moving in this way does not provoke an attack of opportunity from the defender you attack, though it might provoke attacks of opportunity from other creatures, if appropriate. You can't use this feat if you are wearing heavy armor.

You must move at least 1.5 meters both before and after you make your attack in order to utilize the benefits of Spring Attack.}

\GFeat[Fighter]{Stunning Fist}
{Dex 13, Wis 13, \feat{Improved Unarmed Strike}, base attack bonus +8.}
{You must declare that you are using this feat before you make your attack roll (thus, a failed attack roll ruins the attempt). Stunning Fist forces a foe damaged by your unarmed attack to make a Fortitude saving throw (DC 10 + \onehalf your character level + your Wis modifier), in addition to dealing damage normally. A defender who fails this saving throw is stunned for 1 round (until just before your next action). A stunned creature drops everything held, can't take actions, takes a $-2$ penalty to AC, and loses his Dexterity bonus to AC. You may attempt a stunning attack once per day for every four levels you have attained (but see Special), and no more than once per round. Constructs, oozes, plants, undead, incorporeal creatures, and creatures immune to critical hits cannot be stunned.}

\Feat[Fighter]{Tactical Expertise}
{You are expert in war tactics.}
{\skill{Knowledge} (warcraft) 7 ranks.}
{You can use coordinate allies as a standard action.}
{Using coordinate allies is a full-round action.}{}

\Feat[Fighter]{Teamwork}
{You are trained in group combat. You have an easier time protecting your allies, and creating openings in an enemy's defense for others to exploit.}
{Base attack bonus +1.}
{You may aid another as a move action.}
{Aid another is a standard action.}{}

\GFeat[Fighter]{Trample}
{\skill{Ride} 1 rank, \feat{Mounted Combat}.}
{When you attempt to overrun an opponent while mounted, your target may not choose to avoid you. Your mount may make one hoof attack against any target you knock down, gaining the standard +4 bonus on attack rolls against prone targets.}

\GFeat[Fighter]{Two-Weapon Defense}
{Dex 15, \feat{Two-Weapon Fighting}.}
{When wielding a double weapon or two weapons (not including natural weapons or unarmed strikes), you gain a +1 shield bonus to your AC. See the Two-Weapon Fighting special attack.

When you are fighting defensively or using the total defense action, this shield bonus increases to +2.}

\Feat[Fighter]{Two-Weapon Fighting}
{You can fight with a weapon in each hand. You can make one extra attack each round with the second weapon.}
{Dex 15.}
{Your penalties on attack rolls for fighting with two weapons are reduced. The penalty for your primary hand lessens by 2 and the one for your off hand lessens by 6. See the Two-Weapon Fighting special attack.}
{If you wield a second weapon in your off hand, you can get one extra attack per round with that weapon. When fighting in this way you suffer a $-6$ penalty with your regular attack or attacks with your primary hand and a $-10$ penalty to the attack with your off hand. If your off-hand weapon is light the penalties are reduced by 2 each. (An unarmed strike is always considered light.)}
{A 2nd-level ranger who has chosen the two-weapon combat style is treated as having Two-Weapon Fighting, even if he does not have the prerequisite for it, but only when he is wearing light or no armor.}

\GFeat[Fighter]{Weapon Finesse}
{Base attack bonus +1.}
{With a light weapon, rapier, whip, or spiked chain made for a creature of your size category, you may use your Dexterity modifier instead of your Strength modifier on attack rolls. If you carry a shield, its armor check penalty applies to your attack rolls.

Natural weapons are always considered light weapons.}

\Feat[Fighter]{Weapon Focus}
{Choose one type of weapon. You can also choose unarmed strike or grapple (or ray, if you are a spellcaster) as your weapon for purposes of this feat.}
{Proficiency with selected weapon, base attack bonus +1.}
{You gain a +1 bonus on all attack rolls you make using the selected weapon.}{}
{You can gain this feat multiple times. Its effects do not stack. Each time you take the feat, it applies to a new type of weapon.}

\Feat[Fighter]{Weapon Specialization}
{Choose one type of weapon for which you have already selected the \feat{Weapon Focus} feat. You can also choose unarmed strike or grapple as your weapon for purposes of this feat. You deal extra damage when using this weapon.}
{Proficiency with selected weapon, \feat{Weapon Focus} with selected weapon, fighter level 4th.}
{You gain a +2 bonus on all damage rolls you make using the selected weapon.}{}
{You can gain this feat multiple times. Its effects do not stack. Each time you take the feat, it applies to a new type of weapon.}

\GFeat[Fighter]{Whirlwind Attack}
{Dex 13, Int 13, \feat{Combat Expertise}, \feat{Dodge}, \feat{Mobility}, \feat{Spring Attack}, base attack bonus +4.}
{When you use the full attack action, you can give up your regular attacks and instead make one melee attack at your full base attack bonus against each opponent within reach.

When you use the Whirlwind Attack feat, you also forfeit any bonus or extra attacks granted by other feats, spells, or abilities.}


\FeatTable{Psionic}{
	\feat{Aligned Attack}\footnotemark[1] & Base attack bonus +6 & +1d6 damage and attack becomes aligned\\
	\feat{Body Fuel} && Spend power points to recover ability burn damage\\
	\feat{Boost Construct} && Give one additional special ability to astral construct\\
	\feat{Combat Manifestation} && +4 bonus on \skill{Concentration} to manifest on the defensive\\
	\feat{Elemental Manifestation}\footnotemark[1] & Access to domain spells, manifester level 3rd & +2 DC to power with same descriptor as patron element\\
	\feat{Expanded Knowledge} & Manifester level 3rd & Gain one power known\\
	\feat{Focused Mind}\footnotemark[2] & Int 13 & +2 on \skill{Appraise}, \skill{Decipher Script} and \skill{Search} checks\\
	\feat{Focused Sunder}\footnotemark[1] & Str 13, \feat{Power Attack}, \feat{Improved Sunder} & Ignore half of the weapon's hardness\\
	\feat{Ghost Attack}\footnotemark[2] & Base attack bonus +3 & Make two rolls for miss chance against incorporeal foes\\
	\feat{Improved Dwarven Focus}\footnotemark[2] & Dwarf & +2 bonus on all checks related to your dwaven focus\\
	\feat{Improved Elf Run}\footnotemark[2] & Elf & +4.5 m to speed while in elf run state\\
	\feat{Inquisitor}\footnotemark[1] & Wis 13 & +10 bonus on a \skill{Sense Motive} check to oppose a \skill{Bluff}\\
	\feat{Jump Charge}\footnotemark[1] & \feat{Psionic Fist} or \feat{Psionic Weapon}, \skill{Jump} 8 ranks &  Increase damage of \feat{Psionic Fist} or \feat{Psionic Weapon} by 50\% or 100\% after a charge\\
	\feat{Mental Leap}\footnotemark[1] & Str 13, \skill{Jump} 5 ranks &  +10 bonus on a \skill{Jump} check\\
	\feat{Metamorphic Transfer} & Wis 13, manifester level 5th & Gain one supernatural ability of \psionic{metamorphosis} form\\
	\feat{Narrow Mind} & Wis 13 & +4 bonus to checks to become psionically focused\\
	\feat{Overchannel} && Take damage to increase your effective manifester level\\
	~ \feat{Talented} & \feat{Overchannel} & Take no damage to increase a power of 3rd level or lower\\
	\feat{Power Penetration}\footnotemark[1] && +4 bonus to level checks to overcome power resistance\\
	~ \feat{Greater Power Penetration}\footnotemark[1] & \feat{Power Penetration} & +8 bonus to level checks to overcome power resistance\\
	\feat{Power Specialization} & \feat{Weapon Focus} (ray), manifester level 4th & +2 damage to rays and ranged touch attacks\\
	~ \feat{Greater Power Specialization} & \feat{Power Specialization}, \feat{Weapon Focus} (ray), manifester level 12th & +2 damage to any power\\
	\feat{Psicrystal Affinity} & Manifester level 1st & Gain a psicrystal\\
	~ \feat{Improved Psicrystal} & \feat{Psicrystal Affinity} & Implant another personality in psicrystal\\
	~ \feat{Psicrystal Containment} & \feat{Psicrystal Affinity}, manifester level 3rd & Store psionic focus on psicrystal\\
	\feat{Psionic Body} && +2 hit points per psionic feat\\
	\feat{Psionic Dodge}\footnotemark[2] & Dex 13, \feat{Dodge} & +1 dodge bonus to AC\\
	\feat{Psionic Endowment}\footnotemark[1] && +1 DC to a power\\
	~ \feat{Greater Psionic Endowment}\footnotemark[1] & \feat{Psionic Endowment} & +2 DC to a power\\
	\feat{Psionic Fist}\footnotemark[1] & Str 13 & +2d6 damage to a unarmed strike or natural attack\\
	~ \feat{Greater Psionic Fist}\footnotemark[1] & Str 13, \feat{Psionic Fist}, base attack bonus +5 & +4d6 damage to a unarmed strike or natural attack\\
	~ \feat{Unavoidable Strike}\footnotemark[1] & Str 13, \feat{Psionic Fist}, base attack bonus +5 & Resolve a unarmed or natural attack as touch attack\\
	\feat{Psionic Meditation} & Wis 13, \skill{Concentration} 7 ranks & Move action to become psionically focused\\
	\feat{Psionic Shot}\footnotemark[1] & \feat{Point Blank Shot} & +2d6 to ranged attack damage\\
	~ \feat{Fell Shot}\footnotemark[1] & Dex 13, \feat{Point Blank Shot}, \feat{Psionic Shot}, base attack bonus +5 & Resolve a ranged attack as ranged touch attack\\
	~ ~ \feat{Return Shot}\footnotemark[1] & \feat{Point Blank Shot}, \feat{Psionic Shot}, \feat{Fell Shot}, base attack bonus +3 & Return ranged attack to foe using original attack bonus\\
	~ \feat{Greater Psionic Shot}\footnotemark[1] & \feat{Point Blank Shot}, \feat{Psionic Shot}, base attack bonus +5 & +4d6 to ranged attack damage\\
	\feat{Psionic Talent} & Having a power point reserve & Gain 2 additional power points\\
	\feat{Psionic Weapon}\footnotemark[1] & Str 13 & +2d6 to melee damage\\
	~ \feat{Deep Impact}\footnotemark[1] & Str 13, \feat{Psionic Weapon}, base attack bonus +5 & Resolve a melee attack as touch attack\\
	~ \feat{Greater Psionic Weapon}\footnotemark[1] & Str 13, \feat{Psionic Weapon}, base attack bonus +5 & +4d6 damage to melee attack damage\\
	\feat{Pterran Telepathy} & Pterran, \psionic{missive} psi-like ability & Use \psionic{missive} with humanoid in addition to repitiles\\
	\feat{Shipfloater} && +3 on \skill{Profession} (sailor), control \emph{obsidian engines} as standard action\\
	\feat{Speed of Thought}\footnotemark[2] & Wis 13 & +3 m to speed while not in heavy armor\\
	~ \feat{Psionic Charge}\footnotemark[1] & Dex 13, \feat{Speed of Thought} & Turn 90 degrees while charging\\
	\feat{Up The Walls}\footnotemark[2] & Wis 13 & Move on walls for brief distances\\
	\feat{Wind Racer}\footnotemark[2] & \skill{Balance} 2 ranks, \skill{Profession} (sailor) 1 rank & Double sail cart's speed for 1 round after skill check\\
	\feat{Wounding Attack}\footnotemark[1] & Base attack bonus +8 & Attack deals additional 1 point of Con damage\\
	\rowcolor{white}\multicolumn{3}{l}{1 Spend psionic focus}\\
	\rowcolor{white}\multicolumn{3}{l}{2 Must be psionic focused}\\
	\rowcolor{white}\vspace{1em}
}

\subsectionA{Psionic Feats}

\Feat[Psionic]{Aligned Attack}
{Your melee or ranged attack overcomes your opponent's alignment-based damage reduction and deals additional damage.}
{Base attack bonus +6.}
{When you take this feat, choose either chaos, good, evil or law. Your choice must match one of your alignment components. Once you've made this alignment choice, it cannot be changed.

To use this feat, you must expend your psionic focus. When you make a successful melee or ranged attack, you deal an extra 1d6 points of damage, and your attack is treated as either a good, evil, chaotic, or lawful attack (depending on your original choice) for the purpose of overcoming damage reduction.

You must decide whether or not to use this feat prior to making an attack. If your attack misses, you still expend your psionic focus.}
{}{}

\Feat[Psionic]{Body Fuel}
{You can expand your power point total at the expense of your health.}
{}
{You can recover 10 power points by taking 1 point of ability burn damage to each of your three ability scores: Strength, Dexterity, and Constitution.

You can recover additional power points for a proportional cost to Strength, Dexterity, and Constitution. These recovered points are added to your power point reserve as if you had gained them by resting overnight.}
{}
{Only living creatures can use this feat. You can take advantage of this feat only while in your own body.}

\Feat[Psionic]{Boost Construct}
{Your \emph{astral constructs} have more abilities.}
{}
{When you create an \psionic{astral construct}, you can give it one additional special ability from any menu that the construct currently has an ability from.}
{}{}

\Feat[Psionic]{Combat Manifestation}
{You are adept at manifesting powers in combat.}
{}
{You get a +4 bonus on \skill{Concentration} checks made to manifest a power or use a psi-like ability while on the defensive or while you are grappling or pinned.}
{}{}

\Feat[Psionic]{Deep Impact}
{You can strike your foe with a melee weapon as if making a touch attack.}
{Str 13, \feat{Psionic Weapon}, base attack bonus +5.}
{To use this feat, you must expend your psionic focus. You can resolve your attack with a melee weapon as a touch attack. You must decide whether or not to use this feat prior to making an attack. If your attack misses, you still expend your psionic focus.}
{}{}

\Feat[Psionic]{Elemental Manifestation}
{Your patron element aids you in your energy manifestations.}
{Access to domain spells, manifester level 3rd.}
{To use this feat, you must expend your psionic focus. You add 2 to the save DC of a power you manifest if that power has the same descriptor as your patron element.}
{}{}

\Feat[Psionic]{Expanded Knowledge}
{You learn another power.}
{Manifester level 3rd.}
{Add to your powers known one additional power of any level up to one level lower than the highest-level power you can manifest. You can choose any power, including powers from another discipline's list or even from another class's list.}
{}
{You can gain this feat multiple times. Each time, you learn one new power at any level up to one less than the highest-level power you can manifest.}

\Feat[Psionic]{Fell Shot}
{You can strike your foe with a ranged weapon as if making a touch attack.}
{Dex 13, \feat{Point Blank Shot}, \feat{Psionic Shot}, base attack bonus +5.}
{To use this feat, you must expend your psionic focus. You can resolve your ranged attack as a ranged touch attack.

You must decide whether or not to use this feat prior to making an attack. If your attack misses, you still expend your psionic focus.}
{}{}

\Feat[Psionic]{Focused Mind}
{Your meditations strengthen your reasoning.}
{Int 13.}
{As long as you are psionically focused, you receive a +2 bonus on any Intelligence-based skill checks.}
{}{}

\Feat[Psionic]{Focused Sunder}
{You can sense the stress points on others' weapons.}
{Str 13, \feat{Power Attack}, \feat{Improved Sunder}.}
{To use this feat, you must expend your psionic focus.

When you strike at an opponent's weapon, you ignore half of the weapon's total hardness (round down). Total hardness includes any magical or psionic enhancements possessed by the weapon that increase its hardness.}
{}
{You can also sense the stress points in any hard construction, such as wooden doors or stone walls, and can ignore half of the object's total hardness (round down) when attacking that object.}

\Feat[Psionic]{Ghost Attack}
{Your deadly strikes against incorporeal foes always find their mark.}
{Base attack bonus +3.}
{You must be psionically focused to use this feat. When you make a melee attack or a ranged attack against an incorporeal creature, you can make two rolls to check for the miss chance. If either is successful, the attack is treated as if it were made with a ghost touch weapon for the purpose of affecting the creature. Your weapon or natural weapon actually appears to become briefly incorporeal as the attack is made.}
{}{}

% \Feat[Psionic]{Greater Power Penetration}
% {Your powers are especially potent at breaking through power resistance.}
% {\feat{Power Penetration}.}
% {To use this feat, you must expend your psionic focus. You get a +8 bonus on manifester level checks to overcome a creature's power resistance. This bonus overlaps with the bonus from \feat{Power Penetration}.}
% {}{}

\Feat[Psionic]{Greater Power Specialization}
{You deal more damage with your powers.}
{\feat{Power Specialization}, \feat{Weapon Focus} (ray), manifester level 12th.}
{Your powers that deal damage deal an extra 2 points of damage. This damage stacks with other bonuses on damage rolls to powers, including the one from \feat{Power Specialization}. The damage bonus applies only if the target or targets are within 9 meters.}
{}{}

\Feat[Psionic]{Greater Psionic Endowment}
{You can use meditation to focus your powers.}
{\feat{Psionic Endowment}.}
{When you use the \feat{Psionic Endowment} feat, you add +2 to the save DC of a power you manifest instead of +1.}
{}{}

\Feat[Psionic]{Greater Psionic Fist}
{You can charge your unarmed strike or natural weapon with additional damage potential.}
{Str 13, \feat{Psionic Fist}, base attack bonus +5.}
{When you use the \feat{Psionic Fist} feat, your unarmed attack or attack with a natural weapon deals an extra 4d6 points of damage instead of an extra 2d6 points.}
{}{}

\Feat[Psionic]{Greater Psionic Shot}
{You can charge your ranged attacks with additional damage potential.}
{\feat{Point Blank Shot}, \feat{Psionic Shot}, base attack bonus +5.}
{When you use the \feat{Psionic Shot} feat, your ranged attack deals an extra 4d6 points of damage instead of an extra 2d6 points.}
{}{}

\Feat[Psionic]{Greater Psionic Weapon}
{You can charge your melee weapon with additional damage potential.}
{Str 13, \feat{Psionic Weapon}, base attack bonus +5.}
{When you use the \feat{Psionic Weapon} feat, your attack with a melee weapon deals an extra 4d6 points of damage instead of an extra 2d6 points.}
{}{}

\Feat[Psionic]{Improved Dwarven Focus}
{You can use the Way to help fulfill your focus.}
{Dwarf.}
{You must be psionically focused to use this feat. While actively pursuing your dwarven focus, you receive a +2 morale bonus on all checks related to your focus.}
{You receive a +1 morale bonus on checks related to completing your focus.}
{}

\Feat[Psionic]{Improved Elf Run}
{You can use the Way to run faster.}
{Elf.}
{You must be psionically focused to use this feat. While in an elf run state, you gain an insight bonus to your speed of 4.5 meters.}
{}{}

\Feat[Psionic]{Improved Psicrystal}
{You can upgrade your psicrystal.}
{\feat{Psicrystal Affinity}.}
{You can implant another personality fragment in your psicrystal. You gain the benefits of both psicrystal personalities. Your psicrystal's personality adjusts and becomes a blend between all implanted personality fragments. From now on, when determining the abilities of your psicrystal, treat your manifester level as one higher than your normal manifester level.}
{}
{You can gain this feat multiple times. Each time, you implant a new personality fragment in your psicrystal, from which you derive the noted benefits, and you treat your level as one higher for the purpose of determining your psicrystal's abilities.}

\Feat[Psionic]{Inquisitor}
{You know when others lie.}
{Wis 13.}
{To use this feat, you must expend your psionic focus.

You gain a +10 bonus on a \skill{Sense Motive} check to oppose a \skill{Bluff} check.

You must decide whether or not to use this feat prior to making a \skill{Sense Motive} check. If your check fails, or if the opponent isn't lying, you still expend your psionic focus.}
{}{}

\Feat[Psionic]{Jump Charge}
{You can charge an opponent by jumping at them, hitting the enemy with a powerful attack.}
{\feat{Psionic Fist} or \feat{Psionic Weapon}, \skill{Jump} 8 ranks.}
{To use this feat, you must expend your psionic focus. When charging an opponent, you may jump at them as part of the movement. Make a \skill{Jump} check. If your horizontal jump is at least 3 meters and you end your jump in a square in which you may threaten the opponent, you may increase by one-half the damage dealt by your \feat{Psionic Weapon} or \feat{Psionic Fist}. While using a two-handed weapon, damage is doubled instead. This attack must follow all the rules for charging and the \skill{Jump} skill, with the exception that you ignore the ground terrain in any spaces you jump over.}
{}{}

\Feat[Psionic]{Mental Leap}
{You can make amazing jumps.}
{Str 13, \skill{Jump} 5 ranks.}
{To use this feat, you must expend your psionic focus. You gain a +10 bonus on a \skill{Jump} check.}
{}{}

\Feat[Psionic]{Metamorphic Transfer}
{You can gain a supernatural ability of a metamorphed form.}
{Wis 13, manifester level 5th.}
{Each time you change your form, such as through the \psionic{metamorphosis} power, you gain one of the new form's supernatural abilities, if it has any.

You gain only three uses of the metamorphic ability per day, even if the creature into which you metamorph has a higher limit on uses (you are still subject to other restrictions on the use of the ability.) The save DC to resist a supernatural ability gained through Metamorphic Transfer (if it is an attack) is 10 + your Cha modifier + \onehalf your Hit Dice. No matter how many times you manifest the \psionic{metamorphosis} power on a given day, you can gain only a total of three supernatural ability transfers per day.}
{You cannot use the supernatural abilities of creatures whose form you assume.}
{You can gain this feat multiple times. Each time, you can gain one additional supernatural ability.}

\Feat[Psionic]{Narrow Mind}
{Your ability to concentrate is as keen as an arrowhead, allowing you to gain your psionic focus even in the most turbulent situations.}
{Wis 13.}
{You gain a +4 bonus on \skill{Concentration} checks you make to become psionically focused.}
{}{}

\Feat[Psionic]{Overchannel}
{You burn your life force to strengthen your powers.}
{}
{While manifesting a power, you can increase your total power check bonus by one, but in so doing you take 1d8 points of damage. At 8th level, you can choose to increase your total power check bonus by two, but you take 3d8 points of damage. At 15th level, you can increase your total power check bonus by three, but you take 5d8 points of damage.

% The effective increase in manifester level increases the number of power points you can expend on a single power manifestation, as well as increasing all manifester level-dependent effects, such as range, duration, and overcoming power resistance.
}
{Your total power check bonus is equal to the key ability score minus the power level.}
{}

% \Feat[Psionic]{Power Penetration}
% {Your powers are especially potent, breaking through power resistance more readily than normal.}
% {}
% {To use this feat, you must expend your psionic focus. You get a +4 bonus on manifester level checks made to overcome a creature's power resistance.}
% {}{}

\Feat[Psionic]{Power Specialization}
{You deal more damage with your powers.}
{\feat{Weapon Focus} (ray), manifester level 4th.}
{With rays and ranged touch attack powers that deal damage, you deal an extra 2 points of damage. If you expend your psionic focus when you manifest a ray or a ranged touch attack power that deals damage, you add your key ability bonus to the damage (instead of adding 2).}
{}{}

\Feat[Psionic]{Psicrystal Affinity}
{You have created a psicrystal.}
{Manifester level 1st.}
{This feat allows you to gain a psicrystal.}
{}{}

\Feat[Psionic]{Psicrystal Containment}
{Your psicrystal has advanced enough that it can hold a psionic focus that you store within it.}
{\feat{Psicrystal Affinity}, manifester level 3rd.}
{You can spend a full-round action attempting to psionically focus your psicrystal. At any time when you need to expend your psionic focus, you can expend your psicrystal's psionic focus instead, as long as the crystal is within 1.5 meter of you. Psionically focusing your psicrystal works just like focusing yourself. The psicrystal cannot focus itself---only the owner can spend the time to focus the crystal.}
{}{}

\Feat[Psionic]{Psionic Body}
{Your mind reinforces your body.}
{}
{When you take this feat, you gain 2 hit points for each psionic feat you have (including this one). Whenever you take a new psionic feat, you gain 2 more hit points.}
{}{}

\Feat[Psionic]{Psionic Charge}
{You can charge in a crooked line.}
{Dex 13, \feat{Speed of Thought}.}
{To use this feat, you must expend your psionic focus. When you charge, you can make one turn of up to 90 degrees during your movement. All other restrictions on charges still apply; for instance, you cannot pass through a square that blocks or slows movement, or that contains a creature. You must have line of sight to the opponent at the start of your turn.}
{}{}

\Feat[Psionic]{Psionic Dodge}
{You are proficient at dodging blows.}
{Dex 13, \feat{Dodge}.}
{You must be psionically focused to use this feat. You receive a +1 dodge bonus to your Armor Class. This bonus stacks with the bonus from the \feat{Dodge} feat (but only applies on attacks made by the opponent you have designated).}
{}{}

\Feat[Psionic]{Psionic Endowment}
{You can endow your manifestations with more concentrated focus.}
{}
{To use this feat, you must expend your psionic focus. You add 1 to the save DC of a power you manifest.}
{}{}

\Feat[Psionic]{Psionic Fist}
{You can charge your unarmed strike or natural weapon with additional damage potential.}
{Str 13.}
{To use this feat, you must expend your psionic focus. Your unarmed strike or attack with a natural weapon deals an extra 2d6 points of damage.

You must decide whether or not to use this feat prior to making an attack. If your attack misses, you still expend your psionic focus.}
{}{}

\Feat[Psionic]{Psionic Meditation}
{You can focus your mind faster than normal, even under duress.}
{Wis 13, \skill{Concentration} 7 ranks.}
{You can take a move action to become psionically focused.}
{A character without this feat must take a full-round action to become psionically focused.}
{}

\Feat[Psionic]{Psionic Shot}
{You can charge your ranged attacks with additional damage potential.}
{\feat{Point Blank Shot}.}
{To use this feat, you must expend your psionic focus. Your ranged attack deals +2d6 points of damage. You must decide whether or not to use this feat prior to making an attack. If your attack misses, you still expend your psionic focus.}
{}{}

\Feat[Psionic]{Psionic Talent}
{You gain additional power points to supplement those you already had.}
{Having a power point reserve.}
{When you take this feat for the first time, you gain 10 power points.}
{}
{You can take this feat multiple times. Each time you take the feat after the first time, the number of power points you gain increases by 5.}

\Feat[Psionic]{Psionic Weapon}
{You can charge your melee weapon with additional damage potential.}
{Str 13.}
{To use this feat, you must expend your psionic focus.

Your attack with a melee weapon deals an extra 2d6 points of damage. You must decide whether or not to use this feat prior to making an attack. If your attack misses, you still expend your psionic focus.}
{}{}

\Feat[Psionic]{Pterran Telepathy}
{You can leverage your missive psi-like ability to communicate with other creatures.}
{Pterran, missive psi-like ability.}
{You can use your missive ability to communicate with all humanoid creatures in addition to reptiles. Your manifester level for this effect is equal to \onehalf your Hit Dice (minimum 1st).}
{}{}

\Feat[Psionic]{Return Shot}
{You can return incoming arrows, as well as crossbow bolts, spears, and other projectile or thrown weapons.}
{\feat{Point Blank Shot}, \feat{Psionic Shot}, \feat{Fell Shot}, base attack bonus +3.}
{To use this feat, you must expend your psionic focus and have at least one hand free. Once per round when you would normally be hit by a projectile or a thrown weapon no more than one size category larger than your size, you can deflect the attack so that you take no damage from it. The attack is deflected back at your attacker, using the attack bonus of the original attack on you. You must be aware of the attack and not flat-footed. Attempting to return a shot is a free action.}
{}
{If you also have the \feat{Deflect Arrows} feat, the deflected attack is made with the original attack bonus plus your Dexterity bonus.}

% \Feat[Psionic]{Shipfloater}
% {You are a trained shipfloater, even knowing how to keep control of an obsidian engine even while taking other actions.}
% {}
% {You gain a +3 bonus on \skill{Profession} (sailor) check. You can also control an obsidian engine as a standard action. You must stay psionically focused and remain in touch with the engine as normal.}
% {You can control an obsidian engine as a full-round action. }
% {}

\Feat[Psionic]{Speed of Thought}
{The energy of your mind energizes the alacrity of your body.}
{Wis 13.}
{As long as you are psionically focused and not wearing heavy armor, you gain an insight bonus to your speed of 3 meters.}
{}{}

\Feat[Psionic]{Talented}
{You can overchannel powers with less cost to yourself.}
{\feat{Overchannel}.}
{To use this feat, you must expend your psionic focus. When manifesting a power of 3rd level or lower, you do not take damage from overchanneling.}
{}{}

\Feat[Psionic]{Unavoidable Strike}
{You can make an unarmed strike or use a natural weapon against your foe as if delivering a touch attack.}
{Str 13, \feat{Psionic Fist}, base attack bonus +5.}
{To use this feat, you must expend your psionic focus. You can resolve your unarmed strike or attack with a natural weapon as a touch attack.

You must decide whether or not to use this feat prior to making an attack. If your attack misses, you still expend your psionic focus.}
{}{}

\Feat[Psionic]{Up The Walls}
{You can run on walls for brief distances.}
{Wis 13.}
{While you are psionically focused, you can take part of one of your move actions to traverse a wall or other relatively smooth vertical surface if you begin and end your move on a horizontal surface. The height you can achieve on the wall is limited only by this movement restriction. If you do not end your move on a horizontal surface, you fall prone, taking falling damage as appropriate for your distance above the ground. Treat the wall as a normal floor for the purpose of measuring your movement. Passing from floor to wall or wall to floor costs no movement; you can change surfaces freely. Opponents on the ground can make attacks of opportunity as you move up the wall.}
{}
{You can take other move actions in conjunction with moving along a wall. For instance, the \feat{Spring Attack} feat allows you to make an attack from the wall against a foe standing on the ground who is within the area you threaten; however, if you are somehow prevented from completing your move, you fall. Likewise, you could tumble along the wall to avoid attacks of opportunity.}

\Feat[Psionic]{Wild Telepath}
{}
{Cha 13.}
{You use your Charisma as the key ability for metacreativity and telepathy disciplines. This applies to power DCs, power checks, and the minimum ability value to learn and manifest a power.}
{You use Intelligence as key ability for metacreativity powers, and Wisdom as key ability for telepathy powers.}
{}

\Feat[Psionic]{Wind Racer}
{You can achieve fantastic speeds with your sail cart in salt flats or sandy wastes.}
{\skill{Balance} 2 ranks, \skill{Profession} (sailor) 1 rank.}
{You must be psionically focused to use this feat. By making a successful \skill{Profession} (sailor) check, you can double your sail cart's speed for 1 round.}
{}{}

\Feat[Psionic]{Wounding Attack}
{Your vicious attacks wound your foe.}
{Base attack bonus +8.}
{To use this feat, you must expend your psionic focus. You can make an attack with such vicious force that you wound your opponent. A wound deals 1 point of Constitution damage to your foe in addition to the usual damage dealt.

You must decide whether or not to use this feat prior to making an attack. If your attack misses, you still expend your psionic focus.}
{}{}

%\newpage
\FeatTable[p{3cm}]{Divine}{
	\feat{Elemental Affinity} & Cha 13, ability to turn or rebuke undead & Gain +4 to Fort saves against patron element for some rounds\\
	\feat{Elemental Cleansing} & Ability to turn or rebuke undead & Turn undead deals 2d6 energy damage\\
	\feat{Elemental Might} & Str 13, ability to turn or rebuke undead, \feat{Power Attack} & Add Cha modifer as energy damage for 1 full round\\
	\feat{Elemental Vengeance} & Ability to turn undead, \feat{Extra Turning} & +2d6 energy damage to melee attacks against undead\\
	\feat{Superior Blessing} & Ability to turn or rebuke undead & Creatures exposed to blessed element take double damage
}

\FeatTable[p{3cm}]{Raze}{
	\feat{Agonizing Radius} & Defiler & Increase penalties for being caught in defiling radius\\
	~ \feat{Sickening Raze} & \feat{Agonizing Radius}, defiler & Creatures within defiling radius become nauseated\\
	\feat{Controlled Raze} & Defiler & Increases defiling radius by 1.5 m one spot may be unaffected\\
	\feat{Distance Raze} & Defiler & Move the center of the defiling radius\\
	\feat{Destructive Raze} & Defiler & +1 damage per die of evocation spells while defiling\\
	\feat{Efficient Raze} & Defiler & Treat terrain as one category better\\
	\feat{Exterminating Raze} & Defiler & Double damage for plant creatures caugt in defiling radius\\
	\feat{Fast Raze} & Defiler & Get +1 caster level as move action when defiling\\
}

\FeatTable[p{3cm}]{Metamagic}{
	\feat{Disguise Spell} & \skill{Perform} 12 ranks & Spell can't be identified by \skill{Spellcraft}\\
	\feat{Empower Spell} && Increase all variables of a spell by 50\%\\
	\feat{Energy Substitution} & Any other metamagic feat, \skill{Knowledge} (arcana) 5 ranks & Energy spells deal different energy damage\\
	\feat{Enlarge Spell} && Increase range of a spell by 100\%\\
	\feat{Extend Spell} && Double duration of a spell\\
	~ \feat{Persistent Spell} & \feat{Extend Spell} & Fixed or personal spells last 24 hours\\
	\feat{Heighten Spell} && Increase spell effective level\\
	\feat{Maximize Spell} && Maximize all variables of a spell\\
	\feat{Quicken Spell} && Cast spell as swift action\\
	\feat{Reach Spell} && Touch spell becomes a ray with 9-meter range\\
	\feat{Repeat Spell} & Any other metamagic feat & Spell is automatically cast again next round\\
	\feat{Silent Spell} && Cast spell without verbal component\\
	\feat{Still Spell} && Cast spell without somatic component\\
	\feat{Widen Spell} && Increase area of a spell by 100\%\\
}

\FeatTable[p{3cm}]{Metapsionic}{
	\feat{Burrowing Power} && Manifest power through barriers\\
	\feat{Chain Power} && Power that deals energy damage chains to secondary targets\\
	\feat{Delay Power} && Power activates after a trigger\\
	\feat{Empower Power} && Increase all variables of a power by 50\%\\
	\feat{Enlarge Power} && Increase range of a power by 100\%\\
	\feat{Extend Power} && Double duration of a power\\
	\feat{Maximize Power} && Maximize all variables of a power\\
	\feat{Opportunity Power} && Manifest power as attack of opportunity\\
	\feat{Quicken Power} && Manifest power as swift action\\
	\feat{Split Psionic Ray} & Any other metapsionic feat & Split psionic ray into two\\
	\feat{Twin Power} && Manifest a power twice at the same time\\
	\feat{Unconditional Power} && Manifest power in dazed, confused, nauseated, or stunned\\
	\feat{Widen Power} && Increase area of a power by 100\%
}

\FeatTable{Item Creation}{
	\feat{Brew Potion} & Caster level 3rd & Create magic potions\\
	\feat{Craft Cognizance Crystal} & Manifester level 3rd & Create power point storage\\
	\feat{Craft Construct} & \feat{Craft Magic Arms and Armor}, \feat{Craft Wondrous Item} & Create golems and other magical constructs\\
	% \feat{Craft Dorje} & Manifester level 5th & Create psionic wand\\
	\feat{Craft Magic Arms and Armor} & Caster level 5th & Create magic weapons, armor and shields\\
	\feat{Craft Psionic Item} & Manifester level 9th & Create intelligent psionic items\\
	% \feat{Craft Psicrown} & Manifester level 12th & Create psionic staff\\
	% \feat{Craft Psionic Arms and Armor} & Manifester level 5th & Create psionic weapons, armor and shields\\
	% \feat{Craft Psionic Construct} & \feat{Craft Psionic Arms and Armor}, \feat{Craft Universal Item} & Create golems and constructs using psionism\\
	\feat{Craft Rod} & Caster level 9th & Create magic rods\\
	\feat{Craft Staff} & Caster level 12th & Create magic staffs\\
	% \feat{Craft Universal Item} & Manifester level 3rd & Create universal items\\
	\feat{Craft Wand} & Caster level 5th & Create magic wand\\
	\feat{Craft Wondrous Item} & Caster level 3rd & Create wondrous items\\
	\feat{Forge Ring} & Caster level 12th & Create magic rings\\
	% \feat{Imprint Stone} & Manifester level 1st & \\
	\feat{Scribe Scroll} & Caster level 1st, literacy & Scribe magic scrolls\\
	% \feat{Scribe Tattoo} & Manifester level 3rd & Scribe psionic tattoos\\
	\vspace{.2em}
}

\subsectionA{Divine Feats}

\Feat[Divine]{Elemental Affinity}
{You are protected by your patron element.}
{Cha 13, ability to turn or rebuke undead.}
{As a free action, spend one of your turn or rebuke undead attempts to add a +4 sacred bonus to Fortitude saving throws against attacks by the energy type associated with your patron element for a number of rounds equal to your Charisma modifier.}
{}{}

\Feat[Divine]{Elemental Cleansing}
{Undead you turn or rebuke suffer elemental damage.}
{Ability to turn or rebuke undead.}
{Any undead that you successfully turn or rebuke takes 2d6 points of energy damage in addition to the normal turning or rebuking effect. The type of damage dealt is the one associated with your patron element.}
{}{}

\Feat[Divine]{Elemental Might}
{You can channel your element's energy to increase the damage you deal in combat.}
{Str 13, ability to turn or rebuke undead, \feat{Power Attack}.}
{As a free action, spend one of your turn or rebuke undead attempts to add 2 $\times$ your Charisma bonus as energy damage to your weapon for 1 full round. The type of damage dealt must match your patron element's descriptor.}
{}{}

\Feat[Divine]{Elemental Vengeance}
{You can channel your element's energy to deal extra damage against undead in melee.}
{Ability to turn undead, \feat{Extra Turning}.}
{As a free action, spend one of your turn undead attempts to add 2d6 points of energy damage to all your successful melee attacks against undead until the end of your next action. This is a supernatural ability. The type of damage dealt must match your patron element's descriptor.}
{}{}

\Feat[Divine]{Superior Blessing}
{You can bless your element with stronger positive energies.}
{Ability to turn or rebuke undead.}
{As a standard action, spend one of your turn or rebuke undead attempts to double the potency of your blessed element. Creatures exposed to your blessed element take 4d4 points of damage.}
{Creatures exposed to blessed elements receive 2d4 points of damage.}
{}
\section{Raze Feats}

\Feat[Raze]{Agonizing Radius}
{Your defiling techniques are particularly painful.}
{Defiler.}
{The penalties for being caught within your defiling radius increase by one (i.e. from $-1$ to $-2$).}
{}{}

\Feat[Raze]{Controlled Raze}
{You increase your defiling radius and may specify unaffected squares.}
{Defiler.}
{Your defiling radius increases by 1.5 meter. You can specify one 1.5 m square per 1.5 m radius of your defiling circle that is unaffected by your energy gathering. Creatures in unaffected squares do not suffer the adverse effects of being caught in the defiling circle, nor is vegetation in that square turned to ash.}
{}{}

\Feat[Raze]{Distance Raze}
{You can gather energy for spells at a distance.}
{Defiler.}
{You can move the center of your defiling circle (on the ground) up to 3 meters per caster level, in effect moving the entire circle of defiling.}
{Your defiling circle is centered on you.}
{}

\Feat[Raze]{Destructive Raze}
{You can focus the energy you absorb from plants to increase the damage your spells inflict.}
{Defiler.}
{Add +1 to damage per damage die inflicted by evocation spells when defiling.}
{}{}

\Feat[Raze]{Efficient Raze}
{You can gather energy more efficiently, utilizing the maximum energy potential of a given terrain.}
{Defiler.}
{Treat the terrain you gather energy in as one category better when you defile. e.g. a spell cast in barren terrain ($-1$ spell save DC and $-1$ penalty to caster level checks) is treated as if cast in infertile terrain (no spell save modifier and no penalty to caster level checks). In abundant terrain the bonuses to spell save DCs and spell checks are increased by an additional +1. This feat has no effect in the Obsidian Plains.}
{}{}

\Feat[Raze]{Exterminating Raze}
{Your defiling techniques are particularly damaging to plant creatures.}
{Defiler.}
{Plant creatures caught in your defiling radius suffer 4 points of damage per spell level.}
{Plant creatures caught in your defiling radius suffer 2 points of damage per spell level.}
{}

\Feat[Raze]{Fast Raze}
{You can gather energy faster.}
{Defiler.}
{When defiling, you can spend a move action to gain a +1 caster level bonus. Spells with a normal casting time of 1 round or longer still require an extra round to be cast in this manner. Your defiling radius increases by 1.5 m when using Fast Raze.}
{It takes one round to gain the caster level benefit.}
{}

\Feat[Raze]{Sickening Raze}
{Your defilement makes others sick.}
{\feat{Agonizing Radius}, defiler.}
{Creatures within your defiling radius become nauseated in addition to any other penalties for 1 round.}
{}{}
\section{Metamagic Feats}

\GFeat[Metamagic]{Empower Spell}{}
{All variable, numeric effects of an empowered spell are increased by one-half.

Saving throws and opposed rolls are not affected, nor are spells without random variables. An empowered spell uses up a spell slot two levels higher than the spell's actual level.}

\GFeat[Metamagic]{Enlarge Spell}{}
{You can alter a spell with a range of close, medium, or long to increase its range by 100\%. An enlarged spell with a range of close now has a range of 50 ft. + 5 ft./level, while medium-range spells have a range of 200 ft. + 20 ft./level and long-range spells have a range of 800 ft. + 80 ft./level. An enlarged spell uses up a spell slot one level higher than the spell's actual level.

Spells whose ranges are not defined by distance, as well as spells whose ranges are not close, medium, or long, do not have increased ranges.}

\GFeat[Metamagic]{Extend Spell}{}
{An extended spell lasts twice as long as normal. A spell with a duration of concentration, instantaneous, or permanent is not affected by this feat. An extended spell uses up a spell slot one level higher than the spell's actual level.}

\GFeat[Metamagic]{Heighten Spell}{}
{A heightened spell has a higher spell level than normal (up to a maximum of 9th level). Unlike other metamagic feats, Heighten Spell actually increases the effective level of the spell that it modifies. All effects dependent on spell level (such as saving throw DCs and ability to penetrate a lesser globe of invulnerability) are calculated according to the heightened level. The heightened spell is as difficult to prepare and cast as a spell of its effective level.}

\GFeat[Metamagic]{Maximize Spell}{}
{All variable, numeric effects of a spell modified by this feat are maximized. Saving throws and opposed rolls are not affected, nor are spells without random variables. A maximized spell uses up a spell slot three levels higher than the spell's actual level.

An empowered, maximized spell gains the separate benefits of each feat: the maximum result plus:  ne-half the normally rolled result.}

\Feat[Metamagic]{Quicken Spell}{}{}
{Casting a quickened spell is an swift action. You can perform another action, even casting another spell, in the same round as you cast a quickened spell. You may cast only one quickened spell per round. A spell whose casting time is more than 1 full round action cannot be quickened. A quickened spell uses up a spell slot four levels higher than the spell's actual level. Casting a quickened spell doesn't provoke an attack of opportunity.}
{}
{This feat can't be applied to any spell cast spontaneously (including templar spells, and cleric or druid spells cast spontaneously), since applying a metamagic feat to a spontaneously cast spell automatically increases the casting time to a full-round action.}

\GFeat[Metamagic]{Silent Spell}{}
{A silent spell can be cast with no verbal components. Spells without verbal components are not affected. A silent spell uses up a spell slot one level higher than the spell's actual level.}

\GFeat[Metamagic]{Still Spell}{}
{A stilled spell can be cast with no somatic components.

Spells without somatic components are not affected. A stilled spell uses up a spell slot one level higher than the spell's actual level.}

\GFeat[Metamagic]{Widen Spell}{}
{You can alter a burst, emanation, line, or spread shaped spell to increase its area. Any numeric measurements of the spell's area increase by 100\%. A widened spell uses up a spell slot three levels higher than the spell's actual level.

Spells that do not have an area of one of these four sorts are not affected by this feat.}

\section{Metapsionic Feats}

\Feat[Metapsionic]{Burrowing Power}
{Your powers sometimes bypass barriers.}{}
{To use this feat, you must expend your psionic focus. You can attempt to manifest your powers against targets that are sheltered behind a wall or force effect. Your power briefly skips through the Astral Plane to bypass the barrier.

The strength and thickness of the barrier determine your chance of success. To successfully bypass the barrier with your power, you make a \skill{Psicraft} check against a DC equal to 10 + the hardness of the barrier + 1 per 30 centimeters of thickness (minimum 1). Assign a hardness of 20 to barriers without a hardness rating, such as force effects (or a \psionic{wall of ectoplasm}). Force walls or \emph{walls of ectoplasm} are assumed to have less than 30 centimeters of thickness unless noted otherwise.

If a power requires line of sight (which includes most powers that affect a target or targets instead of an area), you cannot manifest it as a burrowing power unless you can somehow see the target, such as with \psionic{clairvoyant sense}.

Using this feat increases the power point cost of the power by 2. The power's total cost cannot exceed your manifester level.}{}{}

\Feat[Metapsionic]{Chain Power}
{You can manifest powers that arc to hit other targets in addition to the primary target.}
{}
{To use this feat, you must expend your psionic focus. You can chain any power that affects a single target and that deals either acid, cold, electricity, fire, or sonic damage. After the primary target is struck, the power can arc to a number of secondary targets equal to your manifester level (maximum twenty). The secondary arcs each strike one target and deal half as much damage as the primary one did (round down).

Each target gets to make a saving throw, if one is allowed by the power. You choose secondary targets as you like, but they must all be within 9 meters of the primary target, and no target can be struck more than once. You can choose to affect fewer secondary targets than the maximum (to avoid allies in the area, for example).

Using this feat increases the power point cost of the power by 6. The power's total cost cannot exceed your manifester level.}{}{}

\Feat[Metapsionic]{Delay Power}
{You can manifest powers that go off up to 5 rounds later.}
{}
{To use this feat, you must expend your psionic focus. You can manifest a power as a delayed power. A delayed power doesn't activate immediately. When you manifest the power, you choose one of three trigger mechanisms: (1) The power activates when you take a standard action to activate it; (2) It activates when a creature enters the area that the power will affect (only powers that affect areas can use this trigger condition); or (3) It activates on your turn after 5 rounds pass. If you choose one of the first two triggers and the conditions are not met within 5 rounds, the power activates automatically on the fifth round.

Only area and personal powers can be delayed.

Any decisions you would make about the delayed power, including attack rolls, designating targets, or determining or shaping an area, are decided when the power is manifested. Any effects resolved by those affected by the power, including saving throws, are decided when the delay period ends.

A delayed power can be dispelled normally during the delay, and can be detected normally in the area or on the target by the use of powers that can detect psionic effects.

Using this feat increases the power point cost of the power by 2. The power's total cost cannot exceed your manifester level.}{}{}

\Feat[Metapsionic]{Empower Power}
{You can manifest powers to greater effect.}
{}
{To use this feat, you must expend your psionic focus.

You can empower a power. All variable, numeric effects of an empowered power are increased by one-half. An empowered power deals half again as much damage as normal, cures half again as many hit points, affects half again as many targets, and so forth, as appropriate. Augmented powers can also be empowered (multiply 1\onehalf times the damage total of the augmented power). Saving throws and opposed checks (such as the one you make when you manifest dispel psionics) are not affected, nor are powers without random variables.

Using this feat increases the power point cost of the power by 2. The power's total cost cannot exceed your manifester level.}{}{}

\Feat[Metapsionic]{Enlarge Power}
{You can manifest powers farther than normal.}
{}
{To use this feat, you must expend your psionic focus. You can alter a power with a range of close, medium, or long to increase its range by 100\%. An enlarged power with a range of close has a range of 15 m + 1.5m per level, a medium-range power has a range of 60 m + 6 m per level, and a long-range power has a range of 240 m + 24 m per level.

Powers whose ranges are not defined by distance, as well as powers whose ranges are not close, medium, or long, are not affected.

Using this feat does not increase the power point cost of the power.}{}{}

\Feat[Metapsionic]{Extend Power}
{You can manifest powers that last longer than normal.}
{}
{To use this feat, you must expend your psionic focus.

You can manifest an extended power. An extended power lasts twice as long as normal. A power with a duration of concentration, instantaneous, or permanent is not affected by this feat.

Using this feat increases the power point cost of the power by 2. The power's total cost cannot exceed your manifester level.}{}{}

\Feat[Metapsionic]{Maximize Power}
{You can manifest powers to maximum effect.}
{}
{To use this feat, you must expend your psionic focus.

You can maximize a power. All variable, numeric effects of a power modified by this feat are maximized. A maximized power deals maximum damage, cures the maximum number of hit points, affects the maximum number of targets, and so on, as appropriate. Saving throws and opposed checks are not affected, nor are powers without random variables.

Augmented powers can be maximized; a maximized augmented power deals the maximum damage (or cures the maximum hit points, and so on) of the augmented power.

An empowered and maximized power gains the separate benefits of each feat: the maximum result plus one-half the normally rolled result.

Using this feat increases the power point cost of the power by 4. The power's total cost cannot exceed your manifester level.}{}{}

\Feat[Metapsionic]{Opportunity Power}
{You can make power-enhanced attacks of opportunity.}
{}
{To use this feat, you must expend your psionic focus. When you make an attack of opportunity, you can use any power you know with a range of touch, if you have at least one hand free.

Manifesting this power is an immediate action.

You cannot use this feat with a touch power whose manifesting time is longer than 1 full-round action.

Using this feat increases the power point cost of the power by 6. The power's total cost cannot exceed your manifester level.}
{Attacks of opportunity can be made only with melee weapons.}{}

\Feat[Metapsionic]{Quicken Power}
{You can manifest a power with a moment's thought.}
{}
{To use this feat, you must expend your psionic focus. You can quicken a power. You can perform another action, even manifest another power, in the same round that you manifest a quickened power. You can manifest only one quickened power per round. A power whose manifesting time is longer than 1 round cannot be quickened.

Using this feat increases the power point cost of the power by 6. The power's total cost cannot exceed your manifester level.

Manifesting a quickened power does not provoke attacks of opportunity.}{}{}

\Feat[Metapsionic]{Split Psionic Ray}
{You can affect two targets with a single ray.}
{Any other metapsionic feat.}
{To use this feat, you must expend your psionic focus. You can split psionic rays you manifest. The split ray affects any two targets that are both within the power's range and within 9 meters of each other. If the ray deals damage, each target takes as much damage as a single target would take.

Using this feat increases the power point cost of the power by 2.}{}{}

\Feat[Metapsionic]{Twin Power}
{You can manifest a power simultaneously with another power just like it.}
{}
{To use this feat, you must expend your psionic focus. You can twin a power. Manifesting a power altered by this feat causes the power to take effect twice on the area or target, as if you were simultaneously manifesting the same power two times on the same location or target. Any variables in the power (such as duration, number of targets, and so on) are the same for both of the resulting powers. The target experiences all the effects of both powers individually and receives a saving throw (if applicable) for each. In some cases, such as a twinned psionic charm, failing both saving throws results in redundant effects (although, in this example, any ally of the target would have to succeed on two dispel attempts to free the target from the charm effect).

Using this feat increases the power point cost of the power by 6. The power's total cost cannot exceed your manifester level.}{}{}

\Feat[Metapsionic]{Unconditional Power}
{Disabling conditions do not hold you back.}
{}
{To use this feat, you must expend your psionic focus. Your mental strength is enough to overcome some otherwise disabling conditions. You can manifest an unconditional power when you are dazed, confused, nauseated, or stunned.

Only personal powers and powers that affect your person can be manifested as unconditional powers.

Using this feat increases the power point cost of the power by 8. The power's total cost cannot exceed your manifester level.}{}{}

\Feat[Metapsionic]{Widen Power}
{You can increase the area of your powers.}
{}
{To use this feat, you must expend your psionic focus. You can alter a burst, emanation, line, or spread-shaped power to increase its area. (Powers that do not have an area of one of these four sorts are not affected by this feat.) Any numeric measurements of the power's area increase by 100\%.

Using this feat increases the power point cost of the power by 4. The power's total cost cannot exceed your manifester level.}{}{}
\section{Item Creation Feats}

\Feat[Item Creation]{Brew Potion}
{}
{Caster level 3rd.}
{You can create a potion of any 3rd-level or lower spell that you know and that targets one or more creatures. Brewing a potion takes one day. When you create a potion, you set the caster level, which must be sufficient to cast the spell in question and no higher than your own level. The base price of a potion is its spell level $\times$ its caster level $\times$ 50 gp. To brew a potion, you must spend 1/25 of this base price in XP and use up raw materials costing one half this base price.

When you create a potion, you make any choices that you would normally make when casting the spell. Whoever drinks the potion is the target of the spell.

Any potion that stores a spell with a costly material component or an XP cost also carries a commensurate cost. In addition to the costs derived from the base price, you must expend the material component or pay the XP when creating the potion.}
{}
{On Athas, potions take many different forms. The most common form is an enchanted fruit, often called a \emph{potionfruit}. Other common items for enchantment include obsidian orbs, packs of herbs, and bone fetishes. The potion, regardless of material used to make it, is consumed or destroyed when used.

Due to the nature of their magic, defilers cannot enchant organic materials, such as fruits, as a potion. As a consequence, most non-defilers use those receptacles almost exclusively, as a way of assuring the potion is not a product of defiler magic.}

\Feat[Item Creation]{Craft Cognizance Crystal}
{You can create psionic cognizance crystals that store power points.}
{Manifester level 3rd.}
{You can create a cognizance crystal. Doing so takes one day for each 1,000 gp in its base price. The base price of a cognizance crystal is equal to the highest-level power it could manifest using all its stored power points, squared, multiplied by 1,000 gp. To create a cognizance crystal, you must spend 1/25 of its base price in XP and use up raw materials costing one-half its base price.}{}{}

\GFeat[Item Creation]{Craft Construct}
{\feat{Craft Magic Arms and Armor}, \feat{Craft Wondrous Item}.}
{A creature with this feat can create any construct whose prerequisites it meets. Enchanting a construct takes one day for each 1,000 gp in its market price. To enchant a construct, a spellcaster must spend 1/25 the item's price in XP and use up raw materials costing half of this price (see individual construct monster entries for details).

A creature with this feat can repair constructs that have taken damage. In one day of work, the creature can repair up to 20 points of damage by expending 50 gp per point of damage repaired.

A newly created construct has average hit points for its Hit Dice.}

\Feat[Item Creation]{Craft Dorje}
{You can create slender crystal wands called dorjes than manifest powers when charges are expended.}
{Manifester level 5th.}
{You can create a dorje of any psionic power you know (barring exceptions, such as bestow power, as noted in a power's description). Crafting a dorje takes one day for each 1,000 gp in its base price. The base price of a dorje is its manifester level $\times$ the power level $\times$ 750 gp. To craft a dorje, you must spend 1/25 of this base price in XP and use up raw materials costing one-half of this base price.

A newly created dorje has 50 charges.

Any dorje that stores a power with an XP cost also carries a commensurate cost. In addition to the XP cost derived from the base price, you must pay fifty times the XP cost.}{}{}

\GFeat[Item Creation]{Craft Magic Arms and Armor}
{Caster level 5th.}
{You can create any magic weapon, armor, or shield whose prerequisites you meet. Enhancing a weapon, suit of armor, or shield takes one day for each 1,000 gp in the price of its magical features. To enhance a weapon, suit of armor, or shield, you must spend 1/25 of its features' total price in XP and use up raw materials costing one-half of this total price.

The weapon, armor, or shield to be enhanced must be a masterwork item that you provide. Its cost is not included in the above cost.

You can also mend a broken magic weapon, suit of armor, or shield if it is one that you could make. Doing so costs half the XP, half the raw materials, and half the time it would take to craft that item in the first place.}

\Feat[Item Creation]{Craft Psicrown}
{You can create psicrowns, which have multiple psionic effects.}
{Manifester level 12th.}
{You can create any psicrown whose prerequisites you meet. Crafting a psicrown takes one day for each 1,000 gp in its base price. To craft a psicrown, you must spend 1/25 of its base price in XP and use up raw materials costing one-half of its base price. Some psicrowns incur extra costs in XP as noted in their descriptions. These costs are in addition to those derived from the psicrown's base price.}{}{}

\Feat[Item Creation]{Craft Psionic Arms and Armor}
{You can create psionic weapons, armor, and shields.}
{Manifester level 5th.}
{You can create any psionic weapon, armor, or shield whose prerequisites you meet. Enhancing a weapon, suit of armor, or shield takes one day for each 1,000 gp in the price of its psionic features. To enhance a weapon, you must spend 1/25 of its features' total price in XP and use up raw materials costing one-half of this total price.

The weapon, armor, or shield to be enhanced must be a masterwork item that you provide. Its cost is not included in the above cost.

You can also mend a broken psionic weapon, suit of armor, or shield if it is one that you could make. Doing so costs half the XP, half the raw materials, and half the time it would take to enhance that item in the first place.}{}{}

\Feat[Item Creation]{Craft Psionic Construct}
{You can create golems and other psionic automatons that obey your orders.}
{\feat{Craft Psionic Arms and Armor}, \feat{Craft Universal Item}.}
{You can create any psionic construct whose prerequisites you meet. Creating a construct takes one day for each 1,000 gp in its base price. To create a construct, you must spend 1/25 of the construct's base price in XP and use up raw materials costing one-half of this price. A newly created construct has average hit points for its Hit Dice.}{}{}

\GFeat[Item Creation]{Craft Rod}
{Caster level 9th.}
{You can create any rod whose prerequisites you meet. Crafting a rod takes one day for each 1,000 gp in its base price. To craft a rod, you must spend 1/25 of its base price in XP and use up raw materials costing one-half of its base price.

Some rods incur extra costs in material components or XP, as noted in their descriptions. These costs are in addition to those derived from the rod's base price.}

\GFeat[Item Creation]{Craft Staff}
{Caster level 12th.}
{You can create any staff whose prerequisites you meet.

Crafting a staff takes one day for each 1,000 gp in its base price. To craft a staff, you must spend 1/25 of its base price in XP and use up raw materials costing one-half of its base price. A newly created staff has 50 charges.

Some staffs incur extra costs in material components or XP, as noted in their descriptions. These costs are in addition to those derived from the staff's base price.}

\Feat[Item Creation]{Craft Universal Item}
{You can create universal psionic items.}
{Manifester level 3rd.}
{You can create any universal psionic item whose prerequisites you meet. Crafting a universal psionic item takes one day for each 1,000 gp in its base price. To craft a universal psionic item, you must spend 1/25 of the item's base price in XP and use up raw materials costing one-half of this price.

You can also mend a broken universal item if it is one that you could make. Doing so costs half the XP, half the raw materials, and half the time it would take to craft that item in the first place.

Some universal items incur extra costs in XP, as noted in their descriptions. These costs are in addition to those derived from the item's base price. You must pay such a cost to create an item or to mend a broken one.}{}{}

\GFeat[Item Creation]{Craft Wand}
{Caster level 5th.}
{You can create a wand of any 4th-level or lower spell that you know. Crafting a wand takes one day for each 1,000 gp in its base price. The base price of a wand is its caster level $\times$ the spell level $\times$ 750 gp. To craft a wand, you must spend 1/25 of this base price in XP and use up raw materials costing one-half of this base price. A newly created wand has 50 charges.

Any wand that stores a spell with a costly material component or an XP cost also carries a commensurate cost. In addition to the cost derived from the base price, you must expend fifty copies of the material component or pay fifty times the XP cost.}

\GFeat[Item Creation]{Craft Wondrous Item}
{Caster level 3rd.}
{You can create any wondrous item whose prerequisites you meet. Enchanting a wondrous item takes one day for each 1,000 gp in its price. To enchant a wondrous item, you must spend 1/25 of the item's price in XP and use up raw materials costing half of this price.

You can also mend a broken wondrous item if it is one that you could make. Doing so costs half the XP, half the raw materials, and half the time it would take to craft that item in the first place.

Some wondrous items incur extra costs in material components or XP, as noted in their descriptions. These costs are in addition to those derived from the item's base price. You must pay such a cost to create an item or to mend a broken one.}

\GFeat[Item Creation]{Forge Ring}
{Caster level 12th.}
{You can create any ring whose prerequisites you meet. Crafting a ring takes one day for each 1,000 gp in its base price. To craft a ring, you must spend 1/25 of its base price in XP and use up raw materials costing one-half of its base price.

You can also mend a broken ring if it is one that you could make. Doing so costs half the XP, half the raw materials, and half the time it would take to forge that ring in the first place.

Some magic rings incur extra costs in material components or XP, as noted in their descriptions. You must pay such a cost to forge such a ring or to mend a broken one.}

\Feat[Item Creation]{Imprint Stone}
{You can create power stones to store psionic powers.}
{Manifester level 1st.}
{You can create a power stone of any power that you know. Encoding a power stone takes one day for each 1,000 gp in its base price. The base price of a power stone is the level of the stored power $\times$ its manifester level $\times$ 25 gp. To imprint a power stone, you must spend 1/25 of this base price in XP and use up raw materials costing one-half of this base price.

Any power stone that stores a power with an XP cost also carries a commensurate cost. In addition to the costs derived from the base price, you must pay the XP when encoding the stone.}{}{}

\Feat[Item Creation]{Scribe Scroll}
{}
{Caster level 1st.}
{You can create a scroll of any spell that you know. Scribing a scroll takes one day for each 1,000 gp in its base price. The base price of a scroll is its spell level $\times$ its caster level $\times$ 25 gp. To scribe a scroll, you must spend 1/25 of this base price in XP and use up raw materials costing one-half of this base price.

Any scroll that stores a spell with a costly material component or an XP cost also carries a commensurate cost. In addition to the costs derived from the base price, you must expend the material component or pay the XP when scribing the scroll.}
{}
{On Athas, scrolls take many different forms. Common forms include paper or papyrus sheets, clay tablets, and woven cloth.}

\Feat[Item Creation]{Scribe Tattoo}
{You can create psionic tattoos, which store powers within their designs.}
{Manifester level 3rd.}
{You can create a psionic tattoo of any power of 3rd level or lower that you know and that targets one or more creatures. Scribing a psionic tattoo takes one day. When you create a psionic tattoo, you set the manifester level. The manifester level must be sufficient to manifest the power in question and no higher than your own level. The base price of a psionic tattoo is its power level $\times$ its manifester level $\times$ 50 gp. To scribe a tattoo, you must spend 1/25 of this base price in XP and use up raw materials (special inks, masterwork needles, and so on) costing one-half of this base price.

When you create a psionic tattoo, you make any choices that you would normally make when manifesting the power.

When its wearer physically activates the tattoo, the wearer is the target of the power.

Any psionic tattoo that stores a power with an XP cost also carries a commensurate cost. In addition to the costs derived from the base price, you must pay the XP when creating the tattoo.}{}{}

