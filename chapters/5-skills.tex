\Chapter{Skills}
{You can learn much from observing another being. The way the gith hunches before it leaps at you, or how the aarakocra circles before it dives. The way the halfling inhales and pauses briefly before shooting his poisoned needles, or how the Urikite trader licks his lips before making his final offer. But appearances can deceive. No two creatures are alike. Remember that when the gith hunches before casting a defiler spell, or the Urikite trader moistens his lips and spits a needle at you.}{The Oracle, Blue Shrine Scrolls}
\section{Skill Summary}
\Capitalize{I}{f} you buy a class skill, your character gets 1 rank (equal to a +1 bonus on checks with that skill) for each skill point. If you buy other classes' skills (cross-class skills), you get \onehalf rank per skill point.

Your maximum rank in a class skill is your character level + 3.

Your maximum rank in a cross-class skill is one-half of this number (do not round up or down).

\subsection{Using Skills}
To make a skill check, roll:

{
	\centering
	\vskip1em
	{\Large 1d20 + \textit{skill modifier}}
	\vskip.3em
	\Cell{\small\textit{Skill modifier} = skill rank + ability modifier\\\small + miscellaneous modifiers.}
	\vskip1em
}

This roll works just like an attack roll or a saving throw---the higher the roll, the better. Either you're trying to match or exceed a certain Difficulty Class (DC), or you're trying to beat another character's check result.

\textbf{Skill Ranks}: A character's number of ranks in a skill is based on how many skill points a character has invested in a skill. Many skills can be used even if the character has no ranks in them; doing this is called making an untrained skill check.

\textbf{Ability Modifier}: The ability modifier used in a skill check is the modifier for the skill's key ability (the ability associated with the skill's use). The key ability of each skill is noted in its description.

\textbf{Miscellaneous Modifiers}: Miscellaneous modifiers include racial bonuses, armor check penalties, and bonuses provided by feats, among others.

\subsection{Acquiring Skill Ranks}
Each skill point you spend on a class skill gets you 1 rank in that skill. Class skills are the skills found on your character's class skill list. Each skill point you spend on a cross-class skill gets your character \onehalf rank in that skill. Cross-class skills are skills not found on your character's class skill list. (Half ranks do not improve your skill check, but two \onehalf ranks make 1 rank.) You can't save skill points to spend later.

The maximum rank in a class skill is the character's level + 3. If it's a cross-class skill, the maximum rank is half of that number (do not round up or down).

Regardless of whether a skill is purchased as a class skill or a cross-class skill, if it is a class skill for any of your classes, your maximum rank equals your total character level + 3.

\section{Using Skills}
When your character uses a skill, you make a skill check to see how well he or she does. The higher the result of the skill check, the better. Based on the circumstances, your result must match or beat a particular number (a DC or the result of an opposed skill check) for the check to be successful. The harder the task, the higher the number you need to roll.

Circumstances can affect your check. A character who is free to work without distractions can make a careful attempt and avoid simple mistakes. A character who has lots of time can try over and over again, thereby assuring the best outcome. If others help, the character may succeed where otherwise he or she would fail.

\subsection{Skill Checks}
A skill check takes into account a character's training (skill rank), natural talent (ability modifier), and luck (the die roll). It may also take into account his or her race's knack for doing certain things (racial bonus) or what armor he or she is wearing (armor check penalty), or a certain feat the character possesses, among other things.

To make a skill check, roll 1d20 and add your character's skill modifier for that skill. The skill modifier incorporates the character's ranks in that skill and the ability modifier for that skill's key ability, plus any other miscellaneous modifiers that may apply, including racial bonuses and armor check penalties. The higher the result, the better. Unlike with attack rolls and saving throws, a natural roll of 20 on the d20 is not an automatic success, and a natural roll of 1 is not an automatic failure.

\subsubsection{Difficulty Class}
Some checks are made against a Difficulty Class (DC). The DC is a number (set using the skill rules as a guideline) that you must score as a result on your skill check in order to succeed.

\Table{Difficulty Class Examples}{l X}{
\tableheader Difficulty (DC) & \tableheader Example (Skill Used)\\
Very easy (0) & Notice something large in plain sight (Spot)\\
Easy (5) & Climb a knotted rope (Climb)\\
Average (10) & Hear an approaching guard (Listen)\\
Tough (15) & Rig a wagon wheel to fall off (Disable Device)\\
Challenging (20) & Swim in stormy water (Swim)\\
Formidable (25) & Open an average lock (Open Lock)\\
Heroic (30) & Leap across a 30-foot chasm (Jump)\\
Nearly impossible (40) & Track a squad of orcs across hard ground after 24 hours of rainfall (Survival)
}

\subsubsection{Opposed Checks}
An opposed check is a check whose success or failure is determined by comparing the check result to another character's check result. In an opposed check, the higher result succeeds, while the lower result fails. In case of a tie, the higher skill modifier wins. If these scores are the same, roll again to break the tie.

\Table{Example Opposed Checks}{X b{2.67cm} b{2.5cm}}{
\tableheader Task & \tableheader Skill (Key Ability) & \tableheader Opposing Skill (Key Ability)\\
Con someone & Bluff (Cha) & Sense Motive (Wis)\\
Pretend to be someone else & Disguise (Cha) & Spot (Wis)\\
Create a false map & Forgery (Int) & Forgery (Int)\\
Hide from someone & Hide (Dex) & Spot (Wis)\\
Sneak up on someone & Move Silently (Dex) & Listen (Wis)\\
Steal a coin pouch & Sleight of Hand (Dex) & Spot (Wis)\\
Tie a prisoner securely & Use Rope (Dex) & Escape Artist (Dex)
}

\subsubsection{Trying Again}
In general, you can try a skill check again if you fail, and you can keep trying indefinitely. Some skills, however, have consequences of failure that must be taken into account. A few skills are virtually useless once a check has failed on an attempt to accomplish a particular task. For most skills, when a character has succeeded once at a given task, additional successes are meaningless.

\subsubsection{Untrained Skill Checks}
Generally, if your character attempts to use a skill he or she does not possess, you make a skill check as normal. The skill modifier doesn't have a skill rank added in because the character has no ranks in the skill. Any other applicable modifiers, such as the modifier for the skill's key ability, are applied to the check.

Many skills can be used only by someone who is trained in them.

\subsubsection{Favorable And Unfavorable Conditions}
Some situations may make a skill easier or harder to use, resulting in a bonus or penalty to the skill modifier for a skill check or a change to the DC of the skill check.

The chance of success can be altered in four ways to take into account exceptional circumstances.
\begin{enumerate*}
\item Give the skill user a +2 circumstance bonus to represent conditions that improve performance, such as having the perfect tool for the job, getting help from another character (see Combining Skill Attempts), or possessing unusually accurate information.
\item Give the skill user a $-2$ circumstance penalty to represent conditions that hamper performance, such as being forced to use improvised tools or having misleading information.
\item Reduce the DC by 2 to represent circumstances that make the task easier, such as having a friendly audience or doing work that can be subpar.
\item Increase the DC by 2 to represent circumstances that make the task harder, such as having an uncooperative audience or doing work that must be flawless.
\end{enumerate*}
Conditions that affect your character's ability to perform the skill change the skill modifier. Conditions that modify how well the character has to perform the skill to succeed change the DC. A bonus to the skill modifier and a reduction in the check's DC have the same result: They create a better chance of success. But they represent different circumstances, and sometimes that difference is important.

\subsubsection{Time And Skill Checks}
Using a skill might take a round, take no time, or take several rounds or even longer. Most skill uses are standard actions, move actions, or full-round actions. Types of actions define how long activities take to perform within the framework of a combat round (6 seconds) and how movement is treated with respect to the activity. Some skill checks are instant and represent reactions to an event, or are included as part of an action.

These skill checks are not actions. Other skill checks represent part of movement.

\subsubsection{Checks Without Rolls}
A skill check represents an attempt to accomplish some goal, usually while under some sort of time pressure or distraction. Sometimes, though, a character can use a skill under more favorable conditions and eliminate the luck factor.

\textbf{Taking 10}: When your character is not being threatened or distracted, you may choose to take 10. Instead of rolling 1d20 for the skill check, calculate your result as if you had rolled a 10. For many routine tasks, taking 10 makes them automatically successful. Distractions or threats (such as combat) make it impossible for a character to take 10. In most cases, taking 10 is purely a safety measure---you know (or expect) that an average roll will succeed but fear that a poor roll might fail, so you elect to settle for the average roll (a 10). Taking 10 is especially useful in situations where a particularly high roll wouldn't help.

\textbf{Taking 20:} When you have plenty of time (generally 2 minutes for a skill that can normally be checked in 1 round, one full-round action, or one standard action), you are faced with no threats or distractions, and the skill being attempted carries no penalties for failure, you can take 20. In other words, eventually you will get a 20 on 1d20 if you roll enough times. Instead of rolling 1d20 for the skill check, just calculate your result as if you had rolled a 20.

Taking 20 means you are trying until you get it right, and it assumes that you fail many times before succeeding. Taking 20 takes twenty times as long as making a single check would take.

Since taking 20 assumes that the character will fail many times before succeeding, if you did attempt to take 20 on a skill that carries penalties for failure, your character would automatically incur those penalties before he or she could complete the task. Common ``take 20'' skills include Escape Artist, Open Lock, and Search.

\textbf{Ability Checks and Caster Level Checks:} The normal take 10 and take 20 rules apply for ability checks. Neither rule applies to caster level checks.

\subsection{Combining Skill Attempts}
When more than one character tries the same skill at the same time and for the same purpose, their efforts may overlap.

\subsubsection{Individual Events}
Often, several characters attempt some action and each succeeds or fails independently. The result of one character's Climb check does not influence the results of other characters Climb check.

\subsubsection{Aid Another}
You can help another character achieve success on his or her skill check by making the same kind of skill check in a cooperative effort. If you roll a 10 or higher on your check, the character you are helping gets a +2 bonus to his or her check, as per the rule for favorable conditions. (You can't take 10 on a skill check to aid another.) In many cases, a character's help won't be beneficial, or only a limited number of characters can help at once.

In cases where the skill restricts who can achieve certain results you can't aid another to grant a bonus to a task that your character couldn't achieve alone.

\subsubsection{Skill Synergy}
It's possible for a character to have two skills that work well together. In general, having 5 or more ranks in one skill gives the character a +2 bonus on skill checks with each of its synergistic skills, as noted in the skill description. In some cases, this bonus applies only to specific uses of the skill in question, and not to all checks. Some skills provide benefits on other checks made by a character, such as those checks required to use certain class features.

\Table{Skill Synergies}{L p{4.9cm}}{
\tableheader 5 or more ranks in... & \tableheader Gives a +2 bonus on...\\
Autohypnosis & Knowledge (psionics) checks\\
Bluff & Diplomacy checks\\
Bluff & Disguise checks to act in character\\
Bluff & Intimidate checks\\
Bluff & Sleight Of Hand checks\\
Concentration & Autohypnosis checks\\
Craft & related Appraise checks\\
Decipher Script & Use Magic Device checks involving scrolls\\
Escape Artist & Use Rope checks involving bindings\\
Handle Animal & Ride checks\\
Handle Animal & wild empathy checks\\
Jump & Tumble checks\\
Knowledge (arcana) & Spellcraft checks\\
~ (architecture and engineering) & Search checks involving secret doors and similar compartments\\
~ (dungeoneering) & Survival checks when underground\\
~ (geography) & Survival checks to keep from getting lost or for avoiding hazards\\
~ (history) & bardic knowledge checks\\
~ (local) & Gather Information checks\\
~ (nature) & Survival checks in aboveground natural environments\\
~ (nobility and royalty) & Diplomacy checks\\
~ (psionics) & Psicraft\\
~ (religion) & checks to turn or rebuke undead\\
~ (the planes) & Survival checks when on other planes\\
Psicraft & Use Psionic Device checks involving power stones\\
Search & Survival checks when following tracks\\
Sense Motive & Diplomacy checks\\
Spellcraft & Use Magic Device involving scrolls\\
Survival & Knowledge (nature) checks\\
Tumble & Balance checks\\
Tumble & Jump checks\\
Use Magic Device & Spellcraft checks to decipher scrolls\\
Use Psionic Device & Psicraft checks to address power stones\\
Use Rope & Climb checks involving climbing ropes\\
Use Rope & Escape Artist checks involving ropes
}

\subsection{Ability Checks}
Sometimes a character tries to do something to which no specific skill really applies. In these cases, you make an ability check. An ability check is a roll of 1d20 plus the appropriate ability modifier. Essentially, you're making an untrained skill check.

In some cases, an action is a straight test of one's ability with no luck involved. Just as you wouldn't make a height check to see who is taller, you don't make a Strength check to see who is stronger.
\section{Skill Descriptions}
This section describes each skill, including common uses and typical modifiers. Characters can sometimes use skills for purposes other than those noted here.

Here is the format for skill descriptions.

\subsection{Skill Name}
The skill name line includes (in addition to the name of the skill) the following information.

\textbf{Key Ability:} The abbreviation of the ability whose modifier applies to the skill check. Exception: Speak Language and Literacy have ``None'' as its key ability because the use of those skills does not require a check.

\textbf{Trained Only:} If this notation is included in the skill name line, you must have at least 1 rank in the skill to use it. If it is omitted, the skill can be used untrained (with a rank of 0). If any special notes apply to trained or untrained use, they are covered in the Untrained section (see below).

\textbf{Armor Check Penalty:} If this notation is included in the skill name line, an armor check penalty applies (when appropriate) to checks using this skill. If this entry is absent, an armor check penalty does not apply.

The skill name line is followed by a general description of what using the skill represents. After the description are a few other types of information:

\textbf{Check:} What a character (``you'' in the skill description) can do with a successful skill check and the check’s DC.

\textbf{Action:} The type of action using the skill requires, or the amount of time required for a check.

\textbf{Try Again:} Any conditions that apply to successive attempts to use the skill successfully. If the skill doesn’t allow you to attempt the same task more than once, or if failure carries an inherent penalty (such as with the Climb skill), you can’t take 20. If this paragraph is omitted, the skill can be retried without any inherent penalty, other than the additional time required.

\textbf{Special:} Any extra facts that apply to the skill, such as special effects deriving from its use or bonuses that certain characters receive because of class, feat choices, or race.

\textbf{Synergy:} Some skills grant a bonus to the use of one or more other skills because of a synergistic effect. This entry, when present, indicates what bonuses this skill may grant or receive because of such synergies. See Table: Skill Synergies for a complete list of bonuses granted by synergy between skills (or between a skill and a class feature).

\textbf{Restriction:} The full utility of certain skills is restricted to characters of certain classes or characters who possess certain feats. This entry indicates whether any such restrictions exist for the skill.

\textbf{Untrained:} This entry indicates what a character without at least 1 rank in the skill can do with it. If this entry doesn’t appear, it means that the skill functions normally for untrained characters (if it can be used untrained) or that an untrained character can’t attempt checks with this skill (for skills that are designated as ``Trained Only'').

\Skill{Appraise}{Int}
\textbf{Check:} You can appraise common or well-known objects with a DC 12 Appraise check. Failure means that you estimate the value at 50\% to 150\% (2d6+3 times 10\%,) of its actual value.

Appraising a rare or exotic item requires a successful check against DC 15, 20, or higher. If the check is successful, you estimate the value correctly; failure means you cannot estimate the item’s value.

A magnifying glass gives you a +2 circumstance bonus on Appraise checks involving any item that is small or highly detailed, such as a gem. A merchant’s scale gives you a +2 circumstance bonus on Appraise checks involving any items that are valued by weight, including anything made of precious metals.

These bonuses stack.

\textbf{Action:} Appraising an item takes 1 minute (ten consecutive full-round actions).

\textbf{Try Again:} No. You cannot try again on the same object, regardless of success.

\textbf{Special:} The master of a raven familiar gains a +3 bonus on Appraise checks.

A character with the Diligent feat gets a +2 bonus on Appraise checks.

\textbf{Synergy:} If you have 5 ranks in any Craft skill, you gain a +2 bonus on Appraise checks related to items made with that Craft skill.

\textbf{Untrained:} For common items, failure on an untrained check means no estimate. For rare items, success means an estimate of 50\% to 150\% (2d6+3 times 10\%).
\Skill{Autohypnosis}{Wis; Trained Only}
You have trained your mind to gain mastery over your body and the mind’s own deepest capabilities.

\textbf{Check:} The DC and the effect of a successful check depend on the task you attempt.

\Table{}{X R}{
\tableheader Task & \tableheader DC\\
Ignore caltrop wound & 18\\
Memorize & 15\\
Resist dying & 20\\
Resist fear & Fear effect DC\\
Tolerate poison & Poison’s DC\\
Willpower & 20
}

\textit{Ignore Caltrop Wound:} If you are wounded by stepping on a caltrop, your speed is reduced to one-half normal. A successful Autohypnosis check removes this movement penalty. The wound doesn’t go away---it is just ignored through self-persuasion.

\textit{Memorize:} You can attempt to memorize a long string of numbers, a long passage of verse, or some other particularly difficult piece of information (but you can’t memorize magical writing or similarly exotic scripts). Each successful check allows you to memorize a single page of text (up to 800 words), numbers, diagrams, or sigils (even if you don’t recognize their meaning). If a document is longer than one page, you can make additional checks for each additional page. You always retain this information; however, you can recall it only with another successful Autohypnosis check.

\textit{Resist Dying:} You can attempt to subconsciously prevent yourself from dying. If you have negative hit points and are losing hit points (at 1 per round, 1 per hour), you can substitute a DC 20 Autohypnosis check for your d\% roll to see if you become stable. If the check is successful, you stop losing hit points (you do not gain any hit points, however, as a result of the check). You can substitute this check for the d\% roll in later rounds if you are initially unsuccessful.

\textit{Resist Fear:} In response to any fear effect, you make a saving throw normally. If you fail the saving throw, you can make an Autohypnosis check on your next round even while overcome by fear. If your Autohypnosis check meets or beats the DC for the fear effect, you shrug off the fear. On a failed check, the fear affects you normally, and you gain no further attempts to shrug off that particular fear effect.

\textit{Tolerate Poison:} You can choose to substitute an Autohypnosis check for a saving throw against any standard poison’s secondary damage or effect. This skill has no effect on the initial saving throw against poison.

\textit{Willpower:} If reduced to 0 hit points (disabled), you can make an Autohypnosis check. If successful, you can take a normal action while at 0 hit points without taking 1 point of damage. You must make a check for each strenuous action you want to take. A failed Autohypnosis check in this circumstance carries no direct penalty---you can choose not to take the strenuous action and thus avoid the hit point loss. If you do so anyway, you drop to $-1$ hit points, as normal when disabled.

\textbf{Action:} None. Making an Autohypnosis check doesn’t require an action; it is either a free action (when attempted reactively) or part of another action (when attempted actively).

\textbf{Try Again:} Yes, for memorize and willpower uses, though a success doesn’t cancel the effects of a previous failure. No for the other uses.

\textbf{Special:} If you have the Autonomous feat, you get a +2 bonus on Autohypnosis checks.
\Skill{Balance}{Dex; Armor Check Penalty}
\textbf{Check:} You can walk on a precarious surface. A successful check lets you move at half your speed along the surface for 1 round. A failure by 4 or less means you can’t move for 1 round. A failure by 5 or more means you fall. The difficulty varies with the surface, as follows:

\Table{Balance DCs}{X Z{1cm} X Z{1cm}}{
\tableheader Narrow Surface & \tableheader Balance DC & \tableheader Difficult Surface & \tableheader Balance DC\\
% ¹ Add modifiers from Narrow Surface Modifiers, below, as appropriate.
% ² Only if running or charging. Failure by 4 or less means the character can’t run or charge, but may otherwise act normally.
7-12 inches wide & 10 & Uneven flagstone & 10\\
2-6 inches wide & 15 & Hewn stone floor & 10\\
Less than 2 inches wide & 20 & Sloped or angled floor & 10
}

\Table{Narrow Surface Modifiers}{X R}{
\tableheader Surface & \tableheader DC Modifier \\
% ¹ Add the appropriate modifier to the Balance DC of a narrow surface.

% These modifiers stack.

Lightly obstructed & +2\\
Severely obstructed & +5\\
Lightly slippery & +2\\
Severely slippery & +5\\
Sloped or angled & +2
}

For narrow surfaces, use the DC given on \hyperref[tab:Balance DCs]{Table: Balance DCs} and add modifiers from \hyperref[tab:Narrow Surface Modifiers]{Table: Narrow Surface Modifiers}, as appropriate. Those modifiers stack.

Make checks for difficult surfaces only if running or charging. Failure by 4 or less means the character can’t run or charge, but may otherwise act normally.

\textit{Being Attacked while Balancing:} You are considered flat-footed while balancing, since you can’t move to avoid a blow, and thus you lose your Dexterity bonus to AC (if any). If you have 5 or more ranks in Balance, you aren’t considered flat-footed while balancing. If you take damage while balancing, you must make another Balance check against the same DC to remain standing.

\textit{Accelerated Movement:} You can try to walk across a precarious surface more quickly than normal. If you accept a $-5$ penalty, you can move your full speed as a move action. (Moving twice your speed in a round requires two Balance checks, one for each move action used.) You may also accept this penalty in order to charge across a precarious surface; charging requires one Balance check for each multiple of your speed (or fraction thereof) that you charge.

\textbf{Action:} None. A Balance check doesn’t require an action; it is made as part of another action or as a reaction to a situation.

\textbf{Special:} If you have the Agile feat, you get a +2 bonus on Balance checks.

\textbf{Synergy:} If you have 5 or more ranks in Tumble, you get a +2 bonus on Balance checks.
\Skill{Bluff}{Cha}
\textbf{Check:} A Bluff check is opposed by the target’s Sense Motive check. See the accompanying table for examples of different kinds of bluffs and the modifier to the target’s Sense Motive check for each one.

Favorable and unfavorable circumstances weigh heavily on the outcome of a bluff. Two circumstances can weigh against you: The bluff is hard to believe, or the action that the target is asked to take goes against its self-interest, nature, personality, orders, or the like. If it’s important, you can distinguish between a bluff that fails because the target doesn’t believe it and one that fails because it just asks too much of the target. For instance, if the target gets a +10 bonus on its Sense Motive check because the bluff demands something risky, and the Sense Motive check succeeds by 10 or less, then the target didn’t so much see through the bluff as prove reluctant to go along with it. A target that succeeds by 11 or more has seen through the bluff.

A successful Bluff check indicates that the target reacts as you wish, at least for a short time (usually 1 round or less) or believes something that you want it to believe. Bluff, however, is not a suggestion spell.

A bluff requires interaction between you and the target. Creatures unaware of you cannot be bluffed.

\Table{Bluff Examples}{X r{1.5cm}}{
\tableheader Example Circumstances & \tableheader Sense Motive Modifier\\
The target wants to believe you. & -5\\
The bluff is believable and doesn’t affect the target much. & +0\\
The bluff is a little hard to believe or puts the target at some risk. & +5\\
The bluff is hard to believe or puts the target at significant risk. & +10\\
The bluff is way out there, almost too incredible to consider. & +20
}

\textit{Feinting in Combat:} You can also use Bluff to mislead an opponent in melee combat (so that it can’t dodge your next attack effectively). To feint, make a Bluff check opposed by your target’s Sense Motive check, but in this case, the target may add its base attack bonus to the roll along with any other applicable modifiers.

If your Bluff check result exceeds this special Sense Motive check result, your target is denied its Dexterity bonus to AC (if any) for the next melee attack you make against it. This attack must be made on or before your next turn.

Feinting in this way against a nonhumanoid is difficult because it’s harder to read a strange creature’s body language; you take a -4 penalty on your Bluff check. Against a creature of animal Intelligence (1 or 2) it’s even harder; you take a -8 penalty. Against a nonintelligent creature, it’s impossible.

Feinting in combat does not provoke an attack of opportunity.

\textit{Creating a Diversion to Hide:} You can use the Bluff skill to help you hide. A successful Bluff check gives you the momentary diversion you need to attempt a Hide check while people are aware of you. This usage does not provoke an attack of opportunity.

\textit{Delivering a Secret Message:} You can use Bluff to get a message across to another character without others understanding it. The DC is 15 for simple messages, or 20 for complex messages, especially those that rely on getting across new information. Failure by 4 or less means you can’t get the message across. Failure by 5 or more means that some false information has been implied or inferred. Anyone listening to the exchange can make a Sense Motive check opposed by the Bluff check you made to transmit in order to intercept your message (see Sense Motive).

\textit{Conceal Spellcasting:} Spellcasters may attempt to conceal the fact that they are attempting to cast a spell. This is an especially important skill for wizards, who are all-too-frequently the unfortunate target of impromptu lynch mobs. When casting a spell, a spellcaster may attempt to conceal verbal and somatic components by making a Bluff check as a move action, to distract any witnesses. Onlookers may oppose the roll with a Sense Motive or Spellcraft check.

Attempting to conceal spellcasting if the spellcaster is defiling carries a $-20$ circumstance penalty.
% Bluff
% The spellcaster is defiling & $-20$
% Spellcraft or Sense Motive

\Table{}{X r{2cm}}{
\tableheader Circumstances & \tableheader Spellcraft/Sense Motive Modifier\\
Target is closely observing spellcaster & +5\\
Target knows the character is a spellcaster & +5
}

Casting spells in cities with witnesses can be very dangerous. Lynch mobs, templars and even other wizards generally flock to the scene when someone cries \textit{``Wizard!''}.

\textbf{Action:} Varies. A Bluff check made as part of general interaction always takes at least 1 round (and is at least a full-round action), but it can take much longer if you try something elaborate. A Bluff check made to feint in combat or create a diversion to hide is a standard action. A Bluff check made to conceal spellcasting is a move action. A Bluff check made to deliver a secret message doesn’t take an action; it is part of normal communication.

\textbf{Try Again:} Varies. Generally, a failed Bluff check in social interaction makes the target too suspicious for you to try again in the same circumstances, but you may retry freely on Bluff checks made to feint in combat. Retries are also allowed when you are trying to send a message, but you may attempt such a retry only once per round.

Each retry carries the same chance of miscommunication.

\textbf{Special:} A ranger gains a bonus on Bluff checks when using this skill against a favored enemy.

The master of a snake familiar gains a +3 bonus on Bluff checks.

If you have the Persuasive feat, you get a +2 bonus on Bluff checks.

\textbf{Synergy:} If you have 5 or more ranks in Bluff, you get a +2 bonus on Diplomacy, Intimidate, and Sleight of Hand checks, as well as on Disguise checks made when you know you’re being observed and you try to act in character.


\Skill{Climb}{Str; Armor Check Penalty}
\textbf{Check:} With a successful Climb check, you can advance up, down, or across a slope, a wall, or some other steep incline (or even a ceiling with handholds) at one-quarter your normal speed. A slope is considered to be any incline at an angle measuring less than 60 degrees; a wall is any incline at an angle measuring 60 degrees or more.

A Climb check that fails by 4 or less means that you make no progress, and one that fails by 5 or more means that you fall from whatever height you have already attained.

A climber’s kit gives you a +2 circumstance bonus on Climb checks.

The DC of the check depends on the conditions of the climb. Compare the task with those on the following table to determine an appropriate DC.

\Table{}{l X}{
\tableheader Climb DC & \tableheader Example Surface or Activity\\
0 & A slope too steep to walk up, or a knotted rope with a wall to brace against.\\
5 & A rope with a wall to brace against, or a knotted rope, or a rope affected by the rope trick spell.\\
10 & A surface with ledges to hold on to and stand on, such as a very rough wall or a ship’s rigging.\\
15 & Any surface with adequate handholds and footholds (natural or artificial), such as a very rough natural rock surface or a tree, or an unknotted rope, or pulling yourself up when dangling by your hands.\\
20 & An uneven surface with some narrow handholds and footholds, such as a typical wall in a dungeon or ruins.\\
25 & A rough surface, such as a natural rock wall or a brick wall.\\
25 & An overhang or ceiling with handholds but no footholds.\\
--- & A perfectly smooth, flat, vertical surface cannot be climbed.
}

\Table{}{l X}{
\tableheader Climb DC Modifier & Example Surface or Activity\\
% These modifiers are cumulative; use any that apply.
-10 & Climbing a chimney (artificial or natural) or other location where you can brace against two opposite walls (reduces DC by 10).\\
-5 & Climbing a corner where you can brace against perpendicular walls (reduces DC by 5).\\
+5 & Surface is slippery (increases DC by 5).
}

You need both hands free to climb, but you may cling to a wall with one hand while you cast a spell or take some other action that requires only one hand. While climbing, you can’t move to avoid a blow, so you lose your Dexterity bonus to AC (if any). You also can’t use a shield while climbing.

Any time you take damage while climbing, make a Climb check against the DC of the slope or wall. Failure means you fall from your current height and sustain the appropriate falling damage.

\textit{Accelerated Climbing:} You try to climb more quickly than normal. By accepting a -5 penalty, you can move half your speed (instead of one-quarter your speed).

\textit{Making Your Own Handholds and Footholds:} You can make your own handholds and footholds by pounding pitons into a wall. Doing so takes 1 minute per piton, and one piton is needed per 3 feet of distance. As with any surface that offers handholds and footholds, a wall with pitons in it has a DC of 15. In the same way, a climber with a handaxe or similar implement can cut handholds in an ice wall.

\textit{Catching Yourself When Falling:} It’s practically impossible to catch yourself on a wall while falling. Make a Climb check (DC = wall’s DC + 20) to do so. It’s much easier to catch yourself on a slope (DC = slope’s DC + 10).

\textit{Catching a Falling Character While Climbing:} If someone climbing above you or adjacent to you falls, you can attempt to catch the falling character if he or she is within your reach. Doing so requires a successful melee touch attack against the falling character (though he or she can voluntarily forego any Dexterity bonus to AC if desired). If you hit, you must immediately attempt a Climb check (DC = wall’s DC + 10). Success indicates that you catch the falling character, but his or her total weight, including equipment, cannot exceed your heavy load limit or you automatically fall. If you fail your Climb check by 4 or less, you fail to stop the character’s fall but don’t lose your grip on the wall. If you fail by 5 or more, you fail to stop the character’s fall and begin falling as well.

\textbf{Action:} Climbing is part of movement, so it’s generally part of a move action (and may be combined with other types of movement in a move action). Each move action that includes any climbing requires a separate Climb check. Catching yourself or another falling character doesn’t take an action.

\textbf{Special:} You can use a rope to haul a character upward (or lower a character) through sheer strength. You can lift double your maximum load in this manner.

A halfling has a +2 racial bonus on Climb checks because halflings are agile and surefooted.

The master of a lizard familiar gains a +3 bonus on Climb checks.

If you have the Athletic feat, you get a +2 bonus on Climb checks.

A creature with a climb speed has a +8 racial bonus on all Climb checks. The creature must make a Climb check to climb any wall or slope with a DC higher than 0, but it always can choose to take 10, even if rushed or threatened while climbing. If a creature with a climb speed chooses an accelerated climb (see above), it moves at double its climb speed (or at its land speed, whichever is slower) and makes a single Climb check at a -5 penalty. Such a creature retains its Dexterity bonus to Armor Class (if any) while climbing, and opponents get no special bonus to their attacks against it. It cannot, however, use the run action while climbing.

\textbf{Synergy:} If you have 5 or more ranks in Use Rope, you get a +2 bonus on Climb checks made to climb a rope, a knotted rope, or a rope-and-wall combination.
\Skill{Concentration}{Con}
\textbf{Check:} You must make a Concentration check whenever you might potentially be distracted (by taking damage, by harsh weather, and so on) while engaged in some action that requires your full attention. Such actions include casting a spell, concentrating on an active spell, directing a spell, using a spell-like ability, or using a skill that would provoke an attack of opportunity. In general, if an action wouldn’t normally provoke an attack of opportunity, you need not make a Concentration check to avoid being distracted.

If the Concentration check succeeds, you may continue with the action as normal. If the check fails, the action automatically fails and is wasted. If you were in the process of casting a spell, the spell is lost. If you were concentrating on an active spell, the spell ends as if you had ceased concentrating on it. If you were directing a spell, the direction fails but the spell remains active. If you were using a spell-like ability, that use of the ability is lost. A skill use also fails, and in some cases a failed skill check may have other ramifications as well.

The table below summarizes various types of distractions that cause you to make a Concentration check. If the distraction occurs while you are trying to cast a spell, you must add the level of the spell you are trying to cast to the appropriate Concentration DC. If more than one type of distraction is present, make a check for each one; any failed Concentration check indicates that the task is not completed.

If you are trying to cast, concentrate on, or direct a spell when the distraction occurs, add the level of the spell to the indicated DC.

If the distracting spell or power allows no save, use the save DC it would have if it did allow a save.

\Table{}{Y{3cm} >{\raggedright\arraybackslash}X}{
\tableheader Concentration DC & \tableheader Distraction\\
% 
% Such as during the casting of a spell with a casting time of 1 round or more, or the execution of an activity that takes more than a single full-round action (such as Disable Device). Also, damage stemming from an attack of opportunity or readied attack made in response to the spell being cast (for spells with a casting time of 1 standard action) or the action being taken (for activities requiring no more than a full-round action).
% Such as from acid arrow.
% 
10 + damage dealt & Damaged during the action.\\
10 + half of continuous damage last dealt & Taking continuous damage during the action, such as from \emph{acid arrow}.\\
Distracting spell’s save DC & Distracted by nondamaging spell.\\
10 & Vigorous motion (on a moving mount, taking a bouncy wagon ride, in a small boat in rough water, belowdecks in a stormtossed ship).\\
15 & Violent motion (on a galloping horse, taking a very rough wagon ride, in a small boat in rapids, on the deck of a storm-tossed ship).\\
20 & Extraordinarily violent motion (earthquake).\\
15 & Entangled.\\
20 & Grappling or pinned. (You can cast only spells without somatic components for which you have any required material component in hand.)\\
5 & Weather is a high wind carrying blinding rain or sleet.\\
10 & Weather is wind-driven hail, dust, or debris.\\
Distracting spell’s save DC & Weather caused by a spell, such as \emph{storm of vengeance}.\\
% \hline
Distracting power’s save DC & Distracted by nondamaging power.\\
15 + power level & Attempting to manifest a power without its display.\\
20 & Gain psionic focus.\\
20 & Grappling or pinned. (You can manifest powers normally unless you fail your Concentration check.)\\
Distracting power’s save DC & Weather caused by power
}

\textit{Gain Psionic Focus:} Merely holding a reservoir of psionic power points in mind gives psionic characters a special energy. Psionic characters can put that energy to work without actually paying a power point cost---they can become psionically focused as a special use of the Concentration skill.

If you have 1 or more power points available, you can meditate to attempt to become psionically focused. The DC to become psionically focused is 20. Meditating is a full-round action that provokes attacks of opportunity. When you are psionically focused, you can expend your focus on any single Concentration check you make thereafter. When you expend your focus in this manner, your Concentration check is treated as if you rolled a 15. It’s like taking 10, except that the number you add to your Concentration modifier is 15. You can also expend your focus to gain the benefit of a psionic feat---many psionic feats are activated in this way.

Once you are psionically focused, you remain focused until you expend your focus, become unconscious, or go to sleep (or enter a meditative trance, in the case of elans), or until your power point reserve drops to 0.

\textbf{Action:} Usually none. In most cases, making a Concentration check doesn’t require an action; it is either a free action (when attempted reactively) or part of another action (when attempted actively). Meditating to gain psionic focus is a full-round action.

\textbf{Try Again:} Yes, though a success doesn’t cancel the effect of a previous failure, such as the loss of a spell you were casting or the disruption of a spell you were concentrating on.

\textbf{Special:} You can use Concentration to cast a spell, use a spell-like ability, or use a skill defensively, so as to avoid attacks of opportunity altogether. This doesn’t apply to other actions that might provoke attacks of opportunity.

The DC of the check is 15 (plus the spell’s level, if casting a spell or using a spell-like ability defensively). If the Concentration check succeeds, you may attempt the action normally without provoking any attacks of opportunity. A successful Concentration check still doesn’t allow you to take 10 on another check if you are in a stressful situation; you must make the check normally. If the Concentration check fails, the related action also automatically fails (with any appropriate ramifications), and the action is wasted, just as if your concentration had been disrupted by a distraction.

A character with the Combat Casting feat gets a +4 bonus on Concentration checks made to cast a spell or use a spell-like ability while on the defensive or while grappling or pinned.

You can use Concentration to manifest a power or use a psi-like ability defensively, so as to avoid attacks of opportunity altogether. The DC of the check is 15 + the power’s level. If the Concentration check succeeds, you can manifest normally without provoking any attacks of opportunity. If the Concentration check fails, the power also automatically fails and the power points are wasted, just as if your concentration had been disrupted by a distraction.

A character with the Combat Manifestation feat gets a +4 bonus on Concentration checks made to manifest a power or use a psi-like ability while on the defensive or while grappling or pinned.

\textbf{Synergy:} If you have 5 or more ranks in Concentration, you get a +2 bonus on Autohypnosis checks.
\Skill{Craft}{Int}
Like Knowledge, Perform, and Profession, Craft is actually a number of separate skills. You could have several Craft skills, each with its own ranks, each purchased as a separate skill.

A Craft skill is specifically focused on creating something. If nothing is created by the endeavor, it probably falls under the heading of a Profession skill.

\textbf{Check:} You can practice your trade and make a decent living, earning about half your check result in gold pieces per week of dedicated work. You know how to use the tools of your trade, how to perform the craft's daily tasks, how to supervise untrained helpers, and how to handle common problems. (Untrained laborers and assistants earn an average of 1 silver piece per day.)

The basic function of the Craft skill, however, is to allow you to make an item of the appropriate type. The DC depends on the complexity of the item to be created. The DC, your check results, and the price of the item determine how long it takes to make a particular item. The item's finished price also determines the cost of raw materials.

In some cases, the fabricate spell can be used to achieve the results of a Craft check with no actual check involved. However, you must make an appropriate Craft check when using the spell to make articles requiring a high degree of craftsmanship.

A successful Craft check related to woodworking in conjunction with the casting of the ironwood spell enables you to make wooden items that have the strength of steel.

When casting the spell minor creation, you must succeed on an appropriate Craft check to make a complex item.

All crafts require artisan's tools to give the best chance of success. If improvised tools are used, the check is made with a $-2$ circumstance penalty. On the other hand, masterwork artisan's tools provide a +2 circumstance bonus on the check.

To determine how much time and money it takes to make an item, follow these steps.

\begin{enumerate*}
\item Find the item's price. Put the price in bits (1 ceramic = 10 bits).
\item Find the DC from the table below.
\item Pay one-third of the item's price for the cost of raw materials.
\item Make an appropriate Craft check representing one week's work. If the check succeeds, multiply your check result by the DC. If the result $\times$ the DC equals the price of the item in bits, then you have completed the item. (If the result $\times$ the DC equals double or triple the price of the item in bits, then you've completed the task in one-half or one-third of the time. Other multiples of the DC reduce the time in the same manner.) If the result $\times$ the DC doesn't equal the price, then it represents the progress you've made this week. Record the result and make a new Craft check for the next week. Each week, you make more progress until your total reaches the price of the item in bits.
\end{enumerate*}

Creating metal items takes 10 times longer than crafting non-metal items. It also carries a $-5$ penalty on all Craft checks.

If you fail a check by 4 or less, you make no progress this week.

If you fail by 5 or more, you ruin half the raw materials and have to pay half the original raw material cost again.

\textit{Progress by the Day:} You can make checks by the day instead of by the week. In this case your progress (check result $\times$ DC) is in lead beads instead of bits.

\textit{Creating Masterwork Items:} You can make a masterwork item---a weapon, suit of armor, shield, or tool that conveys a bonus on its use through its exceptional craftsmanship, not through being magical. To create a masterwork item, you create the masterwork component as if it were a separate item in addition to the standard item. The masterwork component has its own price (300 gp for a weapon or 150 gp for a suit of armor or a shield) and a Craft DC of 20. Once both the standard component and the masterwork component are completed, the masterwork item is finished. Note: The cost you pay for the masterwork component is one-third of the given amount, just as it is for the cost in raw materials.

\textit{Repairing Items:} Generally, you can repair an item by making checks against the same DC that it took to make the item in the first place. The cost of repairing an item is one-fifth of the item's price.

When you use the Craft skill to make a particular sort of item, the DC for checks involving the creation of that item are typically as given on the following table.

\Table{}{L l Y{1.2cm}}{
\tableheader Item & \tableheader Craft Skill & \tableheader Craft DC\\
% You must be a spellcaster to craft any of these items.
% Traps have their own rules for construction.
Acid & Alchemy & 15\\
Alchemist's fire, smokestick, or tindertwig & Alchemy & 20\\
Antitoxin, sunrod, tanglefoot bag, or thunderstone & Alchemy & 25\\
Armor or shield & Armorsmithing & 10 + AC bonus\\
Longbow or shortbow & Bowmaking & 12\\
Composite longbow or composite shortbow & Bowmaking & 15\\
Composite longbow or composite shortbow with high strength rating & Bowmaking & 15 + (2 $\times$ rating)\\
Crossbow & Weaponsmithing & 15\\
Simple melee or thrown weapon & Weaponsmithing & 12\\
Martial melee or thrown weapon & Weaponsmithing & 15\\
Exotic melee or thrown weapon & Weaponsmithing & 18\\
Mechanical trap & Trapmaking & Varies\\
Very simple item (wooden spoon) & Varies & 5\\
Typical item (iron pot) & Varies & 10\\
High-quality item (bell) & Varies & 15\\
Complex or superior item (lock) & Varies & 20
}

\textbf{Action:} Does not apply. Craft checks are made by the day or week (see above).

\textbf{Try Again:} Yes, but each time you miss by 5 or more, you ruin half the raw materials and have to pay half the original raw material cost again.

\textbf{Special:} You may voluntarily add +10 to the indicated DC to craft an item. This allows you to create the item more quickly (since you'll be multiplying this higher DC by your Craft check result to determine progress). You must decide whether to increase the DC before you make each weekly or daily check.

To make an item using Craft (alchemy), you must have alchemical equipment and be a bard. If you are working in a city, you can buy what you need as part of the raw materials cost to make the item, but alchemical equipment is difficult or impossible to come by in some places. Purchasing and maintaining an alchemist's lab grants a +2 circumstance bonus on Craft (alchemy) checks because you have the perfect tools for the job, but it does not affect the cost of any items made using the skill.

If you have the Metalsmith feat, you don't receive the $-5$ penalty to craft metal items.

\textbf{Synergy:} If you have 5 ranks in a Craft skill, you get a +2 bonus on Appraise checks related to items made with that Craft skill.

If you have 5 ranks in Heal, you get a +2 bonus when attempting to
manufacture antidotes.

If you have 5 ranks in Craft (alchemy), you get a +2 bonus to Craft (poisonmaking).
\Skill{Decipher Script}{Int; Trained Only}
\textbf{Check:} You can decipher writing in an unfamiliar language or a message written in an incomplete or archaic form. The base DC is 20 for the simplest messages, 25 for standard texts, and 30 or higher for intricate, exotic, or very old writing.

If the check succeeds, you understand the general content of a piece of writing about one page long (or the equivalent). If the check fails, make a DC 5 Wisdom check to see if you avoid drawing a false conclusion about the text. (Success means that you do not draw a false conclusion; failure means that you do.)

Both the Decipher Script check and (if necessary) the Wisdom check are made secretly, so that you can’t tell whether the conclusion you draw is true or false.

\textbf{Action:} Deciphering the equivalent of a single page of script takes 1 minute (ten consecutive full-round actions).

\textbf{Try Again:} No.

\textbf{Special:} A character with the Diligent feat gets a +2 bonus on Decipher Script checks.

\textbf{Synergy:} If you have 5 or more ranks in Decipher Script, you get a +2 bonus on Use Magic Device checks involving scrolls.
\Skill{Diplomacy}{Cha}
\textbf{Check:} You can change the attitudes of others (nonplayer characters) with a successful Diplomacy check. In negotiations, participants roll opposed Diplomacy checks, and the winner gains the advantage. Opposed checks also resolve situations when two advocates or diplomats plead opposite cases in a hearing before a third party.

\textit{Influencing NPC Attitudes:} Use the table below to determine the effectiveness of Diplomacy checks (or Charisma checks) made to influence the attitude of a nonplayer character, or wild empathy checks made to influence the attitude of an animal or magical beast.

\Table{}{p{1.2cm} *{4}{C} c}{
\rowcolor{white}
\multirow{2}{1cm}{\tableheader Initial Attitude} & \multicolumn{5}{c}{\tableheader New Attitude (DC to achieve)}\\
\cmidrule[.5pt]{2-6}
& \tableheader Hostile & \tableheader Unfriendly & \tableheader Indifferent & \tableheader Friendly & \tableheader Helpful\\
Hostile & Less than 20 & 20 & 25 & 35 & 50\\
Unfriendly & Less than 5 & 5 & 15 & 25 & 40\\
Indifferent & --- & Less than 1 & 1 & 15 & 30\\
Friendly & --- & --- & Less than 1 & 1 & 20\\
Helpful & --- & --- & --- & Less than 1 & 1}

\Table{}{l X X}{
\tableheader Attitude & \tableheader Means & \tableheader Possible Actions\\
Hostile & Will take risks to hurt you & Attack, interfere, berate, flee\\
Unfriendly & Wishes you ill & Mislead, gossip, avoid, watch suspiciously, insult\\
Indifferent & Doesn’t much care & Socially expected interaction\\
Friendly & Wishes you well & Chat, advise, offer limited help, advocate\\
Helpful & Will take risks to help you & Protect, back up, heal, aid
}

\textbf{Action:} Changing others' attitudes with Diplomacy generally takes at least 1 full minute (10 consecutive full-round actions). In some situations, this time requirement may greatly increase. A rushed Diplomacy check can be made as a full-round action, but you take a -10 penalty on the check.

\textbf{Try Again:} Optional, but not recommended because retries usually do not work. Even if the initial Diplomacy check succeeds, the other character can be persuaded only so far, and a retry may do more harm than good. If the initial check fails, the other character has probably become more firmly committed to his position, and a retry is futile.

\textbf{Special:} A half-elf has a +2 racial bonus on Diplomacy checks.

If you have the Negotiator feat, you get a +2 bonus on Diplomacy checks.

\textbf{Synergy:} If you have 5 or more ranks in Bluff, Knowledge (nobility and royalty), or Sense Motive, you get a +2 bonus on Diplomacy checks.
\Skill{Disable Device}{Int; Trained Only}
\textbf{Check:} The Disable Device check is made secretly, so that you don’t necessarily know whether you’ve succeeded.

The DC depends on how tricky the device is. Disabling (or rigging or jamming) a fairly simple device has a DC of 10; more intricate and complex devices have higher DCs.

If the check succeeds, you disable the device. If it fails by 4 or less, you have failed but can try again. If you fail by 5 or more, something goes wrong. If the device is a trap, you spring it. If you’re attempting some sort of sabotage, you think the device is disabled, but it still works normally.

You also can rig simple devices such as saddles or wagon wheels to work normally for a while and then fail or fall off some time later (usually after 1d4 rounds or minutes of use).

\Table{}{l b{1.4cm} Z{1.2cm} X}{
\tableheader Device & \tableheader Time & \tableheader Disable Device DC & \tableheader Example\\
Simple & 1 round & 10 & Jam a lock\\
Tricky & 1d4 rounds & 15 & Sabotage a wagon wheel\\
Difficult & 2d4 rounds & 20 & Disarm a trap, reset a trap\\
Wicked & 2d4 rounds & 25 & Disarm a complex trap, cleverly sabotage a clockwork device}

If you attempt to leave behind no trace of your tampering, add 5 to the DC.

\textbf{Action:} The amount of time needed to make a Disable Device check depends on the task, as noted above. Disabling a simple device takes 1 round and is a full-round action. An intricate or complex device requires 1d4 or 2d4 rounds.

\textbf{Try Again:} Varies. You can retry if you have missed the check by 4 or less, though you must be aware that you have failed in order to try again.

\textbf{Special:} If you have the Nimble Fingers feat, you get a +2 bonus on Disable Device checks.

A rogue who beats a trap’s DC by 10 or more can study the trap, figure out how it works, and bypass it (along with her companions) without disarming it.

\textbf{Restriction:} Rogues (and other characters with the trapfinding class feature) can disarm magic traps. A magic trap generally has a DC of 25 + the spell level of the magic used to create it.

The spells \emph{fire trap}, \emph{glyph of warding}, \emph{symbol}, and \emph{teleportation circle} also create traps that a rogue can disarm with a successful Disable Device check. \emph{Spike growth} and \emph{spike stones}, however, create magic traps against which Disable Device checks do not succeed. See the individual spell descriptions for details.

\subsubsection{Other Ways To Beat A Trap}
It’s possible to ruin many traps without making a Disable Device check.

\textbf{Ranged Attack Traps:} Once a trap’s location is known, the obvious way to ruin it is to smash the mechanism---assuming the mechanism can be accessed. Failing that, it’s possible to plug up the holes from which the projectiles emerge. Doing this prevents the trap from firing unless its ammunition does enough damage to break through the plugs.

\textbf{Melee Attack Traps:} These devices can be thwarted by smashing the mechanism or blocking the weapons, as noted above. Alternatively, if a character studies the trap as it triggers, he might be able to time his dodges just right to avoid damage. A character who is doing nothing but studying a trap when it first goes off gains a +4 dodge bonus against its attacks if it is triggered again within the next minute.

\textbf{Pits:} Disabling a pit trap generally ruins only the trapdoor, making it an uncovered pit. Filling in the pit or building a makeshift bridge across it is an application of manual labor, not the Disable Device skill. Characters could neutralize any spikes at the bottom of a pit by attacking them---they break just as daggers do.

\textbf{Magic Traps:} Dispel magic helps here. Someone who succeeds on a caster level check against the level of the trap’s creator suppresses the trap for 1d4 rounds. This works only with a targeted dispel magic, not the area version.