\Chapter{Psionics}
{To one extent or another, every human and demi-human on Athas has psionic powers. Most people are wild talents, with only one power that they have learned to use by trial and error. But anyone can harness their psionic powers through careful practice and study, and every city has at least one training hall dedicated to teaching ``the way of the mind.'' Many warriors, templars, and sorcerers have attended these academies and developed powerful psionic abilities in addition to their normal talents. }
{The Wanderer's Journal}

Psionic powers spring from sentient minds. Even an undead creature or a being that has no physical form can create a reserve of inner strength necessary to manifest powers, as long as it has an Intelligence score of at least 1. Vermin possessed of a hive mind ability are an exception to this rule.

A psionic power is a one-time psionic effect. Psionic characters and creatures need not prepare their powers for use ahead of time. They either have sufficient power points to manifest a power or they do not.

A power is manifested when a psionic character pays its power point cost. Some psionic creatures automatically manifest powers, called psi-like abilities, without paying a power point cost. Other creatures pay power points to manifest their powers, just as characters do.

Each power has a specific effect. A power known to a psionic character can be used whenever he or she has power points to pay for it.

\section{Manifesting Powers}
Psionic characters and creatures manifest powers. Whether they cost power points when manifest by a psionic character, or are manifested as psi-like abilities, powers' effects remain the same. The process of manifesting a power is akin to casting a spell, but with significant differences.

\subsection{Choosing A Power}
First you must choose which power to manifest. You can select any power you know, provided you are capable of manifesting powers of that level or higher. To manifest a power, you must pay power points, which count against your daily total. You can manifest the same power multiple times if you have points left to pay for it.

\subsection{Concentration}
To manifest a power, you must concentrate. If something threatens to interrupt your concentration while you're manifesting a power, you must succeed on a Concentration check or lose the power points without manifesting the power. The more distracting the interruption and the higher the level of the power that you are trying to manifest, the higher the DC. (Higher-level powers require more mental effort.)

\textbf{Injury}: Getting hurt or being affected by hostile psionics while trying to manifest a power can break your concentration and ruin a power. If you take damage while trying to manifest a power, you must make a Concentration check (DC 10 + points of damage taken + the level of the power you're manifesting). The interrupting event strikes during manifestation if it occurs between when you start and when you complete manifesting a power (for a power with a manifesting time of 1 round or longer) or if it comes in response to your manifesting the power (such as an attack of opportunity provoked by the manifesting of the power or a contingent attack from a readied action).

If you are taking continuous damage half the damage is considered to take place while you are manifesting a power. You must make a Concentration check (DC 10 + \onehalf the damage that the continuous source last dealt + the level of the power you're manifesting).

If the last damage dealt was the last damage that the effect could deal then the damage is over, and it does not distract you.

Repeated damage does not count as continuous damage.

\textbf{Power}: If you are affected by a power while attempting to manifest a power of your own, you must make a Concentration check or lose the power you are manifesting. If the power affecting you deals damage, the Concentration DC is 10 + points of damage + the level of the power you're manifesting. If the power interferes with you or distracts you in some other way, the Concentration DC is the power's save DC + the level of the power you're manifesting. For a power with no saving throw, it's the DC that the power's saving throw would have if a save were allowed.

\textbf{Grappling or Pinned}: To manifest a power while grappling or pinned, you must make a Concentration check (DC 20 + the level of the power you're manifesting) or lose the power.

\textbf{Vigorous Motion}: If you are riding on a moving mount, taking a bouncy ride in a wagon, on a small boat in rough water, belowdecks in a storm-tossed ship, or simply being jostled in a similar fashion, you must make a Concentration check (DC 10 + the level of the power you're manifesting) or lose the power.

\textbf{Violent Motion}: If you are on a galloping horse, taking a very rough ride in a wagon, on a small boat in rapids or in a storm, on deck in a storm-tossed ship, or being tossed roughly about in a similar fashion, you must make a Concentration check (DC 15 + the level of the power you're manifesting) or lose the power.

\textbf{Violent Weather}: If you are in a high wind carrying blinding rain or sleet, the DC is 5 + the level of the power you're manifesting. If you are in wind-driven hail, dust, or debris, the DC is 10 + the level of the power you're manifesting. In either case, you lose the power if you fail the Concentration check. If the weather is caused by a power, use the rules in the Power subsection above.

\textbf{Manifesting Powers on the Defensive}: If you want to manifest a power without provoking attacks of opportunity, you need to dodge and weave. You must make a Concentration check (DC 15 + the level of the power you're manifesting) to succeed. You lose the power points without successful manifestation if you fail.

\textbf{Entangled}: If you want to manifest a power while entangled in a net or while affected by a power with similar effects you must make a DC 15 Concentration check to manifest the power. You lose the power if you fail.

\subsection{Manifester Level}
The variables of a power's effect often depend on its manifester level, which is equal to your psionic class level. A power that can be augmented for additional effect is also limited by your manifester level (you can't spend more power points on a power than your manifester level). See Augment under Descriptive Text, below.

You can manifest a power at a lower manifester level than normal, but the manifester level must be high enough for you to manifest the power in question, and all level-dependent features must be based on the same manifester level.

In the event that a class feature or other special ability provides an adjustment to your manifester level, this adjustment applies not only to all effects based on manifester level (such as range, duration, and augmentation potential) but also to your manifester level check to overcome your target's power resistance and to the manifester level used in dispel checks (both the dispel check and the DC of the check).

\subsection{Power Failure}
If you try to manifest a power in conditions where the characteristics of the power (range, area, and so on) cannot be made to conform, the manifestation fails and the power points are wasted.

Powers also fail if your concentration is broken (see Concentration, above).

\subsection{The Power's Result}
Once you know which creatures (or objects or areas) are affected, and whether those creatures have made successful saving throws (if any were allowed), you can apply whatever results a power entails.

\subsection{Special Power Effects}
Certain special features apply to all powers.

\textbf{Attacks}: Some powers refer to attacking. All offensive combat actions, even those that don't damage opponents, such as disarm and bull rush, are considered attacks. All powers that opponents can resist with saving throws, that deal damage, or that otherwise harm or hamper subjects are considered attacks. Astral construct and similar powers are not considered attacks because the powers themselves don't harm anyone.

\textbf{Bonus Types}: Many powers give creatures bonuses to ability scores, Armor Class, attacks, and other attributes. Each bonus has a type that indicates how the power grants the bonus. The important aspect of bonus types is that two bonuses of the same type don't generally stack. With the exception of dodge bonuses, most circumstance bonuses, and racial bonuses, only the better bonus works (see Combining Psionic and Magical Effects, below). The same principle applies to penalties—a character taking two or more penalties of the same type applies only the worst one.

\textbf{Bringing Back the Dead}: Various psionic powers, such as reality revision and psionic revivify, have the ability to restore slain characters to life. When a living creature dies, its soul departs the body, leaves the Material Plane, travels through the Astral Plane, and goes to abide on the plane where the creature's deity resides. If the creature did not worship a deity, its soul departs to the plane corresponding to its alignment. Bringing someone back from the dead means retrieving his or her soul and returning it to his or her body.

\textit{Level Loss}: The passage from life to death and back again is a wrenching journey for a being's soul. Consequently, any creature brought back to life usually loses one level of experience. The character's new experience point total is midway between the minimum needed for his or her new (reduced) level and the minimum needed for the next one. If the character was 1st level at the time of death, he or she loses 2 points of Constitution instead of losing a level. This level loss or Constitution loss cannot be repaired by any mortal means, even the spells wish or miracle. A revived character can regain a lost level by earning XP through further adventuring. A revived character who was 1st level at the time of death can regain lost points of Constitution by improving his or her Constitution score when he or she attains a level that allows an ability score increase.

\textit{Preventing Revivification}: Enemies can take steps to make it more difficult for a character to be returned from the dead. Keeping the body prevents others from using a single manifestation of reality revision to restore the slain character to life.

\textit{Revivification Against One's Will}: A soul cannot be returned to life if it does not wish to be. A soul knows the name, alignment, and patron deity (if any) of the character attempting to revive it and may refuse to return on that basis.