\Capitalize{A}{thas} is a world of many races, from the gith who wander the deserts, to the tareks, too stubborn to know when they have died. Giants terrorize the Silt Sea, while belgoi steal grown men in the night. The magic of the Pristine Tower produces the New Races; most never see a second generation. Despite the variety of intelligent life, only a few races have the numbers to significantly impact the politics of the Tablelands.

Though the races of the {\tableheader Dark Sun} campaign setting resemble those of other campaign worlds, it is frequently in name only. The insular elves roam the Tablelands, trusted by no one but their own tribe-mates. Halflings are feral creatures, possessed of a taste for human flesh. Hairless dwarves work endlessly, their entire perception of the world filtered through the lens of a single, all-consuming task. Unsleeping thri-kreen roam the wastes, always hunting their next meal.

\section{Racial Characteristics}

\subsection{Ability Adjustments}

Find your character's race on \tabref{Athasian Racial Ability Adjustments} and apply the adjustments to your character's ability scores. If these changes put your score above 18 or below 3, that's okay, except in the case of Intelligence, which does not go below 3 for characters.

\subsection{Level Adjustment}

To determine the effective character level (ECL) of a character, add its race's level adjustment (LA) to its character class levels.

Use ECL instead of character level to determine how many experience points a character needs to reach its next level. Also use ECL to determine starting wealth for a character.

\subsection{Favored Class}

Each race's favored class is also given on \tabref{Athasian Racial Ability Adjustments}. A character's favored class doesn't count against him or her when determining experience point penalties for multiclassing.

\BigTablePair{Athasian Racial Ability Adjustments}{l l C p{6cm} p{2cm} l}{
\tableheader Race & \tableheader Type (Subtype) & \tableheader LA & \tableheader Ability Adjustments & \tableheader Favored Class & \tableheader Languages \\
Human & Humanoid (human) & --- & --- & Any & Common \\
Aarakocra & Monstrous Humanoid & +1 & $-2$ Strength, +4 Dexterity, $-2$ Charisma & Cleric & Auran, Common \\
Dwarf & Humanoid (dwarf) & --- & +2 Constitution, $-2$ Charisma & Fighter & Common, Dwarven \\
Elf & Humanoid (elf) & --- & +2 Dexterity, $-2$ Constitution & Rogue & Common, Elven \\
Half-elf & Humanoid (elf) & --- & +2 Dexterity, $-2$ Charisma & Any & Common, Elven \\
Half-giant & Giant & +2 & +8 Strength, $-2$ Dexterity, +4 Constitution, \newline $-4$ Intelligence, $-4$ Wisdom, $-4$ Charisma & Barbarian & Common \\
Halfling & Humanoid (halfling) & --- & $-2$ Strength, +2 Dexterity & Ranger & Halfling \\
Mul & Humanoid (dwarf) & +1 & +4 Strength, +2 Constitution, $-2$ Charisma & Gladiator & Common \\
Pterran & Humanoid (pterran) & --- & $-2$ Dexterity, +2 Wisdom, +2 Charisma & Druid, ranger,\newline or telepath & Saurian \\
Thri-kreen & Monstrous Humanoid & +2 & +2 Strength, +4 Dexterity, $-2$ Intelligence, \newline +2 Wisdom, $-4$ Charisma & Psychic warrior & Kreen}

\subsection{Race And Languages}

Only races that live in the reach of the city-states know how to speak Common. A aarakocra, dwarf, elf, half-elf, halfling, pterran, or thri-kreen also speaks a racial language, as appropriate. A character who has an Intelligence bonus at 1st level speaks other languages as well, one extra language per point of Intelligence bonus as a starting character.

\textbf{Literacy}: The ability to read has been outlawed for thousands of years by the sorcerer-kings. All characters in a {\tableheader Dark Sun} campaign start without the ability to read or write.

\textbf{Class-Related Languages}: Clerics, druids, templars, and wizards can choose certain languages as bonus languages even if they're not on the lists found in the race descriptions. These class-related languages are as follows:

\textit{Cleric}: Aquan, Auran, Ignan, Terran.

\textit{Druid}: Sylvan.

\textit{Templar}: Templar's City-State language.

\textit{Wizard}: Draconic.

\section{Humans}
\Quote{Humans are fools, and hopelessly naive as well. They outnumber us; they are everywhere, and yet they have no more sense of their strength than a rat. Let us hope that the Datto remain that way.}{Dukkoti Nightrunner, elven warrior}

While not the strongest race, nor the quickest, humans have dominated the Tablelands for the last three thousand years.

\textbf{Personality:} More than other races, human personality is shaped by their social standing and background.

\textbf{Physical Description:} Human males average 6 feet tall and 200 lbs, while smaller females average 5 \onehalf feet and 140 pounds. Color of eyes, skin, and hair, and other physical features vary wildly; enlarged noses, webbed feet or extra digits are not uncommon.

\textbf{Relations:} Human treatment of other races is usually based on what their culture has taught them. In large settlements, such as in city-states, close proximity with many races leads to a suspicious unfriendly tolerance.

\textbf{Alignment:} Humans have no racial tendency toward any specific alignment.

\textbf{Human Lands:} Humans can be found anywhere, from the great city-states to the barren wastes.

\textbf{Magic:} Most humans fear and hate arcane magic, forming mobs to kill vulnerable wizards.

\textbf{Psionics:} Humans see the Way as a natural part of daily life, and readily become psions.

\textbf{Religion:} Most humans pay homage to the elements. Draji and Gulgs often worship their monarchs.

\textbf{Language:} Most humans speak the common tongue. Nobles and artisans within a given city-state usually speak the city language, but slaves typically only speak Common.

\textbf{Names:} Nobles, artisans and traders use titles or surnames; others some simply use one name.

\textbf{Male Names:} Agis of Asticles, King Tithian, Lord Vordon, Pavek, Trenbull Al'Raam'ke

\textbf{Female Names:} Akassia, General Zanthiros, Lady Essen of Rees, Neeva, Sadira

\textbf{Adventurers:} Some human adventurers seek treasure; others adventure for religious purposes as clerics or druids; others seek companionship or simply survival.

\subsection{Human Racial Traits}
\begin{itemize*}
  \item Medium: As Medium creatures, humans have no special bonuses or penalties due to their size. 
  \item Human base land speed is 30 feet.
  \item 1 extra feat at 1st level.
  \item 4 extra skill points at 1st level and 1 extra skill point at each additional level.
  \item Automatic Language: Common. Bonus Languages: Any (other than secret languages, such as Druidic). See the Speak Language skill.
  \item Favored Class: Any. When determining whether a multiclass human takes an experience point penalty, his or her highest-level class does not count.
\end{itemize*}

\section{Aarakocra}
\Quote{You are all slaves. You all suffer from the tyranny of the ground. Only in the company of clouds will you find the true meaning of freedom.}{Kekko Cloud‐Brother, aarakocra cleric}

Aarakocra are the most commonly encountered bird-people of the Tablelands. Some are from Winter Nest in the White Mountains near Kurn, while others are from smaller tribes scattered in the Ringing Mountains and elsewhere. These freedom‐loving creatures rarely leave their homes high in the mountains, but sometimes, either as young wanderers or cautious adventurers, they venture into the inhabited regions of the Tablelands.

\textbf{Personality:} These bird‐people can spend hours riding the wind currents of the mountains, soaring in the olive‐tinged Athasian sky. While traveling, aarakocra prefer to fly high above to get a good view all around their location and detect any threats well in advance. When they stop to rest, they tend to perch on high peaks or tall buildings. Enclosed spaces threaten the aarakocra, who have a racial fear of being anywhere they cannot stretch their wings. This claustrophobia affects their behavior. Unless it is absolutely necessary, no aarakocra will enter a cave or enclosed building, or even a narrow canyon.

\textbf{Physical Description:} Aarakocra stand 6 \onehalf to 8 feet tall, with a wingspan of about 20 feet. They have black eyes, gray beaks, and from a distance they resemble lanky disheveled vultures. Aarakocran plumage ranges from silver white to brown, even pale blue. Male aarakocra weigh around 100 pounds, while females average 85 pounds. An aarakocra's beak comprises much of its head, and it can be used in combat. At the center of their wings, aarakocra have three‐fingered hands with an opposable thumb, and the talons of their feet are just as dexterous. While flying, aarakocra can use their feet as hands, but while walking, they use their wing‐hands to carry weapons or equipment. Aarakocra have a bony plate in their chest (the breastbone), which provides protection from blows. However, most of their bones are hollow and brittle and break more easily than most humanoids. The aarakocra's unusual build means they have difficulty finding armor, unless it has been specifically made for aarakocra. Aarakocra usually live between 30 and 40 years.

\textbf{Relations:} Aarakocra zealously defend their homeland. They are distrustful of strangers that venture onto their lands. Many of the southern tribes exact tolls on all caravans passing through their lands, sometimes kidnapping scouts or lone riders until tribute is paid. Tribute can take the form of livestock or shiny objects, which aarakocra covet. Some evil tribes may attack caravans without provocation. Aarakocra have great confidence and pride in their ability to fly, but have little empathy for land-bound races.

\textbf{Alignment:} Aarakocra tend towards neutrality with regard to law or chaos. With respect to good and evil, Aarakocran tribes usually follow the alignment of their leader. A tribe whose leader is neutral good will contain lawful good, neutral good, chaotic good and neutral members, with most members being neutral good. Aarakocra, even good ones, rarely help out strangers.

\textbf{Aarakocran Lands:} Most Aarakocran communities are small nomadic tribes. Some prey on caravans, while others or build isolated aeries high in the mountains. The least xenophobic aarakocra generally come from Winter Nest, in the White Mountains, a tribe allied with the city‐state of Kurn. Of all the human communities, only Kurn builds perches especially made for aarakocra to rest and do business. In contrast, king Daskinor of Eldaarich has ordered the capture and extermination of all aarakocra. Other human communities tolerate Aarakocran characters but do not welcome them. Merchants will do business with aarakocra as long as they remain on foot. Most land‐bound creatures are suspicious of strange creatures that fly over their herds or lands unannounced, and templars, even in Kurn, have standing orders to attack creatures that fly over the city walls without permission.

\textbf{Magic:} Most Aarakocran tribes shun wizardly magic, but a few evil tribes have defilers, and one prominent good‐aligned tribe, Winter's Nest, has several preservers.

\textbf{Psionics:} Aarakocra are as familiar with psionics as other races of the tablelands. They particularly excel in the psychoportation discipline. In spite of their low strength and constitutions, they excel as psychic warriors, often using ranged touch powers from above to terrifying effect.

\textbf{Religion:} Aarakocran shamans are usually air clerics, sometimes sun clerics, and occasionally druids. Most rituals of Aarakocran society involve the summoning of an air elemental, or Hraak'thunn in Auran (although an aarakocra would call their language Silvaarak, and not Auran). Summoned air elementals are often used in an important ritual, the Hunt. The Aarakocran coming of age ceremony involves hunting the great beasts found in the Silt Sea.

\textbf{Language:} Athasian aarakocra speak Auran. Aarakocra have no written language of their own, though some of the more sophisticated tribes have borrowed alphabets from their land‐bound neighbors. Regardless of the language spoken, aarakocra do not possess lips, and therefore cannot even approximate the ‘m', ‘b' or ‘p' sounds. They have difficulty also with their ‘f's and ‘v's, and tend to pronounce these as ‘th' sounds.

\textbf{Male Names:} Akthag, Awnunaak, Cawthra, Driikaak, Gazziija, Kraah, Krekkekelar, Nakaaka, Thraka.

\textbf{Female Names:} Arraako, Kariko, Kekko, Lisako, Troho.

\textbf{Tribal Names:} Cloud Gliders, Sky Divers, Peak Masters, Far Eyes, Brothers of the Sun.

\textbf{Adventurers:} Adventuring aarakocra are usually young adults with a taste for the unknown. They are usually curious, strong‐minded individuals that wish to experience the lives of the land‐bound peoples. Good tribes see these young ones as undisciplined individuals, but can tolerate this behavior. Evil tribes may view this sort of adventurous behavior as treacherous, and may even hunt down the rogue member.

\subsection{Aarakocra Society}
The aarakocra have a tribal society. The civilized tribes of Winter Nest form the largest known community of aarakocra in the Tyr region. Though their communities are lead by a chieftain, the aarakocra have a great love of personal freedom. So while the chieftain makes all major decisions for the community, unless she consults with the tribal elders and builds a strong consensus within the tribe first, her decisions may be ignored.

Air and sun shamans play an important role in aarakocra societies. Aarakocra worship the sun because it provides them with the thermals they need to soar. The air shamans of Winter Nest lead their community in daily worship of the air spirits.

Aarakocra of Winter Nest have a deep and abiding respect for the gifts of nature and little patience for those who abuse those gifts. They look after the natural resources of the White Mountains and have been known to punish those who despoil or abuse them.

In more primitive societies, female aarakocra rarely travel far from the safety of the nest, and focus solely on raising the young. In Winter Nest, both sexes participate in all aspects of society, with females more often elected by the elders to be chieftains.

Aarakocra believe that their ability to fly makes them superior to all other races and thus they have great confidence and pride in themselves. Though they often express sympathy for people unable to fly, this more often comes across as condescending.

Aarakocra are carnivores, but do not eat intelligent prey.
\subsection{Roleplaying Suggestions}
Loneliness doesn't bother you like it bothers people of other races. You loathe the heat and stink of the cities, and long for cold, clean mountain air. The spectacle and movement of so many sentient beings fascinates you, but watching them from above satisfies your curiosity. The very thought of being caught in a crowd of creatures, pinned so tight that you can't move your own wings, fills you with terror.

You are friendly enough with people of other races, provided they respect your physical distance, and are willing to be the ones that approach you. You form relationships with individuals, but don't involve yourself in the politics of other racial communities - in such matters you prefer to watch from above and to keep your opinions to yourself unless asked.

You prefer to enter buildings through a window rather than through a door. Your instincts are to keep several scattered, hidden, nests throughout the areas that you travel regularly: one never knows when one might need a high place to rest. Remember your love of heights and claustrophobia, and rely on Aarakocran skills and tactics (dive‐bombing). Take advantage of your flying ability to scout out the area and keep a ``bird's eye view'' of every situation.

\subsection{Aarakocra Racial Traits}
\begin{itemize*}
    \item $-2$ Strength, +4 Dexterity, $-2$ Constitution: Aarakocra have keen reflexes, but their lightweight bones are fragile.
    \item Monstrous Humanoid: Aarakocra are not subject to spells or effects that affect humanoids only, such as charm person or dominate person.
    \item Medium: As Medium creatures, aarakocra have no special bonuses or penalties due to size.
    \item Low‐light vision: Aarakocra can see twice as far as a human in moonlight and similar conditions of poor illumination, retaining the ability to distinguish color and detail.
    \item Aarakocra base land speed is 20 feet, and can fly with a movement rate of 90 feet (average maneuverability).
    \item +6 racial bonus to Spot checks in daylight. Aarakocra have excellent vision.
    \item Natural Armor: Aarakocra have +1 natural armor bonus due to their bone chest plate that provides some protection from blows.
    \item Natural Weaponry: An aarakocra can rake with its claws for 1d3 points of damage, and use its secondary bite attack for 1d2 points of damage.
    \item Claustrophobic: Aarakocra receive a $-2$ morale penalty on all rolls when in an enclosed space. Being underground or in enclosed buildings is extremely distressing for them.
    \item Aerial Dive: Aarakocra can make dive attacks. A dive attack works just like a charge, but the diving creature must move a minimum of 30 feet. If attacking with a lance, the aarakocra deals double damage on a successful attack. Optionally, the aarakocra can make a full attack with its natural weapons (two claws and one bite) at the end of the charge, dealing normal damage.
    \item Automatic Languages: Auran and Common. Bonus Languages: Elven, Gith, and Saurian. Aarakocra often learn the languages of their allies and enemies.
    \item Favored Class: Cleric. A multiclass aarakocra's cleric class does not count when determining whether he takes an experience point for multiclassing.
    \item Level Adjustment: +1. Aarakocra are slightly more powerful and gain levels more slowly than most of the humanoid races of the Tablelands.
\end{itemize*}
\input{sections/2.3-dwarves.tex}
\input{sections/2.4-elves.tex}
\section{Half-Elves}
\Quote{People are no good. You can only trust animals and the bottle.}{Delmao, half‐Elven thief}
Unlike the parents of muls, elves and humans are often attracted to each other. Half‐elves are typically the unwanted product of a casual interracial encounter.

\textbf{Personality:} Half‐elves are notorious loners. Many Athasians believe that half‐elves combine the worst traits of both races, but the most difficult aspect of half‐elves --- their lack of self-confidence comes not from their mixed origins but rather from a life of rejection from both parent races. Half‐elves try in vain to gain the respect of humans or elves.

\textbf{Physical Description:} Averaging over six feet tall, half‐elves combine Elven dexterity with human resilience. Bulkier than elves, most half‐elves find it easier to pass themselves off as full humans than as full elves, but all have some features that hint at their Elven heritage.

\textbf{Relations:} Humans distrust the half‐elf's Elven nature, while elves have no use for their mixed-blood children; Elven traditions demand that such children be left behind. Human society gives half‐elves have a better chance of survival, but even less kindness. Half‐elves sometimes find friendship among muls or even Thri‐kreen. Half‐elves will cooperate with companions when necessary, but find it difficult to rely on anyone. Many half‐elves also turn to the animal world for company, training creatures to be servants and friends. Ironically, the survival skills and animal affinity that half‐elves developed to cope with isolation make them valuable beast handlers in human society.

\textbf{Alignment:} Lawful and neutral half‐elves labor for acceptance from a parent race, while chaotic ones have given up on acceptance, electing instead to reject the society that has rejected them.

\textbf{Half‐Elven Lands:} Despite their unique nature, half‐elves don't form communities. The few half‐elves that settle down tend to live among humans who, unlike elves, at least find a use for them.

\textbf{Magic:} Half‐elves often take up arcane studies, because it is a solitary calling.

\textbf{Psionics:} Mastery of the Way often provides the independence and self-knowledge that half‐elves seek, and membership in a psionic academy can provide the half‐elf with acceptance.

\textbf{Religion:} Because of their alienation from society and their affinity with animals, half‐elves make excellent druids. Some half‐elves turn their resentment of society into a profession and become sullen, bullying templars. As clerics, they are drawn to water's healing influence.

\textbf{Language:} Half‐elves all speak the Common tongue. A few half‐elves pick up the Elven language.

\textbf{Names:} Half‐elves nearly always have human names. Unable to run as elves, they never receive Elven given names, or acceptance in an Elven tribe that they could use as surname.

\textbf{Adventurers:} In a party, half‐elves often seem detached and aloof.

\subsection{Half-Elf Society}
Unlike other races, half‐elves do not consider themselves a separate race, and, with very few exceptions, do not try to form half‐Elven communities. A half‐elf's life is typically harder than either a human's or an elf's. It is difficult for half‐elves to find acceptance within either Elven or human society. Elves have not tolerance for those of mixed heritage, while humans do not trust their Elfish side. On the whole, humans are far more tolerant of half‐elves than elves, who often refuse to allow such children into their tribes, and are likely to cast the half‐elf's mother from the tribe as well.

Most half‐elves consider themselves outsiders to all society and tend to wander throughout their entire lives, going through life as an outsider and loner. Half‐elves are forced to develop a high level of self‐reliance. Most half‐elves take great pride in their self‐reliance, but this pride often makes half‐elves seem aloof to others. For many half‐elves the detachment is a defensive mechanism to deal with a desire for acceptance from either human or Elven society that will likely never come. Some half‐elves turn to the animal world for company, training creatures to be servants and friends.

\subsection{Roleplaying Suggestions}
Desperate for the approval of either elves or humans, you are even more desperate to appear independent and self-reliant, to cover your desire for approval. As a result, you tend towards a feisty, insecure, sullen self-reliance, refusing favors. You take every opportunity to show off your skills in front of elves and humans, but if an elf or a human were to actually praise you, you would probably react awkwardly or suspiciously. From your childhood, your closest friendships have been with animals. Other half‐elves do not interest you. As time goes by and you learn from experience, you will find that you can also get along with other races neither human nor Elven: dwarves, pterran, muls, even thri‐kreen. You don't feel the terrible need for their approval, and yet they give it more readily.

\subsection{Half-Elf Racial Traits}
\begin{itemize}
    \item +2 Dexterity, $-2$ Charisma: Half‐elves are limber like their Elven parents, but their upbringing leaves them with a poor sense of self, and affects their relations with others.
    \item Humanoid (elf): Half‐elves are humanoid creatures with the elf subtype.
    \item Medium: As Medium creatures, half‐elves have no bonuses or penalties due to size.
    \item Half‐elf base land speed is 30 feet.
    \item Low-Light Vision: A half-elf can see twice as far as a human in starlight, moonlight, torchlight, and similar conditions of poor illumination. She retains the ability to distinguish color and detail under these conditions..
    \item Half‐elves gain a +2 racial bonus to Disguise checks when impersonating elves or humans.
    \item +1 racial bonus on Listen, Search and Spot checks. Half‐elves have keen senses, but not as keen as those of an elf.
    \item +2 racial bonus on all Survival and Handle Animal checks. Half‐elves spend a lot of time in the wilds of the tablelands.
    \item Elven Blood: For all effects related to race, a half‐elf is considered an elf. Half‐elves, for example, are just as vulnerable to effects that affect elves as their elf ancestors are, and they can use magic items that are only usable by elves.
    \item Automatic Languages: Common and Elven. Bonus Languages: Any.
    \item Favored Class: Any. When determining whether a multiclass half‐elf takes an experience point penalty, his highest-level class does not count when determining whether he takes an experience point for multiclassing.
\end{itemize}
\input{sections/2.6-half-giants.tex}
\section{Halflings}
\Quote{Be wary of the forest ridge. The halflings who live there would as soon eat you alive as look at you. Chances are you won't even notice them until you've become the main course.}{Mo'rune, half‐Elven ranger}

Halflings are masters of the jungles of the Ringing Mountains. They are small, quick and agile creatures steeped in an ancient and rich culture that goes back far into Athas' past. Although they are not common in the Tablelands, some halflings leave their homes in the forests to adventure under the Dark Sun. As carnivores, halflings prefer to eat flesh raw.

\textbf{Personality:} Halflings have difficulty understanding others' customs or points of view, but curiosity helps some halflings overcome their xenophobia. Little concerned with material wealth, halflings are more concerned with how their actions will affect other halflings.

\textbf{Physical Description:} Halflings are small creatures, standing only about 3 1/2 feet tall and weighing 50 to 60 pounds. Rarely affected by age, halfling faces are often mistaken for the faces of human children. They dress in loincloths, sometimes with a shirt or vest, and paint their skins with bright reds and greens. Forest halflings rarely tend to their hair, and some let it grow to great lengths, though it can be unkempt and dirty. They live to be about 120 years old.

\textbf{Relations:} Halfling's culture dominates their relations with others. They relate very well to each other, since they all have the same cultural traits and are able to understand each other. Halflings of different tribes still share a tradition of song, art and poetry, which serves as a basis of communication. Creatures that do not know these cultural expressions are often at a loss to understand the halfling's expressions, analogies and allusions to well-known halfling stories. Halflings can easily become frustrated with such ``uncultured'' creatures. They abhor slavery and most halflings will starve themselves rather than accept slavery.

\textbf{Alignment:} Halflings tend towards law and evil. Uncomfortable with change, halflings tend to rely on intangible constants, such as racial identity, family, clan ties and personal honor. On the other hand, halflings have little respect for the laws of the big people.

\textbf{Halfling Lands:} Halflings villages are rare in the tablelands. Most halflings live in tribes or clans in the Forest Ridge, or in the Rohorind forest west of Kurn. Many dwell in treetop villages. Non-halflings typically only see these villages from within a halfling cooking pot.

\textbf{Magic:} Many halfling tribes reject arcane magic. Tribes that accept wizards tend to have preserver chieftains. Only renegade halfling tribes are ever known to harbor defilers.

\textbf{Psionics:} Many halflings become seers or nomads. In the forest ridge, many tribal halflings become multiclassed seer/rangers, and become some of the deadliest trackers on Athas.

\textbf{Religion:} Halflings' bond with nature extends into most aspects of their culture. A shaman or witch doctor, who also acts as a spiritual leader, often rules their clans. This leader is obeyed without question. Halfling fighters willingly sacrifice themselves to obey their leader.

\textbf{Language:} Halflings rarely teach others their language, but some individuals of the Tablelands have learned the wild speech. Halflings found in the Tablelands often learn to speak Common.

\textbf{Names:} Halflings tend to have only one given name.

\textbf{Male Names:} Basha, Cerk, Derlan, Drassu, Entrok, Kakzim, Lokee, Nok, Pauk, Plool, Sala, Tanuka, Ukos, Zol.

\textbf{Female Names:} Alansa, Anezka, Dokala, Grelzen, Horga, Jikx, Joura, Nasaha, Vensa.

\textbf{Adventurers:} Exploring the Tablelands gives curious halflings the opportunity to learn other customs. Although they may at first have difficulty in understanding the numerous practices of the races of the Tablelands, their natural curiosity enables them to learn and interact with others. Other halflings may be criminals, renegades or other tribal outcasts, venturing into the Tablelands to escape persecution by other halflings.

\subsection{Halfling Society}
Most halflings have a common outlook on life that results in considerable racial unity across tribal and regional ties. Rarely will one halfling draw the blood of another even during extreme disagreements. Only renegade halflings do not share this racial unity, and are cast out of their tribes because of it.

Halfling society is difficult for other races to understand, as such concepts as conquest and plundering have no place. The most important value in halfling society is the abilities of the inner self as it harmonizes with the environment and the rest of the halfling race.

Halflings are extremely conscious of their environment. They are sickened by the ruined landscape of the Tyr region and desperately try to avoid having similar devastation occur to their homelands in the Forest Ridge. Most halflings believe that care must be taken to understand and respect nature and what it means to all life on Athas.

Halfling culture is expressed richly through art and song. Story telling in which oral history is passed on to the next generation is an important part of each halfling community. Halflings rely on this shared culture to express abstract thoughts and complicated concepts. This causes problems and frustration when dealing with non‐halflings. Typically halflings assume that whomever they are talking to have the same cultural background to draw upon, and find it difficult to compensate for a listener who is not intimately familiar with the halfling history and ``lacks culture.''

Generally open‐minded, wandering halflings are curious about outside societies and will attempt to learn all they can about other cultures. Never, will they adopt aspects of those cultures as their own, believing halfling culture to be innately superior to all others. Nor do they seek to change others' culture or views.

While halflings are omnivorous, they vastly prefer meat. Their meat heavy diet means that halflings view all living creatures, both humanoid and animal, as more food than equals. At the same time, most halflings believe that other races have the same perception of them. As a result, halflings are rarely likely to trust another member of any other race.

\subsection{Roleplaying Suggestions}
Remember to consistently take your height into account. Roleplay the halfling culture described above: eating opponents, treating fellow halflings with trust and kindness, suspicion of big people, and general lack of interest in money.

\subsection{Halfling Racial Traits}
\begin{itemize}
    \item $-2$ Strength, +2 Dexterity: Halflings are quick and stealthy, but weaker than humans.
    \item Halflings receive a $-2$ penalty to all Diplomacy skill checks when dealing with other races.
    \item Small: As a Small creature, a halfling gains a +1 size bonus to Armor Class, a +1 size bonus on attack rolls, and a +4 size bonus on Hide checks, but she uses smaller weapons than humans use, and her lifting and carrying limits are three-quarters of those of a Medium character.%Halflings gain a +1 size bonus to Armor Class and a +1 size bonus on all attack rolls.
    \item Halfling base land speed is 20 feet.
    \item +2 racial bonus on Climb, Jump and Move Silently checks: Halflings are agile.
    \item +2 racial bonus on saving throws against spells and spell‐like effects.
    \item +1 racial attack bonus with a thrown weapon: javelins and slings are common weapons in feral halfling society, and many halflings are taught to throw at an early age.
    \item +4 racial bonus on Listen checks: Halflings have keen ears. Their senses of smell and taste are equally keen; they receive a +4 to all Wisdom checks that assess smell or taste.
    \item Automatic Languages: Halfling. Bonus Languages: Common, Dwarven, Elven, Gith, Kreen, Rhul‐thaun, Sylvan, and Yuan-ti.
    \item Favored Class: Ranger. A multiclass halfling's ranger class does not count when determining whether he takes an experience point for multiclassing.
\end{itemize}
\section{Muls}
\Quote{See, the trick is to break their will. Not too much, mind you. Nobody wants to watch a docile gladiator, and muls are too expensive to waste as labor slaves. But, you don't want them trying to escape every other day. Would you like to tell the arena crowd that their favorite champion will not be appearing in today's match because he died trying to escape your pens?}{Gaal, Urikite arena trainer}

Born from the unlikely parentage of dwarves and humans, muls combine the height and adaptable nature of humans with the musculature and resilience of dwarves. Muls enjoy traits that are uniquely their own, such as their robust metabolism and almost inexhaustible capacity for work. The hybrid has disadvantages in a few areas as well: sterility, and the social repercussions of being created for a life of slavery. Humans and dwarves are not typically attracted to each other. The only reason that muls are so common in the Tablelands is because of their value as laborers and gladiators: slave-sellers force-breed humans and dwarves for profit. While mul-breeding practices are exorbitantly lucrative, they are often lethal to both the mother and the baby. Conception is difficult and impractical, often taking months to achieve. Even once conceived, the mul takes a full twelve months to carry to term; fatalities during this period are high. As likely as not, anxious overseers cut muls from the dying bodies of their mothers.

\textbf{Personality:} All gladiators who perform well in the arenas receive some degree of pampered treatment, but muls receive more pampering than others. Some mul gladiators even come to see slavery as an acceptable part of their lives. However, those that acquire a taste of freedom will fight for it. Stoic and dull to pain, muls are not easily intimidated by the lash. Masters are loath to slay or maim a mul who tries repeatedly to escape, although those who help the mul's escape will be tormented in order to punish the mul without damaging valuable property. Once a mul escapes or earns his freedom, slavery remains a dominant part of his life. Most muls are heavily marked with tattoos that mark his ownership, history, capabilities and disciplinary measures. Even untattooed muls are marked as a potential windfall for slavers: it is clearly cheaper to ``retrieve'' a mul who slavers can claim had run away, than to start from scratch in the breeding pits.

\textbf{Physical Description:} Second only to the half-giant, the mul is the strongest of the common humanoid races of the tablelands. Muls grow as high as seven feet, weighing upwards of 250 pounds, but carry almost no fat at all on their broad muscular frames. Universal mul characteristics include angular, almost protrusive eye ridges, and ears that point sharply backwards against the temples. Most muls have dark copper-colored skin and hairless bodies.

\textbf{Relations:} Most mul laborers master the conventions of slave life, figuring out through painful experience who can be trusted and who cannot. (Muls learn from their mistakes in the slave pits to a greater extent than other races not because they are cleverer, but because unlike slaves of other races they tend to survive their mistakes, while other slave races are less expensive and therefore disposable. Only the most foolish and disobedient mul would be killed. Most masters will sell a problem mul slave rather than kill him.) Their mastery of the rules of slave life and their boundless capacity for hard work allows them to gain favor with their masters and reputation among their fellow slaves.

\textbf{Alignment:} Muls tend towards neutrality with respect to good and evil, but run the gamut with respect to law or chaos. Many lawful muls adapt well to the indignities of slavery, playing the game for the comforts that they can win as valued slaves. A few ambitious lawful muls use the respect won from their fellow-slaves to organize rebellions and strike out for freedom. Chaotic muls, on the other hand, push their luck and their value as slaves to the breaking point, defying authority, holding little fear for the lash.

\textbf{Mul Lands:} As a collective group, muls have no lands to call their own. Occasionally, escaped muls band together as outlaws and fugitives, because of their common ex-slave backgrounds, and because their mul metabolism makes it easier for them to survive as fugitives while other races cannot keep up. Almost without exception, muls are born in the slave pits of the merchants and nobles of the city-states. Most are set to work as laborers, some as gladiators, and fewer yet as soldier-slaves. Very few earn their freedom, a greater number escape to freedom among the tribes of ex-slave that inhabit the wastes.

\textbf{Magic:} Muls dislike what they fear, and they fear wizards. They also resent that a wizard's power comes from without, with no seeming effort on the wizard's part, while the mul's power is born of pain and labor. Mul wizards are unheard of.

\textbf{Psionics:} Since most slave owners take steps to ensure that their property does not get schooled in the Way, it is rare for a mul to receive any formal training. Those that get this training tend to excel in psychometabolic powers.

\textbf{Religion:} Even if muls were to create a religion of their own, as sterile hybrids, they would have no posterity to pass it on to. Some cities accept muls as templars. Mul clerics tend to be drawn towards the strength of elemental earth.

\textbf{Language:} Muls speak the Common tongue of slaves, but those favored muls that stay in one city long enough before being sold to the next, sometimes pick up the city language. Because of their tireless metabolism, muls have the capacity to integrate with peoples that other races could not dream of living with, such as elves and Thri-kreen.

\textbf{Names:} Muls sold as laborers will have common slave names. Muls sold as gladiators will often be given more striking and exotic names. Draji names (such as Atlalak) are often popular for gladiators, because of the Draji reputation for violence. Masters who change their mul slaves' professions usually change their names as well, since it is considered bad form to have a gladiator with a farmer's name, and a dangerous incitement of slave rebellions to give a common laborer the name of a gladiator.

\textbf{Adventurers:} Player character muls are assumed to have already won their freedom. Most freed mul gladiators take advantage of their combat skills, working as soldiers or guards. Some turn to crime, adding rogue skills to their repertoire. A few muls follow other paths, such as psionics, templar orders or elemental priesthoods.

\subsection{Mul Society}
Muls have no racial history or a separate culture. They are sterile and cannot reproduce, preventing them from forming family groups and clans. The vast majority of muls are born in slavery, through breeding programs. Often the parents resent their roles in the breeding program and shun the child, leaving the mul to a lonely, hard existence. The taskmaster's whip takes the place of a family. For these reasons, many muls never seek friends or companionship, and often have rough personalities with tendencies towards violence.

The mul slave trade is very profitable, and thus the breeding programs continue. A slave trader can make as much on the sale of a mul as he could with a dozen humans. As slaves, a mul has his profession selected for him and is given extensive training as he grows.

Mul gladiators are often very successful, and win a lot of money for their owners. Highly successful gladiators are looked after by their owners, receiving a large retinue of other slaves to tend to their whims and needs. This has lead to the expression, ``pampered like a mul,'' being used often by the common folk.

Muls not trained as gladiators are often assigned to hard labor and other duties that can take advantage of the mul's hardy constitution and endurance.

\subsection{Roleplaying Suggestions}
Born to the slave pens, you never knew love or affection; the taskmaster's whip took the place of loving parents. As far as you have seen, all of life's problems that can be solved are solved by sheer brute force. You know to bow to force when you see it, especially the veiled force of wealth, power and privilege. The noble and templar may not look strong, but they can kill a man with a word. You tend towards gruffness. In the slave pits, you knew some muls that never sought friends or companionship, but lived in bitter, isolated servitude. You knew other muls who found friendship in an arena partner or co-worker. You are capable of affection, trust and friendship, but camaraderie is easier for you to understand and express - warriors slap each other on the shoulder after a victory, or give their lives for each other in battle. You don't think of that sort of event as ``friendship'' - it just happens.

\subsection{Mul Racial Traits}
\begin{itemize*}
    \item +4 Strength, +2 Constitution, $-2$ Charisma: Combining the human height with the Dwarven musculature, muls end up stronger than either parent race, but their status as born-to-be slaves makes them insecure in their dealings with others.
    \item Humanoid (dwarf): Muls are humanoid creatures with the dwarf subtype.
    \item Medium: As Medium creatures, muls have no bonuses or penalties due to size.
    \item Mul base land speed is 30 feet.
    \item Darkvision: Muls can see in the dark up to 30 feet. Darkvision is black and white only, but is otherwise like normal sight, and muls can function just fine with no light at all.
    \item Tireless: Muls get a +4 racial bonus to checks for performing a physical action that extends over a period of time (running, swimming, holding breath, and so on). This bonus stacks with the \feat{Endurance} feat. This bonus may also be applied to savings throws against spells and magical effects that cause weakness, fatigue, exhaustion or enfeeblement.
    \item Extended Activity: Muls may engage in up to 12 hours of hard labor or forced marching without suffering from fatigue.
    \item Dwarven Blood: For all effects related to race, a mul is considered a dwarf. Muls, for example, are just as vulnerable to effects that affect dwarves as their dwarf ancestors are, and they can use magic items that are only usable by dwarves.
    \item Nonlethal Damage Resistance 1/---. Muls are difficult to subdue, and do not notice minor bruises, scrapes, and other discomforts that pain creatures of other races.
    \item Favored Class: Gladiator. A multiclass mul's gladiator class does not count when determining whether he takes an experience point for multiclassing.
    \item Automatic Language: Common. Bonus Languages: Dwarven, Elven, Gith, and Giant.
    \item Level Adjustment: +1. As a hybrid half-race, muls are considerably more powerful than either of their parent races, thus they gain levels more slowly.
\end{itemize*}
\section{Pterrans}
\Quote{The people of the Tablelands know nothing of life. They choose no Path for themselves, and consume everything until they are dead.}{Keltruch, pterran ranger}

Pterrans are rarely seen in the Tablelands. They live their lives in the Hinterlands, rarely leaving the safety of their villages. However, the recent earthquake and subsequent storms have brought disruption into the pterran's lives. More pterrans now venture outside their homes, and come to the Tyr region to seek trade and information.

\textbf{Personality:} Among strangers, pterrans seem like subdued, cautious beings, but once others earn a pterran's trust, they will find an individual that is open, friendly, inquisitive, and optimistic. In other respects, a pterran's personality is largely shaped by her chosen life path: Pterrans who choose the path of the warrior are less disturbed by the brutality of the Tablelands; they are constantly examining their surroundings and considering how the terrain where they are standing could be defended; they take greatest satisfaction from executing a combat strategy that results in victory without friendly casualties. Pterrans who choose the path of the druid are most interested in plants, animals, and the state of the land; they take greatest satisfaction when they eliminate a threat to nature. Pterrans that choose the path of the mind are most interested in befriending and understanding other individuals and societies; these telepaths take greatest satisfaction from intellectual accomplishments such as solving mysteries, exposing deception, resolving quarrels between individuals, and establishing trade routes between communities.

\textbf{Physical Description:} Pterrans are 5 to 6 \onehalf feet tall reptiles with light brown scaly skin, sharp teeth, and a short tail. Pterrans wear little clothing, preferring belts and loincloths, or sashes. They walk upright, like humanoids, and have opposing thumbs and three-fingered, talon-clawed hands. Pterrans have two shoulder stumps, remnants of wings they possessed long ago, and a finlike growth juts out at the back of their heads. Pterrans weigh between 180 to 220 pounds. There is no visible distinction between male and female pterrans.

\textbf{Relations:} Pterrans are new to the Tablelands, and unaccustomed to cultures and practices of the region. They have learned to not judge too quickly. Their faith in the Earth Mother means they undertake their adventure with open minds, but they will remain subdued and guarded around people they do not trust. A pterran's respect for the Earth Mother governs all his behavior. Creatures that openly destroy the land or show disrespect for the creatures of the wastes are regarded suspiciously. Pterrans understand the natural cycle of life and death, but have difficulty with some aspects of the city life, such as cramped living spaces, piled refuse, and the smells of unwashed humanoids.

\textbf{Alignment:} Pterrans tend towards lawful, well-structured lives, and most of them are good. Evil pterran adventurers are usually outcasts who have committed some horrible offense.

\textbf{Pterran Lands:} Most adventuring pterrans come from one of two villages in the Hinterlands, southwest of the Tyr regions: Pterran Vale and Lost Scale.

\textbf{Magic:} The wizard's use of the environment as a source of power conflicts with a pterran's religious beliefs. Pterrans will cautiously tolerate members of other races who practice preserving magic, if the difference is explained to them.

\textbf{Psionics:} Virtually all pterrans have a telepathic talent, and pterran psions are nearly universally telepaths. Telepathy is considered one of the honored pterran ``life paths.''

\textbf{Religion:} Pterrans worship the Earth Mother, a representation of the whole world of Athas. Their devotion to the Earth Mother is deeply rooted in all aspects of their culture, and it defines a pterran's behavior. All rituals and religious events are related to their worship of the Earth Mother. Religious events include festivals honoring hunts or protection from storms, with a priest presiding over the celebration. Most pterran priests are druids.

\textbf{Language:} Pterran speak Saurian with an accent that is difficult for other races to understand. The long appendage at the back of their head enables them to create sounds that no other race in the Tablelands can reproduce. The sounds are low, and resonate through the pterran's crest. Humanoid vocal chords cannot reproduce such sounds. Pterrans learn the Common tongue easily, but speak it with a slight, odd accent.

\textbf{Names:} Pterrans earn their first name just after they hatch, based on the weather and season of their hatching. After the pterran has decided upon a Life Path and has completed their apprenticeship, she receives title that becomes the first part of her name. This marks her transition into pterran society. There are a number of traditional names associated with each Life Path, but names do not always come from these ranks.

\textbf{Male Names:} Airson, Darksun, Earthsong, Suntail, Goldeye, Onesight, Terrorclaw.

\textbf{Female Names:} Cloudrider, Greenscale, Lifehearth, Rainkeeper, Spiritally, Watertender.

\textbf{Path Name:} Aandu, Caril, Dsar, Everin, Illik, Myril, Odten, Qwes, Pex, Ptellac, Ristu, Ssrui, Tilla, Xandu.

\textbf{Tribe or Village Names:} Pterran Vale, Lost Scale

\textbf{Adventurers:} Pterrans adventure because they believe the recent earthquake and disturbing events are signs from the Earth Mother that they should get more involved in the planet's affairs. They believe that these recent upheavals of nature are signs that the Earth Mother needs help, and this is a call the pterrans will gladly accept. As such, the most brave and adventurous of the pterrans have begun to establish contact with Tyr and some merchant houses, hoping to expand their contacts and information.

\subsection{Pterran Society}
Pterran society is based largely on ceremony and celebrations. An area is set aside in the center of each village for ceremonies. Pterrans revere the world of Athas as the Earth Mother, and believe themselves to be her favored children. Throughout the day, they engage in a number of ceremonies that give thanks to the Earth Mother. These are led by druids who play a very important role in pterran society.

A pterran village is a collection of many smaller family dwellings. Pterrans always bear young in pairs.

At age 15 every pterran chooses a ``life path.'' The three main life paths are the path of the warrior, the path of the druid and the path of the psionicist, though lesser life paths exist as well.

More pterrans follow the path of the warrior than any of the other paths, and become protectors of their villages as well as the tribe's weapon makers.

Pterrans that choose the path of the druid provide an important role in the daily ceremonies to the Earth Mother.

Fewer pterrans choose the path of the psionicist than the other two major paths, as psionics are viewed as outside of nature. Psionicists are viewed with suspicion by the rest of the tribe; however, they do provide valuable skills to the tribe and are often the tribe's negotiators when they meet outsiders.

Pterrans are omnivores. Much of their diet comes from hunting animals and raising crops. Kirre, id fiend, and flailer are all considered pterran delicacies.

\subsection{Roleplaying Suggestions}
Remember your character class is your ``life path.'' You think of yourself, and present yourself first and foremost as a druid, a warrior or a psion. Remember your daily celebrations and giving of thanks to the Earth Mother. You can usually find a reason to be grateful. Disrespect for the land angers you, since the whole land has withered under the disrespect of foolish humans and others. You celebrate with song and with dance. You have a good sense of humor but it does not extend to blasphemies such as defiling. In initial role-playing situations, you are unfamiliar with the customs and practices of the societies of the Tyr Region. However, you are not primitive by any definition of the word. You look upon differences with curiosity and a willingness to learn, as long as the custom doesn't harm the Earth Mother or her works.

\subsection{Pterran Racial Traits}
\begin{itemize*}
    \item $-2$ Dexterity, +2 Wisdom, +2 Charisma: Pterrans' strong confidence and keen instincts for others' motives make them keen diplomats, and when they take the path of the psion, powerful telepaths.
    \item Humanoid (psionic, reptilian): Pterrans are humanoid creatures with the psionic and reptilian subtypes.
    \item Medium: As Medium creatures, pterrans have no special bonuses or penalties due to size.
    \item Pterran base land speed is 30 feet.
    \item Poor Hearing: Pterrans have only slits for ears, and their hearing sense is diminished. Pterrans suffer a $-2$ penalty to Listen checks.
    \item Natural Weaponry: Pterrans can use their natural weapons instead of fighting with crafted weapons if they so choose. A pterran can rake with their primary claw attack for 1d3 of damage for each claw, and they bite for 1d4 points of damage as a secondary attack. For more on natural attacks, see MM section on natural weapons.
    \item Psi-Like Ability: At will---\psionic{missive}. All pterrans are gifted from the day they hatch with the ability to communicate telepathically, but only with their fellow reptiles. Manifester level is equal to \onehalf Hit Dice (minimum 1st).
    \item Weapon Familiarity: The following weapon is treated as martial rather than as an exotic weapon: thanak. This weapon is more common among pterrans than among other races.
    \item Automatic Languages: Saurian. Bonus Languages: Common, Dwarven, Elven, Halfling, Giant, Gith, Kreen, and Yuan-ti. Pterran know the languages of the few intelligent creatures that live in the Hinterlands.
    \item Life Path: A pterran's life path determines his favored class. Those following the Path of the Druid have druid as a favored class; the Path of the Mind gives psion as a favored class, while the Path of the Warrior gives ranger as a favored class. A Pterran chooses a life path upon coming of age, and the path cannot be changed once chosen at character creation time.
\end{itemize*}
\input{sections/2.10-thri-kreen.tex}

\section{Other Races}
Athas is a place where members of different races are usually found in the same city-state, usually because they are going to be used in gladiatorial games as an exotic attraction, or to become slaves due to their physical might. Even though they are not usually concentrated on a specific area, these races are significant players in the Tablelands. It is only fitting then, that belgoi, gith, jozhals, ssurrans, tareks, taris, yuan-tis, and a variety of other creatures commonly viewed as monsters might appear as player characters in a {\tableheader Dark Sun} campaign.

All the rules you need to play a character belonging to one of these races can be found in Terrors of Athas, Expanded Psionics Handbook, and the Dungeon Master's Guide. Cultural information about several monstrous races appears in Chapter 7: Life on Athas.

\section{Vital Statistics}
The details of your character's age, gender, height, weight, and appearance are up to you. However, if you prefer some rough guidelines in determining those details, refer to the tables in this section.

\subsection{Character Age}
You can choose or randomly generate your character's age. If you choose it, it must be at least the minimum for the character's race and class (see \hyperref[tab:Random Starting Ages]{Table: Random Starting Ages}). Your character's minimum starting age is the adulthood age of their race plus the number of dice indicated in the entry corresponding to the character's race and class on \hyperref[tab:Random Starting Ages]{Table: Random Starting Ages}.

Alternatively, you may roll dice to determine how old your character is, as specified in \hyperref[tab:Random Starting Ages]{Table: Random Starting Ages}.

With age, a character's physical ability scores decrease and his or her mental ability scores increase. The effects of each aging step are cumulative. However, none of a character's ability scores can be reduced below 1 in this way.

\begin{itemize}
\setlength\itemsep{0em}
\item \textbf{Middle Age:} $-1$ to Str, Dex, and Con; $+1$ to Int, Wis, and Cha.
\item \textbf{Old:} $-2$ to Str, Dex, and Con; $+1$ to Int, Wis, and Cha.
\item \textbf{Venerable:} $-3$ to Str, Dex, and Con; $+1$ to Int, Wis, and Cha.
\end{itemize}

Aarakocra, pterrans, and thri-kreens do not suffer aging penalties or gain aging bonuses until they reach venerable age, at which point all cumulative effects apply.

When a character reaches venerable age, secretly roll his or her maximum age, which is the number from the Venerable column on \hyperref[tab:Aging Effects]{Table: Aging Effects} plus the result of the dice roll indicated on the Maximum Age column on that table, and records the result, which the player does not know. A character who reaches his or her maximum age dies of old age at some time during the following year.

The maximum ages are for player characters. Most people in the world at large die from pestilence, accidents, infections, or violence before getting to venerable age.

\Table{Random Starting Ages}{X l Z{1.2cm} Z{1.2cm} Z{1.2cm}}{
\tableheader Race & \tableheader Adulthood & \tableheader Barbarian Rogue & \tableheader Bard Fighter Gladiator Ps.Warrior Ranger & \tableheader Cleric Druid Psion Templar Wizard \\
Human & 15 years & $+1d4$ & $+1d6$ & $+2d6$\\
Aarakocra & 8 years & $+1d4$ & $+1d6$ & $+2d4$\\
Dwarf & 30 years & $+2d6$ & $+4d6$ & $+6d6$\\
Elf & 20 years & $+1d4$ & $+1d6$ & $+2d6$\\
Half-elf & 15 years & $+1d6$ & $+2d6$ & $+3d6$\\
Half-giant & 25 years & $+1d6$ & $+2d6$ & $+4d6$\\
Halfling & 20 years & $+3d6$ & $+3d6$ & $+4d6$\\
Mul & 14 years & $+1d4$ & $+1d6$ & $+2d6$\\
Pterran & 10 years & $+1d6$ & $+1d6$ & $+1d6$\\
Thri-kreen & 4 years & $+1d4$ & $+1d4$ & $+1d4$}

\Table{Aging Effects}{X c c c c}{
\tableheader Race & \tableheader Middle Age & \tableheader Old & \tableheader Venerable & \tableheader Max. Age\\
Human & 35 yrs. & 53 yrs. & 70 yrs. & +2d20 yrs.\\
Aarakocra & & & 36 yrs. & +1d10 yrs.\\
Dwarf & 100 yrs. & 150 yrs. & 200 yrs. & +4d20 yrs.\\
Elf & 50 yrs. & 75 yrs. & 100 yrs. & +3d20 yrs.\\
Half-elf & 45 yrs. & 60 yrs. & 90 yrs. & +2d20 yrs.  \\
Half-giant & 60 yrs. & 90 yrs. & 120 yrs. & +1d100 yrs. \\
Halfling & 50 yrs. & 75 yrs. & 100 yrs. & +5d10 yrs. \\
Mul & 30 yrs. & 45 yrs. & 60 yrs. & +2d10 yrs. \\
Pterran & & & 40 yrs. & +1d10 yrs. \\
Thri-kreen & & & 25 yrs. & +1d10 yrs.}

\subsection{Height and Weight}
Choose your character's height and weight from the ranges mentioned on the racial description, or roll randomly on \hyperref[tab:Random Height and Weight]{Table: Random Height and Weight}.

The dice roll given in the Height Modifier column determines the character's extra height beyond the base height. That same number multiplied by the dice roll or quantity given in the Weight Modifier column determines the character's extra weight beyond the base weight.

Thri-kreen characters are 48 inches longer than they are tall.

\Table{Random Height and Weight}{X Z{1cm} Z{1.2cm} Z{1cm} Z{1.2cm}}{
\tableheader Race & \tableheader Base Height & \tableheader Height Modifier ($\times 2.5$ cm) & \tableheader Base Weight & \tableheader Weight Modifier ($\times 0.5$ kg)\\
Human, male & 1.47m & $+2d10$ & 60 kg & $\times 2d4$\\
Human, female & 1.35m & $+2d10$ & 42.5 kg & $\times 2d4$\\
Aarakocra, male & 1.93m & $+2d8$ & 35 kg & $\times 1d4$\\
Aarakocra, female & 1.87m & $+2d8$ & 30 kg & $\times 1d4$\\
Dwarf, male & 1.30m & $+2d4$ & 65 kg & $\times 2d6$\\
Dwarf, female & 1.25m & $+2d4$ & 50 kg & $\times 2d6$\\
Elf, male & 2m & $+2d6$ & 65 kg & $\times 2d4$\\
Elf, female & 1.95m & $+2d6$ & 55 kg & $\times 2d4$\\
Half-elf, male & 1.52m & $+2d10$ & 65 kg & $\times 2d4$\\
Half-elf, female & 1.47m & $+2d10$ & 45 kg & $\times 2d4$\\
Half-giant, male & 3m & $+2d12$ & 700 kg & $\times 3d4$\\
Half-giant, female & 3m & $+2d12$ & 500 kg & $\times 3d4$\\
Halfling, male & 0.81m & $+2d4$ & 15 kg & $\times 1$\\
Halfling, female & 0.76m & $+2d4$ & 12.5 kg & $\times 1$\\
Mul, male & 1.47m & $+2d10$ & 65 kg & $\times 2d6$\\
Mul, female & 1.37m & $+2d10$ & 50 kg & $\times 2d6$\\
Pterran, male & 1.47m & $+2d10$ & 65 kg & $\times 2d6$\\
Pterran, female & 1.40m & $+2d10$ & 55 kg & $\times 2d6$\\
Thri-kreen & 2.08m & $+1d6$ & 225 kg & $\times 1d4$}
\section{Region of Origin}

In Athas, where you character comes from can help dictate his speech, clothing, world view, and values. In the context of the game, these cultural differences are expressed in the choices of class, skills, feats, and prestige classes that characters from different regions make.

This section describes the most common choices of game-related options for several known regions of Athas. These choices are not meant to be restrictive, since exceptions always exist to such general rules. They simply offer guidelines for making a character seem like a true representative of his native culture.

\subsection{Balic}
The city-state of Balic sits at the eastern tip of the Balican Peninsula, the piece of land which splits the Estuary of the Forked Tongue into its northern and southern branches. Balic is currently ruled by a triumvirate made up of its largest merchant houses.

\textbf{Classes:} Bard, gladiator, templar.

\textbf{Skills:} Perform (any).

\textbf{Feats:} Performance Artist.

\textbf{Prestige Classes:} Dune trader, master shipfloater, shadow dancer, shadow templar, shadow wizard.

\subsection{Barrier Wastes}
The Barrier Wastes is the desolate area that cuts across a massive portion of the Jagged Cliffs region, and it is home to the Bandit States, a collection of violent humanoid raiding tribes.

\textbf{Classes:} Barbarian, fighter.

\textbf{Skills:} Intimidate, Survival.

\textbf{Feats:} Intimidating Presence, Wastelander.

\textbf{Prestige Classes:} Master scout, kik, savage.

\subsection{Draj}
Draj is a warrior city-state mostly human but intermingled with the other common races, situated on a vast mud flat east of Raam.

\textbf{Classes:} Fighter, gladiator, templar.

\textbf{Skills:} Intimidate, Knowledge (nature).

\textbf{Feats:} Astrologer, Mekillothead.

\textbf{Prestige Classes:} Arrow knight, cerulean, dune trader, eagle knight, jaguar knight, moon priest.

\subsection{Eldaarich}
Eldaarich occupies a small island in the Sea of Silt, just off the mainland. This human city-state is ruled by the mad Daskinor and his ruthless red guards.

\textbf{Classes:} Gladiator, templar.

\textbf{Skills:} Intimidate, Sense Motive.

\textbf{Feats:} Grovel, Paranoid, Reign of Terror.

\textbf{Prestige Classes:} Brown cloak, executioner, red guard.


\subsection{Forest Ridge}
The Forest Ridge stretches all along the western side of the Ringing Mountains, hugging the spine of the range from north to south. This is where most halflings come from.

\textbf{Classes:} Druid, ranger.

\textbf{Skills:} Knowledge (nature), Survival.

\textbf{Feats:} Cannibalism Ritual, Jungle Fighter, Nature's Child.

\textbf{Prestige Classes:} Elite sniper, grove master, halfling protector, tribal psionicist.


\subsection{Gulg}
The predominantly human city-state of Gulg sits inside the southern portion of the Crescent Forest, almost directly east of Tyr.

\textbf{Classes:} Gladiator, templar, ranger.

\textbf{Skills:} Knowledge (nature), Survival.

\textbf{Feats:} Jungle Fighter, Nature's Child.

\textbf{Prestige Classes:} Ambofari, dune trader, hunter noble, elite judaga, master scout, Oba's servant.


\subsection{Jagged Cliffs}
One of the mist isolated places in Athas, the Jagged Cliffs are home to the rhul-thaun, descendants of the rhulisti, and the distant relatives of the modern halflings, and keepers of the life-shaping arts.

\textbf{Classes:} Fighter, ranger.

\textbf{Skills:} Climb, Craft (life-shaped), Knowledge (life-shaping).

\textbf{Feats:} Cliff Combat, Vertical Orientation.

\textbf{Prestige Classes:} Cliffclimber, life-shaper, windrider.


\subsection{Kurn}
Kurn lies in a fertile valley hidden among the White Mountains themselves. The mostly human residents of Kurn are among the most sophisticated and cultured people of Athas.

\textbf{Classes:} Cleric (air), templar, wizard.

\textbf{Skills:} Bluff, Knowledge (arcana).

\textbf{Feats:} Companion.

\textbf{Prestige Classes:} Dune trader, double templar, Kurnan maker, Kurnan spymaster, loremaster.

\subsection{Nibenay}
The city-state of Nibenay is located east of Tyr at the northern tip of the Crescent Forest and it is famous for its artisans and musicians.

\textbf{Classes:} Bard, gladiator, templar.

\textbf{Skills:} Craft (any), Knowledge (nature), Perform (any).

\textbf{Feats:} Artisan, Astrologer, Performance Artist.

\textbf{Prestige Classes:} Dune trader, mystic dancer, soulknife, wife of Nibenay.

\subsection{Raam}
The city-state of Raam is located east of Urik and is one of the largest and most chaotic cities in the Tablelands. It also has one of the most mixed populations.

\textbf{Classes:} Gladiator, templar, psion.

\textbf{Skills:} Craft (any), Intimidate.

\textbf{Feats:} Artisan, Mansadbar, Tarandan Method.

\textbf{Prestige Classes:} Dune trader, kuotagha, servant of Badna, psiologist.

\subsection{Saragar}
Home to the Last Sea of Athas, Saragar is the legendary region where the Green Age still exists. Their residents are mostly human and elves and psionics is everyday life.

\textbf{Classes:} Druid, psion, psychic warrior, wilder.

\textbf{Skills:} Autohipnosys, Knowledge (psionics).

\textbf{Feats:} Psionic Schooling.

\textbf{Prestige Classes:} Metamind, psion uncarnate.


\subsection{Sea of Silt}
An endless plain of pearly powder, The Silt Sea is home to powerful, aggressive, and primitive giants. Most humanoids use it as a way of transportation using specially crafted vehicles.

\textbf{Classes:} Barbarian, cleric (silt), ranger.

\textbf{Skills:} Knowledge (nature), Survival.

\textbf{Feats:} Giant Killer.

\textbf{Prestige Classes:} Elementalist, master shipfloater.


\subsection{Trembling Plains}
The Trembling Plains are named for the enormous herds of mekillot that stampede across the plains during early Fruitbirth season, shaking the ground. It is home to humans, dwarves and half-elves known as Eloy.

\textbf{Classes:} Fighter, ranger.

\textbf{Skills:} Survival.

\textbf{Feats:} Elfish Eloy, Longshanks, Wind Racer.

\textbf{Prestige Classes:} Wind walker.


\subsection{Tyr}
Located in a fertile valley in the foothills of the Ringing Mountains, Tyr was the first city-state to successfully revolt against its sorcerer-king and to unban preserving magic.

\textbf{Classes:} Gladiator, wizard, templar.

\textbf{Skills:} Craft (any), Diplomacy.

\textbf{Feats:} Companion, Freedom, Metalsmith.

\textbf{Prestige Classes:} Black cassock, draqoman, dune trader, templar knight.

\subsection{Urik}
Located northeast of Tyr, between the Dragon's Bowl and the Smoking Crown Mountains, the city-state of Urik is to fighters what Draj is to barbarians and cerulean wizards.

\textbf{Classes:} Fighter, gladiator, psychic warrior, templar.

\textbf{Skills:} Concentration, Craft (any), Knowledge (warcraft).

\textbf{Feats:} Artisan, Disciplined.

\textbf{Prestige Classes:} Dune trader, templar knight, war mind, yellow robe.
