\Chapter{Spells}
{Magic is arguably the mightiest force in Athas. Those wielding it call fire storms out of a calm sky, change one object into another, or kill enemies with a mere gesture. They dictate the wills of entire mobs, make the dead walk, and have even been known to stop time. Magic can expose traitors, destroy rivals, and exact unquestioning obedience from subjects. It can also conceal secret activities, uncover the king's spies, be used to assassinate royal officers, and foster general rebellion.}{The Wanderer's Journal}

An M or F appearing at the end of a spell's name in the spell lists denotes a spell with a material or focus component, respectively, that is not normally included in a spell component pouch. An X denotes a spell with an XP component paid by the caster.

\textbf{Order of Presentation:} In the spell lists and the spell descriptions that follow them, the spells are presented in alphabetical order by name except for those belonging to certain spell chains.

When a spell's name begins with ``lesser,'' ``greater,'' or ``mass,'' the spell description is alphabetized under the second word of the spell name instead.

\textbf{Hit Dice:} The term ``Hit Dice'' is used synonymously with ``character levels'' for effects that affect a number of Hit Dice of creatures. Creatures with Hit Dice only from their race, not from classes, have character levels equal to their Hit Dice.

\textbf{Caster Level:} A spell's power often depends on caster level, which is defined as the caster's class level for the purpose of casting a particular spell. A creature with no classes has a caster level equal to its Hit Dice unless otherwise specified. The word ``level'' in the spell lists that follow always refers to caster level.

\textbf{Creatures and Characters:} The words ``creature'' and ``character'' are used synonymously in the spell descriptions.

\section{Cleric Spells}



\subsection{0-Level Cleric Spells (Orisons)}

\spellList{Create Element}: Create a small amount of patron element.

\spellList{Cure Minor Wounds}: Cures 1 point of damage.

\spellList{Detect Magic}: Detects spells and magic items within 18 m.

\spellList{Detect Poison}: Detects poison in one creature or object.

\spellList{Guidance}: +1 on one attack roll, saving throw, or skill check.

\spellList{Inflict Minor Wounds}: Touch attack, 1 point of damage.

\spellList{Light}: Object shines like a torch.

\spellList{Mending}: Makes minor repairs on an object.

\spellList{Purify Food and Drink}: Purifies 0.3 m$^3$/level of food or water.

\spellList{Read Magic}: Read scrolls and spellbooks.

\spellList{Resistance}: Subject gains +1 on saving throws.

\spellList{Virtue}: Subject gains 1 temporary hp.



\subsection{1st-Level Cleric Spells}

\spellList{Bane}: Enemies take $-1$ on attack rolls and saves against fear.

\spellList{Bless}: Allies gain +1 on attack rolls and saves against fear.

% \spellList{Bless Water}\textsuperscript{M}: Makes holy water.
\spellList{Bless Element}\textsuperscript{M}: Makes holy element. %%

\spellList{Cause Fear}: One creature of 5 HD or less flees for 1d4 rounds.

\spellList{Command}: One subject obeys selected command for 1 round.

\spellList{Comprehend Languages}: You understand all spoken and written languages.

\spellList{Cure Light Wounds}: Cures 1d8 damage +1/level (max +5).

% \spellList{Curse Water}\textsuperscript{M}: Makes unholy water.
\spellList{Curse Element}\textsuperscript{M}: Makes unholy element. %%

\spellList{Deathwatch}: Reveals how near death subjects within 9 m are.

\spellList{Detect Chaos/Evil/Good/Law}: Reveals creatures, spells, or objects of selected alignment.

\spellList{Detect Undead}: Reveals undead within 18 m.

\spellList{Divine Favor}: You gain +1 per three levels on attack and damage rolls.

\spellList{Doom}: One subject takes $-2$ on attack rolls, saves, and checks.

\spellList{Endure Elements}: Exist comfortably in hot or cold environments.

\spellList{Entropic Shield}: Ranged attacks against you have 20\% miss chance.

\spellList{Heat Lash}: Creature suffers 1d4+1 damage and is knocked back 1.5 m. %%

\spellList{Hide from Undead}: Undead can't perceive one subject/level.

\spellList{Inflict Light Wounds}: Touch deals 1d8 damage +1/level (max +5).

% \spellList{Magic Stone}: Three stones gain +1 on attack, deal 1d6 +1 damage.

\spellList{Magic Weapon}: Weapon gains +1 bonus.

\spellList{Obscuring Mist}: Fog surrounds you.

\spellList{Protection from Chaos/Evil/Good/Law}: +2 to AC and saves, counter mind control, hedge out elementals and outsiders.

\spellList{Remove Fear}: Suppresses fear or gives +4 on saves against fear for one subject + one per four levels.

\spellList{Sanctuary}: Opponents can't attack you, and you can't attack.

\spellList{Shield of Faith}: Aura grants +2 or higher deflection bonus.

\spellList{Summon Monster I}: Calls extraplanar creature to fight for you.



\subsection{2nd-Level Cleric Spells}

\spellList{Aid}: +1 on attack rolls and saves against fear, 1d8 temporary hp +1/level (max +10).

\spellList{Align Weapon}: Weapon becomes good, evil, lawful, or chaotic.

\spellList{Augury}\textsuperscript{MF}: Learns whether an action will be good or bad.

\spellList{Bear's Endurance}: Subject gains +4 to Con for 1 min./level.

\spellList{Bull's Strength}: Subject gains +4 to Str for 1 min./level.

\spellList{Calm Emotions}: Calms creatures, negating emotion effects.

\spellList{Consecrate}\textsuperscript{M}: Fills area with positive energy, making undead weaker.

\spellList{Cure Moderate Wounds}: Cures 2d8 damage +1/level (max +10).

\spellList{Darkness}: 6-m radius of supernatural shadow.

\spellList{Death Knell}: Kills dying creature; you gain 1d8 temporary hp, +2 to Str, and +1 level.

\spellList{Delay Poison}: Stops poison from harming subject for 1 hour/level.

\spellList{Desecrate}\textsuperscript{M}: Fills area with negative energy, making undead stronger.

\spellList{Eagle's Splendor}: Subject gains +4 to Cha for 1 min./level.

\spellList{Enthrall}: Captivates all within 30 m + 3 m/level.

\spellList{Find Traps}: Notice traps as a rogue does.

\spellList{Gentle Repose}: Preserves one corpse.

\spellList{Hold Person}: Paralyzes one humanoid for 1 round/level.

\spellList{Inflict Moderate Wounds}: Touch attack, 2d8 damage +1/level (max +10).

\spellList{Make Whole}: Repairs an object.

\spellList{Owl's Wisdom}: Subject gains +4 to Wis for 1 min./level.

\spellList{Remove Paralysis}: Frees one or more creatures from paralysis or slow effect.

\spellList{Resist Energy}: Ignores 10 (or more) points of damage/attack from specified energy type.

\spellList{Restoration, Lesser}: Dispels magical ability penalty or repairs 1d4 ability damage.

\spellList{Return to the Earth}: Turns dead and undead bodies into dust. %%

\spellList{Shatter}: Sonic vibration damages objects or crystalline creatures.

\spellList{Shield Other}\textsuperscript{F}: You take half of subject's damage.

\spellList{Silence}: Negates sound in 6-m radius.

\spellList{Sound Burst}: Deals 1d8 sonic damage to subjects; may stun them.

\spellList{Spiritual Weapon}: Magic weapon attacks on its own.

\spellList{Status}: Monitors condition, position of allies.

\spellList{Summon Monster II}: Calls extraplanar creature to fight for you.

\spellList{Undetectable Alignment}: Conceals alignment for 24 hours.

\spellList{Zone of Truth}: Subjects within range cannot lie.



\subsection{3rd-Level Cleric Spells}

\spellList{Animate Dead}\textsuperscript{M}: Creates undead skeletons and zombies.

\spellList{Bestow Curse}: $-6$ to an ability score; $-4$ on attack rolls, saves, and checks; or 50\% chance of losing each action.

\spellList{Blindness/Deafness}: Makes subject blinded or deafened.

\spellList{Contagion}: Infects subject with chosen disease.

% \spellList{Continual Flame}\textsuperscript{M}: Makes a permanent, heatless torch.

% \spellList{Create Food and Water}: Feeds three humans (or one horse)/level.

\spellList{Cure Serious Wounds}: Cures 3d8 damage +1/level (max +15).

% \spellList{Daylight}: 18-m radius of bright light.

\spellList{Deeper Darkness}: Object sheds supernatural shadow in 18-m radius.

\spellList{Dispel Magic}: Cancels spells and magical effects.

\spellList{Eye of the Storm}: Protects 9-m radius from effects of storm for 1 hour/level. %%

\spellList{Glyph of Warding}\textsuperscript{M}: Inscription harms those who pass it.

\spellList{Helping Hand}: Ghostly hand leads subject to you.

\spellList{Inflict Serious Wounds}: Touch attack, 3d8 damage +1/level (max +15).

\spellList{Invisibility Purge}: Dispels invisibility within 1.5 m/level.

\spellList{Lighten Load}: Increases Strength for carrying capacity only. %%

\spellList{Locate Object}: Senses direction toward object (specific or type).

\spellList{Magic Circle against Chaos/Evil/Good/Law}: As \emph{protection} spells, but 3-m radius and 10 min./level.

\spellList{Magic Vestment}: Armor or shield gains +1 enhancement per four levels.

% \spellList{Meld into Stone}: You and your gear merge with stone.

\spellList{Obscure Object}: Masks object against scrying.

\spellList{Prayer}: Allies +1 bonus on most rolls, enemies $-1$ penalty.

\spellList{Protection from Energy}: Absorb 12 points/level of damage from one kind of energy.

\spellList{Remove Blindness/Deafness}: Cures normal or magical conditions.

\spellList{Remove Curse}: Frees object or person from curse.

\spellList{Remove Disease}: Cures all diseases affecting subject.

\spellList{Sand Pit}: Excavates sand in a 9 m wide and 15 m deep cone. %%

\spellList{Searing Light}: Ray deals 1d8/two levels damage, more against undead.

\spellList{Speak with Dead}: Corpse answers one question/two levels.

% \spellList{Stone Shape}: Sculpts stone into any shape.

\spellList{Summon Monster III}: Calls extraplanar creature to fight for you.

\spellList{Surface Walk}: Subject treads on unstable surfaces as if solid. %%

\spellList{Telepathic Bond, Lesser}: As \spell{telepathic bond}, but you and one other creature.

% \spellList{Water Breathing}: Subjects can breathe underwater.

% \spellList{Water Walk}: Subject treads on water as if solid.

% \spellList{Wind Wall}: Deflects arrows, smaller creatures, and gases.



\subsection{4th-Level Cleric Spells}

% \spellList{Air Walk}: Subject treads on air as if solid (climb at 45-degree angle).

% \spellList{Control Water}: Raises or lowers bodies of water.

\spellList{Cure Critical Wounds}: Cures 4d8 damage +1/level (max +20).

\spellList{Death Ward}: Grants immunity to death spells and negative energy effects.

\spellList{Dimensional Anchor}: Bars extradimensional movement.

\spellList{Discern Lies}: Reveals deliberate falsehoods.

\spellList{Dismissal}: Forces a creature to return to native plane.

\spellList{Divination}\textsuperscript{M}: Provides useful advice for specific proposed actions.

\spellList{Divine Power}: You gain attack bonus, +6 to Str, and 1 hp/level.

\spellList{Dweomer of Transference}: Convert spellcasting into psionic power points.

\spellList{Elemental Armor}: Armor or shield gains enhancement bonus and special quality. %%

\spellList{Elemental Weapon}: Weapon gains enhancement bonus and special quality. %%

\spellList{Freedom of Movement}: Subject moves normally despite impediments.

\spellList{Giant Vermin}: Turns centipedes, scorpions, or spiders into giant vermin.

\spellList{Imbue with Spell Ability}: Transfer spells to subject.

\spellList{Inflict Critical Wounds}: Touch attack, 4d8 damage +1/level (max +20).

\spellList{Magic Weapon, Greater}: +1 bonus/four levels (max +5).

\spellList{Neutralize Poison}: Immunizes subject against poison, detoxifies venom in or on subject.

\spellList{Planar Ally, Lesser}\textsuperscript{X}: Exchange services with a 6 HD extraplanar creature.

\spellList{Poison}: Touch deals 1d10 Con damage, repeats in 1 min.

\spellList{Repel Vermin}: Insects, spiders, and other vermin stay 3 m away.

\spellList{Restoration}\textsuperscript{M}: Restores level and ability score drains.

\spellList{Sending}: Delivers short message anywhere, instantly.

\spellList{Spell Immunity}: Subject is immune to one spell per four levels.

\spellList{Summon Monster IV}: Calls extraplanar creature to fight for you.

\spellList{Tongues}: Speak any language.



\subsection{5th-Level Cleric Spells}

\spellList{Atonement}\textsuperscript{FX}: Removes burden of misdeeds from subject.

\spellList{Break Enchantment}: Frees subjects from enchantments, alterations, curses, and petrification.

\spellList{Command, Greater}: As \spell{command}, but affects one subject/level.

\spellList{Commune}\textsuperscript{X}: Deity answers one yes-or-no question/level.

\spellList{Cure Light Wounds, Mass}: Cures 1d8 damage +1/level for many creatures.

\spellList{Dispel Chaos/Evil/Good/Law}: +4 bonus against attacks.

\spellList{Disrupting Weapon}: Melee weapon destroys undead.

% \spellList{Flame Strike}: Smite foes with divine fire (1d6/level damage).

\spellList{Hallow}\textsuperscript{M}: Designates location as holy.

\spellList{Inflict Light Wounds, Mass}: Deals 1d8 damage +1/level to many creatures.

\spellList{Insect Plague}: Locust swarms attack creatures.

\spellList{Mark of Justice}: Designates action that will trigger curse on subject.

\spellList{Plane Shift}\textsuperscript{F}: As many as eight subjects travel to another plane.

\spellList{Psychic Turmoil}: Invisible field leeches psionic power points away.

\spellList{Raise Dead}\textsuperscript{M}: Restores life to subject who died as long as one day/level ago.

\spellList{Rangeblade}: Can strike with melee weapons at a distance. %%

\spellList{Righteous Might}: Your size increases, and you gain combat bonuses.

\spellList{Scrying}\textsuperscript{F}: Spies on subject from a distance.

\spellList{Slay Living}: Touch attack kills subject.

\spellList{Spell Resistance}: Subject gains SR 12 + level.

\spellList{Summon Monster V}: Calls extraplanar creature to fight for you.

\spellList{Symbol of Pain}\textsuperscript{M}: Triggered rune wracks nearby creatures with pain.

\spellList{Symbol of Sleep}\textsuperscript{M}: Triggered rune puts nearby creatures into catatonic slumber.

\spellList{True Seeing}\textsuperscript{M}: Lets you see all things as they really are.

\spellList{Unhallow}\textsuperscript{M}: Designates location as unholy.

% \spellList{Wall of Stone}: Creates a stone wall that can be shaped.



\subsection{6th-Level Cleric Spells}

\spellList{Animate Objects}: Objects attack your foes.

\spellList{Antilife Shell}: 3-m field hedges out living creatures.

\spellList{Banishment}: Banishes 2 HD/level of extraplanar creatures.

\spellList{Bear's Endurance, Mass}: As \spell{bear's endurance}, affects one subject/ level.

\spellList{Blade Barrier}: Wall of blades deals 1d6/level damage.

\spellList{Braxatskin}: Your skin hardens, granting armor bonus and damage reduction. %%

\spellList{Bull's Strength, Mass}: As \spell{bull's strength}, affects one subject/level.

\spellList{Create Undead}: Create ghouls, ghasts, mummies, or mohrgs.

\spellList{Cure Moderate Wounds, Mass}: Cures 2d8 damage +1/level for many creatures.

\spellList{Dispel Magic, Greater}: As \spell{dispel magic}, but up to +20 on check.

\spellList{Eagle's Splendor, Mass}: As \spell{eagle's splendor}, affects one subject/level.

\spellList{Find the Path}: Shows most direct way to a location.

\spellList{Forbiddance}\textsuperscript{M}: Blocks planar travel, damages creatures of different alignment.

\spellList{Geas/Quest}: As \spell{lesser geas}, plus it affects any creature.

\spellList{Glyph of Warding, Greater}\textsuperscript{M}: As \spell{glyph of warding}, but up to 10d8 damage or 6th-level spell.

\spellList{Harm}: Deals 10 points/level damage to target.

\spellList{Heal}: Cures 10 points/level of damage, all diseases and mental conditions.

\spellList{Heroes' Feast}: Food for one creature/level cures and grants combat bonuses.

\spellList{Inflict Moderate Wounds, Mass}: Deals 2d8 damage +1/level to many creatures.

\spellList{Owl's Wisdom, Mass}: As \spell{owl's wisdom}, affects one subject/level.

\spellList{Planar Ally}\textsuperscript{X}: As \spell{lesser planar ally}, but up to 12 HD.

\spellList{Summon Monster VI}: Calls extraplanar creature to fight for you.

\spellList{Symbol of Fear}\textsuperscript{M}: Triggered rune panics nearby creatures.

\spellList{Symbol of Persuasion}\textsuperscript{M}: Triggered rune charms nearby creatures.

\spellList{Undeath to Death}\textsuperscript{M}: Destroys 1d4 HD/level undead (max 20d4).

% \spellList{Wind Walk}: You and your allies turn vaporous and travel fast.

\spellList{Word of Recall}: Teleports you back to designated place.



\subsection{7th-Level Cleric Spells}

\spellList{Blasphemy}: Kills, paralyzes, weakens, or dazes nonevil subjects.

\spellList{Control Weather}: Changes weather in local area.

\spellList{Cure Serious Wounds, Mass}: Cures 3d8 damage +1/level for many creatures.

\spellList{Destruction}\textsuperscript{F}: Kills subject and destroys remains.

\spellList{Dictum}: Kills, paralyzes, slows, or deafens nonlawful subjects.

\spellList{Elemental Chariot}: Enhances chariot with elemental effects. %%

\spellList{Ethereal Jaunt}: You become ethereal for 1 round/level.

\spellList{Holy Word}: Kills, paralyzes, blinds, or deafens nongood subjects.

\spellList{Inflict Serious Wounds, Mass}: Deals 3d8 damage +1/level to many creatures.

\spellList{Psychic Turmoil, Greater}: As \spell{psychic turmoil}, but you gain power points as temporary hp.

\spellList{Refuge}\textsuperscript{M}: Alters item to transport its possessor to you.

\spellList{Regenerate}: Subject's severed limbs grow back, cures 4d8 damage +1/level (max +35).

\spellList{Repulsion}: Creatures can't approach you.

\spellList{Restoration, Greater}\textsuperscript{X}: As \spell{restoration}, plus restores all levels and ability scores.

\spellList{Resurrection}\textsuperscript{M}: Fully restore dead subject.

\spellList{Sands of Time}\textsuperscript{F}: Reverses or accelerates aging of a non-living object. %%

\spellList{Scrying, Greater}: As \spell{scrying}, but faster and longer.

\spellList{Summon Monster VII}: Calls extraplanar creature to fight for you.

\spellList{Symbol of Stunning}\textsuperscript{M}: Triggered rune stuns nearby creatures.

\spellList{Symbol of Weakness}\textsuperscript{M}: Triggered rune weakens nearby creatures.

\spellList{Unliving Identity}\textsuperscript{MX}: Transforms a zombie into a thinking zombie. %%

\spellList{Word of Chaos}: Kills, confuses, stuns, or deafens nonchaotic subjects.



\subsection{8th-Level Cleric Spells}

\spellList{Antimagic Field}: Negates magic within 3 m.

\spellList{Brain Spider}: Listen to thoughts of up to eight other creatures.

\spellList{Cloak of Chaos}\textsuperscript{F}: +4 to AC, +4 resistance, and SR 25 against lawful spells.

\spellList{Create Greater Undead}\textsuperscript{M}: Create shadows, wraiths, spectres, or devourers.

\spellList{Cure Critical Wounds, Mass}: Cures 4d8 damage +1/level for many creatures.

\spellList{Dimensional Lock}: Teleportation and interplanar travel blocked for one day/level.

\spellList{Discern Location}: Reveals exact location of creature or object.

% \spellList{Earthquake}: Intense tremor shakes 24-m radius.

% \spellList{Fire Storm}: Deals 1d6/level fire damage.

\spellList{Holy Aura}\textsuperscript{F}: +4 to AC, +4 resistance, and SR 25 against evil spells.

\spellList{Inflict Critical Wounds, Mass}: Deals 4d8 damage +1/level to many creatures.

\spellList{Planar Ally, Greater}\textsuperscript{X}: As \spell{lesser planar ally}, but up to 18 HD.

\spellList{Shield of Law}\textsuperscript{F}: +4 to AC, +4 resistance, and SR 25 against chaotic spells.

\spellList{Spell Immunity, Greater}: As \spell{spell immunity}, but up to 8th-level spells.

\spellList{Summon Monster VIII}: Calls extraplanar creature to fight for you.

\spellList{Symbol of Death}\textsuperscript{M}: Triggered rune slays nearby creatures.

\spellList{Symbol of Insanity}\textsuperscript{M}: Triggered rune renders nearby creatures insane.

\spellList{Unholy Aura}\textsuperscript{F}: +4 to AC, +4 resistance, and SR 25 against good spells.



\subsection{9th-Level Cleric Spells}

\spellList{Astral Projection}\textsuperscript{M}: Projects you and companions onto Astral Plane.

\spellList{Elemental Chariot, Greater}: As \spell{elemental chariot}, but with greater effects. %%

\spellList{Energy Drain}: Subject gains 2d4 negative levels.

\spellList{Etherealness}: Travel to Ethereal Plane with companions.

\spellList{Gate}\textsuperscript{X}: Connects two planes for travel or summoning.

\spellList{Heal, Mass}: As \spell{heal}, but with several subjects.

\spellList{Implosion}: Kills one creature/round.

\spellList{Miracle}\textsuperscript{X}: Requests a deity's intercession.

\spellList{Soul Bind}\textsuperscript{F}: Traps newly dead soul to prevent resurrection.

% \spellList{Storm of Vengeance}: Storm rains acid, lightning, and hail.

\spellList{Summon Monster IX}: Calls extraplanar creature to fight for you.

\spellList{True Resurrection}\textsuperscript{M}: As \spell{resurrection}, plus remains aren't needed.
\section{Cleric Domains}

\Domain{Agriculture}
{Earth}
{Whenever a wizard tries to defile the ground with a spell, you can halve the defiled radius with a successful Wisdom check (DC 10 + spell level).}
{
	\item \spellList{Pass without Trace}: One subject/level leaves no tracks.
	\item \spellList{Soften Earth and Stone}: Turns stone to clay or dirt to sand or mud.
	\item \spellList{Plant Growth}: Grows vegetation, improves crops.
	\item \spellList{Wall of Stone}: Creates a stone wall that can be shaped.
	\item \spellList{Transmute Rock to Mud}: Transforms two 3 m cubes per level.
	\item \spellList{Rejuvenate}: Increase the fertility of the land.
	\item \spellList{Transmute Metal to Wood}: Metal within 12 m becomes wood.
	\item \spellList{Antipathy}: Object or location affected by spell repels certain creatures.
	\item \spellList{Heartseeker}\textsuperscript{X}: Creates a deadly piercing weapon.
}

\Domain{Air}
{Air}
{For a total time per day of 1 round per cleric level you possess, you may hold your breath regardless of any effects. You are immune to any effect or attack that requires breathing.

This effect occurs automatically as soon as it applies, lasts until it runs out or is no longer needed, and can operate multiple times per day (up to the total daily limit of rounds). These rounds do not count on the number of rounds you can hold your breath, as you are treated as if you were breathing normally but without any penalties related.}
{
	\item \spellList{Endure Elements}: Exist comfortably in hot or cold environments.
	\item \spellList{Whispering Wind}: Sends a short message 1.5 km/level.
	\item \spellList{Gust of Wind}: Blows away or knocks down smaller creatures.
	\item \spellList{Air Lens}: Directs intensified sunlight at foes within range.
	\item \spellList{Control Winds}: Change wind direction and speed.
	\item \spellList{Drown on Dry Land}: Targets can only breathe water.
	\item \spellList{Sirocco}: You conjure a legendary desert wind.
	\item \spellList{Whirlwind}: Cyclone deals damage and can pick up creatures.
	\item \spellList{Etherealness}: Travel to Ethereal Plane with companions.
}

\Domain{Cleansing}
{Fire}
{Once per day, you may step into a fire the size of a campfire. This fulfills half your daily need for food and water, cleans your body of dirt and filth, cures 1d4 points of damage, and allows for a second save to resist poison.}
{
	\item \spellList{Faerie Fire}: Outlines subjects with light, canceling blur, concealment, and the like.
	\item \spellList{Fire Trap}\textsuperscript{M}: Opened object deals 1d4 +1/level damage.
	\item \spellList{Invisibility Purge}: Dispels invisibility within 1.5 m/level.
	\item \spellList{Clues of Ash}: You receive a vision of an item's destruction.
	\item \spellList{Cleansing Flame}: 1d6/level fire damage (max 10d6).
	\item \spellList{Heroes' Feast}: Food for one creature/level cures and grants combat bonuses.
	\item \spellList{Flame Harvest}: Creates a timed fire trap.
	\item \spellList{Mind Blank}: Subject is immune to mental/emotional magic and scrying.
	\item \spellList{Blazing Wreath}: Shrouds you in elemental flame.
}

\Domain{Cycle}
{Earth}
{You leave no trail and cannot be tracked while in sand or hard earth. You may choose to leave a trail if so desired.}
{
	\item \spellList{Return to the Earth}: Turns dead and undead bodies into dust.
	\item \spellList{Rusting Grasp}: Your touch corrodes iron and alloys.
	\item \spellList{Curse of the Black Sands}: Target leaves black oily footprints.
	\item \spellList{Giant Vermin}: Turns centipedes, scorpions, or spiders into giant vermin.
	\item \spellList{Stoneskin}\textsuperscript{M}: Ignore 10 points of damage per attack.
	\item \spellList{Infestation}: Tiny parasites infect creatures within area.
	\item \spellList{Sands of Time}\textsuperscript{F}: Reverses or accelerates aging of a nonliving object.
	\item \spellList{Iron Body}: Your body becomes living iron.
	\item \spellList{Imprisonment}: Entombs subject beneath the earth.
}

\Domain{Decay}
{Silt}
{Once per day, you can control nearby silt to drown a creature unless it succeeds against a Fortitude save (DC 10 + \onehalf your cleric level + your Wis modifier). This ability only works near silt.}
{
	\item \spellList{Death Knell}: Kills dying creature; you gain 1d8 temporary hp, +2 Str and +1 level.
	\item \spellList{Rusting Grasp}: Your touch corrodes iron and alloys.
	\item \spellList{Vampiric Touch}: Touch deals 1d6/two levels damage; caster gains damage as hp.
	\item \spellList{Curse of the Choking Sands}: Victim's touch turns water to dust.
	\item \spellList{Slay Living}: Touch attack kills subject.
	\item \spellList{Drown on Dry Land}: Targets can only breathe water.
	\item \spellList{Sands of Time}\textsuperscript{F}: Reverses or accelerates aging of a nonliving object.
	\item \spellList{Horrid Wilting}: Deals 1d6/level damage within 9 m.
	\item \spellList{Vampiric Youthfulness}: Increases your lifespan at the expense of another's.
}

\Domain{Drought}
{Magma}
{Once per day, you can evaporate up to 20 liters of water per cleric level. You need to touch the water source. This ability takes one hour for each 20 liters of water evaporated.}
{
	\item \spellList{Return to the Earth}: Turns dead and undead bodies into dust.
	\item \spellList{Pyrotechnics}: Turns fire into blinding light or choking smoke.
	\item \spellList{Curse of the Black Sands}: Target leaves black oily footprints.
	\item \spellList{Clues of Ash}: You receive a vision of an item's destruction.
	\item \spellList{Blindscorch}: Fire burns the face of one opponent.
	\item \spellList{Flesh to Stone}: Turns subject creature into statue.
	\item \spellList{Statue}: Subject can become a statue at will.
	\item \spellList{Horrid Wilting}: Deals 1d6/level damage within 9 m.
	\item \spellList{Magma Tunnel}: Tunnels through solid rock.
}

\Domain{Earth}
{Earth}
{You can bury yourself beneath loose earth, sand, or top soil for up to a total of eight hours per day. If you slept eight hours buried this way, you heal double the amount of a full night's rest. You may bury yourself multiple times per day (up to the total daily limit of hours).}
{
	\item \spellList{Magic Stone}: Three stones gain +1 on attack rolls, deal 1d6+1 damage.
	\item \spellList{Meld into Stone}: You and your gear merge with stone.
	\item \spellList{Stone Shape}: Sculpts stone into any shape.
	\item \spellList{Spike Stones}: Creatures in area take 1d8 damage, may be slowed.
	\item \spellList{Stoneskin}\textsuperscript{M}: Ignore 10 points of damage per attack.
	\item \spellList{Stone Tell}: Talk to natural or worked stone.
	\item \spellList{Create Oasis}: Conjures a temporary oasis.
	\item \spellList{Earthquake}: Intense tremor shakes 24-m radius.
	\item \spellList{Elemental Swarm}\footnotemark[1]: Summons multiple elementals.
}
\noindent 1 Cast as an earth spell only.

\Domain{Fire}
{Fire}
{As a standard action, you may create a flame the size of a match stick, at a range of up to 1.5 m for each two cleric levels (minimum 1.5 m). This ability ignites materials normally. You may use this ability at will.}
{
	\item \spellList{Produce Flame}: 1d6 damage +1/ level, touch or thrown.
	\item \spellList{Pyrotechnics}: Turns fire into blinding light or choking smoke.
	\item \spellList{Continual Flame}\textsuperscript{M}: Makes a permanent, heatless torch.
	\item \spellList{Watch Fire}: Spies through campfires within range.
	\item \spellList{Wall of Fire}: Deals 2d4 fire damage out to 3 m and 1d4 out to 6 m. Passing through wall deals 2d6 damage +1/level.
	\item \spellList{Fire Seeds}: Acorns and berries become grenades and bombs.
	\item \spellList{Fire Storm}: Deals 1d6/level fire damage.
	\item \spellList{Incendiary Cloud}: Cloud deals 4d6 fire damage/round.
	\item \spellList{Meteor Swarm}: Four exploding spheres each deal 6d6 fire damage.
}

\Domain{Forecasting}
{Air}
{You gain the ranged smite power, the supernatural ability to make a single ranged attack with a +4 bonus on attack rolls and a bonus on damage rolls equal to your cleric level (if you hit). You must declare the ranged smite before making the attack. This ability is usable once per day.}
{
	\item \spellList{Feather Fall}: Objects or creatures fall slowly.
	\item \spellList{Wind Wall}: Deflects arrows, smaller creatures, and gases.
	\item \spellList{Eye of the Storm}: Protects 9-m radius from effects of storm for 1 hour/level.
	\item \spellList{Divination}\textsuperscript{M}: Provides useful advice for specific proposed actions.
	\item \spellList{True Seeing}\textsuperscript{M}: See all things as they really are.
	\item \spellList{Ragestorm}\textsuperscript{M}: Summons a storm with high winds, hail, and lightning.
	\item \spellList{Control Weather}: Changes weather in local area.
	\item \spellList{Reverse Gravity}: Objects and creatures fall upward.
	\item \spellList{Storm of Vengeance}: Storm rains acid, lightning, and hail.
}

\Domain{Freedom}
{Air}
{You gain +1 dodge AC against ranged attacks made with missile weapons.}
{
	\item \spellList{Slave Scent}: Divines target's social class.
	\item \spellList{Air Walk}: Subject treads on air as if solid (climb at 45-degree angle). 
	\item \spellList{Fly}: Subject flies at speed of 18 m.
	\item \spellList{Freedom of Movement}: Subject moves normally despite impediments.
	\item \spellList{Plane Shift}\textsuperscript{F}: As many as eight subjects travel to another plane.
	\item \spellList{Wind Walk}: You and your allies turn vaporous and travel fast.
	\item \spellList{Repulsion}: Creatures can't approach you.
	\item \spellList{Storm Legion}: Transports willing creatures via a natural storm.
	\item \spellList{Freedom}: Releases creature from \spell{imprisonment}.
}

\Domain{Growth}
{Rain}
{You move and attack normally over mudded areas.}
{
	\item \spellList{Entangle}: Plants entangle everyone in 12-m radius.
	\item \spellList{Grease}: Makes 3-m square or one object slippery.
	\item \spellList{Plant Growth}: Grows vegetation, improves crops.
	\item \spellList{Freedom of Movement}: Subject moves normally despite impediments.
	\item \spellList{Transmute Rock to Mud}: Transforms two 3 m cubes per level.
	\item \spellList{Rejuvenate}: Increase the fertility of the land.
	\item \spellList{Animate Plants}: One or more plants animate and fight for you.
	\item \spellList{Flash Flood}: Conjures a flood.
	\item \spellList{Miracle}\textsuperscript{X}: Requests a deity's intercession.
}

\Domain{Magma}
{Magma}
{You do not suffer armor check penalties and encumbrance penalties to \skill{Climb} checks.}
{
	\item \spellList{Magic Stone}: Three stones gain +1 on attack rolls, deal 1d6+1 damage.
	\item \spellList{Heat Metal}: Hot metal damages those who touch it.
	\item \spellList{Fire Shield}: Creatures attacking you take fire damage; you're protected from heat or cold.
	\item \spellList{Oil Spray}: A fountain of flammable oil gushes from the ground.
	\item \spellList{Stinking Cloud}: Nauseating vapors, 1 round/level.
	\item \spellList{Flesh to Stone}: Turns subject creature into statue.
	\item \spellList{Fire Storm}: Deals 1d6/level fire damage.
	\item \spellList{Molten}: Melt sand into glass, or rock into magma.
	\item \spellList{Fissure}\textsuperscript{M}: Opens a volcanic fissure in natural stone.
}

\Domain{Mirage}
{Sun}
{You are not affected by the natural light of the sun. You also gain +2 bonus on any save against illusion and magical blinding effects.}
{
	\item \spellList{Color Spray}: Knocks unconscious, blinds, and/or stuns weak creatures.
	\item \spellList{Hypnotic Pattern}: Fascinates (2d4 + level) HD of creatures.
	\item \spellList{Invisibility Purge}: Dispels invisibility within 1.5 m/level.
	\item \spellList{Hallucinatory Terrain}: Makes one type of terrain appear like another (field into forest, or the like).
	\item \spellList{Shining Sands}: Affected sand reflects light, blinding foes.
	\item \spellList{Mislead}: Turns you invisible and creates illusory double.
	\item \spellList{Power Word Blind}: Blinds creature with 200 hp or less.
	\item \spellList{Simulacrum}\textsuperscript{MX}: Creates partially real double of a creature.
	\item \spellList{Screen}: Illusion hides area from vision, scrying.
}

\Domain{Purity}
{Water}
{You gain +4 bonus on Fortitude saving throws against poison.}
{
	\item \spellList{Clear Water}: Doubles the benefit of water.
	\item \spellList{Surface Tension}: Affected water acquires rubbery surface.
	\item \spellList{Clear-River}: Blows aways or knocks creatures.
	\item \spellList{Freedom of Movement}: Subject moves normally despite impediments.
	\item \spellList{Coat of Mists}\textsuperscript{M}: Coalesce a magical mist about the subject's body.
	\item \spellList{Heal}: Cures 10 points/level of damage, all diseases and mental conditions.
	\item \spellList{Spell Turning}: Reflect 1d4+6 spell levels back at caster.
	\item \spellList{Antimagic Field}: Negates magic within 3 m
	\item \spellList{Heal, Mass}: As \spell{heal}, but with several subjects.
}

\Domain{Rain}
{Rain}
{You are automatically successful in all Fortitude saves against wind effects from natural weather. You are also never hit by natural lightning.}
{
	\item \spellList{Cooling Canopy}: Summons a cloud to provide shade and prevent dehydration.
	\item \spellList{Fog Cloud}: Fog obscures vision.
	\item \spellList{Call Lightning}: Calls down lightning bolts (3d6 per bolt) from sky.
	\item \spellList{Acid Rain}: Conjures an acidic shower.
	\item \spellList{Water Trap}: Body of water becomes death trap.
	\item \spellList{Call Lightning Storm}: As \spell{call lightning}, but 5d6 damage per bolt.
	\item \spellList{Control Weather}: Changes weather in local area.
	\item \spellList{Storm Legion}: Transports willing creatures via a natural storm.
	\item \spellList{Storm of Vengeance}: Storm rains acid, lightning, and hail.
}

\Domain{Replenishment}
{Water}
{You can turn a skin of water into a special healing potion, after a 1 hour ritual. Anyone who drinks from the skin will be healed 1 point of damage. The holy skin can heal up to 10 people. You can have only one holy skin at a time.

If you are 11th level or higher, you may heal a fallen ally to 1 hit point. The target must be dying or unconscious. This use spends the whole holy skin. }
{
	\item \spellList{Cooling Canopy}: Summons a cloud to provide shade and prevent dehydration.
	\item \spellList{Restoration, Lesser}: Dispels magical ability penalty or repairs 1d4 ability damage.
	\item \spellList{Create Food and Water}: Feeds three humans (or one horse)/level.
	\item \spellList{Control Tides}\footnotemark[1]: Raises, lowers, or parts bodies of water or silt.
	\item \spellList{Sweet Water}: Enhances water with life-preserving properties.
	\item \spellList{Rejuvenate}: Increase the fertility of the land.
	\item \spellList{Create Oasis}: Conjures a temporary oasis.
	\item \spellList{Flash Flood}: Conjures a flood.
	\item \spellList{Heal, Mass}: As \spell{heal}, but with several subjects.
}
\noindent 1 Cast only to control water.

\Domain{Silt}
{Silt}
{For a total time per day of 1 round per cleric level you possess, you can walk on silt as if it was solid. This effect occurs automatically as soon as it applies, lasts until it runs out or is no longer needed, and can operate multiple times per day (up to the total daily limit of rounds).}
{
	\item \spellList{Worm's Breath}\footnotemark[1]: Subjects can breathe underwater, in silt or earth.
	\item \spellList{Curse of the Black Sands}: Target leaves black oily footprints.
	\item \spellList{Sand Spray}: Sprays sand or silt as an area attack.
	\item \spellList{Whirlpool of Doom}: You stir the ground into a whirlpool.
	\item \spellList{Control Tides}\footnotemark[2]: Raises, lowers, or parts bodies of water or silt.
	\item \spellList{Gray Beckoning}: Summons zombies from the Gray.
	\item \spellList{Glass Storm}: Creates a whirlwind of broken glass.
	\item \spellList{Sirocco}: You conjure a legendary desert wind.
	\item \spellList{Storm of Vengeance}: Storm rains acid, lightning, and hail.
}
\noindent 1 Cast only to breathe in silt or earth.

\noindent 2 Cast only to control silt.

\Domain{Sun}
{Sun}
{You are automatically successful in all Fortitude saves against heat from natural weather. You are also treated as if you were continually in the shade for water consumption.}
{
	\item \spellList{Faerie Fire}: Outlines subjects with light, canceling blur, concealment, and the like.
	\item \spellList{Daylight}: 18-m radius of bright light.
	\item \spellList{Searing Light}: Ray deals 1d8/two levels, more against undead.
	\item \spellList{Air Lens}: Directs intensified sunlight at foes within range.
	\item \spellList{Flame Strike}: Smite foes with divine fire (1d6/level damage).
	\item \spellList{Sunstroke}\textsuperscript{F}: Ray attacks induce sunstroke.
	\item \spellList{Sunbeam}: Beam blinds and deals 4d6 damage.
	\item \spellList{Sunburst}: Blinds all within 3 m, deals 6d6 damage.
	\item \spellList{Prismatic Sphere}: As \spell{prismatic wall}, but surrounds on all sides.
}

\Domain{Travel}
{}
{You are always considered to be in a highway for overland movement.}
{
	\item \spellList{Longstrider}: Increases your speed.
	\item \spellList{Whispering Wind}: Sends a short message 1.5 km/level.
	\item \spellList{Phantom Steed}: Magic horse appears for 1 hour/level.
	\item \spellList{Haste}: One creature/level moves faster, +1 on attack rolls, AC, and Reflex saves.
	\item \spellList{Tree Stride}: Step from one tree to another far away.
	\item \spellList{Find the Path}: Shows most direct way to a location.
	\item \spellList{Legend Lore}\textsuperscript{MF}: Lets you learn tales about a person, place, or thing.
	\item \spellList{Discern Location}: Reveals exact location of creature or object.
	\item \spellList{Foresight}: ``Sixth sense'' warns of impending danger.
}

\Domain{Water}
{Water}
{You can extract up to half your daily needs of water from any natural source. Mud, plant, even rock.}
{
	\item \spellList{Worm's Breath}\footnotemark[1]: Subjects can breathe underwater, in silt or earth.
	\item \spellList{Water Shock}: Entraps water with electric charge.
	\item \spellList{Quench}: Extinguishes nonmagical fires or one magic item.
	\item \spellList{Lungs of Water}: Conjures water inside victim's lungs.
	\item \spellList{Water Trap}: Body of water becomes death trap.
	\item \spellList{Waters of Life}\textsuperscript{M}: Absorb another creature's ailments.
	\item \spellList{Water Light}: Water within range emits light.
	\item \spellList{Horrid Wilting}: Deals 1d6/level damage within 9 m.
	\item \spellList{Heartseeker}\textsuperscript{X}: Creates a deadly piercing weapon.
}
\noindent 1 Cast only to breathe underwater.

\Domain{Wrath}
{Fire}
{You gain the ability to ignite your melee weapon in divine flames. Wrath strike deals bonus fire damage equal to twice your cleric level (if you hit). You must declare the wrath strike before making the attack. This ability is usable once per day. This ability breaks nonmetal weapons after the strike, even if it misses.}
{
	\item \spellList{Divine Favor}: You gain attack, damage bonus, +1/three levels.
	\item \spellList{Flame Blade}: Touch attack deals 1d8 +1/two levels damage.
	\item \spellList{Heroism}: Gives +2 bonus on attack rolls, saves, skill checks.
	\item \spellList{Fire Track}: Fiery spark follows tracks.
	\item \spellList{Flame Strike}: Smite foes with divine fire (1d6/level damage).
	\item \spellList{Blade Barrier}: Wall of blades deals 1d6/level damage.
	\item \spellList{Heroism, Greater}: Gives +4 bonus on attack rolls, saves, skill checks; immunity to fear; temporary hp.
	\item \spellList{Confessor's Flame}: Uses threat of flame to extract confession.
	\item \spellList{Conflagration}: Incinerates a living creature and animates its remains.
}
\section{Druid Spells}



\subsection{0-Level Druid Spells (Orisons)}

\spellList{Create Water}: Creates 2 liters/level of pure water.

\spellList{Cure Minor Wounds}: Cures 1 point of damage.

\spellList{Defiler Scent}: Smells the presence or absence of defilers.

\spellList{Detect Magic}: Detects spells and magic items within 18 m.

\spellList{Detect Poison}: Detects poison in one creature or object.

\spellList{Flare}: Dazzles one creature (-1 penalty on attack rolls).

\spellList{Guidance}: +1 on one attack roll, saving throw, or skill check.

\spellList{Know Direction}: You discern north.

\spellList{Light}: Object shines like a torch.

\spellList{Mending}: Makes minor repairs on an object.

\spellList{Nurturing Seeds}: Makes 10 seeds or cuttings hardy and easy to transplant.

\spellList{Purify Food and Drink}: Purifies 30 cm$^3$/level of food or water.

\spellList{Read Magic}: Read scrolls and spellbooks.

\spellList{Resistance}: Subject gains +1 bonus on saving throws.

\spellList{Virtue}: Subject gains 1 temporary hp.



\subsection{1st-Level Druid Spells}

\spellList{Backlash}\textsuperscript{M}: 1d6+1 damage/2 levels to defilers.

\spellList{Black Cairn}\textsuperscript{F}: Locates a corpse nearby.

\spellList{Calm Animals}: Calms (2d4 + level) HD of animals.

\spellList{Charm Animal}: Makes one animal your friend.

\spellList{Cooling Canopy}: Summons a cloud to provide shade and prevent dehydration.

\spellList{Cure Light Wounds}: Cures 1d8 damage +1/level (max +5).

\spellList{Detect Animals or Plants}: Detects kinds of animals or plants.

\spellList{Detect Snares and Pits}: Reveals natural or primitive traps.

\spellList{Detect Undead}: Reveals undead within 60 feet.

\spellList{Endure Elements}: Exist comfortably in hot or cold environments.

\spellList{Entangle}: Plants entangle everyone in 12-m-radius.

\spellList{Faerie Fire}: Outlines subjects with light, canceling blur, concealment, and the like.

\spellList{Goodberry}: 2d4 berries each cure 1 hp (max 8 hp/24 hours).

\spellList{Hide from Animals}: Animals can't perceive one subject/level.

\spellList{Jump}: Subject gets bonus on Jump checks.

\spellList{Longstrider}: Your speed increases by 3 m.

\spellList{Magic Fang}: One natural weapon of subject creature gets +1 on attack and damage rolls.

\spellList{Magic Stone}: Three stones gain +1 on attack rolls, deal 1d6+1 damage.

\spellList{Obscuring Mist}: Fog surrounds you.

\spellList{Pass without Trace}: One subject/level leaves no tracks.

\spellList{Plant Renewal}: Brings one plant back from near destruction.

\spellList{Produce Flame}: 1d6 damage +1/level, touch or thrown.

\spellList{Proof Against Undeath}: Prevents dead subject from being raised as undead.

\spellList{Shillelagh}: Cudgel or quarterstaff becomes +1 weapon and deals damage as if two sizes larger.

\spellList{Speak with Animals}: You can communicate with animals.

\spellList{Summon Nature's Ally I}: Calls creature to fight.



\subsection{2nd-Level Druid Spells}

\spellList{Animal Messenger}: Sends a Tiny animal to a specific place.

\spellList{Animal Trance}: Fascinates 2d6 HD of animals.

\spellList{Barkskin}: Grants +2 (or higher) enhancement to natural armor.

\spellList{Bear's Endurance}: Subject gains +4 to Con for 1 min./level.

\spellList{Bull's Strength}: Subject gains +4 to Str for 1 min./level.

\spellList{Cat's Grace}: Subject gains +4 to Dex for 1 min./level.

\spellList{Chill Metal}: Cold metal damages those who touch it.

\spellList{Delay Poison}: Stops poison from harming subject for 1 hour/level.

\spellList{Fire Trap}\textsuperscript{M}: Opened object deals 1d4 +1/level damage.

\spellList{Flame Blade}: Touch attack deals 1d8 +1/two levels damage.

\spellList{Flaming Sphere}: Creates rolling ball of fire, 2d6 damage, lasts 1 round/level.

\spellList{Fog Cloud}: Fog obscures vision.

\spellList{Gust of Wind}: Blows away or knocks down smaller creatures.

\spellList{Heat Metal}: Make metal so hot it damages those who touch it.

\spellList{Hold Animal}: Paralyzes one animal for 1 round/level.

\spellList{Owl's Wisdom}: Subject gains +4 to Wis for 1 min./level.

\spellList{Reduce Animal}: Shrinks one willing animal.

\spellList{Resist Energy}: Ignores 10 (or more) points of damage/attack from specified energy type.

\spellList{Restoration, Lesser}: Dispels magical ability penalty or repairs 1d4 ability damage.

\spellList{Soften Earth and Stone}: Turns stone to clay or dirt to sand or mud.

\spellList{Spider Climb}: Grants ability to walk on walls and ceilings.

\spellList{Summon Nature's Ally II}: Calls creature to fight.

\spellList{Summon Swarm}: Summons swarm of bats, rats, or spiders.

\spellList{Tree Shape}: You look exactly like a tree for 1 hour/level.

\spellList{Warp Wood}: Bends wood (shaft, handle, door, plank).

\spellList{Wood Shape}: Rearranges wooden objects to suit you.



\subsection{3rd-Level Druid Spells}

\spellList{Boneclaw's Cut}\textsuperscript{F}: Deals damage that continues to cause bleeding damage.%%

\spellList{Call Lightning}: Calls down lightning bolts (3d6 per bolt) from sky.

\spellList{Claws of the Tembo}: Deals 1d6 Str damage and transfers hp.

\spellList{Clear‐river}: Blows aways or knocks creatures.

\spellList{Contagion}: Infects subject with chosen disease.

\spellList{Cure Moderate Wounds}: Cures 2d8 damage +1/level (max +10).

\spellList{Curse of the Black Sands}: Target leaves black oily footprints.%%

\spellList{Daylight}: 18-m radius of bright light.

\spellList{Deeper Darkness}: Object sheds absolute darkness in 18-m radius.

\spellList{Diminish Plants}: Reduces size or blights growth of normal plants.

\spellList{Dominate Animal}: Subject animal obeys silent mental commands.

\spellList{Eye of the Storm}: Protects 9-m radius from effects of storm for 1 hour/level.%%

\spellList{Magic Fang, Greater}: One natural weapon of subject creature gets +1/four levels on attack and damage rolls (max +5).

\spellList{Meld into Stone}: You and your gear merge with stone.

\spellList{Neutralize Poison}: Immunizes subject against poison, detoxifies venom in or on subject.

\spellList{Plant Growth}: Grows vegetation, improves crops.

\spellList{Poison}: Touch deals 1d10 Con damage, repeats in 1 min.

\spellList{Protection from Energy}: Absorb 12 points/level of damage from one kind of energy.

\spellList{Quench}: Extinguishes nonmagical fires or one magic item.

\spellList{Remove Curse}: Frees object or person from curse.

\spellList{Remove Disease}: Cures all diseases affecting subject.

\spellList{Return to the Earth}: Turns dead and undead bodies into dust.

\spellList{Searing Light}: Ray deals 1d8/two levels against undead.

\spellList{Sleet Storm}: Hampers vision and movement.

\spellList{Snare}: Creates a magic booby trap.

\spellList{Speak with Plants}: You can talk to normal plants and plant creatures.

\spellList{Spike Growth}: Creatures in area take 1d4 damage, may be slowed.

\spellList{Stone Shape}: Sculpts stone into any shape.

\spellList{Summon Nature's Ally III}: Calls creature to fight.

\spellList{Surface Walk}: Subject treads on unstable surfaces as if solid.

% \spellList{Water Breathing}: Subjects can breathe underwater.

\spellList{Wind Wall}: Deflects arrows, smaller creatures, and gases.

\spellList{Worm's Breath}: Subjects can breathe underwater, in silt, or earth.

\spellList{Zombie Berry}: 1d4 berries from the zombie plant become attuned to you.



\subsection{4th-Level Druid Spells}

\spellList{Air Walk}: Subject treads on air as if solid (climb at 45-degree angle).

\spellList{Antiplant Shell}: Keeps animated plants at bay.

\spellList{Blight}: Withers one plant or deals 1d6/level damage to plant creature.

\spellList{Command Plants}: Sway the actions of one or more plant creatures.

\spellList{Control Tides}: Raises, lowers, or parts bodies of water or silt.

\spellList{Cure Serious Wounds}: Cures 3d8 damage +1/level (max +15).

\spellList{Dispel Magic}: Cancels spells and magical effects.

\spellList{Flame Strike}: Smite foes with divine fire (1d6/level damage).

\spellList{Freedom of Movement}: Subject moves normally despite impediments.

\spellList{Giant Vermin}: Turns centipedes, scorpions, or spiders into giant vermin.

\spellList{Ice Storm}: Hail deals 5d6 damage in cylinder 12 m across.

\spellList{Klar's Heart}: Enhances combat abilities of all creatures within range.

\spellList{Nondetection}: Hides subject from divination, scrying.

\spellList{Reincarnate}: Brings dead subject back in a random body.

\spellList{Repel Vermin}: Insects, spiders, and other vermin stay 3 m away.

\spellList{Rusting Grasp}: Your touch corrodes iron and alloys.

\spellList{Scrying}\textsuperscript{F}: Spies on subject from a distance.

\spellList{Spike Stones}: Creatures in area take 1d8 damage, may be slowed.

\spellList{Summon Nature's Ally IV}: Calls creature to fight.



\subsection{5th-Level Druid Spells}

\spellList{Animal Growth}: One animal/two levels doubles in size.

\spellList{Atonement}: Removes burden of misdeeds from subject.

\spellList{Awaken}\textsuperscript{X}: Animal or tree gains human intellect.

\spellList{Baleful Polymorph}: Transforms subject into harmless animal.

\spellList{Braxatskin}: Your skin hardens, granting armor bonus and damage reduction.

\spellList{Call Lightning Storm}: As call lightning, but 5d6 damage per bolt.

\spellList{Cleansing Flame}: 1d6/level fire damage (max 10d6).

\spellList{Coat of Mists}\textsuperscript{M}: Coalesce a magical mist about the subject’s body.

\spellList{Commune with Nature}: Learn about terrain for 1 mile/level.

\spellList{Control Winds}: Change wind direction and speed.

\spellList{Conversion}\textsuperscript{FX}: Removes burden of acts of defiling from a wizard.

\spellList{Cure Critical Wounds}: Cures 4d8 damage +1/level (max +20).

\spellList{Death Ward}: Grants immunity to all death spells and negative energy effects.

\spellList{Groundflame}: Mist deals 1d6/level acid damage (max 15d6).

\spellList{Hallow}\textsuperscript{M}: Designates location as holy.

\spellList{Insect Plague}: Locust swarms attack creatures.

\spellList{Mark of Justice}: Designates action that will trigger curse on subject.

\spellList{Rejuvenate}: Increase the fertility of the land.

\spellList{Righteous Might}: Your size increases, and you gain +4 Str.

\spellList{Skyfire}: Three exploding spheres each deal 3d6 fire damage.

\spellList{Stoneskin}\textsuperscript{M}: Ignore 10 points of damage per attack.

\spellList{Summon Nature's Ally V}: Calls creature to fight.

\spellList{Transmute Mud to Rock}: Transforms two 3-m cubes per level.

\spellList{Transmute Rock to Mud}: Transforms two 3-m cubes per level.

\spellList{Tree Stride}: Step from one tree to another far away.

\spellList{Unhallow}\textsuperscript{M}: Designates location as unholy.

\spellList{Wall of Fire}: Deals 2d4 fire damage out to 3 m and 1d4 out to 6 m. Passing through wall deals 2d6 damage +1/level.

\spellList{Wall of Thorns}: Thorns damage anyone who tries to pass.



\subsection{6th-Level Druid Spells}

\spellList{Allegiance of the Land}: Grants bonus to AC, temporary hit points, and energy resistance.

\spellList{Antilife Shell}: 3-m radius field hedges out living creatures.

\spellList{Awaken Water Spirits}: Gives sentience to a natural body of water.

\spellList{Bear's Endurance, Mass}: As bear's endurance, affects one subject/ level.

\spellList{Bull's Strength, Mass}: As bull's strength, affects one subject/level.

\spellList{Cat's Grace, Mass}: As cat's grace, affects one subject/level.

\spellList{Create Oasis}: Conjures a temporary oasis.

\spellList{Cure Light Wounds, Mass}: Cures 1d8 damage +1/level for many creatures.

\spellList{Dispel Magic, Greater}: As dispel magic, but +20 on check.

\spellList{Find the Path}: Shows most direct way to a location.

\spellList{Fire Seeds}: Acorns and berries become grenades and bombs.

\spellList{Infestation}: Tiny parasites infect creatures within area.

\spellList{Ironwood}: Magic wood is strong as steel.

\spellList{Liveoak}: Oak becomes treant guardian.

\spellList{Move Earth}: Digs trenches and builds hills.

\spellList{Owl's Wisdom, Mass}: As owl's wisdom, affects one subject/level.

\spellList{Raise Dead}: Restores life to subject who died up to 1 day/level ago.

\spellList{Repel Wood}: Pushes away wooden objects.

\spellList{Spellstaff}: Stores one spell in wooden quarterstaff.

\spellList{Stone Tell}: Talk to natural or worked stone.

\spellList{Summon Nature's Ally VI}: Calls creature to fight.

\spellList{Transport via Plants}: Move instantly from one plant to another of the same kind.

\spellList{Wall of Stone}: Creates a stone wall that can be shaped.



\subsection{7th-Level Druid Spells}

\spellList{Animate Plants}: One or more plants animate and fight for you.

\spellList{Changestaff}: Your staff becomes a treant on command.

\spellList{Control Weather}: Changes weather in local area.

\spellList{Creeping Doom}: Swarms of centipedes attack at your command.

\spellList{Cure Moderate Wounds, Mass}: Cures 2d8 damage +1/level for many creatures.

\spellList{Fire Storm}: Deals 1d6/level fire damage.

\spellList{Heal}: Cures 10 points/level of damage, all diseases and mental conditions.

\spellList{Scrying, Greater}: As scrying, but faster and longer.

\spellList{Summon Nature's Ally VII}: Calls creature to fight.

\spellList{Sunbeam}: Beam blinds and deals 4d6 damage.

\spellList{Transmute Metal to Wood}: Metal within 12 m becomes wood.

\spellList{True Seeing}\textsuperscript{M}: Lets you see all things as they really are.

\spellList{Waters of Life}\textsuperscript{M}: Absorb another creature’s ailments.

\spellList{Wind Walk}: You and your allies turn vaporous and travel fast.



\subsection{8th-Level Druid Spells}

\spellList{Animal Shapes}: One ally/level polymorphs into chosen animal.

\spellList{Control Plants}: Control actions of one or more plant creatures.

\spellList{Cure Serious Wounds, Mass}: Cures 3d8 damage +1/level for many creatures.

\spellList{Earthquake}: Intense tremor shakes 24-m radius.

\spellList{Finger of Death}: Kills one subject.

\spellList{Flame Harvest}: Creates a timed fire trap.

\spellList{Repel Metal or Stone}: Pushes away metal and stone.

\spellList{Reverse Gravity}: Objects and creatures fall upward.

\spellList{Sirocco}: You conjure a legendary desert wind.

\spellList{Summon Nature's Ally VIII}: Calls creature to fight.

\spellList{Sunburst}: Blinds all within 3 m, deals 6d6 damage.

\spellList{Whirlwind}: Cyclone deals damage and can pick up creatures.

\spellList{Word of Recall}: Teleports you back to designated place.



\subsection{9th-Level Druid Spells}

\spellList{Antipathy}: Object or location affected by spell repels certain creatures.

\spellList{Cure Critical Wounds, Mass}: Cures 4d8 damage +1/level for many creatures.

\spellList{Elemental Swarm}: Summons multiple elementals.

\spellList{Flash Flood}: Conjures a flood.

\spellList{Foresight}: “Sixth sense” warns of impending danger.

\spellList{Heartseeker}\textsuperscript{X}: Creates a deadly piercing weapon.

\spellList{Regenerate}: Subject's severed limbs grow back, cures 4d8 damage +1/level (max +35).

\spellList{Shambler}: Summons 1d4+2 shambling mounds to fight for you.

\spellList{Shapechange}\textsuperscript{F}: Transforms you into any creature, and change forms once per round.

\spellList{Storm Legion}: Transports willing creatures via a natural storm.

\spellList{Storm of Vengeance}: Storm rains acid, lightning, and hail.

\spellList{Summon Nature's Ally IX}: Calls creature to fight.

\spellList{Swarm of Anguish}: Transforms you into a swarm of agony beetles.

\spellList{Sympathy}\textsuperscript{M}: Object or location attracts certain creatures.

\spellList{Wild Lands}: Attract wild creatures to an area.