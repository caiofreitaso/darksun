\Capitalize{E}{ach} character in {\tableheader Dark Sun} has six abilities: Strength (abbreviated Str), Dexterity (Dex), Constitution (Con), Intelligence (Int), Wisdom (Wis), and Charisma (Cha). Each of your abilities above average gives you a bonus on certain die rolls, and abilities below average give you a penalty on other die rolls.

\section{Ability Scores}

Previous editions used a rolling method that produced, on average, higher stats. This was supposed to convey that Athas was a much harsher world than normal D\&D campaign worlds, and that its denizens had adapted to compensate. However, the meaning of an attribute has changed in 3rd edition, and attributes start having a positive effect much sooner than they did in 2nd edition. Whereas many stats didn't start having a positive effect until they were at least 14, now as low as 12 have a positive effect. Using higher overall attributes for characters in {\tableheader Dark Sun} actually makes it easier for characters to survive and overcome obstacles that should be challenging, which would mean that the effective difficulty of a campaign would actually be lower using this stat generation method.

% \subsection{Creating Ability Scores}
% There are two methods for creating ability scores for your character: randomly or via point buy.  The average ability score for the typical commoner is 10 or 11, but your character is not typical. The most common ability scores for player characters (PCs) are 12 and 13.

% \subsubsection{Random Generation}
% To create an ability score for your character, roll 4d6. Disregard the lowest die roll and sum the rest. The result is a number between 3 (horrible) and 18 (tremendous)
% .
% Make this roll six times, recording each result on a piece of paper. Once you have six scores, assign each score to one of the six abilities.

% \subsubsection{Point Buy}
% All abilities scores start at 8. Take a number of points according to the type of your campaign to spread out among all abilities. For ability scores of 14 or lower, you buy additional points on a 1-for-1 basis. For ability scores higher than 14, it costs a little more.

% \Table{}{C C C C}{\tableheader Ability Score & \tableheader Point Cost & \tableheader Ability Score & \tableheader Point Cost \\
%   9 & 1 & 14 & 6 \\
%   10 & 2 & 15 & 8 \\
%   11 & 3 & 16 & 10 \\
%   12 & 4 & 17 & 13 \\
%   13 & 5 & 18 & 16}

% \Table{}{X r}{\tableheader Type of Campaign & \tableheader Points Allowed\\
%   Low-powered campaign & 15 points \\
%   Challenging campaign & 22 points \\
%   Normal campaign & 25 points \\
%   Tougher campaign & 28 points \\
%   High-powered campaign & 32 points}

\subsection{Ability Modifiers}

Each ability, after changes made because of race, has a modifier ranging from $-5$ to +5. \hyperref[tab:Ability Modifiers and Bonus Spells]{Table: Ability Modifiers and Bonus Spells} shows the modifier for each score. It also shows bonus spells, which you'll need to know about if your character is a spellcaster.

The modifier is the number you apply to the die roll when your character tries to do something related to that ability. You also use the modifier with some numbers that aren't die rolls. A positive modifier is called a bonus, and a negative modifier is called a penalty.

\BigTable{Ability Modifiers and Bonus Spells}{X c C C C C C C C C C C}{
  \hline
  \rowcolor{white}
  & & \multicolumn{10}{c}{\tableheader Bonus Spells (by spell level)} \\
  \hline
  \rowcolor{white}
  \tableheader Score & \tableheader Modifier & \tableheader 0 & \tableheader 1st & \tableheader 2nd & \tableheader 3rd & \tableheader 4th & \tableheader 5th & \tableheader 6th & \tableheader 7th & \tableheader 8th & \tableheader 9th \\
  1 & $-5$ & \cellcolor{TableColor}&\cellcolor{TableColor}&\cellcolor{TableColor}&\cellcolor{TableColor}&\cellcolor{TableColor}&\cellcolor{TableColor}&\cellcolor{TableColor}&\cellcolor{TableColor}&\cellcolor{TableColor}&\cellcolor{TableColor}\\
  2-3 & $-4$ & \cellcolor{TableColor}&\cellcolor{TableColor}&\cellcolor{TableColor}&\cellcolor{TableColor}&\cellcolor{TableColor}&\cellcolor{TableColor}&\cellcolor{TableColor}&\cellcolor{TableColor}&\cellcolor{TableColor}&\cellcolor{TableColor}\\
  4-5 & $-3$ & \cellcolor{TableColor}&\cellcolor{TableColor}&\cellcolor{TableColor}&\cellcolor{TableColor}&\cellcolor{TableColor}&\cellcolor{TableColor}&\cellcolor{TableColor}&\cellcolor{TableColor}&\cellcolor{TableColor}&\cellcolor{TableColor}\\
  6-7 & $-2$ & \cellcolor{TableColor}&\cellcolor{TableColor}&\cellcolor{TableColor}&\cellcolor{TableColor}&\cellcolor{TableColor}&\cellcolor{TableColor}&\cellcolor{TableColor}&\cellcolor{TableColor}&\cellcolor{TableColor}&\cellcolor{TableColor}\\
  8-9 & $-1$ &\multicolumn{10}{c}{\multirow{-5}{*}{\bfseries Can't cast spells tied to this ability}}\\
  10-11 & 0 & --- & --- & --- & --- & --- & --- & --- & --- & --- & --- \\
  12-13 & +1 & --- & 1 & --- & --- & --- & --- & --- & --- & --- & --- \\
  14-15 & +2 & --- & 1 & 1 & --- & --- & --- & --- & --- & --- & --- \\
  16-17 & +3 & --- & 1 & 1 & 1 & --- & --- & --- & --- & --- & --- \\
  18-19 & +4 & --- & 1 & 1 & 1 & 1 & --- & --- & --- & --- & --- \\
  20-21 & +5 & --- & 2 & 1 & 1 & 1 & 1 & --- & --- & --- & --- \\
  22-23 & +6 & --- & 2 & 2 & 1 & 1 & 1 & 1 & --- & --- & --- \\
  24-25 & +7 & --- & 2 & 2 & 2 & 1 & 1 & 1 & 1 & --- & --- \\
  26-27 & +8 & --- & 2 & 2 & 2 & 2 & 1 & 1 & 1 & 1 & --- \\
  28-29 & +9 & --- & 3 & 2 & 2 & 2 & 2 & 1 & 1 & 1 & 1 \\
  30-31 & +10 & --- & 3 & 3 & 2 & 2 & 2 & 2 & 1 & 1 & 1 \\
  32-33 & +11 & --- & 3 & 3 & 3 & 2 & 2 & 2 & 2 & 1 & 1 \\
  34-35 & +12 & --- & 3 & 3 & 3 & 3 & 2 & 2 & 2 & 2 & 1 \\
  36-37 & +13 & --- & 4 & 3 & 3 & 3 & 3 & 2 & 2 & 2 & 2 \\
  38-39 & +14 & --- & 4 & 4 & 3 & 3 & 3 & 3 & 2 & 2 & 2 \\
  40-41 & +15 & --- & 4 & 4 & 4 & 3 & 3 & 3 & 3 & 2 & 2
}

\BigTable{Ability Scores and Bonus Power Points}{l C C C C C C C C C C C C C C C C C C C C}{
  \hline
  \rowcolor{white}
  & \multicolumn{20}{c}{\tableheader Bonus Power Points (by class level)} \\
  \hline
  \rowcolor{white}
\tableheader Score & \tableheader 1st & \tableheader 2nd & \tableheader 3rd & \tableheader 4th & \tableheader 5th & \tableheader 6th & \tableheader 7th & \tableheader 8th & \tableheader 9th & \tableheader 10th & \tableheader 11th & \tableheader 12th & \tableheader 13th & \tableheader 14th & \tableheader 15th & \tableheader 16th & \tableheader 17th & \tableheader 18th & \tableheader 19th & \tableheader 20th \\
10-11 & --- & --- & --- & --- & --- & --- & --- & --- & --- & --- & --- & --- & --- & --- & --- & --- & --- & --- & --- & --- \\
12-13 & --- & 1 & 1 & 2 & 2 & 3 & 3 & 4 & 4 & 5 & 5 & 6 & 6 & 7 & 7 & 8 & 8 & 9 & 9 & 10 \\
14-15 & 1 & 2 & 3 & 4 & 5 & 6 & 7 & 8 & 9 & 10 & 11 & 12 & 13 & 14 & 15 & 16 & 17 & 18 & 19 & 20 \\
16-17 & 1 & 3 & 4 & 6 & 7 & 9 & 10 & 12 & 13 & 15 & 16 & 18 & 19 & 21 & 22 & 24 & 25 & 27 & 28 & 30 \\
18-19 & 2 & 4 & 6 & 8 & 10 & 12 & 14 & 16 & 18 & 20 & 22 & 24 & 26 & 28 & 30 & 32 & 34 & 36 & 38 & 40 \\
20-21 & 2 & 5 & 7 & 10 & 12 & 15 & 17 & 20 & 22 & 25 & 27 & 30 & 32 & 35 & 37 & 40 & 42 & 45 & 47 & 50 \\
22-23 & 3 & 6 & 9 & 12 & 15 & 18 & 21 & 24 & 27 & 30 & 33 & 36 & 39 & 42 & 45 & 48 & 51 & 54 & 57 & 60 \\
24-25 & 3 & 7 & 10 & 14 & 17 & 21 & 24 & 28 & 31 & 35 & 38 & 42 & 45 & 49 & 52 & 56 & 59 & 63 & 66 & 70 \\
26-27 & 4 & 8 & 12 & 16 & 20 & 24 & 28 & 32 & 36 & 40 & 44 & 48 & 52 & 56 & 60 & 64 & 68 & 72 & 76 & 80 \\
28-29 & 4 & 9 & 13 & 18 & 22 & 27 & 31 & 36 & 40 & 45 & 49 & 54 & 58 & 63 & 67 & 72 & 76 & 81 & 85 & 90 \\
30-31 & 5 & 10 & 15 & 20 & 25 & 30 & 35 & 40 & 45 & 50 & 55 & 60 & 65 & 70 & 75 & 80 & 85 & 90 & 95 & 100 \\
32-33 & 5 & 11 & 16 & 22 & 27 & 33 & 38 & 44 & 49 & 55 & 60 & 66 & 71 & 77 & 82 & 88 & 93 & 99 & 104 & 110 \\
34-35 & 6 & 12 & 18 & 24 & 30 & 36 & 42 & 48 & 54 & 60 & 66 & 72 & 78 & 84 & 90 & 96 & 102 & 108 & 114 & 120 \\
36-37 & 6 & 13 & 19 & 26 & 32 & 39 & 45 & 52 & 58 & 65 & 71 & 78 & 84 & 91 & 97 & 104 & 110 & 117 & 123 & 130 \\
38-39 & 7 & 14 & 21 & 28 & 35 & 42 & 49 & 56 & 63 & 70 & 77 & 84 & 91 & 98 & 105 & 112 & 119 & 126 & 133 & 140 \\
40-41 & 7 & 15 & 22 & 30 & 37 & 45 & 52 & 60 & 67 & 75 & 82 & 90 & 97 & 105 & 112 & 120 & 127 & 135 & 142 & 150}

\subsection{Abilities, Spellcasters and \hskip4em Manifesters}
The ability that governs bonus spells depends on what type of spellcaster your character is: Intelligence for wizards; Wisdom for clerics, druids, and rangers; or Charisma for templars. In addition to having a high ability score, a spellcaster must be of high enough class level to be able to cast spells of a given spell level.

Psionic classes also depend on abilities for additional power: Intelligence for psions, Wisdom for psychic warriors, and Charisma for wilders. The modifier for this ability is referred to as your key ability modifier. If your character's key ability score is 9 or lower, you can't manifest powers from that psionic class.

Just as a high Intelligence score grants bonus spells to a wizard and a high Wisdom score grants bonus spells to a cleric, a character who manifests powers (psions, psychic warriors, and wilders) gains bonus power points according to his key ability score. Refer to \hyperref[tab:Ability Scores and Bonus Power Points]{Table: Ability Scores and Bonus Power Points}.

\subsubsection{How To Determine Bonus Power Points}
Your key ability score grants you additional power points equal to your key ability modifier $\times$ your manifester level $\times$ \onehalf. \hyperref[tab:Ability Scores and Bonus Power Points]{Table: Ability Scores and Bonus Power Points} shows these calculations for class levels 1st through 20th and key ability scores from 10 to 41.

\section{The Abilities}
Each ability partially describes your character and affects some of his or her actions.

\subsection{Strength (Str)}
Strength measures your character's muscle and physical power. This ability is especially important for fighters, barbarians, paladins, rangers, and monks because it helps them prevail in combat. Strength also limits the amount of equipment your character can carry.

You apply your character's Strength modifier to:
\begin{itemize*}
\item Melee attack rolls.
\item Damage rolls when using a melee weapon or a thrown weapon (including a sling). (Exceptions: Off-hand attacks receive only one-half the character's Strength bonus, while two-handed attacks receive one and a half times the Strength bonus. A Strength penalty, but not a bonus, applies to attacks made with a bow that is not a composite bow.)
\item Climb, Jump, and Swim checks. These are the skills that have Strength as their key ability.
\item Strength checks (for breaking down doors and the like).
\end{itemize*}

\subsection{Dexterity (Dex)}
Dexterity measures hand-eye coordination, agility, reflexes, and balance. This ability is the most important one for rogues, but it's also high on the list for characters who typically wear light or medium armor (rangers and barbarians) or no armor at all (monks, wizards, and sorcerers), and for anyone who wants to be a skilled archer.

You apply your character's Dexterity modifier to:
\begin{itemize*}
\item Ranged attack rolls, including those for attacks made with bows, crossbows, throwing axes, and other ranged weapons.
\item Armor Class (AC), provided that the character can react to the attack.
\item Reflex saving throws, for avoiding fireballs and other attacks that you can escape by moving quickly.
\item Balance, Escape Artist, Hide, Move Silently, Open Lock, Ride, Sleight of Hand, Tumble, and Use Rope checks. These are the skills that have Dexterity as their key ability.
\end{itemize*}
\subsection{Constitution (Con)}
Constitution represents your character's health and stamina. A Constitution bonus increases a character's hit points, so the ability is important for all classes.

You apply your character's Constitution modifier to:
\begin{itemize*}
\item Each roll of a Hit Die (though a penalty can never drop a result below 1---that is, a character always gains at least 1 hit point each time he or she advances in level).
\item Fortitude saving throws, for resisting poison and similar threats.
\item Concentration checks. Concentration is a skill, important to spellcasters, that has Constitution as its key ability.
\end{itemize*}

If a character's Constitution score changes enough to alter his or her Constitution modifier, the character's hit points also increase or decrease accordingly.

\subsection{Intelligence (Int)}
Intelligence determines how well your character learns and reasons. This ability is important for wizards because it affects how many spells they can cast, how hard their spells are to resist, and how powerful their spells can be. It's also important for any character who wants to have a wide assortment of skills.

You apply your character's Intelligence modifier to:
\begin{itemize*}
\item The number of languages your character knows at the start of the game.
\item The number of skill points gained each level. (But your character always gets at least 1 skill point per level.)
\item Appraise, Craft, Decipher Script, Disable Device, Forgery, Knowledge, Search, and Spellcraft checks. These are the skills that have Intelligence as their key ability.
\end{itemize*}

A wizard gains bonus spells based on her Intelligence score. The minimum Intelligence score needed to cast a wizard spell is 10 + the spell's level.

An animal has an Intelligence score of 1 or 2. A creature of human-like intelligence has a score of at least 3.

\subsection{Wisdom (Wis)}
Wisdom describes a character's willpower, common sense, perception, and intuition. While Intelligence represents one's ability to analyze information, Wisdom represents being in tune with and aware of one's surroundings. Wisdom is the most important ability for clerics and druids, and it is also important for paladins and rangers. If you want your character to have acute senses, put a high score in Wisdom. Every creature has a Wisdom score.

You apply your character's Wisdom modifier to:
\begin{itemize*}
\item Will saving throws (for negating the effect of charm person and other spells).
\item Heal, Listen, Profession, Sense Motive, Spot, and Survival checks. These are the skills that have Wisdom as their key ability.
\item Clerics, druids, paladins, and rangers get bonus spells based on their Wisdom scores. The minimum Wisdom score needed to cast a cleric, druid, paladin, or ranger spell is 10 + the spell's level.
\end{itemize*}

\subsection{Charisma (Cha)}
Charisma measures a character's force of personality, persuasiveness, personal magnetism, ability to lead, and physical attractiveness. This ability represents actual strength of personality, not merely how one is perceived by others in a social setting. Charisma is most important for paladins, sorcerers, and bards. It is also important for clerics, since it affects their ability to turn undead. Every creature has a Charisma score.

You apply your character's Charisma modifier to:
\begin{itemize*}
\item Bluff, Diplomacy, Disguise, Gather Information, Handle Animal, Intimidate, Perform, and Use Magic Device checks. These are the skills that have Charisma as their key ability.
\item Checks that represent attempts to influence others.
\item Turning checks for clerics and paladins attempting to turn zombies, vampires, and other undead.
\end{itemize*}

Sorcerers and bards get bonus spells based on their Charisma scores. The minimum Charisma score needed to cast a sorcerer or bard spell is 10 + the spell's level.

\subsection{Changing Ability Scores}
When an ability score changes, all attributes associated with that score change accordingly. A character does not retroactively get additional skill points for previous levels if she increases her intelligence.
