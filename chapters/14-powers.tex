\Chapter{Powers}
{Geryas drifted in the cloudy realm of his own mind, contemplating the sun-like nexus of energy that glowed at the core of his being. He perceived his sentience as a shower of golden sparks that whirled and darted around the nexus, and the various disciplines of the Way as colorful spheres of light that danced and bobbed in the gray vapor. He selected a large scarlet orb---his psychokinetic powers---and concentrated on it.

Geryas's body sat cross-legged on a rocky shelf overlooking the Sea of Silt. His eyes stared blankly ahead. As he touched the red sphere with his consciousness. his body began to levitate, rising from the rocky crag. Without stirring from his lotus position, Geryas began to imagine flight. The golden sparks to surrounded the red sphere, exploring it.


When Geryas opened his eyes at last, he was far above the Sea of Silt, the air whistling past his ears as he flew forward. At last, he thought, \emph{telekinetic flight}.}
{}

\Capitalize{T}{his} chapter begins with the power lists for all psion's disciplines. An \textsuperscript{A} appearing at the end of a power's name in the power lists denotes an augmentable power. A \textsuperscript{P} denotes a power with prerequisites the manifester must fulfill before learning it. An \textsuperscript{X} denotes a power with an XP component paid by the manifester.

\textbf{Power Chains:} Some powers reference other powers that they are based upon. Only information in a power later in the power chain that is different from the base power is covered in the power being described. Header entries and other information that are the same as the base power are not repeated. The same holds true for powers that are the equivalents of spells, only the way the power varies from the spell is noted, such as power point cost.

\textbf{Order of Presentation:} In the power lists and the power descriptions that follow them, the powers are presented in alphabetical order by name---except for those belonging to certain power chains and those that are psionic equivalents of spells. When a power's name begins with ``lesser,'' ``greater,'' ``mass,'' or a similar kind of qualifier, the power description is alphabetized under the second word of the power description instead. When the effect of a power is essentially the same as that of a spell, the power's name is simply ``Psionic'' followed by the name of the spell, and it is alphabetized according to the spell name.

% \textbf{Manifester Level:} A power's effect often depends on the manifester level, which is the manifester's psionic class level. A creature with no classes has a manifester level equal to its Hit Dice unless otherwise specified. The word ``level'' in the power lists always refers to manifester level.
\textbf{Manifester Level:} A few powers require a specific manifester level, which is the manifester's psionic class level. A creature with no classes has a manifester level equal to its Hit Dice unless otherwise specified. The word ``level'' in the power lists always refers to manifester level.

\textbf{Creatures and Characters:} ``Creatures'' and ``characters'' are used synonymously in the power descriptions.

\textbf{Prerequisites:} Some powers require previous knowledge of another power. The manifester must fulfill all prerequisites before learning the power.

\textbf{Augment:} Many powers vary in strength depending on how many power points you put into them. The more power points you spend, the more powerful the manifestation. %However, you can spend only a total number of points on a power equal to your manifester level, unless you have an ability that increases your effective manifester level.

Many powers can be augmented in more than one way. When the Augment section contains numbered paragraphs, you need to spend power points separately for each of the numbered options. When a paragraph in the Augment section begins with ``In addition,'' you gain the indicated benefit according to how many power points you have already decided to spend on manifesting the power.

% \section{Psion Powers}



\subsection{1st-Level Psion Powers}


\subsubsection{Clairscience (Wis)}

\psionicList{Aura Reading}: Reveal personal details about the target.

\psionicList{Detect Poison, Psionic}: Detects poison in one creature or object.

\psionicList{Detect Psionics}: You detect the presence of psionics.

\psionicList{Know Direction and Location}: You discover where you are and what direction you face.

\psionicList{Precognition, Defensive}\textsuperscript{A}: Gain +1 insight bonus to AC and saving throws.

\psionicList{Precognition, Offensive}\textsuperscript{A}: Gain +1 insight bonus on your attack rolls.

\psionicList{Prescience, Offensive}\textsuperscript{A}: Gain +2 insight bonus on your damage rolls.

\psionicList{Psychic Tracking}: Track a creature using \skill{Psicraft}.

\psionicList{Trail of Destruction}: Detects recent defiling.


\subsubsection{Metacreativity (Int)}

\psionicList{Bolt}\textsuperscript{A}: You create a few enhanced short-lived bolts, arrows, or bullets.

\psionicList{Create Sound}: Create the sound you desire.

\psionicList{Crystal Shard}\textsuperscript{A}: Ranged touch attack for 1d6 points of piercing damage.

\psionicList{Ecto Protection}\textsuperscript{A}: An astral construct gains bonus against \psionic{dismiss ectoplasm}.

\psionicList{Entangling Ectoplasm}: You entangle a foe in sticky goo.

\psionicList{Ghost Writing}: Creatures writing on a distant surface or creature touched.

\psionicList{Grease, Psionic}: Makes 3-m square or one object slippery.


\subsubsection{Psychokinesis (Int)}

\psionicList{Cast Missiles}: You can launch missiles without a bow or other weapon.

\psionicList{Control Flames}\textsuperscript{A}: Take control of nearby open flame.

\psionicList{Control Light}: Adjust ambient light levels.

\psionicList{Cryokinesis}: You cool a creature or object.

\psionicList{Deflect Strike}: You psychokinetically deflect the next attack of a creature within range.

\psionicList{Energy Ray}\textsuperscript{A}: Deal 1d6 energy (cold, electricity, fire, or sonic) damage.

\psionicList{Far Hand}\textsuperscript{A}: Move small objects at a limited distance.

\psionicList{Force Screen}\textsuperscript{A}: Invisible disc provides +4 shield bonus to AC.

\psionicList{Inertial Armor}\textsuperscript{A}: Tangible field of force provides you with +4 armor bonus to AC.

\psionicList{Matter Agitation}: You heat a creature or object.

\psionicList{My Light}\textsuperscript{A}: Your eyes emit 6-m cone of light.

\psionicList{Psionic Draw}: Instantly draw a weapon.


\subsubsection{Psychometabolism (Con)}

\psionicList{Bioflexibilty}: You gain +10 competence bonus to \skill{Escape Artist} checks.

\psionicList{Cause Sleep}\textsuperscript{A}: Puts 4 HD of creatures into deep slumber.

\psionicList{Hammer}\textsuperscript{A}: Melee touch attack deals 1d8/round.

\psionicList{Photosynthesis}\textsuperscript{A}: Transform light into healing.

\psionicList{Synesthete}: You receive one kind of sense when another sense is stimulated.

\psionicList{Vigor}\textsuperscript{A}: Gain 5 temporary hit points.


\subsubsection{Psychoportation (Con)}

\psionicList{Astral Traveler}: Enable yourself or another to join an \psionic{astral caravan}-enabled trip.

\psionicList{Catfall}\textsuperscript{A}: Instantly save yourself from a fall.

\psionicList{Deceleration}\textsuperscript{A}: Target's speed is halved.

\psionicList{Dissipating Touch}\textsuperscript{A}: Touch deals 1d6 damage.

\psionicList{Float}: You buoy yourself in water or other liquid.

\psionicList{Skate}: Subject slides skillfully along the ground.

\psionicList{Wild Leap}\textsuperscript{A}: Make an additional leap and gain a bonus to \skill{Jump} checks.


\subsubsection{Telepathy (Wis)}

\psionicList{Attraction}\textsuperscript{A}: Subject has an attraction you specify.

\psionicList{Call to Mind}: Gain additional \skill{Knowledge} check with +4 competence bonus.

\psionicList{Conceal Thoughts}: You conceal your motives.

\psionicList{Daze, Psionic}\textsuperscript{A}: Humanoid creature of 4 HD or less loses next action.

\noindent\textit{\hyperref[psionic:Deja Vu]{Déjà Vu}}\textsuperscript{A}: Your target repeats his last action.

\psionicList{Demoralize}\textsuperscript{A}: Enemies become shaken.

\psionicList{Disable}\textsuperscript{A}: Subjects incorrectly believe they are disabled.

\psionicList{Distract}: Target gets $-4$ bonus on \skill{Listen}, \skill{Search}, \skill{Sense Motive}, and \skill{Spot} checks.

\psionicList{Empathy}\textsuperscript{A}: You know the subject's surface emotions.

\psionicList{Empty Mind}\textsuperscript{A}: You gain +2 on Will saves until your next action.

\psionicList{Hush}\textsuperscript{A}: Subjects become utterly silent.

\psionicList{Mind Thrust}\textsuperscript{A}: Deal 1d10 damage.

\psionicList{Missive}\textsuperscript{A}: Send a one-way telepathic message to subject.

\psionicList{Sense Link}\textsuperscript{A}: You sense what the subject senses (single sense).

\psionicList{Telempathic Projection}: Alter the subject's mood.
% 
\psionicList{Tattoo Animation}\textsuperscript{A}: Animates your tattoos or steals another's.



\subsection{2nd-Level Psion Powers}


\subsubsection{Clairscience (Wis)}

\psionicList{Detect Life}: Reveals living creatures.

\psionicList{Feat Leech}\textsuperscript{A}: Borrow another's psionic or metapsionic feats.

\psionicList{Identify, Psionic}: Learn the properties of a psionic item.

\psionicList{Recall Agony}\textsuperscript{A}: Foe takes 2d6 damage.

\psionicList{Watcher Ward}\textsuperscript{A}: You are aware of creature within the warded area.

\psionicList{Weather Prediction}: Predicts weather for next 24 hours.


\subsubsection{Metacreativity (Int)}

\psionicList{Concealing Amorpha}: Quasi-real membrane grants you concealment.

\psionicList{Swarm of Crystals}\textsuperscript{A}: Crystal shards are sprayed forth doing 3d4 slashing damage.


\subsubsection{Psychokinesis (Int)}

\psionicList{Concentrate Water}: Collects water from surrounding area.

\psionicList{Concussion Blast}\textsuperscript{A}: Deal 1d6 force damage to target.

\psionicList{Control Sound}: Create very specific sounds.

\psionicList{Energy Push}\textsuperscript{A}: Deal 2d6 damage and knock subject back.

\psionicList{Energy Stun}\textsuperscript{A}: Deal 1d6 damage and stun target if it fails both saves.

\psionicList{Molecular Bonding}: Temporarily glue two surfaces together.

\psionicList{Return Missile}\textsuperscript{A}: Make one weapon return to you after thrown.

\psionicList{Sever the Tie}\textsuperscript{A}: Disrupt an undead's tie to the Gray, damaging or destroying it.


\subsubsection{Psychometabolism (Con)}

\psionicList{Alter Self, Psionic}\textsuperscript{A}: Assume form of a similar creature.

\psionicList{Biofeedback}\textsuperscript{A}: Gain damage reduction 2/--.

\psionicList{Body Equilibrium}: You can walk on nonsolid surfaces.

\psionicList{Elfsight}: Gain low-light vision, +2 bonus on Search and Spot checks, and notice secret doors.

\psionicList{Energy Adaptation, Specified}\textsuperscript{A}: Gain resistance 10 against one energy type.

\psionicList{Pheromone Discharge}: Vermin react well to you.

\psionicList{Share Pain}: Willing subject takes some of your damage.

\psionicList{Sustenance}: Go without food and water for one day.


\subsubsection{Psychoportation (Con)}

\psionicList{Knock, Psionic}: Opens locked or psionically sealed door.

\psionicList{Levitate, Psionic}: Subject moves up and down at your direction.

\psionicList{Psionic Lock}: Secure a door, chest, or portal.


\subsubsection{Telepathy (Wis)}

\psionicList{Bestow Power}\textsuperscript{A}: Subject receives 2 power points.

\psionicList{Calm Emotions, Psionic}: Calms creatures, negating emotion effects.

\psionicList{Cloud Mind}: You erase knowledge of your presence from target's mind.

\psionicList{Detect Hostile Intent}: You can detect hostile creatures within 9 m of you.

\psionicList{Ego Whip}\textsuperscript{A}: Deal 1d4 Cha damage and daze for 1 round.

\psionicList{Id Insinuation}\textsuperscript{A}: Swift tendrils of thought disrupt and confuse your target.

\psionicList{Inflict Pain}\textsuperscript{A}: Telepathic stab gives your foe $-4$ on attack rolls, or $-2$ if he makes the save.

\psionicList{Mental Disruption}\textsuperscript{A}: Daze creatures within 3 meters for 1 round.

\psionicList{Missive, Mass}\textsuperscript{A}: You send a one-way telepathic message to an area.

\psionicList{Sense Link, Forced}: Sense what subject senses.

\psionicList{Sensory Suppression}: Victim loses one sense---sight, hearing, smell.

\psionicList{Thought Shield}\textsuperscript{A}: Gain PR 13 against mind-affecting powers.

\psionicList{Tongues, Psionic}: You can communicate with intelligent creatures.



\subsection{3rd-Level Psion Powers}


\subsubsection{Clairscience (Wis)}

\psionicList{Danger Sense}\textsuperscript{A}: You gain +4 bonus against traps.

\psionicList{Darkvision, Psionic}: See 18 m in total darkness.

\psionicList{Mental Barrier}\textsuperscript{A}: Gain +4 deflection bonus to AC until your next action.

\psionicList{Psionic Sight}\textsuperscript{A}: Psionic auras become visible to you.

\psionicList{Ubiquitous Vision}: You have all-around vision.


\subsubsection{Metacreativity (Int)}

\psionicList{Dismiss Ectoplasm}: Dissipates ectoplasmic targets and effects.

\psionicList{Energy Wall}: Create wall of your chosen energy type.

\psionicList{Keen Edge, Psionic}: Doubles normal weapon's threat range.


\subsubsection{Psychokinesis (Int)}

\psionicList{Beacon}\textsuperscript{A}: Creates a ball on light that can become much larger with concentration.

\psionicList{Dispel Psionics}\textsuperscript{A}: Cancels psionic powers and effects.

\psionicList{Energy Bolt}\textsuperscript{A}: Deal 5d6 energy damage in 36-m line.

\psionicList{Energy Burst}\textsuperscript{A}: Deal 5d6 energy damage in 12-m burst.

\psionicList{Energy Retort}\textsuperscript{A}: Ectoburst of energy automatically targets your attacker for 4d6 damage once each round.

\psionicList{Eradicate Invisibility}\textsuperscript{A}: Negate invisibility in 15-m burst.

\psionicList{Mass Manipulation}\textsuperscript{A}: Alter the weight of a creature or object.

\psionicList{Telekinetic Force}\textsuperscript{A}: Move an object with the sustained force of your mind.

\psionicList{Telekinetic Thrust}\textsuperscript{A}: Hurl objects with the force of your mind.


\subsubsection{Psychometabolism (Con)}

\psionicList{Antidote Simulation}\textsuperscript{A}: Detoxifies venom in your system.

\psionicList{Body Adjustment}\textsuperscript{A}: You heal 1d12 damage.

\psionicList{Body Purification}\textsuperscript{A}: You restore 2 points of ability damage.

\psionicList{Lighten Load, Psionic}: Increases Strength for carrying capacity only.

\psionicList{Nerve Manipulation}\textsuperscript{A}: Disrupts a creature nervous system.

\psionicList{Share Pain, Forced}\textsuperscript{A}: Unwilling subject takes some of your damage.

\psionicList{Touchsight}\textsuperscript{A}: Your telekinetic field tells you where everything is.


\subsubsection{Psychoportation (Con)}

\psionicList{Blink, Psionic}\textsuperscript{A}: You randomly vanish and reappear for 1 round/level.

\psionicList{Time Hop}\textsuperscript{A}: Subject hops forward in time 1 round/level.


\subsubsection{Telepathy (Wis)}

\psionicList{Mind Trap}\textsuperscript{A}: Drain 1d6 power points from anyone who attacks you with a telepathy power.

\psionicList{Psionic Blast}\textsuperscript{A}: Stun creatures in 9-m cone for 1 round.

\psionicList{Solicit Psicrystal}\textsuperscript{A}: Your psicrystal takes over your concentration power.



\subsection{4th-Level Psion Powers}


\subsubsection{Clairscience (Wis)}

\psionicList{Aura Sight}\textsuperscript{A}: Reveals creatures, objects, powers, or spells of selected alignment axis.

\psionicList{Detect Remote Viewing}: You know when others spy on you remotely.

\psionicList{Divination, Psionic}: Provides useful advice for specific proposed action.

\psionicList{Trace Teleport}\textsuperscript{A}: Learn destination of subject's teleport.


\subsubsection{Metacreativity (Int)}

\psionicList{Wall of Ectoplasm}: You create a protective barrier.


\subsubsection{Psychokinesis (Int)}

\psionicList{Detonate}\textsuperscript{A}: Explode one object.

\psionicList{Intellect Fortress}\textsuperscript{A}: Those inside fortress take only half damage from all powers and psi-like abilities until your next action.

\psionicList{Magnetize}\textsuperscript{A}: Make metallic object magnetic.

\psionicList{Telekinetic Maneuver}\textsuperscript{A}: Telekinetically bull rush, disarm, grapple, or trip your target.


\subsubsection{Psychometabolism (Con)}

\psionicList{Energy Adaptation}\textsuperscript{A}: Your body converts energy to harmless light.


\subsubsection{Psychoportation (Con)}

\psionicList{Dimension Door, Psionic}: Teleports you short distance.

\psionicList{Freedom of Movement, Psionic}: You cannot be held or otherwise rendered immobile.

\psionicList{Shadow Jump}\textsuperscript{A}: Jump into shadow to travel rapidly.


\subsubsection{Telepathy (Wis)}

\psionicList{Correspond}: Hold mental conversation with another creature at any distance.

\psionicList{Death Urge}\textsuperscript{A}: Implant a self-destructive compulsion.

\psionicList{Empathic Feedback}\textsuperscript{A}: When you are hit in melee, your attacker takes damage.

\psionicList{Mindwipe}\textsuperscript{A}: Subject's recent experiences wiped away, bestowing negative levels.

\psionicList{Personality Parasite}: Subject's mind calves self-antagonistic splinter personality for 1 round/level.

\psionicList{Power Leech}: Drain 1d6 power points/round while you maintain concentration; you gain 1/round.

\psionicList{Psychic Reformation}\textsuperscript{X}: Subject can choose skills, feats, and powers anew for previous levels.

\psionicList{Repugnance}: Make a creature repugnant to others.



\subsection{5th-Level Psion Powers}


\subsubsection{Clairscience (Wis)}

\psionicList{Power Resistance}: Grant PR equal to 12 + level.

\psionicList{True Seeing, Psionic}: See all things as they really are.


\subsubsection{Metacreativity (Int)}

\psionicList{Ectoplasmic Shambler}: Foglike predator deals 1 point of damage/two levels each round to an area.

\psionicList{Incarnate}\textsuperscript{X}: Make some powers permanent.

\psionicList{Major Creation, Psionic}: As \psionic{psionic minor creation}, plus stone and metal.


\subsubsection{Psychokinesis (Int)}

\psionicList{Electroerosion}\textsuperscript{A}: Create a ray that erodes iron and alloys.


\subsubsection{Psychometabolism (Con)}

\psionicList{Adapt Body}: Your body automatically adapts to hostile environments.

\psionicList{Leech Field}\textsuperscript{A}: Leech power points each time you make a saving throw.


\subsubsection{Psychoportation (Con)}

\psionicList{Plane Shift, Psionic}: Travel to other planes.


\subsubsection{Telepathy (Wis)}

\psionicList{Catapsi}\textsuperscript{A}: Psychic static inhibits power manifestation.

\psionicList{Psychic Crush}\textsuperscript{A}: Brutally crush subject's mental essence, reducing subject to $-1$ hit points.

\psionicList{Shatter Mind Blank}: Cancels target's mind blank effect.

\psionicList{Tower of Iron Will}\textsuperscript{A}: Grant PR 19 against mind-affecting powers to all creatures within 3 m until your next turn.



\subsection{6th-Level Psion Powers}


\subsubsection{Clairscience (Wis)}

\psionicList{Contingency, Psionic}\textsuperscript{X}: Sets trigger condition for another power.

\psionicList{Remote View Trap}: Deal 8d6 points electricity damage to those who seek to view you at a distance.


% \subsubsection{Metacreativity (Int)}


% \subsubsection{Psychokinesis (Int)}


\subsubsection{Psychometabolism (Con)}

\psionicList{Breath of the Black Dragon}\textsuperscript{A}: Breathe acid for 11d6 damage.

\psionicList{Fuse Flesh}\textsuperscript{A}: Fuse subject's flesh, creating a helpless mass.

\psionicList{Suspend Life}: Put yourself in a state akin to suspended animation.


\subsubsection{Psychoportation (Con)}

\psionicList{Dimensional Screen}: Create a shimmering screen that diverts attacks.

\psionicList{Disintegrate, Psionic}\textsuperscript{A}: Turn one creature or object to dust.

\psionicList{Overland Flight, Psionic}: You fly at a speed of 12 m and can hustle over long distances.

\psionicList{Retrieve}\textsuperscript{A}: Teleport to your hand an item you can see.

\psionicList{Temporal Acceleration}\textsuperscript{A}: Your time frame accelerates for 1 round.


\subsubsection{Telepathy (Wis)}

\psionicList{Aura Alteration}\textsuperscript{A}: Repairs psyche or makes subject seem to be something it is not.

\psionicList{Cloud Mind, Mass}: Erase knowledge of your presence from the minds of one creature/level.

\psionicList{Co-opt Concentration}: Take control of foe's concentration power.




\subsection{7th-Level Psion Powers}


\subsubsection{Clairscience (Wis)}

\psionicList{Moment of Prescience, Psionic}: You gain insight bonus on single attack roll, check, or save.

\psionicList{Sequester, Psionic}\textsuperscript{X}: Subject invisible to sight and \psionic{remote viewing}; renders subject comatose.


% \subsubsection{Metacreativity (Int)}


\subsubsection{Psychokinesis (Int)}

\psionicList{Energy Wave}\textsuperscript{A}: Deal 13d4 damage of your chosen energy type in 36-m cone.


\subsubsection{Psychometabolism (Con)}

\psionicList{Energy Conversion}: Offensively channel energy you've absorbed.

\psionicList{Evade Burst}\textsuperscript{A}: You take no damage from a burst on a successful Reflex save.

\psionicList{Oak Body}\textsuperscript{A}: Your body becomes as hard as oak.


\subsubsection{Psychoportation (Con)}

\psionicList{Decerebrate}: Remove portion of subject's brain stem.

\psionicList{Divert Teleport}: Choose destination for another's teleport.

\psionicList{Incorporeality}\textsuperscript{A}: You become incorporeal for 1 round/level.

\psionicList{Phase Door, Psionic}: Invisible passage through wood or stone.


\subsubsection{Telepathy (Wis)}

\psionicList{Insanity}\textsuperscript{A}: Subject is permanently confused.

\psionicList{Mind Blank, Personal}: You are immune to \spell{scrying} and mental effects.

\psionicList{Mindflame}: Kills, paralyzes, weakens, or dazes subjects.

\psionicList{Ultrablast}\textsuperscript{A}: Deal 13d6 damage in 4.5-m radius.



\subsection{8th-Level Psion Powers}


\subsubsection{Clairscience (Wis)}

\psionicList{Bend Reality}\textsuperscript{X}: Alters reality within power limits.

\psionicList{Recall Death}: Subject dies or takes 5d6 damage.


\subsubsection{Metacreativity (Int)}

\psionicList{Iron Body, Psionic}: Your body becomes living iron.

\psionicList{Matter Manipulation}\textsuperscript{X}: Increase or decrease an object's base hardness by 5.


% \subsubsection{Psychokinesis (Int)}


\subsubsection{Psychometabolism (Con)}

\psionicList{Shadow Body}: You become a living shadow (not the creature).

\psionicList{True Metabolism}: You regenerate 10 hit points/round.


\subsubsection{Psychoportation (Con)}

\psionicList{Teleport, Psionic Greater}: As \psionic{psionic teleport}, but no range limit and no off-target arrival.


\subsubsection{Telepathy (Wis)}

\psionicList{Mind Blank, Psionic}: Subject immune to mental/emotional effects, \spell{scrying}, and \psionic{remote viewing}.



\subsection{9th-Level Psion Powers}


\subsubsection{Clairscience (Wis)}

\psionicList{Reality Revision}\textsuperscript{X}: As \psionic{bend reality}, but fewer limits.


% \subsubsection{Metacreativity (Int)}


% \subsubsection{Psychokinesis (Int)}


\subsubsection{Psychometabolism (Con)}

\psionicList{Affinity Field}: Effects that affect you also affect others.

\psionicList{Assimilate}: Incorporate creature into your own body.


\subsubsection{Psychoportation (Con)}

\psionicList{Etherealness, Psionic}: Become ethereal for 1 min./level.

\psionicList{Timeless Body}: Ignore all harmful, and helpful, effects for 1 round.


\subsubsection{Telepathy (Wis)}

\psionicList{Apopsi}\textsuperscript{X}: You delete target's psionic powers.

\psionicList{Microcosm}\textsuperscript{A}: Creature or creature lives forevermore in world of his own imagination.

% \section{Psion Discipline Powers}



\subsection{Egoist Discipline Powers {\normalsize(Psychometabolism)}}
\begin{enumerate*}
\item \psionicList{Thicken Skin}\textsuperscript{A}: Gain +1 enhancement bonus to your AC for 10 min./level.
\item \psionicList{Animal Affinity}\textsuperscript{A}: Gain +4 enhancement to one ability.

\psionicList{Chameleon}: Gain +10 enhancement bonus on \skill{Hide} checks.

\psionicList{Empathic Transfer}\textsuperscript{A}: Transfer another's wounds to yourself.

\psionicList{Share Strength}\textsuperscript{A}: Temporarily transfer your Strength to another. %

\item \psionicList{Aging}\textsuperscript{A}: Make subject older. %

\psionicList{Death Field}\textsuperscript{A}: Release an energy burst from the Gray that drains vital energy. %

\psionicList{Ectoplasmic Form}: You gain benefits of being insubstantial and can fly slowly.

\psionicList{Hustle}: Instantly gain a move action.

\item \psionicList{Accelerate}\textsuperscript{A}: Move faster, +1 on attack rolls, AC, and Reflex saves. %

\psionicList{Metamorphosis}: Assume shape of creature or object.

\psionicList{Psychic Vampire}: Touch attack drains 2 power points/level from foe.

\item \psionicList{Revivify, Psionic}\textsuperscript{AX}: Return the dead to life before the psyche leaves the corpse.

\psionicList{Psychofeedback}: Boost Strength, Dexterity, or Constitution at the expense of one or more other scores.

\psionicList{Restore Extremity}: Return a lost digit, limb, or other appendage to subject.

\item \psionicList{Restoration, Psionic}: Restores level and ability score drains.
\item \psionicList{Complete Healing}\textsuperscript{A}: Heals all damage. %

\psionicList{Fission}: You briefly duplicate yourself.

\psionicList{Poison Simulation}\textsuperscript{A}: Coat surface with potent poisons. %

\item \psionicList{Fusion}\textsuperscript{X}: You combine your abilities and form with another.
\item \psionicList{Metamorphosis, Greater}\textsuperscript{X}: Assume shape of any nonunique creature or object each round.
\end{enumerate*}



\subsection{Kineticist Discipline Powers {\normalsize(Psychokinesis)}}
\begin{enumerate*}
\item \psionicList{Control Object}: Telekinetically animate a small object.
\item \psionicList{Control Air}\textsuperscript{A}: You have control over wind speed and direction.

\psionicList{Energy Missile}\textsuperscript{A}: Deal 3d6 energy damage to up to five subjects.

\item \psionicList{Energy Cone}\textsuperscript{A}: Deal 5d6 energy damage in 18-m cone.
\item \psionicList{Control Body}\textsuperscript{A}: Take rudimentary control of your foe's limbs.

\psionicList{Energy Ball}\textsuperscript{A}: Deal 7d6 energy damage in 6-m radius.

\psionicList{Inertial Barrier}: Gain DR 5/--.

\item \psionicList{Energy Current}\textsuperscript{A}: Deal 9d6 damage to one foe and half to another foe as long as you concentrate.

\psionicList{Fiery Discorporation}\textsuperscript{A}: Cheat death by discorporating into nearby fire for one day.

\item \psionicList{Dispelling Buffer}: Subject is buffered from one \psionic{dispel psionics} effect.

\psionicList{Null Psionics Field}: Create a field where psionic power does not function.

\item \psionicList{Reddopsi}: Powers targeting you rebound on manifester.
\item \psionicList{Telekinetic Sphere, Psionic}: Mobile force globe encapsulates creature and moves it.
\item \psionicList{Tornado Blast}\textsuperscript{A}: Vortex of air subjects your foes to 17d6 damage and moves them.
\end{enumerate*}



\subsection{Nomad Discipline Powers {\normalsize(Psychoportation)}}
\begin{enumerate*}
\item \psionicList{Burst}: Gain +3 m to speed this round.

\psionicList{Detect Teleportation}\textsuperscript{A}: Know when teleportation powers are used in close range.

\item \psionicList{Dimension Swap}\textsuperscript{A}: You and ally or two allies switch positions.

\psionicList{Levitate, Psionic}: Subject moves up and down at your direction.

\item \psionicList{Astral Caravan}\textsuperscript{A}: You lead \psionic{astral traveler}-enabled group to a planar destination.
\item \psionicList{Dimensional Anchor, Psionic}: Bars extra dimensional movement.

\psionicList{Dismissal, Psionic}: Forces a creature to return to its native plane.

\psionicList{Fly, Psionic}: You fly at a speed of 18 m.

\item \psionicList{Baleful Teleport}\textsuperscript{A}: Destructive teleport deals 9d6 damage.

\psionicList{Teleport, Psionic}: Instantly transports you as far as 100 miles/level.

\psionicList{Teleport Trigger}: Predetermined event triggers teleport.

\item \psionicList{Banishment, Psionic}\textsuperscript{A}: Banishes extraplanar creatures.
\item \psionicList{Dream Travel}\textsuperscript{A}: Travel to other places through dreams.

\psionicList{Ethereal Jaunt, Psionic}: Become ethereal for 1 round/level.

\psionicList{Teleport Object, Psionic}: As \psionic{psionic teleport}, but affects a touched object. %

\item \psionicList{Time Hop, Mass}\textsuperscript{A}: Willing subjects hop forward in time.
\item \psionicList{Teleportation Circle, Psionic}: Circle teleports any creatures inside to designated spot.

\psionicList{Time Regression}\textsuperscript{X}: Relive the last round.
\end{enumerate*}



\subsection{Seer Discipline Powers {\normalsize(Clairsentience)}}
\begin{enumerate*}
\item \psionicList{Destiny Dissonance}: Your dissonant touch sickens a foe.

\psionicList{Precognition}: Gain +2 insight bonus to one roll.

\item \psionicList{Clairvoyant Sense}: See and hear a distant location.

\psionicList{Locate, Psionic}\textsuperscript{A}: Indicates direction to familiar objects and creatures. %

\psionicList{Object Reading}\textsuperscript{A}: Learn details about an object's previous owner.

\psionicList{Sensitivity to Psychic Impressions}: You can find out about an area's past.

\item Detect Moisture\textsuperscript{A}: Reveals moisture within 18 m. %

\psionicList{Escape Detection}: You become difficult to detect with clairsentience powers.

\psionicList{Fate Link}\textsuperscript{A}: You link the fates of two targets.

\psionicList{Truthear}: Receive +20 insight bonus to \skill{Sense Motive} checks. %

\item \psionicList{Anchored Navigation}\textsuperscript{A}: Establish a mishap-free teleport beacon.

\psionicList{Remote Viewing}\textsuperscript{X}: See, hear, and potentially interact with 
subjects at a distance.
\item \psionicList{Clairtangent Hand}\textsuperscript{A}: Emulate \psionic{far hand} at a distance.

\psionicList{Second Chance}: Gain a reroll.

\item \psionicList{Precognition, Greater}: Gain +4 insight bonus to one roll.
\item \psionicList{Fate of One}: Reroll any roll you just failed.
\item \psionicList{Hypercognition}: You can deduce almost anything.
\item \psionicList{Cosmic Awareness}: You perceive all things in range. %

\psionicList{Metafaculty}\textsuperscript{X}: You learn details about any one creature.
\end{enumerate*}



\subsection{Shaper Discipline Powers {\normalsize(Metacreativity)}}
\begin{enumerate*}
\item \psionicList{Astral Construct}\textsuperscript{A}: Creates astral construct to fight for you.

\psionicList{Minor Creation, Psionic}: Creates one cloth or wood object.

\item \psionicList{Psionic Repair Damage}\textsuperscript{A}: Repairs construct of 3d8 hit points +1 hp/level.
\item \psionicList{Concealing Amorpha, Greater}: Quasi-real membrane grants you total concealment.

\psionicList{Ectoplasmic Cocoon}\textsuperscript{A}: You encapsulate a foe so it can't move.

\item \psionicList{Fabricate, Psionic}: Transforms raw goods to finished items.

\psionicList{Quintessence}: You collapse a bit of time into a physical substance.

\item \psionicList{Hail of Crystals}\textsuperscript{A}: A crystal explodes in an area, dealing 9d4 slashing damage.

\psionicList{Pocket Dimension}\textsuperscript{A}: Create a small storage area in an extradimensional space. %

\item \psionicList{Crystallize}: Turn subject permanently to crystal.

\psionicList{Fabricate, Greater Psionic}: Transforms a lot of raw goods to finished items.

\item \psionicList{Ectoplasmic Cocoon, Mass}\textsuperscript{A}: You encapsulate all foes in a 6-m radius.
\item \psionicList{Astral Seed}: You plant the seed of your rebirth from the Astral Plane.
\item \psionicList{Genesis}\textsuperscript{X}: You instigate a new demiplane on the Astral Plane.

\psionicList{True Creation}\textsuperscript{X}: As \psionic{psionic major creation}, except items are completely real.
\end{enumerate*}



\subsection{Telepath Discipline Powers {\normalsize(Telepathy)}}
\begin{enumerate*}
\item \psionicList{Charm, Psionic}\textsuperscript{A}: Makes one person your friend.

\psionicList{Mindlink}\textsuperscript{A}: You forge a limited mental bond with another creature.

\item \psionicList{Aversion}\textsuperscript{A}: Subject has aversion you specify.

\psionicList{Brain Lock}\textsuperscript{A}: Subject cannot move or take any mental actions.

\psionicList{Read Thoughts}: Detect surface thoughts of creatures in range.

\psionicList{Suggestion, Psionic}\textsuperscript{A}: Compels subject to follow stated course of 
action.
\item \psionicList{Crisis of Breath}\textsuperscript{A}: Disrupt subject's breathing.

\psionicList{Empathic Transfer, Hostile}\textsuperscript{A}: Your touch transfers your hurt to 
another.

\psionicList{False Sensory Input}\textsuperscript{A}: Subject sees what isn't there.

\item \psionicList{Dominate, Psionic}\textsuperscript{A}: Control target telepathically.

\psionicList{Hallucination}\textsuperscript{A}: Phantasm cause psychosomatic damage. %

\psionicList{Mindlink, Thieving}\textsuperscript{A}: Borrow knowledge of a subject's power.

\psionicList{Modify Memory, Psionic}: Changes 5 minutes of subject's memories.

\psionicList{Schism}: Your partitioned mind can manifest lower level powers.

\item \psionicList{Metaconcert}\textsuperscript{A}: Mental concert of two or more increases the total power of the participants.

\psionicList{Mind Probe}: You discover the subject's secret thoughts.

\item \psionicList{Mind Switch}\textsuperscript{AX}: You switch minds with another.
\item \psionicList{Crisis of Life}\textsuperscript{A}: Stop subject's heart.
\item \psionicList{Mind Seed}\textsuperscript{X}: Subject slowly becomes you.
\item \psionicList{Mind Switch, True}\textsuperscript{X}: A permanent brain swap.

\psionicList{Psychic Chirurgery}\textsuperscript{X}: You repair psychic damage or impart 
knowledge of new powers.
\end{enumerate*}
% \input{sections/powers/war-mind-powers.tex}
\section{Psion Powers}



\subsection{1st-Level Psion Powers}

\psionicList{Thought Shield}: Protect yourself from psionic attacks.



\subsection{2nd-Level Psion Powers}

\psionicList{Mental Barrier}: Protect yourself from psionic attacks.



\subsection{3rd-Level Psion Powers}

\psionicList{Intellect Fortress}: Protect an 3-meter-radius area from psionic attacks.



\subsection{4th-Level Psion Powers}

\psionicList{Tower of Iron Will}: Protect a 1.5-meter-square from psionic attacks.



\subsection{7th-Level Psion Powers}

\psionicList{Mind Blank, Personal}: You are immune to scrying and mental effects.

\section{Psion Discipline Powers}



\subsection{Egoist Discipline Powers {\normalsize(Psychometabolism; Con)}}
\begin{enumerate*}
\item
\item
\item
\item
\item
\item
\item
\item
\item
\end{enumerate*}



\subsection{Kineticist Discipline Powers {\normalsize(Psychokinesis; Int)}}
\begin{enumerate*}
\item \psionicList{Compact}:

      \psionicList{Control Sound}:

      \psionicList{Magnetize}\textsuperscript{A}:

      \psionicList{Soften}:

\item \psionicList{Control Flames}\textsuperscript{P}:

      \psionicList{Create Sound}\textsuperscript{P}:

      \psionicList{Telekinesis, Psionic}\textsuperscript{A}: Moves object, attacks creature, or hurls object or creature.
      
\item \psionicList{Animate Objects, Psionic}\textsuperscript{AP}: Objects attack your foes.

      \psionicList{Ballistic Attack}\textsuperscript{P}: Attack with small objects.

      \psionicList{Concentrate Water}\textsuperscript{P}:

      \psionicList{Deflect}\textsuperscript{P}:

      \psionicList{Levitate, Psionic}\textsuperscript{P}:

      \psionicList{Mass Manipulation}\textsuperscript{P}:

      \psionicList{Project Force}\textsuperscript{AP}: Punch someone from afar.

      \psionicList{Static Discharge}\textsuperscript{P}:

\item \psionicList{Control Body}\textsuperscript{P}: Take rudimentary control of your foe's limbs.

      \psionicList{Create Object}\textsuperscript{P}: Assemble matter from the surrounding area to create a solid object.

      \psionicList{Fly, Psionic}\textsuperscript{P}:

      \psionicList{Inertial Barrier}\textsuperscript{P}:

      \psionicList{Molecular Agitation}:

      \psionicList{Molecular Manipulation}\textsuperscript{P}:

\item \psionicList{Control Wind}\textsuperscript{P}:

      \psionicList{Detonate}\textsuperscript{P}: Explode small objects or destroy one construct or undead creature.

      \psionicList{Molecular Bonding}\textsuperscript{P}:

      \psionicList{Wall of Force, Psionic}\textsuperscript{P}:

\item \psionicList{Kinetic Control}:

      \psionicList{Molecular Rearrangement}\textsuperscript{P}: Change one object's fundamental properties.

\item \psionicList{Disintegrate, Psionic}\textsuperscript{AP}: Turn one creature or object to dust.

\item \psionicList{Momentum Theft}\textsuperscript{AP}:

\item \psionicList{Megakinesis}\textsuperscript{P}:
\end{enumerate*}



\subsection{Nomad Discipline Powers {\normalsize(Psychoportation; Con)}}
\begin{enumerate*}
\item
\item
\item
\item
\item \psionicList{Teleport, Psionic}: Instantly transports you as far as 90,000 km.
\item
\item
\item
\item
\end{enumerate*}



\subsection{Seer Discipline Powers {\normalsize(Clairsentience; Wis)}}
\begin{enumerate*}
\item \psionicList{Detect Poison, Psionic}: Detects poison in one creature or object.

      \psionicList{Know Direction and Location}: You discover where you are and what direction you face.

      \psionicList{Detect Remote Viewing}: You know when others spy on you remotely.

\item \psionicList{Detect Magic, Psionic}: Detects spells and magic items within 6 m.

      \psionicList{Sensitivity to Psychic Impressions}: You can find out about an area's past.

      \psionicList{Watcher's Ward}: You are aware of creatures entering warded area.

\item \psionicList{Clairaudience}: Hear at a distance.

      \psionicList{Danger Sense}\textsuperscript{A}: Gain the uncanny dodge ability.

      \psionicList{Detect Moisture}\textsuperscript{A}: Reveals moisture within 18 m.

      \psionicList{Detect Spirits}: Sense incorporeal creatures within 12 m.

      \psionicList{Environment}: Sense the environment around a known item.

      \psionicList{Synesthete}: You receive one kind of sense when another sense is stimulated.

      \psionicList{Trail of Destruction}\textsuperscript{P}: Detect past use of defiling magic in the area.

      \psionicList{Ubiquitous Vision}: You have all-around vision.

\item \psionicList{Anchored Navigation}\textsuperscript{A}: Establish a mishap-free teleport beacon.

      \psionicList{Aura Sight}: Reveal alignment, type, and hit dice of creatures in sight.

      \psionicList{Clairvoyance}: See at a distance.

      \psionicList{Combat Mind}\textsuperscript{A}: You improve your allies' initiative count by 1.

      \psionicList{Weather Prediction}\textsuperscript{P}: Predicts weather for next 24 hours.

\item \psionicList{Object Reading}: Learn details about an object's previous owner.

      \psionicList{Safe Path}\textsuperscript{AP}: Gain +2 AC bonus, +4 bonus in saving throws and some skill checks.

      \psionicList{Second Chance}: Gain a reroll. %XPH

\item \psionicList{Bone Reading}\textsuperscript{P}: Learn details about a deceased creature.

      \psionicList{Precognition}: Forsee some hours into your future to give you an edge.

\item \psionicList{Fate of One}\textsuperscript{P}: Reroll any roll you just failed. %XPH

      \psionicList{Predestination}\textsuperscript{AP}: Predict the destiny of a single creature.

\item \psionicList{Spirit Lore}\textsuperscript{P}: Commune with spirits to ask one question/round.

      \psionicList{True Seeing, Psionic}\textsuperscript{P}: Lets you see all things as they really are.

\item \psionicList{Cosmic Awareness}\textsuperscript{AP}: You perceive all things in range.
\end{enumerate*}



\subsection{Shaper Discipline Powers {\normalsize(Metacreativity; Int)}}
\begin{enumerate*}
\item \psionicList{Heighten Senses}\textsuperscript{AP}: You gain scent and +6 enhancement bonus in \skill{Spot} and \skill{Listen}.

\item \psionicList{Detect Psionics}\textsuperscript{P}: You detect the presence of psionics.

      \psionicList{Teleport Trigger}\textsuperscript{P}: Predetermined event triggers \psionic{psionic teleport}.

\item \psionicList{Trace Teleport}\textsuperscript{AP}: Learn destination of subject's \emph{teleport}.

\item
\item \psionicList{Astral Construct}\textsuperscript{AP}: Creates astral construct to fight for you.

      \psionicList{Metaconcert}\textsuperscript{AP}: Mental concert of two or more increases the total power of the participants.

\item \psionicList{Null Psionics Field}\textsuperscript{P}: Create a field where psionic power does not function.
\item \psionicList{Psychic Chirurgery}\textsuperscript{PX}: You repair psychic damage or impart knowledge of new powers.

      \psionicList{Reddopsi}\textsuperscript{P}: Powers targeting you rebound on manifester.

\item \psionicList{Bend Reality}\textsuperscript{PX}: Alters reality within power limits.

      \psionicList{Schism}\textsuperscript{P}: Your partitioned mind can manifest lower level powers.

\item \psionicList{Apopsi}\textsuperscript{PX}: You delete target's psionic powers.

      \psionicList{Reality Revision}\textsuperscript{PX}: As \psionic{bend reality}, but fewer limits.
\end{enumerate*}



\subsection{Telepath Discipline Powers {\normalsize(Telepathy; Wis)}}
\begin{enumerate*}
\item
\item
\item
\item
\item
\item
\item
\item
\item
\end{enumerate*}


\clearpage
\section{Power Descriptions}
The powers herein are presented in alphabetical order (with the exception of those whose names begin with ``greater,'' ``lesser,'' or ``mass''; see Order of Presentation).

\Psionic{Anchored Navigation}{anchored navigation}
{Clairsentience}
{
	\textbf{Level:}
	Psion 4\\
	\textbf{Casting Time:}
	1 mental action\\
	\textbf{Range:}
	Personal\\
	\textbf{Target:}
	You\\
	\textbf{Cost:}
	7 power points\\
	\textbf{Maintenance Cost:}
	4 pp/hour\\
	% \textbf{Critical Success:}
	% You can also retrace your steps through a maze automatically while the power lasts, without resorting to a map.\\
	% \textbf{Critical Failure:}
	% You forget where you are for 1d4 rounds\\
}
{
	You know where you are in relation to a fixed starting point, which is essential for setting up a mishap-free teleport beacon. While the duration lasts, you are aware of your exact distance and route (physical or psychoportive) back to a fixed starting point. The “anchored” starting point is your exact location when you manifest the power. To designate other anchored starting points, you must manifest this power multiple times and be present at the desired locations when you do so.

	While maintaining this power, you can make a power check to retrace your steps through a maze. If you fail the check, you are not sure which way you came.
}
\Psionic{Aura Sight}{aura sight}
{Clairsentience}
{
	\textbf{Level:}
	Seer 4\\
	\textbf{Manifesting Time:}
	1 mental action\\
	\textbf{Range:}
	9 meters\\
	\textbf{Area:}
	Cone-shaped emanation centered on you\\
	\textbf{Saving Throw:}
	None\\
	\textbf{Cost:}
	20 power points\\
	\textbf{Maintenance Cost:}
	4 pp/minute\\
	% \textbf{Critical Success:}
	% You can examine up to four auras per round\\
	% \textbf{Critical Failure:}
	% You can't use this power for 24 hours\\
}
{
	You discern auras. Auras are invisible to the naked eye, but to a psionic viewer manifesting this power they appear as glowing halos or envelopes of colored light that surround all objects. The color of each aura reveals information to the psionic character. A creature's aura reveals information about its alignment, its Hit Dice, and its type.

	During each round, you can examine up to two auras. Examining an aura is a free action and requires a power check. Each check reveals one information about the creature: one of its alignment axis (i.e., only chaotic from its chaotic neutral alignment), its type, or its Hit Dice. Failure in this check doesn't give you any information.

	Creatures with many Hit Dice require extra effort to examine:

	\Table{}{CC}{
	\tableheader Hit Dice & \tableheader Power Check DC Modifier \\
	 1--5  & +0 \\
	 6--11 & +1 \\
	12--17 & +2 \\
	18--20 & +3 \\
	21+    & +4 \\
	}

	\textit{Augment:} For every 5 additional power points you spend, this power's range increases by 3 meters and the Difficulty Class of the power check by +1.
}
\Psionic{Bone Reading}{bone reading}
{Clairsentience}
{
	\textbf{Level:}
	Seer 6\\
	\textbf{Manifesting Time:}
	1 minute\\
	\textbf{Range:}
	Touch\\
	\textbf{Target:}
	A creature's remains touched\\
	\textbf{Cost:}
	11 power points\\
	\textbf{Maintenance Cost:}
	6 pp/minute\\
	\textbf{Prerequisites:}
	\psionic{object reading}\\
	% \textbf{Critical Success:}
	% You are able to view the last 10 minutes of the deceased's life from their perspective\\
	% \textbf{Critical Failure:}
	% You anger the spirits and a malicious entity that attempts to use \spell{magic jar} on you\\
}
{
	You can use a creature's remains to learn who they were and what they were doing when they perished. Fragments of bone are usually used, but the power works on any corpse or portion of a corpse.

	The amount of information revealed depends on how long you study the remains. You can maintain this power for one additional minute per point above the power check DC on your power check. You may stop maintaining this power before you reach your time limit.

	\textit{1st Minute:} Deceased's race.

	\textit{2nd Minute:} Deceased's gender.

	\textit{3rd Minute:} Deceased's age.

	\textit{4th Minute:} Deceased's identity.

	\textit{5th Minute:} Deceased's alignment.

	\textit{6th Minute:} Appearance in life.

	\textit{7th Minute:} Date of death.

	\textit{8th Minute:} Method of death.

	The power is difficult to use on very old remains. The more recently a creature died, the more accurate the reading will be.

	\Table{}{Cc}{
	\tableheader Time Since Death & \tableheader Power Check DC Modifier \\
	Up to 1 day       & +0  \\
	Up to 1 week      & +1  \\
	Up to 1 month     & +2  \\
	Up to 1 year      & +3  \\
	Up to 10 years    & +4  \\
	Up to 100 years   & +6  \\
	Up to 1,000 years & +12 \\
	}
}
\Psionic{Clairaudience}{clairaudience}
{Clairsentience (Scrying)}
{
	\textbf{Level:}
	Seer 3\\
	\textbf{Manifesting Time:}
	1 mental action\\
	\textbf{Range:}
	See text\\
	\textbf{Effect:}
	Psionic sensor\\
	\textbf{Saving Throw:}
	None\\
	\textbf{Cost:}
	15 power points\\
	\textbf{Maintenance Cost:}
	3 pp/round\\
	% \textbf{Critical Success:}
	% You also gain the effect of \psionic{clairvoyance} for the duration of this power\\
	% \textbf{Critical Failure:}
	% You become deaf for 1d12 hours\\
}
{
	You can hear a distant location almost as if you were there. You don’t need line of sight or line of effect, but the locale must be known---a place familiar to you or an obvious one, such as behind a door, around a corner, or in a grove of trees. Once you have selected the locale, the focus of your clairvoyant sense doesn’t move, but you can rotate it in all directions to view the area as desired. Unlike other scrying powers, this power does not allow psionically or supernaturally enhanced senses to work through it.

	The farther the ``listening spot'' is from you, the more difficult it is to use this power. This power does not work across planes.

	\Table{}{CC}{
	\tableheader Range & \tableheader Power Check DC Modifier \\
	90 meters        & +0 \\
	900 meters       & +1 \\
	9 kilometers     & +2 \\
	90 kilometers    & +3 \\
	900 kilometers   & +4 \\
	9,000 kilometers & +5 \\
	}
}
\Psionic{Clairvoyance}{clairvoyance}
{Clairsentience (Scrying)}
{
	\textbf{Level:}
	Psion 4\\
	\textbf{Casting Time:}
	1 mental action\\
	\textbf{Range:}
	See text\\
	\textbf{Effect:}
	Psionic sensor\\
	\textbf{Saving Throw:}
	None\\
	\textbf{Cost:}
	7 power points\\
	\textbf{Maintenance Cost:}
	4 pp/round\\
	% \textbf{Critical Success:}
	% Manifester also gains \psionic{clairaudience}\\
	% \textbf{Critical Failure:}
	% Manifester becomes blind for 1d4 hours\\
}
{
	You can see a distant location almost as if you were there. You don't need line of sight or line of effect, but the locale must be known---a place familiar to you or an obvious one, such as behind a door, around a corner, or in a grove of trees. Once you have selected the locale, the focus of your clairvoyant sense doesn't move, but you can rotate it in all directions to view the area as desired. Unlike other scrying powers, this power does not allow psionically or supernaturally enhanced senses to work through it.

	The more distant the viewed area is, the more difficult it is to use \emph{clairvoyance}.

	\Table{}{CC}{
	\tableheader Range & \tableheader Power Check DC Modifier \\
	90 meters        & +0 \\
	900 meters       & +1 \\
	9 kilometers     & +2 \\
	90 kilometers    & +3 \\
	900 kilometers   & +4 \\
	9,000 kilometers & +5 \\
	}

	If the chosen locale is magically or psionically dark, you see nothing. If it is naturally pitch black, you can see in a 3-meter radius around the center of the power's effect or out to the extent of your natural darkvision. The power does not work across planes. 
}
\Psionic{Combat Mind}{combat mind}
{Clairsentience}
{
	\textbf{Level:}
	Seer 4\\
	\textbf{Casting Time:}
	1 mental action\\
	\textbf{Range:}
	Personal\\
	\textbf{Target:}
	You\\
	\textbf{Saving Throw:}
	Will negates (harmless)\\
	\textbf{Cost:}
	7 power points\\
	\textbf{Maintenance Cost:}
	4 pp/round\\
	% \textbf{Critical Success:}
	% You gain +1 bonus on Armor Class\\
	% \textbf{Critical Failure:}
	% You and your allies have $-1$ penalty on initiative\\
}
{
	You can improve all targets' initiative count for the next round and all subsequent rounds you maintain this power. When you manifest this power, their initiative count improves by 1, and their place in the initiative order changes accordingly. This modifier applies at the end of the round. Their place in the initiative order changes to reflect \emph{combat mind}'s effect starting with the next round.

	\textit{Augment:} You can augment this power in one or both of the following ways.
	\begin{enumerate*}
		\item For every 10 additional power points you spend, this power can affect an additional target. Any additional target cannot be more than 6 meters from another target of the power.
		\item For every 3 additional power points you spend, the initiative count improvement increases by 1.
	\end{enumerate*}
}
\Psionic{Cosmic Awareness}{cosmic awareness}
{Clairsentience}
{
	\textbf{Level:}
	Psion 9\\
	\textbf{Casting Time:}
	1 minute\\
	\textbf{Range:}
	Personal\\
	\textbf{Target:}
	You\\
	\textbf{Cost:}
	17 power points\\
	\textbf{Maintenance Cost:}
	9 pp/round\\
	\textbf{Prerequisites:}
	Manifester level 11th, \psionic{clairvoyance}, \psionic{psionic true seeing}, \psionic{ubiquitous vision}\\
	% \textbf{Critical Success:}
	% You begin with a sensorial radius of 9 meters\\
	% \textbf{Critical Failure:}
	% You must make a Will check or become blind and deaf for 1d4 hours\\
}
{
	You expand your senses in a sensorial 3-meter-radius sphere that extends into the Astral and Ethereal Planes. You see all inanimate features within the area of effect; you can see what is on the other side of the hill, detect hidden caves, see secret doors and traps, and even detect lodes of unusual minerals or other geological phenomena. You also perceive the flow of the wind and water currents.

	You learn about the location and alignment of every creature within range, even if they are hiding, invisible, or in the Astral or Ethereal Planes. Their equipment is detected down to the number of gold pieces in their purses.

	You hear all sounds and detect all smells within the area.

	You also detect magic within range as the \spell{detect magic} spell, but without needing to study the area. You learn about the auras and their strength instantly.

	Otherwise, this power functions as the \spell{true seeing} spell, as if everything within range was in your line of sight.

	\textit{Augment:} For every additional power point you spend, the radius of your \emph{cosmic awareness} field increases by 3 meters.
}
\Psionic{Danger Sense}{danger sense}
{Clairsentience}
{
	\textbf{Level:}
	Psion 3\\
	\textbf{Casting Time:}
	1 mental action\\
	\textbf{Range:}
	Personal\\
	\textbf{Target:}
	You\\
	\textbf{Cost:}
	5 power points\\
	\textbf{Maintenance Cost:}
	3 pp/round\\
	% \textbf{Critical Success:}
	% You can't be surprised\\
	% \textbf{Critical Failure:}
	% You can't use this power for 1d6 hours\\
}
{
	You can sense the presence of danger before your senses would normally allow it. This power gives you the uncanny dodge ability.

	\textit{Augment:} If you spend 3 additional power points, this power also gives you the improved uncanny dodge ability.
}
\Psionic{Detect Magic, Psionic}{psionic detect magic}
{Clairsentience}
{
	\textbf{Level:}
	Seer 2\\
	\textbf{Manifesting Time:}
	1 mental action\\
	\textbf{Range:}
	6 meters\\
	\textbf{Area:}
	6-meter cone-shaped emanation, centered on you\\
	\textbf{Cost:}
	10 power points\\
	\textbf{Maintenance Cost:}
	2 pp/round\\
	% \textbf{Critical Success:}
	% You automatically pass in Spellcraft checks to determine the schools\\
	% \textbf{Critical Failure:}
	% You mistakenly believe a random item is highly magical\\
}
{
	As the \spell{detect magic} spell, except as noted here.

	If you succeed in your power check by 5 or more, you can see the potential of arcane users depending on their remaining relative number of spell slots.

	\Table{Psionic Detect Magic}{lCCC}{
	& \multicolumn{3}{c}{\tableheader Aura Power}\\
	\cmidrule[0.5pt]{2-4}
	& \tableheader Faint & \tableheader Moderate & \tableheader Strong\\
	Relative remaining slots & 01--33\% & 34--66\% & 67--100\% \\
	}
}
\Psionic{Detect Moisture}{detect moisture}
{Clairsentience}
{
	\textbf{Level:}
	Psion 3\\
	\textbf{Casting Time:}
	1 mental action\\
	\textbf{Range:}
	15 meters\\
	\textbf{Effect:}
	15-m-radius emanation, centered on you\\
	\textbf{Cost:}
	5 power point\\
	\textbf{Maintenance Cost:}
	3 pp/round\\
	% \textbf{Critical Success:}
	% You receive one minute of observation\\
	% \textbf{Critical Failure:}
	% You get images from a similar but different item\\
}
{
	You can feel the presence of water. You detect all greater than Tiny that have moisture, plus any concentration of one gallon or more. The amount of information revealed depends on how long you search a particular area.

	\textit{1st Round:} Presence or absence of moisture in the area.

	\textit{2nd Round:} Amount of moisture in the area.

	\textit{3rd Round:} The location of each individual with moisture present and all others sources of moisture. If a moisture concentration is outside your line of sight, then you discern its direction but not its exact location.

	The power can penetrate barriers, but 30 centimeters of stone, 2.5 centimeters of common metal, a thin sheet of lead, or 1 meter of wood blocks it. Note that dirt does not block the power.

	\textit{Augment:} For every additional power point you spend, this power's range increases by 15 meters, up to 90 meters.
}
\Psionic{Detect Poison, Psionic}{psionic detect poison}
{Clairsentience}
{
	\textbf{Level:}
	Seer 1\\
	\textbf{Manifesting Time:}
	1 mental action\\
	\textbf{Range:}
	1.5 meters\\
	\textbf{Cost:}
	5 power points\\
	\textbf{Maintenance Cost:}
	None (Instantaneous)\\
	% \textbf{Critical Success:}
	% You know the exact type of poison\\
	% \textbf{Critical Failure:}
	% You feel the effects of the poison, as if you were afflicted by it\\
}
{
	As the \spell{detect poison} spell, except as noted here.
}
\Psionic{Detect Psionics}{detect psionics}
{Metacreativity}
{
	\textbf{Level:}
	Shaper 2\\
	\textbf{Manifesting Time:}
	1 mental action\\
	\textbf{Range:}
	90 meters\\
	\textbf{Effect:}
	90-m-radius emanation, centered on you\\
	\textbf{Cost:}
	3 power points\\
	\textbf{Maintenance Cost:}
	2 pp/round\\
	\textbf{Prerequisites:}
	\psionic{mindlink}, \psionic{watcher's ward}\\
	% \textbf{Critical Success:}
	% You learn information of the second success on your first success\\
	% \textbf{Critical Failure:}
	% You can't use this power for one round\\
}
{
	You can detect psionic activity anywhere within range. Any expenditure of power points constitutes psionic activity, even if it only to maintain a power.

	During each round, you may attempt a power check. The amount of information you gather each round depends on the number of successes:

	\textit{1st success:} You detect if there is any psionic activity besides your own.

	\textit{2nd success:} You know where the sources of psionic activity are (direction and distance).

	\textit{3rd success:} You know how many power points are being spent by each source.
}
\Psionic{Detect Remote Viewing}{detect remote viewing}
{Clairsentience}
{
	\textbf{Level:}
	Seer 1\\
	\textbf{Manifesting Time:}
	1 mental action\\
	\textbf{Range:}
	Personal\\
	\textbf{Target:}
	You\\
	\textbf{Cost:}
	1 power point\\
	\textbf{Maintenance Cost:}
	1 pp/hour\\
	% \textbf{Critical Success:}
	% You know the exact position of your watcher\\
	% \textbf{Critical Failure:}
	% You become certain everyone is watching you for 1d3 days\\
}
{
	You gain an sixth sense to determine whenever you are being observed. Whenever a character is actively using \skill{Spot} checks on you or whenever there is an attempt to observe you by any scrying effect, you may roll a power check. A successful check does not reveal the location of the watcher, it only confirms your suspicions that you are being watched.

	If the observer is using a psionic power to monitor you, you may attempt a psionic contest to obscure their scrying. If you win the contest, the observer can't use any scrying psionic power on you for 1d4 hours.
}
\Psionic{Detect Spirits}{detect spirits}
{Clairsentience}
{
	\textbf{Level:}
	Psion 3\\
	\textbf{Casting Time:}
	1 mental action\\
	\textbf{Range:}
	12 meters\\
	\textbf{Area:}
	12-meter-radius emanation, centered on you\\
	\textbf{Cost:}
	5 power points\\
	\textbf{Maintenance Cost:}
	3 pp/round\\
	% \textbf{Critical Success:}
	% You know the exact type of poison\\
	% \textbf{Critical Failure:}
	% You feel the effects of the poison, as if you were afflicted by it\\
}
{
	As the \spell{detect undead} spell, except as noted here.

	\emph{Detect spirits} can only detect incorporeal creatures, instead of undead.
}
\Psionic{Environment}{environment}
{Clairsentience (Scrying)}
{
	\textbf{Level:}
	Seer 3\\
	\textbf{Manifesting Time:}
	1 mental action\\
	\textbf{Range:}
	Unlimited\\
	\textbf{Effect:}
	Psionic sensor\\
	\textbf{Saving Throw:}
	Will negates (object)\\
	\textbf{Cost:}
	15 power points\\
	\textbf{Maintenance Cost:}
	None (Instantaneous)\\
	% \textbf{Critical Success:}
	% You receive one minute of observation\\
	% \textbf{Critical Failure:}
	% You get images from a similar but different item\\
}
{
	This power lets you get a sensory image of the present surroundings of a particular unliving item. You need not have any idea where the object is when the power is used, but you must concentrate on a specific, familiar item. For instance, ``my friend Krasha's metal dagger'' is fine, but ``the nearest metal dagger'' is not.

	You receive an all-round sensory ``snapshot'' from the item's surroundings: visual, olfactory, aural, and temperature signals. The power itself gives no notion of direction or distance to the item in question, though the character can often deduce its location from the sensory signals.

	If the item has been destroyed before the power is used, the power check automatically fails. A concealed item (for example, in a pocket) gives a black visual image and insignificant olfactory, aural, and temperature signals.
}
\Psionic{Fate of One}{fate of one}
{Clairsentience}
{
	\textbf{Level:}
	Seer 7\\
	\textbf{Manifesting Time:}
	1 immediate action\\
	\textbf{Range:}
	Personal\\
	\textbf{Target:}
	You\\
	\textbf{Cost:}
	13 power points\\
	\textbf{Maintenance Cost:}
	None (Instantaneous)\\
	\textbf{Prerequisites:}
	\psionic{second chance}\\
}
{
	Your limited omniscience allows you to reroll a saving throw, attack roll, or skill check. Whatever the result of the reroll, you must use it even if it is worse than the original roll.

	You can manifest this power instantly, quickly enough to gain its benefits in an emergency. Manifesting this power is an immediate action. If you use the power to reroll a saving throw, you can manifest this power even when it is not your turn.
}
\Psionic{Know Direction and Location}{know direction and location}
{Clairsentience}
{
	\textbf{Level:}
	Seer 1\\
	\textbf{Manifesting Time:}
	5 minutes\\
	\textbf{Range:}
	Personal\\
	\textbf{Target:}
	You\\
	\textbf{Cost:}
	5 power points\\
	\textbf{Maintenance Cost:}
	None (Instantaneous)\\
	% \textbf{Critical Success:}
	% You know your exact location\\
	% \textbf{Critical Failure:}
	% You can't use this power for 24 hours\\
}
{
	You generally know where you are. This power is useful to characters who end up at unfamiliar destinations after teleporting, using a \spell{gate}, or traveling to or from other planes of existence. The power reveals general information about your location as a feeling or presentiment. The information is usually no more detailed than a summary that locates you according to a prominent local or regional site. Using this power also tells you what direction you are facing.
}
\Psionic{Object Reading}{object reading}
{Clairsentience}
{
	\textbf{Level:}
	Seer 5\\
	\textbf{Manifesting Time:}
	1 minute\\
	\textbf{Range:}
	Touch\\
	\textbf{Target:}
	Object touched\\
	\textbf{Cost:}
	9 power points\\
	\textbf{Maintenance Cost:}
	5 pp/minute\\
	% \textbf{Critical Success:}
	% You know all information on the last owner automatically\\
	% \textbf{Critical Failure:}
	% You believe obsessed with the object until you can read it again\\
}
{
	You can learn details of an inanimate object's previous owner. Objects accumulate psychic impressions left by their previous owners, which can be read by use of this power.

	The amount of information revealed depends on how long you study a particular object. You can maintain this power for one additional minute per point above the power check DC on your power check. You may stop maintaining this power before you reach your time limit.

	\textit{1st Minute:} Last owner's race.

	\textit{2nd Minute:} Last owner's gender.

	\textit{3rd Minute:} Last owner's age.

	\textit{4th Minute:} Last owner's alignment.

	\textit{5th Minute:} How last owner gained and lost the object.

	\textit{6th+ Minute:} Next-to-last owner's race, and so on.

	An object can be read only once per character level---additional reading at that level reveal no additional information. When you gain a new character level, you can try reading the same object again, even if your power score has not changed.
}
\Psionic{Precognition}{precognition}
{Clairsentience}
{
	\textbf{Level:}
	Seer 6\\
	\textbf{Casting Time:}
	5 minutes\\
	\textbf{Range:}
	Personal\\
	\textbf{Target:}
	You\\
	\textbf{Cost:}
	11 power points\\
	\textbf{Maintenance Cost:}
	6 pp/hour\\
	% \textbf{Critical Success:}
	% The insight bonus increases to +4\\
	% \textbf{Critical Failure:}
	% No effect\\
}
{
	\emph{Precognition} allows your mind to glimpse fragments of potential future events---what you see will probably happen if no one takes action to change it. However, your vision is incomplete, and it makes no real sense until the actual events you glimpsed begin to unfold. That's when everything begins to come together, and you can act, if you act swiftly, on the information you previously received when you manifested this power.

	In practice, manifesting this power grants you a ``precognitive edge.'' This edge can be used in two different ways:
	\begin{enumerate*}
	\item +2 insight bonus that you can apply at any time to either an attack roll, a damage roll, or a saving throw;
	\item You can have advantage in a single skill check.
	\end{enumerate*}

	You can elect to apply the edge to the roll after you determine that your unmodified roll is lower than desired.

	You can maintain the precognitive edge for one hour plus one hour for each three points you exceed the power check DC. The power is discharged when you use the edge.

	\emph{Precognition} is tiring. Regardless of the edge chosen, you must rest for at least 10 minutes before you can use another clairsentient power (the use of other disciplines is not affected).

	You can have only a single precognitive edge at one time.
}
\Psionic{Predestination}{predestination}
{Clairsentience}
{
	\textbf{Level:}
	Seer 7\\
	\textbf{Manifesting Time:}
	1 hour\\
	\textbf{Range:}
	Touch\\
	\textbf{Target:}
	Creature touched\\
	\textbf{Cost:}
	35 power points\\
	\textbf{Maintenance Cost:}
	7 pp/year\\
	\textbf{Prerequisites:}
	\psionic{precognition}\\
	% \textbf{Critical Success:}
	% You peer twice as far into the future\\
	% \textbf{Critical Failure:}
	% No effect\\
}
{
	You predict the general destiny of a single creature. That destiny is based on the current situation only---future actions may change the target's predicted destiny.

	Before making the power check, you must spend a full hour alone with the subject (or isolated if you are using the power on yourself). The two converse about the past and present, and the subject makes known their plans for at least the immediate future. If the subject is not completely honest with you, the power check automatically fails.

	If you succeed, you gain a broad understanding of the target creature's prospects for one year in the future. By instructing the target, you infuse them with three ``predestination edges'' per year. This edge can be used in two different ways:
	\begin{enumerate*}
	\item +5 insight bonus that they can apply at any time to either an attack roll, a damage roll, or a saving throw;
	\item They can take 10 with advantage in a single skill check, even if distracted or threatened.
	\end{enumerate*}

	A creature can't have more than one \emph{predestination} active. Both you and the target can dismiss this power at any time. The power is discharged when the target uses all predestination edges.

	\textit{Augment:} For every 5 additional power points you spend, this power can foresee another year in advance and the Difficulty Class of the power check increases by +1.
}
\Psionic{Safe Path}{safe path}
{Clairsentience}
{
	\textbf{Level:}
	Psion 5\\
	\textbf{Casting Time:}
	1 mental action\\
	\textbf{Range:}
	Personal\\
	\textbf{Target:}
	You\\
	\textbf{Cost:}
	9 power points\\
	\textbf{Maintenance Cost:}
	5 pp/round\\
	\textbf{Prerequisites:}
	\psionic{danger sense}\\
	% \textbf{Critical Success:}
	% You succeed on your first ability or skill check\\
	% \textbf{Critical Failure:}
	% You can't use this power for 1d6 hours\\
}
{
	By listening for the warning tingle of your \emph{danger sense} at work, you know when to duck, when to dodge, and when to move forward. This ability is more tactical than \psionic{danger sense}, provided you trust your instincts and move when you are supposed to.

	You receive +4 insight bonus on \skill{Climb}, \skill{Disable Device}, and all Dexterity-based skill checks.

	You also receive +2 insight bonus to Armor Class, and +4 insight bonus on all saving throws.

	\textit{Augment:} For every 3 additional power points you spend, the insight bonus to Armor Class gained increases by 1.
}
\Psionic{Second Chance}{second chance}
{Clairsentience}
{
	\textbf{Level:}
	Seer 5\\
	\textbf{Manifesting Time:}
	1 mental action\\
	\textbf{Range:}
	Personal\\
	\textbf{Target:}
	You\\
	\textbf{Cost:}
	9 power points\\
	\textbf{Maintenance Cost:}
	5 pp/round\\
}
{
	You take a hand in influencing the probable outcomes of your immediate environment. You see the many alternative branches that reality could take in the next few seconds, and with this foreknowledge you gain the ability to reroll one attack roll, one saving throw, one ability check, or one skill check each round. You must take the result of the reroll, even if it’s worse than the original roll. You do not have to make another roll if satisfied with your original roll.
}
\Psionic{Sensitivity to Psychic Impressions}{sensitivity to psychic impressions}
{Clairsentience}
{
	\textbf{Level:}
	Seer 2\\
	\textbf{Casting Time:}
	2 minutes\\
	\textbf{Range:}
	18 meters\\
	\textbf{Effect:}
	18-m-radius emanation, centered on you\\
	\textbf{Cost:}
	3 power points\\
	\textbf{Maintenance Cost:}
	2 pp/minute\\
	% \textbf{Critical Success:}
	% You gain an unusally clear understanding of each event\\
	% \textbf{Critical Failure:}
	% You are a target of a \spell{magic jar} from an angry ghost\\
}
{
	You gain historical vision in a given location. Rooms, streets, tunnels, and other discrete locations accumulate psychic impressions left by powerful emotions experienced in a given area. These impressions offer you a picture of the location's past.

	The types of events most likely to leave psychic impressions are those that elicited strong emotions: battles and betrayals, marriages and murders, births and great pain, or any other event where one emotion dominates. Everyday occurrences leave no residue for a manifester to detect.

	The vision of the event is dreamlike and shadowy. You do not gain special knowledge of those involved in the vision, though you might be able to read large banners or other writing if they are in your language.

	Beginning with the most recent significant event at a location and working backward in time, you can sense one distinct event for every minute you maintain concentration, if any such events exist to be sensed. Your sensitivity depends on your power check: for every point exceeding the power check DC, you may sense an additional strong event.
}
\Psionic{Spirit Lore}{spirit lore}
{Clairsentience}
{
	\textbf{Level:}
	Seer 8\\
	\textbf{Manifesting Time:}
	3 minutes\\
	\textbf{Range:}
	Personal\\
	\textbf{Target:}
	You\\
	\textbf{Cost:}
	40 power points\\
	\textbf{Maintenance Cost:}
	8 pp/round\\
	\textbf{Prerequisites:}
	\psionic{detect spirits}\\
	% \textbf{Critical Success:}
	% You contact a very knowledgeable and helpful spirit\\
	% \textbf{Critical Failure:}
	% You contact a malicious entity that attempts to use \spell{magic jar} on you\\
}
{
	As the \spell{commune} spell, except as noted here.

	You contact the spirits and can gather knowledge from them.	You must ask very specific questions when using this power; the spirits will never volunteer information and often seek to distort or confuse the truth. The spirits possess a lot of knowledge about a lot of different things and matters, but they are not omniscient. To determine the quality of the information received, roll a d\% and check the following table:

	\Table{}{cL}{
	\tableheader d\% & \tableheader Result \\
	01--70  & The spirit knows the answer to the question and answers truthfully \\
	71--80  & The spirit knows the answer, but tries to disguise the truth in deceptive riddles \\
	81--90  & The spirit doesn't know, but admits its ignorance \\
	91--100 & The spirit knows only part of the answer and embellishes the truth \\
	}
}
\Psionic{Synesthete}{synesthete}
{Clairsentience}
{
	\textbf{Level:}
	Seer 3\\
	\textbf{Casting Time:}
	1 mental action\\
	\textbf{Range:}
	Personal\\
	\textbf{Target:}
	You\\
	\textbf{Cost:}
	5 power points\\
	\textbf{Maintenance Cost:}
	3 pp/round\\
	% \textbf{Critical Success:}
	% You gain uncanny dodge\\
	% \textbf{Critical Failure:}
	% Any sound or light causes 1 point of damage per round for 1d8 rounds\\
}
{
	You receive one kind of sensory input when a different sense is stimulated. In particular, you can either feel light or feel sound. You can shift your stimulated sense between these two options once per round as a swift action. Your senses continue to work normally as well, unless they are impaired for some reason.

	Your face must be uncovered to use this power, because it is the skin of your face that acts as the sensory receiver.

	If you are feeling light by absorbing ambient light onto your skin, you have your normal visual abilities (except for darkvision), even if your eyes are closed or you are blinded. If your eyes are working normally, you gain a +4 circumstance bonus on all \skill{Spot} and \skill{Search} checks. While feeling light, you are immune to gaze attacks.

	If you are feeling sound by absorbing sound onto your skin and your ears are working normally, the expanded audio input provides you with a +4 circumstance bonus on \skill{Listen} checks.

	Psionic or magical displacement effects, invisibility effects, illusions, and other similar effects confuse your synesthete senses just as they would your normal senses.

	You can also use this power to see sound if you are deafened, or hear light if you are blinded, thus removing all penalties associated with either condition (though you gain no bonuses for using the power in this way if you are not deafened or blinded).
}
\Psionic{Teleport, Psionic}{psionic teleport}
{Psychoportation (Teleportation)}
{
	\textbf{Level:}
	Nomad 5\\
	\textbf{Manifesting Time:}
	1 mental action\\
	\textbf{Range:}
	9 meters (see text)\\
	\textbf{Target:}
	You (see text)\\
	\textbf{Saving Throw:}
	None\\
	\textbf{Cost:}
	See text\\
	\textbf{Maintenance Cost:}
	None (Instantaneous)\\
	% \textbf{Critical Success:}
	% Manifestation cost reduced by 20\%\\
	% \textbf{Critical Failure:}
	% You and affected creatures become dazed for 1 round, and sickened for 2 rounds\\
}
{
	You instantly transfer yourself from your current location to any other spot within range. You always arrive at exactly the spot desired---whether by simply visualizing the area or by stating direction. You can bring along objects as long as their weight doesn't exceed your light load.

	You simply cease to exist in your previous location and springs into being at the destination. There is a slight, audible ``pop'' at both ends, as air rushes into the sudden vacuum or is instantly displaced.

	If you arrive in a place that is already occupied by a solid body, you and each creature traveling with you take 1d6 points of damage and are shunted to a random open space on a suitable surface within 30 meters of the intended location.

	If there is no free space within 30 meters, you and each creature traveling with you take an additional 2d6 points of damage and are shunted to a free space within 300 meters. If there is no free space within 300 meters, you and each creature traveling with you take an additional 4d6 points of damage and the spell simply fails.

	\textit{Augment:} You can augment this power in one or more of the following ways.
	\begin{enumerate*}
	\item If you spend 20 additional power points, you may also bring two additional willing Medium or smaller creatures (carrying gear or objects up to their light load). A Large creature counts as two Medium creatures. All creatures to be transported must be in contact with one another, and at least one of those creatures must be in contact with you.
	\item If you spend 10 additional power points, you can bring along objects if their weight doesn't exceed your heavy load.
	\item If you spend additional power points, you can increase this power's range. The range also increases the Difficulty Class of the power check: 
	\end{enumerate*}

	\Table{}{cCc}{
	\tableheader Cost & \tableheader Range & \tableheader Power Check DC Modifier \\
	 9 & 9 meters         & $-1$ \\
	20 & 90 meters        &  +0  \\
	30 & 900 meters       &  +1  \\
	40 & 9 kilometers     &  +2  \\
	50 & 90 kilometers    &  +3  \\
	60 & 900 kilometers   &  +4  \\
	70 & 9,000 kilometers &  +5  \\
	}

}
\Psionic{Teleport Trigger}{teleport trigger}
{Metacreativity}
{
	\textbf{Level:}
	Shaper 2\\
	\textbf{Manifesting Time:}
	1 mental action\\
	\textbf{Range:}
	Personal\\
	\textbf{Target:}
	You\\
	\textbf{Cost:}
	3 power points\\
	\textbf{Maintenance Cost:}
	2 pp/hour\\
	\textbf{Prerequisites:}
	\psionic{anchored navigation}, \psionic{psionic teleport}\\
	% \textbf{Critical Success:}
	% You don't pay \psionic{psionic teleport}'s augmented cost for distance\\
	% \textbf{Critical Failure:}
	% No effect\\
}
{
	You specify a situation that triggers your automatic manifestation of a \psionic{psionic teleport}, taking you to a predetermined location. You must have sufficient power points to manifest it when the specified situation occurs.

	The \emph{teleport trigger} goes off on the initiative count immediately after the specified situation occurs, even if you are flat-footed or you have already taken your turn in the current round. The specified situation can be described in general terms or specific terms.
}
\Psionic{Trail of Destruction}{trail of destruction}
{Clairsentience}
{
	\textbf{Level:}
	Seer 3\\
	\textbf{Manifesting Time:}
	1 mental action\\
	\textbf{Range:}
	18 meters\\
	\textbf{Effect:}
	18-m-radius emanation, centered on you\\
	\textbf{Cost:}
	15 power point\\
	\textbf{Maintenance Cost:}
	3 pp/round\\
	\textbf{Prerequisites:}
	\psionic{psionic detect magic}\\
	% \textbf{Critical Success:}
	% You know about the spells and their effects\\
	% \textbf{Critical Failure:}
	% You see all defiling magic ever cast in the area, but you cannot pick out which are more recent\\
}
{
	You can feel the past use of defiling magic. This power shows all locations where defiling magic was used within range in the past month. The sites illuminate only for your eyes only.

	You learn the level of each spell cast using defiling magic, and the day in which they were cast. You do not learn, however, about the actual spell or the caster.
}
\Psionic{True Seeing, Psionic}{psionic true seeing}
{Clairsentience}
{
	\textbf{Level:}
	Seer 8\\
	\textbf{Manifesting Time:}
	1 mental action\\
	\textbf{Range:}
	Personal\\
	\textbf{Target:}
	You\\
	\textbf{Cost:}
	40 power points\\
	\textbf{Maintenance Cost:}
	8 pp/round\\
	\textbf{Prerequisites:}
	\psionic{clairvoyance}\\
	% \textbf{Critical Success:}
	% You can see into the Ethereal Plane and all true forms\\
	% \textbf{Critical Failure:}
	% You see through everything you look at as everything is an illusion for 2d4 rounds\\
}
{
	As the \spell{true seeing} spell, except as noted here.

	You do not see into the Ethereal Plane, or the true form of polymorphed, changed, or transmuted things.
}
\Psionic{Ubiquitous Vision}{ubiquitous vision}
{Clairsentience}
{
	\textbf{Level:}
	Psion 3\\
	\textbf{Casting Time:}
	1 mental action\\
	\textbf{Range:}
	Personal\\
	\textbf{Target:}
	You\\
	\textbf{Saving Throw:}
	None\\
	\textbf{Cost:}
	5 power points\\
	\textbf{Maintenance Cost:}
	3 pp/round\\
	\textbf{Critical Success:}
	Manifester also gains darkvision 18 m\\
	\textbf{Critical Failure:}
	Manifester becomes blind for 1d4 hours\\
}
{
	You have metaphoric ``eyes in the back of your head,'' and on the sides and top as well, granting you benefits in specific situations. In effect, you have a 360-degree sphere of sight, allowing you a perfect view of creatures that might otherwise flank you. Thus, flanking opponents gain no bonus on their attack rolls, and rogues are denied their sneak attack ability while flanking (but they may still sneak attack you if you are denied your Dexterity bonus). Your \skill{Spot} and \skill{Search} checks gain a +4 enhancement bonus. Concurrently, you take a $-4$ penalty on saves against all gaze attacks during the power's duration. 
}
\Psionic{Watcher's Ward}{watcher's ward}
{Clairsentience}
{
	\textbf{Level:}
	Seer 2\\
	\textbf{Manifesting Time:}
	1 minute\\
	\textbf{Range:}
	0 meter\\
	\textbf{Effect:}
	18-m-radius emanation\\
	\textbf{Cost:}
	10 power points\\
	\textbf{Maintenance Cost:}
	2 pp/hour\\
	% \textbf{Critical Success:}
	% You know the exact position and type of any intruder\\
	% \textbf{Critical Failure:}
	% You believe you can't be surprised\\
}
{
	You set an area with 18 meters as your warded area. While you are within the area, you are instantly aware of any changes in the area, you can't be surprised by any creature that enters it. You don't automatically spot any invader---you only know that something is coming.

	If you move out of the area, the power is dismissed.
}
\Psionic{Weather Prediction}{weather prediction}
{Clairsentience}
{
	\textbf{Level:}
	Seer 4\\
	\textbf{Manifesting Time:}
	1 mental action\\
	\textbf{Range:}
	5 kilometers\\
	\textbf{Target:}
	5 kilometers radius, centered on you\\
	\textbf{Cost:}
	20 power points\\
	\textbf{Maintenance Cost:}
	None (Instantaneous)\\
	\textbf{Prerequisites:}
	\psionic{detect moisture}\\
	% \textbf{Critical Success:}
	% You predict one week in advance, instead\\
	% \textbf{Critical Failure:}
	% You receive complete erroneous information, but believe is accurate\\
}
{
	You may accurately predict the natural weather up to 24 hours into the future.

	This power only predicts naturally occurring weather, not magically induced conditions.
}

