\Chapter{Powers}
{}{}

\Capitalize{T}{his} chapter begins with the power lists for all manifesting classes and prestiges classes, as well as the description of psion's disciplines. An \textsuperscript{A} appearing at the end of a power's name in the power lists denotes an augmentable power. An \textsuperscript{X} denotes a power with an XP component paid by the manifester.

\textbf{Power Chains:} Some powers reference other powers that they are based upon. Only information in a power later in the power chain that is different from the base power is covered in the power being described. Header entries and other information that are the same as the base power are not repeated. The same holds true for powers that are the equivalents of spells, only the way the power varies from the spell is noted, such as power point cost.

\textbf{Order of Presentation:} In the power lists and the power descriptions that follow them, the powers are presented in alphabetical order by name---except for those belonging to certain power chains and those that are psionic equivalents of spells. When a power's name begins with ``lesser,'' ``greater,'' ``mass,'' or a similar kind of qualifier, the power description is alphabetized under the second word of the power description instead. When the effect of a power is essentially the same as that of a spell, the power's name is simply ``Psionic'' followed by the name of the spell, and it is alphabetized according to the spell name.

\textbf{Manifester Level:} A power's effect often depends on the manifester level, which is the manifester's psionic class level. A creature with no classes has a manifester level equal to its Hit Dice unless otherwise specified. The word ``level'' in the power lists always refers to manifester level.

\textbf{Creatures and Characters:} ``Creatures'' and ``characters'' are used synonymously in the power descriptions.

\textbf{Augment:} Many powers vary in strength depending on how many power points you put into them. The more power points you spend, the more powerful the manifestation. However, you can spend only a total number of points on a power equal to your manifester level, unless you have an ability that increases your effective manifester level.

Many powers can be augmented in more than one way. When the Augment section contains numbered paragraphs, you need to spend power points separately for each of the numbered options. When a paragraph in the Augment section begins with ``In addition,'' you gain the indicated benefit according to how many power points you have already decided to spend on manifesting the power.

\section{Psion/Wilder Powers}




\subsection{1st-Level Psion/Wilder Powers}

\psionicList{Astral Traveler}: Enable yourself or another to join an \psionic{astral caravan}-enabled trip.

\psionicList{Attraction}\textsuperscript{A}: Subject has an attraction you specify.

\psionicList{Aura Reading}: Reveal personal details about the target.

\psionicList{Bioflexibilty}: You gain +10 competence bonus to \skill{Escape Artist} checks.

\psionicList{Bolt}\textsuperscript{A}: You create a few enhanced short-lived bolts, arrows, or bullets.

\psionicList{Call to Mind}: Gain additional \skill{Knowledge} check with +4 competence bonus.

\psionicList{Cast Missiles}: You can launch missiles without a bow or other weapon.

\psionicList{Catfall}\textsuperscript{A}: Instantly save yourself from a fall.

\psionicList{Cause Sleep}\textsuperscript{A}: Puts 4 HD of creatures into deep slumber.

\psionicList{Conceal Thoughts}: You conceal your motives.

\psionicList{Control Flames}\textsuperscript{A}: Take control of nearby open flame.

\psionicList{Control Light}: Adjust ambient light levels.

\psionicList{Create Sound}: Create the sound you desire.

\psionicList{Cryokinesis}: You cool a creature or object.

\psionicList{Crystal Shard}\textsuperscript{A}: Ranged touch attack for 1d6 points of piercing damage.

\psionicList{Daze, Psionic}\textsuperscript{A}: Humanoid creature of 4 HD or less loses next action.

\psionicList{Deceleration}\textsuperscript{A}: Target's speed is halved.

\psionicList{Deflect Strike}: You psychokinetically deflect the next attack of a creature within range.

\noindent\textit{\hyperref[psionic:Deja Vu]{Déjà Vu}}\textsuperscript{A}: Your target repeats his last action.

\psionicList{Demoralize}\textsuperscript{A}: Enemies become shaken.

\psionicList{Detect Poison, Psionic}: Detects poison in one creature or object.

\psionicList{Detect Psionics}: You detect the presence of psionics.

\psionicList{Disable}\textsuperscript{A}: Subjects incorrectly believe they are disabled.

\psionicList{Dissipating Touch}\textsuperscript{A}: Touch deals 1d6 damage.

\psionicList{Distract}: Target gets $-4$ bonus on \skill{Listen}, \skill{Search}, \skill{Sense Motive}, and \skill{Spot} checks.

\psionicList{Ecto Protection}\textsuperscript{A}: An astral construct gains bonus against \psionic{dismiss ectoplasm}.

\psionicList{Empathy}\textsuperscript{A}: You know the subject's surface emotions.

\psionicList{Empty Mind}\textsuperscript{A}: You gain +2 on Will saves until your next action.

\psionicList{Energy Ray}\textsuperscript{A}: Deal 1d6 energy (cold, electricity, fire, or sonic) damage.

\psionicList{Entangling Ectoplasm}: You entangle a foe in sticky goo.

\psionicList{Far Hand}\textsuperscript{A}: Move small objects at a limited distance.

\psionicList{Float}: You buoy yourself in water or other liquid.

\psionicList{Force Screen}\textsuperscript{A}: Invisible disc provides +4 shield bonus to AC.

\psionicList{Ghost Writing}: Creatures writing on a distant surface or creature touched.

\psionicList{Grease, Psionic}: Makes 3-m square or one object slippery.

\psionicList{Hammer}\textsuperscript{A}: Melee touch attack deals 1d8/round.

\psionicList{Hush}\textsuperscript{A}: Subjects become utterly silent.

\psionicList{Inertial Armor}\textsuperscript{A}: Tangible field of force provides you with +4 armor bonus to AC.

\psionicList{Know Direction and Location}: You discover where you are and what direction you face.

\psionicList{Matter Agitation}: You heat a creature or object.

\psionicList{Mind Thrust}\textsuperscript{A}: Deal 1d10 damage.

\psionicList{Missive}\textsuperscript{A}: Send a one-way telepathic message to subject.

\psionicList{My Light}\textsuperscript{A}: Your eyes emit 6-m cone of light.

\psionicList{Photosynthesis}\textsuperscript{A}: Transform light into healing.
% \psionicList{Precognition, Defensive}\textsuperscript{A}: Gain +1 insight bonus to AC and saving throws.
% \psionicList{Precognition, Offensive}\textsuperscript{A}: Gain +1 insight bonus on your attack rolls.
% \psionicList{Prescience, Offensive}\textsuperscript{A}: Gain +2 insight bonus on your damage rolls.

\psionicList{Psionic Draw}: Instantly draw a weapon.

\psionicList{Psychic Tracking}: Track a creature using \skill{Psicraft}.

\psionicList{Sense Link}\textsuperscript{A}: You sense what the subject senses (single sense).

\psionicList{Skate}: Subject slides skillfully along the ground.

\psionicList{Synesthete}: You receive one kind of sense when another sense is stimulated.

\psionicList{Tattoo Animation}\textsuperscript{A}: Animates your tattoos or steals another's.

\psionicList{Telempathic Projection}: Alter the subject's mood.

\psionicList{Trail of Destruction}: Detects recent defiling.

\psionicList{Vigor}\textsuperscript{A}: Gain 5 temporary hit points.

\psionicList{Wild Leap}\textsuperscript{A}: Make an additional leap and gain a bonus to \skill{Jump} checks.




\subsection{2nd-Level Psion/Wilder Powers}

\psionicList{Alter Self, Psionic}\textsuperscript{A}: Assume form of a similar creature.

\psionicList{Bestow Power}\textsuperscript{A}: Subject receives 2 power points.

\psionicList{Biofeedback}\textsuperscript{A}: Gain damage reduction 2/--.

\psionicList{Body Equilibrium}: You can walk on nonsolid surfaces.

\psionicList{Calm Emotions, Psionic}: Calms creatures, negating emotion effects.

\psionicList{Cloud Mind}: You erase knowledge of your presence from target's mind.

\psionicList{Concealing Amorpha}: Quasi-real membrane grants you concealment.

\psionicList{Concentrate Water}: Collects water from surrounding area.

\psionicList{Concussion Blast}\textsuperscript{A}: Deal 1d6 force damage to target.

\psionicList{Control Sound}: Create very specific sounds.

\psionicList{Detect Hostile Intent}: You can detect hostile creatures within 9 m of you.

\psionicList{Detect Life}: Reveals living creatures.

\psionicList{Ego Whip}\textsuperscript{A}: Deal 1d4 Cha damage and daze for 1 round.

\psionicList{Elfsight}: Gain low-light vision, +2 bonus on Search and Spot checks, and notice secret doors.

\psionicList{Energy Adaptation, Specified}\textsuperscript{A}: Gain resistance 10 against one energy type.

\psionicList{Energy Push}\textsuperscript{A}: Deal 2d6 damage and knock subject back.

\psionicList{Energy Stun}\textsuperscript{A}: Deal 1d6 damage and stun target if it fails both saves.

\psionicList{Feat Leech}\textsuperscript{A}: Borrow another's psionic or metapsionic feats.

\psionicList{Id Insinuation}\textsuperscript{A}: Swift tendrils of thought disrupt and confuse your target.

\psionicList{Identify, Psionic}: Learn the properties of a psionic item.

\psionicList{Inflict Pain}\textsuperscript{A}: Telepathic stab gives your foe $-4$ on attack rolls, or $-2$ if he makes the save.

\psionicList{Knock, Psionic}: Opens locked or psionically sealed door.

\psionicList{Levitate, Psionic}: Subject moves up and down at your direction.

\psionicList{Mental Disruption}\textsuperscript{A}: Daze creatures within 3 meters for 1 round.

\psionicList{Missive, Mass}\textsuperscript{A}: You send a one-way telepathic message to an area.

\psionicList{Molecular Bonding}: Temporarily glue two surfaces together.

\psionicList{Pheromone Discharge}: Vermin react well to you.

\psionicList{Psionic Lock}: Secure a door, chest, or portal.

\psionicList{Recall Agony}\textsuperscript{A}: Foe takes 2d6 damage.

\psionicList{Return Missile}\textsuperscript{A}: Make one weapon return to you after thrown.

\psionicList{Sense Link, Forced}: Sense what subject senses.

\psionicList{Sensory Suppression}: Victim loses one sense---sight, hearing, smell.

\psionicList{Sever the Tie}\textsuperscript{A}: Disrupt an undead's tie to the Gray, damaging or destroying it.

\psionicList{Share Pain}: Willing subject takes some of your damage.

\psionicList{Sustenance}: Go without food and water for one day.

\psionicList{Swarm of Crystals}\textsuperscript{A}: Crystal shards are sprayed forth doing 3d4 slashing damage.

\psionicList{Thought Shield}\textsuperscript{A}: Gain PR 13 against mind-affecting powers.

\psionicList{Tongues, Psionic}: You can communicate with intelligent creatures.

\psionicList{Watcher Ward}\textsuperscript{A}: You are aware of creature within the warded area.

\psionicList{Weather Prediction}: Predicts weather for next 24 hours.




\subsection{3rd-Level Psion/Wilder Powers}

\psionicList{Antidote Simulation}\textsuperscript{A}: Detoxifies venom in your system.

\psionicList{Beacon}\textsuperscript{A}: Creates a ball on light that can become much larger with concentration.

\psionicList{Blink, Psionic}\textsuperscript{A}: You randomly vanish and reappear for 1 round/level.

\psionicList{Body Adjustment}\textsuperscript{A}: You heal 1d12 damage.

\psionicList{Body Purification}\textsuperscript{A}: You restore 2 points of ability damage.

\psionicList{Danger Sense}\textsuperscript{A}: You gain +4 bonus against traps.

\psionicList{Darkvision, Psionic}: See 18 m in total darkness.

\psionicList{Dismiss Ectoplasm}: Dissipates ectoplasmic targets and effects.

\psionicList{Dispel Psionics}\textsuperscript{A}: Cancels psionic powers and effects.

\psionicList{Energy Bolt}\textsuperscript{A}: Deal 5d6 energy damage in 36-m line.

\psionicList{Energy Burst}\textsuperscript{A}: Deal 5d6 energy damage in 12-m burst.

\psionicList{Energy Retort}\textsuperscript{A}: Ectoburst of energy automatically targets your attacker for 4d6 damage once each round.

\psionicList{Energy Wall}: Create wall of your chosen energy type.

\psionicList{Eradicate Invisibility}\textsuperscript{A}: Negate invisibility in 15-m burst.

\psionicList{Keen Edge, Psionic}: Doubles normal weapon's threat range.

\psionicList{Lighten Load, Psionic}: Increases Strength for carrying capacity only.

\psionicList{Mass Manipulation}\textsuperscript{A}: Alter the weight of a creature or object.

\psionicList{Mental Barrier}\textsuperscript{A}: Gain +4 deflection bonus to AC until your next action.

\psionicList{Mind Trap}\textsuperscript{A}: Drain 1d6 power points from anyone who attacks you with a telepathy power.

\psionicList{Nerve Manipulation}\textsuperscript{A}: Disrupts a creature nervous system.

\psionicList{Psionic Blast}\textsuperscript{A}: Stun creatures in 9-m cone for 1 round.

\psionicList{Psionic Sight}\textsuperscript{A}: Psionic auras become visible to you.

\psionicList{Share Pain, Forced}\textsuperscript{A}: Unwilling subject takes some of your damage.

\psionicList{Solicit Psicrystal}\textsuperscript{A}: Your psicrystal takes over your concentration power.

\psionicList{Telekinetic Force}\textsuperscript{A}: Move an object with the sustained force of your mind.

\psionicList{Telekinetic Thrust}\textsuperscript{A}: Hurl objects with the force of your mind.

\psionicList{Time Hop}\textsuperscript{A}: Subject hops forward in time 1 round/level.

\psionicList{Touchsight}\textsuperscript{A}: Your telekinetic field tells you where everything is.

\psionicList{Ubiquitous Vision}: You have all-around vision.




\subsection{4th-Level Psion/Wilder Powers}

\psionicList{Aura Sight}\textsuperscript{A}: Reveals creatures, objects, powers, or spells of selected alignment axis.

\psionicList{Correspond}: Hold mental conversation with another creature at any distance.

\psionicList{Death Urge}\textsuperscript{A}: Implant a self-destructive compulsion.

\psionicList{Detect Remote Viewing}: You know when others spy on you remotely.

\psionicList{Detonate}\textsuperscript{A}: Explode one object.

\psionicList{Dimension Door, Psionic}: Teleports you short distance.

\psionicList{Divination, Psionic}: Provides useful advice for specific proposed action.

\psionicList{Empathic Feedback}\textsuperscript{A}: When you are hit in melee, your attacker takes damage.

\psionicList{Energy Adaptation}\textsuperscript{A}: Your body converts energy to harmless light.

\psionicList{Freedom of Movement, Psionic}: You cannot be held or otherwise rendered immobile.

\psionicList{Intellect Fortress}\textsuperscript{A}: Those inside fortress take only half damage from all powers and psi-like abilities until your next action.

\psionicList{Magnetize}\textsuperscript{A}: Make metallic object magnetic.

\psionicList{Mindwipe}\textsuperscript{A}: Subject's recent experiences wiped away, bestowing negative levels.

\psionicList{Personality Parasite}: Subject's mind calves self-antagonistic splinter personality for 1 round/level.

\psionicList{Power Leech}: Drain 1d6 power points/round while you maintain concentration; you gain 1/round.

\psionicList{Psychic Reformation}\textsuperscript{X}: Subject can choose skills, feats, and powers anew for previous levels.

\psionicList{Repugnance}: Make a creature repugnant to others.

\psionicList{Shadow Jump}\textsuperscript{A}: Jump into shadow to travel rapidly.

\psionicList{Telekinetic Maneuver}\textsuperscript{A}: Telekinetically bull rush, disarm, grapple, or trip your target.

\psionicList{Trace Teleport}\textsuperscript{A}: Learn destination of subject's teleport.

\psionicList{Wall of Ectoplasm}: You create a protective barrier.




\subsection{5th-Level Psion/Wilder Powers}

\psionicList{Adapt Body}: Your body automatically adapts to hostile environments.

\psionicList{Catapsi}\textsuperscript{A}: Psychic static inhibits power manifestation.

\psionicList{Ectoplasmic Shambler}: Foglike predator deals 1 point of damage/two levels each round to an area.

\psionicList{Electroerosion}\textsuperscript{A}: Create a ray that erodes iron and alloys.

\psionicList{Incarnate}\textsuperscript{X}: Make some powers permanent.

\psionicList{Leech Field}\textsuperscript{A}: Leech power points each time you make a saving throw.

\psionicList{Major Creation, Psionic}: As \psionic{psionic minor creation}, plus stone and metal.

\psionicList{Plane Shift, Psionic}: Travel to other planes.

\psionicList{Power Resistance}: Grant PR equal to 12 + level.

\psionicList{Psychic Crush}\textsuperscript{A}: Brutally crush subject's mental essence, reducing subject to $-1$ hit points.

\psionicList{Shatter Mind Blank}: Cancels target's mind blank effect.

\psionicList{Tower of Iron Will}\textsuperscript{A}: Grant PR 19 against mind-affecting powers to all creatures within 3 m until your next turn.

\psionicList{True Seeing, Psionic}: See all things as they really are.




\subsection{6th-Level Psion/Wilder Powers}

\psionicList{Aura Alteration}\textsuperscript{A}: Repairs psyche or makes subject seem to be something it is not.

\psionicList{Breath of the Black Dragon}\textsuperscript{A}: Breathe acid for 11d6 damage.

\psionicList{Cloud Mind, Mass}: Erase knowledge of your presence from the minds of one creature/level.

\psionicList{Co-opt Concentration}: Take control of foe's concentration power.

\psionicList{Contingency, Psionic}\textsuperscript{X}: Sets trigger condition for another power.

\psionicList{Dimensional Screen}: Create a shimmering screen that diverts attacks.

\psionicList{Disintegrate, Psionic}\textsuperscript{A}: Turn one creature or object to dust.

\psionicList{Fuse Flesh}\textsuperscript{A}: Fuse subject's flesh, creating a helpless mass.

\psionicList{Overland Flight, Psionic}: You fly at a speed of 12 m and can hustle over long distances.

\psionicList{Remote View Trap}: Deal 8d6 points electricity damage to those who seek to view you at a distance.

\psionicList{Retrieve}\textsuperscript{A}: Teleport to your hand an item you can see.

\psionicList{Suspend Life}: Put yourself in a state akin to suspended animation.

\psionicList{Temporal Acceleration}\textsuperscript{A}: Your time frame accelerates for 1 round.




\subsection{7th-Level Psion/Wilder Powers}

\psionicList{Decerebrate}: Remove portion of subject's brain stem.

\psionicList{Divert Teleport}: Choose destination for another's teleport.

\psionicList{Energy Conversion}: Offensively channel energy you've absorbed.

\psionicList{Energy Wave}\textsuperscript{A}: Deal 13d4 damage of your chosen energy type in 36-m cone.

\psionicList{Evade Burst}\textsuperscript{A}: You take no damage from a burst on a successful Reflex save.

\psionicList{Incorporeality}\textsuperscript{A}: You become incorporeal for 1 round/level.

\psionicList{Insanity}\textsuperscript{A}: Subject is permanently confused.

\psionicList{Mind Blank, Personal}: You are immune to \spell{scrying} and mental effects.

\psionicList{Mindflame}: Kills, paralyzes, weakens, or dazes subjects.

\psionicList{Moment of Prescience, Psionic}: You gain insight bonus on single attack roll, check, or save.

\psionicList{Oak Body}\textsuperscript{A}: Your body becomes as hard as oak.

\psionicList{Phase Door, Psionic}: Invisible passage through wood or stone.

\psionicList{Sequester, Psionic}\textsuperscript{X}: Subject invisible to sight and \psionic{remote viewing}; renders subject comatose.

\psionicList{Ultrablast}\textsuperscript{A}: Deal 13d6 damage in 4.5-m radius.




\subsection{8th-Level Psion/Wilder Powers}

\psionicList{Bend Reality}\textsuperscript{X}: Alters reality within power limits.

\psionicList{Iron Body, Psionic}: Your body becomes living iron.

\psionicList{Matter Manipulation}\textsuperscript{X}: Increase or decrease an object's base hardness by 5.

\psionicList{Mind Blank, Psionic}: Subject immune to mental/emotional effects, \spell{scrying}, and \psionic{remote viewing}.

\psionicList{Recall Death}: Subject dies or takes 5d6 damage.

\psionicList{Shadow Body}: You become a living shadow (not the creature).

\psionicList{Teleport, Psionic Greater}: As \psionic{psionic teleport}, but no range limit and no off-target arrival.

\psionicList{True Metabolism}: You regenerate 10 hit points/round.




\subsection{9th-Level Psion/Wilder Powers}

\psionicList{Affinity Field}: Effects that affect you also affect others.

\psionicList{Apopsi}\textsuperscript{X}: You delete target's psionic powers.

\psionicList{Assimilate}: Incorporate creature into your own body.

\psionicList{Etherealness, Psionic}: Become ethereal for 1 min./level.

\psionicList{Microcosm}\textsuperscript{A}: Creature or creature lives forevermore in world of his own imagination.

\psionicList{Reality Revision}\textsuperscript{X}: As \psionic{bend reality}, but fewer limits.

\psionicList{Timeless Body}: Ignore all harmful, and helpful, effects for 1 round.
\section{Psion Discipline Powers}



\subsection{Egoist Discipline Powers {\normalsize(Psychometabolism)}}
\begin{enumerate*}
\item \psionicList{Thicken Skin}\textsuperscript{A}: Gain +1 enhancement bonus to your AC for 10 min./level.
\item \psionicList{Animal Affinity}\textsuperscript{A}: Gain +4 enhancement to one ability.

\psionicList{Chameleon}: Gain +10 enhancement bonus on \skill{Hide} checks.

\psionicList{Empathic Transfer}\textsuperscript{A}: Transfer another's wounds to yourself.

\psionicList{Share Strength}\textsuperscript{A}: Temporarily transfer your Strength to another. %

\item \psionicList{Aging}\textsuperscript{A}: Make subject older. %

\psionicList{Death Field}\textsuperscript{A}: Release an energy burst from the Gray that drains vital energy. %

\psionicList{Ectoplasmic Form}: You gain benefits of being insubstantial and can fly slowly.

\psionicList{Hustle}: Instantly gain a move action.

\item \psionicList{Accelerate}\textsuperscript{A}: Move faster, +1 on attack rolls, AC, and Reflex saves. %

\psionicList{Metamorphosis}: Assume shape of creature or object.

\psionicList{Psychic Vampire}: Touch attack drains 2 power points/level from foe.

\item \psionicList{Revivify, Psionic}\textsuperscript{AX}: Return the dead to life before the psyche leaves the corpse.

\psionicList{Psychofeedback}: Boost Strength, Dexterity, or Constitution at the expense of one or more other scores.

\psionicList{Restore Extremity}: Return a lost digit, limb, or other appendage to subject.

\item \psionicList{Restoration, Psionic}: Restores level and ability score drains.
\item \psionicList{Complete Healing}\textsuperscript{A}: Heals all damage. %

\psionicList{Fission}: You briefly duplicate yourself.

\psionicList{Poison Simulation}\textsuperscript{A}: Coat surface with potent poisons. %

\item \psionicList{Fusion}\textsuperscript{X}: You combine your abilities and form with another.
\item \psionicList{Metamorphosis, Greater}\textsuperscript{X}: Assume shape of any nonunique creature or object each round.
\end{enumerate*}



\subsection{Kineticist Discipline Powers {\normalsize(Psychokinesis)}}
\begin{enumerate*}
\item \psionicList{Control Object}: Telekinetically animate a small object.
\item \psionicList{Control Air}\textsuperscript{A}: You have control over wind speed and direction.

\psionicList{Energy Missile}\textsuperscript{A}: Deal 3d6 energy damage to up to five subjects.

\item \psionicList{Energy Cone}\textsuperscript{A}: Deal 5d6 energy damage in 18-m cone.
\item \psionicList{Control Body}\textsuperscript{A}: Take rudimentary control of your foe's limbs.

\psionicList{Energy Ball}\textsuperscript{A}: Deal 7d6 energy damage in 6-m radius.

\psionicList{Inertial Barrier}: Gain DR 5/--.

\item \psionicList{Energy Current}\textsuperscript{A}: Deal 9d6 damage to one foe and half to another foe as long as you concentrate.

\psionicList{Fiery Discorporation}\textsuperscript{A}: Cheat death by discorporating into nearby fire for one day.

\item \psionicList{Dispelling Buffer}: Subject is buffered from one \psionic{dispel psionics} effect.

\psionicList{Null Psionics Field}: Create a field where psionic power does not function.

\item \psionicList{Reddopsi}: Powers targeting you rebound on manifester.
\item \psionicList{Telekinetic Sphere, Psionic}: Mobile force globe encapsulates creature and moves it.
\item \psionicList{Tornado Blast}\textsuperscript{A}: Vortex of air subjects your foes to 17d6 damage and moves them.
\end{enumerate*}



\subsection{Nomad Discipline Powers {\normalsize(Psychoportation)}}
\begin{enumerate*}
\item \psionicList{Burst}: Gain +3 m to speed this round.

\psionicList{Detect Teleportation}\textsuperscript{A}: Know when teleportation powers are used in close range.

\item \psionicList{Dimension Swap}\textsuperscript{A}: You and ally or two allies switch positions.

\psionicList{Levitate, Psionic}: Subject moves up and down at your direction.

\item \psionicList{Astral Caravan}\textsuperscript{A}: You lead \psionic{astral traveler}-enabled group to a planar destination.
\item \psionicList{Dimensional Anchor, Psionic}: Bars extra dimensional movement.

\psionicList{Dismissal, Psionic}: Forces a creature to return to its native plane.

\psionicList{Fly, Psionic}: You fly at a speed of 18 m.

\item \psionicList{Baleful Teleport}\textsuperscript{A}: Destructive teleport deals 9d6 damage.

\psionicList{Teleport, Psionic}: Instantly transports you as far as 100 miles/level.

\psionicList{Teleport Trigger}: Predetermined event triggers teleport.

\item \psionicList{Banishment, Psionic}\textsuperscript{A}: Banishes extraplanar creatures.
\item \psionicList{Dream Travel}\textsuperscript{A}: Travel to other places through dreams.

\psionicList{Ethereal Jaunt, Psionic}: Become ethereal for 1 round/level.

\psionicList{Teleport Object, Psionic}: As \psionic{psionic teleport}, but affects a touched object. %

\item \psionicList{Time Hop, Mass}\textsuperscript{A}: Willing subjects hop forward in time.
\item \psionicList{Teleportation Circle, Psionic}: Circle teleports any creatures inside to designated spot.

\psionicList{Time Regression}\textsuperscript{X}: Relive the last round.
\end{enumerate*}



\subsection{Seer Discipline Powers {\normalsize(Clairsentience)}}
\begin{enumerate*}
\item \psionicList{Destiny Dissonance}: Your dissonant touch sickens a foe.

\psionicList{Precognition}: Gain +2 insight bonus to one roll.

\item \psionicList{Clairvoyant Sense}: See and hear a distant location.

\psionicList{Locate, Psionic}\textsuperscript{A}: Indicates direction to familiar objects and creatures. %

\psionicList{Object Reading}\textsuperscript{A}: Learn details about an object's previous owner.

\psionicList{Sensitivity to Psychic Impressions}: You can find out about an area's past.

\item Detect Moisture\textsuperscript{A}: Reveals moisture within 18 m. %

\psionicList{Escape Detection}: You become difficult to detect with clairsentience powers.

\psionicList{Fate Link}\textsuperscript{A}: You link the fates of two targets.

\psionicList{Truthear}: Receive +20 insight bonus to \skill{Sense Motive} checks. %

\item \psionicList{Anchored Navigation}\textsuperscript{A}: Establish a mishap-free teleport beacon.

\psionicList{Remote Viewing}\textsuperscript{X}: See, hear, and potentially interact with 
subjects at a distance.
\item \psionicList{Clairtangent Hand}\textsuperscript{A}: Emulate \psionic{far hand} at a distance.

\psionicList{Second Chance}: Gain a reroll.

\item \psionicList{Precognition, Greater}: Gain +4 insight bonus to one roll.
\item \psionicList{Fate of One}: Reroll any roll you just failed.
\item \psionicList{Hypercognition}: You can deduce almost anything.
\item \psionicList{Cosmic Awareness}: You perceive all things in range. %

\psionicList{Metafaculty}\textsuperscript{X}: You learn details about any one creature.
\end{enumerate*}



\subsection{Shaper Discipline Powers {\normalsize(Metacreativity)}}
\begin{enumerate*}
\item \psionicList{Astral Construct}\textsuperscript{A}: Creates astral construct to fight for you.

\psionicList{Minor Creation, Psionic}: Creates one cloth or wood object.

\item \psionicList{Psionic Repair Damage}\textsuperscript{A}: Repairs construct of 3d8 hit points +1 hp/level.
\item \psionicList{Concealing Amorpha, Greater}: Quasi-real membrane grants you total concealment.

\psionicList{Ectoplasmic Cocoon}\textsuperscript{A}: You encapsulate a foe so it can't move.

\item \psionicList{Fabricate, Psionic}: Transforms raw goods to finished items.

\psionicList{Quintessence}: You collapse a bit of time into a physical substance.

\item \psionicList{Hail of Crystals}\textsuperscript{A}: A crystal explodes in an area, dealing 9d4 slashing damage.

\psionicList{Pocket Dimension}\textsuperscript{A}: Create a small storage area in an extradimensional space. %

\item \psionicList{Crystallize}: Turn subject permanently to crystal.

\psionicList{Fabricate, Greater Psionic}: Transforms a lot of raw goods to finished items.

\item \psionicList{Ectoplasmic Cocoon, Mass}\textsuperscript{A}: You encapsulate all foes in a 6-m radius.
\item \psionicList{Astral Seed}: You plant the seed of your rebirth from the Astral Plane.
\item \psionicList{Genesis}\textsuperscript{X}: You instigate a new demiplane on the Astral Plane.

\psionicList{True Creation}\textsuperscript{X}: As \psionic{psionic major creation}, except items are completely real.
\end{enumerate*}



\subsection{Telepath Discipline Powers {\normalsize(Telepathy)}}
\begin{enumerate*}
\item \psionicList{Charm, Psionic}\textsuperscript{A}: Makes one person your friend.

\psionicList{Mindlink}\textsuperscript{A}: You forge a limited mental bond with another creature.

\item \psionicList{Aversion}\textsuperscript{A}: Subject has aversion you specify.

\psionicList{Brain Lock}\textsuperscript{A}: Subject cannot move or take any mental actions.

\psionicList{Read Thoughts}: Detect surface thoughts of creatures in range.

\psionicList{Suggestion, Psionic}\textsuperscript{A}: Compels subject to follow stated course of 
action.
\item \psionicList{Crisis of Breath}\textsuperscript{A}: Disrupt subject's breathing.

\psionicList{Empathic Transfer, Hostile}\textsuperscript{A}: Your touch transfers your hurt to 
another.

\psionicList{False Sensory Input}\textsuperscript{A}: Subject sees what isn't there.

\item \psionicList{Dominate, Psionic}\textsuperscript{A}: Control target telepathically.

\psionicList{Hallucination}\textsuperscript{A}: Phantasm cause psychosomatic damage. %

\psionicList{Mindlink, Thieving}\textsuperscript{A}: Borrow knowledge of a subject's power.

\psionicList{Modify Memory, Psionic}: Changes 5 minutes of subject's memories.

\psionicList{Schism}: Your partitioned mind can manifest lower level powers.

\item \psionicList{Metaconcert}\textsuperscript{A}: Mental concert of two or more increases the total power of the participants.

\psionicList{Mind Probe}: You discover the subject's secret thoughts.

\item \psionicList{Mind Switch}\textsuperscript{AX}: You switch minds with another.
\item \psionicList{Crisis of Life}\textsuperscript{A}: Stop subject's heart.
\item \psionicList{Mind Seed}\textsuperscript{X}: Subject slowly becomes you.
\item \psionicList{Mind Switch, True}\textsuperscript{X}: A permanent brain swap.

\psionicList{Psychic Chirurgery}\textsuperscript{X}: You repair psychic damage or impart 
knowledge of new powers.
\end{enumerate*}
\section{Psychic Warrior Powers}




\subsection{1st-Level Psychic Warrior Powers}

\psionicList{Astral Traveler}: Enable yourself or another to join an astral caravan-enabled trip.

\psionicList{Biofeedback}\textsuperscript{A}: Gain DR 2/--.

\psionicList{Bioflexibility}: Gain +10 competence bonus on \skill{Escape Artist} checks. %

\psionicList{Bite of the Wolf}: Gain bite attack for 1d8 damage.

\psionicList{Burst}: Gain +3 m to speed this round.

\psionicList{Call Weaponry}\textsuperscript{A}: Create temporary weapon.

\psionicList{Cast Missiles}: You can launch missiles without a bow or other weapon. %

\psionicList{Catfall}\textsuperscript{A}: Instantly save yourself from a fall.

\psionicList{Chameleon}: Gain +10 enhancement bonus on \skill{Hide} checks.

\psionicList{Claws of the Beast}\textsuperscript{A}: Your hands become deadly claws.

\psionicList{Compression}\textsuperscript{A}: You grow smaller.

\psionicList{Conceal Thoughts}: You conceal your motives.

\psionicList{Deflect Strike}: You psychokinetically deflect the next attack of a creature within range. %

\psionicList{Detect Psionics}: You detect the presence of psionics.

\psionicList{Dissipating Touch}\textsuperscript{A}: Touch deals 1d6 damage.

\psionicList{Distract}: Subject gets -4 on Listen, Search, Sense Motive, and Spot checks.

\psionicList{Elfsight}: Gain low-light vision, +2 bonus on Search and Spot checks, and notice secret doors.

\psionicList{Empty Mind}\textsuperscript{A}: Gain +2 on Will saves until your next action.

\psionicList{Expansion}\textsuperscript{A}: Become one size category larger.

\psionicList{Float}: Buoy yourself in water or other liquid.

\psionicList{Force Screen}\textsuperscript{A}: Invisible disc provides +4 shield bonus to AC.

\psionicList{Grip of Iron}\textsuperscript{A}: Your iron grip gives +4 bonus on grapple checks.

\psionicList{Hammer}\textsuperscript{A}: Melee touch attack deals 1d8/round.

\psionicList{Inertial Armor}\textsuperscript{A}: Tangible field of force provides you with +4 armor bonus to AC.

\psionicList{Metaphysical Claw}\textsuperscript{A}: Your natural weapon gains +1 bonus.

\psionicList{Metaphysical Weapon}\textsuperscript{A}: Weapon gains +1 bonus.

\psionicList{My Light}\textsuperscript{A}: Your eyes emit 6-m cone of light.

\psionicList{Precognition, Defensive}\textsuperscript{A}: Gain +1 insight bonus to AC and saving throws.

\psionicList{Precognition, Offensive}\textsuperscript{A}: Gain +1 insight bonus on your attack rolls.

\psionicList{Prescience, Offensive}\textsuperscript{A}: Gain +2 insight bonus on your damage rolls.

\psionicList{Prevenom Weapon}\textsuperscript{A}: Your weapon is mildly venomous.

\psionicList{Prevenom}\textsuperscript{A}: Your claws gain a poison coating.

\psionicList{Psionic Draw}: Instantly draw a weapon. %

\psionicList{Skate}: Subject slides skillfully along the ground.

\psionicList{Stomp}\textsuperscript{A}: Subjects fall prone and take 1d4 nonlethal damage.

\psionicList{Synesthete}: You receive one kind of sense when another sense is stimulated.

\psionicList{Tattoo Animation}\textsuperscript{A}: Animates your tattoos or steals another's. %

\psionicList{Thicken Skin}\textsuperscript{A}: Gain +1 enhancement bonus to your AC for 10 min./level.

\psionicList{Vigor}\textsuperscript{A}: Gain 5 temporary hit points.

\psionicList{Wild Leap}\textsuperscript{A}: Make an additional leap and gain a bonus to Jump checks. %




\subsection{2nd-Level Psychic Warrior Powers}

\psionicList{Animal Affinity}\textsuperscript{A}: Gain +4 enhancement to one ability.

\psionicList{Antidote Simulation}\textsuperscript{A}: Detoxifies venom in your system. %

\psionicList{Body Adjustment}\textsuperscript{A}: Heal 1d12 damage.

\psionicList{Body Equilibrium}: You can walk on nonsolid surfaces.

\psionicList{Body Purification}\textsuperscript{A}: Restore 2 points of ability damage.

\psionicList{Concealing Amorpha}: Quasi-real membrane grants you concealment.

\psionicList{Darkvision, Psionic}: See 60 ft. in total darkness.

\psionicList{Detect Hostile Intent}: You can detect hostile creatures within 30 ft. of you.

\psionicList{Dimension Swap}\textsuperscript{A}: You and an ally switch positions.

\psionicList{Dissolving Touch}\textsuperscript{A}: Your touch deals 4d6 acid damage.

\psionicList{Dissolving Weapon}\textsuperscript{A}: Your weapon deals 4d6 acid damage.

\psionicList{Empathic Transfer}\textsuperscript{A}: Transfer another's wounds to yourself.

\psionicList{Energy Adaptation, Specified}\textsuperscript{A}: Gain resistance 10 to one energy type.

\psionicList{Feat Leech}\textsuperscript{A}: Borrow another's psionic or metapsionic feats.

\psionicList{Hustle}: Instantly gain a move action.

\psionicList{Levitate, Psionic}: Subject moves up and down at your direction.

\psionicList{Lion's Charge, Psionic}\textsuperscript{A}: You can make full attack in same round you charge.

\psionicList{Painful Strike}\textsuperscript{A}: Your natural weapons deal an extra 1d6 nonlethal damage.

\psionicList{Prowess}: Instantly gain another attack of opportunity.

\psionicList{Psionic Scent}: Gain the scent ability.

\psionicList{Return Missile}\textsuperscript{A}: Make one weapon return to you after thrown. %

\psionicList{Share Strength}\textsuperscript{A}: Temporarily transfer your Strength to another. %

\psionicList{Strength of My Enemy}\textsuperscript{A}: Siphon away your enemy's strength and grow stronger.

\psionicList{Sustenance}: You can go without food and water for one day.

\psionicList{Thought Shield}\textsuperscript{A}: Gain PR 13 against mind-affecting powers.

\psionicList{Wall Walker}: Grants ability to walk on walls and ceilings.



\subsection{3rd-Level Psychic Warrior Powers}

\psionicList{Accelerate}\textsuperscript{A}: Move faster, +1 on attack rolls, AC, and Reflex saves. %

\psionicList{Claws of the Vampire}: Heal half of your claw's base damage.

\psionicList{Concealing Amorpha, Greater}: Quasi-real membrane grants you total concealment.

\psionicList{Danger Sense}\textsuperscript{A}: Gain +4 bonus against traps.

\psionicList{Death Field}\textsuperscript{A}: Release an energy burst from the Gray that drains vital energy. %

\psionicList{Dimension Slide}\textsuperscript{A}: Teleports you very short distance.

\psionicList{Duodimensional Claw}: Increases your natural weapon's threat range.

\psionicList{Ectoplasmic Form}: You gain benefits of being insubstantial and can fly slowly.

\psionicList{Empathic Feedback}\textsuperscript{A}: When you are hit in melee, your attacker takes damage.

\psionicList{Empathic Transfer, Hostile}\textsuperscript{A}: Your touch transfers your hurt to another.

\psionicList{Escape Detection}: You become difficult to detect with clairsentience powers.

\psionicList{Evade Burst}\textsuperscript{A}: You take no damage from a burst on a successful Reflex save.

\psionicList{Exhalation of the Black Dragon}\textsuperscript{A}: Your acid breath deals 3d6 damage to a close target.

\psionicList{Graft Weapon}: Your hand is replaced seamlessly by your weapon.

\psionicList{Keen Edge, Psionic}: Doubles normal weapon's threat range.

\psionicList{Mental Barrier}\textsuperscript{A}: Gain +4 deflection bonus to AC until your next action.

\psionicList{Ubiquitous Vision}: You have all-around vision.

\psionicList{Vampiric Blade}: You heal half of your base weapon damage.




\subsection{4th-Level Psychic Warrior Powers}

\psionicList{Claw of Energy}: Your claws deal additional energy damage.

\psionicList{Dimension Door, Psionic}: Teleports you short distance.

\psionicList{Energy Adaptation}\textsuperscript{A}: Your body converts energy to harmless light.

\psionicList{Freedom of Movement, Psionic}: You cannot be held or otherwise rendered immobile.

\psionicList{Immovability}\textsuperscript{A}: You are almost impossible to move and gain DR 15/-.

\psionicList{Inertial Barrier}: Gain DR 5/--.

\psionicList{Psychic Vampire}: Touch attack drains 2 power points/level from foe.

\psionicList{Shadow Jump}\textsuperscript{A}: Jump into shadow to travel rapidly. %

\psionicList{Steadfast Perception}: Gain immunity to illusory effects, +6 bonus on \skill{Spot} and \skill{Search} checks.

\psionicList{Truevenom}: Your natural weapons are covered in horrible poison.

\psionicList{Truevenom Weapon}: Your weapon is horribly poisonous.

\psionicList{Weapon of Energy}: Weapon deals additional energy damage.




\subsection{5th-Level Psychic Warrior Powers}

\psionicList{Adapt Body}: Your body automatically adapts to hostile environments.

\psionicList{Catapsi}\textsuperscript{A}: Psychic static inhibits power manifestation.

\psionicList{Metaconcert}\textsuperscript{A}: Mental concert of two or more increases the total power of the participants.

\psionicList{Nerve Manipulation}\textsuperscript{A}: Disrupts a creature nervous system. %

\psionicList{Oak Body}\textsuperscript{A}: Your body becomes as hard as oak.

\psionicList{Psychofeedback}: Boost Str, Dex, or Con at the expense of one or more other scores.




\subsection{6th-Level Psychic Warrior Powers}

\psionicList{Breath of the Black Dragon}\textsuperscript{A}: Breathe acid for 11d6 damage.

\psionicList{Dispelling Buffer}: You are buffered from one \psionic{dispel psionics} effect.

\psionicList{Form of Doom}\textsuperscript{A}: You transform into a frightening tentacled beast.

\psionicList{Mind Blank, Personal}: You are immune to scrying and mental effects.

\psionicList{Poison Simulation}\textsuperscript{A}: Coat surface with potent poisons. %

\psionicList{Suspend Life}: Put yourself into a state akin to suspended animation.

