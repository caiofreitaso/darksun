\section{Prerequisites}
Some feats have prerequisites. Your character must have the indicated ability score, class feature, feat, skill, base attack bonus, or other quality designated in order to select or use that feat. A character can gain a feat at the same level at which he or she gains the prerequisite.

A character can’t use a feat if he or she has lost a prerequisite.

\section{Types Of Feats}
Some feats are general, meaning that no special rules govern them as a group. Others are item creation feats, which allow spellcasters to create magic items of all sorts. A metamagic feat lets a spellcaster prepare and cast a spell with greater effect, albeit as if the spell were a higher spell level than it actually is.

\subsection{Fighter Bonus Feats}
Any feat designated as a fighter feat can be selected as a fighter’s bonus feat. This designation does not restrict characters of other classes from selecting these feats, assuming that they meet any prerequisites.

\subsection{Item Creation Feats}
An item creation feat lets a spellcaster create a magic item of a certain type. Regardless of the type of items they involve, the various item creation feats all have certain features in common.

\textbf{XP Cost:} Experience that the spellcaster would normally keep is expended when making a magic item. The XP cost equals 1/25 of the cost of the item in gold pieces. A character cannot spend so much XP on an item that he or she loses a level. However, upon gaining enough XP to attain a new level, he or she can immediately expend XP on creating an item rather than keeping the XP to advance a level.

\textbf{Raw Materials Cost:} The cost of creating a magic item equals one-half the sale cost of the item.

Using an item creation feat also requires access to a laboratory or magical workshop, special tools, and so on. A character generally has access to what he or she needs unless unusual circumstances apply.

\textbf{Time:} The time to create a magic item depends on the feat and the cost of the item. The minimum time is one day.

\textbf{Item Cost:} Brew Potion, Craft Wand, and Scribe Scroll create items that directly reproduce spell effects, and the power of these items depends on their caster level---that is, a spell from such an item has the power it would have if cast by a spellcaster of that level. The price of these items (and thus the XP cost and the cost of the raw materials) also depends on the caster level. The caster level must be high enough that the spellcaster creating the item can cast the spell at that level. To find the final price in each case, multiply the caster level by the spell level, then multiply the result by a constant, as shown below:

\textit{Scrolls:} Base price = spell level $\times$ caster level $\times$ 25 gp.

\textit{Potions:} Base price = spell level $\times$ caster level $\times$ 50 gp.

\textit{Wands:} Base price = spell level $\times$ caster level $\times$ 750 gp.

A 0-level spell is considered to have a spell level of \onehalf for the purpose of this calculation.

\textbf{Extra Costs:} Any potion, scroll, or wand that stores a spell with a costly material component or an XP cost also carries a commensurate cost. For potions and scrolls, the creator must expend the material component or pay the XP cost when creating the item.

For a wand, the creator must expend fifty copies of the material component or pay fifty times the XP cost.

Some magic items similarly incur extra costs in material components or XP, as noted in their descriptions.

\subsection{Metamagic Feats}
As a spellcaster’s knowledge of magic grows, she can learn to cast spells in ways slightly different from the ways in which the spells were originally designed or learned. Preparing and casting a spell in such a way is harder than normal but, thanks to metamagic feats, at least it is possible. Spells modified by a metamagic feat use a spell slot higher than normal. This does not change the level of the spell, so the DC for saving throws against it does not go up.

\textbf{Wizards and Divine Spellcasters:} Wizards and divine spellcasters must prepare their spells in advance. During preparation, the character chooses which spells to prepare with metamagic feats (and thus which ones take up higher-level spell slots than normal).

\textbf{Templars:} Templars choose spells as they cast them. They can choose when they cast their spells whether to apply their metamagic feats to improve them. As with other spellcasters, the improved spell uses up a higher-level spell slot. But because the sorcerer or bard has not prepared the spell in a metamagic form in advance, he must apply the metamagic feat on the spot. Therefore, such a character must also take more time to cast a metamagic spell (one enhanced by a metamagic feat) than he does to cast a regular spell. If the spell’s normal casting time is 1 standard action, casting a metamagic version is a full-round action for a sorcerer or bard. (This isn’t the same as a 1-round casting time.)

For a spell with a longer casting time, it takes an extra full-round action to cast the spell.

\textbf{Spontaneous Casting and Metamagic Feats:} A cleric spontaneously casting a cure or inflict spell can cast a metamagic version of it instead. Extra time is also required in this case. Casting a 1-action metamagic spell spontaneously is a full-round action, and a spell with a longer casting time takes an extra full-round action to cast.

\textbf{Effects of Metamagic Feats on a Spell:} In all ways, a metamagic spell operates at its original spell level, even though it is prepared and cast as a higher-level spell. Saving throw modifications are not changed unless stated otherwise in the feat description.

The modifications made by these feats only apply to spells cast directly by the feat user. A spellcaster can’t use a metamagic feat to alter a spell being cast from a wand, scroll, or other device.

Metamagic feats that eliminate components of a spell don’t eliminate the attack of opportunity provoked by casting a spell while threatened. However, casting a spell modified by Quicken Spell does not provoke an attack of opportunity.

Metamagic feats cannot be used with all spells. See the specific feat descriptions for the spells that a particular feat can’t modify.

\textbf{Multiple Metamagic Feats on a Spell:} A spellcaster can apply multiple metamagic feats to a single spell. Changes to its level are cumulative. You can’t apply the same metamagic feat more than once to a single spell.

\textbf{Magic Items and Metamagic Spells:} With the right item creation feat, you can store a metamagic version of a spell in a scroll, potion, or wand. Level limits for potions and wands apply to the spell’s higher spell level (after the application of the metamagic feat). A character doesn’t need the metamagic feat to activate an item storing a metamagic version of a spell.

\textbf{Counterspelling Metamagic Spells:} Whether or not a spell has been enhanced by a metamagic feat does not affect its vulnerability to counterspelling or its ability to counterspell another spell.

\section{Feat Descriptions}
Here is the format for feat descriptions.

\subsection{Feat Name {\normalsize[Type Of Feat]}}
\textbf{Prerequisite:} A minimum ability score, another feat or feats, a minimum base attack bonus, a minimum number of ranks in one or more skills, or a class level that a character must have in order to acquire this feat. This entry is absent if a feat has no prerequisite. A feat may have more than one prerequisite.

\textbf{Benefit:} What the feat enables the character (``you'' in the feat description) to do. If a character has the same feat more than once, its benefits do not stack unless indicated otherwise in the description.

In general, having a feat twice is the same as having it once.

\textbf{Normal:} What a character who does not have this feat is limited to or restricted from doing. If not having the feat causes no particular drawback, this entry is absent.

\textbf{Special:} Additional facts about the feat that may be helpful when you decide whether to acquire the feat.