\section{Prerequisites}
Some feats have prerequisites. Your character must have the indicated ability score, class feature, feat, skill, base attack bonus, or other quality designated in order to select or use that feat. A character can gain a feat at the same level at which he or she gains the prerequisite.

A character can’t use a feat if he or she has lost a prerequisite.

\section{Types Of Feats}
Some feats are general, meaning that no special rules govern them as a group. Others are item creation feats, which allow spellcasters to create magic items of all sorts. A metamagic feat lets a spellcaster prepare and cast a spell with greater effect, albeit as if the spell were a higher spell level than it actually is.

\subsection{Fighter Bonus Feats}
Any feat designated as a fighter feat can be selected as a fighter’s bonus feat. This designation does not restrict characters of other classes from selecting these feats, assuming that they meet any prerequisites.

\subsection{Item Creation Feats}
An item creation feat lets a spellcaster create a magic item of a certain type. Regardless of the type of items they involve, the various item creation feats all have certain features in common.

\textbf{XP Cost:} Experience that the spellcaster would normally keep is expended when making a magic item. The XP cost equals 1/25 of the cost of the item in gold pieces. A character cannot spend so much XP on an item that he or she loses a level. However, upon gaining enough XP to attain a new level, he or she can immediately expend XP on creating an item rather than keeping the XP to advance a level.

\textbf{Raw Materials Cost:} The cost of creating a magic item equals one-half the sale cost of the item.

Using an item creation feat also requires access to a laboratory or magical workshop, special tools, and so on. A character generally has access to what he or she needs unless unusual circumstances apply.

\textbf{Time:} The time to create a magic item depends on the feat and the cost of the item. The minimum time is one day.

\textbf{Item Cost:} Brew Potion, Craft Wand, and Scribe Scroll create items that directly reproduce spell effects, and the power of these items depends on their caster level---that is, a spell from such an item has the power it would have if cast by a spellcaster of that level. The price of these items (and thus the XP cost and the cost of the raw materials) also depends on the caster level. The caster level must be high enough that the spellcaster creating the item can cast the spell at that level. To find the final price in each case, multiply the caster level by the spell level, then multiply the result by a constant, as shown below:

\textit{Scrolls:} Base price = spell level $\times$ caster level $\times$ 25 gp.

\textit{Potions:} Base price = spell level $\times$ caster level $\times$ 50 gp.

\textit{Wands:} Base price = spell level $\times$ caster level $\times$ 750 gp.

A 0-level spell is considered to have a spell level of \onehalf for the purpose of this calculation.

\textbf{Extra Costs:} Any potion, scroll, or wand that stores a spell with a costly material component or an XP cost also carries a commensurate cost. For potions and scrolls, the creator must expend the material component or pay the XP cost when creating the item.

For a wand, the creator must expend fifty copies of the material component or pay fifty times the XP cost.

Some magic items similarly incur extra costs in material components or XP, as noted in their descriptions.

\subsection{Metamagic Feats}
As a spellcaster’s knowledge of magic grows, she can learn to cast spells in ways slightly different from the ways in which the spells were originally designed or learned. Preparing and casting a spell in such a way is harder than normal but, thanks to metamagic feats, at least it is possible. Spells modified by a metamagic feat use a spell slot higher than normal. This does not change the level of the spell, so the DC for saving throws against it does not go up.

\textbf{Wizards and Divine Spellcasters:} Wizards and divine spellcasters must prepare their spells in advance. During preparation, the character chooses which spells to prepare with metamagic feats (and thus which ones take up higher-level spell slots than normal).

\textbf{Templars:} Templars choose spells as they cast them. They can choose when they cast their spells whether to apply their metamagic feats to improve them. As with other spellcasters, the improved spell uses up a higher-level spell slot. But because the templar has not prepared the spell in a metamagic form in advance, he must apply the metamagic feat on the spot. Therefore, such a character must also take more time to cast a metamagic spell (one enhanced by a metamagic feat) than he does to cast a regular spell. If the spell’s normal casting time is 1 standard action, casting a metamagic version is a full-round action for a templar. (This isn’t the same as a 1-round casting time.)

For a spell with a longer casting time, it takes an extra full-round action to cast the spell.

\textbf{Spontaneous Casting and Metamagic Feats:} A cleric spontaneously casting a cure or inflict spell can cast a metamagic version of it instead. Extra time is also required in this case. Casting a 1-action metamagic spell spontaneously is a full-round action, and a spell with a longer casting time takes an extra full-round action to cast.

\textbf{Effects of Metamagic Feats on a Spell:} In all ways, a metamagic spell operates at its original spell level, even though it is prepared and cast as a higher-level spell. Saving throw modifications are not changed unless stated otherwise in the feat description.

The modifications made by these feats only apply to spells cast directly by the feat user. A spellcaster can’t use a metamagic feat to alter a spell being cast from a wand, scroll, or other device.

Metamagic feats that eliminate components of a spell don’t eliminate the attack of opportunity provoked by casting a spell while threatened. However, casting a spell modified by Quicken Spell does not provoke an attack of opportunity.

Metamagic feats cannot be used with all spells. See the specific feat descriptions for the spells that a particular feat can’t modify.

\textbf{Multiple Metamagic Feats on a Spell:} A spellcaster can apply multiple metamagic feats to a single spell. Changes to its level are cumulative. You can’t apply the same metamagic feat more than once to a single spell.

\textbf{Magic Items and Metamagic Spells:} With the right item creation feat, you can store a metamagic version of a spell in a scroll, potion, or wand. Level limits for potions and wands apply to the spell’s higher spell level (after the application of the metamagic feat). A character doesn’t need the metamagic feat to activate an item storing a metamagic version of a spell.

\textbf{Counterspelling Metamagic Spells:} Whether or not a spell has been enhanced by a metamagic feat does not affect its vulnerability to counterspelling or its ability to counterspell another spell.

\subsection{Psionic Feats}
Psionic feats are available only to characters and creatures with the ability to manifest powers. (In other words, they either have a power point reserve or have psi-like abilities.)

Because psionic feats are supernatural abilities---a departure from the general rule that feats do not grant supernatural abilities---they cannot be disrupted in combat (as powers can be) and generally do not provoke attacks of opportunity (except as noted in their descriptions). Supernatural abilities are not subject to power resistance and cannot be dispelled; however, they do not function in areas where psionics is suppressed, such as a null psionics field. Leaving such an area immediately allows psionic feats to be used.

Many psionic feats can be used only when you are psionically focused; others require you to expend your psionic focus to gain their benefit. Expending your psionic focus does not require an action; it is part of another action (such as using a feat). When you expend your psionic focus, it applies only to the action for which you expended it.

\subsection{Psionic Item Creation Feats}
Manifesters can use their personal power to create lasting psionic items. Doing so, however, is draining. A manifester must put a little of himself or herself into every psionic item he or she creates.

A psionic item creation feat lets a manifester create a psionic item of a certain type. Regardless of the type of items they involve, the various item creation feats all have certain features in common.

\textbf{XP Cost:} Power and energy that the manifester would normally keep is expended when making a psionic item. The experience point cost of using a psionic item creation feat equals 1/25 the cost of the item in gold pieces. A character cannot spend so much XP on an item that he or she loses a level. However, upon gaining enough XP to attain a new level, he or she can immediately expend XP on creating an item rather than keeping the XP to advance a level.

\textbf{Raw Materials Cost:} Creating a psionic item requires costly components, most of which are consumed in the process. The cost of these materials equals ½ the cost of the item.

Using a psionic item creation feat also requires access to a laboratory or psionic workshop, special tools, and other equipment. A character generally has access to what he or she needs unless unusual circumstances apply (such as if he’s traveling far from home).

\textbf{Time:} The time to create a psionic item depends on the feat and the cost of the item. The minimum time is one day.

\textbf{Item Cost:} Craft Dorje, Imprint Stone, and Scribe Tattoo create items that directly reproduce the effects of powers, and the strength of these items depends on their manifester level---that is, a power from such an item has the strength it would have if manifested by a manifester of that level. Often, that is the minimum manifester level necessary to manifest the power. (Randomly discovered items usually follow this rule.) However, when making such an item, the item’s strength can be set higher than the minimum. Any time a character creates an item using a power augmented by spending additional power points, the character’s effective manifester level for the purpose of calculating the item’s cost increases by 1 for each 1 additional power point spent. (Augmentation is a feature of many powers that allows the power to be amplified in various ways if additional power points are spent.) All other level-dependent parameters of the power forged into the item are set according to the effective manifester level.

The price of psionic items (and thus the XP cost and the cost of the raw materials) depends on the level of the power and a character’s manifester level. The character’s manifester level must be high enough that the item creator can manifest the power at the chosen level. To find the final price in each case, multiply the character’s manifester level by the power level, then multiply the result by a constant, as shown below.

\textit{Power Stones:} Base price = power level $\times$ manifester level $\times$ 25 gp.

\textit{Psionic Tattoos:} Base price = power level $\times$ manifester level $\times$ 50 gp.

\textit{Dorjes:} Base price = power level $\times$ manifester level $\times$ 750 gp.

\textbf{Extra Costs:} Any dorje, power stone, or psionic tattoo that stores a power with an XP cost also carries a commensurate cost.

For psionic tattoos and power stones, the creator must pay the XP cost when creating the item. For a dorje, the creator must pay fifty times the XP cost.

Some psionic items similarly incur extra costs in XP, as noted in their descriptions.

\subsection{Metapsionic Feats}
As a manifester’s knowledge of psionics grows, he can learn to manifest powers in ways slightly different from how the powers were originally designed or learned. Of course, manifesting a power while using a metapsionic feat is more expensive than manifesting the power normally.

\textbf{Manifesting Time:} Powers manifested using metapsionic feats take the same time as manifesting the powers normally unless the feat description specifically says otherwise.

\textbf{Manifestation Cost:} To use a metapsionic feat, a psionic character must both expend his psionic focus (see the Concentration skill description) and pay an increased power point cost as given in the feat description.

\textbf{Limits on Use:} As with all powers, you cannot spend more power points on a power than your manifester level. Metapsionic feats merely let you manifest powers in different ways; they do not let you violate this rule.

\textbf{Effects of Metapsionic Feats on a Power:} In all ways, a metapsionic power operates at its original power level, even though it costs additional power points. The modifications to a power made by a metapsionic feat have only their noted effect on the power. A manifester can’t use a metapsionic feat to alter a power being cast from a power stone, dorje, or other device.

Manifesting a power modified by the Quicken Power feat does not provoke attacks of opportunity.

Some metapsionic feats apply only to certain powers, as described in each specific feat entry.

\textbf{Psionic Items and Metapsionic Powers:} With the right psionic item creation feat, you can store a metapsionic power in a power stone, psionic tattoo, or dorje. Level limits for psionic tattoos apply to the power’s higher metapsionic level.

A character doesn’t need the appropriate metapsionic feat to activate an item in which a metapsionic power is stored, but does need the metapsionic feat to create such an item.


\section{Feat Descriptions}
Here is the format for feat descriptions.

\subsection{Feat Name {\normalsize[Type Of Feat]}}
\textbf{Prerequisite:} A minimum ability score, another feat or feats, a minimum base attack bonus, a minimum number of ranks in one or more skills, or a class level that a character must have in order to acquire this feat. This entry is absent if a feat has no prerequisite. A feat may have more than one prerequisite.

\textbf{Benefit:} What the feat enables the character (``you'' in the feat description) to do. If a character has the same feat more than once, its benefits do not stack unless indicated otherwise in the description.

In general, having a feat twice is the same as having it once.

\textbf{Normal:} What a character who does not have this feat is limited to or restricted from doing. If not having the feat causes no particular drawback, this entry is absent.

\textbf{Special:} Additional facts about the feat that may be helpful when you decide whether to acquire the feat.

\FeatTable{General}{
	\feat{Ability Focus} & Special attack & +2 DC to chosen special attack \\
	\feat{Ancestral Knowledge} & Int 13, \skill{Knowledge} (history) 10 ranks & +10 bonus on \skill{Knowledge} (history) checks or bardic knowledge checks about a time period\\
	\feat{Antipsionic Magic} & \skill{Spellcraft} 5 ranks & +2 bonus on caster level check to defeat power resistence\\
	\feat{Arena Clamor} & Cha 13, \feat{Improved Critical}, Perform 5 ranks & +2 bonus on attack rolls after a critical hit\\
	\feat{Armor Proficiency (Light)} && No armor check penalty on attack rolls\\
	~\feat{Armor Proficiency (Medium)} & Armor Proficiency (Light) & No armor check penalty on attack rolls\\
	~ ~\feat{Armor Proficiency (Heavy)} & Armor Proficiency (Medium) & No armor check penalty on attack rolls\\
	\feat{Augment Summoning} & \feat{Spell Focus} (conjuration) & Summoned creatures gain +4 Str, +4 Con\\
	\feat{Brutal Attack} & Cha 13, \feat{Improved Critical}, \skill{Perform} 5 ranks & Enemies become shaken after a critical hit\\
	\feat{Bug Trainer} & \skill{Handle Animal} 5 ranks, \skill{Knowledge} (nature) 5 ranks & Use \skill{Handle Animal} on vermin\\
	\feat{Chaotic Mind} & Chaotic alignment, Cha 15 & Creatures do not gain insight bonus against you\\
	\feat{Cloak Dance} & \skill{Hide} 10 ranks, \skill{Perform} (dance) 2 ranks & Gain concealment for one turn\\
	\feat{Closed Mind} && +2 bonus on saves to resist powers\\
	\feat{Combat Casting} && +4 bonus on checks for defensive casting\\
	\feat{Cornered Fighter} & Base attack bonus +5 & +2 on attack rolls and AC when flanked \\
	\feat{Deadly Precision} & Dex 15, base attack bonus +5 & Reroll 1 on sneak attack's dice\\
	\feat{Defender of the Land} & Wild shape class feature & Bonus damage in spells cast against defilers\\
	\feat{Dissimulated} & Int 13, Cha 13, \skill{Bluff} 5 ranks & Add Int modifier to \skill{Bluff} checks\\
	\feat{Drake's Child} & Str 13, Wis 13 & +1 bonus to Will and Fortitude saves\\
	\feat{Elemental Cleansing} & Ability to turn or rebuke undead & Turn undead deals 2d6 energy damage\\
	\feat{Empower Spell-Like Ability} & Spell-like ability at caster level 6th or higher & +50\% damage to chosen spell-like ability\\
	\feat{Endurance} && +4 bonus on checks to resist nonlethal damage\\
	~ \feat{Diehard} & \feat{Endurance} & Become stable between $-1$ and $-9$ hit points\\
	\feat{Eschew Materials} && Cast spells without material components\\
	\feat{Extra Turning} & Ability to turn or rebuke creatures & Can use turn or rebuke 4 more times per day\\
	\feat{Faithful Follower} && +5 bonus on saves against fear if near an ally with \feat{Leadership} feat\\
	\feat{Favorite} & \feat{Secular Authority}, \skill{Diplomacy} 10 ranks & Can use \feat{Secular Autority} 4 more times per day\\
	\feat{Fearsome} & Str 15 & Use Str on \skill{Intimidade} checks\\
	\feat{Flyby Attack} & Fly speed & Take a standard action at any point during flight\\
	\feat{Gladiatorial Entertainer} & Gladiatorial performance class feature & Can use gladiatorial performance 4 more times per day\\
	\feat{Greasing the Wheels} & Cha 13, \skill{Diplomacy} 7 ranks, \skill{Knowledge} (local) 5 ranks & Use \skill{Diplomacy} to bribe a character\\
	\feat{Great Fortitude} && +2 bonus on Fortitude saves\\
	\feat{Grovel} & Eldaarich, \skill{Perform} 1 rank & +3 on \skill{Diplomacy} and \skill{Bluff} checks\\
	\feat{Hard as Rock} & Con 15, \feat{Diehard}, \feat{Great Fortitude} & Become immune to death from massive damage\\
	\feat{Hover} & Fly speed & Stay in place while flying\\
	\feat{Hostile Mind} & Cha 15 & Deal damage to telepathic manifesters\\
	\feat{Improved Counterspell} && Use spell from the same school to counterspell\\
	\feat{Improved Familiar} & Ability to acquire a new familiar & Acquire nonstandard familiar\\
	\feat{Improved Natural Armor} & Natural armor, Con 13 & +1 natural armor\\
	\feat{Improved Natural Attack} & Natural attack, base attack bonus +4 & Increase dice damage of chosen natural attack\\
	\feat{Improved Sigil} & Sigil ability, \skill{Diplomacy} 9 ranks & Use two 1st-level spells as spell-like ability\\
	\feat{Improved Turning} & Ability to turn or rebuke creatures & +1 level on turning checks\\
	\feat{Improviser} & Wis 13, base attack bonus +3 & Reduce penalty of improvised weapons\\
	\feat{Innate Hunter} & \feat{Track}, \skill{Survival} 5 ranks & +4 bonus on \skill{Survival} for hunting, +1 bonus on attack rolls versus animals\\
	\feat{Iron Will} && +2 bonus on Will saves\\
	~ \feat{Force of Will} & \feat{Iron Will} & Can make a Will save instead of a Fort or Ref save against psionics powers\\
	\feat{Jaguar Roar} & Cha 13, Draj, \skill{Intimidate} 9 ranks & Affected creatures become shaken for 2d4 rounds\\
	\feat{Kiltektet} && All \skill{Knowledge} skills are class skills\\
	\feat{Leadership} & Character level 6th & Gain followers and cohort\\
	\feat{Lightning Reflexes} && +2 bonus on Reflex saves\\
	\feat{Linguist} && Gain two spoken languages\\
	\feat{Martial Weapon Proficiency} && No penalties attacking with specific weapon\\
	\feat{Mastyrial Blood} & Con 13 & +4 bonus on saves against poison\\
	\feat{Mental Resistance} & Base Will save bonus +2 & 3/--- against psionic attacks\\
	\feat{Mind Over Body} & Con 13 & Heal ability damage more quickly\\
	\feat{Multiattack} & Three or more natural attacks & Reduce secondary attack's penalty\\
	\feat{Multiweapon Fighting} & Dex 13, three or more hands & Reduce penalty for fighting with multiple weapons\\
	\feat{Natural Spell} & Wis 13, wild shape ability & Cast spells while in wild shape\\
	\feat{Open Minded} && +5 skill points\\
	\feat{Path Dexter} & Preserver & +1 caster level for chosen abjuration/divination spells\\
	\feat{Path Sinister} & Defiler & +1 caster level for chosen evocation/necromancy spells\\
	\feat{Protective} && +4 bonus to saves to items\\
	\feat{Psionic Hole} & Con 15 & Foes lose psionic focus and power points\\
	\feat{Psionic Mimicry} & \skill{Bluff} 8 ranks, \skill{Knowledge} (psionics) 4 ranks, \skill{Psicraft} 4 ranks & Use \skill{Bluff} to disguise spell as psionics\\
	\feat{Psionic Schooling} && One psionic class becomes an additional favored class\\
}

\FeatTable{General}{
	\feat{Quicken Spell-Like Ability} & Spell-like ability at caster level 10th or higher & Use chosen spell-like ability as swift action\\
	\feat{Raised by Beasts} && Wild empathy for a kind of animal\\
	\feat{Rapid Metabolism} & Con 13 & Heal hit points more quickly\\
	\feat{Reckless Offense} & Base attack bonus +1 & Take $-4$ penalty to AC to gain +2 on melee attack rolls\\
	\feat{Reign of Terror} & \skill{Intimidate} 5 ranks, member of Takrits, Savak, or Neshtap Order & +4 bonus on \feat{Secular Authority} checks\\
	\feat{Run} && Move 5$\times$ normal speed, +4 bonus on \skill{Jump} checks\\
	\feat{Secular Authority} & Cha 13, \skill{Diplomacy} 6 ranks, \feat{Negotiator}, accepted into city-state's templarate & New uses for \skill{Diplomacy}\\
	\feat{Shield Proficiency} && No shield penalty on attack rolls\\
	~ \feat{Tower Shield Proficiency} && No shield penalty on attack rolls using tower shield\\
	\feat{Sidestep Charge} & Dex 13, \feat{Dodge} & +4 bonus to AC against charge attacks\\
	\feat{Simple Weapon Proficiency} && No $-4$ penalty on attack rolls for simple weapons\\
	\feat{Sniper} & Dex 13, \skill{Hide} 1 rank & +5 bonus on \skill{Hide} checks to stay hidden\\
	\feat{Skill Focus} && +3 bonus on checks of chosen skill\\
	\feat{Spell Focus} && +1 bonus to DC on spells of chosen school\\
	~ \feat{Greater Spell Focus} && +1 bonus to DC on spells of chosen school\\
	\feat{Spell Mastery} & Wizard level 1st & Can prepare some spells without spellbook\\
	\feat{Spell Penetration} && +2 bonus on caster level check to defeat spell resistence\\
	~ \feat{Greater Spell Penetration} & \feat{Spell Penetration} & +2 bonus on caster level check to defeat spell resistence\\
	\feat{Stand Still} & Str 13 & Use attack of opportunity to stop moving foe\\
	\feat{Toughness} && +3 hit points\\
	\feat{Track} && Use \skill{Survival} to track\\
	\feat{Wastelander} && +1 bonus on Fortitude saves, +2 bonus on \skill{Survival} checks\\
	\feat{Wild Talent} && Gain psionic powers and 2 power points\\
	\feat{Wingover} & Fly speed & Change flight direction using 10 ft of movement\\
}

\FeatTable[p{3cm}]{Skill}{
	\feat{Acrobatic} && +2 bonus on \skill{Jump} and \skill{Tumble} checks \\
	\feat{Agile} && +2 bonus on \skill{Balance} and \skill{Escape Artist} checks \\
	\feat{Alertness} && +2 bonus on \skill{Listen} and \skill{Spot} checks \\
	\feat{Animal Affinity} && +2 bonus on \skill{Handle Animal} and \skill{Ride} checks \\
	\feat{Athletic} && +2 bonus on \skill{Climb} and \skill{Swim} checks \\
	\feat{Autonomous} && +2 bonus on \skill{Autohypnosis} and \skill{Knowledge} (psionics) checks \\
	\feat{Deceitful} && +2 bonus on \skill{Disguise} and \skill{Forgery} checks \\
	\feat{Deft Hands} && +2 bonus on \skill{Sleight of Hand} and \skill{Use Rope} checks \\
	\feat{Diligent} && +2 bonus on \skill{Appraise} and Decipher \skill{Script} checks \\
	\feat{Field Officer} && +2 bonus on \skill{Diplomacy} and \skill{Knowledge} (warcraft) checks \\
	\feat{Investigator} && +2 bonus on \skill{Gather Information} and \skill{Search} checks \\
	\feat{Magical Aptitude} && +2 bonus on \skill{Spellcraft} and \skill{Use Magic Device} checks \\
	\feat{Negotiator} && +2 bonus on \skill{Diplomacy} and \skill{Sense Motive} checks \\
	\feat{Nimble Fingers} && +2 bonus on \skill{Disable Device} and \skill{Open Lock} checks \\
	\feat{Persuasive} && +2 bonus on \skill{Bluff} and \skill{Intimidate} checks \\
	\feat{Psionic Affinity} && +2 bonus on \skill{Psicraft} and \skill{Use Psionic Device} checks \\
	\feat{Self-Sufficient} && +2 bonus on \skill{Heal} and \skill{Survival} checks \\
	\feat{Stealthy} && +2 bonus on \skill{Hide} and \skill{Move Silently} checks \\
	\feat{Trader} && +2 bonus on \skill{Appraise} and \skill{Bluff} checks
}

\FeatTable[p{3cm}]{Racial}{
	\feat{Active Glands} & Thri-kreen & Use poison two additional times per day\\
	\feat{Advanced Antennae} & Thri-kreen & Gain scent ability\\
	\feat{Blend} & Thri-kreen & +3 on \skill{Hide} checks in sandy or arid terrain\\
	\feat{Blessed by the Ancestors} & Thri-kreen & +1 bonus on all saves\\
	\feat{Cannibalism Ritual} & Wis 13, halfling & Gain ability bonus for 1 day after devouring slain foe\\
	\feat{Dwarven Vision} & Mul & Gain darkvision 60 ft\\
	\feat{Elfeater} & Thri-kreen & +1 on attack rolls and +2 on some skill checks against elves\\
	\feat{Improved Gyth'sa} & Thri-kreen, Con 13 & Recover double hit points after a night's rest\\
	\feat{Longshanks} & Half-elf, both parents must be half-elves & +10 ft speed\\
	\feat{Tikchak} & Thri-kreen, character level 5th & Add Wis to \skill{Survival} checks, gain proficiency with chatkcha\\
	\feat{Tokchak} & Thri-kreen & Adjacent allies gain +1 bonus on Ref saves
}

\FeatTable[p{3cm}]{Regional}{
	\feat{Artisan} & {Nibenay, Raam, Urik} & +3 on \skill{Concentration} and \skill{Craft} checks\\
	\feat{Astrologer} & {Draj, Nibenay} & +3 on \skill{Knowledge} (nature) checks, +5 to avoid getting lost\\
	\feat{Companion} & {Kurn, Tyr} & Your aid another action grants +3 bonus\\
	\feat{Disciplined} & {Dwarf, Urik} & +1 to Will saves, +3 on \skill{Concentration} checks\\
	\feat{Elfish Eloy} & Half-elf, both parents must be half-elves & +3 on \skill{Hide} while aboveground\\
	\feat{Freedom} & {Tyr} & Take extra actions per day\\
	\feat{Giant Killer} & {Sea of Silt} & +4 to confirm criticals and +2 to AC against giants\\
	\feat{Jungle Fighter} & {Forest Ridge, Gulg} & +2 dodge bonus to AC in forests\\
	\feat{Legerdemain} & {Elf, Salt View} & +3 bonus on \skill{Open Lock} and \skill{Sleight of Hand} checks\\
	\feat{Mansabdar} & {Raam} & +1 to Fort saves, +3 bonus on \skill{Intimidate} checks\\
	\feat{Mekillothead} & {Draj, mul} & +1 to Will saves, +3 bonus on \skill{Intimidate} checks\\
	\feat{Metalsmith} & {Dwarf, Tyr} & You do not suffer $-5$ penalty to forge metal items\\
	\feat{Nature's Child} & {Gulg, halfling} & +3 on \skill{Knowledge} (nature) and \skill{Survival} checks\\
	\feat{Paranoid} & {Eldaarich} &  +1 to Ref saves, +3 bonus on \skill{Sense 
	motive} checks\\
	\feat{Performance Artist} & {Balic, Nibenay, Salt View} & +3 bonus on \skill{Knowledge} (local) and \skill{Perform} checks\\
	\feat{Tarandan Method} & {Raam} & +2 DC to powers from chosen discipline
}

\section{General Feats}

\Feat{Ancestral Knowledge}
{You know legends and facts about long past events that have been shrouded by the sands of time.}
{Int 13, \skill{Knowledge} (history) 10 ranks.}
{Choose one of the following time periods: Blue Age, Green Age, or Cleansing Wars. You gain a +10 on bonus on \skill{Knowledge} (history) checks or bardic knowledge checks to gain information about the chosen category.}{}
{You can take this feat more than once, but the bonus doesn't stack. Each time you take this feat, you choose another time period.}

\Feat{Arena Clamor}
{With your savage blows, you can make your companions give their best.}
{Cha 13, \feat{Improved Critical}, Perform 5 ranks.}
{Whenever you confirm a critical hit, all allies within a 60‐foot radius who have line of sight on you receive a +2 morale bonus on attack rolls for 1 round. This is a mind‐affecting effect. This effect is not cumulative. Characters cannot be affected more than once in this way in the same combat.}{}{}

\Feat{Armor Proficiency (Heavy)}{}
{Armor Proficiency (light), Armor Proficiency (medium).}
{See Armor Proficiency (light).}
{See Armor Proficiency (light).}
{Fighters, psychic warriors, and clerics automatically have Armor Proficiency (heavy) as a bonus feat. They need not select it.}

\Feat{Armor Proficiency (Light)}{}{}
{When you wear a type of armor with which you are proficient, the armor check penalty for that armor applies only to \skill{Balance}, \skill{Climb}, \skill{Escape Artist}, \skill{Hide}, \skill{Jump}, \skill{Move Silently}, \skill{Sleight of Hand}, and \skill{Tumble} checks.}
{A character who is wearing armor with which she is not proficient applies its armor check penalty to attack rolls and to all skill checks that involve moving, including Ride.}
{All characters except wizards, and psions automatically have Armor Proficiency (light) as a bonus feat. They need not select it.}

\Feat{Armor Proficiency (Medium)}{}
{Armor Proficiency (light).}
{See Armor Proficiency (light).}
{See Armor Proficiency (light).}
{Fighters, barbarians, gladiators, psychic warriors, clerics, druids, and templars automatically have Armor Proficiency (medium) as a bonus feat. They need not select it.}

\GFeat{Augment Summoning}
{\feat{Spell Focus} (conjuration).}
{Each creature you conjure with any summon spell gains a +4 enhancement bonus to Strength and Constitution for the duration of the spell that summoned it.}

\Feat{Brutal Attack}
{Your decisive attacks are especially frightening for those who watch.}
{Cha 13, \feat{Improved Critical}, \skill{Perform} 5 ranks.}
{Whenever you confirm a critical hit, all enemies within a 10‐foot radius who have line of sight on you must make a Will save (DC 10 + \onehalf your character level + your Cha modifier) or become shaken for a
number of rounds equal to your Cha modifier. This is a mind‐affecting fear effect.

Whether or not the save is successful, that creature cannot be affected again by the same character's brutal
attack for 24 hours.}{}{}

\Feat{Bug Trainer}
{You can train vermin creatures, such as kanks and cilops.}
{\skill{Handle Animal} 5 ranks, \skill{Knowledge} (nature) 5 ranks.}
{You can use the \skill{Handle Animal} skill for vermin as though they were animals with an Intelligence score of 1.}
{You can use the \skill{Handle Animal} skill only on creatures with an Intelligence score of 1 or 2.}
{}

\GFeat{Combat Casting}{}
{You get a +4 bonus on Concentration checks made to cast a spell or use a spell-like ability while on the defensive or while you are grappling or pinned.}

\Feat{Cornered Fighter}
{You fight better when you freedom is put at risk.}
{Base attack bonus +5.}
{You receive a +2 bonus on attack rolls and a +2 bonus to AC when fighting against opponents who flank you.}
{}{}

\Feat{Defender of the Land}
{You share power with the spirit from your guarded land, to nurture and protect the land to which the spirit is tied.}
{Wild shape class feature.}
{You receive a +1 caster level on spells you cast against defilers and your spells damage is increased by 1 per die against defilers.}
{}{}

\Feat{Diehard}{}
{\feat{Endurance}.}
{When reduced to between $-1$ and $-9$ hit points, you automatically become stable. You don't have to roll d\% to see if you lose 1 hit point each round.

When reduced to negative hit points, you may choose to act as if you were disabled, rather than dying. You must make this decision as soon as you are reduced to negative hit points (even if it isn't your turn). If you do not choose to act as if you were disabled, you immediately fall unconscious.

When using this feat, you can take either a single move or standard action each turn, but not both, and you cannot take a full round action. You can take a move action without further injuring yourself, but if you perform any standard action (or any other action deemed as strenuous, including some free actions, swift actions, or immediate actions, such as casting a quickened spell) you take 1 point of damage after completing the act. If you reach $-10$ hit points, you immediately die.}
{A character without this feat who is reduced to between $-1$ and $-9$ hit points is unconscious and dying.}
{}

\Feat{Dissimulated}
{Your ability to speak what others want to hear increases the credibility of your words.}
{Int 13, Cha 13, \skill{Bluff} 5 ranks.}
{In addition to your Charisma modifier, you can add your Intelligence modifier to your \skill{Bluff} checks.}
{}{}

\Feat{Drake's Child}
{You are what is known as a drake's child, an individual who shows both exceptional strength and wisdom.}
{Str 13, Wis 13.}
{You get a +1 bonus to Will saves and a +1 bonus to Fortitude saves. You gain an additional +1 bonus to saving throws against ability drain, ability damage, energy drain, and death effects.}
{}{}

\Feat{Endurance}{}
{You gain a +4 bonus on the following checks and saves: Swim checks made to resist nonlethal damage, Constitution checks made to continue running, Constitution checks made to avoid nonlethal damage from a forced march, Constitution checks made to hold your breath, Constitution checks made to avoid nonlethal damage from starvation or thirst, Fortitude saves made to avoid nonlethal damage from hot or cold environments, and Fortitude saves made to resist damage from suffocation. Also, you may sleep in light or medium armor without becoming fatigued.}
{A character without this feat who sleeps in medium or heavier armor is automatically fatigued the next day.}{}
{A ranger automatically gains Endurance as a bonus feat at 3rd level. He need not select it.}

\GFeat{Eschew Materials}{}
{You can cast any spell that has a material component costing 1 gp or less without needing that component. (The casting of the spell still provokes attacks of opportunity as normal.) If the spell requires a material component that costs more than 1 gp, you must have the material component at hand to cast the spell, just as normal.}

\Feat{Extra Turning}{}
{Ability to turn or rebuke creatures.}
{Each time you take this feat, you can use your ability to turn or rebuke creatures four more times per day than normal.

If you have the ability to turn or rebuke more than one kind of creature each of your turning or rebuking abilities gains four additional uses per day.}
{Without this feat, a character can typically turn or rebuke undead (or other creatures) a number of times per day equal to 3 + his or her Charisma modifier.}
{You can gain Extra Turning multiple times. Its effects stack. Each time you take the feat, you can use each of your turning or rebuking abilities four additional times per day.}

\GFeat{Great Fortitude}{}
{You get a +2 bonus on all Fortitude saving throws.}

\Feat{Greater Spell Focus}
{Choose a school of magic to which you already have applied the Spell Focus feat.}
{Add +1 to the Difficulty Class for all saving throws against spells from the school of magic you select. This bonus stacks with the bonus from Spell Focus.}{}
{You can gain this feat multiple times. Its effects do not stack. Each time you take the feat, it applies to a new school of magic to which you already have applied the Spell Focus feat.}{}

\GFeat{Greater Spell Penetration}
{\feat{Spell Penetration}.}
{You get a +2 bonus on caster level checks (1d20 + caster level) made to overcome a creature's spell resistance. This bonus stacks with the one from Spell Penetration.}

\Feat{Improved Counterspell}{}{}
{When counterspelling, you may use a spell of the same school that is one or more spell levels higher than the target spell.}
{Without this feat, you may counter a spell only with the same spell or with a spell specifically designated as countering the target spell.}{}

\Feat{Improved Familiar}
{This feat allows spellcasters to acquire a new familiar from a nonstandard list, but only when they could normally acquire a new familiar.}
{Ability to acquire a new familiar, sufficiently high level (see below).}
{When choosing a familiar, the creatures listed below are also available to the spellcaster. The spellcaster may choose a familiar with an alignment up to one step away on each of the alignment axes (lawful through chaotic, good through evil).

\Table{}{p{2cm} X Z{1.2cm}}{
\tableheader Familiar & \tableheader Condition & \tableheader Arcane Spellcaster Level\\
Black/Gray Touched & Ability to channel energy from the Black or the Grey & 3rd\\
Boneclaw, lesser &  Neutral Alignment & 3rd\\
Pterrax & Reptilian subtype or ability to manifest psionic powers & 3rd\\
Elemental incarnation & Matching subtype or patron element, or preserver & 5th\\
Paraelemental incarnation & Matching subtype or patron element, or defiler & 5th\\
Tagster & Preserver & 5th\\
Dagorran & Neutral alignment & 5th\\
Elemental, Small & Matching subtype or patron element, or preserver & 5th\\
Paraelemental, Small & Matching subtype or patron element, or defiler & 5th\\
Boneclaw, greater & Neutral alignment or ability to manifest psionic powers & 7th\\
Tigone & Neutral alignment or preserver & 7th\\
Tembo & Defiler or ability to manifest psionic powers & 7th\\
Wall walker & Neutral alignment or defiler & 7th\\
Psionocus & Ability to manifest psionic powers (The master must first create the psionocus.) & 7th
}

Apply the Elemental incarnation or Paraelemental incarnation template to a familiar from the standard list.

Improved familiars otherwise use the rules for regular familiars, with two exceptions: If the creature's type is something other than animal, its type does not change; and improved familiars do not gain the ability to speak with other creatures of their kind (although many of them already have the ability to communicate).}{}{}

\GFeat{Improved Turning}
{Ability to turn or rebuke creatures.}
{You turn or rebuke creatures as if you were one level higher than you are in the class that grants you the ability.}

\GFeat{Iron Will}{}
{You get a +2 bonus on all Will saving throws.}

\GFeat{Leadership}
{Character level 6th.}
{Having this feat enables the character to attract loyal companions and devoted followers, subordinates who assist her. See the table below for what sort of cohort and how many followers the character can recruit.

\Table{Leadership}{p{1.5cm} C *{6}{Z{.4cm}}}{
\rowcolor{white}
\multirow{2}{1.5cm}{\tableheader Leadership Score} & \multirow{2}{*}{\tableheader Cohort Level} & \multicolumn{6}{c}{\tableheader Number of Followers by Level}\\
\cmidrule[0.5pt]{3-8}
& & 1st & 2nd & 3rd & 4th & 5th & 6th\\
1 or lower & --- & --- & --- & --- & --- & --- & ---\\
2 & 1st & --- & --- & --- & --- & --- & ---\\
3 & 2nd & --- & --- & --- & --- & --- & ---\\
4 & 3rd & --- & --- & --- & --- & --- & ---\\
5 & 3rd & --- & --- & --- & --- & --- & ---\\
6 & 4th & --- & --- & --- & --- & --- & ---\\
7 & 5th & --- & --- & --- & --- & --- & ---\\
8 & 5th & --- & --- & --- & --- & --- & ---\\
9 & 6th & --- & --- & --- & --- & --- & ---\\
10 & 7th & 5 & --- & --- & --- & --- & ---\\
11 & 7th & 6 & --- & --- & --- & --- & ---\\
12 & 8th & 8 & --- & --- & --- & --- & ---\\
13 & 9th & 10 & 1 & --- & --- & --- & ---\\
14 & 10th & 15 & 1 & --- & --- & --- & ---\\
15 & 10th & 20 & 2 & 1 & --- & --- & ---\\
16 & 11th & 25 & 2 & 1 & --- & --- & ---\\
17 & 12th & 30 & 3 & 1 & 1 & --- & ---\\
18 & 12th & 35 & 3 & 1 & 1 & --- & ---\\
19 & 13th & 40 & 4 & 2 & 1 & 1 & ---\\
20 & 14th & 50 & 5 & 3 & 2 & 1 & ---\\
21 & 15th & 60 & 6 & 3 & 2 & 1 & 1\\
22 & 15th & 75 & 7 & 4 & 2 & 2 & 1\\
23 & 16th & 90 & 9 & 5 & 3 & 2 & 1\\
24 & 17th & 110 & 11 & 6 & 3 & 2 & 1\\
25 or higher & 17th & 135 & 13 & 7 & 4 & 2 & 2
}

\textit{Leadership Score:} A character's base Leadership score equals his level plus any Charisma modifier. In order to take into account negative Charisma modifiers, this table allows for very low Leadership scores, but the character must still be 6th level or higher in order to gain the Leadership feat. Outside factors can affect a character's Leadership score, as detailed above.

\textit{Cohort Level:} The character can attract a cohort of up to this level. Regardless of a character's Leadership score, he can only recruit a cohort who is two or more levels lower than himself. The cohort should be equipped with gear appropriate for its level. A character can try to attract a cohort of a particular race, class, and alignment. The cohort's alignment may not be opposed to the leader's alignment on either the law-vs-chaos or good-vs-evil axis, and the leader takes a Leadership penalty if he recruits a cohort of an alignment different from his own.

Cohorts earn XP as follows:

\begin{enumerate*}
\item The cohort does not count as a party member when determining the party's XP.
\item Divide the cohort's level by the level of the PC with whom he or she is associated (the character with the Leadership feat who attracted the cohort).
\item Multiply this result by the total XP awarded to the PC and add that number of experience points to the cohort's total.
\end{enumerate*}

If a cohort gains enough XP to bring it to a level one lower than the associated PC's character level, the cohort does not gain the new level---its new XP total is 1 less than the amount needed attain the next level.

\textit{Number of Followers by Level:} The character can lead up to the indicated number of characters of each level. Followers are similar to cohorts, except they're generally low-level NPCs. Because they're generally five or more levels behind the character they follow, they're rarely effective in combat.

Followers don't earn experience and thus don't gain levels. However, when a character with Leadership attains a new level, the player consults the table above to determine if she has acquired more followers, some of which may be higher level than the existing followers. (You don't consult the table to see if your cohort gains levels, however, because cohorts earn experience on their own.)

\textit{Leadership Modifiers:} Several factors can affect a character's Leadership score, causing it to vary from the base score (character level + Cha modifier). A character's reputation (from the point of view of the cohort or follower he is trying to attract) raises or lowers his Leadership score, see Table: Reputation.

\Table{Reputation}{X c}{
\tableheader Leader's Reputation & \tableheader Modifier\\
Great renown & +2\\
Fairness and generosity & +1\\
Special power & +1\\
Failure & $-1$\\
Aloofness & $-1$\\
Cruelty & $-2$
}

Other modifiers may apply when the character tries to attract a cohort, see \tabref{Attracting Cohorts}.

\Table{Attracting Cohorts}{X c}{
\tableheader The Leader… & \tableheader Modifier\\
Has a familiar, special mount, or animal companion & $-2$\\
Recruits a cohort of a different alignment & $-1$\\
Caused the death of a cohort & $-2$ per cohort killed
}

Followers have different priorities from cohorts. When the character tries to attract a new follower, use any of the modifiers that apply on \tabref{Attracting Followers}.

\Table{Attracting Followers}{X c}{
\tableheader The Leader… & \tableheader Modifier\\
Has a stronghold, base of operations, guildhouse, or the like & +2\\
Moves around a lot & $-1$\\
Caused the death of other followers & $-1$
}}

\GFeat{Lightning Reflexes}{}
{You get a +2 bonus on all Reflex saving throws.}

\Feat{Martial Weapon Proficiency}
{Choose a type of martial weapon. You understand how to use that type of martial weapon in combat.}
{}
{You make attack rolls with the selected weapon normally.}
{When using a weapon with which you are not proficient, you take a $-4$ penalty on attack rolls.}
{Barbarians, fighters, gladiators, psychic warriors, and rangers are proficient with all martial weapons. They need not select this feat.

You can gain Martial Weapon Proficiency multiple times. Each time you take the feat, it applies to a new type of weapon.

A Hamanu's templar, because of the the War domain, automatically gains the Martial Weapon Proficiency feat related to his sorcerer-monarchs's favored weapon as a bonus feat, the longsword. He need not select it.}

\GFeat{Natural Spell}
{Wis 13, wild shape ability.}
{You can complete the verbal and somatic components of spells while in a wild shape. You substitute various noises and gestures for the normal verbal and somatic components of a spell.

You can also use any material components or focuses you possess, even if such items are melded within your current form. This feat does not permit the use of magic items while you are in a form that could not ordinarily use them, and you do not gain the ability to speak while in a wild shape.}

\Feat{Run}{}{}
{When running, you move five times your normal speed (if wearing medium, light, or no armor and carrying no more than a medium load) or four times your speed (if wearing heavy armor or carrying a heavy load). If you make a jump after a running start (see the Jump skill description), you gain a +4 bonus on your Jump check. While running, you retain your Dexterity bonus to AC.}
{You move four times your speed while running (if wearing medium, light, or no armor and carrying no more than a medium load) or three times your speed (if wearing heavy armor or carrying a heavy load), and you lose your Dexterity bonus to AC.}{}

\Feat{Shield Proficiency}{}{}
{You can use a shield and take only the standard penalties.}
{When you are using a shield with which you are not proficient, you take the shield's armor check penalty on attack rolls and on all skill checks that involve moving, including Ride checks.}
{Barbarians, clerics, druids, fighters, gladiators, psychic warriors, rangers, and templars automatically have Shield Proficiency as a bonus feat. They need not select it.}

\Feat{Simple Weapon Proficiency}{}{}
{You make attack rolls with simple weapons normally.}
{When using a weapon with which you are not proficient, you take a $-4$ penalty on attack rolls.}
{All characters except for druids, psions, and wizards are automatically proficient with all simple weapons. They need not select this feat.}

\Feat{Skill Focus}
{Choose a skill.}{}
{You get a +3 bonus on all checks involving that skill.}{}
{You can gain this feat multiple times. Its effects do not stack. Each time you take the feat, it applies to a new skill.}

\Feat{Spell Focus}
{Choose a school of magic.}{}
{Add +1 to the Difficulty Class for all saving throws against spells from the school of magic you select.}{}
{You can gain this feat multiple times. Its effects do not stack. Each time you take the feat, it applies to a new school of magic.}

\Feat[Special]{Spell Mastery}{}
{Wizard level 1st.}
{Each time you take this feat, choose a number of spells equal to your Intelligence modifier that you already know. From that point on, you can prepare these spells without referring to a spellbook.}
{Without this feat, you must use a spellbook to prepare all your spells, except read magic.}{}

\GFeat{Spell Penetration}{}
{You get a +2 bonus on caster level checks (1d20 + caster level) made to overcome a creature's spell resistance.}

\Feat{Toughness}{}{}
{You gain +3 hit points.}{}
{A character may gain this feat multiple times. Its effects stack.}

\Feat{Tower Shield Proficiency}{}
{\feat{Shield Proficiency}.}
{You can use a tower shield and suffer only the standard penalties.}
{A character who is using a shield with which he or she is not proficient takes the shield's armor check penalty on attack rolls and on all skill checks that involve moving, including Ride.}
{Fighters automatically have Tower Shield Proficiency as a bonus feat. They need not select it.}

\Feat{Track}{}{}
{To find tracks or to follow them for 1 mile requires a successful Survival check. You must make another Survival check every time the tracks become difficult to follow.

You move at half your normal speed (or at your normal speed with a $-5$ penalty on the check, or at up to twice your normal speed with a $-20$ penalty on the check). The DC depends on the surface and the prevailing conditions, as given on \tabref{Track DC}.


\Table{Track DC}{X Z{1.4cm} X Z{1.4cm}}{
\tableheader Surface & \tableheader Survival DC & \tableheader Surface & \tableheader Survival DC\\
Very soft ground & 5 & Firm ground & 15\\
Soft ground & 10 & Hard ground & 20
}

\textit{Very Soft Ground:} Any surface (fresh snow, thick dust, wet mud) that holds deep, clear impressions of footprints.

\textit{Soft Ground:} Any surface soft enough to yield to pressure, but firmer than wet mud or fresh snow, in which a creature leaves frequent but shallow footprints.

\textit{Firm Ground:} Most normal outdoor surfaces (such as lawns, fields, woods, and the like) or exceptionally soft or dirty indoor surfaces (thick rugs and very dirty or dusty floors). The creature might leave some traces (broken branches or tufts of hair), but it leaves only occasional or partial footprints.

\textit{Hard Ground:} Any surface that doesn't hold footprints at all, such as bare rock or an indoor floor. Most streambeds fall into this category, since any footprints left behind are obscured or washed away. The creature leaves only traces (scuff marks or displaced pebbles).

Several modifiers may apply to the \skill{Survival} check, as given on \tabref{Track DC Modifiers}.

\Table{Track DC Modifiers}{X Z{1.4cm}}{
\tableheader Condition & \tableheader Survival DC Modifier\\
Every three creatures in the group being tracked & $-1$\\
Size of creature or creatures being tracked & \\
~ Fine & +8\\
~ Diminutive & +4\\
~ Tiny & +2\\
~ Small & +1\\
~ Medium & 0\\
~ Large & $-1$\\
~ Huge & $-2$\\
~ Gargantuan & $-4$\\
~ Colossal & $-8$\\
Every 24 hours since the trail was made & +1\\
Every hour of rain since the trail was made & +1\\
Fresh snow cover since the trail was made & +10\\
Poor visibility (Apply only the largest modifier from this category.) & \\
~ Overcast or moonless night & +6\\
~ Moonlight & +3\\
~ Fog or precipitation & +3\\
Tracked party hides trail (and moves at half speed) & +5
}

For a group of mixed sizes, apply only the modifier for the largest size category.

If you fail a \skill{Survival} check, you can retry after 1 hour (outdoors) or 10 minutes (indoors) of searching.}
{Without this feat, you can use the \skill{Survival} skill to find tracks, but you can follow them only if the DC for the task is 10 or lower. Alternatively, you can use the \skill{Search} skill to find a footprint or similar sign of a creature's passage using the DCs given above, but you can't use Search to follow tracks, even if someone else has already found them.}
{A ranger automatically has Track as a bonus feat. He need not select it.

This feat does not allow you to find or follow the tracks made by a subject of a pass without trace spell.}

\section{Skill Feats}

\GFeat[Skill]{Acrobatic}{}
{You get a +2 bonus on all Jump checks and Tumble checks.}

\GFeat[Skill]{Agile}{}
{You get a +2 bonus on all Balance checks and Escape Artist checks.}

\Feat[Skill]{Alertness}{}{}
{You get a +2 bonus on all Listen checks and Spot checks.}
{}
{The master of a familiar gains the benefit of the Alertness feat whenever the familiar is within arm’s reach.}

\GFeat[Skill]{Animal Affinity}{}
{You get a +2 bonus on all Handle Animal checks and Ride checks.}

\GFeat[Skill]{Athletic}{}
{You get a +2 bonus on all Climb checks and Swim checks.}

\GFeat[Skill]{Deceitful}{}
{You get a +2 bonus on all Disguise checks and Forgery checks.}

\GFeat[Skill]{Deft Hands}{}
{You get a +2 bonus on all Sleight of Hand checks and Use Rope checks.}

\GFeat[Skill]{Diligent}{}
{You get a +2 bonus on all Appraise checks and Decipher Script checks.}

\GFeat[Skill]{Investigator}{}
{You get a +2 bonus on all Gather Information checks and Search checks.}

\GFeat[Skill]{Magical Aptitude}{}
{You get a +2 bonus on all Spellcraft checks and Use Magic Device checks.}

\GFeat[Skill]{Negotiator}{}
{You get a +2 bonus on all Diplomacy checks and Sense Motive checks.}

\GFeat[Skill]{Nimble Fingers}{}
{You get a +2 bonus on all Disable Device checks and Open Lock checks.}

\GFeat[Skill]{Persuasive}{}
{You get a +2 bonus on all Bluff checks and Intimidate checks.}

\GFeat[Skill]{Self-Sufficient}{}
{You get a +2 bonus on all Heal checks and Survival checks.}

\GFeat[Skill]{Stealthy}{}
{You get a +2 bonus on all Hide checks and Move Silently checks.}
\section{Fighter Feats}

\Feat[Fighter]{Blind-Fight}{}
{In melee, every time you miss because of concealment, you can reroll your miss chance percentile roll one time to see if you actually hit.

An invisible attacker gets no advantages related to hitting you in melee. That is, you don't lose your Dexterity bonus to Armor Class, and the attacker doesn't get the usual +2 bonus for being invisible. The invisible attacker's bonuses do still apply for ranged attacks, however.

You take only half the usual penalty to speed for being unable to see. Darkness and poor visibility in general reduces your speed to three-quarters normal, instead of one-half.}
{Regular attack roll modifiers for invisible attackers trying to hit you apply, and you lose your Dexterity bonus to AC. The speed reduction for darkness and poor visibility also applies.}
{The Blind-Fight feat is of no use against a character who is the subject of a blink spell.}{}

\GFeat[Fighter]{Cleave}
{Str 13, \feat{Power Attack}.}
{If you deal a creature enough damage to make it drop (typically by dropping it to below 0 hit points or killing it), you get an immediate, extra melee attack against another creature within reach. You cannot take a 5-foot step before making this extra attack. The extra attack is with the same weapon and at the same bonus as the attack that dropped the previous creature. You can use this ability once per round.}

\Feat[Fighter]{Combat Expertise}{}
{Int 13.}
{When you use the attack action or the full attack action in melee, you can take a penalty of as much as $-5$ on your attack roll and add the same number (+5 or less) as a dodge bonus to your Armor Class. This number may not exceed your base attack bonus. The changes to attack rolls and Armor Class last until your next action.}
{A character without the Combat Expertise feat can fight defensively while using the attack or full attack action to take a $-4$ penalty on attack rolls and gain a +2 dodge bonus to Armor Class.}{}

\Feat[Fighter]{Combat Reflexes}{}
{You may make a number of additional attacks of opportunity equal to your Dexterity bonus.

With this feat, you may also make attacks of opportunity while flat-footed.}
{A character without this feat can make only one attack of opportunity per round and can't make attacks of opportunity while flat-footed.}
{The Combat Reflexes feat does not allow a rogue to use her opportunist ability more than once per round.}{}

\Feat[Fighter]{Commanding Presence}
{Your mere presence can enable your allies.}
{\skill{Diplomacy} 7 ranks, \skill{Knowledge} (warcraft) 5 ranks.}
{This feat grants a new use for the \skill{Diplomacy} skill.

\textit{Enabling an Ally:} You can remove harmful conditions from an ally as a move action by making a DC 20 \skill{Diplomacy} check. If the check succeeds, you can negate any one of the following conditions: cowering, dazed, fatigued, nauseated, panicked, shaken, or stunned.

You cannot use this ability on yourself.}{}{}

\Feat[Fighter]{Concentrated Fire}
{You are trained in formation archery and taking out specific targets through joint efforts.}
{Base attack bonus +1.}
{When readying and firing projectile weapons at a single target, you add a +1 bonus to your attack roll for every other participant with this feat who readies and fires at the same target on your initiative count. The total bonus cannot exceed +4.}{}{}

\GFeat[Fighter]{Deflect Arrows}
{Dex 13, \feat{Improved Unarmed Strike}.}
{You must have at least one hand free (holding nothing) to use this feat. Once per round when you would normally be hit with a ranged weapon, you may deflect it so that you take no damage from it. You must be aware of the attack and not flat-footed.

Attempting to deflect a ranged weapon doesn't count as an action. Unusually massive ranged weapons and ranged attacks generated by spell effects can't be deflected.}

\GFeat[Fighter]{Dodge}
{Dex 13.}
{During your action, you designate an opponent and receive a +1 dodge bonus to Armor Class against attacks from that opponent. You can select a new opponent on any action.

A condition that makes you lose your Dexterity bonus to Armor Class (if any) also makes you lose dodge bonuses. Also, dodge bonuses stack with each other, unlike most other types of bonuses.}

\Feat[Fighter]{Exotic Weapon Proficiency}
{Choose a type of exotic weapon. You understand how to use that type of exotic weapon in combat.}
{Base attack bonus +1 (plus Str 13 for bastard sword or dwarven waraxe).}
{You make attack rolls with the weapon normally.}
{A character who uses a weapon with which he or she is not proficient takes a $-4$ penalty on attack rolls.}
{You can gain Exotic Weapon Proficiency multiple times. Each time you take the feat, it applies to a new type of exotic weapon. Proficiency with the bastard sword or the dwarven waraxe has an additional prerequisite of Str 13.}

\GFeat[Fighter]{Far Shot}
{\feat{Point Blank Shot}.}
{When you use a projectile weapon, such as a bow, its range increment increases by one-half (multiply by 1\onehalf). When you use a thrown weapon, its range increment is doubled.}

\GFeat[Fighter]{Great Cleave}
{Str 13, \feat{Cleave}, \feat{Power Attack}, base attack bonus +4.}
{This feat works like \feat{Cleave}, except that there is no limit to the number of times you can use it per round.}

\Feat[Fighter]{Greater Two-Weapon Fighting}{}
{Dex 19, \feat{Improved Two-Weapon Fighting}, \feat{Two-Weapon Fighting}, base attack bonus +11.}
{You get a third attack with your off-hand weapon, albeit at a $-10$ penalty. See the Two-Weapon Fighting special attack.}{}
{An 11th-level ranger who has chosen the two-weapon combat style is treated as having Greater Two-Weapon Fighting, even if he does not have the prerequisites for it, but only when he is wearing light or no armor.}

\Feat[Fighter]{Greater Weapon Focus}
{Choose one type of weapon for which you have already selected Weapon Focus. You can also choose unarmed strike or grapple as your weapon for purposes of this feat.}
{Proficiency with selected weapon, \feat{Weapon Focus} with selected weapon, fighter level 8th.}
{You gain a +1 bonus on all attack rolls you make using the selected weapon. This bonus stacks with other bonuses on attack rolls, including the one from Weapon Focus (see below).}{}
{You can gain Greater Weapon Focus multiple times. Its effects do not stack. Each time you take the feat, it applies to a new type of weapon.

A fighter must have Greater Weapon Focus with a given weapon to gain the \feat{Greater Weapon Specialization} feat for that weapon.}

\Feat[Fighter]{Greater Weapon Specialization}
{Choose one type of weapon for which you have already selected Weapon Specialization. You can also choose unarmed strike or grapple as your weapon for purposes of this feat.}
{Proficiency with selected weapon, \feat{Greater Weapon Focus} with selected weapon, \feat{Weapon Focus} with selected weapon, \feat{Weapon Specialization} with selected weapon, fighter level 12th.}
{You gain a +2 bonus on all damage rolls you make using the selected weapon. This bonus stacks with other bonuses on damage rolls, including the one from Weapon Specialization (see below).}{}
{You can gain Greater Weapon Specialization multiple times. Its effects do not stack. Each time you take the feat, it applies to a new type of weapon.}

\Feat[Fighter]{Improved Bull Rush}{}
{Str 13, \feat{Power Attack}.}
{When you perform a bull rush you do not provoke an attack of opportunity from the defender. You also gain a +4 bonus on the opposed Strength check you make to push back the defender.}{}{}

\Feat[Fighter]{Improved Critical}
{Choose one type of weapon.}
{Proficient with weapon, base attack bonus +8.}
{When using the weapon you selected, your threat range is doubled.}{}
{You can gain Improved Critical multiple times. The effects do not stack. Each time you take the feat, it applies to a new type of weapon.

This effect doesn't stack with any other effect that expands the threat range of a weapon.}{}

\Feat[Fighter]{Improved Disarm}{}
{Int 13, \feat{Combat Expertise}.}
{You do not provoke an attack of opportunity when you attempt to disarm an opponent, nor does the opponent have a chance to disarm you. You also gain a +4 bonus on the opposed attack roll you make to disarm your opponent.}
{See the normal disarm rules.}{}

\Feat[Fighter]{Improved Feint}{}
{Int 13, \feat{Combat Expertise}.}
{You can make a Bluff check to feint in combat as a move action.}
{Feinting in combat is a standard action.}{}

\Feat[Fighter]{Improved Grapple}{}
{Dex 13, \feat{Improved Unarmed Strike}.}
{You do not provoke an attack of opportunity when you make a touch attack to start a grapple. You also gain a +4 bonus on all grapple checks, regardless of whether you started the grapple.}
{Without this feat, you provoke an attack of opportunity when you make a touch attack to start a grapple.}{}

\GFeat[Fighter]{Improved Initiative}{}
{You get a +4 bonus on initiative checks.}

\Feat[Fighter]{Improved Overrun}{}
{Str 13, \feat{Power Attack}.}
{When you attempt to overrun an opponent, the target may not choose to avoid you. You also gain a +4 bonus on your Strength check to knock down your opponent.}
{Without this feat, the target of an overrun can choose to avoid you or to block you.}{}

\Feat[Fighter]{Improved Precise Shot}{}
{Dex 19, \feat{Point Blank Shot}, \feat{Precise Shot}, base attack bonus +11.}
{Your ranged attacks ignore the AC bonus granted to targets by anything less than total cover, and the miss chance granted to targets by anything less than total concealment. Total cover and total concealment provide their normal benefits against your ranged attacks:. }
In addition, when you shoot or throw ranged weapons at a grappling opponent, you automatically strike at the opponent you have chosen.
{See the normal rules on the effects of cover and concealment. Without this feat, a character who shoots or throws a ranged weapon at a target involved in a grapple must roll randomly to see which grappling combatant the attack strikes.}
{An 11th-level ranger who has chosen the archery combat style is treated as having Improved Precise Shot, even if he does not have the prerequisites for it, but only when he is wearing light or no armor.}

\Feat[Fighter]{Improved Shield Bash}{}
{\feat{Shield Proficiency}.}
{When you perform a shield bash, you may still apply the shield's shield bonus to your AC.}
{Without this feat, a character who performs a shield bash loses the shield's shield bonus to AC until his or her next turn.}{}

\Feat[Fighter]{Improved Sunder}{}
{Str 13, \feat{Power Attack}.}
{When you strike at an object held or carried by an opponent (such as a weapon or shield), you do not provoke an attack of opportunity.

You also gain a +4 bonus on any attack roll made to attack an object held or carried by another character.}
{Without this feat, you provoke an attack of opportunity when you strike at an object held or carried by another character.}{}

\Feat[Fighter]{Improved Trip}{}
{Int 13, \feat{Combat Expertise}.}
{You do not provoke an attack of opportunity when you attempt to trip an opponent while you are unarmed. You also gain a +4 bonus on your Strength check to trip your opponent.

If you trip an opponent in melee combat, you immediately get a melee attack against that opponent as if you hadn't used your attack for the trip attempt.}
{Without this feat, you provoke an attack of opportunity when you attempt to trip an opponent while you are unarmed.}{}

\Feat[Fighter]{Improved Two-Weapon Fighting}{}
{Dex 17, \feat{Two-Weapon Fighting}, base attack bonus +6.}
{In addition to the standard single extra attack you get with an off-hand weapon, you get a second attack with it, albeit at a $-5$ penalty. See the Two-Weapon Fighting special attack.}
{Without this feat, you can only get a single extra attack with an off-hand weapon.}
{A 6th-level ranger who has chosen the two-weapon combat style is treated as having Improved Two-Weapon Fighting, even if he does not have the prerequisites for it, but only when he is wearing light or no armor.}

\Feat[Fighter]{Improved Unarmed Strike}{}{}
{You are considered to be armed even when unarmed---that is, you do not provoke attacks or opportunity from armed opponents when you attack them while unarmed. However, you still get an attack of opportunity against any opponent who makes an unarmed attack on you.

In addition, your unarmed strikes can deal lethal or nonlethal damage, at your option.}
{Without this feat, you are considered unarmed when attacking with an unarmed strike, and you can deal only nonlethal damage with such an attack.}{}

\Feat[Fighter]{Manyshot}{}
{Dex 17, \feat{Point Blank Shot}, \feat{Rapid Shot}, base attack bonus +6}
{As a standard action, you may fire two arrows at a single opponent within 30 feet. Both arrows use the same attack roll (with a $-4$ penalty) to determine success and deal damage normally (but see Special).

For every five points of base attack bonus you have above +6, you may add one additional arrow to this attack, to a maximum of four arrows at a base attack bonus of +16. However, each arrow after the second adds a cumulative $-2$ penalty on the attack roll (for a total penalty of $-6$ for three arrows and $-8$ for four).

Damage reduction and other resistances apply separately against each arrow fired.}{}
{Regardless of the number of arrows you fire, you apply precision-based damage only once. If you score a critical hit, only the first arrow fired deals critical damage; all others deal regular damage.

A 6th-level ranger who has chosen the archery combat style is treated as having Manyshot even if he does not have the prerequisites for it, but only when he is wearing light or no armor.}

\GFeat[Fighter]{Mobility}
{Dex 13, \feat{Dodge}.}
{You get a +4 dodge bonus to Armor Class against attacks of opportunity caused when you move out of or within a threatened area. A condition that makes you lose your Dexterity bonus to Armor Class (if any) also makes you lose dodge bonuses.

Dodge bonuses stack with each other, unlike most types of bonuses.}

\GFeat[Fighter]{Mounted Archery}
{Ride 1 rank, \feat{Mounted Combat}.}
{The penalty you take when using a ranged weapon while mounted is halved: $-2$ instead of $-4$ if your mount is taking a double move, and $-4$ instead of $-8$ if your mount is running.}

\GFeat[Fighter]{Mounted Combat}
{Ride 1 rank.}
{Once per round when your mount is hit in combat, you may attempt a Ride check (as a reaction) to negate the hit. The hit is negated if your Ride check result is greater than the opponent's attack roll. (Essentially, the Ride check result becomes the mount's Armor Class if it's higher than the mount's regular AC.)}

\GFeat[Fighter]{Point Blank Shot}{}
{You get a +1 bonus on attack and damage rolls with ranged weapons at ranges of up to 30 feet.}

\Feat[Fighter]{Power Attack}{}
{Str 13.}
{On your action, before making attack rolls for a round, you may choose to subtract a number from all melee attack rolls and add the same number to all melee damage rolls. This number may not exceed your base attack bonus. The penalty on attacks and bonus on damage apply until your next turn.}
{If you attack with a two-handed weapon, or with a one-handed weapon wielded in two hands, instead add twice the number subtracted from your attack rolls. You can't add the bonus from Power Attack to the damage dealt with a light weapon (except with unarmed strikes or natural weapon attacks), even though the penalty on attack rolls still applies. (Normally, you treat a double weapon as a one-handed weapon and a light weapon. If you choose to use a double weapon like a two-handed weapon, attacking with only one end of it in a round, you treat it as a two-handed weapon.)}{}

\GFeat[Fighter]{Precise Shot}
{\feat{Point Blank Shot}.}
{You can shoot or throw ranged weapons at an opponent engaged in melee without taking the standard $-4$ penalty on your attack roll.}

\Feat[Fighter]{Quick Draw}{}
{Base attack bonus +1.}
{You can draw a weapon as a free action instead of as a move action. You can draw a hidden weapon (see the Sleight of Hand skill) as a move action.

A character who has selected this feat may throw weapons at his full normal rate of attacks (much like a character with a bow).}
{Without this feat, you may draw a weapon as a move action, or (if your base attack bonus is +1 or higher) as a free action as part of movement. Without this feat, you can draw a hidden weapon as a standard action.}{}

\Feat[Fighter]{Rapid Reload}
{Choose a type of crossbow (hand, light, or heavy).}
{\feat{Weapon Proficiency} (crossbow type chosen).}
{The time required for you to reload your chosen type of crossbow is reduced to a free action (for a hand or light crossbow) or a move action (for a heavy crossbow). Reloading a crossbow still provokes an attack of opportunity.

If you have selected this feat for hand crossbow or light crossbow, you may fire that weapon as many times in a full attack action as you could attack if you were using a bow.}
{A character without this feat needs a move action to reload a hand or light crossbow, or a full-round action to reload a heavy crossbow.}
{You can gain Rapid Reload multiple times. Each time you take the feat, it applies to a new type of crossbow.}

\Feat[Fighter]{Rapid Shot}{}
{Dex 13, \feat{Point Blank Shot}.}
{You can get one extra attack per round with a ranged weapon. The attack is at your highest base attack bonus, but each attack you make in that round (the extra one and the normal ones) takes a $-2$ penalty. You must use the full attack action to use this feat.}{}
{A 2nd-level ranger who has chosen the archery combat style is treated as having Rapid Shot, even if he does not have the prerequisites for it, but only when he is wearing light or no armor.}

\GFeat[Fighter]{Ride-By Attack}
{Ride 1 rank, \feat{Mounted Combat}.}
{When you are mounted and use the charge action, you may move and attack as if with a standard charge and then move again (continuing the straight line of the charge). Your total movement for the round can't exceed double your mounted speed. You and your mount do not provoke an attack of opportunity from the opponent that you attack.}

\GFeat[Fighter]{Shot On The Run}
{Dex 13, \feat{Dodge}, \feat{Mobility}, \feat{Point Blank Shot}, base attack bonus +4.}
{When using the attack action with a ranged weapon, you can move both before and after the attack, provided that your total distance moved is not greater than your speed.}

\GFeat[Fighter]{Snatch Arrows}
{Dex 15, \feat{Deflect Arrows}, \feat{Improved Unarmed Strike}.}
{When using the \feat{Deflect Arrows} feat you may catch the weapon instead of just deflecting it. Thrown weapons can immediately be thrown back at the original attacker (even though it isn't your turn) or kept for later use.

You must have at least one hand free (holding nothing) to use this feat.}

\GFeat[Fighter]{Spirited Charge}
{Ride 1 rank, \feat{Mounted Combat}, \feat{Ride-By Attack}.}
{When mounted and using the charge action, you deal double damage with a melee weapon (or triple damage with a lance).}

\GFeat[Fighter]{Spring Attack}
{Dex 13, \feat{Dodge}, \feat{Mobility}, base attack bonus +4.}
{When using the attack action with a melee weapon, you can move both before and after the attack, provided that your total distance moved is not greater than your speed. Moving in this way does not provoke an attack of opportunity from the defender you attack, though it might provoke attacks of opportunity from other creatures, if appropriate. You can't use this feat if you are wearing heavy armor.

You must move at least 5 feet both before and after you make your attack in order to utilize the benefits of Spring Attack.}

\GFeat[Fighter]{Stunning Fist}
{Dex 13, Wis 13, \feat{Improved Unarmed Strike}, base attack bonus +8.}
{You must declare that you are using this feat before you make your attack roll (thus, a failed attack roll ruins the attempt). Stunning Fist forces a foe damaged by your unarmed attack to make a Fortitude saving throw (DC 10 + \onehalf your character level + your Wis modifier), in addition to dealing damage normally. A defender who fails this saving throw is stunned for 1 round (until just before your next action). A stunned creature drops everything held, can't take actions, takes a $-2$ penalty to AC, and loses his Dexterity bonus to AC. You may attempt a stunning attack once per day for every four levels you have attained (but see Special), and no more than once per round. Constructs, oozes, plants, undead, incorporeal creatures, and creatures immune to critical hits cannot be stunned.}

\GFeat[Fighter]{Trample}
{Ride 1 rank, \feat{Mounted Combat}.}
{When you attempt to overrun an opponent while mounted, your target may not choose to avoid you. Your mount may make one hoof attack against any target you knock down, gaining the standard +4 bonus on attack rolls against prone targets.}

\GFeat[Fighter]{Two-Weapon Defense}
{Dex 15, \feat{Two-Weapon Fighting}.}
{When wielding a double weapon or two weapons (not including natural weapons or unarmed strikes), you gain a +1 shield bonus to your AC. See the Two-Weapon Fighting special attack.

When you are fighting defensively or using the total defense action, this shield bonus increases to +2.}

\Feat[Fighter]{Two-Weapon Fighting}
{You can fight with a weapon in each hand. You can make one extra attack each round with the second weapon.}
{Dex 15.}
{Your penalties on attack rolls for fighting with two weapons are reduced. The penalty for your primary hand lessens by 2 and the one for your off hand lessens by 6. See the Two-Weapon Fighting special attack.}
{If you wield a second weapon in your off hand, you can get one extra attack per round with that weapon. When fighting in this way you suffer a $-6$ penalty with your regular attack or attacks with your primary hand and a $-10$ penalty to the attack with your off hand. If your off-hand weapon is light the penalties are reduced by 2 each. (An unarmed strike is always considered light.)}
{A 2nd-level ranger who has chosen the two-weapon combat style is treated as having Two-Weapon Fighting, even if he does not have the prerequisite for it, but only when he is wearing light or no armor.}

\GFeat[Fighter]{Weapon Finesse}
{Base attack bonus +1.}
{With a light weapon, rapier, whip, or spiked chain made for a creature of your size category, you may use your Dexterity modifier instead of your Strength modifier on attack rolls. If you carry a shield, its armor check penalty applies to your attack rolls.

Natural weapons are always considered light weapons.}

\Feat[Fighter]{Weapon Focus}
{Choose one type of weapon. You can also choose unarmed strike or grapple (or ray, if you are a spellcaster) as your weapon for purposes of this feat.}
{Proficiency with selected weapon, base attack bonus +1.}
{You gain a +1 bonus on all attack rolls you make using the selected weapon.}{}
{You can gain this feat multiple times. Its effects do not stack. Each time you take the feat, it applies to a new type of weapon.}

\Feat[Fighter]{Weapon Specialization}
{Choose one type of weapon for which you have already selected the Weapon Focus feat. You can also choose unarmed strike or grapple as your weapon for purposes of this feat. You deal extra damage when using this weapon.}
{Proficiency with selected weapon, Weapon Focus with selected weapon, fighter level 4th.}
{You gain a +2 bonus on all damage rolls you make using the selected weapon.}{}
{You can gain this feat multiple times. Its effects do not stack. Each time you take the feat, it applies to a new type of weapon.}

\GFeat[Fighter]{Whirlwind Attack}
{Dex 13, Int 13, \feat{Combat Expertise}, \feat{Dodge}, \feat{Mobility}, \feat{Spring Attack}, base attack bonus +4.}
{When you use the full attack action, you can give up your regular attacks and instead make one melee attack at your full base attack bonus against each opponent within reach.

When you use the Whirlwind Attack feat, you also forfeit any bonus or extra attacks granted by other feats, spells, or abilities.}

\section{Metamagic Feats}

\GFeat[Metamagic]{Empower Spell}{}
{All variable, numeric effects of an empowered spell are increased by one-half.

Saving throws and opposed rolls are not affected, nor are spells without random variables. An empowered spell uses up a spell slot two levels higher than the spell’s actual level.}

\GFeat[Metamagic]{Enlarge Spell}{}
{You can alter a spell with a range of close, medium, or long to increase its range by 100\%. An enlarged spell with a range of close now has a range of 50 ft. + 5 ft./level, while medium-range spells have a range of 200 ft. + 20 ft./level and long-range spells have a range of 800 ft. + 80 ft./level. An enlarged spell uses up a spell slot one level higher than the spell’s actual level.

Spells whose ranges are not defined by distance, as well as spells whose ranges are not close, medium, or long, do not have increased ranges.}

\GFeat[Metamagic]{Extend Spell}{}
{An extended spell lasts twice as long as normal. A spell with a duration of concentration, instantaneous, or permanent is not affected by this feat. An extended spell uses up a spell slot one level higher than the spell’s actual level.}

\GFeat[Metamagic]{Heighten Spell}{}
{A heightened spell has a higher spell level than normal (up to a maximum of 9th level). Unlike other metamagic feats, Heighten Spell actually increases the effective level of the spell that it modifies. All effects dependent on spell level (such as saving throw DCs and ability to penetrate a lesser globe of invulnerability) are calculated according to the heightened level. The heightened spell is as difficult to prepare and cast as a spell of its effective level.}

\GFeat[Metamagic]{Maximize Spell}{}
{All variable, numeric effects of a spell modified by this feat are maximized. Saving throws and opposed rolls are not affected, nor are spells without random variables. A maximized spell uses up a spell slot three levels higher than the spell’s actual level.

An empowered, maximized spell gains the separate benefits of each feat: the maximum result plus:  ne-half the normally rolled result.}

\Feat[Metamagic]{Quicken Spell}{}{}
{Casting a quickened spell is an swift action. You can perform another action, even casting another spell, in the same round as you cast a quickened spell. You may cast only one quickened spell per round. A spell whose casting time is more than 1 full round action cannot be quickened. A quickened spell uses up a spell slot four levels higher than the spell’s actual level. Casting a quickened spell doesn’t provoke an attack of opportunity.}
{}
{This feat can’t be applied to any spell cast spontaneously (including templar spells, and cleric or druid spells cast spontaneously), since applying a metamagic feat to a spontaneously cast spell automatically increases the casting time to a full-round action.}

\GFeat[Metamagic]{Silent Spell}{}
{A silent spell can be cast with no verbal components. Spells without verbal components are not affected. A silent spell uses up a spell slot one level higher than the spell’s actual level.}

\GFeat[Metamagic]{Still Spell}{}
{A stilled spell can be cast with no somatic components.

Spells without somatic components are not affected. A stilled spell uses up a spell slot one level higher than the spell’s actual level.}

\GFeat[Metamagic]{Widen Spell}{}
{You can alter a burst, emanation, line, or spread shaped spell to increase its area. Any numeric measurements of the spell’s area increase by 100\%. A widened spell uses up a spell slot three levels higher than the spell’s actual level.

Spells that do not have an area of one of these four sorts are not affected by this feat.}

\section{Item Creation Feats}

\Feat[Item Creation]{Brew Potion}
{}
{Caster level 3rd.}
{You can create a potion of any 3rd-level or lower spell that you know and that targets one or more creatures. Brewing a potion takes one day. When you create a potion, you set the caster level, which must be sufficient to cast the spell in question and no higher than your own level. The base price of a potion is its spell level $\times$ its caster level $\times$ 50 gp. To brew a potion, you must spend 1/25 of this base price in XP and use up raw materials costing one half this base price.

When you create a potion, you make any choices that you would normally make when casting the spell. Whoever drinks the potion is the target of the spell.

Any potion that stores a spell with a costly material component or an XP cost also carries a commensurate cost. In addition to the costs derived from the base price, you must expend the material component or pay the XP when creating the potion.}
{}
{On Athas, potions take many different forms. The most common form is an enchanted fruit, often called a \emph{potionfruit}. Other common items for enchantment include obsidian orbs, packs of herbs, and bone fetishes. The potion, regardless of material used to make it, is consumed or destroyed when used.

Due to the nature of their magic, defilers cannot enchant organic materials, such as fruits, as a potion. As a consequence, most non-defilers use those receptacles almost exclusively, as a way of assuring the potion is not a product of defiler magic.}

\Feat[Item Creation]{Craft Cognizance Crystal}
{You can create psionic cognizance crystals that store power points.}
{Manifester level 3rd.}
{You can create a cognizance crystal. Doing so takes one day for each 1,000 gp in its base price. The base price of a cognizance crystal is equal to the highest-level power it could manifest using all its stored power points, squared, multiplied by 1,000 gp. To create a cognizance crystal, you must spend 1/25 of its base price in XP and use up raw materials costing one-half its base price.}{}{}

\GFeat[Item Creation]{Craft Construct}
{\feat{Craft Magic Arms and Armor}, \feat{Craft Wondrous Item}.}
{A creature with this feat can create any construct whose prerequisites it meets. Enchanting a construct takes one day for each 1,000 gp in its market price. To enchant a construct, a spellcaster must spend 1/25 the item's price in XP and use up raw materials costing half of this price (see individual construct monster entries for details).

A creature with this feat can repair constructs that have taken damage. In one day of work, the creature can repair up to 20 points of damage by expending 50 gp per point of damage repaired.

A newly created construct has average hit points for its Hit Dice.}

\Feat[Item Creation]{Craft Dorje}
{You can create slender crystal wands called dorjes than manifest powers when charges are expended.}
{Manifester level 5th.}
{You can create a dorje of any psionic power you know (barring exceptions, such as bestow power, as noted in a power's description). Crafting a dorje takes one day for each 1,000 gp in its base price. The base price of a dorje is its manifester level $\times$ the power level $\times$ 750 gp. To craft a dorje, you must spend 1/25 of this base price in XP and use up raw materials costing one-half of this base price.

A newly created dorje has 50 charges.

Any dorje that stores a power with an XP cost also carries a commensurate cost. In addition to the XP cost derived from the base price, you must pay fifty times the XP cost.}{}{}

\GFeat[Item Creation]{Craft Magic Arms and Armor}
{Caster level 5th.}
{You can create any magic weapon, armor, or shield whose prerequisites you meet. Enhancing a weapon, suit of armor, or shield takes one day for each 1,000 gp in the price of its magical features. To enhance a weapon, suit of armor, or shield, you must spend 1/25 of its features' total price in XP and use up raw materials costing one-half of this total price.

The weapon, armor, or shield to be enhanced must be a masterwork item that you provide. Its cost is not included in the above cost.

You can also mend a broken magic weapon, suit of armor, or shield if it is one that you could make. Doing so costs half the XP, half the raw materials, and half the time it would take to craft that item in the first place.}

\Feat[Item Creation]{Craft Psicrown}
{You can create psicrowns, which have multiple psionic effects.}
{Manifester level 12th.}
{You can create any psicrown whose prerequisites you meet. Crafting a psicrown takes one day for each 1,000 gp in its base price. To craft a psicrown, you must spend 1/25 of its base price in XP and use up raw materials costing one-half of its base price. Some psicrowns incur extra costs in XP as noted in their descriptions. These costs are in addition to those derived from the psicrown's base price.}{}{}

\Feat[Item Creation]{Craft Psionic Arms and Armor}
{You can create psionic weapons, armor, and shields.}
{Manifester level 5th.}
{You can create any psionic weapon, armor, or shield whose prerequisites you meet. Enhancing a weapon, suit of armor, or shield takes one day for each 1,000 gp in the price of its psionic features. To enhance a weapon, you must spend 1/25 of its features' total price in XP and use up raw materials costing one-half of this total price.

The weapon, armor, or shield to be enhanced must be a masterwork item that you provide. Its cost is not included in the above cost.

You can also mend a broken psionic weapon, suit of armor, or shield if it is one that you could make. Doing so costs half the XP, half the raw materials, and half the time it would take to enhance that item in the first place.}{}{}

\Feat[Item Creation]{Craft Psionic Construct}
{You can create golems and other psionic automatons that obey your orders.}
{\feat{Craft Psionic Arms and Armor}, \feat{Craft Universal Item}.}
{You can create any psionic construct whose prerequisites you meet. Creating a construct takes one day for each 1,000 gp in its base price. To create a construct, you must spend 1/25 of the construct's base price in XP and use up raw materials costing one-half of this price. A newly created construct has average hit points for its Hit Dice.}{}{}

\GFeat[Item Creation]{Craft Rod}
{Caster level 9th.}
{You can create any rod whose prerequisites you meet. Crafting a rod takes one day for each 1,000 gp in its base price. To craft a rod, you must spend 1/25 of its base price in XP and use up raw materials costing one-half of its base price.

Some rods incur extra costs in material components or XP, as noted in their descriptions. These costs are in addition to those derived from the rod's base price.}

\GFeat[Item Creation]{Craft Staff}
{Caster level 12th.}
{You can create any staff whose prerequisites you meet.

Crafting a staff takes one day for each 1,000 gp in its base price. To craft a staff, you must spend 1/25 of its base price in XP and use up raw materials costing one-half of its base price. A newly created staff has 50 charges.

Some staffs incur extra costs in material components or XP, as noted in their descriptions. These costs are in addition to those derived from the staff's base price.}

\Feat[Item Creation]{Craft Universal Item}
{You can create universal psionic items.}
{Manifester level 3rd.}
{You can create any universal psionic item whose prerequisites you meet. Crafting a universal psionic item takes one day for each 1,000 gp in its base price. To craft a universal psionic item, you must spend 1/25 of the item's base price in XP and use up raw materials costing one-half of this price.

You can also mend a broken universal item if it is one that you could make. Doing so costs half the XP, half the raw materials, and half the time it would take to craft that item in the first place.

Some universal items incur extra costs in XP, as noted in their descriptions. These costs are in addition to those derived from the item's base price. You must pay such a cost to create an item or to mend a broken one.}{}{}

\GFeat[Item Creation]{Craft Wand}
{Caster level 5th.}
{You can create a wand of any 4th-level or lower spell that you know. Crafting a wand takes one day for each 1,000 gp in its base price. The base price of a wand is its caster level $\times$ the spell level $\times$ 750 gp. To craft a wand, you must spend 1/25 of this base price in XP and use up raw materials costing one-half of this base price. A newly created wand has 50 charges.

Any wand that stores a spell with a costly material component or an XP cost also carries a commensurate cost. In addition to the cost derived from the base price, you must expend fifty copies of the material component or pay fifty times the XP cost.}

\GFeat[Item Creation]{Craft Wondrous Item}
{Caster level 3rd.}
{You can create any wondrous item whose prerequisites you meet. Enchanting a wondrous item takes one day for each 1,000 gp in its price. To enchant a wondrous item, you must spend 1/25 of the item's price in XP and use up raw materials costing half of this price.

You can also mend a broken wondrous item if it is one that you could make. Doing so costs half the XP, half the raw materials, and half the time it would take to craft that item in the first place.

Some wondrous items incur extra costs in material components or XP, as noted in their descriptions. These costs are in addition to those derived from the item's base price. You must pay such a cost to create an item or to mend a broken one.}

\GFeat[Item Creation]{Forge Ring}
{Caster level 12th.}
{You can create any ring whose prerequisites you meet. Crafting a ring takes one day for each 1,000 gp in its base price. To craft a ring, you must spend 1/25 of its base price in XP and use up raw materials costing one-half of its base price.

You can also mend a broken ring if it is one that you could make. Doing so costs half the XP, half the raw materials, and half the time it would take to forge that ring in the first place.

Some magic rings incur extra costs in material components or XP, as noted in their descriptions. You must pay such a cost to forge such a ring or to mend a broken one.}

\Feat[Item Creation]{Imprint Stone}
{You can create power stones to store psionic powers.}
{Manifester level 1st.}
{You can create a power stone of any power that you know. Encoding a power stone takes one day for each 1,000 gp in its base price. The base price of a power stone is the level of the stored power $\times$ its manifester level $\times$ 25 gp. To imprint a power stone, you must spend 1/25 of this base price in XP and use up raw materials costing one-half of this base price.

Any power stone that stores a power with an XP cost also carries a commensurate cost. In addition to the costs derived from the base price, you must pay the XP when encoding the stone.}{}{}

\Feat[Item Creation]{Scribe Scroll}
{}
{Caster level 1st.}
{You can create a scroll of any spell that you know. Scribing a scroll takes one day for each 1,000 gp in its base price. The base price of a scroll is its spell level $\times$ its caster level $\times$ 25 gp. To scribe a scroll, you must spend 1/25 of this base price in XP and use up raw materials costing one-half of this base price.

Any scroll that stores a spell with a costly material component or an XP cost also carries a commensurate cost. In addition to the costs derived from the base price, you must expend the material component or pay the XP when scribing the scroll.}
{}
{On Athas, scrolls take many different forms. Common forms include paper or papyrus sheets, clay tablets, and woven cloth.}

\Feat[Item Creation]{Scribe Tattoo}
{You can create psionic tattoos, which store powers within their designs.}
{Manifester level 3rd.}
{You can create a psionic tattoo of any power of 3rd level or lower that you know and that targets one or more creatures. Scribing a psionic tattoo takes one day. When you create a psionic tattoo, you set the manifester level. The manifester level must be sufficient to manifest the power in question and no higher than your own level. The base price of a psionic tattoo is its power level $\times$ its manifester level $\times$ 50 gp. To scribe a tattoo, you must spend 1/25 of this base price in XP and use up raw materials (special inks, masterwork needles, and so on) costing one-half of this base price.

When you create a psionic tattoo, you make any choices that you would normally make when manifesting the power.

When its wearer physically activates the tattoo, the wearer is the target of the power.

Any psionic tattoo that stores a power with an XP cost also carries a commensurate cost. In addition to the costs derived from the base price, you must pay the XP when creating the tattoo.}{}{}
