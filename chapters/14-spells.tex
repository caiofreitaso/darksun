\Chapter{Spells}
{Magic is arguably the mightiest force in Athas. Those wielding it call fire storms out of a calm sky, change one object into another, or kill enemies with a mere gesture. They dictate the wills of entire mobs, make the dead walk, and have even been known to stop time. Magic can expose traitors, destroy rivals, and exact unquestioning obedience from subjects. It can also conceal secret activities, uncover the king's spies, be used to assassinate royal officers, and foster general rebellion.}{The Wanderer's Journal}

An M or F appearing at the end of a spell's name in the spell lists denotes a spell with a material or focus component, respectively, that is not normally included in a spell component pouch. An X denotes a spell with an XP component paid by the caster.

\textbf{Order of Presentation:} In the spell lists and the spell descriptions that follow them, the spells are presented in alphabetical order by name except for those belonging to certain spell chains.

When a spell's name begins with ``lesser,'' ``greater,'' or ``mass,'' the spell description is alphabetized under the second word of the spell name instead.

\textbf{Hit Dice:} The term ``Hit Dice'' is used synonymously with ``character levels'' for effects that affect a number of Hit Dice of creatures. Creatures with Hit Dice only from their race, not from classes, have character levels equal to their Hit Dice.

\textbf{Caster Level:} A spell's power often depends on caster level, which is defined as the caster's class level for the purpose of casting a particular spell. A creature with no classes has a caster level equal to its Hit Dice unless otherwise specified. The word ``level'' in the spell lists that follow always refers to caster level.

\textbf{Creatures and Characters:} The words ``creature'' and ``character'' are used synonymously in the spell descriptions.

\section{Cleric Spells}



\subsection{0-Level Cleric Spells (Orisons)}

\textit{Create Element}: Create a small amount of patron element.

\textit{Cure Minor Wounds}: Cures 1 point of damage.

\textit{Detect Magic}: Detects spells and magic items within 18 m.

\textit{Detect Poison}: Detects poison in one creature or object.

\textit{Guidance}: +1 on one attack roll, saving throw, or skill check.

\textit{Inflict Minor Wounds}: Touch attack, 1 point of damage.

\textit{Light}: Object shines like a torch.

\textit{Mending}: Makes minor repairs on an object.

\textit{Purify Food and Drink}: Purifies 30 cm$^3$/level of food or water.

\textit{Read Magic}: Read scrolls and spellbooks.

\textit{Resistance}: Subject gains +1 on saving throws.

\textit{Virtue}: Subject gains 1 temporary hp.



\subsection{1st-Level Cleric Spells}

\textit{Bane}: Enemies take $-1$ on attack rolls and saves against fear.

\textit{Bless}: Allies gain +1 on attack rolls and saves against fear.

% \textit{Bless Water}\textsuperscript{M}: Makes holy water.
\textit{Bless Element}\textsuperscript{M}: Makes holy element.

\textit{Cause Fear}: One creature of 5 HD or less flees for 1d4 rounds.

\textit{Command}: One subject obeys selected command for 1 round.

\textit{Comprehend Languages}: You understand all spoken and written languages.

\textit{Cure Light Wounds}: Cures 1d8 damage +1/level (max +5).

% \textit{Curse Water}\textsuperscript{M}: Makes unholy water.
\textit{Curse Element}\textsuperscript{M}: Makes unholy element.

\textit{Deathwatch}: Reveals how near death subjects within 9 m are.

\textit{Detect Chaos/Evil/Good/Law}: Reveals creatures, spells, or objects of selected alignment.

\textit{Detect Undead}: Reveals undead within 18 m.

\textit{Divine Favor}: You gain +1 per three levels on attack and damage rolls.

\textit{Doom}: One subject takes $-2$ on attack rolls, saves, and checks.

\textit{Endure Elements}: Exist comfortably in hot or cold environments.

\textit{Entropic Shield}: Ranged attacks against you have 20\% miss chance.

%%
\textit{Heat Lash}: Creature suffers 1d4+1 damage and is knocked back 1.5 m.

\textit{Hide from Undead}: Undead can't perceive one subject/level.

\textit{Inflict Light Wounds}: Touch deals 1d8 damage +1/level (max +5).

% \textit{Magic Stone}: Three stones gain +1 on attack, deal 1d6 +1 damage.

\textit{Magic Weapon}: Weapon gains +1 bonus.

\textit{Obscuring Mist}: Fog surrounds you.

\textit{Protection from Chaos/Evil/Good/Law}: +2 to AC and saves, counter mind control, hedge out elementals and outsiders.

\textit{Remove Fear}: Suppresses fear or gives +4 on saves against fear for one subject + one per four levels.

\textit{Sanctuary}: Opponents can't attack you, and you can't attack.

\textit{Shield of Faith}: Aura grants +2 or higher deflection bonus.

\textit{Summon Monster I}: Calls extraplanar creature to fight for you.



\subsection{2nd-Level Cleric Spells}

\textit{Aid}: +1 on attack rolls and saves against fear, 1d8 temporary hp +1/level (max +10).

\textit{Align Weapon}: Weapon becomes good, evil, lawful, or chaotic.

\textit{Augury}\textsuperscript{MF}: Learns whether an action will be good or bad.

\textit{Bear's Endurance}: Subject gains +4 to Con for 1 min./level.

\textit{Bull's Strength}: Subject gains +4 to Str for 1 min./level.

\textit{Calm Emotions}: Calms creatures, negating emotion effects.

\textit{Consecrate}\textsuperscript{M}: Fills area with positive energy, making undead weaker.

\textit{Cure Moderate Wounds}: Cures 2d8 damage +1/level (max +10).

\textit{Darkness}: 6-m radius of supernatural shadow.

\textit{Death Knell}: Kills dying creature; you gain 1d8 temporary hp, +2 to Str, and +1 level.

\textit{Delay Poison}: Stops poison from harming subject for 1 hour/level.

\textit{Desecrate}\textsuperscript{M}: Fills area with negative energy, making undead stronger.

\textit{Eagle's Splendor}: Subject gains +4 to Cha for 1 min./level.

\textit{Enthrall}: Captivates all within 30 m + 3 m/level.

\textit{Find Traps}: Notice traps as a rogue does.

\textit{Gentle Repose}: Preserves one corpse.

\textit{Hold Person}: Paralyzes one humanoid for 1 round/level.

\textit{Inflict Moderate Wounds}: Touch attack, 2d8 damage +1/level (max +10).

\textit{Make Whole}: Repairs an object.

\textit{Owl's Wisdom}: Subject gains +4 to Wis for 1 min./level.

\textit{Remove Paralysis}: Frees one or more creatures from paralysis or slow effect.

\textit{Resist Energy}: Ignores 10 (or more) points of damage/attack from specified energy type.

\textit{Restoration, Lesser}: Dispels magical ability penalty or repairs 1d4 ability damage.

%%
\textit{Return to the Earth}: Turns dead and undead bodies into dust.

\textit{Shatter}: Sonic vibration damages objects or crystalline creatures.

\textit{Shield Other}\textsuperscript{F}: You take half of subject's damage.

\textit{Silence}: Negates sound in 6-m radius.

\textit{Sound Burst}: Deals 1d8 sonic damage to subjects; may stun them.

\textit{Spiritual Weapon}: Magic weapon attacks on its own.

\textit{Status}: Monitors condition, position of allies.

\textit{Summon Monster II}: Calls extraplanar creature to fight for you.

\textit{Undetectable Alignment}: Conceals alignment for 24 hours.

\textit{Zone of Truth}: Subjects within range cannot lie.



\subsection{3rd-Level Cleric Spells}

\textit{Animate Dead}\textsuperscript{M}: Creates undead skeletons and zombies.

\textit{Bestow Curse}: $-6$ to an ability score; $-4$ on attack rolls, saves, and checks; or 50\% chance of losing each action.

\textit{Blindness/Deafness}: Makes subject blinded or deafened.

\textit{Contagion}: Infects subject with chosen disease.

% \textit{Continual Flame}\textsuperscript{M}: Makes a permanent, heatless torch.

% \textit{Create Food and Water}: Feeds three humans (or one horse)/level.

\textit{Cure Serious Wounds}: Cures 3d8 damage +1/level (max +15).

% \textit{Daylight}: 18-m radius of bright light.

\textit{Deeper Darkness}: Object sheds supernatural shadow in 18-m radius.

\textit{Dispel Magic}: Cancels spells and magical effects.

%%
\textit{Eye of the Storm}: Protects 9-m radius from effects of storm for 1 hour/level.

\textit{Glyph of Warding}\textsuperscript{M}: Inscription harms those who pass it.

\textit{Helping Hand}: Ghostly hand leads subject to you.

\textit{Inflict Serious Wounds}: Touch attack, 3d8 damage +1/level (max +15).

\textit{Invisibility Purge}: Dispels invisibility within 1.5 m/level.

%%
\textit{Lighten Load}: Increases Strength for carrying capacity only.

\textit{Locate Object}: Senses direction toward object (specific or type).

\textit{Magic Circle against Chaos/Evil/Good/Law}: As protection spells, but 3-m radius and 10 min./level.

\textit{Magic Vestment}: Armor or shield gains +1 enhancement per four levels.

% \textit{Meld into Stone}: You and your gear merge with stone.

\textit{Obscure Object}: Masks object against scrying.

\textit{Prayer}: Allies +1 bonus on most rolls, enemies $-1$ penalty.

\textit{Protection from Energy}: Absorb 12 points/level of damage from one kind of energy.

\textit{Remove Blindness/Deafness}: Cures normal or magical conditions.

\textit{Remove Curse}: Frees object or person from curse.

\textit{Remove Disease}: Cures all diseases affecting subject.

%%
\textit{Sand Pit}: Excavates sand in a 9 m wide and 15 m deep cone.

\textit{Searing Light}: Ray deals 1d8/two levels damage, more against undead.

\textit{Speak with Dead}: Corpse answers one question/two levels.

% \textit{Stone Shape}: Sculpts stone into any shape.

\textit{Summon Monster III}: Calls extraplanar creature to fight for you.

%%
\textit{Surface Walk}: Subject treads on unstable surfaces as if solid.

% \textit{Water Breathing}: Subjects can breathe underwater.

% \textit{Water Walk}: Subject treads on water as if solid.

% \textit{Wind Wall}: Deflects arrows, smaller creatures, and gases.



\subsection{4th-Level Cleric Spells}

% \textit{Air Walk}: Subject treads on air as if solid (climb at 45-degree angle).

% \textit{Control Water}: Raises or lowers bodies of water.

\textit{Cure Critical Wounds}: Cures 4d8 damage +1/level (max +20).

\textit{Death Ward}: Grants immunity to death spells and negative energy effects.

\textit{Dimensional Anchor}: Bars extradimensional movement.

\textit{Discern Lies}: Reveals deliberate falsehoods.

\textit{Dismissal}: Forces a creature to return to native plane.

\textit{Divination}\textsuperscript{M}: Provides useful advice for specific proposed actions.

\textit{Divine Power}: You gain attack bonus, +6 to Str, and 1 hp/level.

%%
\textit{Elemental Armor}: Armor or shield gains enhancement bonus and special quality.

%%
\textit{Elemental Weapon}: Weapon gains enhancement bonus and special quality.

\textit{Freedom of Movement}: Subject moves normally despite impediments.

\textit{Giant Vermin}: Turns centipedes, scorpions, or spiders into giant vermin.

\textit{Imbue with Spell Ability}: Transfer spells to subject.

\textit{Inflict Critical Wounds}: Touch attack, 4d8 damage +1/level (max +20).

\textit{Magic Weapon, Greater}: +1 bonus/four levels (max +5).

\textit{Neutralize Poison}: Immunizes subject against poison, detoxifies venom in or on subject.

\textit{Planar Ally, Lesser}\textsuperscript{X}: Exchange services with a 6 HD extraplanar creature.

\textit{Poison}: Touch deals 1d10 Con damage, repeats in 1 min.

\textit{Repel Vermin}: Insects, spiders, and other vermin stay 3 m away.

\textit{Restoration}\textsuperscript{M}: Restores level and ability score drains.

\textit{Sending}: Delivers short message anywhere, instantly.

\textit{Spell Immunity}: Subject is immune to one spell per four levels.

\textit{Summon Monster IV}: Calls extraplanar creature to fight for you.

\textit{Tongues}: Speak any language.



\subsection{5th-Level Cleric Spells}

\textit{Atonement}\textsuperscript{FX}: Removes burden of misdeeds from subject.

\textit{Break Enchantment}: Frees subjects from enchantments, alterations, curses, and petrification.

\textit{Command, Greater}: As command, but affects one subject/level.

\textit{Commune}\textsuperscript{X}: Deity answers one yes-or-no question/level.

\textit{Cure Light Wounds, Mass}: Cures 1d8 damage +1/level for many creatures.

\textit{Dispel Chaos/Evil/Good/Law}: +4 bonus against attacks.

\textit{Disrupting Weapon}: Melee weapon destroys undead.

% \textit{Flame Strike}: Smite foes with divine fire (1d6/level damage).

\textit{Hallow}\textsuperscript{M}: Designates location as holy.

\textit{Inflict Light Wounds, Mass}: Deals 1d8 damage +1/level to many creatures.

\textit{Insect Plague}: Locust swarms attack creatures.

\textit{Mark of Justice}: Designates action that will trigger curse on subject.

\textit{Plane Shift}\textsuperscript{F}: As many as eight subjects travel to another plane.

\textit{Raise Dead}\textsuperscript{M}: Restores life to subject who died as long as one day/level ago.

\textit{Rangeblade}: Can strike with melee weapons at a distance.

\textit{Righteous Might}: Your size increases, and you gain combat bonuses.

\textit{Scrying}\textsuperscript{F}: Spies on subject from a distance.

\textit{Slay Living}: Touch attack kills subject.

\textit{Spell Resistance}: Subject gains SR 12 + level.

\textit{Summon Monster V}: Calls extraplanar creature to fight for you.

\textit{Symbol of Pain}\textsuperscript{M}: Triggered rune wracks nearby creatures with pain.

\textit{Symbol of Sleep}\textsuperscript{M}: Triggered rune puts nearby creatures into catatonic slumber.

\textit{True Seeing}\textsuperscript{M}: Lets you see all things as they really are.

\textit{Unhallow}\textsuperscript{M}: Designates location as unholy.

% \textit{Wall of Stone}: Creates a stone wall that can be shaped.



\subsection{6th-Level Cleric Spells}

\textit{Animate Objects}: Objects attack your foes.

\textit{Antilife Shell}: 3-m field hedges out living creatures.

\textit{Banishment}: Banishes 2 HD/level of extraplanar creatures.

\textit{Bear's Endurance, Mass}: As bear's endurance, affects one subject/ level.

\textit{Blade Barrier}: Wall of blades deals 1d6/level damage.

\textit{Braxatskin}: Your skin hardens, granting armor bonus and damage reduction.

\textit{Bull's Strength, Mass}: As bull's strength, affects one subject/level.

\textit{Create Undead}: Create ghouls, ghasts, mummies, or mohrgs.

\textit{Cure Moderate Wounds, Mass}: Cures 2d8 damage +1/level for many creatures.

\textit{Dispel Magic, Greater}: As dispel magic, but up to +20 on check.

\textit{Eagle's Splendor, Mass}: As eagle's splendor, affects one subject/level.

\textit{Find the Path}: Shows most direct way to a location.

\textit{Forbiddance}\textsuperscript{M}: Blocks planar travel, damages creatures of different alignment.

\textit{Geas/Quest}: As lesser geas, plus it affects any creature.

\textit{Glyph of Warding, Greater}: As glyph of warding, but up to 10d8 damage or 6th-level spell.

\textit{Harm}: Deals 10 points/level damage to target.

\textit{Heal}: Cures 10 points/level of damage, all diseases and mental conditions.

\textit{Heroes' Feast}: Food for one creature/level cures and grants combat bonuses.

\textit{Inflict Moderate Wounds, Mass}: Deals 2d8 damage +1/level to many creatures.

\textit{Owl's Wisdom, Mass}: As owl's wisdom, affects one subject/level.

\textit{Planar Ally}\textsuperscript{X}: As lesser planar ally, but up to 12 HD.

\textit{Summon Monster VI}: Calls extraplanar creature to fight for you.

\textit{Symbol of Fear}\textsuperscript{M}: Triggered rune panics nearby creatures.

\textit{Symbol of Persuasion}\textsuperscript{M}: Triggered rune charms nearby creatures.

\textit{Undeath to Death}\textsuperscript{M}: Destroys 1d4 HD/level undead (max 20d4).

% \textit{Wind Walk}: You and your allies turn vaporous and travel fast.

\textit{Word of Recall}: Teleports you back to designated place.



\subsection{7th-Level Cleric Spells}

\textit{Blasphemy}: Kills, paralyzes, weakens, or dazes nonevil subjects.

\textit{Control Weather}: Changes weather in local area.

\textit{Cure Serious Wounds, Mass}: Cures 3d8 damage +1/level for many creatures.

\textit{Destruction}\textsuperscript{F}: Kills subject and destroys remains.

\textit{Dictum}: Kills, paralyzes, slows, or deafens nonlawful subjects.

\textit{Elemental Chariot}: Enhances chariot with elemental effects.

\textit{Ethereal Jaunt}: You become ethereal for 1 round/level.

\textit{Holy Word}: Kills, paralyzes, blinds, or deafens nongood subjects.

\textit{Inflict Serious Wounds, Mass}: Deals 3d8 damage +1/level to many creatures.

\textit{Refuge}\textsuperscript{M}: Alters item to transport its possessor to you.

\textit{Regenerate}: Subject's severed limbs grow back, cures 4d8 damage +1/level (max +35).

\textit{Repulsion}: Creatures can't approach you.

\textit{Restoration, Greater}\textsuperscript{X}: As restoration, plus restores all levels and ability scores.

\textit{Resurrection}\textsuperscript{M}: Fully restore dead subject.

\textit{Sands of Time}\textsuperscript{F}: Reverses or accelerates aging of a non-living object.

\textit{Scrying, Greater}: As scrying, but faster and longer.

\textit{Summon Monster VII}: Calls extraplanar creature to fight for you.

\textit{Symbol of Stunning}\textsuperscript{M}: Triggered rune stuns nearby creatures.

\textit{Symbol of Weakness}\textsuperscript{M}: Triggered rune weakens nearby creatures.

\textit{Unliving Identity}\textsuperscript{MX}: Transforms a zombie into a thinking zombie.

\textit{Word of Chaos}: Kills, confuses, stuns, or deafens nonchaotic subjects.



\subsection{8th-Level Cleric Spells}

\textit{Antimagic Field}: Negates magic within 3 m.

\textit{Cloak of Chaos}\textsuperscript{F}: +4 to AC, +4 resistance, and SR 25 against lawful spells.

\textit{Create Greater Undead}\textsuperscript{M}: Create shadows, wraiths, spectres, or devourers.

\textit{Cure Critical Wounds, Mass}: Cures 4d8 damage +1/level for many creatures.

\textit{Dimensional Lock}: Teleportation and interplanar travel blocked for one day/level.

\textit{Discern Location}: Reveals exact location of creature or object.

% \textit{Earthquake}: Intense tremor shakes 24-m radius.

% \textit{Fire Storm}: Deals 1d6/level fire damage.

\textit{Holy Aura}\textsuperscript{F}: +4 to AC, +4 resistance, and SR 25 against evil spells.

\textit{Inflict Critical Wounds, Mass}: Deals 4d8 damage +1/level to many creatures.

\textit{Planar Ally, Greater}\textsuperscript{X}: As lesser planar ally, but up to 18 HD.

\textit{Shield of Law}\textsuperscript{F}: +4 to AC, +4 resistance, and SR 25 against chaotic spells.

\textit{Spell Immunity, Greater}: As spell immunity, but up to 8th-level spells.

\textit{Summon Monster VIII}: Calls extraplanar creature to fight for you.

\textit{Symbol of Death}\textsuperscript{M}: Triggered rune slays nearby creatures.

\textit{Symbol of Insanity}\textsuperscript{M}: Triggered rune renders nearby creatures insane.

\textit{Unholy Aura}\textsuperscript{F}: +4 to AC, +4 resistance, and SR 25 against good spells.



\subsection{9th-Level Cleric Spells}

\textit{Astral Projection}\textsuperscript{M}: Projects you and companions onto Astral Plane.

\textit{Elemental Chariot, Greater}: As \spell{elemental chariot}, but with greater effects.

\textit{Energy Drain}: Subject gains 2d4 negative levels.

\textit{Etherealness}: Travel to Ethereal Plane with companions.

\textit{Gate}\textsuperscript{X}: Connects two planes for travel or summoning.

\textit{Heal, Mass}: As heal, but with several subjects.

\textit{Implosion}: Kills one creature/round.

\textit{Miracle}\textsuperscript{X}: Requests a deity's intercession.

\textit{Soul Bind}\textsuperscript{F}: Traps newly dead soul to prevent resurrection.

% \textit{Storm of Vengeance}: Storm rains acid, lightning, and hail.

\textit{Summon Monster IX}: Calls extraplanar creature to fight for you.

\textit{True Resurrection}\textsuperscript{M}: As resurrection, plus remains aren't needed.
\section{Cleric Domains}

\Domain{Air}
{Air}
{???}
{
	\item \textit{Feather Fall}: Objects or creatures fall slowly.
	\item \textit{Wind Wall}: Deflects arrows, smaller creatures, and gases.
	\item \textit{Gaseous Form}: Subject becomes insubstantial and can fly slowly.
	\item 
	\item \textit{Control Winds}: Change wind direction and speed.
	\item \textit{Wind Walk}: You and your allies turn vaporous and travel fast.
	\item \textit{Control Weather}: Changes weather in local area.
	\item 
	\item \textit{Whirlwind}: Cyclone deals damage and can pick up creatures.
}

\Domain{Freedom}
{Air}
{???}
{
	\item 
	\item 
	\item 
	\item \textit{Fly}: Subject flies at speed of 60 ft.
	\item 
	\item 
	\item 
	\item 
	\item 
}

\Domain{Rain}
{Rain}
{???}
{
	\item 
	\item 
	\item 
	\item 
	\item 
	\item 
	\item \textit{Control Weather}: Changes weather in local area.
	\item 
	\item 
}

\Domain{Travel}
{}
{???}
{
	\item \textit{Longstrider}: Increases your speed.
	\item 
	\item 
	\item \textit{Haste}: One creature/level moves faster, +1 on attack rolls, AC, and Reflex saves.
	\item 
	\item \textit{Find the Path}: Shows most direct way to a location.
	\item 
	\item 
	\item 
}
\section{Druid Spells}



\subsection{0-Level Druid Spells (Orisons)}

\spellList{Create Water}: Creates 2 liters/level of pure water.

\spellList{Cure Minor Wounds}: Cures 1 point of damage.

\spellList{Defiler Scent}: Smells the presence or absence of defilers.

\spellList{Detect Magic}: Detects spells and magic items within 18 m.

\spellList{Detect Poison}: Detects poison in one creature or object.

\spellList{Flare}: Dazzles one creature ($-1$ penalty on attack rolls).

\spellList{Guidance}: +1 on one attack roll, saving throw, or skill check.

\spellList{Know Direction}: You discern north.

\spellList{Light}: Object shines like a torch.

\spellList{Mending}: Makes minor repairs on an object.

\spellList{Nurturing Seeds}: Makes 10 seeds or cuttings hardy and easy to transplant.

\spellList{Purify Food and Drink}: Purifies 0.3 m$^3$/level of food or water.

\spellList{Read Magic}: Read scrolls and spellbooks.

\spellList{Resistance}: Subject gains +1 bonus on saving throws.

\spellList{Virtue}: Subject gains 1 temporary hp.



\subsection{1st-Level Druid Spells}

\spellList{Backlash}\textsuperscript{M}: 1d6+1 damage/2 levels to defilers.

\spellList{Black Cairn}\textsuperscript{F}: Locates a corpse nearby.

\spellList{Calm Animals}: Calms (2d4 + level) HD of animals.

\spellList{Charm Animal}: Makes one animal your friend.

\spellList{Cooling Canopy}: Summons a cloud to provide shade and prevent dehydration.

\spellList{Cure Light Wounds}: Cures 1d8 damage +1/level (max +5).

\spellList{Detect Animals or Plants}: Detects kinds of animals or plants.

\spellList{Detect Snares and Pits}: Reveals natural or primitive traps.

\spellList{Detect Undead}: Reveals undead within 18 meters.

\spellList{Endure Elements}: Exist comfortably in hot or cold environments.

\spellList{Entangle}: Plants entangle everyone in 12-m-radius.

\spellList{Faerie Fire}: Outlines subjects with light, canceling blur, concealment, and the like.

\spellList{Goodberry}: 2d4 berries each cure 1 hp (max 8 hp/24 hours).

\spellList{Hide from Animals}: Animals can't perceive one subject/level.

\spellList{Jump}: Subject gets bonus on \skill{Jump} checks.

\spellList{Longstrider}: Your speed increases by 3 m.

\spellList{Magic Fang}: One natural weapon of subject creature gets +1 on attack and damage rolls.

\spellList{Magic Stone}: Three stones gain +1 on attack rolls, deal 1d6+1 damage.

\spellList{Obscuring Mist}: Fog surrounds you.

\spellList{Pass without Trace}: One subject/level leaves no tracks.

\spellList{Plant Renewal}: Brings one plant back from near destruction.

\spellList{Produce Flame}: 1d6 damage +1/level, touch or thrown.

\spellList{Proof Against Undeath}: Prevents dead subject from being raised as undead.

\spellList{Shillelagh}: Cudgel or quarterstaff becomes +1 weapon and deals damage as if two sizes larger.

\spellList{Speak with Animals}: You can communicate with animals.

\spellList{Summon Nature's Ally I}: Calls creature to fight.



\subsection{2nd-Level Druid Spells}

\spellList{Animal Messenger}: Sends a Tiny animal to a specific place.

\spellList{Animal Trance}: Fascinates 2d6 HD of animals.

\spellList{Barkskin}: Grants +2 (or higher) enhancement to natural armor.

\spellList{Bear's Endurance}: Subject gains +4 to Con for 1 min./level.

\spellList{Bull's Strength}: Subject gains +4 to Str for 1 min./level.

\spellList{Cat's Grace}: Subject gains +4 to Dex for 1 min./level.

\spellList{Chill Metal}: Cold metal damages those who touch it.

\spellList{Delay Poison}: Stops poison from harming subject for 1 hour/level.

\spellList{Fire Trap}\textsuperscript{M}: Opened object deals 1d4 +1/level damage.

\spellList{Flame Blade}: Touch attack deals 1d8 +1/two levels damage.

\spellList{Flaming Sphere}: Creates rolling ball of fire, 2d6 damage, lasts 1 round/level.

\spellList{Fog Cloud}: Fog obscures vision.

\spellList{Gust of Wind}: Blows away or knocks down smaller creatures.

\spellList{Heat Metal}: Make metal so hot it damages those who touch it.

\spellList{Hold Animal}: Paralyzes one animal for 1 round/level.

\spellList{Owl's Wisdom}: Subject gains +4 to Wis for 1 min./level.

\spellList{Reduce Animal}: Shrinks one willing animal.

\spellList{Resist Energy}: Ignores 10 (or more) points of damage/attack from specified energy type.

\spellList{Restoration, Lesser}: Dispels magical ability penalty or repairs 1d4 ability damage.

\spellList{Soften Earth and Stone}: Turns stone to clay or dirt to sand or mud.

\spellList{Spider Climb}: Grants ability to walk on walls and ceilings.

\spellList{Summon Nature's Ally II}: Calls creature to fight.

\spellList{Summon Swarm}: Summons swarm of bats, rats, or spiders.

\spellList{Tree Shape}: You look exactly like a tree for 1 hour/level.

\spellList{Warp Wood}: Bends wood (shaft, handle, door, plank).

\spellList{Wood Shape}: Rearranges wooden objects to suit you.



\subsection{3rd-Level Druid Spells}

\spellList{Boneclaw's Cut}\textsuperscript{F}: Deals damage that continues to cause bleeding damage.%%

\spellList{Call Lightning}: Calls down lightning bolts (3d6 per bolt) from sky.

\spellList{Claws of the Tembo}: Deals 1d6 Str damage and transfers hp.

\spellList{Clear-river}: Blows aways or knocks creatures.

\spellList{Contagion}: Infects subject with chosen disease.

\spellList{Cure Moderate Wounds}: Cures 2d8 damage +1/level (max +10).

\spellList{Curse of the Black Sands}: Target leaves black oily footprints.%%

\spellList{Daylight}: 18-m radius of bright light.

\spellList{Deeper Darkness}: Object sheds absolute darkness in 18-m radius.

\spellList{Diminish Plants}: Reduces size or blights growth of normal plants.

\spellList{Dominate Animal}: Subject animal obeys silent mental commands.

\spellList{Eye of the Storm}: Protects 9-m radius from effects of storm for 1 hour/level.%%

\spellList{Magic Fang, Greater}: One natural weapon of subject creature gets +1/four levels on attack and damage rolls (max +5).

\spellList{Meld into Stone}: You and your gear merge with stone.

\spellList{Neutralize Poison}: Immunizes subject against poison, detoxifies venom in or on subject.

\spellList{Plant Growth}: Grows vegetation, improves crops.

\spellList{Poison}: Touch deals 1d10 Con damage, repeats in 1 min.

\spellList{Protection from Energy}: Absorb 12 points/level of damage from one kind of energy.

\spellList{Quench}: Extinguishes nonmagical fires or one magic item.

\spellList{Remove Curse}: Frees object or person from curse.

\spellList{Remove Disease}: Cures all diseases affecting subject.

\spellList{Return to the Earth}: Turns dead and undead bodies into dust.

\spellList{Searing Light}: Ray deals 1d8/two levels against undead.

\spellList{Sleet Storm}: Hampers vision and movement.

\spellList{Snare}: Creates a magic booby trap.

\spellList{Speak with Plants}: You can talk to normal plants and plant creatures.

\spellList{Spike Growth}: Creatures in area take 1d4 damage, may be slowed.

\spellList{Stone Shape}: Sculpts stone into any shape.

\spellList{Summon Nature's Ally III}: Calls creature to fight.

\spellList{Surface Walk}: Subject treads on unstable surfaces as if solid.

% \spellList{Water Breathing}: Subjects can breathe underwater.

\spellList{Wind Wall}: Deflects arrows, smaller creatures, and gases.

\spellList{Worm's Breath}: Subjects can breathe underwater, in silt, or earth.

\spellList{Zombie Berry}: 1d4 berries from the zombie plant become attuned to you.



\subsection{4th-Level Druid Spells}

\spellList{Air Walk}: Subject treads on air as if solid (climb at 45-degree angle).

\spellList{Antiplant Shell}: Keeps animated plants at bay.

\spellList{Blight}: Withers one plant or deals 1d6/level damage to plant creature.

\spellList{Command Plants}: Sway the actions of one or more plant creatures.

\spellList{Control Tides}: Raises, lowers, or parts bodies of water or silt.

\spellList{Cure Serious Wounds}: Cures 3d8 damage +1/level (max +15).

\spellList{Dispel Magic}: Cancels spells and magical effects.

\spellList{Flame Strike}: Smite foes with divine fire (1d6/level damage).

\spellList{Freedom of Movement}: Subject moves normally despite impediments.

\spellList{Giant Vermin}: Turns centipedes, scorpions, or spiders into giant vermin.

\spellList{Ice Storm}: Hail deals 5d6 damage in cylinder 12 m across.

\spellList{Klar's Heart}: Enhances combat abilities of all creatures within range.

\spellList{Nondetection}\textsuperscript{M}: Hides subject from divination, scrying.

\spellList{Reincarnate}: Brings dead subject back in a random body.

\spellList{Repel Vermin}: Insects, spiders, and other vermin stay 3 m away.

\spellList{Rusting Grasp}: Your touch corrodes iron and alloys.

\spellList{Scrying}\textsuperscript{F}: Spies on subject from a distance.

\spellList{Spike Stones}: Creatures in area take 1d8 damage, may be slowed.

\spellList{Summon Nature's Ally IV}: Calls creature to fight.



\subsection{5th-Level Druid Spells}

\spellList{Animal Growth}: One animal/two levels doubles in size.

\spellList{Atonement}: Removes burden of misdeeds from subject.

\spellList{Awaken}\textsuperscript{X}: Animal or tree gains human intellect.

\spellList{Baleful Polymorph}: Transforms subject into harmless animal.

\spellList{Braxatskin}: Your skin hardens, granting armor bonus and damage reduction.

\spellList{Call Lightning Storm}: As call lightning, but 5d6 damage per bolt.

\spellList{Cleansing Flame}: 1d6/level fire damage (max 10d6).

\spellList{Coat of Mists}\textsuperscript{M}: Coalesce a magical mist about the subject's body.

\spellList{Commune with Nature}: Learn about terrain for 1.5 kilometer/level.

\spellList{Control Winds}: Change wind direction and speed.

\spellList{Conversion}\textsuperscript{FX}: Removes burden of acts of defiling from a wizard.

\spellList{Cure Critical Wounds}: Cures 4d8 damage +1/level (max +20).

\spellList{Death Ward}: Grants immunity to all death spells and negative energy effects.

\spellList{Groundflame}: Mist deals 1d6/level acid damage (max 15d6).

\spellList{Hallow}\textsuperscript{M}: Designates location as holy.

\spellList{Insect Plague}: Locust swarms attack creatures.

\spellList{Mark of Justice}: Designates action that will trigger curse on subject.

\spellList{Rejuvenate}: Increase the fertility of the land.

\spellList{Righteous Might}: Your size increases, and you gain +4 Str.

\spellList{Skyfire}: Three exploding spheres each deal 3d6 fire damage.

\spellList{Stoneskin}\textsuperscript{M}: Ignore 10 points of damage per attack.

\spellList{Summon Nature's Ally V}: Calls creature to fight.

\spellList{Transmute Mud to Rock}: Transforms two 3-m cubes per level.

\spellList{Transmute Rock to Mud}: Transforms two 3-m cubes per level.

\spellList{Tree Stride}: Step from one tree to another far away.

\spellList{Unhallow}\textsuperscript{M}: Designates location as unholy.

\spellList{Wall of Fire}: Deals 2d4 fire damage out to 3 m and 1d4 out to 6 m. Passing through wall deals 2d6 damage +1/level.

\spellList{Wall of Thorns}: Thorns damage anyone who tries to pass.



\subsection{6th-Level Druid Spells}

\spellList{Allegiance of the Land}: Grants bonus to AC, temporary hit points, and energy resistance.

\spellList{Antilife Shell}: 3-m radius field hedges out living creatures.

\spellList{Awaken Water Spirits}: Gives sentience to a natural body of water.

\spellList{Bear's Endurance, Mass}: As bear's endurance, affects one subject/ level.

\spellList{Bull's Strength, Mass}: As bull's strength, affects one subject/level.

\spellList{Cat's Grace, Mass}: As cat's grace, affects one subject/level.

\spellList{Create Oasis}: Conjures a temporary oasis.

\spellList{Cure Light Wounds, Mass}: Cures 1d8 damage +1/level for many creatures.

\spellList{Dispel Magic, Greater}: As \spell{dispel magic}, but +20 on check.

\spellList{Find the Path}: Shows most direct way to a location.

\spellList{Fire Seeds}: Acorns and berries become grenades and bombs.

\spellList{Infestation}: Tiny parasites infect creatures within area.

\spellList{Ironwood}: Magic wood is strong as steel.

\spellList{Liveoak}: Oak becomes treant guardian.

\spellList{Move Earth}: Digs trenches and builds hills.

\spellList{Owl's Wisdom, Mass}: As owl's wisdom, affects one subject/level.

\spellList{Raise Dead}\textsuperscript{M}: Restores life to subject who died up to 1 day/level ago.

\spellList{Repel Wood}: Pushes away wooden objects.

\spellList{Spellstaff}: Stores one spell in wooden quarterstaff.

\spellList{Stone Tell}: Talk to natural or worked stone.

\spellList{Summon Nature's Ally VI}: Calls creature to fight.

\spellList{Transport via Plants}: Move instantly from one plant to another of the same kind.

\spellList{Wall of Stone}: Creates a stone wall that can be shaped.



\subsection{7th-Level Druid Spells}

\spellList{Animate Plants}: One or more plants animate and fight for you.

\spellList{Changestaff}: Your staff becomes a treant on command.

\spellList{Control Weather}: Changes weather in local area.

\spellList{Creeping Doom}: Swarms of centipedes attack at your command.

\spellList{Cure Moderate Wounds, Mass}: Cures 2d8 damage +1/level for many creatures.

\spellList{Fire Storm}: Deals 1d6/level fire damage.

\spellList{Heal}: Cures 10 points/level of damage, all diseases and mental conditions.

\spellList{Scrying, Greater}: As scrying, but faster and longer.

\spellList{Summon Nature's Ally VII}: Calls creature to fight.

\spellList{Sunbeam}: Beam blinds and deals 4d6 damage.

\spellList{Transmute Metal to Wood}: Metal within 12 m becomes wood.

\spellList{True Seeing}\textsuperscript{M}: Lets you see all things as they really are.

\spellList{Waters of Life}\textsuperscript{M}: Absorb another creature's ailments.

\spellList{Wind Walk}: You and your allies turn vaporous and travel fast.



\subsection{8th-Level Druid Spells}

\spellList{Animal Shapes}: One ally/level polymorphs into chosen animal.

\spellList{Control Plants}: Control actions of one or more plant creatures.

\spellList{Cure Serious Wounds, Mass}: Cures 3d8 damage +1/level for many creatures.

\spellList{Earthquake}: Intense tremor shakes 24-m radius.

\spellList{Finger of Death}: Kills one subject.

\spellList{Flame Harvest}: Creates a timed fire trap.

\spellList{Repel Metal or Stone}: Pushes away metal and stone.

\spellList{Reverse Gravity}: Objects and creatures fall upward.

\spellList{Sirocco}: You conjure a legendary desert wind.

\spellList{Summon Nature's Ally VIII}: Calls creature to fight.

\spellList{Sunburst}: Blinds all within 3 m, deals 6d6 damage.

\spellList{Whirlwind}: Cyclone deals damage and can pick up creatures.

\spellList{Word of Recall}: Teleports you back to designated place.



\subsection{9th-Level Druid Spells}

\spellList{Antipathy}: Object or location affected by spell repels certain creatures.

\spellList{Cure Critical Wounds, Mass}: Cures 4d8 damage +1/level for many creatures.

\spellList{Elemental Swarm}: Summons multiple elementals.

\spellList{Flash Flood}: Conjures a flood.

\spellList{Foresight}: ``Sixth sense'' warns of impending danger.

\spellList{Heartseeker}\textsuperscript{X}: Creates a deadly piercing weapon.

\spellList{Regenerate}: Subject's severed limbs grow back, cures 4d8 damage +1/level (max +35).

\spellList{Shambler}: Summons 1d4+2 shambling mounds to fight for you.

\spellList{Shapechange}\textsuperscript{F}: Transforms you into any creature, and change forms once per round.

\spellList{Storm Legion}: Transports willing creatures via a natural storm.

\spellList{Storm of Vengeance}: Storm rains acid, lightning, and hail.

\spellList{Summon Nature's Ally IX}: Calls creature to fight.

\spellList{Swarm of Anguish}: Transforms you into a swarm of agony beetles.

\spellList{Sympathy}\textsuperscript{M}: Object or location attracts certain creatures.

\spellList{Wild Lands}: Attract wild creatures to an area.
\section{Ranger Spells}



\subsection{1st-Level Ranger Spells}

\spellList{Alarm}: Wards an area for 2 hours/level.

\spellList{Animal Messenger}: Sends a Tiny animal to a specific place.

\spellList{Calm Animals}: Calms (2d4 + level) HD of animals.

\spellList{Charm Animal}: Makes one animal your friend.

\spellList{Cooling Canopy}: Summons a cloud to provide shade and prevent dehydration. %%

\spellList{Delay Poison}: Stops poison from harming subject for 1 hour/level.

\spellList{Detect Animals or Plants}: Detects kinds of animals or plants.

\spellList{Detect Poison}: Detects poison in one creature or object.

\spellList{Detect Snares and Pits}: Reveals natural or primitive traps.

\spellList{Endure Elements}: Exist comfortably in hot or cold environments.

\spellList{Entangle}: Plants entangle everyone in 12-m radius circle.

\spellList{Hide from Animals}: Animals can't perceive one subject/level.

\spellList{Jump}: Subject gets bonus on Jump checks.

\spellList{Longstrider}: Increases your speed.

\spellList{Magic Fang}: One natural weapon of subject creature gets +1 on attack and damage rolls.

\spellList{Nurturing Seeds}: Makes 10 seeds or cuttings hardy and easy to transplant. %%

\spellList{Pass without Trace}: One subject/level leaves no tracks.

\spellList{Plant Renewal}: Brings one plant back from near destruction. %%

\spellList{Read Magic}: Read scrolls and spellbooks.

\spellList{Resist Energy}: Ignores 10 (or more) points of damage/attack from specified energy type.

\spellList{Speak with Animals}: You can communicate with animals.

\spellList{Summon Nature's Ally I}: Calls animal to fight for you.



\subsection{2nd-Level Ranger Spells}

\spellList{Barkskin}: Grants +2 (or higher) enhancement to natural armor.

\spellList{Bear's Endurance}: Subject gains +4 to Con for 1 min./level.

\spellList{Cat's Grace}: Subject gains +4 to Dex for 1 min./level.

\spellList{Cure Light Wounds}: Cures 1d8 damage +1/level (max +5).

\spellList{Echo of the Lirr}: Stuns creatures in a cone. %%

\spellList{Footsteps of the Quarry}\textsuperscript{M}: Track a specific creature or person. %%

\spellList{Hold Animal}: Paralyzes one animal for 1 round/level.

\spellList{Owl's Wisdom}: Subject gains +4 to Wis for 1 min./level.

\spellList{Protection from Energy}: Absorb 12 points/level of damage from one kind of energy.

\spellList{Snare}: Creates a magic booby trap.

\spellList{Speak with Plants}: You can talk to normal plants and plant creatures.

\spellList{Spike Growth}: Creatures in area take 1d4 damage, may be slowed.

\spellList{Sting of the Gold Scorpion}: Enlivens scorpion barb to strike with poison of real scorpion once. %%

\spellList{Summon Nature's Ally II}: Calls animal to fight for you.

\spellList{Wind Wall}: Deflects arrows, smaller creatures, and gases.



\subsection{3rd-Level Ranger Spells}

% \spellList{Water Walk}: Subject treads on water as if solid.

\spellList{Claws of the Tembo}: Deals 1d6 Str damage and transfers hp. %%

\spellList{Command Plants}: Sway the actions of one or more plant creatures.

\spellList{Cure Moderate Wounds}: Cures 2d8 damage +1/level (max +10).

\spellList{Darkvision}: See 18 m in total darkness.

\spellList{Diminish Plants}: Reduces size or blights growth of normal plants.

\spellList{Eye of the Storm}: Protects 9 m radius from effects of storm for 1 hour/level. %%

\spellList{Magic Fang, Greater}: One natural weapon of subject creature gets +1/three caster levels on attack and damage rolls (max +5).

\spellList{Neutralize Poison}: Immunizes subject against poison, detoxifies venom in or on subject.

\spellList{Plant Growth}: Grows vegetation, improves crops.

\spellList{Reduce Animal}: Shrinks one willing animal.

\spellList{Remove Disease}: Cures all diseases affecting subject.

\spellList{Repel Vermin}: Insects, spiders, and other vermin stay 3 m away.

\spellList{Summon Nature's Ally III}: Calls animal to fight for you.

\spellList{Surface Walk}: Subject treads on unstable surfaces as if solid. %%

\spellList{Tree Shape}: You look exactly like a tree for 1 hour/level.

\spellList{Worm's Breath}: Subjects can breathe underwater, in silt or earth. %%



\subsection{4th-Level Ranger Spells}

\spellList{Animal Growth}: One animal/two levels doubles in size.

\spellList{Commune with Nature}: Learn about terrain for 1 mile/level.

\spellList{Cure Serious Wounds}: Cures 3d8 damage +1/level (max +15).

\spellList{Freedom of Movement}: Subject moves normally despite impediments.

\spellList{Nondetection}\textsuperscript{M}: Hides subject from divination, scrying.

\spellList{Summon Nature's Ally IV}: Calls animal to fight for you.

\spellList{Tree Stride}: Step from one tree to another far away.

\section{Templar Spells}



\subsection{0-Level Templar Spells (Orisons)}

\spellList{Cure Minor Wounds}: Cures 1 point of damage.

\spellList{Defiler Scent}: Smells presence or absence of defilers.

\spellList{Detect Magic}: Detects spells and magical items within 18 m.

\spellList{Detect Poison}: Detects poison in one creature or small object.

\spellList{Guidance}: +1 on one attack roll, saving throw, or skill check.

\spellList{Inflict Minor Wounds}: Touch attack, 1 point of damage.

\spellList{Light}: Object shines like a torch.

\spellList{Mending}: Makes minor repairs on an object.

\spellList{Read Magic}: Read scrolls and spellbooks

\spellList{Resistance}: Subject gains +1 on saving throws.

\spellList{Virtue}: Subject gains 1 temporary hp.



\subsection{1st-Level Templar Spells}

\spellList{Black Cairn}\textsuperscript{F}: Locates a corpse nearby.

\spellList{Command}: One subject obeys selected command for 1 round.

\spellList{Comprehend Languages}: Understand all spoken and written languages.

\spellList{Cure Light Wounds}: Cures 1d8+1/level damage (max +5).

\spellList{Deathwatch}: Sees how wounded subjects within 9 m are.

\spellList{Detect Undead}: Reveals undead within 18 m.

\spellList{Divine Favor}: You gain attack, damage bonus, +1/three levels.

\spellList{Doom}: One subject suffers $-2$ on attacks, damage, saves, and checks.

\spellList{Endure Elements}: Exist comfortably in hot or cold environments.

\spellList{Hand of the Sorcerer-King}: Protects caster from spells.

\spellList{Hide From Undead}: Undead can't perceive one subject/level.

\spellList{Inflict Light Wounds}: Touch deals 1d8 damage +1/level (max +5).

\spellList{Remove Fear}: +4 on saves against fear for one subject +1/four levels.

\spellList{Protection from Evil/Good}: +2 to AC and saves, counter mind control, hedge out elementals and outsiders.

\spellList{Shield of Faith}: Aura grants +2 or higher deflection bonus.



\subsection{2nd-Level Templar Spells}

\spellList{Battlefield Healing}: Stabilizes one creature/level.

\spellList{Bear's Endurance}: Subject gains +4 Con for 1 min./level.

\spellList{Cure Moderate Wounds}: Cures 2d8+1/level damage (max +10).

\spellList{Delay Poison}: Stops poison from harming subject for 1 hour/level.

\spellList{Enthrall}: Captivates all within 30 m + 3 m/level.

\spellList{Footsteps of the Quarry}\textsuperscript{M}: Track a specific creature or person.

\spellList{Hold Person}: Holds one person helpless; 1 round/level.

\spellList{Inflict Moderate Wounds}: Touch attack, 2d8 damage +1/level (max +10).

\spellList{Remove Paralysis}: Frees one or more creatures from paralysis, hold, or slow.

\spellList{Resist Energy}: Ignores 10 (or more) points of damage/attack from specified energy type.

\spellList{Restoration, Lesser}: Dispels magic ability penalty or repairs 1d4 ability damage.

\spellList{Return to the Earth}: Turns dead and undead bodies into dust.

\spellList{Silence}: Negates sound in 4.5-mradius.

\spellList{Undetectable Alignment}: Conceals alignment for 24 hours.

\spellList{Zone of Truth}: Subjects within range cannot lie.



\subsection{3rd-Level Templar Spells}

\spellList{Cure Serious Wounds}: Cures 3d8+1/level damage (max +15).

\spellList{Dedication}: Allows target to avoid sleep, consume half food and water, and +1 to attack, damage, saves, ability, and skill checks while pursuing a specified task.

\spellList{Discern Lies}: Reveals deliberate falsehoods.

\spellList{Dispel Magic}: Cancels magical spells and effects.

\spellList{Glyph of Warding}: Inscription harms those who pass it.

\spellList{Image of the Sorcerer-King}: Touched creatures must save or become affected by cause fear.

\spellList{Inflict Serious Wounds}: Touch attack, 3d8 damage +1/level (max +15).

\spellList{Lightning Bolt}: Electricity deals 1d6/level damage.

\spellList{Locate Object}: Senses direction toward object (specific or type).

\spellList{Magic Circle against Evil/Good}: As \emph{protection} spells, but 3-m radius and 10 min./level.

\spellList{Magic Vestment}: Armor or shield gains +1 enhancement per four levels.

\spellList{Protection from Energy}: Absorb 12 points/level of damage from one kind of energy.

\spellList{Remove Disease}: Cures all diseases affecting subject.

\spellList{Sand Pit}: Excavates sand in a 9 m wide and 15 m deep cone.

\spellList{Speak with Dead}: Corpse answers one question/two levels.

\spellList{Surface Walk}: Subject treads on unstable surfaces as if solid.

\spellList{Wind Wall}: Deflects arrows, smaller creatures, and gases.

\spellList{Worm's Breath}: Subjects can breathe underwater, in silt, or earth.



\subsection{4th-Level Templar Spells}

\spellList{Air Walk}: Subject treads on air as if solid (climb at 45-degree angle).

\spellList{Command, Greater}: As command, but affects one subject/level.

\spellList{Cure Critical Wounds}: Cures 4d8+1/level damage (max +20).

\spellList{Dimensional Anchor}: Bars extradimensional movement.

\spellList{Fool's Feast}: Enhances food for one creature/level and blesses.

\spellList{Freedom of Movement}: Subject moves normally despite impediments.

\spellList{Geas, Lesser}: Commands subject of 7 HD or less.

\spellList{Inflict Critical Wounds}: Touch attack, 4d8 damage +1/level (max +20).

\spellList{Mage Seeker}\textsuperscript{F}: Locate nearby wizard.

\spellList{Magic Weapon, Greater}: +1 bonus/four levels (max +5).

\spellList{Neutralize Poison}: Detoxifies venom in or on subject.

\spellList{Sending}: Delivers a short message anywhere, instantly.

\spellList{Status}: Monitors condition, position of allies.

\spellList{Tongues}: Speak any language.

\spellList{Wrath of the Sorcerer-King}: Know if a creature has broken the law, and punish him.



\subsection{5th-Level Templar Spells}

\spellList{Air Lens}: Directs intensified sunlight at foes within range.

\spellList{Break Enchantment}: Frees subjects from enchantments, alterations, curses, and petrifaction.

\spellList{Elemental Strike}: Smites foes with 1d6/level of divine and elemental energy (max 15d6).

\spellList{Fire Track}: Fiery spark follows tracks.

\spellList{Klar's Heart}: Enhances combat abilities of all creatures within range.

\spellList{Mark of Justice}: Designates action that will trigger curse on subject.

\spellList{Scrying}: Spies on subject from a distance.

\spellList{Symbol of Pain}\textsuperscript{M}: Triggered rune wracks nearby creatures with pain.

\spellList{Symbol of Sleep}\textsuperscript{M}: Triggered rune puts nearby creatures into catatonic slumber.

\spellList{True Seeing}: See all things as they really are.



\subsection{6th-Level Templar Spells}

\spellList{Control Tides}: Raises, lowers, or parts bodies of water or silt.

\spellList{Dispel Magic, Greater}: As \spell{dispel magic}, but up to +20 on check.

\spellList{Forbiddance}\textsuperscript{M}: Blocks planar travel, damages creatures of different alignment.

\spellList{Glyph of Warding, Greater}: As glyph of warding, but up to 10d8 damage or 6th level spell.

\spellList{Raise Dead}: Restores life to subject who died up to 1 day/level ago.

\spellList{Symbol of Fear}\textsuperscript{M}: Triggered rune panics nearby creatures.

\spellList{Symbol of Persuasion}\textsuperscript{M}: Triggered rune charms nearby creatures.

\spellList{Wisdom of the Sorcerer-King}: Apply metamagic to one spell of up to 4th level.

\spellList{Word of Recall}: Teleports you back to designated place.



\subsection{7th-Level Templar Spells}

\spellList{Confessor's Flame}: Uses threat of flame to extract confession.

\spellList{Crusade}: Allies receive +4 bonus to attack rolls, damage rolls, and saving throws, 2d8 hit points, and immunity to magical fear.

\spellList{Refuge}: Alters item to transport its possessor to you.

\spellList{Scrying, Greater}: As scrying, but faster and longer.

\spellList{Symbol of Stunning}\textsuperscript{M}: Triggered rune stuns nearby creatures.

\spellList{Symbol of Weakness}\textsuperscript{M}: Triggered rune weakens nearby creatures.



\subsection{8th-Level Templar Spells}

\spellList{Antipathy}: Object or location affected by spell repels certain creatures.

\spellList{Discern Location}: Exact location of creature or object.

\spellList{Finger of Death}: Kills one subject.

\spellList{Poisoned Gale}: Poisonous cloud (3 m wide, 3 m high) emanates out from you to the extreme of the range.

\spellList{Regenerate}: Subject's severed limbs grow back.

\spellList{Symbol of Death}\textsuperscript{M}: Triggered rune slays nearby creature.



\subsection{9th-Level Templar Spells}

\spellList{Energy Drain}: Subject gains 2d4 negative levels.

\spellList{Gray Rift}: A hovering rift to the Gray bolsters
undead.

\spellList{Power Word, Blind}: Blinds 200 hp worth of creatures.

\spellList{Soul Bind}: Traps newly dead soul to prevent
resurrection.
\section{Templar Domains}

\Domain{Animal}
{Lalali-Puy}
{You can use \spell{speak with animals} once per day as a spell-like ability.

Add \skill{Knowledge} (nature) to your list of cleric class skills.}
{
	\item \spellList{Calm Animals}: Calms (2d4 + level) HD of animals.
	\item \spellList{Hold Animal}: Paralyzes one animal for 1 round/level.
	\item \spellList{Dominate Animal}: Subject animal obeys silent mental commands.
	\item \spellList{Summon Nature's Ally IV}\footnotemark[1]: Calls creature to fight.
	\item \spellList{Commune with Nature}: Learn about terrain for 1 mile/level.
	\item \spellList{Antilife Shell}: 3-m field hedges out living creatures.
	\item \spellList{Animal Shapes}: One ally/level polymorphs into chosen animal.
	\item \spellList{Summon Nature's Ally VIII}\footnotemark[1]: Calls creature to fight.
	\item \spellList{Shapechange}\textsuperscript{F}: Transforms you into any creature, and change forms once per round.
}
\noindent 1 Can only summon animals.


\Domain{Chaos}
{Abalach-Re, Daskinor}
{You cast chaos spells at +1 caster level.}
{
	\item \spellList{Protection from Law}: +2 to AC and saves, counter mind control, hedge out elementals and outsiders.
	\item \spellList{Shatter}: Sonic vibration damages objects or crystalline creatures.
	\item \spellList{Magic Circle against Law}: As \spell{protection spells}, but 3-m radius and 10 min./level.
	\item \spellList{Chaos Hammer}: Damages and staggers lawful creatures.
	\item \spellList{Dispel Law}: +4 bonus against attacks by lawful creatures.
	\item \spellList{Animate Objects}: Objects attack your foes.
	\item \spellList{Word of Chaos}: Kills, confuses, stuns, or deafens nonchaotic subjects.
	\item \spellList{Cloak of Chaos}\textsuperscript{F}: +4 to AC, +4 resistance, SR 25 against lawful spells.
	\item \spellList{Summon Monster IX}\footnotemark[1]: Calls extraplanar creature to fight for you.
}
\noindent 1 Cast as a chaos spell only.

\Domain{Charm}
{Abalach-Re}
{You can boost your Charisma by 4 points once per day. Activating this power is a free action. The Charisma increase lasts 1 minute.}
{
	\item \spellList{Charm Person}: Makes one person your friend.
	\item \spellList{Calm Emotions}: Calms creatures, negating emotion effects.
	\item \spellList{Suggestion}: Compels subject to follow stated course of action.
	\item \spellList{Heroism}: Gives +2 on attack rolls, saves, skill checks.
	\item \spellList{Charm Monster}: Makes monster believe it is your ally.
	\item \spellList{Geas/Quest}: As \spell{lesser geas}, plus it affects any creature.
	\item \spellList{Insanity}: Subject suffers continuous confusion.
	\item \spellList{Demand}: As \spell{sending}, plus you can send suggestion.
	\item \spellList{Dominate Monster}: As \spell{dominate person}, but any creature.
}

\Domain{Death}
{Dregoth}
{You may use a death touch once per day. Your death touch is a supernatural ability that produces a death effect. You must succeed on a melee touch attack against a living creature (using the rules for touch spells). When you touch, roll 1d6 per cleric level you possess. If the total at least equals the creature's current hit points, it dies (no save).}
{
	\item \spellList{Cause Fear}: One creature of 5 HD or less flees for 1d4 rounds.
	\item \spellList{Death Knell}: Kill dying creature and gain 1d8 temporary hp, +2 to Str, and +1 caster level.
	\item \spellList{Animate Dead}\textsuperscript{M}: Creates undead skeletons and zombies.
	\item \spellList{Death Ward}: Grants immunity to death spells and negative energy effects.
	\item \spellList{Slay Living}: Touch attack kills subject.
	\item \spellList{Create Undead}\textsuperscript{M}: Create ghouls, ghasts, mummies, or mohrgs.
	\item \spellList{Destruction}\textsuperscript{F}: Kills subject and destroys remains.
	\item \spellList{Create Greater Undead}\textsuperscript{M}: Create shadows, wraiths, spectres, or devourers.
	\item \spellList{Wail of the Banshee}: Kills one creature/level.
}

\Domain{Destruction}
{Borys, Dregoth}
{You gain the smite power, the supernatural ability to make a single melee attack with a +4 bonus on attack rolls and a bonus on damage rolls equal to your cleric level (if you hit). You must declare the smite before making the attack. This ability is usable once per day.}
{
	\item \spellList{Inflict Light Wounds}: Touch attack, 1d8 damage +1/level (max +5).
	\item \spellList{Shatter}: Sonic vibration damages objects or crystalline creatures.
	\item \spellList{Contagion}: Infects subject with chosen disease.
	\item \spellList{Inflict Critical Wounds}: Touch attack, 4d8 damage +1/level (max +20).
	\item \spellList{Inflict Light Wounds, Mass}: Deals 1d8 damage +1/level to any creatures.
	\item \spellList{Harm}: Deals 10 points/level damage to target.
	\item \spellList{Disintegrate}: Makes one creature or object vanish.
	\item \spellList{Earthquake}: Intense tremor shakes 24-m radius.
	\item \spellList{Implosion}: Kills one creature/round.
}

\Domain{Glory}
{Tectuktilay}
{You gain a +2 bonus on turning checks and +1d6 to turning damage rolls.}
{
	\item \spellList{Disrupt Undead}: Deals 1d6 damage to one undead.
	\item \spellList{Bless Weapon}: Weapon strikes true against evil foes.
	\item \spellList{Searing Light}: Ray deals 1d8/two levels damage, more against undead.
	\item \spellList{Holy Smite}: Damages and blinds evil creatures.
	\item \spellList{Holy Sword}: Weapon becomes +5, deals +2d6 damage against evil.
	\item \spellList{Bolt of Glory}: Positive energy ray deals extra damage to evil outsiders and undead.
	\item \spellList{Sunbeam}: Beam blinds and deals 4d6 damage.
	\item \spellList{Crown of Glory}: You gain +4 Charisma and enthrall those who hear you.
	\item \spellList{Gate}\textsuperscript{X}: Connects two planes for travel or summoning.
}

\Domain{Knowledge}
{Oronis}
{Add all \skill{Knowledge} skills to your list of cleric class skills.

You cast divination spells at +1 caster level.}
{
	\item \spellList{Detect Secret Doors}: Reveals hidden doors within 18 m.
	\item \spellList{Detect Thoughts}: Allows ``listening'' to surface thoughts.
	\item \spellList{Clairaudience/Clairvoyance}: Hear or see at a distance for 1 min./level.
	\item \spellList{Divination}\textsuperscript{M}: Provides useful advice for specific proposed actions.
	\item \spellList{True Seeing}\textsuperscript{M}: Lets you see all things as they really are.
	\item \spellList{Find the Path}: Shows most direct way to a location.
	\item \spellList{Legend Lore}\textsuperscript{MF}: Lets you learn tales about a person, place, or thing.
	\item \spellList{Discern Location}: Reveals exact location of creature or object.
	\item \spellList{Foresight}: ``Sixth sense'' warns of impending danger.
}

\Domain{Law}
{Andropinis}
{You cast law spells at +1 caster level.}
{
	\item \spellList{Protection from Chaos}: +2 to AC and saves, counter mind control, hedge out elementals and outsiders.
	\item \spellList{Calm Emotions}: Calms creatures, negating emotion effects.
	\item \spellList{Magic Circle against Chaos}: As \spell{protection spells}, but 3-m radius and 10 min./level.
	\item \spellList{Order's Wrath}: Damages and dazes chaotic creatures.
	\item \spellList{Dispel Chaos}: +4 bonus against attacks by chaotic creatures.
	\item \spellList{Hold Monster}: As \spell{hold person}, but any creature.
	\item \spellList{Dictum}: Kills, paralyzes, slows, or deafens nonlawful subjects.
	\item \spellList{Shield of Law}\textsuperscript{F}: +4 to AC, +4 resistance, and SR 25 against chaotic spells.
	\item \spellList{Summon Monster IX}\footnotemark[1]: Calls extraplanar creature to fight for you.
}
\noindent 1 Cast as a law spell only.

\Domain{Madness}
{Daskinor}
{You have a $-1$ penalty for each three cleric levels on all Wisdom-based skills.

Once per day, you can see and act with the clarity of true madness. You can add one-half your cleric levels on a single Wisdom-based skill check or Will save. You must choose to use this power before the roll is made.}
{
	\item \spellList{Confusion, Lesser}: One creature is confused for 1 round.
	\item \spellList{Touch of Madness}: Dazes one creature for 1 round/level.
	\item \spellList{Rage}: Gives +2 to Str and Con, +1 on Will saves, $-2$ to AC.
	\item \spellList{Confusion}: Subjects behave oddly for 1 round/level.
	\item \spellList{Bolts of Bedevilment}: One ray/round, dazes 1d3 rounds.
	\item \spellList{Phantasmal Killer}: Fearsome illusion kills subject or deals 3d6 damage.
	\item \spellList{Insanity}: Subject suffers continuous confusion.
	\item \spellList{Maddening Scream}: Subject has $-4$ AC, no shield, Reflex save on 20 only.
	\item \spellList{Weird}: As \spell{phantasmal killer}, but affects all within 30 ft.
}

\Domain{Magic}
{Kalak, Nibenay}
{Use scrolls, wands, and other devices with spell completion or spell trigger activation as a wizard of one-half your cleric level (at least 1st level). For the purpose of using a scroll or other magic device, if you are also a wizard, actual wizard levels and these effective wizard levels stack.}
{
	\item \spellList{Magic Aura}: Alters object's magic aura.
	\item \spellList{Identify}: Determines properties of magic item.
	\item \spellList{Dispel Magic}: Cancels magical spells and effects.
	\item \spellList{Imbue with Spell Ability}: Transfer spells to subject.
	\item \spellList{Spell Resistance}: Subject gains SR 12 + level.
	\item \spellList{Antimagic Field}: Negates magic within 3 m.
	\item \spellList{Spell Turning}: Reflect 1d4+6 spell levels back at caster.
	\item \spellList{Protection from Spells}\textsuperscript{M}\textsuperscript{F}: Confers +8 resistance bonus.
	\item \spellList{Mage's Disjunction}: Dispels magic, disenchants magic items.
}

\Domain{Mind}
{Nibenay}
{Gain a +2 bonus on \skill{Bluff}, \skill{Diplomacy}, and \skill{Sense Motive} checks. Gain a +2 bonus on Will saves against enchantment spells and effects.}
{
	\item \spellList{Comprehend Languages}: You understand all spoken and written languages.
	\item \spellList{Detect Thoughts}: Allows ``listening'' to surface thoughts.
	\item \spellList{Telepathic Bond, Lesser}: Link with subject within 9 m for 10 min./level.
	\item \spellList{Discern Lies}: Reveals deliberate falsehoods.
	\item \spellList{Telepathic Bond}: Link lets allies communicate.
	\item \spellList{Probe Thoughts}: Read subject's memories, one question/round.
	\item \spellList{Brain Spider}: Eavesdrop on thoughts of up to eight other creatures.
	\item \spellList{Mind Blank}: Subject is immune to mental/emotional magic and scrying.
	\item \spellList{Weird}: Fearful illusion affects all within 9 m, either killing or dealing 3d6 damage.
}

\Domain{Nobility}
{Andropinis}
{You have the spell-like ability to inspire allies, giving you a +2 morale bonus on saving throws, attack rolls, ability checks, skill checks, and weapon damage rolls. Allies must be able to hear the character speak for 1 round. Using this ability is a standard action. It lasts a number of rounds equal to your Charisma bonus and can be used once per day.}
{
	\item \spellList{Divine Favor}: You gain +1 per three levels on attack and damage rolls.
	\item \spellList{Enthrall}: Captivates all within 30 m + 3 m/level.
	\item \spellList{Magic Vestment}: Armor or shield gains +1 enhancement per four levels.
	\item \spellList{Discern Lies}: Reveals deliberate falsehoods.
	\item \spellList{Command, Greater}: As \spell{command}, but affects one subject/level.
	\item \spellList{Geas/Quest}: As \spell{lesser geas}, plus it affects any creature.
	\item \spellList{Repulsion}: Creatures can't approach you.
	\item \spellList{Demand}: As \spell{sending}, plus you can send suggestion.
	\item \spellList{Storm of Vengeance}: Storm rains acid, lightning, and hail.
}

\Domain{Plant}
{Lalali-Puy}
{Rebuke or command plant creatures as an evil cleric rebukes or commands undead. Use this ability a total number of times per day equal to 3 + your Charisma modifier. This granted power is a supernatural ability.

Add \skill{Knowledge} (nature) to your list of cleric class skills.}
{
	\item \spellList{Entangle}: Plants entangle everyone in 12-m radius.
	\item \spellList{Barkskin}: Grants +2 (or higher) enhancement to natural armor.
	\item \spellList{Plant Growth}: Grows vegetation, improves crops.
	\item \spellList{Command Plants}: Sway the actions of one or more plant creatures.
	\item \spellList{Wall of Thorns}: Thorns damage anyone who tries to pass.
	\item \spellList{Repel Wood}: Pushes away wooden objects.
	\item \spellList{Animate Plants}: One or more trees animate and fight for you.
	\item \spellList{Control Plants}: Control actions of one or more plant creatures.
	\item \spellList{Shambler}: Summons 1d4+2 shambling mounds to fight for you.
}

\Domain{Protection}
{Borys, Oronis}
{You can generate a protective ward as a supernatural ability. Grant someone you touch a resistance bonus equal to your cleric level on his or her next saving throw. Activating this power is a standard action. The protective ward is an abjuration effect with a duration of 1 hour that is usable once per day.}
{
	\item \spellList{Sanctuary}: Opponents can't attack you, and you can't attack.
	\item \spellList{Shield Other}\textsuperscript{F}: You take half of subject's damage.
	\item \spellList{Protection from Energy}: Absorb 12 points/level of damage from one kind of energy.
	\item \spellList{Spell Immunity}: Subject is immune to one spell per four levels.
	\item \spellList{Spell Resistance}: Subject gains SR 12 + level.
	\item \spellList{Antimagic Field}: Negates magic within 3 m.
	\item \spellList{Repulsion}: Creatures can't approach you.
	\item \spellList{Mind Blank}: Subject is immune to mental/emotional magic and scrying.
	\item \spellList{Prismatic Sphere}: As \spell{prismatic wall}, but surrounds on all sides.
}

\Domain{Strength}
{Hamanu, Tectuktilay}
{You can perform a feat of strength as a supernatural ability. You gain an enhancement bonus to Strength equal to your cleric level. Activating the power is a free action, the power lasts 1 round, and it is usable once per day.}
{
	\item \spellList{Enlarge Person}: Humanoid creature doubles in size.
	\item \spellList{Bull's Strength}: Subject gains +4 to Str for 1 min./level.
	\item \spellList{Magic Vestment}: Armor or shield gains +1 enhancement per four levels.
	\item \spellList{Spell Immunity}: Subject is immune to one spell per four levels.
	\item \spellList{Righteous Might}: Your size increases, and you gain combat bonuses.
	\item \spellList{Stoneskin}\textsuperscript{M}: Ignore 10 points of damage per attack.
	\item \spellList{Grasping Hand}: Large hand provides cover, pushes, or grapples.
	\item \spellList{Clenched Fist}: Large hand provides cover, pushes, or attacks your foes.
	\item \spellList{Crushing Hand}: Large hand provides cover, pushes, or crushes your foes.
}


\Domain{Trickery}
{Kalak}
{Add \skill{Bluff}, \skill{Disguise}, and \skill{Hide} to your list of cleric class skills.}
{
	\item \spellList{Disguise Self}: Disguise own appearance.
	\item \spellList{Invisibility}: Subject invisible 1 min./level or until it attacks.
	\item \spellList{Nondetection}\textsuperscript{M}: Hides subject from divination, scrying.
	\item \spellList{Confusion}: Subjects behave oddly for 1 round/level.
	\item \spellList{False Vision}\textsuperscript{M}: Fools scrying with an illusion.
	\item \spellList{Mislead}: Turns you invisible and creates illusory double.
	\item \spellList{Screen}: Illusion hides area from vision, scrying.
	\item \spellList{Polymorph Any Object}: Changes any subject into anything else.
	\item \spellList{Time Stop}: You act freely for 1d4+1 rounds.
}


\Domain{War}
{Hamanu}
{Free \feat{Martial Weapon Proficiency} with deity's favored weapon (if necessary) and \feat{Weapon Focus} with the deity's favored weapon.}
{
	\item \spellList{Magic Weapon}: Weapon gains +1 bonus.
	\item \spellList{Spiritual Weapon}: Magical weapon attacks on its own.
	\item \spellList{Magic Vestment}: Armor or shield gains +1 enhancement per four levels.
	\item \spellList{Divine Power}: You gain attack bonus, +6 to Str, and 1 hp/level.
	\item \spellList{Flame Strike}: Smite foes with divine fire (1d6/level damage).
	\item \spellList{Blade Barrier}: Wall of blades deals 1d6/level damage.
	\item \spellList{Power Word Blind}: Blinds creature with 200 hp or less.
	\item \spellList{Power Word Stun}: Stuns creature with 150 hp or less.
	\item \spellList{Power Word Kill}: Kills creature with 100 hp or less.
}

\section{Wizard Spells}



\subsection{0-Level Wizard Spells (Cantrips)}

\subsubsection{Abjuration}
	\spellList{Resistance}: Subject gains +1 on saving throws.

\subsubsection{Conjuration}
	\spellList{Acid Splash}: Orb deals 1d3 acid damage.

\subsubsection{Divination}
	\spellList{Detect Poison}: Detects poison in one creature or small object.

	\spellList{Detect Magic}: Detects spells and magic items within 18 m.

	\spellList{Read Magic}: Read scrolls and spellbooks.

	\spellList{Slave Scent}: Divines target's social class. %%

\subsubsection{Enchantment}
	\spellList{Daze}: Humanoid creature of 4 HD or less loses next action.

\subsubsection{Evocation}
	\spellList{Dancing Lights}: Creates torches or other lights.

	\spellList{Flare}: Dazzles one creature ($-1$ on attack rolls).

	\spellList{Light}: Object shines like a torch.

	\spellList{Ray of Frost}: Ray deals 1d3 cold damage.

\subsubsection{Illusion}
	\spellList{Ghost Sound}: Figment sounds.

\subsubsection{Necromancy}
	\spellList{Disrupt Undead}: Deals 1d6 damage to one undead.

	\spellList{Touch of Fatigue}: Touch attack fatigues target.

\subsubsection{Transmutation}
	\spellList{Mage Hand}: 2-kg telekinesis.

	\spellList{Mending}: Makes minor repairs on an object.

	\spellList{Message}: Whispered conversation at distance.

	\spellList{Open/Close}: Opens or closes small or light things.

\subsubsection{Universal}
	\spellList{Arcane Mark}: Inscribes a personal rune (visible or invisible).

	\spellList{Prestidigitation}: Performs minor tricks.



\subsection{1st-Level Wizard Spells}

\subsubsection{Abjuration}
	\spellList{Alarm}: Wards an area for 2 hours/level.

	\spellList{Endure Elements}: Exist comfortably in hot or cold environments.

	\spellList{Hold Portal}: Holds door shut.

	\spellList{Protection from Chaos/Evil/Good/Law}: +2 to AC and saves, counter mind control, hedge out elementals and outsiders.

	\spellList{Shield}: Invisible disc gives +4 to AC, blocks magic missiles.

\subsubsection{Conjuration}
	\spellList{Cooling Canopy}: Summons a cloud to provide shade and prevent dehydration. %%

	\spellList{Grease}: Makes 3-m square or one object slippery.

	\spellList{Mage Armor}: Gives subject +4 armor bonus.

	\spellList{Mount}: Summons riding horse for 2 hours/level.

	\spellList{Obscuring Mist}: Fog surrounds you.

	\spellList{Summon Monster I}: Calls extraplanar creature to fight for you.

	\spellList{Unseen Servant}: Invisible force obeys your commands.

\subsubsection{Divination}
	\spellList{Comprehend Languages}: You understand all spoken and written languages.

	\spellList{Detect Secret Doors}: Reveals hidden doors within 18 m.

	\spellList{Detect Undead}: Reveals undead within 18 m.

	\spellList{Identify}\textsuperscript{M}: Determines properties of magic item.

	\spellList{True Strike}: +20 on your next attack roll.

\subsubsection{Enchantment}
	\spellList{Charm Person}: Makes one person your friend.

	\spellList{Hypnotism}: Fascinates 2d4 HD of creatures.

	\spellList{Sleep}: Puts 4 HD of creatures into magical slumber.

\subsubsection{Evocation}
	\spellList{Burning Hands}: 1d4/level fire damage (max 5d4).

	\spellList{Floating Disk}: Creates 1-m-diameter horizontal disk that holds 50 kg/level.

	\spellList{Magic Missile}: 1d4+1 damage; +1 missile per two levels above 1st (max 5).

	\spellList{Shocking Grasp}: Touch delivers 1d6/level electricity damage (max 5d6).

\subsubsection{Illusion}
	\spellList{Color Spray}: Knocks unconscious, blinds, and/or stuns weak creatures.

	\spellList{Disguise Self}: Changes your appearance.

	\spellList{Illusory Talent}: Provides the appearance of skill. %%

	\spellList{Magic Aura}: Alters object's magic aura.

	\spellList{Silent Image}: Creates minor illusion of your design.

	\spellList{Ventriloquism}: Throws voice for 1 min./level.

\subsubsection{Necromancy}
	\spellList{Cause Fear}: One creature of 5 HD or less flees for 1d4 rounds.

	\spellList{Chill Touch}: One touch/level deals 1d6 damage and possibly 1 Str damage.

	\spellList{Ray of Enfeeblement}: Ray deals 1d6 +1 per two levels Str damage.

\subsubsection{Transmutation}
	\spellList{Animate Rope}: Makes a rope move at your command.

	\spellList{Enlarge Person}: Humanoid creature doubles in size.

	\spellList{Erase}: Mundane or magical writing vanishes.

	\spellList{Expeditious Retreat}: Your speed increases by 9 m.

	\spellList{Feather Fall}: Objects or creatures fall slowly.

	\spellList{Jump}: Subject gets bonus on \skill{Jump} checks.

	\spellList{Magic Weapon}: Weapon gains +1 bonus.

	\spellList{Reduce Person}: Humanoid creature halves in size.



\subsection{2nd-Level Wizard Spells}

\subsubsection{Abjuration}
	\spellList{Arcane Lock}\textsuperscript{M}: Magically locks a portal or chest.

	\spellList{Backlash}\textsuperscript{M}: 1d6+1 damage/2 levels to defilers. %%

	\spellList{Eye of the Storm}: Protects 9-m radius from effects of storm for 1 hour/level.%%

	\spellList{Obscure Object}: Masks object against scrying.

	\spellList{Protection from Arrows}: Subject immune to most ranged attacks.

	\spellList{Resist Energy}: Ignores first 10 (or more) points of damage/attack from specified energy type.

\subsubsection{Conjuration}
	\spellList{Acid Arrow}: Ranged touch attack; 2d4 damage for 1 round +1 round/three levels.

	\spellList{Fog Cloud}: Fog obscures vision.

	\spellList{Glitterdust}: Blinds creatures, outlines invisible creatures.

	\spellList{Summon Monster II}: Calls extraplanar creature to fight for you.

	\spellList{Summon Swarm}: Summons swarm of bats, rats, or spiders.

	\spellList{Web}: Fills 6-m-radius spread with sticky spiderwebs.

\subsubsection{Divination}
	\spellList{Detect Thoughts}: Allows ``listening'' to surface thoughts.

	\spellList{Footsteps of the Quarry}\textsuperscript{M}: Track a specific creature or person. %%

	\spellList{Locate Object}: Senses direction toward object (specific or type).

	\spellList{See Invisibility}: Reveals invisible creatures or objects.

\subsubsection{Enchantment}
	\spellList{Daze Monster}: Living creature of 6 HD or less loses next action.

	\spellList{Hideous Laughter}: Subject loses actions for 1 round/level.

	\spellList{Touch of Idiocy}: Subject takes 1d6 points of Int, Wis, and Cha damage.

	\spellList{Wakefulness}: Target can postpone sleep. %%

\subsubsection{Evocation}
	\spellList{Cerulean Shock}\textsuperscript{M}: Imbue target with harmful static electricity. %%

	\spellList{Continual Flame}\textsuperscript{M}: Makes a permanent, heatless torch.

	\spellList{Darkness}: 6-m radius of supernatural shadow.

	\spellList{Flaming Sphere}: Creates rolling ball of fire, 2d6 damage, lasts 1 round/level.

	\spellList{Gust of Wind}: Blows away or knocks down smaller creatures.

	\spellList{Scorching Ray}: Ranged touch attack deals 4d6 fire damage, +1 ray/four levels (max 3).

	\spellList{Shatter}: Sonic vibration damages objects or crystalline creatures.

\subsubsection{Illusion}
	\spellList{Blur}: Attacks miss subject 20\% of the time.

	\spellList{Hypnotic Pattern}: Fascinates (2d4 + level) HD of creatures.

	\spellList{Invisibility}: Subject is invisible for 1 min./level or until it attacks.

	\spellList{Magic Mouth}\textsuperscript{M}: Speaks once when triggered.

	\spellList{Magic Trick}\textsuperscript{F}: Conceal your spellcasting. %%

	\spellList{Minor Image}: As \spell{silent image}, plus some sound.

	\spellList{Mirror Image}: Creates decoy duplicates of you (1d4 +1 per three levels, max 8).

	\spellList{Misdirection}: Misleads divinations for one creature or object.

	\spellList{Phantom Trap}\textsuperscript{M}: Makes item seem trapped.

\subsubsection{Necromancy}
	\spellList{Blindness/Deafness}: Makes subject blinded or deafened.

	\spellList{Command Undead}: Undead creature obeys your commands.

	\spellList{Death Mark}: Target becomes sickened. %%

	\spellList{False Life}: Gain 1d10 temporary hp +1/level (max +10).

	\spellList{Ghoul Touch}: Paralyzes one subject, which exudes stench that makes those nearby sickened.

	\spellList{Scare}: Panics creatures of less than 6 HD.

	\spellList{Spectral Hand}: Creates disembodied glowing hand to deliver touch attacks.

	\spellList{Sting of the Gold Scorpion}: Enlivens scorpion barb to strike with poison of real scorpion once. %%

\subsubsection{Transmutation}
	\spellList{Alter Self}: Assume form of a similar creature.

	\spellList{Bear's Endurance}: Subject gains +4 to Con for 1 min./level.

	\spellList{Boneharden}: Hardens bone, making weapons stronger and impairing living beings. %%

	\spellList{Bull's Strength}: Subject gains +4 to Str for 1 min./level.

	\spellList{Cat's Grace}: Subject gains +4 to Dex for 1 min./level.

	\spellList{Darkvision}: See 18 m in total darkness.

	\spellList{Eagle's Splendor}: Subject gains +4 to Cha for 1 min./level.

	\spellList{Fox's Cunning}: Subject gains +4 Int for 1 min./level.

	\spellList{Knock}: Opens locked or magically sealed door.

	\spellList{Levitate}: Subject moves up and down at your direction.

	\spellList{Owl's Wisdom}: Subject gains +4 to Wis for 1 min./level.

	\spellList{Pyrotechnics}: Turns fire into blinding light or choking smoke.

	\spellList{Rope Trick}: As many as eight creatures hide in extradimensional space.

	\spellList{Sandstone}: Touch turns sand into sandstone. %%

	\spellList{Spider Climb}: Grants ability to walk on walls and ceilings.

	\spellList{Whispering Wind}: Sends a short message 1.5 kg/level.



\subsection{3rd-Level Wizard Spells}

\subsubsection{Abjuration}
	\spellList{Conservation}\textsuperscript{M}: Protect the land from defilement. %%

	\spellList{Dispel Magic}: Cancels magical spells and effects.

	\spellList{Explosive Runes}: Deals 6d6 damage when read.

	\spellList{Magic Circle against Chaos/Evil/Good/Law}: As \emph{protection} spells, but 3-m radius and 10 min./level.

	\spellList{Nondetection}\textsuperscript{M}: Hides subject from divination, scrying.

	\spellList{Protection from Energy}: Absorb 12 points/level of damage from one kind of energy.

\subsubsection{Conjuration}
	\spellList{Phantom Steed}: Magic horse appears for 1 hour/level.

	\spellList{Sepia Snake Sigil}\textsuperscript{M}: Creates text symbol that immobilizes reader.

	\spellList{Sleet Storm}: Hampers vision and movement.

	\spellList{Stinking Cloud}: Nauseating vapors, 1 round/level.

	\spellList{Summon Monster III}: Calls extraplanar creature to fight for you.

\subsubsection{Divination}
	\spellList{Arcane Sight}: Magical auras become visible to you.

	\spellList{Clairaudience/Clairvoyance}: Hear or see at a distance for 1 min./level.

	\spellList{Telepathic Bond, Lesser}: As \spell{telepathic bond}, but you and one other creature.

	\spellList{Tongues}: Speak any language.

\subsubsection{Enchantment}
	\spellList{Dedication}: Allows target to avoid sleep, consume half food and water, and +1 to attack, damage, saves, ability, and skill checks while pursuing a specified task. %%

	\spellList{Deep Slumber}: Puts 10 HD of creatures to sleep.

	\spellList{Heroism}: Gives +2 bonus on attack rolls, saves, skill checks.

	\spellList{Hold Person}: Paralyzes one humanoid for 1 round/level.

	\spellList{Rage}: Subjects gains +2 to Str and Con, +1 on Will saves, $-2$ to AC.

	\spellList{Suggestion}: Compels subject to follow stated course of action.

\subsubsection{Evocation}
	\spellList{Clear-river}: Blows away or knocks creatures.

	\spellList{Daylight}: 18-m radius of bright light.

	\spellList{Fireball}: 1d6 damage per level, 6-m radius.

	\spellList{Lightning Bolt}: Electricity deals 1d6/level damage.

	\spellList{Tiny Hut}: Creates shelter for ten creatures.

	\spellList{Wind Wall}: Deflects arrows, smaller creatures, and gases.

\subsubsection{Illusion}
	\spellList{Displacement}: Attacks miss subject 50\%.

	\spellList{Illusory Script}\textsuperscript{M}: Only intended reader can decipher.

	\spellList{Invisibility Sphere}: Makes everyone within 3 m invisible.

	\spellList{Major Image}: As \spell{silent image}, plus sound, smell and thermal effects.

\subsubsection{Necromancy}
	\spellList{Boneclaw's Cut}\textsuperscript{F}: Deals damage that continues to cause bleeding damage. %%

	\spellList{Death Whip}\textsuperscript{F}: Whip deals temporary Strength damage. %%

	\spellList{Gentle Repose}: Preserves one corpse.

	\spellList{Halt Undead}: Immobilizes undead for 1 round/level.

	\spellList{Ray of Exhaustion}: Ray makes subject exhausted.

	\spellList{Vampiric Touch}: Touch deals 1d6/two levels damage; caster gains damage as hp.

\subsubsection{Transmutation}
	\spellList{Blink}: You randomly vanish and reappear for 1 round/level.

	\spellList{Flame Arrow}: Arrows deal +1d6 fire damage.

	\spellList{Fly}: Subject flies at speed of 18 m.

	\spellList{Gaseous Form}: Subject becomes insubstantial and can fly slowly.

	\spellList{Haste}: One creature/level moves faster, +1 on attack rolls, AC, and Reflex saves.

	\spellList{Keen Edge}: Doubles normal weapon's threat range.

	\spellList{Magic Weapon, Greater}: +1/four levels (max +5).

	\spellList{Sand Pit}: Excavates sand in a 9 m wide and 15 m deep cone. %%

	\spellList{Secret Page}: Changes one page to hide its real content.

	\spellList{Shrink Item}: Object shrinks to one-sixteenth size.

	\spellList{Slow}: One subject/level takes only one action/round, $-1$ to AC, reflex saves, and attack rolls.

	\spellList{Water Breathing}: Subjects can breathe underwater.

	\spellList{Worm's Breath}: Subjects can breathe underwater, in silt or earth. %%

	\spellList{Zombie Berry}: Enchants 1d4 berries to act like a berry from the zombie plant. %%



\subsection{4th-Level Wizard Spells}

\subsubsection{Abjuration}
	\spellList{Dimensional Anchor}: Bars extradimensional movement.

	\spellList{Fire Trap}\textsuperscript{M}: Opened object deals 1d4 damage +1/level.

	\spellList{Globe of Invulnerability, Lesser}: Stops 1st- through 3rd-level spell effects.

	\spellList{Remove Curse}: Frees object or person from curse.

	\spellList{Stoneskin}\textsuperscript{M}: Ignore 10 points of damage per attack.

\subsubsection{Conjuration}
	\spellList{Black Tentacles}: Tentacles grapple all within 6 m spread.

	\spellList{Dimension Door}: Teleports you short distance.

	\spellList{Minor Creation}: Creates one cloth or wood object.

	\spellList{Secure Shelter}: Creates sturdy cottage.

	\spellList{Solid Fog}: Blocks vision and slows movement.

	\spellList{Summon Monster IV}: Calls extraplanar creature to fight for you.

\subsubsection{Divination}
	\spellList{Arcane Eye}: Invisible floating eye moves 9 m./round.

	\spellList{Detect Scrying}: Alerts you of magical eavesdropping.

	\spellList{Locate Creature}: Indicates direction to familiar creature.

	\spellList{Mage Seeker}\textsuperscript{F}: Locate nearby wizard. %%

	\spellList{Scrying}\textsuperscript{F}: Spies on subject from a distance.

\subsubsection{Enchantment}
	\spellList{Charm Monster}: Makes monster believe it is your ally.

	\spellList{Confusion}: Subjects behave oddly for 1 round/level.

	\spellList{Crushing Despair}: Subjects take $-2$ on attack rolls, damage rolls, saves, and checks.

	\spellList{Geas, Lesser}: Commands subject of 7 HD or less.

	\spellList{Gloomcloud}\textsuperscript{M}: Make one creature despair.

\subsubsection{Evocation}
	\spellList{Dweomer of Transference}: Convert spellcasting into psionic power points.

	\spellList{Fire Shield}: Creatures attacking you take fire damage; you're protected from heat or cold.

	\spellList{Ice Storm}: Hail deals 5d6 damage in cylinder 12 m across.

	\spellList{Resilient Sphere}: Force globe protects but traps one subject.

	\spellList{Sand Spray}: Sprays sand or silt as an area attack. %%

	\spellList{Shout}: Deafens all within cone and deals 5d6 sonic damage.

	\spellList{Wall of Fire}: Deals 2d4 fire damage out to 3 m and 1d4 out to 6 m. Passing through wall deals 2d6 damage +1/level.

	\spellList{Wall of Ice}: Ice plane creates wall with 15 hp +1/level, or hemisphere can trap creatures inside.

\subsubsection{Illusion}
	\spellList{Hallucinatory Terrain}: Makes one type of terrain appear like another (field into forest, or the like).

	\spellList{Illusory Wall}: Wall, floor, or ceiling looks real, but anything can pass through.

	\spellList{Invisibility, Greater}: As \spell{invisibility}, but subject can attack and stay invisible.

	\spellList{Phantasmal Killer}: Fearsome illusion kills subject or deals 3d6 damage.

	\spellList{Rainbow Pattern}: Lights fascinate 24 HD of creatures.

	\spellList{Shadow Conjuration}: Mimics conjuration below 4th level, but only 20\% real.

\subsubsection{Necromancy}
	\spellList{Animate Dead}\textsuperscript{M}: Creates undead skeletons and zombies.

	\spellList{Bestow Curse}: $-6$ to an ability score; $-4$ on attack rolls, saves, and checks; or 50\% chance of losing each action.

	\spellList{Claws of the Tembo}: Deals 1d6 Str damage and transfers hp. %%

	\spellList{Contagion}: Infects subject with chosen disease.

	\spellList{Enervation}: Subject gains 1d4 negative levels.

	\spellList{Fear}: Subjects within cone flee for 1 round/level.

	\spellList{Ghostfire}: Kills 2 HD/level of creatures (max 40 HD). %%

	\spellList{Touch the Black}\textsuperscript{M}: Cold deals 1d6 damage per level, 12-m radius. %%

\subsubsection{Transmutation}
	\spellList{Enlarge Person, Mass}: Enlarges several creatures.

	\spellList{Mnemonic Enhancer}\textsuperscript{F}: Wizard only. Prepares extra spells or retains one just cast.

	\spellList{Polymorph}: Gives one willing subject a new form.

	\spellList{Reduce Person, Mass}: Reduces several creatures.

	\spellList{Stone Shape}: Sculpts stone into any shape.



\subsection{5th-Level Wizard Spells}

\subsubsection{Abjuration}
	\spellList{Break Enchantment}: Frees subjects from enchantments, alterations, curses, and petrification.

	\spellList{Dismissal}: Forces a creature to return to native plane.

	\spellList{Mage's Private Sanctum}: Prevents anyone from viewing or scrying an area for 24 hours.

	\spellList{Psychic Turmoil}: Invisible field leeches psionic power points away.

\subsubsection{Conjuration}
	\spellList{Cerulean Hail}: A hailstorm appears and does 3d8 damage and causes less than 10 HD creatures to flee. %%

	\spellList{Cloudkill}: Kills 3 HD or less; 4$-6$ HD save or die, 6+ HD take Con damage.

	\spellList{Mage's Faithful Hound}: Phantom dog can guard, attack.

	\spellList{Major Creation}: As \spell{minor creation}, plus stone and metal.

	\spellList{Planar Binding, Lesser}: Traps extraplanar creature of 6 HD or less until it performs a task.

	\spellList{Secret Chest}\textsuperscript{F}: Hides expensive chest on Ethereal Plane; you retrieve it at will.

	\spellList{Summon Monster V}: Calls extraplanar creature to fight for you.

	\spellList{Teleport}: Instantly transports you as far as 150 km/level.

	\spellList{Wall of Stone}: Creates a stone wall that can be shaped.

\subsubsection{Divination}
	\spellList{Contact Other Plane}: Lets you ask question of extraplanar entity.

	\spellList{Prying Eyes}: 1d4 +1/level floating eyes scout for you.

	\spellList{Telepathic Bond}: Link lets allies communicate.

\subsubsection{Enchantment}
	\spellList{Dominate Person}: Controls humanoid telepathically.

	\spellList{Feeblemind}: Subject's Int and Cha drop to 1.

	\spellList{Hold Monster}: As \spell{hold person}, but any creature.

	\spellList{Mind Fog}: Subjects in fog get $-10$ to Wis and Will checks.

	\spellList{Scapegoat}\textsuperscript{M}: Put the blame on a nearby target. %%

	\spellList{Symbol of Sleep}\textsuperscript{M}: Triggered rune puts nearby creatures into catatonic slumber.

\subsubsection{Evocation}
	
	\spellList{Cone of Cold}: 1d6/level cold damage.
	
	\spellList{Interposing Hand}: Hand provides cover against one opponent.
	
	\spellList{Quietstorm}: Ranged touch attack deals 8d6 sonic damage. %%
	
	\spellList{Ragestorm}\textsuperscript{M}: Storm rains hail, winds and lightning. %%
	
	\spellList{Sending}: Delivers short message anywhere, instantly.

	\spellList{Skyfire}: Three exploding spheres each deal 1d6 bludgeoning damage and 3d6 fire damage. %%
	
	\spellList{Sparkrain}: Bolt dispels magical effects. %%
	
	\spellList{Wall of Force}: Wall is immune to damage.

\subsubsection{Illusion}
	\spellList{Dream}: Sends message to anyone sleeping.

	\spellList{False Vision}\textsuperscript{M}: Fools scrying with an illusion.

	\spellList{Mirage Arcana}: As \spell{hallucinatory terrain}, plus structures.

	\spellList{Nightmare}: Sends vision dealing 1d10 damage, fatigue.

	\spellList{Persistent Image}: As \spell{major image}, but no concentration required.

	\spellList{Seeming}: Changes appearance of one person per two levels.

	\spellList{Shadow Evocation}: Mimics evocation below 5th level, but only 20\% real.

\subsubsection{Necromancy}
	\spellList{Blight}: Withers one plant or deals 1d6/level damage to plant creature.

	\spellList{Magic Jar}\textsuperscript{F}: Enables possession of another creature.

	\spellList{Symbol of Pain}\textsuperscript{M}: Triggered rune wracks nearby creatures with pain.

	\spellList{Waves of Fatigue}: Several targets become fatigued.

\subsubsection{Transmutation}
	\spellList{Animal Growth}: One animal/two levels doubles in size.
	
	\spellList{Baleful Polymorph}: Transforms subject into harmless animal.
	
	\spellList{Fabricate}: Transforms raw materials into finished items.
	
	\spellList{Overland Flight}: You fly at a speed of 12 m and can hustle over long distances.
	
	\spellList{Passwall}: Creates passage through wood or stone wall.
	
	\spellList{Rangeblade}: Can strike with melee weapons at a distance. %%
	
	\spellList{Sand Trap}\textsuperscript{M}: You entrap an area of sand. %%
	
	\spellList{Sandflow}: You shift the location of sand dunes. %%
	
	\spellList{Telekinesis}: Moves object, attacks creature, or hurls object or creature.
	
	\spellList{Transmute Mud to Rock}: Transforms two 3-m cubes per level.
	
	\spellList{Transmute Rock to Mud}: Transforms two 3-m cubes per level.

\subsubsection{Universal}
	\spellList{Permanency}\textsuperscript{X}: Makes certain spells permanent.



\subsection{6th-Level Wizard Spells}

\subsubsection{Abjuration}
	\spellList{Antimagic Field}: Negates magic within 3 m.

	\spellList{Banish Tyr-Storm}\textsuperscript{F}: Repel a Tyr-storm. %%

	\spellList{Dispel Magic, Greater}: As \spell{dispel magic}, but +20 on check.

	\spellList{Globe of Invulnerability}: As \spell{lesser globe of invulnerability}, plus 4th-level spell effects.

	\spellList{Guards and Wards}: Array of magic effects protect area.

	\spellList{Repulsion}: Creatures can't approach you.

\subsubsection{Conjuration}
	\spellList{Acid Fog}: Fog deals acid damage.

	\spellList{Planar Binding}: As \spell{lesser planar binding}, but up to 12 HD.

	\spellList{Summon Monster VI}: Calls extraplanar creature to fight for you.

	\spellList{Summon Tyr-Storm}\textsuperscript{F}: Call a powerful Tyr-storm. %%

	\spellList{Wall of Iron}\textsuperscript{M}: 30 hp/four levels; can topple onto foes.

\subsubsection{Divination}
	\spellList{Analyze Dweomer}\textsuperscript{F}: Reveals magical aspects of subject.

	\spellList{Legend Lore}\textsuperscript{MF}: Lets you learn tales about a person, place, or thing.

	\spellList{Probe Thoughts}: Read subject's memories, one question/round.

	\spellList{True Seeing}\textsuperscript{M}: Lets you see all things as they really are.

\subsubsection{Enchantment}
	\spellList{Geas/Quest}: As \spell{lesser geas}, plus it affects any creature.

	\spellList{Heroism, Greater}: Gives +4 bonus on attack rolls, saves, skill checks; immunity to fear; temporary hp.

	\spellList{Suggestion, Mass}: As \spell{suggestion}, plus one subject/level.

	\spellList{Symbol of Persuasion}\textsuperscript{M}: Triggered rune charms nearby creatures.

\subsubsection{Evocation}
	\spellList{Chain Lightning}: 1d6/level damage; 1 secondary bolt/level each deals half damage.

	\spellList{Cleansing Flame}: 1d6/level fire damage (max 10d6). %%

	\spellList{Contingency}\textsuperscript{F}: Sets trigger condition for another spell.

	\spellList{Forceful Hand}: Hand pushes creatures away.

	\spellList{Freezing Sphere}: Freezes water or deals cold damage.

	\spellList{Groundflame}: Mist deals acid damage. %%

\subsubsection{Illusion}
	\spellList{Mislead}: Turns you invisible and creates illusory double.

	\spellList{Permanent Image}: Includes sight, sound, and smell.

	\spellList{Programmed Image}\textsuperscript{M}: As \spell{major image}, plus triggered by event.

	\spellList{Shadow Walk}: Step into shadow to travel rapidly.

	\spellList{Veil}: Changes appearance of group of creatures.

\subsubsection{Necromancy}
	\spellList{Circle of Death}\textsuperscript{M}: Kills 1d4/level HD of creatures.

	\spellList{Create Undead}\textsuperscript{M}: Creates ghouls, ghasts, mummies, or mohrgs.

	\spellList{Eyebite}: Target becomes panicked, sickened, and comatose.

	\spellList{Shroud of Darkness}\textsuperscript{M}: Imbue a cloak with protective qualities. %%

	\spellList{Symbol of Fear}\textsuperscript{M}: Triggered rune panics nearby creatures.

	\spellList{Undeath to Death}\textsuperscript{M}: Destroys 1d4/level HD of undead (max 20d4).

\subsubsection{Transmutation}
	% \spellList{Control Water}: Raises or lowers bodies of water.

	\spellList{Bear's Endurance, Mass}: As \spell{bear's endurance}, affects one subject/level.

	\spellList{Braxatskin}: Your skin hardens, granting armor bonus and damage reduction. %%

	\spellList{Bull's Strength, Mass}: As \spell{bull's strength}, affects one subject/ level.

	\spellList{Cat's Grace, Mass}: As \spell{cat's grace}, affects one subject/level.

	\spellList{Control Tides}: Raises, lowers, or parts bodies of water or silt. %%

	\spellList{Disintegrate}: Makes one creature or object vanish.

	\spellList{Eagle's Splendor, Mass}: As \spell{eagle's splendor}, affects one subject/level.

	\spellList{Flesh to Stone}: Turns subject creature into statue.

	\spellList{Fox's Cunning, Mass}: As \spell{fox's cunning}, affects one subject/ level.

	\spellList{Hardening}: Permanently increase by 1/two levels the hardness of an item.

	\spellList{Mage's Lucubration}: Wizard only. Recalls spell of 5th level or lower.

	\spellList{Mental Pinnacle}\textsuperscript{M}: You gain the mental powers of a psion.

	\spellList{Move Earth}: Digs trenches and build hills.

	\spellList{Owl's Wisdom, Mass}: As \spell{owl's wisdom}, affects one subject/ level.

	\spellList{Sands of Time}\textsuperscript{F}: Reverses or accelerates aging of a nonliving object. %%

	\spellList{Shining Sands}: Grains of sand rotate to reflect light where you wish it to go. %%

	\spellList{Stone to Flesh}: Restores petrified creature.

	\spellList{Transformation}\textsuperscript{M}: You gain combat bonuses.



\subsection{7th-Level Wizard Spells}

\subsubsection{Abjuration}
	\spellList{Banishment}: Banishes 2 HD/level of extraplanar creatures.

	\spellList{Psychic Turmoil, Greater}: As \spell{psychic turmoil}, but you gain power points as temporary hit points.

	\spellList{Sequester}: Subject is invisible to sight and scrying; renders creature comatose.

	\spellList{Spell Turning}: Reflect 1d4+6 spell levels back at caster.

\subsubsection{Conjuration}
	\spellList{Infestation}: Tiny parasites infest creatures within area. %%

	\spellList{Instant Summons}\textsuperscript{M}: Prepared object appears in your hand.

	\spellList{Mage's Magnificent Mansion}\textsuperscript{F}: Door leads to extradimensional mansion.

	\spellList{Phase Door}: Creates an invisible passage through wood or stone.

	\spellList{Plane Shift}\textsuperscript{F}: As many as eight subjects travel to another plane.

	\spellList{Summon Monster VII}: Calls extraplanar creature to fight for you.

	\spellList{Teleport, Greater}: As \spell{teleport}, but no range limit and no off-target arrival.

	\spellList{Teleport Object}: As \spell{teleport}, but affects a touched object.

\subsubsection{Divination}
	\spellList{Arcane Sight, Greater}: As \spell{arcane sight}, but also reveals magic effects on creatures and objects.

	\spellList{Scrying, Greater}: As \spell{scrying}, but faster and longer.

	\spellList{Vision}\textsuperscript{MX}: As \spell{legend lore}, but quicker and strenuous.

\subsubsection{Enchantment}
	\spellList{Hold Person, Mass}: As \spell{hold person}, but all within 9 m.

	\spellList{Insanity}: Subject suffers continuous confusion.

	\spellList{Power Word Blind}: Blinds creature with 200 hp or less.

	\spellList{Symbol of Stunning}\textsuperscript{M}: Triggered rune stuns nearby creatures.

\subsubsection{Evocation}
	\spellList{Delayed Blast Fireball}: 1d6/level fire damage; you can postpone blast for 5 rounds.

	\spellList{Forcecage}\textsuperscript{M}: Cube or cage of force imprisons all inside.

	\spellList{Grasping Hand}: Hand provides cover, pushes, or grapples.

	\spellList{Mage's Sword}\textsuperscript{F}: Floating magic blade strikes opponents.

	\spellList{Prismatic Spray}: Rays hit subjects with variety of effects.

\subsubsection{Illusion}
	\spellList{Invisibility, Mass}: As \spell{invisibility}, but affects all in range.

	\spellList{Project Image}: Illusory double can talk and cast spells.

	\spellList{Shadow Conjuration, Greater}: As \spell{shadow conjuration}, but up to 6th level and 60\% real.

	\spellList{Simulacrum}\textsuperscript{MX}: Creates partially real double of a creature.

\subsubsection{Necromancy}
	\spellList{Control Undead}: Undead don't attack you while under your command.
	
	\spellList{Finger of Death}: Kills one subject.
	
	\spellList{Gray Beckoning}: Summons zombies from the Gray. %%

	\spellList{Symbol of Weakness}\textsuperscript{M}: Triggered rune weakens nearby creatures.

	\spellList{Unliving Identity}\textsuperscript{MX}: Transform a zombie into a thinking 
	zombie. %%
	
	\spellList{Waves of Exhaustion}: Several targets become exhausted.

\subsubsection{Transmutation}
	\spellList{Control Weather}: Changes weather in local area.

	\spellList{Ethereal Jaunt}: You become ethereal for 1 round/level.

	\spellList{Reverse Gravity}: Objects and creatures fall upward.

	\spellList{Statue}: Subject can become a statue at will.

\subsubsection{Universal}
	\spellList{Limited Wish}\textsuperscript{X}: Alters reality---within spell limits.



\subsection{8th-Level Wizard Spells}

\subsubsection{Abjuration}
	\spellList{Dimensional Lock}: Teleportation and interplanar travel blocked for one day/level.

	\spellList{Mind Blank}: Subject is immune to mental/emotional magic and scrying.

	\spellList{Prismatic Wall}: Wall's colors have array of effects.

	\spellList{Protection from Spells}\textsuperscript{MF}: Confers +8 resistance bonus.

	\spellList{Protection from Time}\textsuperscript{MX}: Makes creature or object immune to aging for 1 month/2 levels. %%

\subsubsection{Conjuration}
	\spellList{Incendiary Cloud}: Cloud deals 4d6 fire damage/round.

	\spellList{Maze}: Traps subject in extradimensional maze.

	\spellList{Planar Binding, Greater}: As \spell{lesser planar binding}, but up to 18 HD.

	\spellList{Summon Monster VIII}: Calls extraplanar creature to fight for you.

	\spellList{Trap the Soul}\textsuperscript{MF}: Imprisons subject within gem.

\subsubsection{Divination}
	\spellList{Discern Location}: Reveals exact location of creature or object.

	\spellList{Moment of Prescience}: You gain insight bonus on single attack roll, check, or save.

	\spellList{Prying Eyes, Greater}: As \spell{prying eyes}, but eyes have true seeing.

\subsubsection{Enchantment}
	\spellList{Antipathy}: Object or location affected by spell repels certain creatures.

	\spellList{Binding}\textsuperscript{M}: Utilizes an array of techniques to imprison a creature.

	\spellList{Charm Monster, Mass}: As \spell{charm monster}, but all within 9 m.

	\spellList{Demand}: As \spell{sending}, plus you can send suggestion.

	\spellList{Irresistible Dance}: Forces subject to dance.

	\spellList{Maddening Scream}: Subject has $-4$ AC, no shield, Reflex save on 20 only.

	\spellList{Power Word Stun}: Stuns creature with 150 hp or less.

	\spellList{Symbol of Insanity}\textsuperscript{M}: Triggered rune renders nearby creatures insane.

	\spellList{Sympathy}\textsuperscript{F}: Object or location attracts certain creatures.

\subsubsection{Evocation}
	\spellList{Clenched Fist}: Large hand provides cover, pushes, or attacks your foes.

	\spellList{Polar Ray}: Ranged touch attack deals 1d6/level cold damage.

	\spellList{Shout, Greater}: Devastating yell deals 10d6 sonic damage; stuns creatures, damages objects.

	\spellList{Sunburst}: Blinds all within 3 m, deals 6d6 damage.

	\spellList{Telekinetic Sphere}: As \spell{resilient sphere}, but you move sphere telekinetically.

\subsubsection{Illusion}
	\spellList{Scintillating Pattern}: Twisting colors confuse, stun, or render unconscious.

	\spellList{Screen}: Illusion hides area from vision, scrying.

	\spellList{Shadow Evocation, Greater}: As \spell{shadow evocation}, but up to 7th level and 60\% real.

\subsubsection{Necromancy}
	\spellList{Clone}\textsuperscript{MF}: Duplicate awakens when original dies.

	\spellList{Create Greater Undead}\textsuperscript{M}: Create shadows, wraiths, spectres, or devourers.

	\spellList{Horrid Wilting}: Deals 1d6/level damage within 9 m.

	\spellList{Symbol of Death}\textsuperscript{M}: Triggered rune slays nearby creatures.

\subsubsection{Transmutation}
	\spellList{Iron Body}: Your body becomes living iron.

	\spellList{Polymorph Any Object}: Changes any subject into anything else.

	\spellList{Temporal Stasis}\textsuperscript{M}: Puts subject into suspended animation.



\subsection{9th-Level Wizard Spells}

\subsubsection{Abjuration}
	\spellList{Freedom}: Releases creature from imprisonment.

	\spellList{Imprisonment}: Entombs subject beneath the earth.

	\spellList{Mage's Disjunction}: Dispels magic, disenchants magic items.

	\spellList{Prismatic Sphere}: As \spell{prismatic wall}, but surrounds on all sides.

\subsubsection{Conjuration}
	\spellList{Gate}\textsuperscript{X}: Connects two planes for travel or summoning.

	\spellList{Gray Rift}: A hovering rift to the Gray bolsters undead. %%

	\spellList{Refuge}\textsuperscript{M}: Alters item to transport its possessor to you.

	\spellList{Summon Monster IX}: Calls extraplanar creature to fight for you.

	\spellList{Teleportation Circle}\textsuperscript{M}: Circle teleports any creature inside to designated spot.

\subsubsection{Divination}
	\spellList{Foresight}: ``Sixth sense'' warns of impending danger.

\subsubsection{Enchantment}
	\spellList{Dominate Monster}: As \spell{dominate person}, but any creature.

	\spellList{Hold Monster, Mass}: As \spell{hold monster}, but all within 9 m.

	\spellList{Power Word Kill}: Kills one creature with 100 hp or less.

\subsubsection{Evocation}
	\spellList{Crushing Hand}: Large hand provides cover, pushes, or crushes your foes.

	\spellList{Meteor Swarm}: Four exploding spheres each deal 6d6 fire damage.

	\spellList{Tempest}\textsuperscript{M}: Create an obliterating storm. %%

\subsubsection{Illusion}
	\spellList{Shades}: As \spell{shadow conjuration}, but up to 8th level and 80\% real.

	\spellList{Weird}: As \spell{phantasmal killer}, but affects all within 9 m.

\subsubsection{Necromancy}
	\spellList{Astral Projection}\textsuperscript{M}: Projects you and companions onto Astral Plane.

	\spellList{Energy Drain}: Subject gains 2d4 negative levels.

	\spellList{Pact of Darkness}\textsuperscript{M}: Exchange services with a shadow giant. %%

	\spellList{Soul Bind}\textsuperscript{F}: Traps newly dead soul to prevent resurrection.

	\spellList{Vampiric Youthfulness}: Increases your lifespan at the expense of another's. %%

	\spellList{Wail of the Banshee}: Kills one creature/level.

\subsubsection{Transmutation}
	\spellList{Etherealness}: Travel to Ethereal Plane with companions.

	\spellList{Magma Tunnel}: Tunnels through solid rock. %%

	\spellList{Shapechange}\textsuperscript{F}: Transforms you into any creature, and change forms once per round.

	\spellList{Time Stop}: You act freely for 1d4+1 rounds.

\subsubsection{Universal}
	\spellList{Wish}\textsuperscript{X}: As \spell{limited wish}, but with fewer limits.
\section{Spell Descriptions}
The spells herein are presented in alphabetical order (with the exception of those whose names begin with ``greater,'' ``lesser,'' or ``mass''; see Order of Presentation).

\input{subsections/spells/acid-arrow.tex}
\Spell{Acid Fog}{acid fog}
{Conjuration (Creation) [Acid]}
{
	\textbf{Level:}
	Wiz 6\\
	\textbf{Components:}
	V, S, M/DF\\
	\textbf{Casting Time:}
	1 standard action\\
	\textbf{Range:}
	Medium (30 m + 3 m/level)\\
	\textbf{Effect:}
	Fog spreads in 6-m radius, 6 m high\\
	\textbf{Duration:}
	1 round/level\\
	\textbf{Saving Throw:}
	None\\
	\textbf{Spell Resistance:}
	No\\
}
{
Acid fog creates a billowing mass of misty vapors similar to that produced by a solid fog spell. In addition to slowing creatures down and obscuring sight, this spell's vapors are highly acidic. Each round on your turn, starting when you cast the spell, the fog deals 2d6 points of acid damage to each creature and object within it.

	\textit{Arcane Material Component}:
	A pinch of dried, powdered peas combined with powdered animal hoof.

}

\Spell{Acid Splash}{acid splash}
{Conjuration (Creation) [Acid]}
{
	\textbf{Level:}
	Wiz 0\\
	\textbf{Components:}
	V, S\\
	\textbf{Casting Time:}
	1 standard action\\
	\textbf{Range:}
	Close (7.5 m + 1.5 m/2 levels)\\
	\textbf{Effect:}
	One missile of acid\\
	\textbf{Duration:}
	Instantaneous\\
	\textbf{Saving Throw:}
	None\\
	\textbf{Spell Resistance:}
	No\\
}
{
	You fire a small orb of acid at the target. You must succeed on a ranged touch attack to hit your target. The orb deals 1d3 points of acid damage.

}

\input{subsections/spells/aid.tex}
\Spell{Air Walk}{air walk}
{Transmutation [Air]}
{
	\textbf{Level:}
	Clr 4, Drd 4, Freedom 2\\
	\textbf{Components:}
	V, S, DF\\
	\textbf{Casting Time:}
	1 standard action\\
	\textbf{Range:}
	Touch\\
	\textbf{Target:}
	Creature (Gargantuan or smaller) touched\\
	\textbf{Duration:}
	10 min./level\\
	\textbf{Saving Throw:}
	None\\
	\textbf{Spell Resistance:}
	Yes (harmless)\\
}
{
	The subject can tread on air as if walking on solid ground. Moving upward is similar to walking up a hill. The maximum upward or downward angle possible is 45 degrees, at a rate equal to one-half the air walker's normal speed.

	A strong wind (31+ km/h) can push the subject along or hold it back. At the end of its turn each round, the wind blows the air walker 1.5 meter for each 1.5 kilometers per hour of wind speed. The creature may be subject to additional penalties in exceptionally strong or turbulent winds, such as loss of control over movement or physical damage from being buffeted about.

	Should the spell duration expire while the subject is still aloft, the magic fails slowly. The subject floats downward 18 meters per round for 1d6 rounds. If it reaches the ground in that amount of time, it lands safely. If not, it falls the rest of the distance, taking 1d6 points of damage per 3 meters of fall. Since dispelling a spell effectively ends it, the subject also descends in this way if the air walk spell is dispelled, but not if it is negated by an antimagic field.

	You can cast air walk on a specially trained mount so it can be ridden through the air. You can train a mount to move with the aid of air walk (counts as a trick; see \spell{Handle Animal} skill) with one week of work and a DC 25 \spell{Handle Animal} check.

}

\Spell{Alarm}{alarm}
{Abjuration}
{
	\textbf{Level:}
	Rgr 1, Wiz 1\\
	\textbf{Components:}
	V, S, F/DF\\
	\textbf{Casting Time:}
	1 standard action\\
	\textbf{Range:}
	Close (7.5 m + 1.5 m/2 levels)\\
	\textbf{Area:}
	6-m-radius emanation centered on a point in space\\
	\textbf{Duration:}
	2 hours/level (D)\\
	\textbf{Saving Throw:}
	None\\
	\textbf{Spell Resistance:}
	No\\
}
{
	Alarm sounds a mental or audible alarm each time a creature of Tiny or larger size enters the warded area or touches it. A creature that speaks the password (determined by you at the time of casting) does not set off the alarm. You decide at the time of casting whether the alarm will be mental or audible.

	\textit{Mental Alarm}:
	A mental alarm alerts you (and only you) so long as you remain within 1.5 kilometer of the warded area. You note a single mental ``ping'' that awakens you from normal sleep but does not otherwise disturb concentration. A silence spell has no effect on a mental alarm.

	\textit{Audible Alarm}:
	An audible alarm produces the sound of a hand bell, and anyone within 18 meters of the warded area can hear it clearly. Reduce the distance by 3 meters for each interposing closed door and by 6 meters for each substantial interposing wall.

	In quiet conditions, the ringing can be heard faintly as far as 54 meters away. The sound lasts for 1 round. Creatures within a silence spell cannot hear the ringing.

	Ethereal or astral creatures do not trigger the alarm.

	\emph{Alarm} can be made permanent with a \spell{permanency} spell.

	\textit{Arcane Focus}:
	A tiny bell and a piece of very fine silver wire.

}

\input{subsections/spells/align-weapon.tex}
\input{subsections/spells/alter-self.tex}
\input{subsections/spells/analyze-dweomer.tex}
\input{subsections/spells/animal-growth.tex}
\input{subsections/spells/animal-messenger.tex}
\input{subsections/spells/animal-shapes.tex}
\input{subsections/spells/animal-trance.tex}
\input{subsections/spells/animate-dead.tex}
\input{subsections/spells/animate-objects.tex}
\input{subsections/spells/animate-plants.tex}
\Spell{Animate Rope}{animate rope}
{Transmutation}
{
	\textbf{Level:}
	Wiz 1\\
	\textbf{Components:}
	V, S\\
	\textbf{Casting Time:}
	1 standard action\\
	\textbf{Range:}
	Medium (30 m + 3 m/level)\\
	\textbf{Target:}
	One ropelike object, length up to 15 m + 1.5 m/level; see text\\
	\textbf{Duration:}
	1 round/level\\
	\textbf{Saving Throw:}
	None\\
	\textbf{Spell Resistance:}
	No\\
}
{
	You can animate a nonliving ropelike object. The maximum length assumes a rope with a 2-centimeter diameter.

	Reduce the maximum length by 50\% for every additional inch of thickness, and increase it by 50\% for each reduction of the rope's diameter by half.

	The possible commands are ``coil'' (form a neat, coiled stack), ``coil and knot,'' ``loop,'' ``loop and knot,'' ``tie and knot,'' and the opposites of all of the above (``uncoil,'' and so forth). You can give one command each round as a move action, as if directing an active spell.

	The rope can enwrap only a creature or an object within 30 centimeters of it---it does not snake outward---so it must be thrown near the intended target. Doing so requires a successful ranged touch attack roll (range increment 3 meters). A typical 2-centimeter-diameter hempen rope has 2 hit points, AC 10, and requires a DC 23 Strength check to burst it. The rope does not deal damage, but it can be used as a trip line or to cause a single opponent that fails a Reflex saving throw to become entangled. A creature capable of spellcasting that is bound by this spell must make a DC 15 \skill{Concentration} check to cast a spell. An entangled creature can slip free with a DC 20 \skill{Escape Artist} check.

	The rope itself and any knots tied in it are not magical.

	This spell grants a +2 bonus on any \skill{Use Rope} checks you make when using the transmuted rope.

	The spell cannot animate objects carried or worn by a creature.

}

\input{subsections/spells/antilife-shell.tex}
\input{subsections/spells/antimagic-field.tex}
\input{subsections/spells/antipathy.tex}
\Spell{Antiplant Shell}{antiplant shell}
{Abjuration}
{
	\textbf{Level:}
	Drd 4\\
	\textbf{Components:}
	V, S, DF\\
	\textbf{Casting Time:}
	1 standard action\\
	\textbf{Range:}
	3 m\\
	\textbf{Area:}
	3-m-radius emanation, centered on you\\
	\textbf{Duration:}
	10 min./level (D)\\
	\textbf{Saving Throw:}
	None\\
	\textbf{Spell Resistance:}
	Yes\\
}
{
	The \emph{antiplant shell} spell creates an invisible, mobile barrier that keeps all creatures within the shell protected from attacks by plant creatures or animated plants. As with many abjuration spells, forcing the barrier against creatures that the spell keeps at bay strains and collapses the field.

}

\Spell{Arcane Eye}{arcane eye}
{Divination (Scrying)}
{
	\textbf{Level:}
	Wiz 4\\
	\textbf{Components:}
	V, S, M\\
	\textbf{Casting Time:}
	10 minutes\\
	\textbf{Range:}
	Unlimited\\
	\textbf{Effect:}
	Magical sensor\\
	\textbf{Duration:}
	1 min./level (D)\\
	\textbf{Saving Throw:}
	None\\
	\textbf{Spell Resistance:}
	No\\
}
{
	You create an invisible magical sensor that sends you visual information. You can create the arcane eye at any point you can see, but it can then travel outside your line of sight without hindrance. An arcane eye travels at 9 meters per round (90 meters per minute) if viewing an area ahead as a human would (primarily looking at the floor) or 3 meters per round (30 meters per minute) if examining the ceiling and walls as well as the floor ahead. It sees exactly as you would see if you were there.

	The eye can travel in any direction as long as the spell lasts. Solid barriers block its passage, but it can pass through a hole or space as small as 1 inch in diameter. The eye can't enter another plane of existence, even through a gate or similar magical portal.

	You must concentrate to use an arcane eye. If you do not concentrate, the eye is inert until you again concentrate.

	\textit{Material Component}:
	A bit of bat fur.

}

\Spell{Arcane Lock}{arcane lock}
{Abjuration}
{
	\textbf{Level:}
	Wiz 2\\
	\textbf{Components:}
	V, S, M\\
	\textbf{Casting Time:}
	1 standard action\\
	\textbf{Range:}
	Touch\\
	\textbf{Target:}
	The door, chest, or portal touched, up to 9 sq. m/level in size\\
	\textbf{Duration:}
	Permanent\\
	\textbf{Saving Throw:}
	None\\
	\textbf{Spell Resistance:}
	No\\
}
{
	An \emph{arcane lock} spell cast upon a door, chest, or portal magically locks it. You can freely pass your own \emph{arcane lock} without affecting it; otherwise, a door or object secured with this spell can be opened only by breaking in or with a successful \spell{dispel magic} or \spell{knock} spell. Add 10 to the normal DC to break open a door or portal affected by this spell. (A \spell{knock} spell does not remove an \emph{arcane lock}; it only suppresses the effect for 10 minutes.)

	\textit{Material Component:}
	Gold dust worth 25 gp.

}

\Spell{Arcane Mark}{arcane mark}
{Universal [Ritual]}
{
	\textbf{Level:}
	Wiz 0\\
	\textbf{Components:}
	V, S\\
	\textbf{Casting Time:}
	1 standard action\\
	\textbf{Range:}
	0 m\\
	\textbf{Effect:}
	One personal rune or mark, all of which must fit within 0.3 sq. m\\
	\textbf{Duration:}
	Permanent\\
	\textbf{Saving Throw:}
	None\\
	\textbf{Spell Resistance:}
	No\\
}
{
	This spell allows you to inscribe your personal rune or mark, which can consist of no more than six characters. The writing can be visible or invisible. An \emph{arcane mark} spell enables you to etch the rune upon any substance without harm to the material upon which it is placed. If an invisible mark is made, a detect magic spell causes it to glow and be visible, though not necessarily understandable.

	The spells \spell{see invisibility} and \spell{true seeing}, or the items gem of seeing and robe of eyes likewise allow the user to see an invisible \emph{arcane mark}. A \spell{read magic} spell reveals the words, if any. The mark cannot be dispelled, but it can be removed by the caster or by an erase spell.

	If an \emph{arcane mark} is placed on a living being, normal wear gradually causes the effect to fade in about a month.

	\emph{Arcane mark} must be cast on an object prior to casting instant summons on the same object (see that spell description for details).

}

\Spell{Arcane Sight, Greater}{greater arcane sight}
{Divination}
{
	\textbf{Level:}
	Wiz 7\\
}
{
	This spell functions like \spell{arcane sight}, except that you automatically know which spells or magical effects are active upon any individual or object you see.

	Greater arcane sight doesn't let you identify magic items.

	Unlike \spell{arcane sight}, this spell cannot be made permanent with a \spell{permanency} spell.

}

\input{subsections/spells/arcane-sight.tex}
\input{subsections/spells/astral-projection.tex}
\input{subsections/spells/atonement.tex}
\Spell{Augury}{augury}
{Divination}
{
	\textbf{Level:}
	Clr 2\\
	\textbf{Components:}
	V, S, M, F\\
	\textbf{Casting Time:}
	1 minute\\
	\textbf{Range:}
	Personal\\
	\textbf{Target:}
	You\\
	\textbf{Duration:}
	Instantaneous\\
}
{
	An augury can tell you whether a particular action will bring good or bad results for you in the immediate future.

	\textbf{	The base chance for receiving a meaningful reply is 70\% + 1\% per caster level, to a maximum of 90\%; this roll is made secretly. A question may be so straightforward that a successful result is automatic, or so vague as to have no chance of success. If the augury succeeds, you get one of four results:}


Weal (if the action will probably bring good results).
Woe (for bad results).
Weal and woe (for both).
Nothing (for actions that don't have especially good or bad results).

	If the spell fails, you get the ``nothing'' result. A cleric who gets the ``nothing'' result has no way to tell whether it was the consequence of a failed or successful augury.

	The augury can see into the future only about half an hour, so anything that might happen after that does not affect the result. Thus, the result might not take into account the long-term consequences of a contemplated action. All auguries cast by the same person about the same topic use the same dice result as the first casting.

	\textit{Material Component}:
	Incense worth at least 25 gp.

	\textit{Focus}:
	A set of marked sticks, bones, or similar tokens of at least 25 gp value.

}

\Spell{Awaken}{awaken}
{Transmutation}
{
	\textbf{Level:}
	Drd 5\\
	\textbf{Components:}
	V, S, DF, XP\\
	\textbf{Casting Time:}
	24 hours\\
	\textbf{Range:}
	Touch\\
	\textbf{Target:}
	Animal or tree touched\\
	\textbf{Duration:}
	Instantaneous\\
	\textbf{Saving Throw:}
	Will negates\\
	\textbf{Spell Resistance:}
	Yes\\
}
{
	You awaken a tree or animal to humanlike sentience. To succeed, you must make a Will save (DC 10 + the animal's current HD, or the HD the tree will have once \emph{awakened}).

	The \emph{awakened} animal or tree is friendly toward you. You have no special empathy or connection with a creature you awaken, although it serves you in specific tasks or endeavors if you communicate your desires to it.

	An \emph{awakened} tree has characteristics as if it were an animated object, except that it gains the plant type and its Intelligence, Wisdom, and Charisma scores are each 3d6. An \emph{awakened} plant gains the ability to move its limbs, roots, vines, creepers, and so forth, and it has senses similar to a human's.

	An \emph{awakened} animal gets 3d6 Intelligence, +1d3 Charisma, and +2 HD. Its type becomes magical beast (augmented animal). An \emph{awakened} animal can't serve as an animal companion, familiar, or special mount.

	An \emph{awakened} tree or animal can speak one language that you know, plus one additional language that you know per point of Intelligence bonus (if any).

	\textit{XP Cost}:
	250 XP.

}

\input{subsections/spells/baleful-polymorph.tex}
\Spell{Bane}{bane}
{Enchantment (Compulsion) [Fear, Mind-Affecting]}
{
	\textbf{Level:}
	Clr 1, ElC 1\\
	\textbf{Components:}
	V, S, DF\\
	\textbf{Casting Time:}
	1 standard action\\
	\textbf{Range:}
	15 m\\
	\textbf{Area:}
	All enemies within 15 m\\
	\textbf{Duration:}
	1 min./level\\
	\textbf{Saving Throw:}
	Will negates\\
	\textbf{Spell Resistance:}
	Yes\\
}
{
	\emph{Bane} fills your enemies with fear and doubt. Each affected creature takes a $-1$ penalty on attack rolls and a $-1$ penalty on saving throws against fear effects.

	\emph{Bane} counters and dispels \spell{bless}.

}

\input{subsections/spells/banishment.tex}
\input{subsections/spells/barkskin.tex}
\Spell{Bear's Endurance, Mass}{mass bear's endurance}
{Transmutation}
{
	\textbf{Level:}
	Clr 6, Drd 6, Wiz 6\\
	\textbf{Range:}
	Close (7.5 m + 1.5 m/2 levels)\\
	\textbf{Targets:}
	One creature/level, no two of which can be more than 9 m apart\\
}
{
	\emph{Mass bear's endurance} works like \spell{bear's endurance}, except that it affects multiple creatures.

}

\input{subsections/spells/bears-endurance.tex}
\Spell{Bestow Curse}{bestow curse}
{Necromancy}
{
	\textbf{Level:}
	Clr 3, Wiz 4\\
	\textbf{Components:}
	V, S\\
	\textbf{Casting Time:}
	1 standard action\\
	\textbf{Range:}
	Touch\\
	\textbf{Target:}
	Creature touched\\
	\textbf{Duration:}
	Permanent\\
	\textbf{Saving Throw:}
	Will negates\\
	\textbf{Spell Resistance:}
	Yes\\
}
{
	You place a curse on the subject. Choose one of the following three effects.
\begin{itemize*}
\item $-6$ decrease to an ability score (minimum 1).
\item $-4$ penalty on attack rolls, saves, ability checks, and skill checks.
\item Each turn, the target has a 50\% chance to act normally; otherwise, it takes no action.
\end{itemize*}

	You may also invent your own curse, but it should be no more powerful than those described above.

	The curse bestowed by this spell cannot be dispelled, but it can be removed with a break enchantment, limited wish, miracle, remove curse, or wish spell.

	Bestow curse counters remove curse.

}

\input{subsections/spells/binding.tex}
\input{subsections/spells/black-tentacles.tex}
\Spell{Blade Barrier}{blade barrier}
{Evocation [Force]}
{
	\textbf{Level:}
	Clr 6, Good 6, War 6\\
	\textbf{Components:}
	V, S\\
	\textbf{Casting Time:}
	1 standard action\\
	\textbf{Range:}
	Medium (30 m + 3 m/level)\\
	\textbf{Effect:}
	Wall of whirling blades up to 6 m long/ level, or a ringed wall of whirling blades with a radius of up to 1.5 m per two levels; either form 6 m high\\
	\textbf{Duration:}
	1 min./level (D)\\
	\textbf{Saving Throw:}
	Reflex half or Reflex negates; see text\\
	\textbf{Spell Resistance:}
	Yes\\
}
{
	An immobile, vertical curtain of whirling blades shaped of pure force springs into existence. Any creature passing through the wall takes 1d6 points of damage per caster level (maximum 15d6), with a Reflex save for half damage.

	If you evoke the barrier so that it appears where creatures are, each creature takes damage as if passing through the wall. Each such creature can avoid the wall (ending up on the side of its choice) and thus take no damage by making a successful Reflex save.

	A blade barrier provides cover (+4 bonus to AC, +2 bonus on Reflex saves) against attacks made through it.

}

\Spell{Blasphemy}{blasphemy}
{Evocation [Evil, Sonic]}
{
	\textbf{Level:}
	Clr 7\\
	\textbf{Components:}
	V\\
	\textbf{Casting Time:}
	1 standard action\\
	\textbf{Range:}
	12 m\\
	\textbf{Area:}
	Nonevil creatures in a 12-m-radius spread centered on you\\
	\textbf{Duration:}
	Instantaneous\\
	\textbf{Saving Throw:}
	None or Will negates; see text\\
	\textbf{Spell Resistance:}
	Yes\\
}
{
	Any nonevil creature within the area of a \emph{blasphemy} spell suffers the following ill effects.

\Table{}{XX}{
	\tableheader HD & \tableheader Effect\\
	Equal to caster level & Dazed\\
	Up to caster level $-1$ & Weakened, dazed\\
	Up to caster level $-5$ & Paralyzed, weakened, dazed\\
	Up to caster level $-10$ & Killed, paralyzed, weakened, dazed\\
}

	The effects are cumulative and concurrent.

	No saving throw is allowed against these effects.

	\textit{Dazed}:
	The creature can take no actions for 1 round, though it defends itself normally.

	\textit{Weakened}:
	The creature's Strength score decreases by 2d6 points for 2d4 rounds.

	\textit{Paralyzed}:
	The creature is paralyzed and helpless for 1d10 minutes.

	\textit{Killed}:
	Living creatures die. Undead creatures are destroyed.

	Furthermore, if you are on your home plane when you cast this spell, nonevil extraplanar creatures within the area are instantly banished back to their home planes. Creatures so banished cannot return for at least 24 hours. This effect takes place regardless of whether the creatures hear the \emph{blasphemy}. The banishment effect allows a Will save (at a $-4$ penalty) to negate.

	Creatures whose Hit Dice exceed your caster level are unaffected by \emph{blasphemy}.

}

\Spell{Bless}{bless}
{Enchantment (Compulsion) [Mind-Affecting]}
{
	\textbf{Level:}
	Clr 1\\
	\textbf{Components:}
	V, S, DF\\
	\textbf{Casting Time:}
	1 standard action\\
	\textbf{Range:}
	15 m\\
	\textbf{Area:}
	The caster and all allies within a 15-m burst, centered on the caster\\
	\textbf{Duration:}
	1 min./level\\
	\textbf{Saving Throw:}
	None\\
	\textbf{Spell Resistance:}
	Yes (harmless)\\
}
{
	\emph{Bless} fills your allies with courage. Each ally gains a +1 morale bonus on attack rolls and on saving throws against fear effects.

	\emph{Bless} counters and dispels \spell{bane}.

}

\input{subsections/spells/bless-water.tex}
\input{subsections/spells/bless-weapon.tex}
\input{subsections/spells/blight.tex}
\Spell{Blindness/Deafness}{blindness/deafness}
{Necromancy}
{
	\textbf{Level:}
	Clr 3, Wiz 2\\
	\textbf{Components:}
	V\\
	\textbf{Casting Time:}
	1 standard action\\
	\textbf{Range:}
	Medium (30 m + 3 m/level)\\
	\textbf{Target:}
	One living creature\\
	\textbf{Duration:}
	Permanent (D)\\
	\textbf{Saving Throw:}
	Fortitude negates\\
	\textbf{Spell Resistance:}
	Yes\\
}
{
	You call upon the powers of unlife to render the subject blinded or deafened, as you choose.

}

\input{subsections/spells/blink.tex}
\Spell{Blur}{blur}
{Illusion (Glamer)}
{
	\textbf{Level:}
	Brd 2,Wiz 2\\
	\textbf{Components:}
	V\\
	\textbf{Casting Time:}
	1 standard action\\
	\textbf{Range:}
	Touch\\
	\textbf{Target:}
	Creature touched\\
	\textbf{Duration:}
	1 min./level (D)\\
	\textbf{Saving Throw:}
	Will negates (harmless)\\
	\textbf{Spell Resistance:}
	Yes (harmless)\\
}
{
	The subject's outline appears blurred, shifting and wavering. This distortion grants the subject concealment (20\% miss chance).

	A see invisibility spell does not counteract the blur effect, but a true seeing spell does.

	Opponents that cannot see the subject ignore the spell's effect (though fighting an unseen opponent carries penalties of its own).

}

\Spell{Bolt of Glory}{bolt of glory}
{Evocation [Good]}
{
	\textbf{Level}: Glory 6\\
	\textbf{Components}: V, S, D F\\
	\textbf{Casting Time}: 1 standard action\\
	\textbf{Range}: Close (7.5 m + 1.5 m/level)\\
	\textbf{Effect}: Ray\\
	\textbf{Duration}: Instantaneous\\
	\textbf{Saving Throw}: None\\
	\textbf{Spell Resistance}: Yes\\
}
{
	This spell projects a bolt of energy from the Positive Energy Plane against one creature. The caster must succeed at a ranged touch attack to strike the target. A creature struck suffers varying damage, depending on its nature and home plane of existence:\\

	\Table{}{Xcc}{
		\tableheader Creature's Origin/Nature & \tableheader Damage & \tableheader Maximum Value\\
		Material Plane, Elemental Plane, neutral outsider & 1d6/2 levels & 7d6\\
		Negative Energy Plane, evil outsider, undead creature & 1d6/level & 15d6\\
		Positive Energy Plane, good outsider & --  & -- \\
	}
}
\Spell{Bolts of Bedevilment}{bolts of bedevilment}
{Enchantment [Mind-Affecting]}
{
	\textbf{Level}: Madness 5\\
	\textbf{Components}: V, S\\
	\textbf{Casting Time}: 1 standard action\\
	\textbf{Range}: Medium (30 m + 3 m/level)\\
	\textbf{Effect}: Ray\\
	\textbf{Duration}: 1 round/level\\
	\textbf{Saving Throw}: Will negates\\
	\textbf{Spell Resistance}: Yes\\
}
{
	This spell grants the caster the ability to make one ray attack per round. The ray dazes one living creature, clouding its mind so that it takes no action for 1d3 rounds. The creature is not stunned (so attackers get no special advantage against it), but it can’t move, cast spells, use mental abilities, and so on.
}
\Spell{Brain Spider}{brain spider}
{Divination [Mind-Affecting]}
{
	\textbf{Level}: Clr 8, Mind 7\\
	\textbf{Components}: V, S, M, DF\\
	\textbf{Casting Time}: 1 round\\
	\textbf{Range}: Long (120 m + 12 m/level)\\
	\textbf{Targets}: Up to eight living creatures\\
	\textbf{Duration}: 1 min./level\\
	\textbf{Saving Throw}: Will negates\\
	\textbf{Spell Resistance}: Yes\\
}
{
	This spell allows you to eavesdrop as a standard action on the thoughts of up to eight other creatures at once, hearing as desired:
	\begin{itemize*}
		\item Individual trains of thought in whatever order you desire.
		\item Information from all minds about one particular topic, thing, or being, one nugget of information per caster level.
		\item A study of the thoughts and memories of one creature of the group in detail.
	\end{itemize*}

	Once per round, if you do not perform a detailed study of one creature’s mind, you can attempt (as a standard action) to implant a suggestion in the mind of any one of the affected creatures. The creature can make another Will saving throw to resist the suggestion, using the save DC of the brain spider spell. (Creatures with special resistance to enchantment spells can use this resistance to keep from being affected by the suggestion.) Success on this saving throw does not negate the other effects of the brain spider spell for that creature.

	You can affect all intelligent beings of your choice within range (up to the limit of eight), beginning with known or named beings. Language is not a barrier, and you need not personally know the beings The spell cannot reach those who make a successful Will save.

	\textit{Material Component}: A spider of any size or kind. It can be dead, but must still have all eight legs.
}
\Spell{Break Enchantment}{break enchantment}
{Abjuration}
{
	\textbf{Level:}
	Clr 5, Luck 5, Pal 4, Wiz 5\\
	\textbf{Components:}
	V, S\\
	\textbf{Casting Time:}
	1 minute\\
	\textbf{Range:}
	Close (7.5 m + 1.5 m/2 levels)\\
	\textbf{Targets:}
	Up to one creature per level, all within 9 m of each other\\
	\textbf{Duration:}
	Instantaneous\\
	\textbf{Saving Throw:}
	See text\\
	\textbf{Spell Resistance:}
	No\\
}
{
	This spell frees victims from enchantments, transmutations, and curses. Break enchantment can reverse even an instantaneous effect. For each such effect, you make a caster level check (1d20 + caster level, maximum +15) against a DC of 11 + caster level of the effect. Success means that the creature is free of the spell, curse, or effect. For a cursed magic item, the DC is 25.

	If the spell is one that cannot be dispelled by dispel magic, break enchantment works only if that spell is 5th level or lower.

	If the effect comes from some permanent magic item break enchantment does not remove the curse from the item, but it does frees the victim from the item's effects.

}

\Spell{Bull's Strength, Mass}{bull's strength, mass}
{Transmutation}
{
	\textbf{Level:}
	Clr 6, Drd 6, Wiz 6\\
	\textbf{Range:}
	Close (7.5 m + 1.5 m/2 levels)\\
	\textbf{Targets:}
	One creature/level, no two of which can be more than 9 m apart\\
}
{
	This spell functions like bull's strength, except that it affects multiple creatures.

}

\input{subsections/spells/bulls-strength.tex}
\input{subsections/spells/burning-hands.tex}
\Spell{Call Lightning Storm}{call lightning storm}
{Evocation [Electricity]}
{
	\textbf{Level:}
	Drd 5\\
	\textbf{Range:}
	Long (120 m + 12 m/level)\\
}
{
	This spell functions like call lightning, except that each bolt deals 5d6 points of electricity damage (or 5d10 if created outdoors in a stormy area), and you may call a maximum of 15 bolts.

}

\input{subsections/spells/call-lightning.tex}
\input{subsections/spells/calm-animals.tex}
\Spell{Calm Emotions}{calm emotions}
{Enchantment (Compulsion) [Mind-Affecting]}
{
	\textbf{Level:}
	Charm 2, Clr 2, Law 2\\
	\textbf{Components:}
	V, S, DF\\
	\textbf{Casting Time:}
	1 standard action\\
	\textbf{Range:}
	Medium (30 m + 3 m/level)\\
	\textbf{Area:}
	Creatures in a 6-m-radius spread\\
	\textbf{Duration:}
	Concentration, up to 1 round/level (D)\\
	\textbf{Saving Throw:}
	Will negates\\
	\textbf{Spell Resistance:}
	Yes\\
}
{
	This spell calms agitated creatures. You have no control over the affected creatures, but \emph{calm emotions} can stop raging creatures from fighting or joyous ones from reveling. Creatures so affected cannot take violent actions (although they can defend themselves) or do anything destructive. Any aggressive action against or damage dealt to a calmed creature immediately breaks the spell on all calmed creatures.

	This spell automatically suppresses (but does not dispel) any morale bonuses granted by spells such as \spell{bless}, \spell{good hope}, and \spell{rage}, as well as negating a bard's ability to inspire courage or a barbarian's rage ability. It also suppresses any fear effects and removes the confused condition from all targets. While the spell lasts, a suppressed spell or effect has no effect. When the \emph{calm emotions} spell ends, the original spell or effect takes hold of the creature again, provided that its duration has not expired in the meantime.

}

\Spell{Cat's Grace, Mass}{cat's grace, mass}
{Transmutation}
{
	\textbf{Level:}
	Drd 6, Wiz 6\\
	\textbf{Range:}
	Close (7.5 m + 1.5 m/2 levels)\\
	\textbf{Targets:}
	One creature/level, no two of which can be more than 9 m apart\\
}
{
	This spell functions like cat's grace, except that it affects multiple creatures.

}

\input{subsections/spells/cats-grace.tex}
\input{subsections/spells/cause-fear.tex}
\input{subsections/spells/chain-lightning.tex}
\input{subsections/spells/changestaff.tex}
\input{subsections/spells/chaos-hammer.tex}
\Spell{Charm Animal}{charm animal}
{Enchantment (Charm) [Mind-Affecting]}
{
	\textbf{Level:}
	Drd 1, Rgr 1\\
	\textbf{Target:}
	One animal\\
}
{
	This spell functions like charm person, except that it affects a creature of the animal type.

}

\Spell{Charm Monster, Mass}{mass charm monster}
{Enchantment (Charm) [Mind-Affecting]}
{
	\textbf{Level:}
	Wiz 8\\
	\textbf{Components:}
	V\\
	\textbf{Targets:}
	One or more creatures, no two of which can be more than 9 m apart\\
	\textbf{Duration:}
	One day/level\\
}
{
	This spell functions like \spell{charm monster}, except that mass charm monster affects a number of creatures whose combined HD do not exceed twice your level, or at least one creature regardless of HD. If there are more potential targets than you can affect, you choose them one at a time until you choose a creature with too many HD.

}

\Spell{Charm Monster}{charm monster}
{Enchantment (Charm) [Mind-Affecting]}
{
	\textbf{Level:}
	Wiz 4\\
	\textbf{Target:}
	One living creature\\
	\textbf{Duration:}
	One day/level\\
}
{
	This spell functions like \spell{charm person}, except that the effect is not restricted by creature type or size.

}

\input{subsections/spells/charm-person.tex}
\Spell{Chill Metal}{chill metal}
{Transmutation [Cold]}
{
	\textbf{Level:}
	Drd 2\\
	\textbf{Components:}
	V, S, DF\\
	\textbf{Casting Time:}
	1 standard action\\
	\textbf{Range:}
	Close (7.5 m + 1.5 m/2 levels)\\
	\textbf{Target:}
	Metal equipment of one creature per two levels, no two of which can be more than 9 m apart; or 12.5 kg of metal/level, none of which can be more than 9 m away from any of the rest\\
	\textbf{Duration:}
	7 rounds\\
	\textbf{Saving Throw:}
	Will negates (object)\\
	\textbf{Spell Resistance:}
	Yes (object)\\
}
{
	\emph{Chill metal} makes metal extremely cold. Unattended, nonmagical metal gets no saving throw. Magical metal is allowed a saving throw against the spell. An item in a creature's possession uses the creature's saving throw bonus unless its own is higher.

	A creature takes cold damage if its equipment is chilled. It takes full damage if its armor is affected or if it is holding, touching, wearing, or carrying metal weighing one-fifth of its weight. The creature takes minimum damage (1 point or 2 points; see the table) if it's not wearing metal armor and the metal that it's carrying weighs less than one-fifth of its weight.

\Table{}{lXX}{
\tableheader Round & \tableheader Metal Temperature & \tableheader Damage\\
	1 & Cold & None\\
	2 & Icy & 1d4 points\\
	3--5 & Freezing & 2d4 points\\
	6 & Icy & 1d4 points\\
	7 & Cold & None\\
}

	On the first round of the spell, the metal becomes chilly and uncomfortable to touch but deals no damage. The same effect also occurs on the last round of the spell's duration. During the second (and also the next-to-last) round, icy coldness causes pain and damage. In the third, fourth, and fifth rounds, the metal is freezing cold, causing more damage, as shown on the table below.

	Any heat intense enough to damage the creature negates cold damage from the spell (and vice versa) on a point-for-point basis. Underwater, chill metal deals no damage, but ice immediately forms around the affected metal, making it more buoyant.

	\emph{Chill metal} counters and dispels \spell{heat metal}.

}

\input{subsections/spells/chill-touch.tex}
\Spell{Circle of Death}{circle of death}
{Necromancy [Death]}
{
	\textbf{Level:}
	Wiz 6\\
	\textbf{Components:}
	V, S, M\\
	\textbf{Casting Time:}
	1 standard action\\
	\textbf{Range:}
	Medium (30 m + 3 m/level)\\
	\textbf{Area:}
	Several living creatures within a 12-m-radius burst\\
	\textbf{Duration:}
	Instantaneous\\
	\textbf{Saving Throw:}
	Fortitude negates\\
	\textbf{Spell Resistance:}
	Yes\\
}
{
	A circle of death snuffs out the life force of living creatures, killing them instantly.

	The spell slays 1d4 HD worth of living creatures per caster level (maximum 20d4). Creatures with the fewest HD are affected first; among creatures with equal HD, those who are closest to the burst's point of origin are affected first. No creature of 9 or more HD can be affected, and Hit Dice that are not sufficient to affect a creature are wasted.

	\textit{Material Component}:
	The powder of a crushed black pearl with a minimum value of 500 gp.

}

\Spell{Clairaudience/Clairvoyance}{clairaudience/clairvoyance}
{Divination (Scrying)}
{
	\textbf{Level:}
	Ass 4, Knowledge 3, Wiz 3\\
	\textbf{Components:}
	V, S, F/DF\\
	\textbf{Casting Time:}
	10 minutes\\
	\textbf{Range:}
	Long (120 m + 12 m/level)\\
	\textbf{Effect:}
	Magical sensor\\
	\textbf{Duration:}
	1 min./level (D)\\
	\textbf{Saving Throw:}
	None\\
	\textbf{Spell Resistance:}
	No\\
}
{
	\emph{Clairaudience/clairvoyance} creates an invisible magical sensor at a specific location that enables you to hear or see (your choice) almost as if you were there. You don't need line of sight or line of effect, but the locale must be known---a place familiar to you or an obvious one. Once you have selected the locale, the sensor doesn't move, but you can rotate it in all directions to view the area as desired. Unlike other scrying spells, this spell does not allow magically or supernaturally enhanced senses to work through it. If the chosen locale is magically dark, you see nothing. If it is naturally pitch black, you can see in a 3-meter radius around the center of the spell's effect. \emph{Clairaudience/clairvoyance} functions only on the plane of existence you are currently occupying.

	\textit{Arcane Focus:}
	A small horn (for hearing) or a glass eye (for seeing).

}

\Spell{Clenched Fist}{clenched fist}
{Evocation [Force]}
{
	\textbf{Level:}
	Wiz 8, Strength 8\\
	\textbf{Components:}
	V, S, F/DF\\
}
{
	This spell functions like \spell{interposing hand}, except that the hand can interpose itself, push, or strike one opponent that you select. The floating hand can move as far as 18 meters and can attack in the same round. Since this hand is directed by you, its ability to notice or attack invisible or concealed creatures is no better than yours.

	The hand attacks once per round, and its attack bonus equals your caster level + your Intelligence, or Charisma modifier (for a wizard or a templar, respectively), +11 for the hand's Strength score (33), -1 for being Large. The hand deals 1d8+11 points of damage on each attack, and any creature struck must make a Fortitude save (against this spell's save DC) or be stunned for 1 round. Directing the spell to a new target is a move action.

	The clenched fist can also interpose itself as interposing hand does, or it can bull rush an opponent as forceful hand does, but at a +15 bonus on the Strength check.

	Templars who cast this spell name it for their sorcerer-kings.
	% Clerics who cast this spell name it for their deities.

	\textit{Arcane Focus}:
	A leather glove.

}

\Spell{Cloak of Chaos}{cloak of chaos}
{Abjuration [Chaotic]}
{
	\textbf{Level:}
	Chaos 8, Clr 8\\
	\textbf{Components:}
	V, S, F\\
	\textbf{Casting Time:}
	1 standard action\\
	\textbf{Range:}
	6 m\\
	\textbf{Targets:}
	One creature/level in a 6-m-radius burst centered on you\\
	\textbf{Duration:}
	1 round/level (D)\\
	\textbf{Saving Throw:}
	See text\\
	\textbf{Spell Resistance:}
	Yes (harmless)\\
}
{
	A random pattern of color surrounds the subjects, protecting them from attacks, granting them resistance to spells cast by lawful creatures, and causing lawful creatures that strike the subjects to become confused. This abjuration has four effects.

	First, each warded creature gains a +4 deflection bonus to AC and a +4 resistance bonus on saves. Unlike \spell{protection from law}, the benefit of this spell applies against all attacks, not just against attacks by lawful creatures.

	Second, each warded creature gains spell resistance 25 against lawful spells and spells cast by lawful creatures.

	Third, the abjuration blocks possession and mental influence, just as \spell{protection from law} does.

	Finally, if a lawful creature succeeds on a melee attack against a warded creature, the offending attacker is confused for 1 round (Will save negates, as with the \spell{confusion} spell, but against the save DC of cloak of chaos).

	\textit{Focus}:
	A tiny reliquary containing some sacred relic, such as a scrap of parchment from a chaotic text. The reliquary costs at least 500 gp.

}

\input{subsections/spells/clone.tex}
\input{subsections/spells/cloudkill.tex}
\input{subsections/spells/color-spray.tex}
\input{subsections/spells/command-greater.tex}
\input{subsections/spells/command-plants.tex}
\input{subsections/spells/command.tex}
\input{subsections/spells/command-undead.tex}
\input{subsections/spells/commune.tex}
\input{subsections/spells/commune-with-nature.tex}
\Spell{Comprehend Languages}{comprehend languages}
{Divination [Ritual]}
{
	\textbf{Level:}
	Clr 1, Mind 1, Tmp 1, Wiz 1\\
	\textbf{Components:}
	V, S, M/DF\\
	\textbf{Casting Time:}
	1 standard action\\
	\textbf{Range:}
	Personal\\
	\textbf{Target:}
	You\\
	\textbf{Duration:}
	10 min./level\\
}
{
	You can understand the spoken words of creatures or read otherwise incomprehensible written messages. In either case, you must touch the creature or the writing. The ability to read does not necessarily impart insight into the material, merely its literal meaning. The spell enables you to understand or read an unknown language, not speak or write it.

	Written material can be read at the rate of one page (250 words) per minute. Magical writing cannot be read, though the spell reveals that it is magical. This spell can be foiled by certain warding magic (such as the secret page and illusory script spells). It does not decipher codes or reveal messages concealed in otherwise normal text.

	\emph{Comprehend languages} can be made permanent with a \spell{permanency} spell.

	\textit{Arcane Material Component:}
	A pinch of soot and a few grains of salt.

}

\Spell{Cone of Cold}{cone of cold}
{Evocation [Cold]}
{
	\textbf{Level:}
	Wiz 5\\
	\textbf{Components:}
	V, S, M\\
	\textbf{Casting Time:}
	1 standard action\\
	\textbf{Range:}
	18 m\\
	\textbf{Area:}
	Cone-shaped burst\\
	\textbf{Duration:}
	Instantaneous\\
	\textbf{Saving Throw:}
	Reflex half\\
	\textbf{Spell Resistance:}
	Yes\\
}
{
	\emph{Cone of cold} creates an area of extreme cold, originating at your hand and extending outward in a cone. It drains heat, dealing 1d6 points of cold damage per caster level (maximum 15d6).

	\textit{Arcane Material Component}:
	A very small crystal or glass cone.

}

\Spell{Confusion, Lesser}{confusion, lesser}
{Enchantment (Compulsion) [Mind-Affecting]}
{
	\textbf{Level:}
	Brd 1\\
	\textbf{Components:}
	V, S, DF\\
	\textbf{Range:}
	Close (7.5 m + 1.5 m/2 levels)\\
	\textbf{Target:}
	One living creature\\
	\textbf{Duration:}
	1 round\\
}
{
	This spell causes a single creature to become confused for 1 round. See the confusion spell, above, to determine the exact effect on the subject.

}

\Spell{Confusion}{confusion}
{Enchantment (Compulsion) [Mind-Affecting]}
{
	\textbf{Level:}
	Wiz 4, Trickery 4\\
	\textbf{Components:}
	V, S, M/DF\\
	\textbf{Casting Time:}
	1 standard action\\
	\textbf{Range:}
	Medium (30 m + 3 m/level)\\
	\textbf{Targets:}
	All creatures in a 15-ft. radius burst\\
	\textbf{Duration:}
	1 round/level\\
	\textbf{Saving Throw:}
	Will negates\\
	\textbf{Spell Resistance:}
	Yes\\
}
{
	This spell causes the targets to become confused, making them unable to independently determine what they will do.

	Roll on the following table at the beginning of each subject's turn each round to see what the subject does in that round.

\Table{}{XX}{
	\tableheader d\% & \tableheader Behavior\\
	01--10 & Attack caster with melee or ranged weapons (or close with caster if attack is not possible).\\
	11--20 & Act normally.\\
	21--50 & Do nothing but babble incoherently.\\
	51--70 & Flee away from caster at top possible speed.\\
	71--100 & Attack nearest creature (for this purpose, a familiar counts as part of the subject's self).\\
}
	A confused character who can't carry out the indicated action does nothing but babble incoherently. Attackers are not at any special advantage when attacking a confused character. Any confused character who is attacked automatically attacks its attackers on its next turn, as long as it is still confused when its turn comes. Note that a confused character will not make attacks of opportunity against any creature that it is not already devoted to attacking (either because of its most recent action or because it has just been attacked).

	\textit{Arcane Material Component}:
	A set of three nut shells.

}

\input{subsections/spells/consecrate.tex}
\Spell{Contact Other Plane}{contact other plane}
{Divination}
{
	\textbf{Level:}
	Wiz 5\\
	\textbf{Components:}
	V\\
	\textbf{Casting Time:}
	10 minutes\\
	\textbf{Range:}
	Personal\\
	\textbf{Target:}
	You\\
	\textbf{Duration:}
	Concentration\\
}
{
	You send your mind to another plane of existence (an Elemental Plane or some plane farther removed) in order to receive advice and information from powers there. (See the accompanying table for possible consequences and results of the attempt.) The powers reply in a language you understand, but they resent such contact and give only brief answers to your questions. (All questions are answered with ``yes,'' ``no,'' ``maybe,'' ``never,'' ``irrelevant,'' or some other one-word answer.)

	You must concentrate on maintaining the spell (a standard action) in order to ask questions at the rate of one per round. A question is answered by the power during the same round. For every two caster levels, you may ask one question.

	Contact with minds far removed from your home plane increases the probability that you will incur a decrease to Intelligence and Charisma, but the chance of the power knowing the answer, as well as the probability of the entity answering correctly, are likewise increased by moving to distant planes.

	Once the Outer Planes are reached, the power of the deity contacted determines the effects. (Random results obtained from the table are subject to the personalities of individual deities.)

	On rare occasions, this divination may be blocked by an act of certain deities or forces.

	d\% is rolled for the result shown on \tabref{Contact Other Plane}.

}
\Spell{Contagion}{contagion}
{Necromancy [Evil]}
{
	\textbf{Level:}
	Clr 3, Destruction 3, Drd 3, Wiz 4\\
	\textbf{Components:}
	V, S\\
	\textbf{Casting Time:}
	1 standard action\\
	\textbf{Range:}
	Touch\\
	\textbf{Target:}
	Living creature touched\\
	\textbf{Duration:}
	Instantaneous\\
	\textbf{Saving Throw:}
	Fortitude negates\\
	\textbf{Spell Resistance:}
	Yes\\
}
{
	The subject contracts a disease selected from the table, which strikes immediately (no incubation period). The DC noted is for the subsequent saves (use contagion's normal save DC for the initial saving throw).

\Table{}{XcX}{
\tableheader Disease & \tableheader DC & \tableheader Damage\\
	Blinding sickness & 16 & 1d4 Str\footnotemark[1]\\
	Cackle fever & 16 & 1d6 Wis\\
	Filth fever & 12 & 1d3 Dex and 1d3 Con\\
	Mindfire & 12 & 1d4 Int\\
	Red ache & 15 & 1d6 Str\\
	Shakes & 13 & 1d8 Dex\\
	Slimy doom & 14 & 1d4 Con\\

\TableNote{3}{1 Each time a victim takes 2 or more points of Strength damage from blinding sickness, he or she must make another Fortitude save (using the disease's save DC) or be permanently blinded.}
}

}

\input{subsections/spells/contingency.tex}
\input{subsections/spells/continual-flame.tex}
\input{subsections/spells/control-plants.tex}
\input{subsections/spells/control-undead.tex}
\input{subsections/spells/control-water.tex}
\input{subsections/spells/control-weather.tex}
\Spell{Control Winds}{control winds}
{Transmutation [Air]}
{
	\textbf{Level:}
	Air 5, Drd 5\\
	\textbf{Components:}
	V, S\\
	\textbf{Casting Time:}
	1 standard action\\
	\textbf{Range:}
	12 m/level\\
	\textbf{Area:}
	12 m/level radius cylinder 12 m high\\
	\textbf{Duration:}
	10 min./level\\
	\textbf{Saving Throw:}
	Fortitude negates\\
	\textbf{Spell Resistance:}
	No\\
}
{
	You alter wind force in the area surrounding you. You can make the wind blow in a certain direction or manner, increase its strength, or decrease its strength. The new wind direction and strength persist until the spell ends or until you choose to alter your handiwork, which requires concentration. You may create an ``eye'' of calm air up to 24 meters in diameter at the center of the area if you so desire, and you may choose to limit the area to any cylindrical area less than your full limit.

	\textit{Wind Direction:}
	You may choose one of four basic wind patterns to function over the spell's area.

\begin{itemize*}
\item A downdraft blows from the center outward in equal strength in all directions.
\item An updraft blows from the outer edges in toward the center in equal strength from all directions, veering upward before impinging on the eye in the center.
\item A rotation causes the winds to circle the center in clockwise or counterclockwise fashion.
\item A blast simply causes the winds to blow in one direction across the entire area from one side to the other.
\end{itemize*}

	\textit{Wind Strength:}
	For every three caster levels, you can increase or decrease wind strength by one level. Each round on your turn, a creature in the wind must make a Fortitude save or suffer the effect of being in the windy area.

	Strong winds (35+ km/h) make sailing difficult.

	A severe wind (50+ km/h) causes minor ship and building damage.

	A windstorm (80+ km/h) drives most flying creatures from the skies, uproots small trees, knocks down light wooden structures, tears off roofs, and endangers ships.

	Hurricane force winds (120+ km/h) destroy wooden buildings, sometimes uproot even large trees, and cause most ships to founder.

	A tornado (280+ km/h) destroys all nonfortified buildings and often uproots large trees.

}

\Spell{Create Food and Water}{create food and water}
{Conjuration (Creation)}
{
	\textbf{Level:}
	Clr 3\\
	\textbf{Components:}
	V, S\\
	\textbf{Casting Time:}
	10 minutes\\
	\textbf{Range:}
	Close (7.5 m + 1.5 m/2 levels)\\
	\textbf{Effect:}
	Food and water to sustain three humans or one horse/level for 24 hours\\
	\textbf{Duration:}
	24 hours; see text\\
	\textbf{Saving Throw:}
	None\\
	\textbf{Spell Resistance:}
	No\\
}
{
	The food that this spell creates is simple fare of your choice---highly nourishing, if rather bland. Food so created decays and becomes inedible within 24 hours, although it can be kept fresh for another 24 hours by casting a purify food and drink spell on it. The water created by this spell is just like clean rain water, and it doesn't go bad as the food does.

}

\Spell{Create Greater Undead}{create greater undead}
{Necromancy [Evil]}
{
	\textbf{Level:}
	Clr 8, Death 8, Wiz 8\\
}
{
	This spell functions like \spell{create undead}, except that you can create more powerful and intelligent sorts of undead: shadows, wraiths, spectres, and devourers. The type or types of undead you can create is based on your caster level, as shown on the table below.

\Table{}{XX}{
\tableheader Caster Level & \tableheader Undead Created\\
	15th or lower & Shadow\\
	16th--17th & Wraith\\
	18th--19th & Spectre\\
	20th or higher & Devourer\\
}

}

\Spell{Create Undead}{create undead}
{Necromancy [Evil]}
{
	\textbf{Level:}
	Clr 6, Death 6, Evil 6, Wiz 6\\
	\textbf{Components:}
	V, S, M\\
	\textbf{Casting Time:}
	1 hour\\
	\textbf{Range:}
	Close (7.5 m + 1.5 m/2 levels)\\
	\textbf{Target:}
	One corpse\\
	\textbf{Duration:}
	Instantaneous\\
	\textbf{Saving Throw:}
	None\\
	\textbf{Spell Resistance:}
	No\\
}
{
\Table{}{XX}{
\tableheader Caster Level & \tableheader Undead Created\\
	11th or lower & Ghoul\\
	12th--14th & Ghast\\
	15th--17th & Mummy\\
	18th or higher & Mohrg\\
}
	A much more potent spell than animate dead, this evil spell allows you to create more powerful sorts of undead: ghouls, ghasts, mummies, and mohrgs. The type or types of undead you can create is based on your caster level, as shown on the table below.

	You may create less powerful undead than your level would allow if you choose. Created undead are not automatically under the control of their animator. If you are capable of commanding undead, you may attempt to command the undead creature as it forms.

	This spell must be cast at night.

	\textit{Material Component}:
	A clay pot filled with grave dirt and another filled with brackish water. The spell must be cast on a dead body. You must place a black onyx gem worth at least 50 gp per HD of the undead to be created into the mouth or eye socket of each corpse. The magic of the spell turns these gems into worthless shells.

}

\input{subsections/spells/create-water.tex}
\Spell{Creeping Doom}{creeping doom}
{Conjuration (Summoning)}
{
	\textbf{Level:}
	Drd 7\\
	\textbf{Components:}
	V, S\\
	\textbf{Casting Time:}
	1 round\\
	\textbf{Range:}
	Close (7.5 m + 1.5 m/2 levels)/ 100 ft.; see text\\
	\textbf{Effect:}
	One swarm of centipedes per two levels\\
	\textbf{Duration:}
	1 min./level\\
	\textbf{Saving Throw:}
	None\\
	\textbf{Spell Resistance:}
	No\\
}
{
	When you utter the spell of creeping doom, you call forth a mass of centipede swarms (one per two caster levels, to a maximum of ten swarms at 20th level), which need not appear adjacent to one another.

	You may summon the centipede swarms so that they share the area of other creatures. The swarms remain stationary, attacking any creatures in their area, unless you command the creeping doom to move (a standard action). As a standard action, you can command any number of the swarms to move toward any prey within 30 meters of you. You cannot command any swarm to move more than 30 meters away from you, and if you move more than 30 meters from any swarm, that swarm remains stationary, attacking any creatures in its area (but it can be commanded again if you move within 30 meters).

}

\input{subsections/spells/crown-of-glory.tex}
\Spell{Crushing Despair}{crushing despair}
{Enchantment (Compulsion) [Mind-Affecting]}
{
	\textbf{Level:}
	Wiz 4\\
	\textbf{Components:}
	V, S, M\\
	\textbf{Casting Time:}
	1 standard action\\
	\textbf{Range:}
	9 m\\
	\textbf{Area:}
	Cone-shaped burst\\
	\textbf{Duration:}
	1 min./level\\
	\textbf{Saving Throw:}
	Will negates\\
	\textbf{Spell Resistance:}
	Yes\\
}
{
	An invisible cone of despair causes great sadness in the subjects. Each affected creature takes a -2 penalty on attack rolls, saving throws, ability checks, skill checks, and weapon damage rolls.

	Crushing despair counters and dispels good hope.

	\textit{Material Component}:
	A vial of tears.

}

\Spell{Crushing Hand}{crushing hand}
{Evocation [Force]}
{
	\textbf{Level:}
	Wiz 9, Strength 9\\
	\textbf{Components:}
	V, S, M, F/DF\\
}
{
	This spell functions like \spell{interposing hand}, except that the hand can interpose itself, push, or crush one opponent that you select.

	The crushing hand can grapple an opponent like grasping hand does. Its grapple bonus equals your caster level + your Intelligence or Charisma modifier (for a wizard or templar, respectively), +12 for the hand's Strength score (35), +4 for being Large. The hand deals 2d6+12 points of damage (lethal, not nonlethal) on each successful grapple check against an opponent.

	The crushing hand can also interpose itself as interposing hand does, or it can bull rush an opponent as forceful hand does, but at a +18 bonus.

	Directing the spell to a new target is a move action.

	Templars who cast this spell name it for their sorcerer-kings.
	% Clerics who cast this spell name it for their deities.

	\textit{Arcane Material Component}:
	The shell of an egg.

	\textit{Arcane Focus}:
	A glove of snakeskin.

}

\Spell{Cure Critical Wounds, Mass}{cure critical wounds, mass}
{Conjuration (Healing)}
{
	\textbf{Level:}
	Clr 8, Drd 9, Healing 8\\
}
{
	This spell functions like mass cure light wounds, except that it cures 4d8 points of damage +1 point per caster level (maximum +40).

}

\Spell{Cure Critical Wounds}{cure critical wounds}
{Conjuration (Healing)}
{
	\textbf{Level:}
	Clr 4, Drd 5, Healing 4\\
}
{
	This spell functions like cure light wounds, except that it cures 4d8 points of damage +1 point per caster level (maximum +20).

}

\Spell{Cure Light Wounds, Mass}{mass cure light wounds}
{Conjuration (Healing)}
{
	\textbf{Level:}
	Clr 5, Drd 6\\
	\textbf{Components:}
	V, S\\
	\textbf{Casting Time:}
	1 standard action\\
	\textbf{Range:}
	Close (7.5 m + 1.5 m/2 levels)\\
	\textbf{Target:}
	One creature/level, no two of which can be more than 9 m apart\\
	\textbf{Duration:}
	Instantaneous\\
	\textbf{Saving Throw:}
	Will half (harmless) or Will half; see text\\
	\textbf{Spell Resistance:}
	Yes (harmless) or Yes; see text\\
}
{
	You channel positive energy to cure 1d8 points of damage +1 point per caster level (maximum +25) in each selected creature.

	Like other \spellref{cure light wounds}{cure} spells, \emph{mass cure light wounds} deals damage to undead in its area rather than curing them. Each affected undead may attempt a Will save for half damage.

}

\input{subsections/spells/cure-light-wounds.tex}
\Spell{Cure Minor Wounds}{cure minor wounds}
{Conjuration (Healing)}
{
	\textbf{Level:}
	Clr 0, Drd 0\\
}
{
	This spell functions like cure light wounds, except that it cures only 1 point of damage.

}

\Spell{Cure Moderate Wounds, Mass}{cure moderate wounds, mass}
{Conjuration (Healing)}
{
	\textbf{Level:}
	Clr 6, Drd 7\\
}
{
	This spell functions like mass cure light wounds, except that it cures 2d8 points of damage +1 point per caster level (maximum +30).

}

\Spell{Cure Moderate Wounds}{cure moderate wounds}
{Conjuration (Healing)}
{
	\textbf{Level:}
	Clr 2, Drd 3, ElC 2, Rgr 3, Tmp 2\\
}
{
	This spell functions like \spell{cure light wounds}, except that it cures 2d8 points of damage +1 point per caster level (maximum +10).

}

\Spell{Cure Serious Wounds, Mass}{mass cure serious wounds}
{Conjuration (Healing)}
{
	\textbf{Level:}
	Clr 7, Drd 8\\
}
{
	This spell functions like \spell{mass cure light wounds}, except that it cures 3d8 points of damage +1 point per caster level (maximum +35).

}

\Spell{Cure Serious Wounds}{cure serious wounds}
{Conjuration (Healing)}
{
	\textbf{Level:}
	Clr 3, Drd 4, ElC 3, Rgr 4, Tmp 3\\
}
{
	This spell functions like \spell{cure light wounds}, except that it cures 3d8 points of damage +1 point per caster level (maximum +15).

}

\input{subsections/spells/curse-water.tex}
\Spell{Dancing Lights}{dancing lights}
{Evocation [Light]}
{
	\textbf{Level:}
	Wiz 0\\
	\textbf{Components:}
	V, S\\
	\textbf{Casting Time:}
	1 standard action\\
	\textbf{Range:}
	Medium (30 m + 3 m/level)\\
	\textbf{Effect:}
	Up to four lights, all within a 3-m-radius area\\
	\textbf{Duration:}
	1 minute (D)\\
	\textbf{Saving Throw:}
	None\\
	\textbf{Spell Resistance:}
	No\\
}
{
	Depending on the version selected, you create up to four lights that resemble lanterns or torches (and cast that amount of light), or up to four glowing spheres of light (which look like will-o'-wisps), or one faintly glowing, vaguely humanoid shape. The dancing lights must stay within a 3-meter-radius area in relation to each other but otherwise move as you desire (no concentration required): forward or back, up or down, straight or turning corners, or the like. The lights can move up to 30 meters per round. A light winks out if the distance between you and it exceeds the spell's range.

	\emph{Dancing lights} can be made permanent with a \spell{permanency} spell.

}

\Spell{Darkness}{darkness}
{Evocation [Darkness]}
{
	\textbf{Level:}
	Ass 2, Clr 2, Wiz 2\\
	\textbf{Components:}
	V, M/DF\\
	\textbf{Casting Time:}
	1 standard action\\
	\textbf{Range:}
	Touch\\
	\textbf{Target:}
	Object touched\\
	\textbf{Duration:}
	10 min./level (D)\\
	\textbf{Saving Throw:}
	None\\
	\textbf{Spell Resistance:}
	No\\
}
{
	This spell causes an object to radiate shadowy illumination out to a 6-meter radius. All creatures in the area gain concealment (20\% miss chance). Even creatures that can normally see in such conditions (such as with darkvision or low-light vision) have the miss chance in an area shrouded in magical darkness.

	Normal lights (torches, candles, lanterns, and so forth) are incapable of brightening the area, as are light spells of lower level. Higher level light spells are not affected by \emph{darkness}.

	If \emph{darkness} is cast on a small object that is then placed inside or under a lightproof covering, the spell's effect is blocked until the covering is removed.

	\emph{Darkness} counters or dispels any light spell of equal or lower spell level.

	\textit{Arcane Material Component}:
	A bit of bat fur and either a drop of pitch or a piece of coal.

}

\Spell{Darkvision}{darkvision}
{Transmutation}
{
	\textbf{Level:}
	Rgr 3, Wiz 2\\
	\textbf{Components:}
	V, S, M\\
	\textbf{Casting Time:}
	1 standard action\\
	\textbf{Range:}
	Touch\\
	\textbf{Target:}
	Creature touched\\
	\textbf{Duration:}
	1 hour/level\\
	\textbf{Saving Throw:}
	Will negates (harmless)\\
	\textbf{Spell Resistance:}
	Yes (harmless)\\
}
{
	The subject gains the ability to see 18 meters even in total darkness. Darkvision is black and white only but otherwise like normal sight. \emph{Darkvision} does not grant one the ability to see in magical darkness.

	\emph{Darkvision} can be made permanent with a \spell{permanency} spell.

	\textit{Material Component:}
	Either a pinch of dried carrot or an agate.

}

\Spell{Daylight}{daylight}
{Evocation [Light]}
{
	\textbf{Level:}
	Clr 3, Drd 3, Pal 3, Wiz 3\\
	\textbf{Components:}
	V, S\\
	\textbf{Casting Time:}
	1 standard action\\
	\textbf{Range:}
	Touch\\
	\textbf{Target:}
	Object touched\\
	\textbf{Duration:}
	10 min./level (D)\\
	\textbf{Saving Throw:}
	None\\
	\textbf{Spell Resistance:}
	No\\
}
{
	The object touched sheds light as bright as full daylight in a 60-foot radius, and dim light for an additional 18 meters beyond that. Creatures that take penalties in bright light also take them while within the radius of this magical light. Despite its name, this spell is not the equivalent of daylight for the purposes of creatures that are damaged or destroyed by bright light.

	If daylight is cast on a small object that is then placed inside or under a light-proof covering, the spell's effects are blocked until the covering is removed.

	Daylight brought into an area of magical darkness (or vice versa) is temporarily negated, so that the otherwise prevailing light conditions exist in the overlapping areas of effect.

	Daylight counters or dispels any darkness spell of equal or lower level, such as darkness.

}

\Spell{Daze Monster}{daze monster}
{Enchantment (Compulsion) [Mind-Affecting]}
{
	\textbf{Level:}
	Wiz 2\\
	\textbf{Range:}
	Medium (30 m + 3 m/level)\\
	\textbf{Target:}
	One living creature of 6 HD or less\\
}
{
	This spell functions like daze, but daze monster can affect any one living creature of any type. Creatures of 7 or more HD are not affected.

}

\input{subsections/spells/daze.tex}
\input{subsections/spells/death-knell.tex}
\input{subsections/spells/death-ward.tex}
\input{subsections/spells/deathwatch.tex}
\Spell{Deeper Darkness}{deeper darkness}
{Evocation [Darkness]}
{
	\textbf{Level:}
	Clr 3\\
	\textbf{Duration:}
	One day/level (D)\\
}
{
	This spell functions like \spell{darkness}, except that the object radiates shadowy illumination in a 18-meter radius and the darkness lasts longer.

Daylight brought into an area of deeper darkness (or vice versa) is temporarily negated, so that the otherwise prevailing light conditions exist in the overlapping areas of effect.

	\emph{Deeper darkness} counters and dispels any light spell of equal or lower level, including \spell{daylight} and \spell{light}.

}

\Spell{Deep Slumber}{deep slumber}
{Enchantment (Compulsion) [Mind-Affecting]}
{
	\textbf{Level:}
	Ass 3, Wiz 3\\
	\textbf{Range:}
	Close (7.5 m + 1.5 m/2 levels)\\
}
{
	This spell functions like \spell{sleep}, except that it affects 10 HD of creatures.

}

\input{subsections/spells/delayed-blast-fireball.tex}
\input{subsections/spells/delay-poison.tex}
\Spell{Demand}{demand}
{Enchantment (Compulsion) [Mind-Affecting]}
{
	\textbf{Level:}
	Charm 8, Nobility 8, Wiz 8\\
	\textbf{Saving Throw:}
	Will partial\\
	\textbf{Spell Resistance:}
	Yes\\
}
{
	This spell functions like \spell{sending}, but the message can also contain a \emph{suggestion} (see the \spell{suggestion} spell), which the subject does its best to carry out. A successful Will save negates the \emph{suggestion} effect but not the contact itself. The \emph{demand}, if received, is understood even if the subject's Intelligence score is as low as 1. If the message is impossible or meaningless according to the circumstances that exist for the subject at the time the \emph{demand} is issued, the message is understood but the \emph{suggestion} is ineffective.

	The \emph{demand}'s message to the creature must be twenty-five words or less, including the \emph{suggestion}. The creature can also give a short reply immediately.

	\textit{Material Component}:
	A short piece of copper wire and some small part of the subject---a hair, a bit of nail, or the like.

}

\Spell{Desecrate}{desecrate}
{Evocation [Evil]}
{
	\textbf{Level:}
	Clr 2, Evil 2\\
	\textbf{Components:}
	V, S, M, DF\\
	\textbf{Casting Time:}
	1 standard action\\
	\textbf{Range:}
	Close (7.5 m + 1.5 m/2 levels)\\
	\textbf{Area:}
	6-m-radius emanation\\
	\textbf{Duration:}
	2 hours/level\\
	\textbf{Saving Throw:}
	None\\
	\textbf{Spell Resistance:}
	Yes\\
}
{
	This spell imbues an area with negative energy. Each Charisma check made to turn undead within this area takes a -3 profane penalty, and every undead creature entering a desecrated area gains a +1 profane bonus on attack rolls, damage rolls, and saving throws. An undead creature created within or summoned into such an area gains +1 hit points per HD.

	If the desecrated area contains an altar, shrine, or other permanent fixture dedicated to your deity or aligned higher power, the modifiers given above are doubled (-6 profane penalty on turning checks, +2 profane bonus and +2 hit points per HD for undead in the area).

	Furthermore, anyone who casts animate dead within this area may create as many as double the normal amount of undead (that is, 4 HD per caster level rather than 2 HD per caster level).

	If the area contains an altar, shrine, or other permanent fixture of a deity, pantheon, or higher power other than your patron, the desecrate spell instead curses the area, cutting off its connection with the associated deity or power. This secondary function, if used, does not also grant the bonuses and penalties relating to undead, as given above.

	Desecrate counters and dispels consecrate.

	\textit{Material Component}:
	A vial of unholy water and 25 gp worth (5 pounds) of silver dust, all of which must be sprinkled around the area.

}

\Spell{Destruction}{destruction}
{Necromancy [Death]}
{
	\textbf{Level:}
	Clr 7, Death 7\\
	\textbf{Components:}
	V, S, F\\
	\textbf{Casting Time:}
	1 standard action\\
	\textbf{Range:}
	Close (7.5 m + 1.5 m/2 levels)\\
	\textbf{Target:}
	One creature\\
	\textbf{Duration:}
	Instantaneous\\
	\textbf{Saving Throw:}
	Fortitude partial\\
	\textbf{Spell Resistance:}
	Yes\\
}
{
	This spell instantly slays the subject and consumes its remains (but not its equipment and possessions) utterly. If the target's Fortitude saving throw succeeds, it instead takes 10d6 points of damage. The only way to restore life to a character who has failed to save against this spell is to use \spell{true resurrection}, a carefully worded \spell{wish} spell followed by \spell{resurrection}, or \spell{miracle}.

	\textit{Focus}:
	A special holy (or unholy) symbol of silver marked with verses of anathema (cost 500 cp).

}

\Spell{Detect Animals or Plants}{detect animals or plants}
{Divination [Ritual]}
{
	\textbf{Level:}
	Drd 1, Rgr 1\\
	\textbf{Components:}
	V, S\\
	\textbf{Casting Time:}
	1 standard action\\
	\textbf{Range:}
	Long (120 m + 12 m/level)\\
	\textbf{Area:}
	Cone-shaped emanation\\
	\textbf{Duration:}
	Concentration, up to 10 min./level (D)\\
	\textbf{Saving Throw:}
	None\\
	\textbf{Spell Resistance:}
	No\\
}
{
	You can detect a particular kind of animal or plant in a cone emanating out from you in whatever direction you face. You must think of a kind of animal or plant when using the spell, but you can change the animal or plant kind each round. The amount of information revealed depends on how long you search a particular area or focus on a specific kind of animal or plant.

	\textit{1st Round:}
	Presence or absence of that kind of animal or plant in the area.

	\textit{2nd Round:}
	Number of individuals of the specified kind in the area, and the condition of the healthiest specimen.

	\textit{3rd Round:}
	The condition (see below) and location of each individual present. If an animal or plant is outside your line of sight, then you discern its direction but not its exact location.

	\textit{Conditions:}
	For purposes of this spell, the categories of condition are as follows:

	\emph{Normal:}
	Has at least 90\% of full normal hit points, free of disease.

	\emph{Fair:}
	30\% to 90\% of full normal hit points remaining.

	\emph{Poor:}
	Less than 30\% of full normal hit points remaining, afflicted with a disease, or suffering from a debilitating injury.

	\emph{Weak:}
	0 or fewer hit points remaining, afflicted with a disease in the terminal stage, or crippled.

	If a creature falls into more than one category, the spell indicates the weaker of the two.

	Each round you can turn to detect a kind of animal or plant in a new area. The spell can penetrate barriers, but 30 centimeters of stone, 2.5 centimeters of common metal, a thin sheet of lead, or 1 meter of wood or dirt blocks it.

}

\Spell{Detect Chaos}{detect chaos}
{Divination}
{
	\textbf{Level:}
	Clr 1\\
}
{
	This spell functions like detect evil, except that it detects the auras of chaotic creatures, clerics of chaotic deities, chaotic spells, and chaotic magic items, and you are vulnerable to an overwhelming chaotic aura if you are lawful.

}

\Spell{Detect Evil}{detect evil}
{Divination}
{
	\textbf{Level:}
	Clr 1\\
	\textbf{Components:}
	V, S, DF\\
	\textbf{Casting Time:}
	1 standard action\\
	\textbf{Range:}
	60 ft.\\
	\textbf{Area:}
	Cone-shaped emanation\\
	\textbf{Duration:}
	Concentration, up to 10 min./ level (D)\\
	\textbf{Saving Throw:}
	None\\
	\textbf{Spell Resistance:}
	No\\
}
{
	You can sense the presence of evil. The amount of information revealed depends on how long you study a particular area or subject.

	\textit{1st Round}:
	Presence or absence of evil.

	\textit{2nd Round}:
	Number of evil auras (creatures, objects, or spells) in the area and the power of the most potent evil aura present.

	If you are of good alignment, and the strongest evil aura's power is overwhelming (see below), and the HD or level of the aura's source is at least twice your character level, you are stunned for 1 round and the spell ends.

	\textit{3rd Round}:
	The power and location of each aura. If an aura is outside your line of sight, then you discern its direction but not its exact location.

	\textit{Aura Power}:
	An evil aura's power depends on the type of evil creature or object that you're detecting and its HD, caster level, or (in the case of a cleric) class level; see the accompanying table. If an aura falls into more than one strength category, the spell indicates the stronger of the two.

\Table{}{lcccc}{
& \multicolumn{4}{c}{\tableheader Aura Power}\\
\cmidrule[0.5pt]{2-5}
\tableheader Creature/Object & \tableheader Faint & \tableheader Moderate & \tableheader Strong & \tableheader Overwhelming\\

	Evil creature\footnotemark[1] (HD) & 10 or lower & 11--25 & 26--50 & 51 or higher\\
	Undead (HD) & 2 or lower & 3--8 & 9--20 & 21 or higher\\
	Evil outsider (HD) & 1 or lower & 2--4 & 5--10 & 11 or higher\\
	Cleric of an evil deity\footnotemark[2] (class levels) & 1 & 2--4 & 5--10 & 11 or higher\\
	Evil magic item or spell (caster level) & 2nd or lower & 3rd--8th & 9th--20th & 21st or higher\\
\TableNote{5}{1 Except for undead and outsiders, which have their own entries on the table.}\\
\TableNote{5}{2 Some characters who are not clerics may radiate an aura of equivalent power. The class description will indicate whether this applies.}\\

}
	\textit{Lingering Aura}:
	An evil aura lingers after its original source dissipates (in the case of a spell) or is destroyed (in the case of a creature or magic item). If detect evil is cast and directed at such a location, the spell indicates an aura strength of dim (even weaker than a faint aura). How long the aura lingers at this dim level depends on its original power:

\Table{}{lX}{
\tableheader Original Strength & \tableheader Duration of Lingering Aura\\
	Faint & 1d6 rounds\\
	Moderate & 1d6 minutes\\
	Strong & 1d6 $\times$ 10 minutes\\
	Overwhelming & 1d6 days\\
}

	Animals, traps, poisons, and other potential perils are not evil, and as such this spell does not detect them.

	Each round, you can turn to detect evil in a new area. The spell can penetrate barriers, but 1 foot of stone, 1 inch of common metal, a thin sheet of lead, or 3 feet of wood or dirt blocks it.

}

\Spell{Detect Good}{detect good}
{Divination}
{
	\textbf{Level:}
	Clr 1\\
}
{
	This spell functions like \spell{detect evil}, except that it detects the auras of good creatures, good clerics, good spells, and good magic items, and you are vulnerable to an overwhelming good aura if you are evil. Healing potions, antidotes, and similar beneficial items are not good.

}

\Spell{Detect Law}{detect law}
{Divination}
{
	\textbf{Level:}
	Clr 1\\
}
{
	This spell functions like detect evil, except that it detects the auras of lawful creatures, clerics of lawful deities, lawful spells, and lawful magic items, and you are vulnerable to an overwhelming lawful aura if you are chaotic.

}

\Spell{Detect Magic}{detect magic}
{Divination}
{
	\textbf{Level:}
	Clr 0, Drd 0, Wiz 0\\
	\textbf{Components:}
	V, S\\
	\textbf{Casting Time:}
	1 standard action\\
	\textbf{Range:}
	60 ft.\\
	\textbf{Area:}
	Cone-shaped emanation\\
	\textbf{Duration:}
	Concentration, up to 1 min./level (D)\\
	\textbf{Saving Throw:}
	None\\
	\textbf{Spell Resistance:}
	No\\
}
{
	You detect magical auras. The amount of information revealed depends on how long you study a particular area or subject.

	\textit{1st Round}:
	Presence or absence of magical auras.

	\textit{2nd Round}:
	Number of different magical auras and the power of the most potent aura.

	\textit{3rd Round}:
	The strength and location of each aura. If the items or creatures bearing the auras are in line of sight, you can make Spellcraft skill checks to determine the school of magic involved in each. (Make one check per aura; DC 15 + spell level, or 15 + half caster level for a nonspell effect.)

	Magical areas, multiple types of magic, or strong local magical emanations may distort or conceal weaker auras.

	\textit{Aura Strength}:
	An aura's power depends on a spell's functioning spell level or an item's caster level. If an aura falls into more than one category, detect magic indicates the stronger of the two.

	\textit{Lingering Aura}:
	A magical aura lingers after its original source dissipates (in the case of a spell) or is destroyed (in the case of a magic item). If detect magic is cast and directed at such a location, the spell indicates an aura strength of dim (even weaker than a faint aura). How long the aura lingers at this dim level depends on its original power:

\Table{}{lX}{
\tableheader Original Strength & \tableheader Duration of Lingering Aura\\
	Faint & 1d6 rounds\\
	Moderate & 1d6 minutes\\
	Strong & 1d6 $\times$ 10 minutes\\
	Overwhelming & 1d6 days\\
}

	Outsiders and elementals are not magical in themselves, but if they are summoned, the conjuration spell registers.

	Each round, you can turn to detect magic in a new area. The spell can penetrate barriers, but 30 centimeters of stone, 1 inch of common metal, a thin sheet of lead, or 1 meter of wood or dirt blocks it.

	\emph{Detect magic} can be made permanent with a \spell{permanency} spell.

}

\Spell{Detect Poison}{detect poison}
{Divination [Ritual]}
{
	\textbf{Level:}
	Ass 1, Clr 0, Drd 0, Rgr 1, Tmp 0, Wiz 0\\
	\textbf{Components:}
	V, S\\
	\textbf{Casting Time:}
	1 standard action\\
	\textbf{Range:}
	Close (7.5 m + 1.5 m/2 levels)\\
	\textbf{Target or Area:}
	One creature, one object, or a 1.5-m cube\\
	\textbf{Duration:}
	Instantaneous\\
	\textbf{Saving Throw:}
	None\\
	\textbf{Spell Resistance:}
	No\\
}
{
	You determine whether a creature, object, or area has been poisoned or is poisonous. You can determine the exact type of poison with a DC 20 Wisdom check. A character with the \skill{Craft} (alchemy) skill may try a DC 20 \skill{Craft} (alchemy) check if the Wisdom check fails, or may try the \skill{Craft} (alchemy) check prior to the Wisdom check.

	The spell can penetrate barriers, but 30 centimeters of stone, 2.5 centimeters of common metal, a thin sheet of lead, or 1 meter of wood or dirt blocks it.

}

\input{subsections/spells/detect-scrying.tex}
\input{subsections/spells/detect-secret-doors.tex}
\Spell{Detect Snares and Pits}{detect snares and pits}
{Divination}
{
	\textbf{Level:}
	Drd 1, Rgr 1\\
	\textbf{Components:}
	V, S\\
	\textbf{Casting Time:}
	1 standard action\\
	\textbf{Range:}
	18 m\\
	\textbf{Area:}
	Cone-shaped emanation\\
	\textbf{Duration:}
	Concentration, up to 10 min./level (D)\\
	\textbf{Saving Throw:}
	None\\
	\textbf{Spell Resistance:}
	No\\
}
{
	You can detect simple pits, deadfalls, and snares as well as mechanical traps constructed of natural materials. The spell does not detect complex traps, including trapdoor traps.

	Detect snares and pits does detect certain natural hazards---quicksand (a snare), a sinkhole (a pit), or unsafe walls of natural rock (a deadfall). However, it does not reveal other potentially dangerous conditions. The spell does not detect magic traps (except those that operate by pit, deadfall, or snaring; see the spell \spell{snare}), nor mechanically complex ones, nor those that have been rendered safe or inactive.

	The amount of information revealed depends on how long you study a particular area.

	\textit{1st Round:}
	Presence or absence of hazards.

	\textit{2nd Round:}
	Number of hazards and the location of each. If a hazard is outside your line of sight, then you discern its direction but not its exact location.

	\textit{Each Additional Round:}
	The general type and trigger for one particular hazard closely examined by you.

	Each round, you can turn to detect snares and pits in a new area. The spell can penetrate barriers, but 30 centimeters of stone, 2.5 centimeters of common metal, a thin sheet of lead, or 1 meter of wood or dirt blocks it.

}

\input{subsections/spells/detect-thoughts.tex}
\Spell{Detect Undead}{detect undead}
{Divination}
{
	\textbf{Level:}
	Clr 1, Drd 1, Tmp 1, Wiz 1\\
	\textbf{Components:}
	V, S, M/DF\\
	\textbf{Casting Time:}
	1 standard action\\
	\textbf{Range:}
	18 m\\
	\textbf{Area:}
	Cone-shaped emanation\\
	\textbf{Duration:}
	Concentration, up to 1 minute/ level (D)\\
	\textbf{Saving Throw:}
	None\\
	\textbf{Spell Resistance:}
	No\\
}
{
	You can detect the aura that surrounds undead creatures. The amount of information revealed depends on how long you study a particular area.

	\textit{1st Round}:
	Presence or absence of undead auras.

	\textit{2nd Round}:
	Number of undead auras in the area and the strength of the strongest undead aura present. If you are of good alignment, and the strongest undead aura's strength is overwhelming (see below), and the creature has HD of at least twice your character level, you are stunned for 1 round and the spell ends.

	\textit{3rd Round}:
	The strength and location of each undead aura. If an aura is outside your line of sight, then you discern its direction but not its exact location.

	\textit{Aura Strength}:
	The strength of an undead aura is determined by the HD of the undead creature, as given on the following table:

\Table{}{XX}{
\tableheader HD & \tableheader Strength\\
	1 or lower & Faint\\
	2--4 & Moderate\\
	5--10 & Strong\\
	11 or higher & Overwhelming\\
}
	\textit{Lingering Aura}:
	An undead aura lingers after its original source is destroyed. If detect undead is cast and directed at such a location, the spell indicates an aura strength of dim (even weaker than a faint aura). How long the aura lingers at this dim level depends on its original power:

\Table{}{XX}{
\tableheader Original Strength & \tableheader Duration of Lingering Aura\\
	Faint & 1d6 rounds\\
	Moderate & 1d6 minutes\\
	Strong & 1d6 $\times$ 10 minutes\\
	Overwhelming & 1d6 days\\
}
	Each round, you can turn to detect undead in a new area. The spell can penetrate barriers, but 30 centimeters of stone, 1 inch of common metal, a thin sheet of lead, or 1 meter of wood or dirt blocks it.

	\textit{Arcane Material Component}:
	A bit of earth from a grave.

}

\input{subsections/spells/dictum.tex}
\input{subsections/spells/dimensional-anchor.tex}
\Spell{Dimensional Lock}{dimensional lock}
{Abjuration}
{
	\textbf{Level:}
	Clr 8, Wiz 8\\
	\textbf{Components:}
	V, S\\
	\textbf{Casting Time:}
	1 standard action\\
	\textbf{Range:}
	Medium (30 m + 3 m/level)\\
	\textbf{Area:}
	6-m-radius emanation centered on a point in space\\
	\textbf{Duration:}
	One day/level\\
	\textbf{Saving Throw:}
	None\\
	\textbf{Spell Resistance:}
	Yes\\
}
{
	You create a shimmering emerald barrier that completely blocks extradimensional travel. Forms of movement barred include \spell{astral projection}, \spell{blink}, \spell{dimension door}, \spell{ethereal jaunt}, \spell{etherealness}, \spell{gate}, \spell{maze}, \spell{plane shift}, \spell{shadow walk}, \spell{teleport}, and similar spell-like or psionic abilities. Once \emph{dimensional lock} is in place, extradimensional travel into or out of the area is not possible.

	A \emph{dimensional lock} does not interfere with the movement of creatures already in ethereal or astral form when the spell is cast, nor does it block extradimensional perception or attack forms. Also, the spell does not prevent summoned creatures from disappearing at the end of a summoning spell.

}

\input{subsections/spells/dimension-door.tex}
\Spell{Diminish Plants}{diminish plants}
{Transmutation}
{
	\textbf{Level:}
	Drd 3, Rgr 3\\
	\textbf{Components:}
	V, S, DF\\
	\textbf{Casting Time:}
	1 standard action\\
	\textbf{Range:}
	See text\\
	\textbf{Target or Area:}
	See text\\
	\textbf{Duration:}
	Instantaneous\\
	\textbf{Saving Throw:}
	None\\
	\textbf{Spell Resistance:}
	No\\
}
{
	This spell has two versions.

	\textit{Prune Growth}:
	This version causes normal vegetation within long range (120 meters + 12 meters per level) to shrink to about one-third of their normal size, becoming untangled and less bushy. The affected vegetation appears to have been carefully pruned and trimmed.

	At your option, the area can be a 30-meter-radius circle, a 45-meter-radius semicircle, or a 60-meter-radius quarter-circle.

	You may also designate portions of the area that are not affected.

	\textit{Stunt Growth}:
	This version targets normal plants within a range of \onehalf mile, reducing their potential productivity over the course of the following year to one third below normal.

	\emph{Diminish plants} counters \spell{plant growth}.

	This spell has no effect on plant creatures.

}

\Spell{Discern Lies}{discern lies}
{Divination}
{
	\textbf{Level:}
	Clr 4, Pal 3\\
	\textbf{Components:}
	V, S, DF\\
	\textbf{Casting Time:}
	1 standard action\\
	\textbf{Range:}
	Close (7.5 m + 1.5 m/2 levels)\\
	\textbf{Targets:}
	One creature/level, no two of which can be more than 9 m apart\\
	\textbf{Duration:}
	Concentration, up to 1 round/level\\
	\textbf{Saving Throw:}
	Will negates\\
	\textbf{Spell Resistance:}
	No\\
}
{
	Each round, you concentrate on one subject, who must be within range. You know if the subject deliberately and knowingly speaks a lie by discerning disturbances in its aura caused by lying. The spell does not reveal the truth, uncover unintentional inaccuracies, or necessarily reveal evasions.

	Each round, you may concentrate on a different subject.

}

\input{subsections/spells/discern-location.tex}
\input{subsections/spells/disguise-self.tex}
\Spell{Disintegrate}{disintegrate}
{Transmutation}
{
	\textbf{Level:}
	Wiz 6\\
	\textbf{Components:}
	V, S, M/DF\\
	\textbf{Casting Time:}
	1 standard action\\
	\textbf{Range:}
	Medium (30 m + 3 m/level)\\
	\textbf{Effect:}
	Ray\\
	\textbf{Duration:}
	Instantaneous\\
	\textbf{Saving Throw:}
	Fortitude partial (object)\\
	\textbf{Spell Resistance:}
	Yes\\
}
{
	A thin, green ray springs from your pointing finger. You must make a successful ranged touch attack to hit. Any creature struck by the ray takes 2d6 points of damage per caster level (to a maximum of 40d6). Any creature reduced to 0 or fewer hit points by this spell is entirely disintegrated, leaving behind only a trace of fine dust. A disintegrated creature's equipment is unaffected.

	When used against an object, the ray simply disintegrates as much as one 3-meter cube of nonliving matter. Thus, the spell disintegrates only part of any very large object or structure targeted. The ray affects even objects constructed entirely of force, such as \spell{forceful hand} or a \spell{wall of force}, but not magical effects such as a \spell{globe of invulnerability} or an \spell{antimagic field}.

	A creature or object that makes a successful Fortitude save is partially affected, taking only 5d6 points of damage. If this damage reduces the creature or object to 0 or fewer hit points, it is entirely disintegrated.

	Only the first creature or object struck can be affected; that is, the ray affects only one target per casting.

	\textit{Arcane Material Component}:
	A lodestone and a pinch of dust.

}

\input{subsections/spells/dismissal.tex}
\Spell{Dispel Chaos}{dispel chaos}
{Abjuration [Lawful]}
{
	\textbf{Level:}
	Clr 5, Law 5\\
}
{
	This spell functions like \spell{dispel evil}, except that you are surrounded by constant, blue, lawful energy, and the spell affects chaotic creatures and spells rather than evil ones.

}

\input{subsections/spells/dispel-evil.tex}
\Spell{Dispel Good}{dispel good}
{Abjuration [Evil]}
{
	\textbf{Level:}
	Clr 5, Evil 5\\
}
{
	This spell functions like dispel evil, except that you are surrounded by dark, wavering, unholy energy, and the spell affects good creatures and spells rather than evil ones.

}

\Spell{Dispel Law}{dispel law}
{Abjuration [Chaotic]}
{
	\textbf{Level:}
	Chaos 5, Clr 5\\
}
{
	This spell functions like dispel evil, except that you are surrounded by flickering, yellow, chaotic energy, and the spell affects lawful creatures and spells rather than evil ones.

}

\Spell{Dispel Magic, Greater}{greater dispel magic}
{Abjuration}
{
	\textbf{Level:}
	Clr 6, Drd 6, Tmp 6, Wiz 6\\
}
{
	This spell functions like \spell{dispel magic}, except that the maximum caster level on your dispel check is +20 instead of +10.

	Additionally, \emph{greater dispel magic} has a chance to dispel any effect that \spell{remove curse} can remove, even if \spell{dispel magic} can't dispel that effect.

}

\input{subsections/spells/dispel-magic.tex}
\Spell{Displacement}{displacement}
{Illusion (Glamer)}
{
	\textbf{Level:}
	Wiz 3\\
	\textbf{Components:}
	V, M\\
	\textbf{Casting Time:}
	1 standard action\\
	\textbf{Range:}
	Touch\\
	\textbf{Target:}
	Creature touched\\
	\textbf{Duration:}
	1 round/level (D)\\
	\textbf{Saving Throw:}
	Will negates (harmless)\\
	\textbf{Spell Resistance:}
	Yes (harmless)\\
}
{
	The subject of this spell appears to be about 60 centimeters away from its true location. The creature benefits from a 50\% miss chance as if it had total concealment. However, unlike actual total concealment, \emph{displacement} does not prevent enemies from targeting the creature normally. The spell \spell{true seeing} reveals its true location.

	\textit{Material Component:}
	A small strip of leather twisted into a loop.

}

\input{subsections/spells/disrupting-weapon.tex}
\Spell{Disrupt Undead}{disrupt undead}
{Necromancy}
{
	\textbf{Level:}
	Glory 1, Wiz 0\\
	\textbf{Components:}
	V, S\\
	\textbf{Casting Time:}
	1 standard action\\
	\textbf{Range:}
	Close (7.5 m + 1.5 m/2 levels)\\
	\textbf{Effect:}
	Ray\\
	\textbf{Duration:}
	Instantaneous\\
	\textbf{Saving Throw:}
	None\\
	\textbf{Spell Resistance:}
	Yes\\
}
{
	You direct a ray of positive energy. You must make a ranged touch attack to hit, and if the ray hits an undead creature, it deals 1d6 points of damage to it.

}

\input{subsections/spells/divination.tex}
\input{subsections/spells/divine-favor.tex}
\input{subsections/spells/divine-power.tex}
\input{subsections/spells/dominate-animal.tex}
\Spell{Dominate Monster}{dominate monster}
{Enchantment (Compulsion) [Mind-Affecting]}
{
	\textbf{Level:}
	Charm 9, Wiz 9\\
	\textbf{Target:}
	One creature\\
}
{
	This spell functions like \spell{dominate person}, except that the spell is not restricted by creature type.

}

\input{subsections/spells/dominate-person.tex}
\Spell{Doom}{doom}
{Necromancy [Fear, Mind-Affecting]}
{
	\textbf{Level:}
	Clr 1, Tmp 1\\
	\textbf{Components:}
	V, S, DF\\
	\textbf{Casting Time:}
	1 standard action\\
	\textbf{Range:}
	Medium (30 m + 3 m/level)\\
	\textbf{Target:}
	One living creature\\
	\textbf{Duration:}
	1 min./level\\
	\textbf{Saving Throw:}
	Will negates\\
	\textbf{Spell Resistance:}
	Yes\\
}
{
	This spell fills a single subject with a feeling of horrible dread that causes it to become shaken.

}

\input{subsections/spells/dream.tex}
\Spell{Dweomer of Transference}{dweomer of transference}
{Evocation [Ritual]}
{
	\textbf{Level:}
	Clr 4, Tmp 4, Wiz 4\\
	\textbf{Components:}
	V, S\\
	\textbf{Casting Time:}
	1 minute\\
	\textbf{Range:}
	Close (7.5 m + 1.5 m/2 levels)\\
	\textbf{Target:}
	One willing psionic creature\\
	\textbf{Duration:}
	1 round/level\\
	\textbf{Saving Throw:}
	Will negates (harmless)\\
	\textbf{Spell Resistance:}
	Yes (harmless)\\
}
{
	With this spell, you form a radiating corona around the head of a psionic ally, then convert some of your spells into psionic power points. When you finish casting \emph{dweomer of transference}, a red-orange glow surrounds the psionic creature's head. For the duration of the spell, any spells cast at the subject don't have their usual effect, instead converting themselves harmlessly into psionic energy that the subject can use as energy for psionic powers. You can cast any spell you like at the subject, even area spells, effect spells, and spells for whom the subject would ordinarily not be a legitimate target. The spells don't do anything other than provide the subject with power points, but you must still cast them normally, obeying the component and range requirements listed in the description of each spell.

	For each spell you cast into the \emph{dweomer of transference}, the psionic creature gets temporary power points, according to the following table. The transference isn't perfectly efficient. The temporary power points acquired through a \emph{dweomer of transference} dissipate after 1 hour if they haven't already been spent.

	\Table{}{CC} {
		\tableheader Spell Level & \tableheader Power Points Acquired\\
		0 & 1\\
		1st & 4\\
		2nd & 9\\
		3rd & 14\\
		4th & 19\\
		5th & 24\\
		6th & 29\\
		7th & 34\\
		8th & 39\\
		9th & 44\\
	}
}

\Spell{Eagle's Splendor, Mass}{eagle's splendor, mass}
{Transmutation}
{
	\textbf{Level:}
	Clr 6, Wiz 6\\
	\textbf{Range:}
	Close (7.5 m + 1.5 m/2 levels)\\
	\textbf{Target:}
	One creature/level, no two of which can be more than 9 m apart\\
}
{
	This spell functions like eagle's splendor, except that it affects multiple creatures.

}

\input{subsections/spells/eagles-splendor.tex}
\input{subsections/spells/earthquake.tex}
\input{subsections/spells/elemental-swarm.tex}
\Spell{Endure Elements}{endure elements}
{Abjuration}
{
	\textbf{Level:}
	Clr 1, Drd 1, Pal 1, Rgr 1, Wiz 1, Sun 1\\
	\textbf{Components:}
	V, S\\
	\textbf{Casting Time:}
	1 standard action\\
	\textbf{Range:}
	Touch\\
	\textbf{Target:}
	Creature touched\\
	\textbf{Duration:}
	24 hours\\
	\textbf{Saving Throw:}
	Will negates (harmless)\\
	\textbf{Spell Resistance:}
	Yes (harmless)\\
}
{
	A creature protected by endure elements suffers no harm from being in a hot or cold environment. It can exist comfortably in conditions between $-50$ and 140 degrees Fahrenheit without having to make Fortitude saves). The creature's equipment is likewise protected.

	Endure elements doesn't provide any protection from fire or cold damage, nor does it protect against other environmental hazards such as smoke, lack of air, and so forth.

}

\Spell{Energy Drain}{energy drain}
{Necromancy}
{
	\textbf{Level:}
	Clr 9, Wiz 9\\
	\textbf{Saving Throw:}
	Fortitude partial; see text for enervation\\
}
{
	This spell functions like enervation, except that the creature struck gains 2d4 negative levels, and the negative levels last longer.

	There is no saving throw to avoid gaining the negative levels, but 24 hours after gaining them, the subject must make a Fortitude saving throw (DC = energy drain spell's save DC) for each negative level. If the save succeeds, that negative level is removed. If it fails, the negative level also goes away, but one of the subject's character levels is permanently drained.

	An undead creature struck by the ray gains 2d4 $\times$ 5 temporary hit points for 1 hour.

}

\input{subsections/spells/enervation.tex}
\Spell{Enlarge Person, Mass}{mass enlarge person}
{Transmutation}
{
	\textbf{Level:}
	Wiz 4\\
	\textbf{Target:}
	One humanoid creature/level, no two of which can be more than 9 m apart\\
}
{
	This spell functions like \spell{enlarge person}, except that it affects multiple creatures.

}

\Spell{Enlarge Person}{enlarge person}
{Transmutation}
{
	\textbf{Level:}
	Wiz 1, Strength 1\\
	\textbf{Components:}
	V, S, M\\
	\textbf{Casting Time:}
	1 round\\
	\textbf{Range:}
	Close (7.5 m + 1.5 m/2 levels)\\
	\textbf{Target:}
	One humanoid creature\\
	\textbf{Duration:}
	1 min./level (D)\\
	\textbf{Saving Throw:}
	Fortitude negates\\
	\textbf{Spell Resistance:}
	Yes\\
}
{
	This spell causes instant growth of a humanoid creature, doubling its height and multiplying its weight by 8. This increase changes the creature's size category to the next larger one. The target gains a +2 size bonus to Strength, a $-2$ size penalty to Dexterity (to a minimum of 1), and a -1 penalty on attack rolls and AC due to its increased size.

	A humanoid creature whose size increases to Large has a space of 3 meters and a natural reach of 3 meters. This spell does not change the target's speed.

	If insufficient room is available for the desired growth, the creature attains the maximum possible size and may make a Strength check (using its increased Strength) to burst any enclosures in the process. If it fails, it is constrained without harm by the materials enclosing it---the spell cannot be used to crush a creature by increasing its size.

	All equipment worn or carried by a creature is similarly enlarged by the spell. Melee and projectile weapons affected by this spell deal more damage. Other magical properties are not affected by this spell. Any enlarged item that leaves an enlarged creature's possession (including a projectile or thrown weapon) instantly returns to its normal size. This means that thrown weapons deal their normal damage, and projectiles deal damage based on the size of the weapon that fired them. Magical properties of enlarged items are not increased by this spell.

	Multiple magical effects that increase size do not stack,.

	\emph{Enlarge person} counters and dispels \spell{reduce person}.

	\emph{Enlarge person} can be made permanent with a \spell{permanency} spell.

	\textit{Material Component}:
	A pinch of powdered iron.

}

\Spell{Entangle}{entangle}
{Transmutation}
{
	\textbf{Level:}
	Drd 1, Growth 1, Plant 1, Rgr 1\\
	\textbf{Components:}
	V, S, DF\\
	\textbf{Casting Time:}
	1 standard action\\
	\textbf{Range:}
	Long (120 m + 12 m/level)\\
	\textbf{Area:}
	Plants in a 12-m-radius spread\\
	\textbf{Duration:}
	1 min./level (D)\\
	\textbf{Saving Throw:}
	Reflex partial; see text\\
	\textbf{Spell Resistance:}
	No\\
}
{
	Grasses, weeds, bushes, and even trees wrap, twist, and entwine about creatures in the area or those that enter the area, holding them fast and causing them to become entangled. The creature can break free and move half its normal speed by using a full-round action to make a DC 20 Strength check or a DC 20 \skill{Escape Artist} check. A creature that succeeds on a Reflex save is not entangled but can still move at only half speed through the area. Each round on your turn, the plants once again attempt to entangle all creatures that have avoided or escaped entanglement.

	\textit{Note:} The effects of the spell may be altered somewhat, based on the nature of the entangling plants.

}

\input{subsections/spells/enthrall.tex}
\input{subsections/spells/entropic-shield.tex}
\Spell{Erase}{erase}
{Transmutation}
{
	\textbf{Level:}
	Wiz 1\\
	\textbf{Components:}
	V, S\\
	\textbf{Casting Time:}
	1 standard action\\
	\textbf{Range:}
	Close (7.5 m + 1.5 m/2 levels)\\
	\textbf{Target:}
	One scroll or two pages\\
	\textbf{Duration:}
	Instantaneous\\
	\textbf{Saving Throw:}
	See text\\
	\textbf{Spell Resistance:}
	No\\
}
{
	Erase removes writings of either magical or mundane nature from a scroll or from one or two pages of paper, parchment, or similar surfaces. With this spell, you can remove explosive runes, a glyph of warding, a sepia snake sigil, or an arcane mark, but not illusory script or a symbol spell. Nonmagical writing is automatically erased if you touch it and no one else is holding it. Otherwise, the chance of erasing nonmagical writing is 90\%.

	Magic writing must be touched to be erased, and you also must succeed on a caster level check (1d20 + caster level) against DC 15. (A natural 1 or 2 is always a failure on this check.) If you fail to erase explosive runes, a glyph of warding, or a sepia snake sigil, you accidentally activate that writing instead.

}

\input{subsections/spells/ethereal-jaunt.tex}
\Spell{Etherealness}{etherealness}
{Transmutation}
{
	\textbf{Level:}
	Air 9, Clr 9, Wiz 9\\
	\textbf{Range:}
	Touch; see text\\
	\textbf{Targets:}
	You and one other touched creature per three levels\\
	\textbf{Duration:}
	1 min./level (D)\\
	\textbf{Spell Resistance:}
	Yes\\
}
{
	This spell functions like \spell{ethereal jaunt}, except that you and other willing creatures joined by linked hands (along with their equipment) become ethereal. Besides yourself, you can bring one creature per three caster levels to the Ethereal Plane. Once ethereal, the subjects need not stay together.

	When the spell expires, all affected creatures on the Ethereal Plane return to material existence.

}

\Spell{Expeditious Retreat}{expeditious retreat}
{Transmutation}
{
	\textbf{Level:}
	Wiz 1\\
	\textbf{Components:}
	V, S\\
	\textbf{Casting Time:}
	1 standard action\\
	\textbf{Range:}
	Personal\\
	\textbf{Target:}
	You\\
	\textbf{Duration:}
	1 min./level (D)\\
}
{
	This spell increases your base land speed by 9 meters. (This adjustment is treated as an enhancement bonus.) There is no effect on other modes of movement, such as burrow, climb, fly, or swim. As with any effect that increases your speed, this spell affects your jumping distance (see the Jump skill).

}

\Spell{Explosive Runes}{explosive runes}
{Abjuration [Force]}
{
	\textbf{Level:}
	Wiz 3\\
	\textbf{Components:}
	V, S\\
	\textbf{Casting Time:}
	1 standard action\\
	\textbf{Range:}
	Touch\\
	\textbf{Target:}
	One touched object weighing no more than 10 lb.\\
	\textbf{Duration:}
	Permanent until discharged (D)\\
	\textbf{Saving Throw:}
	See text\\
	\textbf{Spell Resistance:}
	Yes\\
}
{
	You trace these mystic runes upon a book, map, scroll, or similar object bearing written information. The runes detonate when read, dealing 6d6 points of force damage. Anyone next to the runes (close enough to read them) takes the full damage with no saving throw; any other creature within 3 meters of the runes is entitled to a Reflex save for half damage. The object on which the runes were written also takes full damage (no saving throw).

	You and any characters you specifically instruct can read the protected writing without triggering the runes. Likewise, you can remove the runes whenever desired. Another creature can remove them with a successful dispel magic or erase spell, but attempting to dispel or erase the runes and failing to do so triggers the explosion.

	\textbf{Note: Magic traps such as explosive runes are hard to detect and disable. A rogue (only) can use the Search skill to find the runes and Disable Device to thwart them. The DC in each case is 25 + spell level, or 28 for explosive runes.}

}

\Spell{Eyebite}{eyebite}
{Necromancy [Evil]}
{
	\textbf{Level:}
	Wiz 6\\
	\textbf{Components:}
	V, S\\
	\textbf{Casting Time:}
	1 standard action\\
	\textbf{Range:}
	Close (7.5 m + 1.5 m/2 levels)\\
	\textbf{Target:}
	One living creature\\
	\textbf{Duration:}
	1 round per three levels; see text\\
	\textbf{Saving Throw:}
	Fortitude negates\\
	\textbf{Spell Resistance:}
	Yes\\
}
{
	Each round, you may target a single living creature, striking it with waves of evil power. Depending on the target's HD, this attack has as many as three effects.

\Table{}{XX}{
\tableheader HD & \tableheader Effect\\
	10 or more & Sickened\\
	5--9 & Panicked, sickened\\
	4 or less & Comatose, panicked, sickened\\
}

	The effects are cumulative and concurrent.

	\textit{Sickened}:
	Sudden pain and fever sweeps over the subject's body. A sickened creature takes a $-2$ penalty on attack rolls, weapon damage rolls, saving throws, skill checks, and ability checks. A creature affected by this spell remains sickened for 10 minutes per caster level. The effects cannot be negated by a remove disease or heal spell, but a remove curse is effective.

	\textit{Panicked}:
	The subject becomes panicked for 1d4 rounds. Even after the panic ends, the creature remains shaken for 10 minutes per caster level, and it automatically becomes panicked again if it comes within sight of you during that time. This is a fear effect.

	\textit{Comatose}:
	The subject falls into a catatonic coma for 10 minutes per caster level. During this time, it cannot be awakened by any means short of dispelling the effect. This is not a sleep effect, and thus elves are not immune to it.

	The spell lasts for 1 round per three caster levels. You must spend a move action each round after the first to target a foe.

}

\Spell{Fabricate}{fabricate}
{Transmutation}
{
	\textbf{Level:}
	Wiz 5\\
	\textbf{Components:}
	V, S, M\\
	\textbf{Casting Time:}
	See text\\
	\textbf{Range:}
	Close (7.5 m + 1.5 m/2 levels)\\
	\textbf{Target:}
	Up to 0.3 m$^3$/level; see text\\
	\textbf{Duration:}
	Instantaneous\\
	\textbf{Saving Throw:}
	None\\
	\textbf{Spell Resistance:}
	No\\
}
{
	You convert material of one sort into a product that is of the same material. Creatures or magic items cannot be created or transmuted by the \emph{fabricate} spell. The quality of items made by this spell is commensurate with the quality of material used as the basis for the new fabrication. If you work with a mineral, the target is reduced to 0.03 cubic meter per level instead of 0.3 cubic meter.

	You must make an appropriate \skill{Craft} check to fabricate articles requiring a high degree of craftsmanship.

	Casting requires 1 round per 0.3 cubic meter (or 0.03 cubic meter) of material to be affected by the spell.

	\textit{Material Component}:
	The original material, which costs the same amount as the raw materials required to craft the item to be created.

}

\Spell{Faerie Fire}{faerie fire}
{Evocation [Light]}
{
	\textbf{Level:}
	Cleasing 1, Drd 1, Sun 1\\
	\textbf{Components:}
	V, S, DF\\
	\textbf{Casting Time:}
	1 standard action\\
	\textbf{Range:}
	Long (120 m + 12 m/level)\\
	\textbf{Area:}
	Creatures and objects within a 1.5-m-radius burst\\
	\textbf{Duration:}
	1 min./level (D)\\
	\textbf{Saving Throw:}
	None\\
	\textbf{Spell Resistance:}
	Yes\\
}
{
	A pale glow surrounds and outlines the subjects. Outlined subjects shed light as candles. Outlined creatures do not benefit from the concealment normally provided by \spell{darkness} (though a 2nd-level or higher magical darkness effect functions normally), \spell{blur}, \spell{displacement}, \spell{invisibility}, or similar effects. The light is too dim to have any special effect on undead or dark-dwelling creatures vulnerable to light. The \emph{faerie fire} can be blue, green, or violet, according to your choice at the time of casting. The \emph{faerie fire} does not cause any harm to the objects or creatures thus outlined.

}

\input{subsections/spells/false-life.tex}
\input{subsections/spells/false-vision.tex}
\Spell{Fear}{fear}
{Necromancy [Fear, Mind-Affecting]}
{
	\textbf{Level:}
	Wiz 4\\
	\textbf{Components:}
	V, S, M\\
	\textbf{Casting Time:}
	1 standard action\\
	\textbf{Range:}
	9 m\\
	\textbf{Area:}
	Cone-shaped burst\\
	\textbf{Duration:}
	1 round/level or 1 round; see text\\
	\textbf{Saving Throw:}
	Will partial\\
	\textbf{Spell Resistance:}
	Yes\\
}
{
	An invisible cone of terror causes each living creature in the area to become panicked unless it succeeds on a Will save. If cornered, a panicked creature begins cowering. If the Will save succeeds, the creature is shaken for 1 round.

	\textit{Material Component}:
	Either the heart of a hen or a white feather.

}

\Spell{Feather Fall}{feather fall}
{Transmutation}
{
	\textbf{Level:}
	Ass 1, Forecasting 1, Wiz 1\\
	\textbf{Components:}
	V\\
	\textbf{Casting Time:}
	1 immediate action\\
	\textbf{Range:}
	Close (7.5 m + 1.5 m/2 levels)\\
	\textbf{Targets:}
	One Medium or smaller freefalling object or creature/level, no two of which may be more than 6 m apart\\
	\textbf{Duration:}
	Until landing or 1 round/level\\
	\textbf{Saving Throw:}
	Will negates (harmless) or Will negates (object)\\
	\textbf{Spell Resistance:}
	Yes (object)\\
}
{
	The affected creatures or objects fall slowly. \emph{Feather fall} instantly changes the rate at which the targets fall to a mere 18 meters per round (equivalent to the end of a fall from a few feet), and the subjects take no damage upon landing while the spell is in effect. However, when the spell duration expires, a normal rate of falling resumes.

	The spell affects one or more Medium or smaller creatures (including gear and carried objects up to each creature's maximum load) or objects, or the equivalent in larger creatures: A Large creature or object counts as two Medium creatures or objects, a Huge creature or object counts as two Large creatures or objects, and so forth.

	You can cast this spell with an instant utterance, quickly enough to save yourself if you unexpectedly fall. Casting the spell is a immediate action, allowing you to cast this spell even when it isn't your turn.

	This spell has no special effect on ranged weapons unless they are falling quite a distance. If the spell is cast on a falling item the object does half normal damage based on its weight, with no bonus for the height of the drop.

	\emph{Feather fall} works only upon free-falling objects. It does not affect a sword blow or a charging or flying creature.

}

\Spell{Feeblemind}{feeblemind}
{Enchantment (Compulsion) [Mind-Affecting]}
{
	\textbf{Level:}
	Wiz 5\\
	\textbf{Components:}
	V, S, M\\
	\textbf{Casting Time:}
	1 standard action\\
	\textbf{Range:}
	Medium (30 m + 3 m/level)\\
	\textbf{Target:}
	One creature\\
	\textbf{Duration:}
	Instantaneous\\
	\textbf{Saving Throw:}
	Will negates; see text\\
	\textbf{Spell Resistance:}
	Yes\\
}
{
	If the target creature fails a Will saving throw, its Intelligence and Charisma scores each drop to 1. The affected creature is unable to use Intelligence- or Charisma-based skills, cast spells, understand language, or communicate coherently. Still, it knows who its friends are and can follow them and even protect them. The subject remains in this state until a \spell{heal}, \spell{limited wish}, \spell{miracle}, or \spell{wish} spell is used to cancel the effect of the \emph{feeblemind}. A creature that can cast arcane spells, such as an assassin or a wizard, takes a $-4$ penalty on its saving throw.

	\textit{Material Component:}
	A handful of clay, crystal, glass, or mineral spheres.

}

\input{subsections/spells/find-the-path.tex}
\input{subsections/spells/find-traps.tex}
\Spell{Finger of Death}{finger of death}
{Necromancy [Death]}
{
	\textbf{Level:}
	Drd 8, Tmp 8, Wiz 7\\
	\textbf{Components:}
	V, S\\
	\textbf{Casting Time:}
	1 standard action\\
	\textbf{Range:}
	Close (7.5 m + 1.5 m/2 levels)\\
	\textbf{Target:}
	One living creature\\
	\textbf{Duration:}
	Instantaneous\\
	\textbf{Saving Throw:}
	Fortitude partial\\
	\textbf{Spell Resistance:}
	Yes\\
}
{
	You can slay any one living creature within range. The target is entitled to a Fortitude saving throw to survive the attack. If the save is successful, the creature instead takes 3d6 points of damage +1 point per caster level (maximum +25).

	The subject might die from damage even if it succeeds on its saving throw.

}

\Spell{Fireball}{fireball}
{Evocation [Fire]}
{
	\textbf{Level:}
	Wiz 3\\
	\textbf{Components:}
	V, S, M\\
	\textbf{Casting Time:}
	1 standard action\\
	\textbf{Range:}
	Long (120 m + 12 m/level)\\
	\textbf{Area:}
	6-m-radius spread\\
	\textbf{Duration:}
	Instantaneous\\
	\textbf{Saving Throw:}
	Reflex half\\
	\textbf{Spell Resistance:}
	Yes\\
}
{
	A \emph{fireball} spell is an explosion of flame that detonates with a low roar and deals 1d6 points of fire damage per caster level (maximum 10d6) to every creature within the area. Unattended objects also take this damage. The explosion creates almost no pressure.

	You point your finger and determine the range (distance and height) at which the \emph{fireball} is to burst. A glowing, pea-sized bead streaks from the pointing digit and, unless it impacts upon a material body or solid barrier prior to attaining the prescribed range, blossoms into the \emph{fireball} at that point. (An early impact results in an early detonation.) If you attempt to send the bead through a narrow passage, such as through an arrow slit, you must ``hit'' the opening with a ranged touch attack, or else the bead strikes the barrier and detonates prematurely.

	The \emph{fireball} sets fire to combustibles and damages objects in the area. It can melt metals with low melting points, such as lead, gold, copper, silver, and bronze. If the damage caused to an interposing barrier shatters or breaks through it, the \emph{fireball} may continue beyond the barrier if the area permits; otherwise it stops at the barrier just as any other spell effect does.

	\textit{Material Component}:
	A tiny ball of bat guano and sulfur.

}

\input{subsections/spells/fire-seeds.tex}
\input{subsections/spells/fire-shield.tex}
\Spell{Fire Storm}{fire storm}
{Evocation [Fire]}
{
	\textbf{Level:}
	Drd 7, Fire 7, Magma 7\\
	\textbf{Components:}
	V, S\\
	\textbf{Casting Time:}
	1 round\\
	\textbf{Range:}
	Medium (30 m + 3 m/level)\\
	\textbf{Area:}
	Two 3-m cubes per level (S)\\
	\textbf{Duration:}
	Instantaneous\\
	\textbf{Saving Throw:}
	Reflex half\\
	\textbf{Spell Resistance:}
	Yes\\
}
{
	When a fire storm spell is cast, the whole area is shot through with sheets of roaring flame. The raging flames do not harm natural vegetation, ground cover, and any plant creatures in the area that you wish to exclude from damage. Any other creature within the area takes 1d6 points of fire damage per caster level (maximum 20d6).

}

\Spell{Fire Trap}{fire trap}
{Abjuration [Fire, Ritual]}
{
	\textbf{Level:}
	Cleansing 2, Drd 2, Rng 2, Wiz 4\\
	\textbf{Components:}
	V, S, M\\
	\textbf{Casting Time:}
	10 minutes\\
	\textbf{Range:}
	Touch\\
	\textbf{Target:}
	Object touched\\
	\textbf{Duration:}
	Permanent until discharged (D)\\
	\textbf{Saving Throw:}
	Reflex half; see text\\
	\textbf{Spell Resistance:}
	Yes\\
}
{
	\emph{Fire trap} creates a fiery explosion when an intruder opens the item that the trap protects. A \emph{fire trap} can ward any object that can be opened and closed.

	When casting \emph{fire trap}, you select a point on the object as the spell's center. When someone other than you opens the object, a fiery explosion fills the area within a 1.5-meter radius around the spell's center. The flames deal 1d4 points of fire damage +1 point per caster level (maximum +20). The item protected by the trap is not harmed by this explosion.

	A \emph{fire trapped} item cannot have a second closure or warding spell placed on it.

	A \spell{knock} spell does not bypass a \emph{fire trap}.  An unsuccessful \spell{dispel magic} spell does not detonate the spell.

	Underwater, this ward deals half damage and creates a large cloud of steam.

	You can use the \emph{fire trapped} object without discharging it, as can any individual to whom the object was specifically attuned when cast. Attuning a \emph{fire trapped} object to an individual usually involves setting a password that you can share with friends.

	\textit{Note:} Magic traps such as \emph{fire trap} are hard to detect and disable. A rogue (only) can use the \skill{Search} skill to find a \emph{fire trap} and \skill{Disable Device} to thwart it. The DC in each case is 25 + spell level (DC 27 for a druid's fire trap or DC 29 for the arcane version).

	\textit{Material Component:}
	A half-pound of gold dust (cost 25 gp) sprinkled on the warded object.

}

\input{subsections/spells/flame-arrow.tex}
\input{subsections/spells/flame-blade.tex}
\Spell{Flame Strike}{flame strike}
{Evocation [Fire]}
{
	\textbf{Level:}
	Drd 4, Sun 5, War 5, Wrath 5\\
	\textbf{Components:}
	V, S, DF\\
	\textbf{Casting Time:}
	1 standard action\\
	\textbf{Range:}
	Medium (30 m + 3 m/level)\\
	\textbf{Area:}
	Cylinder (3-m radius, 12 m high)\\
	\textbf{Duration:}
	Instantaneous\\
	\textbf{Saving Throw:}
	Reflex half\\
	\textbf{Spell Resistance:}
	Yes\\
}
{
	A \emph{flame strike} produces a vertical column of divine fire roaring downward. The spell deals 1d6 points of damage per caster level (maximum 15d6). Half the damage is fire damage, but the other half results directly from divine power and is therefore not subject to being reduced by resistance to fire-based attacks.

}

\Spell{Flaming Sphere}{flaming sphere}
{Evocation [Fire]}
{
	\textbf{Level:}
	Drd 2, Wiz 2\\
	\textbf{Components:}
	V, S, M/DF\\
	\textbf{Casting Time:}
	1 standard action\\
	\textbf{Range:}
	Medium (30 m + 3 m/level)\\
	\textbf{Effect:}
	1.5-m-diameter sphere\\
	\textbf{Duration:}
	1 round/level\\
	\textbf{Saving Throw:}
	Reflex negates\\
	\textbf{Spell Resistance:}
	Yes\\
}
{
	A burning globe of fire rolls in whichever direction you point and burns those it strikes. It moves 9 meters per round. As part of this movement, it can ascend or jump up to 9 meters to strike a target. If it enters a space with a creature, it stops moving for the round and deals 2d6 points of fire damage to that creature, though a successful Reflex save negates that damage. A \emph{flaming sphere} rolls over barriers less than 1.2 meter tall. It ignites flammable substances it touches and illuminates the same area as a torch would.

	The sphere moves as long as you actively direct it (a move action for you); otherwise, it merely stays at rest and burns. It can be extinguished by any means that would put out a normal fire of its size. The surface of the sphere has a spongy, yielding consistency and so does not cause damage except by its flame. It cannot push aside unwilling creatures or batter down large obstacles. A \emph{flaming sphere} winks out if it exceeds the spell's range.

	\textit{Arcane Material Component}:
	A bit of tallow, a pinch of brimstone, and a dusting of powdered iron.

}

\Spell{Flare}{flare}
{Evocation [Light]}
{
	\textbf{Level:}
	Drd 0, Wiz 0\\
	\textbf{Components:}
	V\\
	\textbf{Casting Time:}
	1 standard action\\
	\textbf{Range:}
	Close (7.5 m + 1.5 m/2 levels)\\
	\textbf{Effect:}
	Burst of light\\
	\textbf{Duration:}
	Instantaneous\\
	\textbf{Saving Throw:}
	Fortitude negates\\
	\textbf{Spell Resistance:}
	Yes\\
}
{
	This cantrip creates a burst of light. If you cause the light to burst directly in front of a single creature, that creature is dazzled for 1 minute unless it makes a successful Fortitude save. Sightless creatures, as well as creatures already dazzled, are not affected by flare.

}

\input{subsections/spells/flesh-to-stone.tex}
\Spell{Floating Disk}{floating disk}
{Evocation [Force, Ritual]}
{
	\textbf{Level:}
	Wiz 1\\
	\textbf{Components:}
	V, S, M\\
	\textbf{Casting Time:}
	1 standard action\\
	\textbf{Range:}
	Close (7.5 m + 1.5 m/2 levels)\\
	\textbf{Effect:}
	1-m-diameter disk of force\\
	\textbf{Duration:}
	1 hour/level\\
	\textbf{Saving Throw:}
	None\\
	\textbf{Spell Resistance:}
	No\\
}
{
	You create a slightly concave, circular plane of force that follows you about and carries loads for you. The disk is 1 meter in diameter and 2.5 centimeters deep at its center. It can hold 50 kilograms of weight per caster level. (If used to transport a liquid, its capacity is 9 liters.) The disk floats approximately 1 meter above the ground at all times and remains level. It floats along horizontally within spell range and will accompany you at a rate of no more than your normal speed each round. If not otherwise directed, it maintains a constant interval of 1.5 meter between itself and you. The disk winks out of existence when the spell duration expires. The disk also winks out if you move beyond range or try to take the disk more than 1 meter away from the surface beneath it. When the disk winks out, whatever it was supporting falls to the surface beneath it.

	\textit{Material Component:}
	A drop of mercury.

}

\Spell{Fly}{fly}
{Transmutation}
{
	\textbf{Level:}
	Freedom 3, Wiz 3\\
	\textbf{Components:}
	V, S, F/DF\\
	\textbf{Casting Time:}
	1 standard action\\
	\textbf{Range:}
	Touch\\
	\textbf{Target:}
	Creature touched\\
	\textbf{Duration:}
	1 min./level\\
	\textbf{Saving Throw:}
	Will negates (harmless)\\
	\textbf{Spell Resistance:}
	Yes (harmless)\\
}
{
	The subject can fly at a speed of 18 meters (or 12 meters if it wears medium or heavy armor, or if it carries a medium or heavy load). It can ascend at half speed and descend at double speed, and its maneuverability is good. Using a \emph{fly} spell requires only as much concentration as walking, so the subject can attack or cast spells normally. The subject of a \emph{fly} spell can charge but not run, and it cannot carry aloft more weight than its maximum load, plus any armor it wears.

	Should the spell duration expire while the subject is still aloft, the magic fails slowly. The subject floats downward 18 meters per round for 1d6 rounds. If it reaches the ground in that amount of time, it lands safely. If not, it falls the rest of the distance, taking 1d6 points of damage per 3 meters of fall. Since dispelling a spell effectively ends it, the subject also descends in this way if the \emph{fly} spell is dispelled, but not if it is negated by an \spell{antimagic field}.

	\textit{Arcane Focus}:
	A wing feather from any bird.

}

\Spell{Fog Cloud}{fog cloud}
{Conjuration (Creation)}
{
	\textbf{Level:}
	Drd 2, Rain 2, Wiz 2\\
	\textbf{Components:}
	V, S\\
	\textbf{Casting Time:}
	1 standard action\\
	\textbf{Range:}
	Medium (30 m + 3 m level)\\
	\textbf{Effect:}
	Fog spreads in 6-m radius, 6 m high\\
	\textbf{Duration:}
	10 min./level\\
	\textbf{Saving Throw:}
	None\\
	\textbf{Spell Resistance:}
	No\\
}
{
	A bank of fog billows out from the point you designate. The fog obscures all sight, including darkvision, beyond 1.5 meter. A creature within 1.5 meter has concealment (attacks have a 20\% miss chance). Creatures farther away have total concealment (50\% miss chance, and the attacker can't use sight to locate the target).

	A moderate wind (11+ mph) disperses the fog in 4 rounds; a strong wind (21+ mph) disperses the fog in 1 round.

	The spell does not function underwater.

}

\Spell{Forbiddance}{forbiddance}
{Abjuration}
{
	\textbf{Level:}
	Clr 6, Tmp 6\\
	\textbf{Components:}
	V, S, M, DF\\
	\textbf{Casting Time:}
	6 rounds\\
	\textbf{Range:}
	Medium (30 m + 3 m/level)\\
	\textbf{Area:}
	18-m cube/level (S)\\
	\textbf{Duration:}
	Permanent\\
	\textbf{Saving Throw:}
	See text\\
	\textbf{Spell Resistance:}
	Yes\\
}
{
	\emph{Forbiddance} seals an area against all planar travel into or within it. This includes all teleportation spells (such as \spell{dimension door} and \spell{teleport}), \spell{plane shifting}, \spell{astral travel}, \spell{ethereal travel}, and all summoning spells. Such effects simply fail automatically.

	In addition, it damages entering creatures whose alignments are different from yours. The effect on those attempting to enter the warded area is based on their alignment relative to yours (see below). A creature inside the area when the spell is cast takes no damage unless it exits the area and attempts to reenter, at which time it is affected as normal.

	\textit{Alignments identical}:
	No effect. The creature may enter the area freely (although not by planar travel).

	\textit{Alignments different with respect to either law/chaos or good/evil}:
	The creature takes 6d6 points of damage. A successful Will save halves the damage, and spell resistance applies.

	\textit{Alignments different with respect to both law/chaos and good/evil}:
	The creature takes 12d6 points of damage. A successful Will save halves the damage, and spell resistance applies.

	At your option, the abjuration can include a password, in which case creatures of alignments different from yours can avoid the damage by speaking the password as they enter the area. You must select this option (and the password) at the time of casting.

	\spell{Dispel Magic} does not dispel a \emph{forbiddance} effect unless the dispeller's level is at least as high as your caster level.

	You can't have multiple overlapping \emph{forbiddance} effects. In such a case, the more recent effect stops at the boundary of the older effect.

	\textit{Material Component}:
	A sprinkling of holy water and rare incenses worth at least 1,500 cp, plus 1,500 cp per 18-meter cube. If a password is desired, this requires the burning of additional rare incenses worth at least 1,000 cp, plus 1,000 cp per 18-meter cube.

}

\Spell{Forcecage}{forcecage}
{Evocation [Force]}
{
	\textbf{Level:}
	Wiz 7\\
	\textbf{Components:}
	V, S, M\\
	\textbf{Casting Time:}
	1 standard action\\
	\textbf{Range:}
	Close (7.5 m + 1.5 m/2 levels)\\
	\textbf{Area:}
	Barred cage (6-m cube) or windowless cell (10-ft. cube)\\
	\textbf{Duration:}
	2 hours/level (D)\\
	\textbf{Saving Throw:}
	None\\
	\textbf{Spell Resistance:}
	No\\
}
{
	This powerful spell brings into being an immobile, invisible cubical prison composed of either bars of force or solid walls of force (your choice).

	Creatures within the area are caught and contained unless they are too big to fit inside, in which case the spell automatically fails. Teleportation and other forms of astral travel provide a means of escape, but the force walls or bars extend into the Ethereal Plane, blocking ethereal travel.

	Like a wall of force spell, a forcecage resists dispel magic, but it is vulnerable to a disintegrate spell, and it can be destroyed by a sphere of annihilation or a rod of cancellation.

	\textit{Barred Cage}:
	This version of the spell produces a 6-meter cube made of bands of force (similar to a wall of force spell) for bars. The bands are a half-inch wide, with half-inch gaps between them. Any creature capable of passing through such a small space can escape; others are confined. You can't attack a creature in a barred cage with a weapon unless the weapon can fit between the gaps. Even against such weapons (including arrows and similar ranged attacks), a creature in the barred cage has cover. All spells and breath weapons can pass through the gaps in the bars.

	\textit{Windowless Cell}:
	This version of the spell produces a 3-meter cube with no way in and no way out. Solid walls of force form its six sides.

	\textit{Material Component}:
	Ruby dust worth 1,500 gp, which is tossed into the air and disappears when you cast the spell.

}

\input{subsections/spells/forceful-hand.tex}
\input{subsections/spells/foresight.tex}
\Spell{Fox's Cunning, Mass}{mass fox's cunning}
{Transmutation}
{
	\textbf{Level:}
	Wiz 6\\
	\textbf{Range:}
	Close (7.5 m + 1.5 m/2 levels)\\
	\textbf{Target:}
	One creature/level, no two of which can be more than 9 m apart\\
}
{
	This spell functions like \spell{fox's cunning}, except that it affects multiple creatures.

}

\input{subsections/spells/foxs-cunning.tex}
\Spell{Freedom of Movement}{freedom of movement}
{Abjuration}
{
	\textbf{Level:}
	Ass 4, Clr 4, Drd 4, Freedom 4, Growth 4, Purity 4, Rgr 4, Tmp 4\\
	\textbf{Components:}
	V, S, M, DF\\
	\textbf{Casting Time:}
	1 standard action\\
	\textbf{Range:}
	Personal or touch\\
	\textbf{Target:}
	You or creature touched\\
	\textbf{Duration:}
	10 min./level\\
	\textbf{Saving Throw:}
	Will negates (harmless)\\
	\textbf{Spell Resistance:}
	Yes (harmless)\\
}
{
	This spell enables you or a creature you touch to move and attack normally for the duration of the spell, even under the influence of magic that usually impedes movement, such as paralysis, \spell{solid fog}, \spell{slow}, and \spell{web}. The subject automatically succeeds on any grapple check made to resist a grapple attempt, as well as on grapple checks or \skill{Escape Artist} checks made to escape a grapple or a pin.

	The spell also allows the subject to move and attack normally while underwater, even with slashing weapons such as axes and swords or with bludgeoning weapons such as flails, hammers, and maces, provided that the weapon is wielded in the hand rather than hurled. The \emph{freedom of movement} spell does not, however, allow water breathing.

	\textit{Material Component:}
	A leather thong, bound around the arm or a similar appendage.

}

\Spell{Freedom}{freedom}
{Abjuration}
{
	\textbf{Level:}
	Wiz 9\\
	\textbf{Components:}
	V, S\\
	\textbf{Casting Time:}
	1 standard action\\
	\textbf{Range:}
	Close (7.5 m + 1.5 m/2 levels) or see text\\
	\textbf{Target:}
	One creature\\
	\textbf{Duration:}
	Instantaneous\\
	\textbf{Saving Throw:}
	Will negates (harmless)\\
	\textbf{Spell Resistance:}
	Yes\\
}
{
	The subject is freed from spells and effects that restrict its movement, including binding, entangle, grappling, imprisonment, maze, paralysis, petrification, pinning, sleep, slow, stunning, temporal stasis, and web. To free a creature from imprisonment or maze, you must know its name and background, and you must cast this spell at the spot where it was entombed or banished into the maze.

}

\input{subsections/spells/freezing-sphere.tex}
\Spell{Gaseous Form}{gaseous form}
{Transmutation}
{
	\textbf{Level:}
	Wiz 3\\
	\textbf{Components:}
	S, M\\
	\textbf{Casting Time:}
	1 standard action\\
	\textbf{Range:}
	Touch\\
	\textbf{Target:}
	Willing corporeal creature touched\\
	\textbf{Duration:}
	2 min./level (D)\\
	\textbf{Saving Throw:}
	None\\
	\textbf{Spell Resistance:}
	No\\
}
{
	The subject and all its gear become insubstantial, misty, and translucent. Its material armor (including natural armor) becomes worthless, though its size, Dexterity, deflection bonuses, and armor bonuses from force effects still apply. The subject gains damage reduction 10/magic and becomes immune to poison and critical hits. It can't attack or cast spells with verbal, somatic, material, or focus components while in gaseous form. (This does not rule out the use of certain spells that the subject may have prepared using the feats \feat{Silent Spell}, \feat{Still Spell}, and \feat{Eschew Materials}.) The subject also loses supernatural abilities while in gaseous form. If it has a touch spell ready to use, that spell is discharged harmlessly when the gaseous form spell takes effect.

	A gaseous creature can't run, but it can fly at a speed of 3 meters (maneuverability perfect). It can pass through small holes or narrow openings, even mere cracks, with all it was wearing or holding in its hands, as long as the spell persists. The creature is subject to the effects of wind, and it can't enter water or other liquid. It also can't manipulate objects or activate items, even those carried along with its gaseous form. Continuously active items remain active, though in some cases their effects may be moot.

	\textit{Arcane Material Component}:
	A bit of gauze and a wisp of smoke.

}

\input{subsections/spells/gate.tex}
\Spell{Geas, Lesser}{lesser geas}
{Enchantment (Compulsion) [Language-Dependent, Mind-Affecting]}
{
	\textbf{Level:}
	Wiz 4\\
	\textbf{Components:}
	V\\
	\textbf{Casting Time:}
	1 round\\
	\textbf{Range:}
	Close (7.5 m + 1.5 m/2 levels)\\
	\textbf{Target:}
	One living creature with 7 HD or less\\
	\textbf{Duration:}
	One day/level or until discharged (D)\\
	\textbf{Saving Throw:}
	Will negates\\
	\textbf{Spell Resistance:}
	Yes\\
}
{
	A \emph{lesser geas} places a magical command on a creature to carry out some service or to refrain from some action or course of activity, as desired by you. The creature must have 7 or fewer Hit Dice and be able to understand you. While a geas cannot compel a creature to kill itself or perform acts that would result in certain death, it can cause almost any other course of activity.

	The geased creature must follow the given instructions until the geas is completed, no matter how long it takes.

	If the instructions involve some open-ended task that the recipient cannot complete through his own actions the spell remains in effect for a maximum of one day per caster level. A clever recipient can subvert some instructions:

	If the subject is prevented from obeying the \emph{lesser geas} for 24 hours, it takes a $-2$ penalty to each of its ability scores. Each day, another $-2$ penalty accumulates, up to a total of $-8$. No ability score can be reduced to less than 1 by this effect. The ability score penalties are removed 24 hours after the subject resumes obeying the \emph{lesser geas}.

	A \emph{lesser geas} (and all ability score penalties) can be ended by \spell{break enchantment}, \spell{limited wish}, \spell{remove curse}, \spell{miracle}, or \spell{wish}. The \spell{dispel magic} spell does not affect a \emph{lesser geas}.

}

\Spell{Geas/Quest}{geas/quest}
{Enchantment (Compulsion) [Language-Dependent, Mind-Affecting]}
{
	\textbf{Level:}
	Clr 6, Wiz 6\\
	\textbf{Casting Time:}
	10 minutes\\
	\textbf{Target:}
	One living creature\\
	\textbf{Saving Throw:}
	None\\
}
{
	This spell functions similarly to lesser geas, except that it affects a creature of any HD and allows no saving throw.

	Instead of taking penalties to ability scores (as with lesser geas), the subject takes 3d6 points of damage each day it does not attempt to follow the geas/quest. Additionally, each day it must make a Fortitude saving throw or become sickened. These effects end 24 hours after the creature attempts to resume the geas/quest.

	A remove curse spell ends a geas/quest spell only if its caster level is at least two higher than your caster level. Break enchantment does not end a geas/quest, but limited wish, miracle, and wish do.

Bards, sorcerers, and wizards usually refer to this spell as geas, while clerics call the same spell quest.

}

\input{subsections/spells/gentle-repose.tex}
\Spell{Ghost Sound}{ghost sound}
{Illusion (Figment)}
{
	\textbf{Level:}
	Wiz 0\\
	\textbf{Components:}
	V, S, M\\
	\textbf{Casting Time:}
	1 standard action\\
	\textbf{Range:}
	Close (7.5 m + 1.5 m/2 levels)\\
	\textbf{Effect:}
	Illusory sounds\\
	\textbf{Duration:}
	1 round/level (D)\\
	\textbf{Saving Throw:}
	Will disbelief (if interacted with)\\
	\textbf{Spell Resistance:}
	No\\
}
{
	Ghost sound allows you to create a volume of sound that rises, recedes, approaches, or remains at a fixed place. You choose what type of sound ghost sound creates when casting it and cannot thereafter change the sound's basic character.

	The volume of sound created depends on your level. You can produce as much noise as four normal humans per caster level (maximum twenty humans). Thus, talking, singing, shouting, walking, marching, or running sounds can be created. The noise a ghost sound spell produces can be virtually any type of sound within the volume limit. A horde of rats running and squeaking is about the same volume as eight humans running and shouting. A roaring lion is equal to the noise from sixteen humans, while a roaring dire tiger is equal to the noise from twenty humans.

	Ghost sound can enhance the effectiveness of a silent image spell.

	Ghost sound can be made permanent with a permanency spell.

	\textit{Material Component}:
	A bit of wool or a small lump of wax.

}

\input{subsections/spells/ghoul-touch.tex}
\Spell{Giant Vermin}{giant vermin}
{Transmutation}
{
	\textbf{Level:}
	Clr 4, Drd 4\\
	\textbf{Components:}
	V, S, DF\\
	\textbf{Casting Time:}
	1 standard action\\
	\textbf{Range:}
	Close (7.5 m + 1.5 m/2 levels)\\
	\textbf{Targets:}
	Up to three vermin, no two of which can be more than 9 m apart\\
	\textbf{Duration:}
	1 min./level\\
	\textbf{Saving Throw:}
	None\\
	\textbf{Spell Resistance:}
	Yes\\
}
{
\Table{}{XX}{
\tableheader Caster Level & \tableheader Vermin Size\\
	9th or lower & Medium\\
	10th--13th & Large\\
	14th--17th & Huge\\
	18th--19th & Gargantuan\\
	20th or higher & Colossal\\
}

	You turn three normal-sized centipedes, two normal-sized spiders, or a single normal-sized scorpion into larger forms. Only one type of vermin can be transmuted (so a single casting cannot affect both a centipede and a spider), and all must be grown to the same size. The size to which the vermin can be grown depends on your level; see the table.

	Any giant vermin created by this spell do not attempt to harm you, but your control of such creatures is limited to simple commands (``Attack,'' ``Defend,'' ``Stop,'' and so forth). Orders to attack a certain creature when it appears or guard against a particular occurrence are too complex for the vermin to understand. Unless commanded to do otherwise, the giant vermin attack whoever or whatever is near them.

}

\input{subsections/spells/glibness.tex}
\input{subsections/spells/glitterdust.tex}
\Spell{Globe of Invulnerability, Lesser}{lesser globe of invulnerability}
{Abjuration}
{
	\textbf{Level:}
	Wiz 4\\
	\textbf{Components:}
	V, S, M\\
	\textbf{Casting Time:}
	1 standard action\\
	\textbf{Range:}
	3 m\\
	\textbf{Area:}
	3-m-radius spherical emanation, centered on you\\
	\textbf{Duration:}
	1 round/level (D)\\
	\textbf{Saving Throw:}
	None\\
	\textbf{Spell Resistance:}
	No\\
}
{
	An immobile, faintly shimmering magical sphere surrounds you and excludes all spell effects of 3rd level or lower. The area or effect of any such spells does not include the area of the \emph{lesser globe of invulnerability}. Such spells fail to affect any target located within the \emph{globe}. Excluded effects include spell-like abilities and spells or spell-like effects from items. However, any type of spell can be cast through or out of the magical \emph{globe}. Spells of 4th level and higher are not affected by the \emph{globe}, nor are spells already in effect when the \emph{globe} is cast. The \emph{globe} can be brought down by a targeted \spell{dispel magic} spell, but not by an area \spell{dispel magic}. You can leave and return to the \emph{globe} without penalty.

	Note that spell effects are not disrupted unless their effects enter the \emph{globe}, and even then they are merely suppressed, not dispelled.

	If a given spell has more than one level depending on which character class is casting it, use the level appropriate to the caster to determine whether \emph{lesser globe of invulnerability} stops it.

	\textit{Material Component:}
	A glass or crystal bead that shatters at the expiration of the spell.

}

\Spell{Globe of Invulnerability}{globe of invulnerability}
{Abjuration}
{
	\textbf{Level:}
	Wiz 6\\
}
{
	This spell functions like \spell{lesser globe of invulnerability}, except that it also excludes 4th-level spells and spell-like effects.

}

\Spell{Glyph of Warding, Greater}{glyph of warding, greater}
{Abjuration}
{
	\textbf{Level:}
	Clr 6\\
}
{
	This spell functions like glyph of warding, except that a greater blast glyph deals up to 10d8 points of damage, and a greater spell glyph can store a spell of 6th level or lower.

	\textit{Material Component}:
	You trace the glyph with incense, which must first be sprinkled with powdered diamond worth at least 400 gp.

}

\input{subsections/spells/glyph-of-warding.tex}
\input{subsections/spells/goodberry.tex}
\Spell{Good Hope}{good hope}
{Enchantment (Compulsion) [Mind-Affecting]}
{
	\textbf{Level:}
	Brd 3\\
	\textbf{Components:}
	V, S\\
	\textbf{Casting Time:}
	1 standard action\\
	\textbf{Range:}
	Medium (30 m + 3 m/level)\\
	\textbf{Targets:}
	One living creature/level, no two of which may be more than 9 m apart\\
	\textbf{Duration:}
	1 min./level\\
	\textbf{Saving Throw:}
	Will negates (harmless)\\
	\textbf{Spell Resistance:}
	Yes (harmless)\\
}
{
	This spell instills powerful hope in the subjects. Each affected creature gains a +2 morale bonus on saving throws, attack rolls, ability checks, skill checks, and weapon damage rolls.

	Good hope counters and dispels crushing despair.

}

\input{subsections/spells/grasping-hand.tex}
\Spell{Grease}{grease}
{Conjuration (Creation)}
{
	\textbf{Level:}
	Growth 2, Wiz 1\\
	\textbf{Components:}
	V, S, M\\
	\textbf{Casting Time:}
	1 standard action\\
	\textbf{Range:}
	Close (7.5 m + 1.5 m/2 levels)\\
	\textbf{Target or Area:}
	One object or a 3-m square\\
	\textbf{Duration:}
	1 round/level (D)\\
	\textbf{Saving Throw:}
	See text\\
	\textbf{Spell Resistance:}
	No\\
}
{
	A \emph{grease} spell covers a solid surface with a layer of slippery grease. Any creature in the area when the spell is cast must make a successful Reflex save or fall. This save is repeated on your turn each round that the creature remains within the area. A creature can walk within or through the area of \emph{grease} at half normal speed with a DC 10 \skill{Balance} check. Failure means it can't move that round (and must then make a Reflex save or fall), while failure by 5 or more means it falls (see the \skill{Balance} skill for details).

	The spell can also be used to create a greasy coating on an item. Material objects not in use are always affected by this spell, while an object wielded or employed by a creature receives a Reflex saving throw to avoid the effect. If the initial saving throw fails, the creature immediately drops the item. A saving throw must be made in each round that the creature attempts to pick up or use the greased item. A creature wearing greased armor or clothing gains a +10 circumstance bonus on \skill{Escape Artist} checks and on grapple checks made to resist or escape a grapple or to escape a pin.

	\textit{Material Component:}
	A bit of pork rind or butter.

}

\Spell{Guards and Wards}{guards and wards}
{Abjuration}
{
	\textbf{Level:}
	Wiz 6\\
	\textbf{Components:}
	V, S, M, F\\
	\textbf{Casting Time:}
	30 minutes\\
	\textbf{Range:}
	Anywhere within the area to be warded\\
	\textbf{Area:}
	Up to 200 sq. ft./level (S)\\
	\textbf{Duration:}
	2 hours/level (D)\\
	\textbf{Saving Throw:}
	See text\\
	\textbf{Spell Resistance:}
	See text\\
}
{
	This powerful spell is primarily used to defend your stronghold. The ward protects 200 square feet per caster level. The warded area can be as much as 6 meters high, and shaped as you desire. You can ward several stories of a stronghold by dividing the area among them; you must be somewhere within the area to be warded to cast the spell. The spell creates the following magical effects within the warded area.

	\textit{Fog}:
	\textbf{	Fog fills all corridors, obscuring all sight, including darkvision, beyond 1.5 meter. A creature within 1.5 meter has concealment (attacks have a 20\% miss chance). Creatures farther away have total concealment (50\% miss chance, and the attacker cannot use sight to locate the target). Saving Throw: None. Spell Resistance: No.}

	\textit{Arcane Locks}:
	\textbf{	All doors in the warded area are arcane locked. Saving Throw: None. Spell Resistance: No.}

	\textit{Webs}:
	\textbf{	Webs fill all stairs from top to bottom. These strands are identical with those created by the web spell, except that they regrow in 10 minutes if they are burned or torn away while the guards and wards spell lasts. Saving Throw: Reflex negates; see text for web. Spell Resistance: No.}

	\textit{Confusion}:
	\textbf{	Where there are choices in direction---such as a corridor intersection or side passage---a minor confusion-type effect functions so as to make it 50\% probable that intruders believe they are going in the opposite direction from the one they actually chose. This is an enchantment, mind-affecting effect. Saving Throw: None. Spell Resistance: Yes.}

	\textit{Lost Doors}:
	\textbf{	One door per caster level is covered by a silent image to appear as if it were a plain wall. Saving Throw: Will disbelief (if interacted with). Spell Resistance: No.}

	In addition, you can place your choice of one of the following five magical effects.

	\textbf{	1. Dancing lights in four corridors. You can designate a simple program that causes the lights to repeat as long as the guards and wards spell lasts. Saving Throw: None. Spell Resistance: No.}

	\textbf{	2. A magic mouth in two places. Saving Throw: None. Spell Resistance: No.}

	\textbf{	3. A stinking cloud in two places. The vapors appear in the places you designate; they return within 10 minutes if dispersed by wind while the guards and wards spell lasts. Saving Throw: Fortitude negates; see text for stinking cloud. Spell Resistance: No.}

	\textbf{	4. A gust of wind in one corridor or room. Saving Throw: Fortitude negates. Spell Resistance: Yes.}

	\textbf{	5. A suggestion in one place. You select an area of up to 1.5 meter square, and any creature who enters or passes through the area receives the suggestion mentally. Saving Throw: Will negates. Spell Resistance: Yes.}

	The whole warded area radiates strong magic of the abjuration school. A dispel magic cast on a specific effect, if successful, removes only that effect. A successful Mage's disjunction destroys the entire guards and wards effect.

	\textit{Material Component}:
	Burning incense, a small measure of brimstone and oil, a knotted string, and a small amount of blood.

	\textit{Focus}:
	A small silver rod.

}

\input{subsections/spells/guidance.tex}
\input{subsections/spells/gust-of-wind.tex}
\Spell{Hallow}{hallow}
{Evocation [Good]}
{
	\textbf{Level:}
	Clr 5, Drd 5\\
	\textbf{Components:}
	V, S, M, DF\\
	\textbf{Casting Time:}
	24 hours\\
	\textbf{Range:}
	Touch\\
	\textbf{Area:}
	12-m radius emanating from the touched point\\
	\textbf{Duration:}
	Instantaneous\\
	\textbf{Saving Throw:}
	See text\\
	\textbf{Spell Resistance:}
	See text\\
}
{
	\emph{Hallow} makes a particular site, building, or structure a holy site. This has four major effects.

	First, the site or structure is guarded by a \spell{magic circle against evil} effect.

	Second, all Charisma checks made to turn undead gain a +4 sacred bonus, and Charisma checks to command undead take a $-4$ penalty. Spell resistance does not apply to this effect. (This provision does not apply to the druid version of the spell.)

	Third, any dead body interred in a hallowed site cannot be turned into an undead creature.

	Finally, you may choose to fix a single spell effect to the hallowed site. The spell effect lasts for one year and functions throughout the entire site, regardless of the normal duration and area or effect. You may designate whether the effect applies to all creatures, creatures who share your faith or alignment, or creatures who adhere to another faith or alignment. At the end of the year, the chosen effect lapses, but it can be renewed or replaced simply by casting \emph{hallow} again.

	Spell effects that may be tied to a hallowed site include \spell{aid}, \spell{bane}, \spell{bless}, \spell{cause fear}, \spell{darkness}, \spell{daylight}, \spell{death ward}, \spell{deeper darkness}, \spell{detect evil}, \spell{detect magic}, \spell{dimensional anchor}, \spell{discern lies}, \spell{dispel magic}, \spell{endure elements}, \spell{freedom of movement}, \spell{invisibility purge}, \spell{protection from energy}, \spell{remove fear}, \spell{resist energy}, \spell{silence}, \spell{tongues}, and \spell{zone of truth}. Saving throws and spell resistance might apply to these spells' effects. (See the individual spell descriptions for details.)

	An area can receive only one \emph{hallow} spell (and its associated spell effect) at a time. \emph{Hallow} counters but does not dispel \spell{unhallow}.

	\textit{Material Component:}
	Herbs, oils, and incense worth at least 1,000 cp, plus 1,000 cp per level of the spell to be included in the hallowed area.

}

\Spell{Hallucinatory Terrain}{hallucinatory terrain}
{Illusion (Glamer)}
{
	\textbf{Level:}
	Mirage 4, Wiz 4\\
	\textbf{Components:}
	V, S, M\\
	\textbf{Casting Time:}
	10 minutes\\
	\textbf{Range:}
	Long (120 m + 12 m/level)\\
	\textbf{Area:}
	One 9-m cube/level (S)\\
	\textbf{Duration:}
	2 hours/level (D)\\
	\textbf{Saving Throw:}
	Will disbelief (if interacted with)\\
	\textbf{Spell Resistance:}
	No\\
}
{
	You make natural terrain look, sound, and smell like some other sort of natural terrain. Structures, equipment, and creatures within the area are not hidden or changed in appearance.

	\textit{Material Component:}
	A stone, a twig, and a bit of green plant.

}

\Spell{Halt Undead}{halt undead}
{Necromancy}
{
	\textbf{Level:}
	Wiz 3\\
	\textbf{Components:}
	V, S, M\\
	\textbf{Casting Time:}
	1 standard action\\
	\textbf{Range:}
	Medium (30 m + 3 m/level)\\
	\textbf{Targets:}
	Up to three undead creatures, no two of which can be more than 9 m apart\\
	\textbf{Duration:}
	1 round/level\\
	\textbf{Saving Throw:}
	Will negates (see text)\\
	\textbf{Spell Resistance:}
	Yes\\
}
{
	This spell renders as many as three undead creatures immobile. A nonintelligent undead creature gets no saving throw; an intelligent undead creature does. If the spell is successful, it renders the undead creature immobile for the duration of the spell (similar to the effect of hold person on a living creature). The effect is broken if the halted creatures are attacked or take damage.

	\textit{Material Component}:
	A pinch of sulfur and powdered garlic.

}

\Spell{Hardening}{hardening}
{Transmutation}
{
	\textbf{Level:} Wiz 6\\	
	\textbf{Components:} V, S\\	
	\textbf{Casting Time:} 1 standard action\\	
	\textbf{Range:} Touch\\	
	\textbf{Target:} One item of a volume no greater than 0.3 m$^3$/level (see text)\\	
	\textbf{Duration:} Permanent\\	
	\textbf{Saving Throw:} None\\	
	\textbf{Spell Resistance:} Yes (object)\\
}
{
	This spell increases the hardness of materials. For every two caster levels, increase by 1 the hardness of the material targeted by the spell. This hardness increase improves only the material's resistance to damage. Nothing else is modified by the improvement.

	The \emph{hardening} spell does not in any way affect resistance to other forms of transformation.

	This spell affects up to 0.3 cubic meter per level of the spellcaster.

	If cast upon a metal or mineral, the volume is reduced to 0.03 cubic meter per level.
}
\Spell{Harm}{harm}
{Necromancy}
{
	\textbf{Level:}
	Clr 6, Destruction 6\\
	\textbf{Components:}
	V, S\\
	\textbf{Casting Time:}
	1 standard action\\
	\textbf{Range:}
	Touch\\
	\textbf{Target:}
	Creature touched\\
	\textbf{Duration:}
	Instantaneous\\
	\textbf{Saving Throw:}
	Will half; see text\\
	\textbf{Spell Resistance:}
	Yes\\
}
{
	\emph{Harm} charges a subject with negative energy that deals 10 points of damage per caster level (to a maximum of 150 points at 15th level). If the creature successfully saves, \emph{harm} deals half this amount, but it cannot reduce the target's hit points to less than 1.

	If used on an undead creature, \emph{harm} acts like \spell{heal}.

}

\input{subsections/spells/haste.tex}
\Spell{Heal, Mass}{mass heal}
{Conjuration (Healing)}
{
	\textbf{Level:}
	Clr 9, Purity 9, Replenishment 9\\
	\textbf{Range:}
	Close (7.5 m + 1.5 m/2 levels)\\
	\textbf{Targets:}
	One or more creatures, no two of which can be more than 9 m apart\\
}
{
	This spell functions like \spell{heal}, except as noted above. The maximum number of hit points restored to each creature is 250.

}

\input{subsections/spells/heal-mount.tex}
\Spell{Heal}{heal}
{Conjuration (Healing)}
{
	\textbf{Level:}
	Clr 6, Drd 7, Healing 6\\
	\textbf{Components:}
	V, S\\
	\textbf{Casting Time:}
	1 standard action\\
	\textbf{Range:}
	Touch\\
	\textbf{Target:}
	Creature touched\\
	\textbf{Duration:}
	Instantaneous\\
	\textbf{Saving Throw:}
	Will negates (harmless)\\
	\textbf{Spell Resistance:}
	Yes (harmless)\\
}
{
	\textbf{	Heal enables you to channel positive energy into a creature to wipe away injury and afflictions. It immediately ends any and all of the following adverse conditions affecting the Target: ability damage, blinded, confused, dazed, dazzled, deafened, diseased, exhausted, fatigued, feebleminded, insanity, nauseated, sickened, stunned, and poisoned. It also cures 10 hit points of damage per level of the caster, to a maximum of 150 points at 15th level.}

	Heal does not remove negative levels, restore permanently drained levels, or restore permanently drained ability score points.

	If used against an undead creature, heal instead acts like harm.

}

\Spell{Heat Metal}{heat metal}
{Transmutation [Fire]}
{
	\textbf{Level:}
	Drd 2, Magma 2\\
	\textbf{Components:}
	V, S, DF\\
	\textbf{Casting Time:}
	1 standard action\\
	\textbf{Range:}
	Close (7.5 m + 1.5 m/2 levels)\\
	\textbf{Target:}
	Metal equipment of one creature per two levels, no two of which can be more than 9 m apart; or 12.5 kg of metal/level, all of which must be within a 9-m circle\\
	\textbf{Duration:}
	7 rounds\\
	\textbf{Saving Throw:}
	Will negates (object)\\
	\textbf{Spell Resistance:}
	Yes (object)\\
}
{
	\emph{Heat metal} makes metal extremely warm. Unattended, nonmagical metal gets no saving throw. Magical metal is allowed a saving throw against the spell. An item in a creature's possession uses the creature's saving throw bonus unless its own is higher.

	A creature takes fire damage if its equipment is heated. It takes full damage if its armor is affected or if it is holding, touching, wearing, or carrying metal weighing one-fifth of its weight. The creature takes minimum damage (1 point or 2 points; see the table) if it's not wearing metal armor and the metal that it's carrying weighs less than one-fifth of its weight.

\Table{}{lXX}{
\tableheader Round & \tableheader Metal Temperature & \tableheader Damage\\
	1 & Warm & None\\
	2 & Hot & 1d4 points\\
	3--5 & Searing & 2d4 points\\
	6 & Hot & 1d4 points\\
	7 & Warm & None\\
}

	On the first round of the spell, the metal becomes warm and uncomfortable to touch but deals no damage. The same effect also occurs on the last round of the spell's duration. During the second (and also the next-to-last) round, intense heat causes pain and damage. In the third, fourth, and fifth rounds, the metal is searing hot, causing more damage, as shown on the table below.

	Any cold intense enough to damage the creature negates fire damage from the spell (and vice versa) on a point-for-point basis. If cast underwater, \emph{heat metal} deals half damage and boils the surrounding water.

	\emph{Heat metal} counters and dispels \spell{chill metal}.

}

\input{subsections/spells/helping-hand.tex}
\input{subsections/spells/heroes-feast.tex}
\Spell{Heroism, Greater}{greater heroism}
{Enchantment (Compulsion) [Mind-Affecting]}
{
	\textbf{Level:}
	Wiz 6\\
	\textbf{Duration:}
	1 min./level\\
}
{
	This spell functions like \spell{heroism}, except the creature gains a +4 morale bonus on attack rolls, saves, and skill checks, immunity to fear effects, and temporary hit points equal to your caster level (maximum 20).

}

\input{subsections/spells/heroism.tex}
\input{subsections/spells/hide-from-animals.tex}
\input{subsections/spells/hide-from-undead.tex}
\Spell{Hideous Laughter}{hideous laughter}
{Enchantment (Compulsion) [Mind-Affecting]}
{
	\textbf{Level:}
	Wiz 2\\
	\textbf{Components:}
	V, S, M\\
	\textbf{Casting Time:}
	1 standard action\\
	\textbf{Range:}
	Close (7.5 m + 1.5 m/2 levels)\\
	\textbf{Target:}
	One creature; see text\\
	\textbf{Duration:}
	1 round/level\\
	\textbf{Saving Throw:}
	Will negates\\
	\textbf{Spell Resistance:}
	Yes\\
}
{
	This spell afflicts the subject with uncontrollable laughter. It collapses into gales of manic laughter, falling prone. The subject can take no actions while laughing, but is not considered helpless. After the spell ends, it can act normally.

	A creature with an Intelligence score of 2 or lower is not affected. A creature whose type is different from the caster's receives a +4 bonus on its saving throw, because humor doesn't ``translate'' well.

	\textit{Material Component}:
	Tiny tarts that are thrown at the target and a feather that is waved in the air.

}

\Spell{Hold Animal}{hold animal}
{Enchantment (Compulsion) [Mind-Affecting]}
{
	\textbf{Level:}
	Animal 2, Drd 2, Rgr 2\\
	\textbf{Components:}
	V, S\\
	\textbf{Target:}
	One animal\\
}
{
	This spell functions like hold person, except that it affects an animal instead of a humanoid.

}

\Spell{Hold Monster, Mass}{mass hold monster}
{Enchantment (Compulsion) [Mind-Affecting]}
{
	\textbf{Level:}
	Wiz 9\\
	\textbf{Targets:}
	One or more creatures, no two of which can be more than 9 m apart\\
}
{
	This spell functions like \spell{hold person}, except that it affects multiple creatures and holds any living creature that fails its Will save.

}

\Spell{Hold Monster}{hold monster}
{Enchantment (Compulsion) [Mind-Affecting]}
{
	\textbf{Level:}
	Law 6, Wiz 5\\
	\textbf{Components:}
	V, S, M/DF\\
	\textbf{Target:}
	One living creature\\
}
{
	This spell functions like \spell{hold person}, except that it affects any living creature that fails its Will save.

	\textit{Arcane Material Component}:
	One hard metal bar or rod, which can be as small as a three-penny nail.

}

\Spell{Hold Person, Mass}{hold person, mass}
{Enchantment (Compulsion) [Mind-Affecting]}
{
	\textbf{Level:}
	Wiz 7\\
	\textbf{Targets:}
	One or more humanoid creatures, no two of which can be more than 9 m apart\\
}
{
	This spell functions like hold person, except as noted above.

}

\input{subsections/spells/hold-person.tex}
\input{subsections/spells/hold-portal.tex}
\input{subsections/spells/holy-aura.tex}
\input{subsections/spells/holy-smite.tex}
\input{subsections/spells/holy-sword.tex}
\Spell{Holy Word}{holy word}
{Evocation [Good, Sonic]}
{
	\textbf{Level:}
	Clr 7\\
	\textbf{Components:}
	V\\
	\textbf{Casting Time:}
	1 standard action\\
	\textbf{Range:}
	12 m\\
	\textbf{Area:}
	Nongood creatures in a 12-m-radius spread centered on you\\
	\textbf{Duration:}
	Instantaneous\\
	\textbf{Saving Throw:}
	None or Will negates; see text\\
	\textbf{Spell Resistance:}
	Yes\\
}
{
	Any nongood creature within the area that hears the \emph{holy word} suffers the following ill effects.

\Table{}{XX}{
\tableheader HD & \tableheader Effect\\
	Equal to caster level & Deafened\\
	Up to caster level $-1$ & Blinded, deafened\\
	Up to caster level $-5$ & Paralyzed, blinded, deafened\\
	Up to caster level $-10$ & Killed, paralyzed, blinded, deafened\\
}

	The effects are cumulative and concurrent. No saving throw is allowed against these effects.

	\textit{Deafened:}
	The creature is deafened for 1d4 rounds.

	\textit{Blinded:}
	The creature is blinded for 2d4 rounds.

	\textit{Paralyzed:}
	The creature is paralyzed and helpless for 1d10 minutes.

	\textit{Killed:}
	Living creatures die. Undead creatures are destroyed.

	Furthermore, if you are on your home plane when you cast this spell, nongood extraplanar creatures within the area are instantly banished back to their home planes. Creatures so banished cannot return for at least 24 hours. This effect takes place regardless of whether the creatures hear the \emph{holy word}. The banishment effect allows a Will save (at a $-4$ penalty) to negate.

	Creatures whose HD exceed your caster level are unaffected by \emph{holy word}.

}

\Spell{Horrid Wilting}{horrid wilting}
{Necromancy}
{
	\textbf{Level:}
	Decay 8, Drought 8, Wiz 8, Water 8\\
	\textbf{Components:}
	V, S, M/DF\\
	\textbf{Casting Time:}
	1 standard action\\
	\textbf{Range:}
	Long (120 m + 12 m/level)\\
	\textbf{Targets:}
	Living creatures, no two of which can be more than 18 m apart\\
	\textbf{Duration:}
	Instantaneous\\
	\textbf{Saving Throw:}
	Fortitude half\\
	\textbf{Spell Resistance:}
	Yes\\
}
{
	This spell evaporates moisture from the body of each subject living creature, dealing 1d6 points of damage per caster level (maximum 20d6). This spell is especially devastating to water elementals and plant creatures, which instead take 1d8 points of damage per caster level (maximum 20d8).

	\textit{Arcane Material Component:}
	A bit of sponge.

}

\Spell{Hypnotic Pattern}{hypnotic pattern}
{Illusion (Pattern) [Mind-Affecting]}
{
	\textbf{Level:}
	Mirage 2, Wiz 2\\
	\textbf{Components:}
	S, M; see text\\
	\textbf{Casting Time:}
	1 standard action\\
	\textbf{Range:}
	Medium (30 m + 3 m/level)\\
	\textbf{Effect:}
	Colorful lights in a 3-m-radius spread\\
	\textbf{Duration:}
	Concentration + 2 rounds\\
	\textbf{Saving Throw:}
	Will negates\\
	\textbf{Spell Resistance:}
	Yes\\
}
{
	A twisting pattern of subtle, shifting colors weaves through the air, fascinating creatures within it. Roll 2d4 and add your caster level (maximum 10) to determine the total number of Hit Dice of creatures affected. Creatures with the fewest HD are affected first; and, among creatures with equal HD, those who are closest to the spell's point of origin are affected first. Hit Dice that are not sufficient to affect a creature are wasted. Affected creatures become fascinated by the pattern of colors. Sightless creatures are not affected.

	\textit{Material Component:}
	A glowing stick of incense or a crystal rod filled with phosphorescent material.

}

\input{subsections/spells/hypnotism.tex}
\Spell{Ice Storm}{ice storm}
{Evocation [Cold]}
{
	\textbf{Level:}
	Drd 4, Wiz 4, Water 5\\
	\textbf{Components:}
	V, S, M/DF\\
	\textbf{Casting Time:}
	1 standard action\\
	\textbf{Range:}
	Long (120 m + 12 m/level)\\
	\textbf{Area:}
	Cylinder (6-m radius, 12 m high)\\
	\textbf{Duration:}
	1 full round\\
	\textbf{Saving Throw:}
	None\\
	\textbf{Spell Resistance:}
	Yes\\
}
{
	Great magical hailstones pound down for 1 full round, dealing 3d6 points of bludgeoning damage and 2d6 points of cold damage to every creature in the area. A $-4$ penalty applies to each \skill{Listen} check made within the ice storm's effect, and all land movement within its area is at half speed. At the end of the duration, the hail disappears, leaving no aftereffects (other than the damage dealt).

	\textit{Arcane Material Component:}
	A pinch of dust and a few drops of water.

}

\input{subsections/spells/identify.tex}
\input{subsections/spells/illusory-script.tex}
\Spell{Illusory Wall}{illusory wall}
{Illusion (Figment)}
{
	\textbf{Level:}
	Wiz 4\\
	\textbf{Components:}
	V, S\\
	\textbf{Casting Time:}
	1 standard action\\
	\textbf{Range:}
	Close (7.5 m + 1.5 m/2 levels)\\
	\textbf{Effect:}
	Image 1 ft. by 3 m by 3 m\\
	\textbf{Duration:}
	Permanent\\
	\textbf{Saving Throw:}
	Will disbelief (if interacted with)\\
	\textbf{Spell Resistance:}
	No\\
}
{
	This spell creates the illusion of a wall, floor, ceiling, or similar surface. It appears absolutely real when viewed, but physical objects can pass through it without difficulty. When the spell is used to hide pits, traps, or normal doors, any detection abilities that do not require sight work normally. Touch or a probing search reveals the true nature of the surface, though such measures do not cause the illusion to disappear.

}

\Spell{Imbue with Spell Ability}{imbue with spell ability}
{Evocation}
{
	\textbf{Level:}
	Clr 4, Magic 4\\
	\textbf{Components:}
	V, S, DF\\
	\textbf{Casting Time:}
	10 minutes\\
	\textbf{Range:}
	Touch\\
	\textbf{Target:}
	Creature touched; see text\\
	\textbf{Duration:}
	Permanent until discharged (D)\\
	\textbf{Saving Throw:}
	Will negates (harmless)\\
	\textbf{Spell Resistance:}
	Yes (harmless)\\
}
{
	You transfer some of your currently prepared spells, and the ability to cast them, to another creature. Only a creature with an Intelligence score of at least 5 and a Wisdom score of at least 9 can receive this bestowal. Only cleric spells from the schools of abjuration, divination, and conjuration (healing) can be transferred. The number and level of spells that the subject can be granted depends on its Hit Dice; even multiple castings of imbue with spell ability can't exceed this limit.

\Table{}{XX}{
\tableheader HD of Recipient & \tableheader Spells Imbued\\
	2 or lower & One 1st-level spell\\
	3--4 & One or two 1st-level spells\\
	5 or higher & One or two 1st-level spells and one 2nd-level spell\\
}

	The transferred spell's variable characteristics (range, duration, area, and the like) function according to your level, not the level of the recipient.

	Once you cast imbue with spell ability, you cannot prepare a new 4th-level spell to replace it until the recipient uses the imbued spells or is slain, or until you dismiss the imbue with spell ability spell. In the meantime, you remain responsible to your deity or your principles for the use to which the spell is put. If the number of 4th-level spells you can cast decreases, and that number drops below your current number of active imbue with spell ability spells, the more recently cast imbued spells are dispelled.

	To cast a spell with a verbal component, the subject must be able to speak. To cast a spell with a somatic component, it must have humanlike hands. To cast a spell with a material component or focus, it must have the materials or focus.

}

\input{subsections/spells/implosion.tex}
\Spell{Imprisonment}{imprisonment}
{Abjuration}
{
	\textbf{Level:}
	Wiz 9\\
	\textbf{Components:}
	V, S\\
	\textbf{Casting Time:}
	1 standard action\\
	\textbf{Range:}
	Touch\\
	\textbf{Target:}
	Creature touched\\
	\textbf{Duration:}
	Instantaneous\\
	\textbf{Saving Throw:}
	Will negates; see text\\
	\textbf{Spell Resistance:}
	Yes\\
}
{
	When you cast imprisonment and touch a creature, it is entombed in a state of suspended animation (see the temporal stasis spell) in a small sphere far beneath the surface of the earth. The subject remains there unless a freedom spell is cast at the locale where the imprisonment took place. Magical search by a crystal ball, a locate object spell, or some other similar divination does not reveal the fact that a creature is imprisoned, but discern location does. A wish or miracle spell will not free the recipient, but will reveal where it is entombed. If you know the target's name and some facts about its life, the target takes a $-4$ penalty on its save.

}

\Spell{Incendiary Cloud}{incendiary cloud}
{Conjuration (Creation) [Fire]}
{
	\textbf{Level:}
	Fire 8, Wiz 8\\
	\textbf{Components:}
	V, S\\
	\textbf{Casting Time:}
	1 standard action\\
	\textbf{Range:}
	Medium (30 m + 3 m/level)\\
	\textbf{Effect:}
	Cloud spreads in 6-m radius, 6 m high\\
	\textbf{Duration:}
	1 round/level\\
	\textbf{Saving Throw:}
	Reflex half; see text\\
	\textbf{Spell Resistance:}
	No\\
}
{
	An \emph{incendiary cloud} spell creates a cloud of roiling smoke shot through with white-hot embers. The smoke obscures all sight as a \spell{fog cloud} does. In addition, the white-hot embers within the cloud deal 4d6 points of fire damage to everything within the cloud on your turn each round. All targets can make Reflex saves each round to take half damage.

	As with a \spell{cloudkill} spell, the smoke moves away from you at 3 meters per round. Figure out the smoke's new spread each round based on its new point of origin, which is 3 meters farther away from where you were when you cast the spell. By concentrating, you can make the cloud (actually its point of origin) move as much as 18 meters each round. Any portion of the cloud that would extend beyond your maximum range dissipates harmlessly, reducing the remainder's spread thereafter.

	As with \spell{fog cloud}, wind disperses the smoke, and the spell can't be cast underwater.

}

\Spell{Inflict Critical Wounds, Mass}{mass inflict critical wounds}
{Necromancy}
{
	\textbf{Level:}
	Clr 8\\
}
{
	This spell functions like \spell{mass inflict light wounds}, except that it deals 4d8 points of damage +1 point per caster level (maximum +40).

}

\Spell{Inflict Critical Wounds}{inflict critical wounds}
{Necromancy}
{
	\textbf{Level:}
	Clr 4, Destruction 4, Tmp 4\\
}
{
	This spell functions like \spell{inflict light wounds}, except that you deal 4d8 points of damage +1 point per caster level (maximum +20).

}

\Spell{Inflict Light Wounds, Mass}{mass inflict light wounds}
{Necromancy}
{
	\textbf{Level:}
	Clr 5, Destruction 5\\
	\textbf{Components:}
	V, S\\
	\textbf{Casting Time:}
	1 standard action\\
	\textbf{Range:}
	Close (7.5 m + 1.5 m/2 levels)\\
	\textbf{Target:}
	One creature/level, no two of which can be more than 9 m apart\\
	\textbf{Duration:}
	Instantaneous\\
	\textbf{Saving Throw:}
	Will half\\
	\textbf{Spell Resistance:}
	Yes\\
}
{
	Negative energy spreads out in all directions from the point of origin, dealing 1d8 points of damage +1 point per caster level (maximum +25) to nearby living enemies.

	Like other \emph{inflict} spells, \emph{mass inflict light wounds} cures undead in its area rather than damaging them. A cleric capable of spontaneously casting \emph{inflict} spells can also spontaneously cast \emph{mass inflict} spells.

}

\input{subsections/spells/inflict-light-wounds.tex}
\Spell{Inflict Minor Wounds}{inflict minor wounds}
{Necromancy}
{
	\textbf{Level:}
	Clr 0\\
	\textbf{Saving Throw:}
	Will negates\\
}
{
	This spell functions like \spell{inflict light wounds}, except that you deal 1 point of damage and a Will save negates the damage instead of halving it.

}

\Spell{Inflict Moderate Wounds, Mass}{mass inflict moderate wounds}
{Necromancy}
{
	\textbf{Level:}
	Clr 6\\
}
{
	This spell functions like \spell{mass inflict light wounds}, except that it deals 2d8 points of damage +1 point per caster level (maximum +30).

}

\Spell{Inflict Moderate Wounds}{inflict moderate wounds}
{Necromancy}
{
	\textbf{Level:}
	Clr 2\\
}
{
	This spell functions like inflict light wounds, except that you deal 2d8 points of damage +1 point per caster level (maximum +10).

}

\Spell{Inflict Serious Wounds, Mass}{mass inflict serious wounds}
{Necromancy}
{
	\textbf{Level:}
	Clr 7\\
}
{
	This spell functions like \spell{mass inflict light wounds}, except that it deals 3d8 points of damage +1 point per caster level (maximum +35).

}

\Spell{Inflict Serious Wounds}{inflict serious wounds}
{Necromancy}
{
	\textbf{Level:}
	Clr 3\\
}
{
	This spell functions like \spell{inflict light wounds}, except that you deal 3d8 points of damage +1 point per caster level (maximum +15).

}

\Spell{Insanity}{insanity}
{Enchantment (Compulsion) [Mind-Affecting]}
{
	\textbf{Level:}
	Charm 7, Madness 7, Wiz 7\\
	\textbf{Components:}
	V, S\\
	\textbf{Casting Time:}
	1 standard action\\
	\textbf{Range:}
	Medium (30 m + 3 m/level)\\
	\textbf{Target:}
	One living creature\\
	\textbf{Duration:}
	Instantaneous\\
	\textbf{Saving Throw:}
	Will negates\\
	\textbf{Spell Resistance:}
	Yes\\
}
{
	The affected creature suffers from a continuous \spell{confusion} effect, as the spell.

	A \spell{remove curse} does not remove \emph{insanity}. A \spell{greater restoration}, \spell{heal}, \spell{limited wish}, \spell{miracle}, or \spell{wish} can restore the creature.

}

\input{subsections/spells/insect-plague.tex}
\input{subsections/spells/instant-summons.tex}
\input{subsections/spells/interposing-hand.tex}
\Spell{Invisibility, Greater}{invisibility, greater}
{Illusion (Glamer)}
{
	\textbf{Level:}
	Wiz 4\\
	\textbf{Components:}
	V, S\\
	\textbf{Target:}
	You or creature touched\\
	\textbf{Duration:}
	1 round/level (D)\\
	\textbf{Saving Throw:}
	Will negates (harmless)\\
}
{
	This spell functions like \spell{invisibility}, except that it doesn't end if the subject attacks.

}

\Spell{Invisibility, Mass}{mass invisibility}
{Illusion (Glamer)}
{
	\textbf{Level:}
	Wiz 7\\
	\textbf{Components:}
	V, S, M\\
	\textbf{Range:}
	Long (120 m + 12 m/level)\\
	\textbf{Targets:}
	Any number of creatures, no two of which can be more than 54 m apart\\
}
{
	This spell functions like \spell{invisibility}, except that the effect is mobile with the group and is broken when anyone in the group attacks. Individuals in the group cannot see each other. The spell is broken for any individual who moves more than 54 meters from the nearest member of the group. (If only two individuals are affected, the one moving away from the other one loses its invisibility. If both are moving away from each other, they both become visible when the distance between them exceeds 54 meters.)

	\textit{Material Component:}
	An eyelash encased in a bit of gum arabic.

}

\Spell{Invisibility Purge}{invisibility purge}
{Evocation}
{
	\textbf{Level:}
	Cleansing 3, Clr 3, Mirage 3\\
	\textbf{Components:}
	V, S\\
	\textbf{Casting Time:}
	1 standard action\\
	\textbf{Range:}
	Personal\\
	\textbf{Target:}
	You\\
	\textbf{Duration:}
	1 min./level (D)\\
}
{
	You surround yourself with a sphere of power with a radius of 1.5 meter per caster level that negates all forms of invisibility.

	Anything invisible becomes visible while in the area.

}

\Spell{Invisibility Sphere}{invisibility sphere}
{Illusion (Glamer)}
{
	\textbf{Level:}
	Wiz 3\\
	\textbf{Components:}
	V, S, M\\
	\textbf{Area:}
	3-m-radius emanation around the creature or object touched\\
}
{
	This spell functions like \spell{invisibility}, except that this spell confers invisibility upon all creatures within 3 meters of the recipient. The center of the effect is mobile with the recipient.

	Those affected by this spell can see each other and themselves as if unaffected by the spell. Any affected creature moving out of the area becomes visible, but creatures moving into the area after the spell is cast do not become invisible. Affected creatures (other than the recipient) who attack negate the invisibility only for themselves. If the spell recipient attacks, the invisibility sphere ends.

}

\input{subsections/spells/invisibility.tex}
\input{subsections/spells/iron-body.tex}
\input{subsections/spells/ironwood.tex}
\input{subsections/spells/irresistible-dance.tex}
\Spell{Jump}{jump}
{Transmutation}
{
	\textbf{Level:}
	Ass 1, Drd 1, Rgr 1, Wiz 1\\
	\textbf{Components:}
	V, S, M\\
	\textbf{Casting Time:}
	1 standard action\\
	\textbf{Range:}
	Touch\\
	\textbf{Target:}
	Creature touched\\
	\textbf{Duration:}
	1 min./level (D)\\
	\textbf{Saving Throw:}
	Will negates (harmless)\\
	\textbf{Spell Resistance:}
	Yes\\
}
{
	The subject gets a +10 enhancement bonus on \skill{Jump} checks. The enhancement bonus increases to +20 at caster level 5th, and to +30 (the maximum) at caster level 9th.

	\textit{Material Component:}
	A grasshopper's hind leg, which you break when the spell is cast.

}

\input{subsections/spells/keen-edge.tex}
\Spell{Knock}{knock}
{Transmutation}
{
	\textbf{Level:}
	Wiz 2\\
	\textbf{Components:}
	V\\
	\textbf{Casting Time:}
	1 standard action\\
	\textbf{Range:}
	Medium (30 m + 3 m/level)\\
	\textbf{Target:}
	One door, box, or chest with an area of up to 10 sq. ft./level\\
	\textbf{Duration:}
	Instantaneous; see text\\
	\textbf{Saving Throw:}
	None\\
	\textbf{Spell Resistance:}
	No\\
}
{
	The \emph{knock} spell opens stuck, barred, locked, held, or arcane locked doors. It opens secret doors, as well as locked or trick-opening boxes or chests. It also loosens welds, shackles, or chains (provided they serve to hold closures shut). If used to open a \emph{arcane locked} door, the spell does not remove the \spell{arcane lock} but simply suspends its functioning for 10 minutes. In all other cases, the door does not relock itself or become stuck again on its own. \emph{Knock} does not raise barred gates or similar impediments (such as a portcullis), nor does it affect ropes, vines, and the like. The effect is limited by the area. Each spell can undo as many as two means of preventing egress.

}

\input{subsections/spells/know-direction.tex}
\input{subsections/spells/legend-lore.tex}
\input{subsections/spells/levitate.tex}
\input{subsections/spells/lightning-bolt.tex}
\Spell{Light}{light}
{Evocation [Light]}
{
	\textbf{Level:}
	Clr 0, Drd 0, Wiz 0\\
	\textbf{Components:}
	V, M/DF\\
	\textbf{Casting Time:}
	1 standard action\\
	\textbf{Range:}
	Touch\\
	\textbf{Target:}
	Object touched\\
	\textbf{Duration:}
	10 min./level (D)\\
	\textbf{Saving Throw:}
	None\\
	\textbf{Spell Resistance:}
	No\\
}
{
	This spell causes an object to glow like a torch, shedding bright light in a 20-foot radius (and dim light for an additional 6 meters) from the point you touch. The effect is immobile, but it can be cast on a movable object. Light taken into an area of magical darkness does not function.

	A light spell (one with the light descriptor) counters and dispels a darkness spell (one with the darkness descriptor) of an equal or lower level.

	\textit{Arcane Material Component}:
	A firefly or a piece of phosphorescent moss.

}

\Spell{Limited Wish}{limited wish}
{Universal}
{
	\textbf{Level:}
	Wiz 7\\
	\textbf{Components:}
	V, S, XP\\
	\textbf{Casting Time:}
	1 standard action\\
	\textbf{Range:}
	See text\\
	\textbf{Target, Effect, or Area:}
	See text\\
	\textbf{Duration:}
	See text\\
	\textbf{Saving Throw:}
	None; see text\\
	\textbf{Spell Resistance:}
	Yes\\
}
{
	A \emph{limited wish} lets you create nearly any type of effect. For example, a \emph{limited wish} can do any of the following things.

	\begin{itemize*}
	\item Duplicate any wizard spell of 6th level or lower, provided the spell is not of a school prohibited to you.
	\item Duplicate any other spell of 5th level or lower, provided the spell is not of a school prohibited to you.
	\item Duplicate any wizard spell of 5th level or lower, even if it's of a prohibited school.
	\item Duplicate any other spell of 4th level or lower, even if it's of a prohibited school.
	\item Undo the harmful effects of many spells, such as geas/quest or insanity.
	\item Produce any other effect whose power level is in line with the above effects, such as a single creature automatically hitting on its next attack or taking a $-7$ penalty on its next saving throw.
	\end{itemize*}

	A duplicated spell allows saving throws and spell resistance as normal (but the save DC is for a 7th-level spell). When a \emph{limited wish} duplicates a spell that has an XP cost, you must pay that cost or 300 XP, whichever is more. When a \emph{limited wish} spell duplicates a spell with a material component that costs more than 1,000 cp, you must provide that component.

	\textit{XP Cost}:
	300 XP or more (see above).

}

\input{subsections/spells/liveoak.tex}
\Spell{Locate Creature}{locate creature}
{Divination}
{
	\textbf{Level:}
	Wiz 4\\
	\textbf{Components:}
	V, S, M\\
	\textbf{Duration:}
	10 min./level\\
}
{
	This spell functions like locate object, except this spell locates a known or familiar creature.

	You slowly turn and sense when you are facing in the direction of the creature to be located, provided it is within range. You also know in which direction the creature is moving, if any.

	The spell can locate a creature of a specific kind or a specific creature known to you. It cannot find a creature of a certain type. To find a kind of creature, you must have seen such a creature up close (within 9 meters) at least once.

	Running water blocks the spell. It cannot detect objects. It can be fooled by mislead, nondetection, and polymorph spells.

	\textit{Material Component}:
	A bit of fur from a bloodhound.

}

\Spell{Locate Object}{locate object}
{Divination [Ritual]}
{
	\textbf{Level:}
	Clr 3, Tmp 3, Wiz 2\\
	\textbf{Components:}
	V, S, F/DF\\
	\textbf{Casting Time:}
	1 standard action\\
	\textbf{Range:}
	Long (120 m + 12 m/level)\\
	\textbf{Area:}
	Circle, centered on you, with a radius of 120 m + 12 m/level\\
	\textbf{Duration:}
	1 min./level\\
	\textbf{Saving Throw:}
	None\\
	\textbf{Spell Resistance:}
	No\\
}
{
	You sense the direction of a well-known or clearly visualized object. You can search for general items, in which case you locate the nearest one of its kind if more than one is within range. Attempting to find a certain item requires a specific and accurate mental image; if the image is not close enough to the actual object, the spell fails. You cannot specify a unique item unless you have observed that particular item firsthand (not through divination).

	The spell is blocked by even a thin sheet of lead. Creatures cannot be found by this spell. A \spell{polymorph any object} fools it.

	\textit{Arcane Focus:}
	A forked twig.

}

\input{subsections/spells/longstrider.tex}
\Spell{Lullaby}{lullaby}
{Enchantment (Compulsion) [Mind-Affecting]}
{
	\textbf{Level:}
	Brd 0\\
	\textbf{Components:}
	V, S\\
	\textbf{Casting Time:}
	1 standard action\\
	\textbf{Range:}
	Medium (30 m + 3 m/level)\\
	\textbf{Area:}
	Living creatures within a 3-m-radius burst\\
	\textbf{Duration:}
	Concentration + 1 round/level (D)\\
	\textbf{Saving Throw:}
	Will negates\\
	\textbf{Spell Resistance:}
	Yes\\
}
{
	Any creature within the area that fails a Will save becomes drowsy and inattentive, taking a -5 penalty on Listen and Spot checks and a -2 penalty on Will saves against sleep effects while the lullaby is in effect. Lullaby lasts for as long as the caster concentrates, plus up to 1 round per caster level thereafter.

}

\input{subsections/spells/maddening-scream.tex}
\Spell{Mage Armor}{mage armor}
{Conjuration (Creation) [Force]}
{
	\textbf{Level:}
	Wiz 1\\
	\textbf{Components:}
	V, S, F\\
	\textbf{Casting Time:}
	1 standard action\\
	\textbf{Range:}
	Touch\\
	\textbf{Target:}
	Creature touched\\
	\textbf{Duration:}
	1 hour/level (D)\\
	\textbf{Saving Throw:}
	Will negates (harmless)\\
	\textbf{Spell Resistance:}
	No\\
}
{
	An invisible but tangible field of force surrounds the subject of a mage armor spell, providing a +4 armor bonus to AC.

	Unlike mundane armor, mage armor entails no armor check penalty, arcane spell failure chance, or speed reduction. Since mage armor is made of force, incorporeal creatures can't bypass it the way they do normal armor.

	\textit{Focus}:
	A piece of cured leather.

}

\Spell{Mage Hand}{mage hand}
{Transmutation}
{
	\textbf{Level:}
	Wiz 0\\
	\textbf{Components:}
	V, S\\
	\textbf{Casting Time:}
	1 standard action\\
	\textbf{Range:}
	Close (7.5 m + 1.5 m/2 levels)\\
	\textbf{Target:}
	One nonmagical, unattended object weighing up to 5 lb.\\
	\textbf{Duration:}
	Concentration\\
	\textbf{Saving Throw:}
	None\\
	\textbf{Spell Resistance:}
	No\\
}
{
	You point your finger at an object and can lift it and move it at will from a distance. As a move action, you can propel the object as far as 4.5 meters in any direction, though the spell ends if the distance between you and the object ever exceeds the spell's range.

}

\Spell{Mage's Disjunction}{mage's disjunction}
{Abjuration}
{
	\textbf{Level:}
	Magic 9, Wiz 9\\
	\textbf{Components:}
	V\\
	\textbf{Casting Time:}
	1 standard action\\
	\textbf{Range:}
	Close (7.5 m + 1.5 m/2 levels)\\
	\textbf{Area:}
	All magical effects and magic items within a 12-m-radius burst\\
	\textbf{Duration:}
	Instantaneous\\
	\textbf{Saving Throw:}
	Will negates (object)\\
	\textbf{Spell Resistance:}
	No\\
}
{
	All magical effects and magic items within the radius of the spell, except for those that you carry or touch, are disjoined. That is, spells and spell-like effects are separated into their individual components (ending the effect as a \spell{dispel magic} spell does), and each permanent magic item must make a successful Will save or be turned into a normal item. An item in a creature's possession uses its own Will save bonus or its possessor's Will save bonus, whichever is higher.

	You also have a 1\% chance per caster level of destroying an \spell{antimagic field}. If the \spell{antimagic field} survives the disjunction, no items within it are disjoined.

	Even artifacts are subject to disjunction, though there is only a 1\% chance per caster level of actually affecting such powerful items. Additionally, if an artifact is destroyed, you must make a DC 25 Will save or permanently lose all spellcasting abilities. (These abilities cannot be recovered by mortal magic, not even miracle or wish.)

	\textit{Note:} Destroying artifacts is a dangerous business, and it is 95\% likely to attract the attention of some powerful being who has an interest in or connection with the device.

}

\input{subsections/spells/mages-faithful-hound.tex}
\input{subsections/spells/mages-lucubration.tex}
\input{subsections/spells/mages-magnificent-mansion.tex}
\input{subsections/spells/mages-private-sanctum.tex}
\input{subsections/spells/mages-sword.tex}
\input{subsections/spells/magic-aura.tex}
\Spell{Magic Circle against Chaos}{magic circle against chaos}
{Abjuration [Lawful]}
{
	\textbf{Level:}
	Clr 3, Law 3, Pal 3, Wiz 3\\
}
{
	This spell functions like \spell{magic circle against evil}, except that it is similar to protection from chaos instead of protection from evil, and it can imprison a nonlawful called creature.

}

\input{subsections/spells/magic-circle-against-evil.tex}
\Spell{Magic Circle against Good}{magic circle against good}
{Abjuration [Evil]}
{
	\textbf{Level:}
	Clr 3, Evil 3, Wiz 3\\
}
{
	This spell functions like magic circle against evil, except that it is similar to protection from good instead of protection from evil, and it can imprison a nonevil called creature.

}

\Spell{Magic Circle against Law}{magic circle against law}
{Abjuration [Chaotic]}
{
	\textbf{Level:}
	Chaos 3, Clr 3, Wiz 3\\
}
{
	This spell functions like magic circle against evil, except that it is similar to protection from law instead of protection from evil, and it can imprison a nonchaotic called creature.

}

\Spell{Magic Fang, Greater}{magic fang, greater}
{Transmutation}
{
	\textbf{Level:}
	Drd 3, Rgr 3\\
	\textbf{Range:}
	Close (7.5 m + 1.5 m/2 levels)\\
	\textbf{Target:}
	One living creature\\
	\textbf{Duration:}
	1 hour/level\\
}
{
	This spell functions like magic fang, except that the enhancement bonus on attack and damage rolls is +1 per four caster levels (maximum +5).

	Alternatively, you may imbue all of the creature's natural weapons with a +1 enhancement bonus (regardless of your caster level).

	Greater magic fang can be made permanent with a permanency spell.

}

\input{subsections/spells/magic-fang.tex}
\input{subsections/spells/magic-jar.tex}
\input{subsections/spells/magic-missile.tex}
\Spell{Magic Mouth}{magic mouth}
{Illusion (Glamer) [Ritual]}
{
	\textbf{Level:}
	Wiz 2\\
	\textbf{Components:}
	V, S, M\\
	\textbf{Casting Time:}
	1 standard action\\
	\textbf{Range:}
	Close (7.5 m + 1.5 m/2 levels)\\
	\textbf{Target:}
	One creature or object\\
	\textbf{Duration:}
	Permanent until discharged\\
	\textbf{Saving Throw:}
	Will negates (object)\\
	\textbf{Spell Resistance:}
	Yes (object)\\
}
{
	This spell imbues the chosen object or creature with an enchanted mouth that suddenly appears and speaks its message the next time a specified event occurs. The message, which must be twenty-five or fewer words long, can be in any language known by you and can be delivered over a period of 10 minutes. The mouth cannot utter verbal components, use command words, or activate magical effects. It does, however, move according to the words articulated; if it were placed upon a statue, the mouth of the statue would move and appear to speak. Of course, \emph{magic mouth} can be placed upon a tree, rock, or any other object or creature.

	The spell functions when specific conditions are fulfilled according to your command as set in the spell. Commands can be as general or as detailed as desired, although only visual and audible triggers can be used. Triggers react to what appears to be the case. Disguises and illusions can fool them. Normal darkness does not defeat a visual trigger, but magical darkness or invisibility does. Silent movement or magical silence defeats audible triggers. Audible triggers can be keyed to general types of noises or to a specific noise or spoken word. Actions can serve as triggers if they are visible or audible. A \emph{magic mouth} cannot distinguish alignment, level, Hit Dice, or class except by external garb.

	The range limit of a trigger is 4.5 meters per caster level, so a 6th-level caster can command a \emph{magic mouth} to respond to triggers as far as 27 meters away. Regardless of range, the mouth can respond only to visible or audible triggers and actions in line of sight or within hearing distance.

	\emph{Magic mouth} can be made permanent with a \spell{permanency} spell.

	\textit{Material Component:}
	A small bit of honeycomb and jade dust worth 10 cp.

}

\input{subsections/spells/magic-stone.tex}
\input{subsections/spells/magic-vestment.tex}
\input{subsections/spells/magic-weapon-greater.tex}
\input{subsections/spells/magic-weapon.tex}
\Spell{Major Creation}{major creation}
{Conjuration (Creation)}
{
	\textbf{Level:}
	Wiz 5\\
	\textbf{Casting Time:}
	10 minutes\\
	\textbf{Range:}
	Close (7.5 m + 1.5 m/2 levels)\\
	\textbf{Duration:}
	See text\\
}
{
	This spell functions like \spell{minor creation}, except that you can also create an object of mineral nature: stone, crystal, metal, or the like. The duration of the created item varies with its relative hardness and rarity, as indicated on the following table.

\Table{}{XX}{

\tableheader Hardness and Rarity Examples & \tableheader Duration\\
	Vegetable matter & 2 hr./level\\
	Stone, crystal, base metals & 1 hr./level\\
	Precious metals & 20 min./level\\
	Gems & 10 min./level\\
	Rare metal\footnotemark[1] & 1 round/level\\

\TableNote{2}{1 Includes adamantine, alchemical silver, and mithral. You can't use major creation to create a cold iron item.}\\

}

}

\Spell{Major Image}{major image}
{Illusion (Figment)}
{
	\textbf{Level:}
	Wiz 3\\
	\textbf{Duration:}
	Concentration + 3 rounds\\
}
{
	This spell functions like \spell{silent image}, except that sound, smell, and thermal illusions are included in the spell effect. While concentrating, you can move the image within the range.

	The image disappears when struck by an opponent unless you cause the illusion to react appropriately.

}

\input{subsections/spells/make-whole.tex}
\Spell{Mark of Justice}{mark of justice}
{Necromancy}
{
	\textbf{Level:}
	Clr 5, Pal 4\\
	\textbf{Components:}
	V, S, DF\\
	\textbf{Casting Time:}
	10 minutes\\
	\textbf{Range:}
	Touch\\
	\textbf{Target:}
	Creature touched\\
	\textbf{Duration:}
	Permanent; see text\\
	\textbf{Saving Throw:}
	None\\
	\textbf{Spell Resistance:}
	Yes\\
}
{
	You draw an indelible mark on the subject and state some behavior on the part of the subject that will activate the mark. When activated, the mark curses the subject. Typically, you designate some sort of criminal behavior that activates the mark, but you can pick any act you please. The effect of the mark is identical with the effect of bestow curse.

	Since this spell takes 10 minutes to cast and involves writing on the target, you can cast it only on a creature that is willing or restrained.

	Like the effect of bestow curse, a mark of justice cannot be dispelled, but it can be removed with a break enchantment, limited wish, miracle, remove curse, or wish spell. Remove curse works only if its caster level is equal to or higher than your mark of justice caster level. These restrictions apply regardless of whether the mark has activated.

}

\Spell{Maze}{maze}
{Conjuration (Teleportation)}
{
	\textbf{Level:}
	Wiz 8\\
	\textbf{Components:}
	V, S\\
	\textbf{Casting Time:}
	1 standard action\\
	\textbf{Range:}
	Close (7.5 m + 1.5 m/2 levels)\\
	\textbf{Target:}
	One creature\\
	\textbf{Duration:}
	See text\\
	\textbf{Saving Throw:}
	None\\
	\textbf{Spell Resistance:}
	Yes\\
}
{
	You banish the subject into an extradimensional labyrinth of force planes. Each round on its turn, it may attempt a DC 20 Intelligence check to escape the labyrinth as a full-round action. If the subject doesn't escape, the maze disappears after 10 minutes, forcing the subject to leave.

	On escaping or leaving the maze, the subject reappears where it had been when the maze spell was cast. If this location is filled with a solid object, the subject appears in the nearest open space. Spells and abilities that move a creature within a plane, such as teleport and dimension door, do not help a creature escape a maze spell, although a plane shift spell allows it to exit to whatever plane is designated in that spell. Minotaurs are not affected by this spell.

}

\input{subsections/spells/meld-into-stone.tex}
\Spell{Mending}{mending}
{Transmutation}
{
	\textbf{Level:}
	Clr 0, Drd 0, Tmp 0, Wiz 0\\
	\textbf{Components:}
	V, S\\
	\textbf{Casting Time:}
	1 standard action\\
	\textbf{Range:}
	3 m\\
	\textbf{Target:}
	One object of up to 0.5 kg\\
	\textbf{Duration:}
	Instantaneous\\
	\textbf{Saving Throw:}
	Will negates (harmless, object)\\
	\textbf{Spell Resistance:}
	Yes (harmless, object)\\
}
{
	\emph{Mending} repairs small breaks or tears in objects (but not warps, such as might be caused by a warp wood spell). It will weld broken metallic objects such as a ring, a chain link, a medallion, or a slender dagger, providing but one break exists.

	Ceramic or wooden objects with multiple breaks can be invisibly rejoined to be as strong as new. A hole in a leather sack or a wineskin is completely healed over by \emph{mending}. The spell can repair a magic item, but the item's magical abilities are not restored. The spell cannot mend broken magic rods, staffs, or wands, nor does it affect creatures (including constructs).

}

\Spell{Mental Pinnacle}{mental pinnacle}
{Transmutation}
{
	\textbf{Level:} Wiz 6\\
	\textbf{Components:} V, S, M\\
	\textbf{Casting Time:} 1 standard action\\
	\textbf{Range:} Personal\\
	\textbf{Target:} You\\
	\textbf{Duration:} 1 round/level\\
}
{
	For a brief time, you achieve the mental dominance of a powerful psion, able to lash out at enemies using only the power of your mind. Your revel in your new mental powers to the point that you disdain using spells, even in the form of effects from magic items. You gain a +4 enhancement bonus to Intelligence and Wisdom, 3 power points per caster level, and access to the following powers.

	\begin{itemize*}
		\item \psionic{Mind Thrust}\textsuperscript{A:} Deal 1d10 damage.
		\item \psionic{Ego Whip}\textsuperscript{A:} Deal 1d4 Cha damage and daze for 1 round.
		\item \psionic{Psionic Blast}\textsuperscript{A:} Stun creatures in 9-m cone for 1 round.
		\item \psionic{Id Insinuation}\textsuperscript{A:} Swift tendrils of thought disrupt and confuse your target.
		\item \psionic{Psychic Crush}\textsuperscript{A:} Brutally crush subject's mental essence, reducing subject to $-1$ hit points.
	\end{itemize*}

	You manifest the powers as a psion of your caster level does, creating displays as described in each power's description. You lose your spellcasting ability, including your ability to use spell trigger or spell completion magic items, just as if those spells were no longer on your class list. For the duration of this spell, you use magic items and psionic items as if you were a psion with only the five powers given above on your class list.

	Any unspent power points dissipate when the spell ends.

	\textit{Material Component:} A potion of \spell{fox's cunning}, which you drink (its effect is overridden by the effect of this spell).
}
\input{subsections/spells/message.tex}
\input{subsections/spells/meteor-swarm.tex}
\Spell{Mind Blank}{mind blank}
{Abjuration}
{
	\textbf{Level:}
	Cleansing 8, Mind 8, Protection 8, Wiz 8\\
	\textbf{Components:}
	V, S\\
	\textbf{Casting Time:}
	1 standard action\\
	\textbf{Range:}
	Close (7.5 m + 1.5 m/2 levels)\\
	\textbf{Target:}
	One creature\\
	\textbf{Duration:}
	24 hours\\
	\textbf{Saving Throw:}
	Will negates (harmless)\\
	\textbf{Spell Resistance:}
	Yes (harmless)\\
}
{
	The subject is protected from all devices and spells that detect, influence, or read emotions or thoughts. This spell protects against all mind-affecting spells and effects as well as information gathering by divination spells or effects. \emph{Mind blank} even foils \spell{limited wish}, \spell{miracle}, and \spell{wish} spells when they are used in such a way as to affect the subject's mind or to gain information about it. In the case of scrying that scans an area the creature is in, such as \spell{arcane eye}, the spell works but the creature simply isn't detected. Scrying attempts that are targeted specifically at the subject do not work at all.

}

\Spell{Mind Fog}{mind fog}
{Enchantment (Compulsion) [Mind-Affecting]}
{
	\textbf{Level:}
	Wiz 5\\
	\textbf{Components:}
	V, S\\
	\textbf{Casting Time:}
	1 standard action\\
	\textbf{Range:}
	Medium (30 m + 3 m/level)\\
	\textbf{Effect:}
	Fog spreads in 6-m radius, 6 m high\\
	\textbf{Duration:}
	30 minutes and 2d6 rounds; see text\\
	\textbf{Saving Throw:}
	Will negates\\
	\textbf{Spell Resistance:}
	Yes\\
}
{
	Mind fog produces a bank of thin mist that weakens the mental resistance of those caught in it. Creatures in the mind fog take a $-10$ competence penalty on Wisdom checks and Will saves. (A creature that successfully saves against the fog is not affected and need not make further saves even if it remains in the fog.) Affected creatures take the penalty as long as they remain in the fog and for 2d6 rounds thereafter. The fog is stationary and lasts for 30 minutes (or until dispersed by wind).

	A moderate wind (11+ mph) disperses the fog in four rounds; a strong wind (21+ mph) disperses the fog in 1 round.

	The fog is thin and does not significantly hamper vision.

}

\Spell{Minor Creation}{minor creation}
{Conjuration (Creation)}
{
	\textbf{Level:}
	Wiz 4\\
	\textbf{Components:}
	V, S, M\\
	\textbf{Casting Time:}
	1 minute\\
	\textbf{Range:}
	0 m\\
	\textbf{Effect:}
	Unattended, nonmagical object of nonliving plant matter, up to 0.03 m$^3$/level\\
	\textbf{Duration:}
	1 hour/level (D)\\
	\textbf{Saving Throw:}
	None\\
	\textbf{Spell Resistance:}
	No\\
}
{
	You create a nonmagical, unattended object of nonliving, vegetable matter. The volume of the item created cannot exceed 1 cubic foot per caster level. You must succeed on an appropriate skill check to make a complex item.

	Attempting to use any created object as a material component causes the spell to fail.

	\textit{Material Component:}
	A tiny piece of matter of the same sort of item you plan to create with minor creation.

}

\Spell{Minor Image}{minor image}
{Illusion (Figment)}
{
	\textbf{Level:}
	Wiz 2\\
	\textbf{Duration:}
	Concentration +2 rounds\\
}
{
	This spell functions like silent image, except that minor image includes some minor sounds but not understandable speech.

}

\input{subsections/spells/miracle.tex}
\Spell{Mirage Arcana}{mirage arcana}
{Illusion (Glamer)}
{
	\textbf{Level:}
	Wiz 5\\
	\textbf{Components:}
	V, S\\
	\textbf{Casting Time:}
	1 standard action\\
	\textbf{Area:}
	One 6-m cube/level (S)\\
	\textbf{Duration:}
	Concentration +1 hour/ level (D)\\
}
{
	This spell functions like hallucinatory terrain, except that it enables you to make any area appear to be something other than it is. The illusion includes audible, visual, tactile, and olfactory elements. Unlike hallucinatory terrain, the spell can alter the appearance of structures (or add them where none are present). Still, it can't disguise, conceal, or add creatures (though creatures within the area might hide themselves within the illusion just as they can hide themselves within a real location).

}

\input{subsections/spells/mirror-image.tex}
\Spell{Misdirection}{misdirection}
{Illusion (Glamer)}
{
	\textbf{Level:}
	Wiz 2\\
	\textbf{Components:}
	V, S\\
	\textbf{Casting Time:}
	1 standard action\\
	\textbf{Range:}
	Close (7.5 m + 1.5 m/2 levels)\\
	\textbf{Target:}
	One creature or object, up to a 3-m cube in size\\
	\textbf{Duration:}
	1 hour/level\\
	\textbf{Saving Throw:}
	None or Will negates; see text\\
	\textbf{Spell Resistance:}
	No\\
}
{
	By means of this spell, you misdirect the information from divination spells that reveal auras (\spell{detect evil}, \spell{detect magic}, \spell{discern lies}, and the like). On casting the spell, you choose another object within range. For the duration of the spell, the subject of \emph{misdirection} is detected as if it were the other object. (Neither the subject nor the other object gets a saving throw against this effect.) Detection spells provide information based on the second object rather than on the actual target of the detection unless the caster of the detection succeeds on a Will save. For instance, you could make yourself detect as a tree if one were within range at casting: not evil, not lying, not magical, neutral in alignment, and so forth. This spell does not affect other types of divination magic (\spell{augury}, \spell{detect thoughts}, \spell{clairaudience/clairvoyance}, and the like).

}

\Spell{Mislead}{mislead}
{Illusion (Figment, Glamer)}
{
	\textbf{Level:}
	Luck 6, Wiz 6, Trickery 6\\
	\textbf{Components:}
	S\\
	\textbf{Casting Time:}
	1 standard action\\
	\textbf{Range:}
	Close (7.5 m + 1.5 m/2 levels)\\
	\textbf{Target/Effect:}
	You/one illusory double\\
	\textbf{Duration:}
	1 round/level (D) and concentration + 3 rounds; see text\\
	\textbf{Saving Throw:}
	None or Will disbelief (if interacted with); see text\\
	\textbf{Spell Resistance:}
	No\\
}
{
	You become invisible (as greater invisibility, a glamer), and at the same time, an illusory double of you (as major image, a figment) appears. You are then free to go elsewhere while your double moves away. The double appears within range but thereafter moves as you direct it (which requires concentration beginning on the first round after the casting). You can make the figment appear superimposed perfectly over your own body so that observers don't notice an image appearing and you turning invisible. You and the figment can then move in different directions. The double moves at your speed and can talk and gesture as if it were real, but it cannot attack or cast spells, though it can pretend to do so.

	The illusory double lasts as long as you concentrate upon it, plus 3 additional rounds. After you cease concentration, the illusory double continues to carry out the same activity until the duration expires. The greater invisibility lasts for 1 round per level, regardless of concentration.

}

\input{subsections/spells/mnemonic-enhancer.tex}
\input{subsections/spells/modify-memory.tex}
\input{subsections/spells/moment-of-prescience.tex}
\Spell{Mount}{mount}
{Conjuration (Summoning)}
{
	\textbf{Level:}
	Wiz 1\\
	\textbf{Components:}
	V, S, M\\
	\textbf{Casting Time:}
	1 round\\
	\textbf{Range:}
	Close (7.5 m + 1.5 m/2 levels)\\
	\textbf{Effect:}
	One mount\\
	\textbf{Duration:}
	2 hours/level (D)\\
	\textbf{Saving Throw:}
	None\\
	\textbf{Spell Resistance:}
	No\\
}
{
	You summon a light horse or a pony (your choice) to serve you as a mount. The steed serves willingly and well. The mount comes with a bit and bridle and a riding saddle.

	\textit{Material Component:}
	A bit of horse hair.

}

\Spell{Move Earth}{move earth}
{Transmutation [Earth]}
{
	\textbf{Level:}
	Drd 6, Wiz 6\\
	\textbf{Components:}
	V, S, M\\
	\textbf{Casting Time:}
	See text\\
	\textbf{Range:}
	Long (120 m + 12 m/level)\\
	\textbf{Area:}
	Dirt in an area up to 225 m square and up to 3 m deep (S)\\
	\textbf{Duration:}
	Instantaneous\\
	\textbf{Saving Throw:}
	None\\
	\textbf{Spell Resistance:}
	No\\
}
{
	Move earth moves dirt (clay, loam, sand), possibly collapsing embankments, moving hillocks, shifting dunes, and so forth.

	However, in no event can rock formations be collapsed or moved. The area to be affected determines the casting time. For every 45-meter square (up to 3 meters deep), casting takes 10 minutes. The maximum area, 225 meters by 225 meters, takes 4 hours and 10 minutes to move.

	This spell does not violently break the surface of the ground. Instead, it creates wavelike crests and troughs, with the earth reacting with glacierlike fluidity until the desired result is achieved. Trees, structures, rock formations, and such are mostly unaffected except for changes in elevation and relative topography.

	The spell cannot be used for tunneling and is generally too slow to trap or bury creatures. Its primary use is for digging or filling moats or for adjusting terrain contours before a battle.

	This spell has no effect on earth creatures.

	\textit{Material Component:}
	A mixture of soils (clay, loam, and sand) in a small bag, and an iron blade.

}

\input{subsections/spells/neutralize-poison.tex}
\Spell{Nightmare}{nightmare}
{Illusion (Phantasm) [Mind-Affecting, Evil]}
{
	\textbf{Level:}
	Wiz 5\\
	\textbf{Components:}
	V, S\\
	\textbf{Casting Time:}
	10 minutes\\
	\textbf{Range:}
	Unlimited\\
	\textbf{Target:}
	One living creature\\
	\textbf{Duration:}
	Instantaneous\\
	\textbf{Saving Throw:}
	Will negates; see text\\
	\textbf{Spell Resistance:}
	Yes\\
}
{
	You send a hideous and unsettling phantasmal vision to a specific creature that you name or otherwise specifically designate.

	The nightmare prevents restful sleep and causes 1d10 points of damage. The nightmare leaves the subject fatigued and unable to regain arcane spells for the next 24 hours.

\Table{}{XX}{
\tableheader Knowledge & \tableheader Will Save Modifier\\
	None\footnotemark[1] & +10\\
	Secondhand (you have heard of the subject) & +5\\
	Firsthand (you have met the subject) & +0\\
	Familiar (you know the subject well) & $-5$\\

\TableNote{2}{1 You must have some sort of connection to a creature you have no knowledge of.}\\
}
\Table{}{XX}{
\tableheader Connection & \tableheader Will Save Modifier\\
	Likeness or picture & $-2$\\
	Possession or garment & $-4$\\
	Body part, lock of hair, bit of nail, etc. & $-10$\\
}
	The difficulty of the save depends on how well you know the subject and what sort of physical connection (if any) you have to that creature.

Dispel evil cast on the subject while you are casting the spell dispels the nightmare and causes you to be stunned for 10 minutes per caster level of the dispel evil.

	If the recipient is awake when the spell begins, you can choose to cease casting (ending the spell) or to enter a trance until the recipient goes to sleep, whereupon you become alert again and complete the casting. If you are disturbed during the trance, you must succeed on a \skill{Concentration} check as if you were in the midst of casting a spell or the spell ends.

	If you choose to enter a trance, you are not aware of your surroundings or the activities around you while in the trance.

	You are defenseless, both physically and mentally, while in the trance. (You always fail any saving throw, for example.)

	Creatures who don't sleep (such as elves, but not half-elves) or dream are immune to this spell.

}

\Spell{Nondetection}{nondetection}
{Abjuration [Ritual]}
{
	\textbf{Level:}
	Ass 3, Drd 4, Rgr 4, Wiz 3, Trickery 3\\
	\textbf{Components:}
	V, S, M\\
	\textbf{Casting Time:}
	1 standard action\\
	\textbf{Range:}
	Touch\\
	\textbf{Target:}
	Creature or object touched\\
	\textbf{Duration:}
	1 hour/level\\
	\textbf{Saving Throw:}
	Will negates (harmless, object)\\
	\textbf{Spell Resistance:}
	Yes (harmless, object)\\
}
{
	The warded creature or object becomes difficult to detect by divination spells such as \spell{clairaudience/clairvoyance}, \spell{locate object}, and \emph{detect} spells. \emph{Nondetection} also prevents location by such magic items as crystal balls. If a divination is attempted against the warded creature or item, the caster of the divination must succeed on a caster level check (1d20 + caster level) against a DC of 11 + the caster level of the spellcaster who cast \emph{nondetection}. If you cast \emph{nondetection} on yourself or on an item currently in your possession, the DC is 15 + your caster level.

	If cast on a creature, \emph{nondetection} wards the creature's gear as well as the creature itself.

	\textit{Material Component:}
	A pinch of diamond dust worth 50 cp.

}

\input{subsections/spells/obscure-object.tex}
\Spell{Obscuring Mist}{obscuring mist}
{Conjuration (Creation)}
{
	\textbf{Level:}
	Air 1, Clr 1, Drd 1, Wiz 1, Water 1\\
	\textbf{Components:}
	V, S\\
	\textbf{Casting Time:}
	1 standard action\\
	\textbf{Range:}
	6 m\\
	\textbf{Effect:}
	Cloud spreads in 6-m radius from you, 6 m high\\
	\textbf{Duration:}
	1 min./level\\
	\textbf{Saving Throw:}
	None\\
	\textbf{Spell Resistance:}
	No\\
}
{
	A misty vapor arises around you. It is stationary once created. The vapor obscures all sight, including darkvision, beyond 1.5 meter. A creature 1.5 meter away has concealment (attacks have a 20\% miss chance). Creatures farther away have total concealment (50\% miss chance, and the attacker cannot use sight to locate the target).

	A moderate wind (11+ mph), such as from a gust of wind spell, disperses the fog in 4 rounds. A strong wind (21+ mph) disperses the fog in 1 round. A fireball, flame strike, or similar spell burns away the fog in the explosive or fiery spell's area. A wall of fire burns away the fog in the area into which it deals damage.

	This spell does not function underwater.

}

\input{subsections/spells/open-close.tex}
\input{subsections/spells/orders-wrath.tex}
\Spell{Overland Flight}{overland flight}
{Transmutation}
{
	\textbf{Level:}
	Wiz 5\\
	\textbf{Components:}
	V, S\\
	\textbf{Range:}
	Personal\\
	\textbf{Target:}
	You\\
	\textbf{Duration:}
	1 hour/level\\
}
{
	This spell functions like a \spell{fly} spell, except you can fly at a speed of 12 meters (9 meters if wearing medium or heavy armor, or if carrying a medium or heavy load) with average maneuverability. When using this spell for long-distance movement, you can hustle without taking nonlethal damage (a forced march still requires Constitution checks). This means you can cover 96 kilometers in an eight-hour period of flight (or 72 kilometers at a speed of 9 meters).

}

\Spell{Owl's Wisdom, Mass}{mass owl's wisdom}
{Transmutation}
{
	\textbf{Level:}
	Clr 6, Drd 6, Wiz 6\\
	\textbf{Range:}
	Close (7.5 m + 1.5 m/2 levels)\\
	\textbf{Target:}
	One creature/level, no two of which can be more than 9 m apart\\
}
{
	This spell functions like \spell{owl's wisdom}, except that it affects multiple creatures.

}

\input{subsections/spells/owls-wisdom.tex}
\Spell{Passwall}{passwall}
{Transmutation}
{
	\textbf{Level:}
	Wiz 5\\
	\textbf{Components:}
	V, S, M\\
	\textbf{Casting Time:}
	1 standard action\\
	\textbf{Range:}
	Touch\\
	\textbf{Effect:}
	1.5 m by 2.4 m opening, 3 m deep plus 1.5 m deep per three additional levels\\
	\textbf{Duration:}
	1 hour/level (D)\\
	\textbf{Saving Throw:}
	None\\
	\textbf{Spell Resistance:}
	No\\
}
{
	You create a passage through wooden, plaster, or stone walls, but not through metal or other harder materials. The passage is 3 meters deep plus an additional 1.5 meter deep per three caster levels above 9th (4.5 meters at 12th, 6 meters at 15th, and a maximum of 7.5 meters deep at 18th level). If the wall's thickness is more than the depth of the passage created, then a single \emph{passwall} simply makes a niche or short tunnel. Several \emph{passwall} spells can then form a continuing passage to breach very thick walls. When \emph{passwall} ends, creatures within the passage are ejected out the nearest exit. If someone dispels the \emph{passwall} or you dismiss it, creatures in the passage are ejected out the far exit, if there is one, or out the sole exit if there is only one.

	\textit{Material Component:}
	A pinch of sesame seeds.

}

\input{subsections/spells/pass-without-trace.tex}
\Spell{Permanency}{permanency}
{Universal}
{
	\textbf{Level:}
	Wiz 5\\
	\textbf{Components:}
	V, S, XP\\
	\textbf{Casting Time:}
	2 rounds\\
	\textbf{Range:}
	See text\\
	\textbf{Target, Effect, or Area:}
	See text\\
	\textbf{Duration:}
	Permanent; see text\\
	\textbf{Saving Throw:}
	None\\
	\textbf{Spell Resistance:}
	No\\
}
{
	This spell makes certain other spells permanent.

	Depending on the spell, you must be of a minimum caster level and must expend a number of XP.

	You can make the following spells permanent in regard to yourself.

\Table{}{Xll}{
\tableheader Spell & \tableheader Minimum Caster Level & \tableheader XP Cost\\
 	\spell{arcane sight} & 11th & 1,500 XP\\
	\spell{comprehend languages} & 9th & 500 XP\\
	\spell{darkvision} & 10th & 1,000 XP\\
	\spell{detect magic} & 9th & 500 XP\\
	\spell{read magic} & 9th & 500 XP\\
	\spell{see invisibility} & 10th & 1,000 XP\\
	\spell{tongues} & 11th & 1,500 XP\\
}

	You cast the desired spell and then follow it with the permanency spell. You cannot cast these spells on other creatures. This application of permanency can be dispelled only by a caster of higher level than you were when you cast the spell.

	In addition to personal use, permanency can be used to make the following spells permanent on yourself, another creature, or an object (as appropriate).

\Table{}{Xll}{
\tableheader Spell & \tableheader Minimum Caster Level & \tableheader XP Cost\\
	\spell{enlarge person} & 9th & 500 XP\\
	\spell{magic fang} & 9th & 500 XP\\
	\spell{magic fang, greater} & 11th & 1,500 XP\\
	\spell{reduce person} & 9th & 500 XP\\
	\spell{resistance} & 9th & 500 XP\\
	\spell{telepathic bond}\footnotemark[1] & 13th & 2,500 XP\\

\TableNote{3}{1 Only bonds two creatures per casting of permanency.}
}

	Additionally, the following spells can be cast upon objects or areas only and rendered permanent.

\Table{}{Xll}{
\tableheader Spell & \tableheader Minimum Caster Level & \tableheader XP Cost\\
 	\spell{alarm} & 9th & 500 XP\\
	\spell{animate objects} & 14th & 3,000 XP\\
	\spell{dancing lights} & 9th & 500 XP\\
	\spell{ghost sound} & 9th & 500 XP\\
	\spell{gust of wind} & 11th & 1,500 XP\\
	\spell{invisibility} & 10th & 1,000 XP\\
	\spell{mage's private sanctum} & 13th & 2,500 XP\\
	\spell{magic mouth} & 10th & 1,000 XP\\
	\spell{phase door} & 15th & 3,500 XP\\
	\spell{prismatic sphere} & 17th & 4,500 XP\\
	\spell{prismatic wall} & 16th & 4,000 XP\\
	\spell{shrink item} & 11th & 1,500 XP\\
	\spell{solid fog} & 12th & 2,000 XP\\
	\spell{stinking cloud} & 11th & 1,500 XP\\
	\spell{symbol of death} & 16th & 4,000 XP\\
	\spell{symbol of fear} & 14th & 3,000 XP\\
	\spell{symbol of insanity} & 16th & 4,000 XP\\
	\spell{symbol of pain} & 13th & 2,500 XP\\
	\spell{symbol of persuasion} & 14th & 3,000 XP\\
	\spell{symbol of sleep} & 16th & 4,000 XP\\
	\spell{symbol of stunning} & 15th & 3,500 XP\\
	\spell{symbol of weakness} & 15th & 3,500 XP\\
	\spell{teleportation circle} & 17th & 4,500 XP\\
	\spell{wall of fire} & 12th & 2,000 XP\\
	\spell{wall of force} & 13th & 2,500 XP\\
	\spell{web} & 10th & 1,000 XP\\
}

	Spells cast on other creatures, objects, or locations (not on you) are vulnerable to dispel magic as normal.

	\textit{XP Cost}:
	See tables above.

}

\Spell{Permanent Image}{permanent image}
{Illusion (Figment) [Ritual]}
{
	\textbf{Level:}
	Wiz 6\\
	\textbf{Components:}
	V, S, M, F\\
	\textbf{Effect:}
	Figment that cannot extend beyond a 6-m cube + one 3-m cube/level (S)\\
	\textbf{Duration:}
	Permanent (D)\\
}
{
	This spell functions like \spell{silent image}, except that the figment includes visual, auditory, olfactory, and thermal elements, and the spell is permanent. By concentrating, you can move the image within the limits of the range, but it is static while you are not concentrating.

	\textit{Material Component:}
	 Powdered jade worth 100 cp.

	\textit{Focus:}
	A bit of fleece.

}

\input{subsections/spells/persistent-image.tex}
\input{subsections/spells/phantasmal-killer.tex}
\Spell{Phantom Steed}{phantom steed}
{Conjuration (Creation)}
{
	\textbf{Level:}
	Travel 3, Wiz 3\\
	\textbf{Components:}
	V, S\\
	\textbf{Casting Time:}
	10 minutes\\
	\textbf{Range:}
	0 m\\
	\textbf{Effect:}
	One quasi-real, horselike creature\\
	\textbf{Duration:}
	1 hour/level (D)\\
	\textbf{Saving Throw:}
	None\\
	\textbf{Spell Resistance:}
	No\\
}
{
	You conjure a Large, quasi-real, horselike creature. The steed can be ridden only by you or by the one person for whom you specifically created the mount. A phantom steed has a black head and body, gray mane and tail, and smoke-colored, insubstantial hooves that make no sound. It has what seems to be a saddle, bit, and bridle. It does not fight, but animals shun it and refuse to attack it.

	The mount has an AC of 18 (-1 size, +4 natural armor, +5 Dex) and 7 hit points +1 hit point per caster level. If it loses all its hit points, the phantom steed disappears. A phantom steed has a speed of 6 meters per caster level, to a maximum of 72 meters. It can bear its rider's weight plus up to 5 kilograms per caster level.

	These mounts gain certain powers according to caster level. A mount's abilities include those of mounts of lower caster levels.

	\textit{8th Level:}
	The mount can ride over sandy, muddy, or even swampy ground without difficulty or decrease in speed.

	\textit{10th Level:}
	The mount can use water walk at will (as the spell, no action required to activate this ability).

	\textit{12th Level:}
	The mount can use air walk at will (as the spell, no action required to activate this ability) for up to 1 round at a time, after which it falls to the ground.

	\textit{14th Level:}
	The mount can fly at its speed (average maneuverability).

}

\input{subsections/spells/phantom-trap.tex}
\Spell{Phase Door}{phase door}
{Conjuration (Creation)}
{
	\textbf{Level:}
	Wiz 7\\
	\textbf{Components:}
	V\\
	\textbf{Casting Time:}
	1 standard action\\
	\textbf{Range:}
	0 m\\
	\textbf{Effect:}
	Ethereal 1.5 m by 2.4 m opening, 3 m deep + 1.5 m deep per three levels\\
	\textbf{Duration:}
	One usage per two levels\\
	\textbf{Saving Throw:}
	None\\
	\textbf{Spell Resistance:}
	No\\
}
{
	This spell creates an ethereal passage through wooden, plaster, or stone walls, but not other materials. The \emph{phase door} is invisible and inaccessible to all creatures except you, and only you can use the passage. You disappear when you enter the \emph{phase door} and appear when you exit. If you desire, you can take one other creature (Medium or smaller) through the door. This counts as two uses of the door. The door does not allow light, sound, or spell effects through it, nor can you see through it without using it. Thus, the spell can provide an escape route, though certain creatures, such as phase spiders, can follow with ease. A gem of true seeing or similar magic reveals the presence of a \emph{phase door} but does not allow its use.

	A \emph{phase door} is subject to \spell{dispel magic}. If anyone is within the passage when it is dispelled, he is harmlessly ejected just as if he were inside a \spell{passwall} effect.

	You can allow other creatures to use the \emph{phase door} by setting some triggering condition for the door. Such conditions can be as simple or elaborate as you desire. They can be based on a creature's name, identity, or alignment, but otherwise must be based on observable actions or qualities. Intangibles such as level, class, Hit Dice, and hit points don't qualify.

	\emph{Phase door} can be made permanent with a \spell{permanency} spell.

}

\Spell{Planar Ally, Greater}{greater planar ally}
{Conjuration (Calling) [see text for \spell{lesser planar ally}]}
{
	\textbf{Level:}
	Clr 8\\
	\textbf{Effect:}
	Up to three called elementals or outsiders, totaling no more than 18 HD, no two of which can be more than 9 m apart when they appear.\\
}
{
	This spell functions like \spell{lesser planar ally}, except that you may call a single creature of 18 HD or less, or up to three creatures of the same kind whose Hit Dice total no more than 18. The creatures agree to help you and request your return payment together.

	\textit{XP Cost:}
	500 XP.

}

\input{subsections/spells/planar-ally-lesser.tex}
\Spell{Planar Ally}{planar ally}
{Conjuration (Calling) [see text for lesser planar ally]}
{
	\textbf{Level:}
	Clr 6\\
	\textbf{Effect:}
	One or two called elementals or outsiders, totaling no more than 12 HD, which cannot be more than 9 m apart when they appear\\
}
{
	This spell functions like \spell{lesser planar ally}, except you may call a single creature of 12 HD or less, or two creatures of the same kind whose Hit Dice total no more than 12. The creatures agree to help you and request your return payment together.

	\textit{XP Cost}:
	250 XP.

}

\Spell{Planar Binding, Greater}{greater planar binding}
{Conjuration (Calling) [see text for \spell{lesser planar binding}]}
{
	\textbf{Level:}
	Wiz 8\\
	\textbf{Components:}
	V, S\\
	\textbf{Targets:}
	Up to three elementals or outsiders, totaling no more than 18 HD, no two of which can be more than 9 m apart when they appear.\\
}
{
	This spell functions like \spell{lesser planar binding}, except that you may call a single creature of 18 HD or less, or up to three creatures of the same kind whose Hit Dice total no more than 18. Each creature gets a saving throw, makes independent attempts to escape, and must be persuaded to aid you individually.

}

\input{subsections/spells/planar-binding-lesser.tex}
\Spell{Planar Binding}{planar binding}
{Conjuration (Calling) [see text for lesser planar binding]}
{
	\textbf{Level:}
	Wiz 6\\
	\textbf{Components:}
	V, S\\
	\textbf{Targets:}
	Up to three elementals or outsiders, totaling no more than 12 HD, no two of which can be more than 9 m apart when they appear\\
}
{
	This spell functions like \spell{lesser planar binding}, except that you may call a single creature of 12 HD or less, or up to three creatures of the same kind whose Hit Dice total no more than 12. Each creature gets a save, makes an independent attempt to escape, and must be individually persuaded to aid you.

}

\input{subsections/spells/plane-shift.tex}
\Spell{Plant Growth}{plant growth}
{Transmutation [Ritual]}
{
	\textbf{Level:}
	Agriculture 3, Drd 3, Growth 3, Plant 3, Rgr 3\\
	\textbf{Components:}
	V, S, DF\\
	\textbf{Casting Time:}
	1 standard action\\
	\textbf{Range:}
	See text\\
	\textbf{Target or Area:}
	See text\\
	\textbf{Duration:}
	Instantaneous\\
	\textbf{Saving Throw:}
	None\\
	\textbf{Spell Resistance:}
	No\\
}
{
	\emph{Plant growth} has different effects depending on the version chosen.

	\textit{Overgrowth:}
	This effect causes normal vegetation (grasses, briars, bushes, creepers, thistles, trees, vines) within long range (120 meters + 12 meters per caster level) to become thick and overgrown. The plants entwine to form a thicket or jungle that creatures must hack or force a way through. Speed drops to 1.5 meter, or 3 meters for Large or larger creatures. The area must have brush and trees in it for this spell to take effect.

	At your option, the area can be a 30-meter-radius circle, a 45-meter-radius semicircle, or a 60-meter-radius quarter circle.

	You may designate places within the area that are not affected.

	\textit{Enrichment:}
	This effect targets plants within a range of one-half mile, raising their potential productivity over the course of the next year to one-third above normal.

	\emph{Plant growth} counters \spell{diminish plants}.

	This spell has no effect on plant creatures.

}

\input{subsections/spells/poison.tex}
\Spell{Polar Ray}{polar ray}
{Evocation [Cold]}
{
	\textbf{Level:}
	Wiz 8\\
	\textbf{Components:}
	V, S, F\\
	\textbf{Casting Time:}
	1 standard action\\
	\textbf{Range:}
	Close (7.5 m + 1.5 m/2 levels)\\
	\textbf{Effect:}
	Ray\\
	\textbf{Duration:}
	Instantaneous\\
	\textbf{Saving Throw:}
	None\\
	\textbf{Spell Resistance:}
	Yes\\
}
{
	A blue-white ray of freezing air and ice springs from your hand. You must succeed on a ranged touch attack with the ray to deal damage to a target. The ray deals 1d6 points of cold damage per caster level (maximum 25d6).

	\textit{Focus}:
	A small, white ceramic cone or prism.

}

\Spell{Polymorph Any Object}{polymorph any object}
{Transmutation}
{
	\textbf{Level:}
	Wiz 8, Trickery 8\\
	\textbf{Components:}
	V, S, M/DF\\
	\textbf{Casting Time:}
	1 standard action\\
	\textbf{Range:}
	Close (7.5 m + 1.5 m/2 levels)\\
	\textbf{Target:}
	One creature, or one nonmagical object of up to 100 cu. ft./level\\
	\textbf{Duration:}
	See text\\
	\textbf{Saving Throw:}
	Fortitude negates (object); see text\\
	\textbf{Spell Resistance:}
	Yes (object)\\
}
{
	This spell functions like \spell{polymorph}, except that it changes one object or creature into another. The duration of the spell depends on how radical a change is made from the original state to its enchanted state. The duration is determined by using the following guidelines.

\Table{}{XX}{
	\tableheader Changed Subject Is: & \tableheader Increase to Duration Factor\footnotemark[1]\\
	Same kingdom (animal, vegetable, mineral) & +5\\
	Same class (mammals, fungi, metals, etc.) & +2\\
	Same size & +2\\
	Related (twig is to tree, wolf fur is to wolf, etc.) & +2\\
	Same or lower Intelligence & +2\\

\TableNote{2}{1 Add all that apply. Look up the total on the next table.}
}

\Table{}{XXX}{
\tableheader Duration Factor & \tableheader Duration & \tableheader Example\\
	0 & 20 minutes & Pebble to human\\
	2 & 1 hour & Marionette to human\\
	4 & 3 hours & Human to marionette\\
	5 & 12 hours & Lizard to manticore\\
	6 & 2 days & Sheep to wool coat\\
	7 & 1 week & Shrew to manticore\\
	9+ & Permanent & Manticore to shrew\\
}

	Unlike polymorph, polymorph any object does grant the creature the Intelligence score of its new form. If the original form didn't have a Wisdom or Charisma score, it gains those scores as appropriate for the new form.

	Damage taken by the new form can result in the injury or death of the polymorphed creature. In general, damage occurs when the new form is changed through physical force.

	A nonmagical object cannot be made into a magic item with this spell. Magic items aren't affected by this spell.

	This spell cannot create material of great intrinsic value, such as copper, silver, gems, silk, gold, platinum, mithral, or adamantine. It also cannot reproduce the special properties of cold iron in order to overcome the damage reduction of certain creatures.

	This spell can also be used to duplicate the effects of baleful polymorph, polymorph, flesh to stone, stone to flesh, transmute mud to rock, transmute metal to wood, or transmute rock to mud.

	\textit{Arcane Material Component}:
	Mercury, gum arabic, and smoke.

}

\input{subsections/spells/polymorph.tex}
\Spell{Power Word Blind}{power word blind}
{Enchantment (Compulsion) [Mind-Affecting]}
{
	\textbf{Level:}
	Wiz 7, War 7\\
	\textbf{Components:}
	V\\
	\textbf{Casting Time:}
	1 standard action\\
	\textbf{Range:}
	Close (7.5 m + 1.5 m/2 levels)\\
	\textbf{Target:}
	One creature with 200 hp or less\\
	\textbf{Duration:}
	See text\\
	\textbf{Saving Throw:}
	None\\
	\textbf{Spell Resistance:}
	Yes\\
}
{
\Table{}{XX}{
\tableheader Hit Points & \tableheader Duration\\
	50 or less & Permanent\\
	51--100 & 1d4+1 minutes\\
	101-=200 & 1d4+1 rounds\\
}

	You utter a single word of power that causes one creature of your choice to become blinded, whether the creature can hear the word or not. The duration of the spell depends on the target's current hit point total. Any creature that currently has 201 or more hit points is unaffected by power word blind.

}

\Spell{Power Word Kill}{power word kill}
{Enchantment (Compulsion) [Death, Mind-Affecting]}
{
	\textbf{Level:}
	Wiz 9, War 9\\
	\textbf{Components:}
	V\\
	\textbf{Casting Time:}
	1 standard action\\
	\textbf{Range:}
	Close (7.5 m + 1.5 m/2 levels)\\
	\textbf{Target:}
	One living creature with 100 hp or less\\
	\textbf{Duration:}
	Instantaneous\\
	\textbf{Saving Throw:}
	None\\
	\textbf{Spell Resistance:}
	Yes\\
}
{
	You utter a single word of power that instantly kills one creature of your choice, whether the creature can hear the word or not. Any creature that currently has 101 or more hit points is unaffected by power word kill.

}

\Spell{Power Word Stun}{power word stun}
{Enchantment (Compulsion) [Mind-Affecting]}
{
	\textbf{Level:}
	Wiz 8, War 8\\
	\textbf{Components:}
	V\\
	\textbf{Casting Time:}
	1 standard action\\
	\textbf{Range:}
	Close (7.5 m + 1.5 m/2 levels)\\
	\textbf{Target:}
	One creature with 150 hp or less\\
	\textbf{Duration:}
	See text\\
	\textbf{Saving Throw:}
	None\\
	\textbf{Spell Resistance:}
	Yes\\
}
{
\Table{}{XX}{
\tableheader Hit Points & \tableheader Duration\\
	50 or less & 4d4 rounds\\
	51--100 & 2d4 rounds\\
	101--150 & 1d4 rounds\\
}

	You utter a single word of power that instantly causes one creature of your choice to become stunned, whether the creature can hear the word or not. The duration of the spell depends on the target's current hit point total. Any creature that currently has 151 or more hit points is unaffected by power word stun.

}

\input{subsections/spells/prayer.tex}
\Spell{Prestidigitation}{prestidigitation}
{Universal}
{
	\textbf{Level:}
	Wiz 0\\
	\textbf{Components:}
	V, S\\
	\textbf{Casting Time:}
	1 standard action\\
	\textbf{Range:}
	3 m\\
	\textbf{Target, Effect, or Area:}
	See text\\
	\textbf{Duration:}
	1 hour\\
	\textbf{Saving Throw:}
	See text\\
	\textbf{Spell Resistance:}
	No\\
}
{
	Prestidigitations are minor tricks that novice spellcasters use for practice. Once cast, a prestidigitation spell enables you to perform simple magical effects for 1 hour. The effects are minor and have severe limitations. A prestidigitation can slowly lift 0.5 kilogram of material. It can color, clean, or soil items in a 30-centimeter cube each round. It can chill, warm, or flavor 0.5 kilogram of nonliving material. It cannot deal damage or affect the concentration of spellcasters. Prestidigitation can create small objects, but they look crude and artificial. The materials created by a prestidigitation spell are extremely fragile, and they cannot be used as tools, weapons, or spell components. Finally, a prestidigitation lacks the power to duplicate any other spell effects. Any actual change to an object (beyond just moving, cleaning, or soiling it) persists only 1 hour.

}

\Spell{Prismatic Sphere}{prismatic sphere}
{Abjuration}
{
	\textbf{Level:}
	Protection 9, Wiz 9, Sun 9\\
	\textbf{Components:}
	V\\
	\textbf{Range:}
	3 m\\
	\textbf{Effect:}
	3-m-radius sphere centered on you\\
}
{
	This spell functions like \spell{prismatic wall}, except you conjure up an immobile, opaque globe of shimmering, multicolored light that surrounds you and protects you from all forms of attack. The sphere flashes in all colors of the visible spectrum.

	The sphere's blindness effect on creatures with less than 8 HD lasts 2d4 $\times$ 10 minutes.

	You can pass into and out of the prismatic sphere and remain near it without harm. However, when you're inside it, the sphere blocks any attempt to project something through the sphere (including spells). Other creatures that attempt to attack you or pass through suffer the effects of each color, one at a time.

	Typically, only the upper hemisphere of the globe will exist, since you are at the center of the sphere, so the lower half is usually excluded by the floor surface you are standing on.

	The colors of the sphere have the same effects as the colors of a prismatic wall.

	Prismatic sphere can be made permanent with a permanency spell.

}

% \Spell{Prismatic Spray}{prismatic spray}
{Evocation}
{
	\textbf{Level:}
	Wiz 7\\
	\textbf{Components:}
	V, S\\
	\textbf{Casting Time:}
	1 standard action\\
	\textbf{Range:}
	18 m\\
	\textbf{Area:}
	Cone-shaped burst\\
	\textbf{Duration:}
	Instantaneous\\
	\textbf{Saving Throw:}
	See text\\
	\textbf{Spell Resistance:}
	Yes\\
}
{
	This spell causes seven shimmering, intertwined, multicolored beams of light to spray from your hand. Each beam has a different power. Creatures in the area of the spell with 8 HD or less are automatically blinded for 2d4 rounds. Every creature in the area is randomly struck by one or more beams, which have additional effects.

	\Table{}{llX}{
\tableheader 1d8 & \tableheader Color of Beam & \tableheader Effect\\
 	1 & Red & 20 points fire damage (Reflex half)\\
	2 & Orange & 40 points acid damage (Reflex half)\\
	3 & Yellow & 80 points electricity damage (Reflex half)\\
	4 & Green & Poison (Kills; Fortitude partial, take 1d6 points of Con damage instead)\\
	5 & Blue & Turned to stone (Fortitude negates)\\
	6 & Indigo & Insane, as insanity spell (Will negates)\\
	7 & Violet & Sent to another plane (Will negates)\\
	8 & \multicolumn{2}{l}{Struck by two rays; roll twice more, ignoring any `8' results.}\\
}


}

% \Spell{Prismatic Wall}{prismatic wall}
{Abjuration}
{
	\textbf{Level:}
	Wiz 8\\
	\textbf{Components:}
	V, S\\
	\textbf{Casting Time:}
	1 standard action\\
	\textbf{Range:}
	Close (7.5 m + 1.5 m/2 levels)\\
	\textbf{Effect:}
	Wall 4 ft./level wide, 2 ft./level high\\
	\textbf{Duration:}
	10 min./level (D)\\
	\textbf{Saving Throw:}
	See text\\
	\textbf{Spell Resistance:}
	See text\\
}
{
	Prismatic wall creates a vertical, opaque wall---a shimmering, multicolored plane of light that protects you from all forms of attack. The wall flashes with seven colors, each of which has a distinct power and purpose. The wall is immobile, and you can pass through and remain near the wall without harm. However, any other creature with less than 8 HD that is within 6 meters of the wall is blinded for 2d4 rounds by the colors if it looks at the wall.

	The wall's maximum proportions are 4 feet wide per caster level and 2 feet high per caster level. A prismatic wall spell cast to materialize in a space occupied by a creature is disrupted, and the spell is wasted.

	Each color in the wall has a special effect. The accompanying table shows the seven colors of the wall, the order in which they appear, their effects on creatures trying to attack you or pass through the wall, and the magic needed to negate each color.

	The wall can be destroyed, color by color, in consecutive order, by various magical effects; however, the first color must be brought down before the second can be affected, and so on. A rod of cancellation or a mage's disjunction spell destroys a prismatic wall, but an antimagic field fails to penetrate it. Dispel magic and greater dispel magic cannot dispel the wall or anything beyond it. Spell resistance is effective against a prismatic wall, but the caster level check must be repeated for each color present.

	Prismatic wall can be made permanent with a permanency spell.

{

Color
Order
Effect of Color
Negated By




The violet effect makes the special effects of the other six colors redundant, but these six effects are included here because certain magic items can create prismatic effects one color at a time, and spell resistance might render some colors ineffective (see above).

	\\


	Red\\
	1st\\
	Stops nonmagical ranged weapons.Deals 20 points of fire damage (Reflex half).\\
	Cone of cold\\
	Orange\\
	2nd\\
	Stops magical ranged weapons.Deals 40 points of acid damage (Reflex half).\\
	Gust of wind\\
	Yellow\\
	3rd\\
	Stops poisons, gases, and petrification.Deals 80 points of electricity damage (Reflex half).\\
	Disintegrate\\
	Green\\
	4th\\
	Stops breath weapons.Poison (Kills; Fortitude partial for 1d6 points of Con damage instead).\\
	Passwall\\
	Blue\\
	5th\\
	Stops divination and mental attacks.Turned to stone (Fortitude negates).\\
	Magic missile\\
	Indigo\\
	6th\\
	Stops all spells.Will save or become insane (as insanity spell).\\
	Daylight\\
	Violet\\
	7th\\
	Energy field destroys all objects and effects.1Creatures sent to another plane (Will negates).\\
	Dispel magic\\

}
{
}

\Spell{Probe Thoughts}{probe thoughts}
{Divination [Mind-Affecting]}
{
	\textbf{Level}: Mind 6, Wiz 6\\
	\textbf{Components}: V, S\\
	\textbf{Casting Time}: 1 minute\\
	\textbf{Range}: Close (7.5 m + 1.5 m/2 levels)\\
	\textbf{Target}: One living creature\\
	\textbf{Duration}: Concentration\\
	\textbf{Saving Throw}: Fortitude negates; see text\\
	\textbf{Spell Resistance}: Yes\\
}
{
	All the subject's memories and knowledge are accessible to you, ranging from memories deep below the surface to those still easily called to mind. You can learn the answer to one question per round, to the best of the subject's knowledge. You can also probe a sleeping subject, though the subject may make a Will save against the DC of the probe thoughts spell to wake after each question. Subjects who do not wish to be probed can attempt to move beyond the power's range, unless somehow hindered. You pose the questions telepathically, and the answers to those questions are imparted directly to your mind. You and the target do not need to speak the same language, though less intelligent creatures may yield up only appropriate visual images in answer to your questions.
}
\input{subsections/spells/produce-flame.tex}
\input{subsections/spells/programmed-image.tex}
\input{subsections/spells/project-image.tex}
\input{subsections/spells/protection-from-arrows.tex}
\Spell{Protection from Chaos}{protection from chaos}
{Abjuration [Lawful]}
{
	\textbf{Level:}
	Clr 1, Law 1, Pal 1, Wiz 1\\
}
{
	This spell functions like \spell{protection from evil}, except that the deflection and resistance bonuses apply to attacks from chaotic creatures, and chaotic summoned creatures cannot touch the subject.

}

\Spell{Protection from Energy}{protection from energy}
{Abjuration}
{
	\textbf{Level:}
	Clr 3, Drd 3, Protection 3, Rgr 2, Tmp 3, Wiz 3\\
	\textbf{Components:}
	V, S, DF\\
	\textbf{Casting Time:}
	1 standard action\\
	\textbf{Range:}
	Touch\\
	\textbf{Target:}
	Creature touched\\
	\textbf{Duration:}
	10 min./level or until discharged\\
	\textbf{Saving Throw:}
	Fortitude negates (harmless)\\
	\textbf{Spell Resistance:}
	Yes (harmless)\\
}
{
	\emph{Protection from energy} grants temporary immunity to the type of energy you specify when you cast it (acid, cold, electricity, fire, or sonic). When the spell absorbs 12 points per caster level of energy damage (to a maximum of 120 points at 10th level), it is discharged.

	\textit{Note:} \emph{Protection from energy} overlaps (and does not stack with) \spell{resist energy}. If a character is warded by \emph{protection from energy} and \spell{resist energy}, the protection spell absorbs damage until its power is exhausted.

}

\input{subsections/spells/protection-from-evil.tex}
\Spell{Protection from Good}{protection from good}
{Abjuration [Evil]}
{
	\textbf{Level:}
	Clr 1, Tmp 1, Wiz 1\\
}
{
	This spell functions like \spell{protection from evil}, except that the deflection and resistance bonuses apply to attacks from good creatures, and good summoned creatures cannot touch the subject.

}

\Spell{Protection from Law}{protection from law}
{Abjuration [Chaotic]}
{
	\textbf{Level:}
	Chaos 1, Clr 1, Wiz 1\\
}
{
	This spell functions like protection from evil, except that the deflection and resistance bonuses apply to attacks from lawful creatures, and lawful summoned creatures cannot touch the subject.

}

\Spell{Protection from Spells}{protection from spells}
{Abjuration}
{
	\textbf{Level:}
	Magic 8, Wiz 8\\
	\textbf{Components:}
	V, S, M, F\\
	\textbf{Casting Time:}
	1 standard action\\
	\textbf{Range:}
	Touch\\
	\textbf{Targets:}
	Up to one creature touched per four levels\\
	\textbf{Duration:}
	10 min./level\\
	\textbf{Saving Throw:}
	Will negates (harmless)\\
	\textbf{Spell Resistance:}
	Yes (harmless)\\
}
{
	The subject gains a +8 resistance bonus on saving throws against spells and spell-like abilities (but not against supernatural and extraordinary abilities).

	\textit{Material Component}:
	A diamond of at least 500 cp value, which must be crushed and sprinkled over the targets.

	\textit{Focus}:
	One 1,000 cp diamond per creature to be granted the protection. Each subject must carry one such gem for the duration of the spell. If a subject loses the gem, the spell ceases to affect him.

}

\Spell{Prying Eyes, Greater}{greater prying eyes}
{Divination}
{
	\textbf{Level:}
	Wiz 8\\
}
{
	This spell functions like \spell{prying eyes}, except that the eyes can see all things as they actually are, just as if they had true seeing with a range of 36 meters. Thus, they can navigate darkened areas at full normal speed. Also, a greater prying eye's maximum Spot modifier is +25 instead of +15.

}

\input{subsections/spells/prying-eyes.tex}
\Spell{Psychic Turmoil, Greater}{greater psychic turmoil}
{Abjuration}
{
	\textbf{Level:} Clr 7, Wiz 7\\
	\textbf{Duration:} 1 round/level\\
}
{
	As \spell{psychic turmoil}, except you gain 1 temporary hit point for each power point the spell takes from a psionic creature. The temporary hit points last for 1 hour.
}
\Spell{Psychic Turmoil}{psychic turmoil}
{Abjuration}
{
	\textbf{Level:} Clr 5, Wiz 5\\
	\textbf{Components:} V, S, M\\
	\textbf{Casting Time:} 1 standard action\\
	\textbf{Range:} Close (7.5 m + 1.5 m/2 levels)\\
	\textbf{Area:} 12-m-radius emanation centered on a point in space\\
	\textbf{Duration:} 1 round/level\\
	\textbf{Saving Throw:} Will partial; see text\\
	\textbf{Spell Resistance:} Yes\\
}
{
	With this spell, you create an invisible field that leeches away the power points of psionic characters standing within the emanation. Nonpsionic characters are unaffected.

	When the spell is cast and at the beginning of each of your subsequent turns, psionic creatures within the area of the psychic turmoil lose 1 power point per manifester level they have. Characters who succeed on a Will save when they first come into contact with the emanation lose only half as many power points (round down) each round. Characters get only one save attempt against any particular psychic turmoil effect, even if they leave the spell's area and later return.

	\textit{Material Component:} Five playing cards, which are torn in half when the spell is cast.
}
\Spell{Purify Food and Drink}{purify food and drink}
{Transmutation}
{
	\textbf{Level:}
	Clr 0, Drd 0, ElC 1, Rng 1\\
	\textbf{Components:}
	V, S\\
	\textbf{Casting Time:}
	1 standard action\\
	\textbf{Range:}
	3 m\\
	\textbf{Target:}
	0.03 m$^3$/level of contaminated food and water\\
	\textbf{Duration:}
	Instantaneous\\
	\textbf{Saving Throw:}
	Will negates (object)\\
	\textbf{Spell Resistance:}
	Yes (object)\\
}
{
	This spell makes spoiled, rotten, poisonous, or otherwise contaminated food and water pure and suitable for eating and drinking. This spell does not prevent subsequent natural decay or spoilage. Unholy water and similar food and drink of significance is spoiled by \emph{purify food and drink}, but the spell has no effect on creatures of any type nor upon magic potions.

	\textbf{Note:} Water weighs about 1 kilogram per liter. 0.03 m$^3$ of water contains roughly 30 liters and weighs about 30 kilograms.

}

\Spell{Pyrotechnics}{pyrotechnics}
{Transmutation}
{
	\textbf{Level:}
	Wiz 2\\
	\textbf{Components:}
	V, S, M\\
	\textbf{Casting Time:}
	1 standard action\\
	\textbf{Range:}
	Long (120 m + 12 m/level)\\
	\textbf{Target:}
	One fire source, up to a 6-meter cube\\
	\textbf{Duration:}
	1d4+1 rounds, or 1d4+1 rounds after creatures leave the smoke cloud; see text\\
	\textbf{Saving Throw:}
	Will negates or Fortitude negates; see text\\
	\textbf{Spell Resistance:}
	Yes or No; see text\\
}
{
	Pyrotechnics turns a fire into either a burst of blinding fireworks or a thick cloud of choking smoke, depending on the version you choose.

	\textit{Fireworks:} The fireworks are a flashing, fiery, momentary burst of glowing, colored aerial lights. This effect causes creatures within 36 meters of the fire source to become blinded for 1d4+1 rounds (Will negates). These creatures must have line of sight to the fire to be affected. Spell resistance can prevent blindness.

	\textit{Smoke Cloud:} A writhing stream of smoke billows out from the source, forming a choking cloud. The cloud spreads 6 meters in all directions and lasts for 1 round per caster level. All sight, even darkvision, is ineffective in or through the cloud. All within the cloud take $-4$ penalties to Strength and Dexterity (Fortitude negates). These effects last for 1d4+1 rounds after the cloud dissipates or after the creature leaves the area of the cloud. Spell resistance does not apply.

	\textit{Material Component:} The spell uses one fire source, which is immediately extinguished. A fire so large that it exceeds a 6-meter cube is only partly extinguished. Magical fires are not extinguished, although a fire-based creature used as a source takes 1 point of damage per caster level.
}

\input{subsections/spells/quench.tex}
\input{subsections/spells/rage.tex}
\Spell{Rainbow Pattern}{rainbow pattern}
{Illusion (Pattern) [Mind-Affecting]}
{
	\textbf{Level:}
	Wiz 4\\
	\textbf{Components:}
	S, M, F\\
	\textbf{Casting Time:}
	1 standard action\\
	\textbf{Range:}
	Medium (30 m + 3 m/level)\\
	\textbf{Effect:}
	Colorful lights with a 6-m-radius spread\\
	\textbf{Duration:}
	Concentration +1 round/ level (D)\\
	\textbf{Saving Throw:}
	Will negates\\
	\textbf{Spell Resistance:}
	Yes\\
}
{
	A glowing, rainbow-hued pattern of interweaving colors fascinates those within it. Rainbow pattern fascinates a maximum of 24 Hit Dice of creatures. Creatures with the fewest HD are affected first. Among creatures with equal HD, those who are closest to the spell's point of origin are affected first. An affected creature that fails its saves is fascinated by the pattern.

	With a simple gesture (a free action), you can make the \emph{rainbow pattern} move up to 9 meters per round (moving its effective point of origin). All fascinated creatures follow the moving rainbow of light, trying to get or remain within the effect. Fascinated creatures who are restrained and removed from the pattern still try to follow it. If the pattern leads its subjects into a dangerous area each fascinated creature gets a second save. If the view of the lights is completely blocked creatures who can't see them are no longer affected.

	The spell does not affect sightless creatures.

	\textit{Material Component:}
	A piece of phosphor.

	\textit{Focus:}
	A crystal prism.

}

\input{subsections/spells/raise-dead.tex}
\Spell{Ray of Enfeeblement}{ray of enfeeblement}
{Necromancy}
{
	\textbf{Level:}
	Wiz 1\\
	\textbf{Components:}
	V, S\\
	\textbf{Casting Time:}
	1 standard action\\
	\textbf{Range:}
	Close (7.5 m + 1.5 m/2 levels)\\
	\textbf{Effect:}
	Ray\\
	\textbf{Duration:}
	1 min./level\\
	\textbf{Saving Throw:}
	None\\
	\textbf{Spell Resistance:}
	Yes\\
}
{
	A coruscating ray springs from your hand. You must succeed on a ranged touch attack to strike a target. The subject takes a penalty to Strength equal to 1d6+1 per two caster levels (maximum 1d6+5). The subject's Strength score cannot drop below 1.

}

\input{subsections/spells/ray-of-exhaustion.tex}
\Spell{Ray of Frost}{ray of frost}
{Evocation [Cold]}
{
	\textbf{Level:}
	Wiz 0\\
	\textbf{Components:}
	V, S\\
	\textbf{Casting Time:}
	1 standard action\\
	\textbf{Range:}
	Close (7.5 m + 1.5 m/2 levels)\\
	\textbf{Effect:}
	Ray\\
	\textbf{Duration:}
	Instantaneous\\
	\textbf{Saving Throw:}
	None\\
	\textbf{Spell Resistance:}
	Yes\\
}
{
	A ray of freezing air and ice projects from your pointing finger. You must succeed on a ranged touch attack with the ray to deal damage to a target. The ray deals 1d3 points of cold damage.

}

\Spell{Read Magic}{read magic}
{Divination}
{
	\textbf{Level:}
	Clr 0, Drd 0, Rgr 1, Tmp 0, Wiz 0\\
	\textbf{Components:}
	V, S, F\\
	\textbf{Casting Time:}
	1 standard action\\
	\textbf{Range:}
	Personal\\
	\textbf{Target:}
	You\\
	\textbf{Duration:}
	10 min./level\\
}
{
	By means of \emph{read magic}, you can decipher magical inscriptions on objects---books, scrolls, weapons, and the like---that would otherwise be unintelligible. This deciphering does not normally invoke the magic contained in the writing, although it may do so in the case of a cursed scroll. Furthermore, once the spell is cast and you have read the magical inscription, you are thereafter able to read that particular writing without recourse to the use of \emph{read magic}. You can read at the rate of one page (250 words) per minute. The spell allows you to identify a glyph of warding with a DC 13 \skill{Spellcraft} check, a greater glyph of warding with a DC 16 \skill{Spellcraft} check, or any symbol spell with a \skill{Spellcraft} check (DC 10 + spell level).

	\emph{Read magic} can be made permanent with a \spell{permanency} spell.

	\textit{Focus:}
	A clear crystal or mineral prism.

}

\input{subsections/spells/reduce-animal.tex}
\Spell{Reduce Person, Mass}{reduce person, mass}
{Transmutation}
{
	\textbf{Level:}
	Wiz 4\\
	\textbf{Target:}
	One humanoid creature/level, no two of which can be more than 9 m apart\\
}
{
	This spell functions like reduce person, except that it affects multiple creatures.

}

\Spell{Reduce Person}{reduce person}
{Transmutation}
{
	\textbf{Level:}
	Wiz 1\\
	\textbf{Components:}
	V, S, M\\
	\textbf{Casting Time:}
	1 round\\
	\textbf{Range:}
	Close (7.5 m + 1.5 m/2 levels)\\
	\textbf{Target:}
	One humanoid creature\\
	\textbf{Duration:}
	1 min./level (D)\\
	\textbf{Saving Throw:}
	Fortitude negates\\
	\textbf{Spell Resistance:}
	Yes\\
}
{
	This spell causes instant diminution of a humanoid creature, halving its height, length, and width and dividing its weight by 8. This decrease changes the creature's size category to the next smaller one. The target gains a +2 size bonus to Dexterity, a $-2$ size penalty to Strength (to a minimum of 1), and a +1 bonus on attack rolls and AC due to its reduced size.

	A Small humanoid creature whose size decreases to Tiny has a space of 0.75 m and a natural reach of 0 meters (meaning that it must enter an opponent's square to attack). A Large humanoid creature whose size decreases to Medium has a space of 1.5 meter and a natural reach of 1.5 meter. This spell doesn't change the target's speed.

	All equipment worn or carried by a creature is similarly reduced by the spell.

	Melee and projectile weapons deal less damage. Other magical properties are not affected by this spell. Any reduced item that leaves the reduced creature's possession (including a projectile or thrown weapon) instantly returns to its normal size. This means that thrown weapons deal their normal damage (projectiles deal damage based on the size of the weapon that fired them).

	Multiple magical effects that reduce size do not stack.

	\emph{Reduce person} counters and dispels \spell{enlarge person}.

	\emph{Reduce person} can be made permanent with a \spell{permanency} spell.

	\textit{Material Component}:
	A pinch of powdered iron.

}

\input{subsections/spells/refuge.tex}
\input{subsections/spells/regenerate.tex}
% \input{subsections/spells/reincarnate.tex}
\input{subsections/spells/remove-blindness-deafness.tex}
\Spell{Remove Curse}{remove curse}
{Abjuration}
{
	\textbf{Level:}
	Clr 3, Pal 3, Wiz 4\\
	\textbf{Components:}
	V, S\\
	\textbf{Casting Time:}
	1 standard action\\
	\textbf{Range:}
	Touch\\
	\textbf{Target:}
	Creature or item touched\\
	\textbf{Duration:}
	Instantaneous\\
	\textbf{Saving Throw:}
	Will negates (harmless)\\
	\textbf{Spell Resistance:}
	Yes (harmless)\\
}
{
	Remove curse instantaneously removes all curses on an object or a creature. Remove curse does not remove the curse from a cursed shield, weapon, or suit of armor, although the spell typically enables the creature afflicted with any such cursed item to remove and get rid of it. Certain special curses may not be countered by this spell or may be countered only by a caster of a certain level or higher.

	Remove curse counters and dispels bestow curse.

}

\input{subsections/spells/remove-disease.tex}
\Spell{Remove Fear}{remove fear}
{Abjuration}
{
	\textbf{Level:}
	Clr 1\\
	\textbf{Components:}
	V, S\\
	\textbf{Casting Time:}
	1 standard action\\
	\textbf{Range:}
	Close (7.5 m + 1.5 m/2 levels)\\
	\textbf{Targets:}
	One creature plus one additional creature per four levels, no two of which can be more than 9 m apart\\
	\textbf{Duration:}
	10 minutes; see text\\
	\textbf{Saving Throw:}
	Will negates (harmless)\\
	\textbf{Spell Resistance:}
	Yes (harmless)\\
}
{
	You instill courage in the subject, granting it a +4 morale bonus against fear effects for 10 minutes. If the subject is under the influence of a fear effect when receiving the spell, that effect is suppressed for the duration of the spell.

	\emph{Remove fear} counters and dispels \spell{cause fear}.

}

\input{subsections/spells/remove-paralysis.tex}
\input{subsections/spells/repel-metal-or-stone.tex}
\input{subsections/spells/repel-vermin.tex}
\input{subsections/spells/repel-wood.tex}
\input{subsections/spells/repulsion.tex}
\Spell{Resilient Sphere}{resilient sphere}
{Evocation [Force]}
{
	\textbf{Level:}
	Wiz 4\\
	\textbf{Components:}
	V, S, M\\
	\textbf{Casting Time:}
	1 standard action\\
	\textbf{Range:}
	Close (7.5 m + 1.5 m/2 levels)\\
	\textbf{Effect:}
	0.3-m-diameter/level sphere, centered around a creature\\
	\textbf{Duration:}
	1 min./level (D)\\
	\textbf{Saving Throw:}
	Reflex negates\\
	\textbf{Spell Resistance:}
	Yes\\
}
{
	A globe of shimmering force encloses a creature, provided the creature is small enough to fit within the diameter of the sphere. The sphere contains its subject for the spell's duration. The sphere is not subject to damage of any sort except from a rod of cancellation, a rod of negation, a disintegrate spell, or a targeted dispel magic spell. These effects destroy the sphere without harm to the subject. Nothing can pass through the sphere, inside or out, though the subject can breathe normally.

	The subject may struggle, but the sphere cannot be physically moved either by people outside it or by the struggles of those within.

	\textit{Material Component:}
	A hemispherical piece of clear crystal and a matching hemispherical piece of gum arabic.

}

\Spell{Resistance}{resistance}
{Abjuration}
{
	\textbf{Level:}
	Clr 0, Drd 0, Pal 1, Wiz 0\\
	\textbf{Components:}
	V, S, M/DF\\
	\textbf{Casting Time:}
	1 standard action\\
	\textbf{Range:}
	Touch\\
	\textbf{Target:}
	Creature touched\\
	\textbf{Duration:}
	1 minute\\
	\textbf{Saving Throw:}
	Will negates (harmless)\\
	\textbf{Spell Resistance:}
	Yes (harmless)\\
}
{
	You imbue the subject with magical energy that protects it from harm, granting it a +1 resistance bonus on saves.

	\emph{Resistance} can be made permanent with a \spell{permanency} spell.

	\textit{Arcane Material Component:}
	A miniature cloak.

}

\input{subsections/spells/resist-energy.tex}
\input{subsections/spells/restoration-greater.tex}
\input{subsections/spells/restoration-lesser.tex}
\input{subsections/spells/restoration.tex}
\input{subsections/spells/resurrection.tex}
\input{subsections/spells/reverse-gravity.tex}
\input{subsections/spells/righteous-might.tex}
\input{subsections/spells/rope-trick.tex}
\input{subsections/spells/rusting-grasp.tex}
\input{subsections/spells/sanctuary.tex}
\input{subsections/spells/scare.tex}
\input{subsections/spells/scintillating-pattern.tex}
\Spell{Scorching Ray}{scorching ray}
{Evocation [Fire]}
{
	\textbf{Level:}
	Wiz 2\\
	\textbf{Components:}
	V, S\\
	\textbf{Casting Time:}
	1 standard action\\
	\textbf{Range:}
	Close (7.5 m + 1.5 m/2 levels)\\
	\textbf{Effect:}
	One or more rays\\
	\textbf{Duration:}
	Instantaneous\\
	\textbf{Saving Throw:}
	None\\
	\textbf{Spell Resistance:}
	Yes\\
}
{
	You blast your enemies with fiery rays. You may fire one ray, plus one additional ray for every four levels beyond 3rd (to a maximum of three rays at 11th level). Each ray requires a ranged touch attack to hit and deals 4d6 points of fire damage.

	The rays may be fired at the same or different targets, but all bolts must be aimed at targets within 9 meters of each other and fired simultaneously.

}

\input{subsections/spells/screen.tex}
\Spell{Scrying, Greater}{greater scrying}
{Divination (Scrying)}
{
	\textbf{Level:}
	Clr 7, Drd 7, Wiz 7\\
	\textbf{Components:}
	V, S\\
	\textbf{Casting Time:}
	1 standard action\\
	\textbf{Duration:}
	1 hour/level\\
}
{
	This spell functions like \spell{scrying}, except as noted above. Additionally, all of the following spells function reliably through the sensor: \spell{detect chaos}, \spell{detect evil}, \spell{detect good}, \spell{detect law}, \spell{detect magic}, \spell{message}, \spell{read magic}, and \spell{tongues}.

}

% \Spell{Scrying}{scrying}
{Divination (Scrying)}
{
	\textbf{Level:}
	Clr 5, Drd 4, Wiz 4\\
	\textbf{Components:}
	V, S, M/DF, F\\
	\textbf{Casting Time:}
	1 hour\\
	\textbf{Range:}
	See text\\
	\textbf{Effect:}
	Magical sensor\\
	\textbf{Duration:}
	1 min./level\\
	\textbf{Saving Throw:}
	Will negates\\
	\textbf{Spell Resistance:}
	Yes\\
}
{
	You can see and hear some creature, which may be at any distance. If the subject succeeds on a Will save, the scrying attempt simply fails. The difficulty of the save depends on how well you know the subject and what sort of physical connection (if any) you have to that creature. Furthermore, if the subject is on another plane, it gets a +5 bonus on its Will save.

\Table{}{Xl}{
\tableheader Knowledge & \tableheader Will Save Modifier\\
	None\footnotemark[1] & +10\\
	Secondhand (you have heard of the subject) & +5\\
	Firsthand (you have met the subject) & +0\\
	Familiar (you know the subject well) & $-5$\\

\TableNote{2}{1 You must have some sort of connection to a creature you have no knowledge of.}\\
}

\Table{}{Xl}{
\tableheader Connection & \tableheader Will Save Modifier\\
	Likeness or picture & $-2$\\
	Possession or garment & $-4$\\
	Body part, lock of hair, bit of nail, etc. & $-10$\\
}

	If the save fails, you can see and hear the subject and the subject's immediate surroundings (approximately 3 meters in all directions of the subject). If the subject moves, the sensor follows at a speed of up to 45 meters.

	As with all divination (scrying) spells, the sensor has your full visual acuity, including any magical effects. In addition, the following spells have a 5\% chance per caster level of operating through the sensor: detect chaos, detect evil, detect good, detect law, detect magic, and message.

	If the save succeeds, you can't attempt to scry on that subject again for at least 24 hours.

	\textit{Arcane Material Component:}
	The eye of a hawk, an eagle, or a roc, plus nitric acid, copper, and zinc.

	\textit{Wizard Focus:}
	A mirror of finely wrought and highly polished silver costing not less than 1,000 cp. The mirror must be at least 60 centimeters by 1.2 meter.

	\textit{Cleric Focus:}
	A holy water font costing not less than 100 cp.

	\textit{Druid Focus:}
	A natural pool of water.

}

\input{subsections/spells/sculpt-sound.tex}
\input{subsections/spells/searing-light.tex}
\input{subsections/spells/secret-chest.tex}
\Spell{Secret Page}{secret page}
{Transmutation}
{
	\textbf{Level:}
	Wiz 3\\
	\textbf{Components:}
	V, S, M\\
	\textbf{Casting Time:}
	10 minutes\\
	\textbf{Range:}
	Touch\\
	\textbf{Target:}
	Page touched, up to 3 sq. ft. in size\\
	\textbf{Duration:}
	Permanent\\
	\textbf{Saving Throw:}
	None\\
	\textbf{Spell Resistance:}
	No\\
}
{
	Secret page alters the contents of a page so that they appear to be something entirely different. The text of a spell can be changed to show even another spell. Explosive runes or sepia snake sigil can be cast upon the secret page.

	A comprehend languages spell alone cannot reveal a secret page's contents. You are able to reveal the original contents by speaking a special word. You can then peruse the actual page, and return it to its secret page form at will. You can also remove the spell by double repetition of the special word. A detect magic spell reveals dim magic on the page in question but does not reveal its true contents. True seeing reveals the presence of the hidden material but does not reveal the contents unless cast in combination with comprehend languages. A secret page spell can be dispelled, and the hidden writings can be destroyed by means of an erase spell.

	\textit{Material Component:}
	Powdered herring scales and will-o'-wisp essence.

}

\Spell{Secure Shelter}{secure shelter}
{Conjuration (Creation)}
{
	\textbf{Level:}
	Wiz 4\\
	\textbf{Components:}
	V, S, M, F; see text\\
	\textbf{Casting Time:}
	10 minutes\\
	\textbf{Range:}
	Close (7.5 m + 1.5 m/2 levels)\\
	\textbf{Effect:}
	6-m-square structure\\
	\textbf{Duration:}
	2 hours/level (D)\\
	\textbf{Saving Throw:}
	None\\
	\textbf{Spell Resistance:}
	No\\
}
{
	You conjure a sturdy cottage or lodge made of material that is common in the area where the spell is cast. The floor is level, clean, and dry. In all respects the lodging resembles a normal cottage, with a sturdy door, two shuttered windows, and a small fireplace.

	The shelter has no heating or cooling source (other than natural insulation qualities). Therefore, it must be heated as a normal dwelling, and extreme heat adversely affects it and its occupants. The dwelling does, however, provide considerable security otherwise---it is as strong as a normal stone building, regardless of its material composition. The dwelling resists flames and fire as if it were stone. It is impervious to normal missiles (but not the sort cast by siege engines or giants).

	The door, shutters, and even chimney are secure against intrusion, the former two being arcane locked and the latter secured by an iron grate at the top and a narrow flue. In addition, these three areas are protected by an alarm spell. Finally, an unseen servant is conjured to provide service to you for the duration of the shelter.

	The secure shelter contains rude furnishings ---eight bunks, a trestle table, eight stools, and a writing desk.

	\textit{Material Component}:
	A square chip of stone, crushed lime, a few grains of sand, a sprinkling of water, and several splinters of wood. These must be augmented by the components of the unseen servant spell (string and a bit of wood) if this benefit is to be included.

	\textit{Focus}:
	The focus of the alarm spell (silver wire and a tiny bell) if this benefit is to be included.

}

\input{subsections/spells/see-invisibility.tex}
\Spell{Seeming}{seeming}
{Illusion (Glamer)}
{
	\textbf{Level:}
	Wiz 5\\
	\textbf{Components:}
	V, S\\
	\textbf{Casting Time:}
	1 standard action\\
	\textbf{Range:}
	Close (7.5 m + 1.5 m/2 levels)\\
	\textbf{Targets:}
	One creature per two levels, no two of which can be more than 9 m apart\\
	\textbf{Duration:}
	12 hours (D)\\
	\textbf{Saving Throw:}
	Will negates or Will disbelief (if interacted with)\\
	\textbf{Spell Resistance:}
	Yes or No; see text\\
}
{
	This spell functions like disguise self, except that you can change the appearance of other people as well. Affected creatures resume their normal appearances if slain.

	Unwilling targets can negate the spell's effect on them by making Will saves or with spell resistance.

}

\input{subsections/spells/sending.tex}
\input{subsections/spells/sepia-snake-sigil.tex}
\Spell{Sequester}{sequester}
{Abjuration}
{
	\textbf{Level:}
	Wiz 7\\
	\textbf{Components:}
	V, S, M\\
	\textbf{Casting Time:}
	1 standard action\\
	\textbf{Range:}
	Touch\\
	\textbf{Target:}
	One willing creature or object (up to a 0.6-m cube/level) touched\\
	\textbf{Duration:}
	One day/level (D)\\
	\textbf{Saving Throw:}
	None or Will negates (object)\\
	\textbf{Spell Resistance:}
	No or Yes (object)\\
}
{
	When cast, this spell not only prevents divination spells from working to detect or locate the creature or object affected by \emph{sequester}, it also renders the affected creature or object invisible to any form of sight or seeing (as the \spell{invisibility} spell). The spell does not prevent the subject from being discovered through tactile means or through the use of devices. Creatures affected by \emph{sequester} become comatose and are effectively in a state of suspended animation until the spell wears off or is dispelled.

	\textit{Note:} The Will save prevents an attended or magical object from being sequestered. There is no save to see the sequestered creature or object or to detect it with a divination spell.

	\textit{Material Component:}
	A basilisk eyelash, gum arabic, and a dram of whitewash.

}

\Spell{Shades}{shades}
{Illusion (Shadow)}
{
	\textbf{Level:}
	Wiz 9\\
}
{
	This spell functions like shadow conjuration, except that it mimics sorcerer and wizard conjuration spells of 8th level or lower. The illusory conjurations created deal four-fifths (80\%) damage to nonbelievers, and nondamaging effects are 80\% likely to work against nonbelievers.

}

\Spell{Shadow Conjuration, Greater}{shadow conjuration, greater}
{Illusion (Shadow)}
{
	\textbf{Level:}
	Wiz 7\\
}
{
	This spell functions like shadow conjuration, except that it can duplicate any sorcerer or wizard conjuration (summoning) or conjuration (creation) spell of 6th level or lower. The illusory conjurations created deal three-fifths (60\%) damage to nonbelievers, and nondamaging effects are 60\% likely to work against nonbelievers.

}

\input{subsections/spells/shadow-conjuration.tex}
\Spell{Shadow Evocation, Greater}{shadow evocation, greater}
{Illusion (Shadow)}
{
	\textbf{Level:}
	Wiz 8\\
}
{
	This spell functions like \spell{shadow evocation}, except that it enables you to create partially real, illusory versions of sorcerer or wizard evocation spells of 7th level or lower. If recognized as a greater shadow evocation, a damaging spell deals only three-fifths (60\%) damage.

}

\input{subsections/spells/shadow-evocation.tex}
\input{subsections/spells/shadow-walk.tex}
\input{subsections/spells/shambler.tex}
\input{subsections/spells/shapechange.tex}
\input{subsections/spells/shatter.tex}
\input{subsections/spells/shield-of-faith.tex}
\input{subsections/spells/shield-of-law.tex}
\input{subsections/spells/shield-other.tex}
\input{subsections/spells/shield.tex}
\input{subsections/spells/shillelagh.tex}
\input{subsections/spells/shocking-grasp.tex}
\input{subsections/spells/shout-greater.tex}
\input{subsections/spells/shout.tex}
\input{subsections/spells/shrink-item.tex}
\input{subsections/spells/silence.tex}
\Spell{Silent Image}{silent image}
{Illusion (Figment)}
{
	\textbf{Level:}
	Wiz 1\\
	\textbf{Components:}
	V, S, F\\
	\textbf{Casting Time:}
	1 standard action\\
	\textbf{Range:}
	Long (120 m + 12 m/level)\\
	\textbf{Effect:}
	Visual figment that cannot extend beyond four 3-m cubes + one 3-m cube/level (S)\\
	\textbf{Duration:}
	Concentration\\
	\textbf{Saving Throw:}
	Will disbelief (if interacted with)\\
	\textbf{Spell Resistance:}
	No\\
}
{
	This spell creates the visual illusion of an object, creature, or force, as visualized by you. The illusion does not create sound, smell, texture, or temperature. You can move the image within the limits of the size of the effect.

	\textit{Focus}:
	A bit of fleece.

}

\Spell{Simulacrum}{simulacrum}
{Illusion (Shadow)}
{
	\textbf{Level:}
	Mirage 8, Wiz 7\\
	\textbf{Components:}
	V, S, M, XP\\
	\textbf{Casting Time:}
	12 hours\\
	\textbf{Range:}
	0 m\\
	\textbf{Effect:}
	One duplicate creature\\
	\textbf{Duration:}
	Instantaneous\\
	\textbf{Saving Throw:}
	None\\
	\textbf{Spell Resistance:}
	No\\
}
{
	\emph{Simulacrum} creates an illusory duplicate of any creature. The duplicate creature is partially real and formed from ice or snow. It appears to be the same as the original, but it has only one-half of the real creature's levels or Hit Dice (and the appropriate hit points, feats, skill ranks, and special abilities for a creature of that level or HD). You can't create a \emph{simulacrum} of a creature whose Hit Dice or levels exceed twice your caster level. You must make a \skill{Disguise} check when you cast the spell to determine how good the likeness is. A creature familiar with the original might detect the ruse with a successful \skill{Spot} check (opposed by the caster's \skill{Disguise} check) or a DC 20 \skill{Sense Motive} check.

	At all times the \emph{simulacrum} remains under your absolute command. No special telepathic link exists, so command must be exercised in some other manner. A \emph{simulacrum} has no ability to become more powerful. It cannot increase its level or abilities. If reduced to 0 hit points or otherwise destroyed, it reverts to snow and melts instantly into nothingness. A complex process requiring at least 24 hours, 100 cp per hit point, and a fully equipped magical laboratory can repair damage to a \emph{simulacrum}.

	\textit{Material Component:}
	The spell is cast over the rough snow or ice form, and some piece of the creature to be duplicated (hair, nail, or the like) must be placed inside the snow or ice. Additionally, the spell requires powdered ruby worth 100 cp per HD of the \emph{simulacrum} to be created.

	\textit{XP Cost:}
	100 XP per HD of the \emph{simulacrum} to be created (minimum 1,000 XP).

}

\input{subsections/spells/slay-living.tex}
\Spell{Sleep}{sleep}
{Enchantment (Compulsion) [Mind-Affecting]}
{
	\textbf{Level:}
	Ass 1, Wiz 1\\
	\textbf{Components:}
	V, S, M\\
	\textbf{Casting Time:}
	1 round\\
	\textbf{Range:}
	Medium (30 m + 3 m/level)\\
	\textbf{Area:}
	One or more living creatures within a 3-m-radius burst\\
	\textbf{Duration:}
	1 min./level\\
	\textbf{Saving Throw:}
	Will negates\\
	\textbf{Spell Resistance:}
	Yes\\
}
{
	A \emph{sleep} spell causes a magical slumber to come upon 4 Hit Dice of creatures. Creatures with the fewest HD are affected first.

	Among creatures with equal HD, those who are closest to the spell's point of origin are affected first. Hit Dice that are not sufficient to affect a creature are wasted.

	Sleeping creatures are helpless. Slapping or wounding awakens an affected creature, but normal noise does not. Awakening a creature is a standard action (an application of the aid another action).

	Sleep does not target unconscious creatures, constructs, or undead creatures.

	\textit{Material Component:}
	A pinch of fine sand, rose petals, or a live cricket.

}

\Spell{Sleet Storm}{sleet storm}
{Conjuration (Creation) [Cold]}
{
	\textbf{Level:}
	Drd 3, Wiz 3\\
	\textbf{Components:}
	V, S, M/DF\\
	\textbf{Casting Time:}
	1 standard action\\
	\textbf{Range:}
	Long (120 m + 12 m/level)\\
	\textbf{Area:}
	Cylinder (12-m radius, 6 m high)\\
	\textbf{Duration:}
	1 round/level\\
	\textbf{Saving Throw:}
	None\\
	\textbf{Spell Resistance:}
	No\\
}
{
	Driving sleet blocks all sight (even darkvision) within it and causes the ground in the area to be icy. A creature can walk within or through the area of sleet at half normal speed with a DC 10 \skill{Balance} check. Failure means it can't move in that round, while failure by 5 or more means it falls (see the \skill{Balance} skill for details).

	The sleet extinguishes torches and small fires.

	\textit{Arcane Material Component:}
	A pinch of dust and a few drops of water.

}

\input{subsections/spells/slow.tex}
\input{subsections/spells/snare.tex}
\input{subsections/spells/soften-earth-and-stone.tex}
\Spell{Solid Fog}{solid fog}
{Conjuration (Creation)}
{
	\textbf{Level:}
	Wiz 4\\
	\textbf{Components:}
	V, S, M\\
	\textbf{Duration:}
	1 min./level\\
	\textbf{Spell Resistance:}
	No\\
}
{
	This spell functions like \spell{fog cloud}, but in addition to obscuring sight, the \emph{solid fog} is so thick that any creature attempting to move through it progresses at a speed of 1.5 meter, regardless of its normal speed, and it takes a $-2$ penalty on all melee attack and melee damage rolls. The vapors prevent effective ranged weapon attacks (except for magic rays and the like). A creature or object that falls into \emph{solid fog} is slowed, so that each 3 meters of vapor that it passes through reduces falling damage by 1d6. A creature can't take a 1.5-meter step while in \emph{solid fog}.

	However, unlike normal fog, only a severe wind (31+ mph) disperses these vapors, and it does so in 1 round.

	\emph{Solid fog} can be made permanent with a \spell{permanency} spell. A permanent \emph{solid fog} dispersed by wind reforms in 10 minutes.

	\textit{Material Component:}
	A pinch of dried, powdered peas combined with powdered animal hoof.

}

\Spell{Song of Discord}{song of discord}
{Enchantment (Compulsion) [Mind-Affecting, Sonic]}
{
	\textbf{Level:}
	Brd 5\\
	\textbf{Components:}
	V, S\\
	\textbf{Casting Time:}
	1 standard action\\
	\textbf{Range:}
	Medium (30 m + 3 m/level)\\
	\textbf{Area:}
	Creatures within a 6-m-radius spread\\
	\textbf{Duration:}
	1 round/level\\
	\textbf{Saving Throw:}
	Will negates\\
	\textbf{Spell Resistance:}
	Yes\\
}
{
	This spell causes those within the area to turn on each other rather than attack their foes. Each affected creature has a 50\% chance to attack the nearest target each round. (Roll to determine each creature's behavior every round at the beginning of its turn.) A creature that does not attack its nearest neighbor is free to act normally for that round.

	Creatures forced by a song of discord to attack their fellows employ all methods at their disposal, choosing their deadliest spells and most advantageous combat tactics. They do not, however, harm targets that have fallen unconscious.

}

\Spell{Soul Bind}{soul bind}
{Necromancy}
{
	\textbf{Level:}
	Clr 9, Wiz 9\\
	\textbf{Components:}
	V, S, F\\
	\textbf{Casting Time:}
	1 standard action\\
	\textbf{Range:}
	Close (7.5 m + 1.5 m/2 levels)\\
	\textbf{Target:}
	Corpse\\
	\textbf{Duration:}
	Permanent\\
	\textbf{Saving Throw:}
	Will negates\\
	\textbf{Spell Resistance:}
	No\\
}
{
	You draw the soul from a newly dead body and imprison it in a black sapphire gem. The subject must have been dead no more than 1 round per caster level. The soul, once trapped in the gem, cannot be returned through clone, raise dead, reincarnation, resurrection, true resurrection, or even a miracle or a wish. Only by destroying the gem or dispelling the spell on the gem can one free the soul (which is then still dead).

	\textit{Focus}:
	A black sapphire of at least 1,000 cp value for every Hit Die possessed by the creature whose soul is to be bound. If the gem is not valuable enough, it shatters when the binding is attempted. (While creatures have no concept of level or Hit Dice as such, the value of the gem needed to trap an individual can be researched. Remember that this value can change over time as creatures gain more Hit Dice.)

}

\Spell{Sound Burst}{sound burst}
{Evocation [Sonic]}
{
	\textbf{Level:}
	Clr 2\\
	\textbf{Components:}
	V, S, F/DF\\
	\textbf{Casting Time:}
	1 standard action\\
	\textbf{Range:}
	Close (7.5 m + 1.5 m/2 levels)\\
	\textbf{Area:}
	3-m-radius spread\\
	\textbf{Duration:}
	Instantaneous\\
	\textbf{Saving Throw:}
	Fortitude partial\\
	\textbf{Spell Resistance:}
	Yes\\
}
{
	You blast an area with a tremendous cacophony. Every creature in the area takes 1d8 points of sonic damage and must succeed on a Fortitude save to avoid being stunned for 1 round.

	Creatures that cannot hear are not stunned but are still damaged.

	\textit{Arcane Focus}:
	A musical instrument.

}

\input{subsections/spells/speak-with-animals.tex}
\input{subsections/spells/speak-with-dead.tex}
\input{subsections/spells/speak-with-plants.tex}
\input{subsections/spells/spectral-hand.tex}
\Spell{Spell Immunity, Greater}{greater spell immunity}
{Abjuration}
{
	\textbf{Level:}
	Clr 8\\
}
{
	This spell functions like \spell{spell immunity}, except the immunity applies to spells of 8th level or lower.

	A creature can have only one \spell{spell immunity} or \emph{greater spell immunity} spell in effect on it at a time.

}

\input{subsections/spells/spell-immunity.tex}
\input{subsections/spells/spell-resistance.tex}
\input{subsections/spells/spellstaff.tex}
% \input{subsections/spells/spell-turning.tex}
\Spell{Spider Climb}{spider climb}
{Transmutation}
{
	\textbf{Level:}
	Drd 2, Wiz 2\\
	\textbf{Components:}
	V, S, M\\
	\textbf{Casting Time:}
	1 standard action\\
	\textbf{Range:}
	Touch\\
	\textbf{Target:}
	Creature touched\\
	\textbf{Duration:}
	10 min./level\\
	\textbf{Saving Throw:}
	Will negates (harmless)\\
	\textbf{Spell Resistance:}
	Yes (harmless)\\
}
{
	The subject can climb and travel on vertical surfaces or even traverse ceilings as well as a spider does. The affected creature must have its hands free to climb in this manner. The subject gains a climb speed of 6 meters; furthermore, it need not make \skill{Climb} checks to traverse a vertical or horizontal surface (even upside down). A spider climbing creature retains its Dexterity bonus to Armor Class (if any) while climbing, and opponents get no special bonus to their attacks against it. It cannot, however, use the run action while climbing.

	\textit{Material Component}:
	A drop of bitumen and a live spider, both of which must be eaten by the subject.

}

\Spell{Spike Growth}{spike growth}
{Transmutation}
{
	\textbf{Level:}
	Drd 3, Rgr 2\\
	\textbf{Components:}
	V, S, DF\\
	\textbf{Casting Time:}
	1 standard action\\
	\textbf{Range:}
	Medium (30 m + 3 m/level)\\
	\textbf{Area:}
	One 6-m square/level\\
	\textbf{Duration:}
	1 hour/level (D)\\
	\textbf{Saving Throw:}
	Reflex partial\\
	\textbf{Spell Resistance:}
	Yes\\
}
{
	Any ground-covering vegetation in the spell's area becomes very hard and sharply pointed without changing its appearance.

	In areas of bare earth, roots and rootlets act in the same way. Typically, \emph{spike growth} can be cast in any outdoor setting except open water, ice, heavy snow, sandy desert, or bare stone. Any creature moving on foot into or through the spell's area takes 1d4 points of piercing damage for each 1.5 meter of movement through the spiked area.

	Any creature that takes damage from this spell must also succeed on a Reflex save or suffer injuries to its feet and legs that slow its land speed by one-half. This speed penalty lasts for 24 hours or until the injured creature receives a cure spell (which also restores lost hit points). Another character can remove the penalty by taking 10 minutes to dress the injuries and succeeding on a \skill{Heal} check against the spell's save DC.

	\emph{Spike growth} can't be disabled with the \skill{Disable Device} skill.

	\textit{Note:} Magic traps such as \emph{spike growth} are hard to detect. A rogue (only) can use the \skill{Search} skill to find a \emph{spike growth}. The DC is 25 + spell level, or DC 28 for \emph{spike growth} (or DC 27 for \emph{spike growth} cast by a ranger).

}

\Spell{Spike Stones}{spike stones}
{Transmutation [Earth]}
{
	\textbf{Level:}
	Drd 4, Earth 4, Rng 3\\
	\textbf{Components:}
	V, S, DF\\
	\textbf{Casting Time:}
	1 standard action\\
	\textbf{Range:}
	Medium (30 m + 3 m/level)\\
	\textbf{Area:}
	One 6-m square/level\\
	\textbf{Duration:}
	1 hour/level (D)\\
	\textbf{Saving Throw:}
	Reflex partial\\
	\textbf{Spell Resistance:}
	Yes\\
}
{
	Rocky ground, stone floors, and similar surfaces shape themselves into long, sharp points that blend into the background.

	\emph{Spike stones} impede progress through an area and deal damage. Any creature moving on foot into or through the spell's area moves at half speed.

	In addition, each creature moving through the area takes 1d8 points of piercing damage for each 1.5 meter of movement through the spiked area.

	Any creature that takes damage from this spell must also succeed on a Reflex save to avoid injuries to its feet and legs. A failed save causes the creature's speed to be reduced to half normal for 24 hours or until the injured creature receives a cure spell (which also restores lost hit points). Another character can remove the penalty by taking 10 minutes to dress the injuries and succeeding on a \skill{Heal} check against the spell's save DC.

	\emph{Spike stones} is a magic trap that can't be disabled with the \skill{Disable Device} skill.

	\textbf{Note:} Magic traps such as \emph{spike stones} are hard to detect. A rogue (only) can use the \skill{Search} skill to find \emph{spike stones}. The DC is 25 + spell level, or DC 29 for \emph{spike stones} (or DC 28 for a \emph{spike stones} cast by a ranger).

}

\input{subsections/spells/spiritual-weapon.tex}
\input{subsections/spells/statue.tex}
\input{subsections/spells/status.tex}
\Spell{Stinking Cloud}{stinking cloud}
{Conjuration (Creation)}
{
	\textbf{Level:}
	Wiz 3\\
	\textbf{Components:}
	V, S, M\\
	\textbf{Casting Time:}
	1 standard action\\
	\textbf{Range:}
	Medium (30 m + 3 m/level)\\
	\textbf{Effect:}
	Cloud spreads in 6-m radius, 6 m high\\
	\textbf{Duration:}
	1 round/level\\
	\textbf{Saving Throw:}
	Fortitude negates; see text\\
	\textbf{Spell Resistance:}
	No\\
}
{
	\emph{Stinking cloud} creates a bank of fog like that created by \spell{fog cloud}, except that the vapors are nauseating. Living creatures in the cloud become nauseated. This condition lasts as long as the creature is in the cloud and for 1d4+1 rounds after it leaves. (Roll separately for each nauseated character.) Any creature that succeeds on its save but remains in the cloud must continue to save each round on your turn.

	\emph{Stinking cloud} can be made permanent with a \spell{permanency} spell. A permanent \emph{stinking cloud} dispersed by wind reforms in 10 minutes.

	\textit{Material Component}:
	A rotten egg or several skunk cabbage leaves.

}

\input{subsections/spells/stone-shape.tex}
\input{subsections/spells/stoneskin.tex}
\input{subsections/spells/stone-tell.tex}
\Spell{Stone to Flesh}{stone to flesh}
{Transmutation}
{
	\textbf{Level:}
	Wiz 6\\
	\textbf{Components:}
	V, S, M\\
	\textbf{Casting Time:}
	1 standard action\\
	\textbf{Range:}
	Medium (30 m + 3 m/level)\\
	\textbf{Target:}
	One petrified creature or a cylinder of stone from 0.3 m to 1 m in diameter and up to 3 m long\\
	\textbf{Duration:}
	Instantaneous\\
	\textbf{Saving Throw:}
	Fortitude negates (object); see text\\
	\textbf{Spell Resistance:}
	Yes\\
}
{
	This spell restores a petrified creature to its normal state, restoring life and goods. The creature must make a DC 15 Fortitude save to survive the process. Any petrified creature, regardless of size, can be restored.

	The spell also can convert a mass of stone into a fleshy substance. Such flesh is inert and lacking a vital life force unless a life force or magical energy is available. (For example, this spell would turn a stone golem into a flesh golem, but an ordinary statue would become a corpse.) You can affect an object that fits within a cylinder from 30 centimeters to 1 meter in diameter and up to 3 meters long or a cylinder of up to those dimensions in a larger mass of stone.

	\textit{Material Component:}
	A pinch of earth and a drop of blood.

}

\input{subsections/spells/storm-of-vengeance.tex}
\Spell{Suggestion, Mass}{suggestion, mass}
{Enchantment (Compulsion) [Language-Dependent, Mind-Affecting]}
{
	\textbf{Level:}
	Wiz 6\\
	\textbf{Range:}
	Medium (30 m + 3 m/level)\\
	\textbf{Targets:}
	One creature/level, no two of which can be more than 9 m apart\\
}
{
	This spell functions like suggestion, except that it can affect more creatures. The same suggestion applies to all these creatures.

}

\input{subsections/spells/suggestion.tex}
\input{subsections/spells/summon-instrument.tex}
% \Spell{Summon Monster III}{summon monster iii}
{



May be summoned only into an aquatic or watery environment.

	\\


Monster
Alignment
	Celestial black bear\\
	LG\\
	Celestial bison\\
	NG\\
	Celestial dire badger\\
	CG\\
	Celestial hippogriff\\
	CG\\
	Elemental, Small (any)\\
	N\\
	Fiendish ape\\
	LE\\
	Fiendish dire weasel\\
	LE\\
	Hell hound\\
	LE\\
	Fiendish snake, constrictor\\
	LE\\
	Fiendish boar\\
	NE\\
	Fiendish dire bat\\
	NE\\
	Fiendish monstrous centipede, Huge\\
	NE\\
	Fiendish crocodile\\
	CE\\
	Dretch (demon)\\
	CE\\
	Fiendish snake, Large viper\\
	CE\\
	Fiendish wolverine\\
	CE\\

}
{
{Conjuration (Summoning) [see text for summon monster I]}
{
	\textbf{Level:}
	Clr 3, Wiz 3\\
	\textbf{Effect:}
	One or more summoned creatures, no two of which can be more than 9 m apart\\
}
{
	This spell functions like \spell{summon monster I}, except that you can summon one creature from the 3rd-level list, 1d3 creatures of the same kind from the 2nd-level list, or 1d4+1 creatures of the same kind from the 1st-level list.

}

% \Spell{Summon Monster II}{summon monster ii}
{



May be summoned only into an aquatic or watery environment.

	\\


Monster
Alignment
	Celestial giant bee\\
	LG\\
	Celestial giant bombardier beetle\\
	NG\\
	Celestial riding dog\\
	NG\\
	Celestial eagle\\
	CG\\
	Lemure (devil)\\
	LE\\
	Fiendish squid1\\
	LE\\
	Fiendish wolf\\
	LE\\
	Fiendish monstrous centipede, Large\\
	NE\\
	Fiendish monstrous scorpion, Medium\\
	NE\\
	Fiendish shark, Medium1\\
	NE\\
	Fiendish monstrous spider, Medium\\
	CE\\
	Fiendish snake, Medium viper\\
	CE\\

}
{
{Conjuration (Summoning) [see text for summon monster I]}
{
	\textbf{Level:}
	Clr 2, Wiz 2\\
	\textbf{Effect:}
	One or more summoned creatures, no two of which can be more than 9 m apart\\
}
{
	This spell functions like \spell{summon monster I}, except that you can summon one creature from the 2nd-level list or 1d3 creatures of the same kind from the 1st-level list.

}

% \Spell{Summon Monster I}{summon monster i}
{



May be summoned only into an aquatic or watery environment.

	\\


Monster
Alignment
	Celestial dog\\
	LG\\
	Celestial owl\\
	LG\\
	Celestial giant fire beetle\\
	NG\\
	Celestial porpoise1\\
	NG\\
	Celestial badger\\
	CG\\
	Celestial monkey\\
	CG\\
	Fiendish dire rat\\
	LE\\
	Fiendish raven\\
	LE\\
	Fiendish monstrous centipede, Medium\\
	NE\\
	Fiendish monstrous scorpion, Small\\
	NE\\
	Fiendish hawk\\
	CE\\
	Fiendish monstrous spider, Small\\
	CE\\
	Fiendish octopus1\\
	CE\\
	Fiendish snake, Small viper\\
	CE\\

}
{
{Conjuration (Summoning) [see text]}
{
	\textbf{Level:}
	Clr 1, Wiz 1\\
	\textbf{Components:}
	V, S, F/DF\\
	\textbf{Casting Time:}
	1 round\\
	\textbf{Range:}
	Close (7.5 m + 1.5 m/2 levels)\\
	\textbf{Effect:}
	One summoned creature\\
	\textbf{Duration:}
	1 round/level (D)\\
	\textbf{Saving Throw:}
	None\\
	\textbf{Spell Resistance:}
	No\\
}
{
	This spell summons an extraplanar creature (typically an outsider, elemental, or magical beast native to another plane). It appears where you designate and acts immediately, on your turn. It attacks your opponents to the best of its ability. If you can communicate with the creature, you can direct it not to attack, to attack particular enemies, or to perform other actions.

	The spell conjures one of the creatures from the 1st-level list on the accompanying Summon Monster table. You choose which kind of creature to summon, and you can change that choice each time you cast the spell.

	A summoned monster cannot summon or otherwise conjure another creature, nor can it use any teleportation or planar travel abilities. Creatures cannot be summoned into an environment that cannot support them.

	When you use a summoning spell to summon an air, chaotic, earth, evil, fire, good, lawful, or water creature, it is a spell of that type.

	\textit{Arcane Focus:}
	A tiny bag and a small (not necessarily lit) candle.

}

% \input{subsections/spells/summon-monster-iV.tex}
% \Spell{Summon Monster IX}{summon monster ix}
{



May be summoned only into an aquatic or watery environment.

	\\


Monster
Alignment
	Couatl\\
	LG\\
	Leonal (guardinal)\\
	NG\\
	Celestial roc\\
	CG\\
	Elemental, elder (any)\\
	N\\
	Devil, barbed\\
	LE\\
	Fiendish dire shark1\\
	NE\\
	Fiendish monstrous scorpion, Gargantuan\\
	NE\\
	Night hag\\
	NE\\
	Bebilith (demon)\\
	CE\\
	Fiendish monstrous spider, Colossal\\
	CE\\
	Hezrou (demon)\\
	CE\\

}
{
{Conjuration (Summoning) [see text for summon monster I]}
{
	\textbf{Level:}
	Chaos 9, Clr 9, Evil 9, Good 9, Law 9, Wiz 9\\
}
{
	This spell functions like \spell{summon monster I}, except that you can summon one creature from the 9th-level list, 1d3 creatures of the same kind from the 8th-level list, or 1d4+1 creatures of the same kind from a lower-level list.

}

% \Spell{Summon Monster VIII}{summon monster viii}
{



May be summoned only into an aquatic or watery environment.

	\\


Monster
Alignment
	Celestial dire bear\\
	LG\\
	Celestial cachalot whale1\\
	NG\\
	Celestial triceratops\\
	NG\\
	Lillend\\
	CG\\
	Elemental, greater (any)\\
	N\\
	Fiendish giant squid1\\
	LE\\
	Hellcat\\
	LE\\
	Fiendish monstrous centipede, Colossal\\
	NE\\
	Fiendish dire tiger\\
	CE\\
	Fiendish monstrous spider, Gargantuan\\
	CE\\
	Fiendish tyrannosaurus\\
	CE\\
	Vrock (demon)\\
	CE\\

}
{
{Conjuration (Summoning) [see text for summon monster I]}
{
	\textbf{Level:}
	Clr 8, Wiz 8\\
}
{
	This spell functions like summon monster I, except that you can summon one creature from the 8th-level list, 1d3 creatures of the same kind from the 7th-level list, or 1d4+1 creatures of the same kind from a lower-level list.

}

% \Spell{Summon Monster VII}{summon monster vii}
{



May be summoned only into an aquatic or watery environment.

	\\


Monster
Alignment
	Celestial elephant\\
	LG\\
	Avoral (guardinal)\\
	NG\\
	Celestial baleen whale1\\
	NG\\
	Djinni (genie)\\
	CG\\
	Elemental, Huge (any)\\
	N\\
	Invisible stalker\\
	N\\
	Devil, bone\\
	LE\\
	Fiendish megaraptor\\
	LE\\
	Fiendish monstrous scorpion, Huge\\
	NE\\
	Babau (demon)\\
	CE\\
	Fiendish giant octopus1\\
	CE\\
	Fiendish girallon\\
	CE\\

}
{
{Conjuration (Summoning) [see text for summon monster I]}
{
	\textbf{Level:}
	Clr 7, Wiz 7\\
}
{
	This spell functions like \spell{summon monster I}, except that you can summon one creature from the 7th-level list, 1d3 creatures of the same kind from the 6th-level list, or 1d4+1 creatures of the same kind from a lower-level list.

}

% \Spell{Summon Monster VI}{summon monster vi}
{



May be summoned only into an aquatic or watery environment.

	\\


Monster
Alignment
	Celestial polar bear\\
	LG\\
	Celestial orca whale1\\
	NG\\
	Bralani (eladrin)\\
	CG\\
	Celestial dire lion\\
	CG\\
	Elemental, Large (any)\\
	N\\
	Janni (genie)\\
	N\\
	Chaos beast\\
	CN\\
	Devil, chain\\
	LE\\
	Xill\\
	LE\\
	Fiendish monstrous centipede, Gargantuan\\
	NE\\
	Fiendish rhinoceros\\
	NE\\
	Fiendish elasmosaurus1\\
	CE\\
	Fiendish monstrous spider, Huge\\
	CE\\
	Fiendish snake, giant constrictor\\
	CE\\

}
{
{Conjuration (Summoning) [see text for summon monster I]}
{
	\textbf{Level:}
	Clr 6, Wiz 6\\
	\textbf{Effect:}
	One or more summoned creatures, no two of which can be more than 9 m apart\\
}
{
	This spell functions like \spell{summon monster I}, except you can summon one creature from the 6th-level list, 1d3 creatures of the same kind from the 5th-level list, or 1d4+1 creatures of the same kind from a lower-level list.

}

% \Spell{Summon Monster V}{summon monster v}
{



May be summoned only into an aquatic or watery environment.

	\\


Monster
Alignment
	Archon, hound\\
	LG\\
	Celestial brown bear\\
	LG\\
	Celestial giant stag beetle\\
	NG\\
	Celestial sea cat1\\
	NG\\
	Celestial griffon\\
	CG\\
	Elemental, Medium (any)\\
	N\\
	Achaierai\\
	LE\\
	Devil, bearded\\
	LE\\
	Fiendish deinonychus\\
	LE\\
	Fiendish dire ape\\
	LE\\
	Fiendish dire boar\\
	NE\\
	Fiendish shark, Huge\\
	NE\\
	Fiendish monstrous scorpion, Large\\
	NE\\
	Shadow mastiff\\
	NE\\
	Fiendish dire wolverine\\
	CE\\
	Fiendish giant crocodile\\
	CE\\
	Fiendish tiger\\
	CE\\

}
{
{Conjuration (Summoning) [see text for summon monster I]}
{
	\textbf{Level:}
	Clr 5, Wiz 5\\
	\textbf{Effect:}
	One or more summoned creatures, no two of which can be more than 9 m apart\\
}
{
	This spell functions like \spell{summon monster I}, except that you can summon one creature from the 5th-level list, 1d3 creatures of the same kind from the 4th-level list, or 1d4+1 creatures of the same kind from a lower-level list.

}

% \Spell{Summon Nature's Ally III		}{summon nature's ally iii		}
{



May be summoned only into an aquatic or watery environment.

	\\


Summoned Creature
	Ape (animal)\\
	Dire weasel\\
	Dire wolf\\
	Eagle, giant [NG]\\
	Lion\\
	Owl, giant [NG]\\
	Satyr [CN; without pipes]\\
	Shark, Large1 (animal)\\
	Snake, constrictor (animal)\\
	Snake, Large viper (animal)\\
	Thoqqua\\

}
{
{Conjuration (Summoning) [see text]}
{
	\textbf{Level:}
	Drd 3, Rgr 3\\
	\textbf{Effect:}
	One or more creatures, no two of which can be more than 9 m apart\\
}
{
	This spell functions like summon nature's ally I, except that you can summon one 3rd-level creature, 1d3 2nd-level creatures of the same kind, or 1d4+1 1st-level creatures of the same kind.

	When you use a summoning spell to summon an air, chaotic, earth, evil, fire, good, lawful, or water creature, it is a spell of that type.

}

% \Spell{Summon Nature's Ally II		}{summon nature's ally ii		}
{



May be summoned only into an aquatic or watery environment.

	\\


Summoned Creature
	Bear, black (animal)\\
	Crocodile (animal)\\
	Dire badger\\
	Dire bat\\
	Elemental, Small (any)\\
	Hippogriff\\
	Shark, Medium1 (animal)\\
	Snake, Medium viper (animal)\\
	Squid1 (animal)\\
	Wolverine (animal)\\

}
{
{Conjuration (Summoning)}
{
	\textbf{Level:}
	Drd 2, Rgr 2\\
	\textbf{Effect:}
	One or more creatures, no two of which can be more than 9 m apart\\
}
{
	This spell functions like summon nature's ally I, except that you can summon one 2nd-level creature or 1d3 1st-level creatures of the same kind.

}

% \Spell{Summon Nature's Ally I}{summon nature's ally i}
{



May be summoned only into an aquatic or watery environment.

	\\


Summoned Creature
	Dire rat\\
	Eagle (animal)\\
	Monkey (animal)\\
	Octopus1 (animal)\\
	Owl (animal)\\
	Porpoise1 (animal)\\
	Snake, Small viper (animal)\\
	Wolf (animal)\\

}
{
{Conjuration (Summoning)}
{
	\textbf{Level:}
	Drd 1, Rgr 1\\
	\textbf{Components:}
	V, S, DF\\
	\textbf{Casting Time:}
	1 round\\
	\textbf{Range:}
	Close (7.5 m + 1.5 m/2 levels)\\
	\textbf{Effect:}
	One summoned creature\\
	\textbf{Duration:}
	1 round/level (D)\\
	\textbf{Saving Throw:}
	None\\
	\textbf{Spell Resistance:}
	No\\
}
{
	This spell summons a natural creature. It appears where you designate and acts immediately, on your turn. It attacks your opponents to the best of its ability. If you can communicate with the creature, you can direct it not to attack, to attack particular enemies, or to perform other actions.

	A summoned monster cannot summon or otherwise conjure another creature, nor can it use any teleportation or planar travel abilities. Creatures cannot be summoned into an environment that cannot support them.

	The spell conjures one of the creatures from the 1st-level list on the accompanying Summon Nature's Ally table. You choose which kind of creature to summon, and you can change that choice each time you cast the spell. All the creatures on the table are neutral unless otherwise noted.

}

% \Spell{Summon Nature's Ally IV		}{summon nature's ally iv		}
{



May be summoned only into an aquatic or watery environment.

	\\


Summoned Creature
	Arrowhawk, juvenile\\
	Bear, brown (animal)\\
	Crocodile, giant (animal)\\
	Deinonychus (dinosaur)\\
	Dire ape\\
	Dire boar\\
	Dire wolverine\\
	Elemental, Medium (any)\\
	Salamander, flamebrother [NE]\\
	Sea cat1\\
	Shark, Huge1 (animal)\\
	Snake, Huge viper (animal)\\
	Tiger (animal)\\
	Tojanida, juvenile1\\
	Unicorn [CG]\\
	Xorn, minor\\

}
{
{Conjuration (Summoning) [see text]}
{
	\textbf{Level:}
	Animal 4, Drd 4, Rgr 4\\
	\textbf{Effect:}
	One or more creatures, no two of which can be more than 9 m apart\\
}
{
	This spell functions like \spell{summon nature's ally I}, except that you can summon one 4th-level creature, 1d3 3rd-level creatures of the same kind, or 1d4+1 lower-level creatures of the same kind.

	When you use a summoning spell to summon an air, chaotic, earth, evil, fire, good, lawful, or water creature, it is a spell of that type.

}

% \Spell{Summon Nature's Ally IX		}{summon nature's ally ix		}
{



May be summoned only into an aquatic or watery environment.
Can cast irresistible dance

	\\


Summoned Creature
	Elemental, elder\\
	Grig [NG; with fiddle] (sprite)\\
	Pixie2 (sprite) [NG; with sleep and memory loss arrows]\\
	Unicorn, celestial charger [CG]\\

}
{
{Conjuration (Summoning) [see text]}
{
	\textbf{Level:}
	Drd 9\\
	\textbf{Effect:}
	One or more creatures, no two of which can be more than 9 m apart\\
}
{
	This spell functions like \spell{summon nature's ally I}, except that you can summon one 9th-level creature, 1d3 8th-level creatures of the same kind, or 1d4+1 lower-level creatures of the same kind.

	When you use a summoning spell to summon an air, chaotic, earth, evil, fire, good, lawful, or water creature, it is a spell of that type.

}

% \Spell{Summon Nature's Ally VIII		}{summon nature's ally viii		}
{



May be summoned only into an aquatic or watery environment.

	\\


Summoned Creature
	Dire shark1\\
	Roc\\
	Salamander, noble [NE]\\
	Tojanida, elder\\

}
{
{Conjuration (Summoning) [see text]}
{
	\textbf{Level:}
	Animal 8, Drd 8\\
	\textbf{Effect:}
	One or more creatures, no two of which can be more than 9 m apart\\
}
{
	This spell functions like summon nature's ally I, except that you can summon one 8th-level creature, 1d3 7th-level creatures of the same kind, or 1d4+1 lower-level creatures of the same kind.

	When you use a summoning spell to summon an air, chaotic, earth, evil, fire, good, lawful, or water creature, it is a spell of that type.

}

% \Spell{Summon Nature's Ally VII		}{summon nature's ally vii		}
{



May be summoned only into an aquatic or watery environment.
Can't cast irresistible dance

	\\


Summoned Creature
	Arrowhawk, elder\\
	Dire tiger\\
	Elemental, greater (any)\\
	Djinni (genie) [NG]\\
	Invisible stalker\\
	Pixie2 (sprite) [NG; with sleep arrows]\\
	Squid, giant1 (animal)\\
	Triceratops (dinosaur)\\
	Tyrannosaurus (dinosaur)\\
	Whale, cachalot1 (animal)\\
	Xorn, elder\\

}
{
{Conjuration (Summoning) [see text]}
{
	\textbf{Level:}
	Drd 7\\
	\textbf{Effect:}
	One or more creatures, no two of which can be more than 9 m apart\\
}
{
	This spell functions like summon nature's ally I, except that you can summon one 7th-level creature, 1d3 6th-level creatures of the same kind, or 1d4+1 lower-level creatures of the same kind.

	When you use a summoning spell to summon an air, chaotic, earth, evil, fire, good, lawful, or water creature, it is a spell of that type.

}

% \Spell{Summon Nature's Ally VI		}{summon nature's ally vi		}
{



May be summoned only into an aquatic or watery environment.

	\\


Summoned Creature
	Dire bear\\
	Elemental, Huge (any)\\
	Elephant (animal)\\
	Girallon\\
	Megaraptor (dinosaur)\\
	Octopus, giant1 (animal)\\
	Pixie2 (sprite) [NG; no special arrows]\\
	Salamander, average [NE]\\
	Whale, baleen1\\
	Xorn, average\\

}
{
{Conjuration (Summoning) [see text]}
{
	\textbf{Level:}
	Drd 6\\
	\textbf{Effect:}
	One or more creatures, no two of which can be more than 9 m apart\\
}
{
	This spell functions like summon nature's ally I, except that you can summon one 6th-level creature, 1d3 5th-level creatures of the same kind, or 1d4+1 lower-level creatures of the same kind.

	When you use a summoning spell to summon an air, chaotic, earth, evil, fire, good, lawful, or water creature, it is a spell of that type.

}

% \Spell{Summon Nature's Ally V		}{summon nature's ally v		}
{



May be summoned only into an aquatic or watery environment.

	\\


Summoned Creature
	Arrowhawk, adult\\
	Bear, polar (animal)\\
	Dire lion\\
	Elasmosaurus1 (dinosaur)\\
	Elemental, Large (any)\\
	Griffon\\
	Janni (genie)\\
	Rhinoceros (animal)\\
	Satyr [CN; with pipes]\\
	Snake, giant constrictor (animal)\\
	Nixie (sprite)\\
	Tojanida, adult1\\
	Whale, orca1 (animal)\\

}
{
{Conjuration (Summoning) [see text]}
{
	\textbf{Level:}
	Drd 5\\
	\textbf{Effect:}
	One or more creatures, no two of which can be more than 9 m apart\\
}
{
	This spell functions like \spell{summon nature's ally I}, except that you can summon one 5th-level creature, 1d3 4th-level creatures of the same kind, or 1d4+1 lower-level creatures of the same kind.

	When you use a summoning spell to summon an air, chaotic, earth, evil, fire, good, lawful, or water creature, it is a spell of that type.

}

% \Spell{Summon Swarm}{summon swarm}
{Conjuration (Summoning)}
{
	\textbf{Level:}
	Drd 2, Wiz 2\\
	\textbf{Components:}
	V, S, M/DF\\
	\textbf{Casting Time:}
	1 round\\
	\textbf{Range:}
	Close (7.5 m + 1.5 m/2 levels)\\
	\textbf{Effect:}
	One swarm of bats, rats, or spiders\\
	\textbf{Duration:}
	Concentration + 2 rounds\\
	\textbf{Saving Throw:}
	None\\
	\textbf{Spell Resistance:}
	No\\
}
{
	You summon a swarm of bats, rats, or spiders (your choice), which attacks all other creatures within its area. (You may summon the swarm so that it shares the area of other creatures.) If no living creatures are within its area, the swarm attacks or pursues the nearest creature as best it can. The caster has no control over its target or direction of travel.

	\textit{Arcane Material Component:}
	A square of red cloth.

}

\input{subsections/spells/sunbeam.tex}
\input{subsections/spells/sunburst.tex}
\Spell{Symbol of Death}{symbol of death}
{Necromancy [Death]}
{
	\textbf{Level:}
	Clr 8, Wiz 8\\
	\textbf{Components:}
	V, S, M\\
	\textbf{Casting Time:}
	10 minutes\\
	\textbf{Range:}
	0 m; see text\\
	\textbf{Effect:}
	One symbol\\
	\textbf{Duration:}
	See text\\
	\textbf{Saving Throw:}
	Fortitude negates\\
	\textbf{Spell Resistance:}
	Yes\\
}
{
	This spell allows you to scribe a potent rune of power upon a surface. When triggered, a \emph{symbol of death} slays one or more creatures within 18 meters of the symbol (treat as a burst) whose combined total current hit points do not exceed 150. The \emph{symbol of death} affects the closest creatures first, skipping creatures with too many hit points to affect. Once triggered, the symbol becomes active and glows, lasting for 10 minutes per caster level or until it has affected 150 hit points' worth of creatures, whichever comes first. Any creature that enters the area while the \emph{symbol of death} is active is subject to its effect, whether or not that creature was in the area when it was triggered. A creature need save against the symbol only once as long as it remains within the area, though if it leaves the area and returns while the symbol is still active, it must save again.

	Until it is triggered, the \emph{symbol of death} is inactive (though visible and legible at a distance of 18 meters). To be effective, a \emph{symbol of death} must always be placed in plain sight and in a prominent location. Covering or hiding the rune renders the \emph{symbol of death} ineffective, unless a creature removes the covering, in which case the \emph{symbol of death} works normally.

	As a default, a \emph{symbol of death} is triggered whenever a creature does one or more of the following, as you select: looks at the rune; reads the rune; touches the rune; passes over the rune; or passes through a portal bearing the rune. Regardless of the trigger method or methods chosen, a creature more than 18 meters from a \emph{symbol of death} can't trigger it (even if it meets one or more of the triggering conditions, such as reading the rune). Once the spell is cast, a \emph{symbol of death}'s triggering conditions cannot be changed.

	In this case, ``reading'' the rune means any attempt to study it, identify it, or fathom its meaning. Throwing a cover over a \emph{symbol of death} to render it inoperative triggers it if the symbol reacts to touch. You can't use a \emph{symbol of death} offensively; for instance, a touch-triggered \emph{symbol of death} remains untriggered if an item bearing the \emph{symbol of death} is used to touch a creature. Likewise, a \emph{symbol of death} cannot be placed on a weapon and set to activate when the weapon strikes a foe.

	You can also set special triggering limitations of your own. These can be as simple or elaborate as you desire. Special conditions for triggering a \emph{symbol of death} can be based on a creature's name, identity, or alignment, but otherwise must be based on observable actions or qualities. Intangibles such as level, class, Hit Dice, and hit points don't qualify.

	When scribing a \emph{symbol of death}, you can specify a password or phrase that prevents a creature using it from triggering the effect. Anyone using the password remains immune to that particular rune's effects so long as the creature remains within 18 meters of the rune. If the creature leaves the radius and returns later, it must use the password again.

	You also can attune any number of creatures to the \emph{symbol of death}, but doing this can extend the casting time. Attuning one or two creatures takes negligible time, and attuning a small group (as many as ten creatures) extends the casting time to 1 hour. Attuning a large group (as many as twenty-five creatures) takes 24 hours. Attuning larger groups takes proportionately longer. Any creature attuned to a \emph{symbol of death} cannot trigger it and is immune to its effects, even if within its radius when triggered. You are automatically considered attuned to your own symbols of death, and thus always ignore the effects and cannot inadvertently trigger them.

	Read magic allows you to identify a \emph{symbol of death} with a DC 19 \skill{Spellcraft} check. Of course, if the \emph{symbol of death} is set to be triggered by reading it, this will trigger the symbol.

	A \emph{symbol of death} can be removed by a successful dispel magic targeted solely on the rune. An erase spell has no effect on a \emph{symbol of death}. Destruction of the surface where a \emph{symbol of death} is inscribed destroys the symbol but also triggers it.

	\emph{Symbol of death} can be made permanent with a \spell{permanency} spell. A permanent \emph{symbol of death} that is disabled or that has affected its maximum number of hit points becomes inactive for 10 minutes, then can be triggered again as normal.

	\textit{Note:}
	Magic traps such as \emph{symbol of death} are hard to detect and disable. A rogue (only) can use the \skill{Search} skill to find a \emph{symbol of death} and \skill{Disable Device} to thwart it. The DC in each case is 25 + spell level, or 33 for \emph{symbol of death}.

	\textit{Material Component}:
	Mercury and phosphorus, plus powdered diamond and opal with a total value of at least 5,000 cp each.

}

\Spell{Symbol of Fear}{symbol of fear}
{Necromancy [Fear, Mind-Affecting]}
{
	\textbf{Level:}
	Clr 6, Wiz 6\\
	\textbf{Saving Throw:}
	Will negates\\
}
{
	This spell functions like \spell{symbol of death}, except that all creatures within 18 meters of the symbol of fear instead become panicked for 1 round per caster level.

	\textit{Note:} Magic traps such as symbol of fear are hard to detect and disable. A rogue (only) can use the Search skill to find a symbol of fear and Disable Device to thwart it. The DC in each case is 25 + spell level, or 31 for symbol of fear.

	\textit{Material Component:}
	Mercury and phosphorus, plus powdered diamond and opal with a total value of at least 1,000 cp.

}

\Spell{Symbol of Insanity}{symbol of insanity}
{Enchantment (Compulsion) [Mind-Affecting]}
{
	\textbf{Level:}
	Clr 8, Wiz 8\\
	\textbf{Saving Throw:}
	Will negates\\
}
{
	This spell functions like \spell{symbol of death}, except that all creatures within the radius of the symbol of insanity instead become permanently insane (as the insanity spell).

	Unlike symbol of death, symbol of insanity has no hit point limit; once triggered, a symbol of insanity simply remains active for 10 minutes per caster level.

	\textit{Note:} Magic traps such as symbol of insanity are hard to detect and disable. A rogue (only) can use the Search skill to find a symbol of insanity and Disable Device to thwart it. The DC in each case is 25 + spell level, or 33 for symbol of insanity.

	\textit{Material Component:}
	Mercury and phosphorus, plus powdered diamond and opal with a total value of at least 5,000 cp.

}

\Spell{Symbol of Pain}{symbol of pain}
{Necromancy [Evil]}
{
	\textbf{Level:}
	Clr 5, Wiz 5\\
}
{
	This spell functions like \spell{symbol of death}, except that each creature within the radius of a symbol of pain instead suffers wracking pains that impose a $-4$ penalty on attack rolls, skill checks, and ability checks. These effects last for 1 hour after the creature moves farther than 18 meters from the symbol.

	Unlike symbol of death, symbol of pain has no hit point limit; once triggered, a symbol of pain simply remains active for 10 minutes per caster level.

	\textit{Note:} Magic traps such as symbol of pain are hard to detect and disable. A rogue (only) can use the Search skill to find a symbol of pain and Disable Device to thwart it. The DC in each case is 25 + spell level, or 30 for symbol of pain.

	\textit{Material Component:}
	Mercury and phosphorus, plus powdered diamond and opal with a total value of at least 1,000 cp.

}

\Spell{Symbol of Persuasion}{symbol of persuasion}
{Enchantment (Charm) [Mind-Affecting]}
{
	\textbf{Level:}
	Clr 6, Wiz 6\\
	\textbf{Saving Throw:}
	Will negates\\
}
{
	This spell functions like \spell{symbol of death}, except that all creatures within the radius of a symbol of persuasion instead become charmed by the caster (as the charm monster spell) for 1 hour per caster level.

	Unlike symbol of death, symbol of persuasion has no hit point limit; once triggered, a symbol of persuasion simply remains active for 10 minutes per caster level.

	\textit{Note:} Magic traps such as symbol of persuasion are hard to detect and disable. A rogue (only) can use the Search skill to find a symbol of persuasion and Disable Device to thwart it. The DC in each case is 25 + spell level, or 31 for symbol of persuasion.

	\textit{Material Component}:
	Mercury and phosphorus, plus powdered diamond and opal with a total value of at least 5,000 cp.

}

\Spell{Symbol of Sleep}{symbol of sleep}
{Enchantment (Compulsion) [Mind-Affecting]}
{
	\textbf{Level:}
	Clr 5, Wiz 5\\
	\textbf{Saving Throw:}
	Will negates\\
}
{
	This spell functions like symbol of death, except that all creatures of 10 HD or less within 18 meters of the symbol of sleep instead fall into a catatonic slumber for 3d6 $\times$ 10 minutes. Unlike with the sleep spell, sleeping creatures cannot be awakened by nonmagical means before this time expires.

	Unlike symbol of death, symbol of sleep has no hit point limit; once triggered, a symbol of sleep simply remains active for 10 minutes per caster level.

	Note: Magic traps such as symbol of sleep are hard to detect and disable. A rogue (only) can use the Search skill to find a symbol of sleep and Disable Device to thwart it. The DC in each case is 25 + spell level, or 30 for symbol of sleep.

	\textit{Material Component}:
	Mercury and phosphorus, plus powdered diamond and opal with a total value of at least 1,000 gp.

}

\Spell{Symbol of Stunning}{symbol of stunning}
{Enchantment (Compulsion) [Mind-Affecting]}
{
	\textbf{Level:}
	Clr 7, Wiz 7\\
	\textbf{Saving Throw:}
	Will negates\\
}
{
	This spell functions like \spell{symbol of death}, except that all creatures within 18 meters of a symbol of stunning instead become stunned for 1d6 rounds.

	\textit{Note:} Magic traps such as symbol of stunning are hard to detect and disable. A rogue (only) can use the Search skill to find a symbol of stunning and Disable Device to thwart it. The DC in each case is 25 + spell level, or 32 for symbol of stunning.

	\textit{Material Component:}
	Mercury and phosphorus, plus powdered diamond and opal with a total value of at least 5,000 cp.

}

\input{subsections/spells/symbol-of-weakness.tex}
\input{subsections/spells/sympathetic-vibration.tex}
\Spell{Sympathy}{sympathy}
{Enchantment (Compulsion) [Mind-Affecting]}
{
	\textbf{Level:}
	Drd 9, Wiz 8\\
	\textbf{Components:}
	V, S, M\\
	\textbf{Casting Time:}
	1 hour\\
	\textbf{Range:}
	Close (7.5 m + 1.5 m/2 levels)\\
	\textbf{Target:}
	One location (up to a 3-m cube/level) or one object\\
	\textbf{Duration:}
	2 hours/level (D)\\
	\textbf{Saving Throw:}
	Will negates; see text\\
	\textbf{Spell Resistance:}
	Yes\\
}
{
	You cause an object or location to emanate magical vibrations that attract either a specific kind of intelligent creature or creatures of a particular alignment, as defined by you. The particular kind of creature to be affected must be named specifically. A creature subtype is not specific enough. Likewise, the specific alignment must be named.

	Creatures of the specified kind or alignment feel elated and pleased to be in the area or desire to touch or to possess the object. The compulsion to stay in the area or touch the object is overpowering. If the save is successful, the creature is released from the enchantment, but a subsequent save must be made 1d6 $\times$ 10 minutes later. If this save fails, the affected creature attempts to return to the area or object.

	\emph{Sympathy} counters and dispels \spell{antipathy}.

	\textit{Material Component}:
	1,500 cp worth of crushed pearls and a drop of honey.

}

\input{subsections/spells/telekinesis.tex}
\Spell{Telekinetic Sphere}{telekinetic sphere}
{Evocation [Force]}
{
	\textbf{Level:}
	Wiz 8\\
	\textbf{Components:}
	V, S, M\\
	\textbf{Casting Time:}
	1 standard action\\
	\textbf{Range:}
	Close (7.5 m + 1.5 m/2 levels)\\
	\textbf{Effect:}
	1-ft.-diameter/level sphere, centered around creatures or objects\\
	\textbf{Duration:}
	1 min./level (D)\\
	\textbf{Saving Throw:}
	Reflex negates (object)\\
	\textbf{Spell Resistance:}
	Yes (object)\\
}
{
	This spell functions like resilient sphere, with the addition that the creatures or objects inside the globe are nearly weightless. Anything contained within an telekinetic sphere weighs only one-sixteenth of its normal weight. You can telekinetically lift anything in the sphere that normally weighs 2,500 kilograms or less. The telekinetic control extends from you out to medium range (30 meters + 3 meters per caster level) after the sphere has succeeded in encapsulating its contents.

	You can move objects or creatures in the sphere that weigh a total of 2,500 kilograms or less by concentrating on the sphere. You can begin moving a sphere in the round after casting the spell. If you concentrate on doing so (a standard action), you can move the sphere as much as 9 meters in a round. If you cease concentrating, the sphere does not move in that round (if on a level surface) or descends at its falling rate (if aloft) until it reaches a level surface, or the spell's duration expires, or you begin concentrating again. If you cease concentrating (voluntarily or due to failing a \skill{Concentration} check), you can resume concentrating on your next turn or any later turn during the spell's duration.

	The sphere falls at a rate of only 18 meters per round, which is not fast enough to cause damage to the contents of the sphere.

	You can move the sphere telekinetically even if you are in it.

	\textit{Material Component}:
	A hemispherical piece of clear crystal, a matching hemispherical piece of gum arabic, and a pair of small bar magnets.

}

\Spell{Telepathic Bond, Lesser}{lesser telepathic bond}
{Divination [Mind-Affecting]}
{
	\textbf{Level:} Clr 3, Mind 3, Wiz 3\\
	\textbf{Components:} V, S\\
	\textbf{Casting Time:} 1 standard action\\
	\textbf{Range:} 9 m\\
	\textbf{Targets:} You and one willing creature within 9 m\\
	\textbf{Duration:} 10 min./level\\
	\textbf{Saving Throw:} None\\
	\textbf{Spell Resistance:} No\\
}
{
	You forge a telepathic bond with another creature with an Intelligence score of 6 or higher. The bond can be established only with a willing subject. You can communicate telepathically through the bond regardless of language. No special power or influence is established as a result of the bond. Once the bond is formed, it works over any distance (although not from one plane to another).
}
\Spell{Telepathic Bond}{telepathic bond}
{Divination [Ritual]}
{
	\textbf{Level:}
	Wiz 5\\
	\textbf{Components:}
	V, S, M\\
	\textbf{Casting Time:}
	1 standard action\\
	\textbf{Range:}
	Close (7.5 m + 1.5 m/2 levels)\\
	\textbf{Targets:}
	You plus one willing creature per three levels, no two of which can be more than 9 m apart\\
	\textbf{Duration:}
	10 min./level (D)\\
	\textbf{Saving Throw:}
	None\\
	\textbf{Spell Resistance:}
	No\\
}
{
	You forge a \emph{telepathic bond} among yourself and a number of willing creatures, each of which must have an Intelligence score of 3 or higher. Each creature included in the link is linked to all the others. The creatures can communicate telepathically through the bond regardless of language. No special power or influence is established as a result of the bond. Once the bond is formed, it works over any distance (although not from one plane to another).

	If desired, you may leave yourself out of the \emph{telepathic bond} forged. This decision must be made at the time of casting.

	\emph{Telepathic bond} can be made permanent with a \spell{permanency} spell, though it only bonds two creatures per casting of permanency.

	\textit{Material Component:}
	Pieces of eggshell from two different kinds of creatures.

}

\Spell{Teleportation Circle}{teleportation circle}
{Conjuration (Teleportation)}
{
	\textbf{Level:}
	Wiz 9\\
	\textbf{Components:}
	V, M\\
	\textbf{Casting Time:}
	10 minutes\\
	\textbf{Range:}
	0 m\\
	\textbf{Effect:}
	1.5-m-radius circle that teleports those who activate it\\
	\textbf{Duration:}
	10 min./level (D)\\
	\textbf{Saving Throw:}
	None\\
	\textbf{Spell Resistance:}
	Yes\\
}
{
	You create a circle on the floor or other horizontal surface that teleports, as greater teleport, any creature who stands on it to a designated spot. Once you designate the destination for the circle, you can't change it. The spell fails if you attempt to set the circle to teleport creatures into a solid object, to a place with which you are not familiar and have no clear description, or to another plane.

	The circle itself is subtle and nearly impossible to notice. If you intend to keep creatures from activating it accidentally, you need to mark the circle in some way.

	\emph{Teleportation circle} can be made permanent with a \spell{permanency} spell. A permanent \emph{teleportation circle} that is disabled becomes inactive for 10 minutes, then can be triggered again as normal.

	\textit{Note:}
	Magic traps such as \emph{teleportation circle} are hard to detect and disable. A rogue (only) can use the Search skill to find the circle and Disable Device to thwart it. The DC in each case is 25 + spell level, or 34 in the case of \emph{teleportation circle}.

	\textit{Material Component}:
	Amber dust to cover the area of the circle (cost 1,000 cp).

}

\Spell{Teleport, Greater}{greater teleport}
{Conjuration (Teleportation)}
{
	\textbf{Level:}
	Wiz 7, Travel 7\\
}
{
	This spell functions like \spell{teleport}, except that there is no range limit and there is no chance you arrive off target. In addition, you need not have seen the destination, but in that case you must have at least a reliable description of the place to which you are teleporting. If you attempt to teleport with insufficient information (or with misleading information), you disappear and simply reappear in your original location. Interplanar travel is not possible.

}

\Spell{Teleport Object}{teleport object}
{Conjuration (Teleportation)}
{
	\textbf{Level:}
	Wiz 7\\
	\textbf{Range:}
	Touch\\
	\textbf{Target:}
	One touched object of up to 25 kg/level and 0.085 m$^3$/level\\
	\textbf{Saving Throw:}
	Will negates (object)\\
	\textbf{Spell Resistance:}
	Yes (object)\\
}
{
	This spell functions like \spell{teleport}, except that it teleports an object, not you. Creatures and magical forces cannot be teleported.

	If desired, the target object can be sent to a distant location on the Ethereal Plane. In this case, the point from which the object was teleported remains faintly magical until the item is retrieved. A successful targeted dispel magic spell cast on that point brings the vanished item back from the Ethereal Plane.

}

% \input{subsections/spells/teleport.tex}
\input{subsections/spells/temporal-stasis.tex}
\input{subsections/spells/time-stop.tex}
\Spell{Tiny Hut}{tiny hut}
{Evocation [Force]}
{
	\textbf{Level:}
	Wiz 3\\
	\textbf{Components:}
	V, S, M\\
	\textbf{Casting Time:}
	1 standard action\\
	\textbf{Range:}
	6 m\\
	\textbf{Effect:}
	6-m-radius sphere centered on your location\\
	\textbf{Duration:}
	2 hours/level (D)\\
	\textbf{Saving Throw:}
	None\\
	\textbf{Spell Resistance:}
	No\\
}
{
	You create an unmoving, opaque sphere of force of any color you desire around yourself. Half the sphere projects above the ground, and the lower hemisphere passes through the ground. As many as nine other Medium creatures can fit into the field with you; they can freely pass into and out of the hut without harming it. However, if you remove yourself from the hut, the spell ends.

	The temperature inside the hut is 70\textdegree F if the exterior temperature is between 0\textdegree and 100\textdegree F. An exterior temperature below 0\textdegree or above 100\textdegree lowers or raises the interior temperature on a 1-degree-for-1 basis. The hut also provides protection against the elements, such as rain, dust, and sandstorms. The hut withstands any wind of less than hurricane force, but a hurricane (120+ km/h wind speed) or greater force destroys it.

	The interior of the hut is a hemisphere. You can illuminate it dimly upon command or extinguish the light as desired. Although the force field is opaque from the outside, it is transparent from within. Missiles, weapons, and most spell effects can pass through the hut without affecting it, although the occupants cannot be seen from outside the hut (they have total concealment).

	\textit{Material Component:}
	A small crystal bead that shatters when the spell duration expires or the hut is dispelled.

}

\input{subsections/spells/tongues.tex}
\input{subsections/spells/touch-of-fatigue.tex}
\input{subsections/spells/touch-of-idiocy.tex}
\input{subsections/spells/touch-of-madness.tex}
\input{subsections/spells/transformation.tex}
\input{subsections/spells/transmute-metal-to-wood.tex}
\Spell{Transmute Mud to Rock}{transmute mud to rock}
{Transmutation [Earth]}
{
	\textbf{Level:}
	Drd 5, Wiz 5\\
	\textbf{Components:}
	V, S, M/DF\\
	\textbf{Casting Time:}
	1 standard action\\
	\textbf{Range:}
	Medium (30 m + 3 m/level)\\
	\textbf{Area:}
	Up to two 3-m cubes/level (S)\\
	\textbf{Duration:}
	Permanent\\
	\textbf{Saving Throw:}
	See text\\
	\textbf{Spell Resistance:}
	No\\
}
{
	This spell transforms normal mud or quicksand of any depth into soft stone (sandstone or a similar mineral) permanently.

	Any creature in the mud is allowed a Reflex save to escape before the area is hardened to stone.

	Transmute mud to rock counters and dispels transmute rock to mud.

	\textit{Arcane Material Component}:
	Sand, lime, and water.

}

\input{subsections/spells/transmute-rock-to-mud.tex}
\input{subsections/spells/transport-via-plants.tex}
\input{subsections/spells/trap-the-soul.tex}
\input{subsections/spells/tree-shape.tex}
% \Spell{Tree Stride}{tree stride}
{Conjuration (Teleportation)}
{
	\textbf{Level:}
	Drd 5, Rgr 4\\
	\textbf{Components:}
	V, S, DF\\
	\textbf{Casting Time:}
	1 standard action\\
	\textbf{Range:}
	Personal\\
	\textbf{Target:}
	You\\
	\textbf{Duration:}
	1 hour/level or until expended; see text\\
}
{
{
Type of Tree
Transport Range
	Oak, ash, yew\\
	3,000 feet\\
	Elm, linden\\
	2,000 feet\\
	Other deciduous\\
	1,500 feet\\
	Any coniferous\\
	1,000 feet\\
	All other trees\\
	500 feet\\
}
{
	You gain the ability to enter trees and move from inside one tree to inside another tree. The first tree you enter and all others you enter must be of the same kind, must be living, and must have girth at least equal to yours. By moving into an oak tree (for example), you instantly know the location of all other oak trees within transport range (see below) and may choose whether you want to pass into one or simply step back out of the tree you moved into. You may choose to pass to any tree of the appropriate kind within the transport range as shown on the following table.

	You may move into a tree up to one time per caster level (passing from one tree to another counts only as moving into one tree). The spell lasts until the duration expires or you exit a tree. Each transport is a full-round action.

	You can, at your option, remain within a tree without transporting yourself, but you are forced out when the spell ends. If the tree in which you are concealed is chopped down or burned, you are slain if you do not exit before the process is complete.

}

\input{subsections/spells/true-resurrection.tex}
\Spell{True Seeing}{true seeing}
{Divination}
{
	\textbf{Level:}
	Clr 5, Drd 7, Knowledge 5, Wiz 6\\
	\textbf{Components:}
	V, S, M\\
	\textbf{Casting Time:}
	1 standard action\\
	\textbf{Range:}
	Touch\\
	\textbf{Target:}
	Creature touched\\
	\textbf{Duration:}
	1 min./level\\
	\textbf{Saving Throw:}
	Will negates (harmless)\\
	\textbf{Spell Resistance:}
	Yes (harmless)\\
}
{
	You confer on the subject the ability to see all things as they actually are. The subject sees through normal and magical darkness, notices secret doors hidden by magic, sees the exact locations of creatures or objects under blur or displacement effects, sees invisible creatures or objects normally, sees through illusions, and sees the true form of polymorphed, changed, or transmuted things. Further, the subject can focus its vision to see into the Ethereal Plane (but not into extradimensional spaces). The range of true seeing conferred is 36 meters.

	True seeing, however, does not penetrate solid objects. It in no way confers X-ray vision or its equivalent. It does not negate concealment, including that caused by fog and the like. True seeing does not help the viewer see through mundane disguises, spot creatures who are simply hiding, or notice secret doors hidden by mundane means. In addition, the spell effects cannot be further enhanced with known magic, so one cannot use true seeing through a crystal ball or in conjunction with clairaudience/clairvoyance.

	\textit{Material Component}:
	An ointment for the eyes that costs 250 gp and is made from mushroom powder, saffron, and fat.

}

\input{subsections/spells/true-strike.tex}
\Spell{Undeath to Death}{undeath to death}
{Necromancy}
{
	\textbf{Level:}
	Clr 6, Wiz 6\\
	\textbf{Components:}
	V, S, M/DF\\
	\textbf{Area:}
	Several undead creatures within a 12-m-radius burst\\
	\textbf{Saving Throw:}
	Will negates\\
}
{
	This spell functions like \spell{circle of death}, except that it destroys undead creatures as noted above.

	\textit{Material Component}:
	The powder of a crushed diamond worth at least 500 gp.

}

\Spell{Undetectable Alignment}{undetectable alignment}
{Abjuration}
{
	\textbf{Level:}
	Ass 2, Clr 2, Tmp 2\\
	\textbf{Components:}
	V, S\\
	\textbf{Casting Time:}
	1 standard action\\
	\textbf{Range:}
	Close (7.5 m + 1.5 m/2 levels)\\
	\textbf{Target:}
	One creature or object\\
	\textbf{Duration:}
	24 hours\\
	\textbf{Saving Throw:}
	Will negates (object)\\
	\textbf{Spell Resistance:}
	Yes (object)\\
}
{
	An undetectable alignment spell conceals the alignment of an object or a creature from all forms of divination.

}

\input{subsections/spells/unhallow.tex}
\Spell{Unholy Aura}{unholy aura}
{Abjuration [Evil]}
{
	\textbf{Level:}
	Clr 8, Evil 8\\
	\textbf{Components:}
	V, S, F\\
	\textbf{Casting Time:}
	1 standard action\\
	\textbf{Range:}
	6 m\\
	\textbf{Targets:}
	One creature/level in a 6-m-radius burst centered on you\\
	\textbf{Duration:}
	1 round/level (D)\\
	\textbf{Saving Throw:}
	See text\\
	\textbf{Spell Resistance:}
	Yes (harmless)\\
}
{
	A malevolent darkness surrounds the subjects, protecting them from attacks, granting them resistance to spells cast by good creatures, and weakening good creatures when they strike the subjects. This abjuration has four effects.

	First, each warded creature gains a +4 deflection bonus to AC and a +4 resistance bonus on saves. Unlike the effect of protection from good, this benefit applies against all attacks, not just against attacks by good creatures.

	Second, a warded creature gains spell resistance 25 against good spells and spells cast by good creatures.

	Third, the abjuration blocks possession and mental influence, just as protection from good does.

	Finally, if a good creature succeeds on a melee attack against a warded creature, the offending attacker takes 1d6 points of temporary Strength damage (Fortitude negates).

	\textit{Focus:}
	A tiny reliquary containing some sacred relic, such as a piece of parchment from an unholy text. The reliquary costs at least 500 cp.

}

\input{subsections/spells/unholy-blight.tex}
\Spell{Unseen Servant}{unseen servant}
{Conjuration (Creation)}
{
	\textbf{Level:}
	Wiz 1\\
	\textbf{Components:}
	V, S, M\\
	\textbf{Casting Time:}
	1 standard action\\
	\textbf{Range:}
	Close (7.5 m + 1.5 m/2 levels)\\
	\textbf{Effect:}
	One invisible, mindless, shapeless servant\\
	\textbf{Duration:}
	1 hour/level\\
	\textbf{Saving Throw:}
	None\\
	\textbf{Spell Resistance:}
	No\\
}
{
	An unseen servant is an invisible, mindless, shapeless force that performs simple tasks at your command. It can run and fetch things, open unstuck doors, and hold chairs, as well as clean and mend. The servant can perform only one activity at a time, but it repeats the same activity over and over again if told to do so as long as you remain within range. It can open only normal doors, drawers, lids, and the like. It has an effective Strength score of 2 (so it can lift 10 kilograms or drag 50 kilograms). It can trigger traps and such, but it can exert only 10 kilograms of force, which is not enough to activate certain pressure plates and other devices. It can't perform any task that requires a skill check with a DC higher than 10 or that requires a check using a skill that can't be used untrained. Its speed is 4.5 meters.

	The servant cannot attack in any way; it is never allowed an attack roll. It cannot be killed, but it dissipates if it takes 6 points of damage from area attacks. (It gets no saves against attacks.) If you attempt to send it beyond the spell's range (measured from your current position), the servant ceases to exist.

	\textit{Material Component:}
	A piece of string and a bit of wood.

}

\input{subsections/spells/vampiric-touch.tex}
\input{subsections/spells/veil.tex}
\input{subsections/spells/ventriloquism.tex}
\input{subsections/spells/virtue.tex}
\input{subsections/spells/vision.tex}
\Spell{Wail of the Banshee}{wail of the banshee}
{Necromancy [Death, Sonic]}
{
	\textbf{Level:}
	Death 9, Wiz 9\\
	\textbf{Components:}
	V\\
	\textbf{Casting Time:}
	1 standard action\\
	\textbf{Range:}
	Close (7.5 m + 1.5 m/2 levels)\\
	\textbf{Area:}
	One living creature/level within a 12-m-radius spread\\
	\textbf{Duration:}
	Instantaneous\\
	\textbf{Saving Throw:}
	Fortitude negates\\
	\textbf{Spell Resistance:}
	Yes\\
}
{
	You emit a terrible scream that kills creatures that hear it (except for yourself). Creatures closest to the point of origin are affected first.

}

\Spell{Wall of Fire}{wall of fire}
{Evocation [Fire]}
{
	\textbf{Level:}
	Drd 5, Fire 4, Wiz 4\\
	\textbf{Components:}
	V, S, M/DF\\
	\textbf{Casting Time:}
	1 standard action\\
	\textbf{Range:}
	Medium (30 m + 3 m/level)\\
	\textbf{Effect:}
	Opaque sheet of flame up to 6 m long/level or a ring of fire with a radius of up to 1.5 m per two levels; either form 6 m high\\
	\textbf{Duration:}
	Concentration + 1 round/level\\
	\textbf{Saving Throw:}
	None\\
	\textbf{Spell Resistance:}
	Yes\\
}
{
	An immobile, blazing curtain of shimmering violet fire springs into existence. One side of the wall, selected by you, sends forth waves of heat, dealing 2d4 points of fire damage to creatures within 3 meters and 1d4 points of fire damage to those past 3 meters but within 6 meters. The wall deals this damage when it appears and on your turn each round to all creatures in the area. In addition, the wall deals 2d6 points of fire damage +1 point of fire damage per caster level (maximum +20) to any creature passing through it. The wall deals double damage to undead creatures.

	If you evoke the wall so that it appears where creatures are, each creature takes damage as if passing through the wall. If any 1.5-meter length of wall takes 20 points of cold damage or more in 1 round, that length goes out. (Do not divide cold damage by 4, as normal for objects.)

	\emph{Wall of fire} can be made permanent with a \spell{permanency} spell. A permanent \emph{wall of fire} that is extinguished by cold damage becomes inactive for 10 minutes, then reforms at normal strength.

	\textit{Arcane Material Component}:
	A small piece of phosphorus.

}

\Spell{Wall of Force}{wall of force}
{Evocation [Force]}
{
	\textbf{Level:}
	Wiz 5\\
	\textbf{Components:}
	V, S, M\\
	\textbf{Casting Time:}
	1 standard action\\
	\textbf{Range:}
	Close (7.5 m + 1.5 m/2 levels)\\
	\textbf{Effect:}
	Wall whose area is up to one 3-m square/level\\
	\textbf{Duration:}
	1 round /level (D)\\
	\textbf{Saving Throw:}
	None\\
	\textbf{Spell Resistance:}
	No\\
}
{
	A wall of force spell creates an invisible wall of force. The wall cannot move, it is immune to damage of all kinds, and it is unaffected by most spells, including dispel magic. However, disintegrate immediately destroys it, as does a rod of cancellation, a sphere of annihilation, or a mage's disjunction spell. Breath weapons and spells cannot pass through the wall in either direction, although dimension door, teleport, and similar effects can bypass the barrier. It blocks ethereal creatures as well as material ones (though ethereal creatures can usually get around the wall by floating under or over it through material floors and ceilings). Gaze attacks can operate through a wall of force.

	The caster can form the wall into a flat, vertical plane whose area is up to one 3-meter square per level. The wall must be continuous and unbroken when formed. If its surface is broken by any object or creature, the spell fails.

	\emph{Wall of force} can be made permanent with a \spell{permanency} spell.

	\textit{Material Component}:
	A pinch of powder made from a clear gem.

}

\Spell{Wall of Ice}{wall of ice}
{Evocation [Cold]}
{
	\textbf{Level:}
	Wiz 4\\
	\textbf{Components:}
	V, S, M\\
	\textbf{Casting Time:}
	1 standard action\\
	\textbf{Range:}
	Medium (30 m + 3 m/level)\\
	\textbf{Effect:}
	Anchored plane of ice, up to one 3-m square/level, or hemisphere of ice with a radius of up to 1 m + 0.3 m/level\\
	\textbf{Duration:}
	1 min./level\\
	\textbf{Saving Throw:}
	Reflex negates; see text\\
	\textbf{Spell Resistance:}
	Yes\\
}
{
	This spell creates an anchored plane of ice or a hemisphere of ice, depending on the version selected. A wall of ice cannot form in an area occupied by physical objects or creatures. Its surface must be smooth and unbroken when created. Any creature adjacent to the wall when it is created may attempt a Reflex save to disrupt the wall as it is being formed. A successful save indicates that the spell automatically fails. Fire can melt a wall of ice, and it deals full damage to the wall (instead of the normal half damage taken by objects). Suddenly melting a wall of ice creates a great cloud of steamy fog that lasts for 10 minutes.

	\textit{Ice Plane:}
	A sheet of strong, hard ice appears. The wall is 1 inch thick per caster level. It covers up to a 3-meter-square area per caster level (so a 10th-level wizard can create a wall of ice 30 meters long and 3 meters high, a wall 15 meters long and 6 meters high, or some other combination of length and height that does not exceed 90 square meters). The plane can be oriented in any fashion as long as it is anchored. A vertical wall need only be anchored on the floor, while a horizontal or slanting wall must be anchored on two opposite sides.

	Each 3-meter square of wall has 3 hit points per inch of thickness. Creatures can hit the wall automatically. A section of wall whose hit points drop to 0 is breached. If a creature tries to break through the wall with a single attack, the DC for the Strength check is 15 + caster level.

	Even when the ice has been broken through, a sheet of frigid air remains. Any creature stepping through it (including the one who broke through the wall) takes 1d6 points of cold damage +1 point per caster level (no save).

	\textit{Hemisphere:}
	The wall takes the form of a hemisphere whose maximum radius is 1 meter + 30 centimeters per caster level. The hemisphere is as hard to break through as the ice plane form, but it does not deal damage to those who go through a breach.

	\textit{Material Component:}
	A small piece of quartz or similar rock crystal.

}

\input{subsections/spells/wall-of-iron.tex}
\Spell{Wall of Stone}{wall of stone}
{Conjuration (Creation) [Earth]}
{
	\textbf{Level:}
	Clr 5, Drd 6, Earth 5, Wiz 5\\
	\textbf{Components:}
	V, S, M/DF\\
	\textbf{Casting Time:}
	1 standard action\\
	\textbf{Range:}
	Medium (30 m + 3 m/level)\\
	\textbf{Effect:}
	Stone wall whose area is up to one 1.5-m square/level (S)\\
	\textbf{Duration:}
	Instantaneous\\
	\textbf{Saving Throw:}
	See text\\
	\textbf{Spell Resistance:}
	No\\
}
{
	This spell creates a wall of rock that merges into adjoining rock surfaces. A wall of stone is 2.5 centimeters thick per four caster levels and composed of up to one 1.5-meter square per level. You can double the wall's area by halving its thickness. The wall cannot be conjured so that it occupies the same space as a creature or another object.

	Unlike a wall of iron, you can create a wall of stone in almost any shape you desire. The wall created need not be vertical, nor rest upon any firm foundation; however, it must merge with and be solidly supported by existing stone. It can be used to bridge a chasm, for instance, or as a ramp. For this use, if the span is more than 6 meters, the wall must be arched and buttressed. This requirement reduces the spell's area by half. The wall can be crudely shaped to allow crenellations, battlements, and so forth by likewise reducing the area.

	Like any other stone wall, this one can be destroyed by a disintegrate spell or by normal means such as breaking and chipping. Each 1.5-meter square of the wall has 15 hit points per 2.5 centimeters of thickness and hardness 8. A section of wall whose hit points drop to 0 is breached. If a creature tries to break through the wall with a single attack, the DC for the Strength check is 20 + 2 per 2.5 centimeters of thickness.

	It is possible, but difficult, to trap mobile opponents within or under a wall of stone, provided the wall is shaped so it can hold the creatures. Creatures can avoid entrapment with successful Reflex saves.

	\textit{Arcane Material Component:}
	A small block of granite.

}

\input{subsections/spells/wall-of-thorns.tex}
\input{subsections/spells/warp-wood.tex}
\input{subsections/spells/water-breathing.tex}
\input{subsections/spells/water-walk.tex}
\Spell{Waves of Exhaustion}{waves of exhaustion}
{Necromancy}
{
	\textbf{Level:}
	Wiz 7\\
	\textbf{Components:}
	V, S\\
	\textbf{Casting Time:}
	1 standard action\\
	\textbf{Range:}
	18 m\\
	\textbf{Area:}
	Cone-shaped burst\\
	\textbf{Duration:}
	Instantaneous\\
	\textbf{Saving Throw:}
	No\\
	\textbf{Spell Resistance:}
	Yes\\
}
{
	Waves of negative energy cause all living creatures in the spell's area to become exhausted. This spell has no effect on a creature that is already exhausted.

}

\Spell{Waves of Fatigue}{waves of fatigue}
{Necromancy}
{
	\textbf{Level:}
	Wiz 5\\
	\textbf{Components:}
	V, S\\
	\textbf{Casting Time:}
	1 standard action\\
	\textbf{Range:}
	9 m\\
	\textbf{Area:}
	Cone-shaped burst\\
	\textbf{Duration:}
	Instantaneous\\
	\textbf{Saving Throw:}
	No\\
	\textbf{Spell Resistance:}
	Yes\\
}
{
	Waves of negative energy render all living creatures in the spell's area fatigued. This spell has no effect on a creature that is already fatigued.

}

\Spell{Web}{web}
{Conjuration (Creation)}
{
	\textbf{Level:}
	Wiz 2\\
	\textbf{Components:}
	V, S, M\\
	\textbf{Casting Time:}
	1 standard action\\
	\textbf{Range:}
	Medium (30 m + 3 m/level)\\
	\textbf{Effect:}
	Webs in a 6-m-radius spread\\
	\textbf{Duration:}
	10 min./level (D)\\
	\textbf{Saving Throw:}
	Reflex negates; see text\\
	\textbf{Spell Resistance:}
	No\\
}
{
	\emph{Web} creates a many-layered mass of strong, sticky strands. These strands trap those caught in them. The strands are similar to spider webs but far larger and tougher. These masses must be anchored to two or more solid and diametrically opposed points or else the \emph{web} collapses upon itself and disappears. Creatures caught within a web become entangled among the gluey fibers. Attacking a creature in a web won't cause you to become entangled.

	Anyone in the effect's area when the spell is cast must make a Reflex save. If this save succeeds, the creature is entangled, but not prevented from moving, though moving is more difficult than normal for being entangled (see below). If the save fails, the creature is entangled and can't move from its space, but can break loose by spending 1 round and making a DC 20 Strength check or a DC 25 \skill{Escape Artist} check. Once loose (either by making the initial Reflex save or a later Strength check or \skill{Escape Artist} check), a creature remains entangled, but may move through the web very slowly. Each round devoted to moving allows the creature to make a new Strength check or \skill{Escape Artist} check. The creature moves 1.5 meter for each full 5 points by which the check result exceeds 10.

	If you have at least 1.5 meter of web between you and an opponent, it provides cover. If you have at least 6 meters of web between you, it provides total cover.

	The strands of a \emph{web} spell are flammable. A magic flaming sword can slash them away as easily as a hand brushes away cobwebs. Any fire can set the webs alight and burn away 0.5 square meter in 1 round. All creatures within flaming webs take 2d4 points of fire damage from the flames.

	\emph{Web} can be made permanent with a \spell{permanency} spell. A permanent \emph{web} that is damaged (but not destroyed) regrows in 10 minutes.

	\textit{Material Component:}
	A bit of spider web.

}

\Spell{Weird}{weird}
{Illusion (Phantasm) [Fear, Mind-Affecting]}
{
	\textbf{Level:}
	Wiz 9\\
	\textbf{Targets:}
	Any number of creatures, no two of which can be more than 9 m apart\\
}
{
	This spell functions like \spell{phantasmal killer}, except it can affect more than one creature. Only the affected creatures see the phantasmal creatures attacking them, though you see the attackers as shadowy shapes.

	If a subject's Fortitude save succeeds, it still takes 3d6 points of damage and is stunned for 1 round. The subject also takes 1d4 points of temporary Strength damage.

}

\input{subsections/spells/whirlwind.tex}
\input{subsections/spells/whispering-wind.tex}
\input{subsections/spells/wind-walk.tex}
\Spell{Wind Wall}{wind wall}
{Evocation [Air]}
{
	\textbf{Level:}
	Air 2, Clr 3, Drd 3, Rgr 2, Wiz 3\\
	\textbf{Components:}
	V, S, M/DF\\
	\textbf{Casting Time:}
	1 standard action\\
	\textbf{Range:}
	Medium (30 m + 3 m/level)\\
	\textbf{Effect:}
	Wall up to 3 m/level long and 1.5 m/level high (S)\\
	\textbf{Duration:}
	1 round/level\\
	\textbf{Saving Throw:}
	None; see text\\
	\textbf{Spell Resistance:}
	Yes\\
}
{
	An invisible vertical curtain of wind appears. It is 2 feet thick and of considerable strength. It is a roaring blast sufficient to blow away any bird smaller than an eagle, or tear papers and similar materials from unsuspecting hands. (A Reflex save allows a creature to maintain its grasp on an object.) Tiny and Small flying creatures cannot pass through the barrier. Loose materials and cloth garments fly upward when caught in a wind wall. Arrows and bolts are deflected upward and miss, while any other normal ranged weapon passing through the wall has a 30\% miss chance. (A giant-thrown boulder, a siege engine projectile, and other massive ranged weapons are not affected.) Gases, most gaseous breath weapons, and creatures in gaseous form cannot pass through the wall (although it is no barrier to incorporeal creatures).

	While the wall must be vertical, you can shape it in any continuous path along the ground that you like. It is possible to create cylindrical or square wind walls to enclose specific points.

	\textit{Arcane Material Component}:
	A tiny fan and a feather of exotic origin.

}

\Spell{Wish}{wish}
{Universal}
{
	\textbf{Level:}
	Wiz 9\\
	\textbf{Components:}
	V, XP\\
	\textbf{Casting Time:}
	1 standard action\\
	\textbf{Range:}
	See text\\
	\textbf{Target, Effect, or Area:}
	See text\\
	\textbf{Duration:}
	See text\\
	\textbf{Saving Throw:}
	See text\\
	\textbf{Spell Resistance:}
	Yes\\
}
{
	\emph{Wish} is the mightiest spell a wizard can cast. By simply speaking aloud, you can alter reality to better suit you.

	Even \emph{wish}, however, has its limits.

	A \emph{wish} can produce any one of the following effects.

	\begin{itemize*}
	\item Duplicate any wizard spell of 8th level or lower, provided the spell is not of a school prohibited to you.
	\item Duplicate any other spell of 6th level or lower, provided the spell is not of a school prohibited to you.
	\item Duplicate any wizard spell of 7th level or lower even if it's of a prohibited school.
	\item Duplicate any other spell of 5th level or lower even if it's of a prohibited school.
	\item Undo the harmful effects of many other spells, such as geas/quest or insanity.
	\item Create a nonmagical item of up to 25,000 cp in value.
	\item Create a magic item, or add to the powers of an existing magic item.
	\item Grant a creature a +1 inherent bonus to an ability score. Two to five \emph{wish} spells cast in immediate succession can grant a creature a +2 to +5 inherent bonus to an ability score (two \emph{wish}es for a +2 inherent bonus, three for a +3 inherent bonus, and so on). Inherent bonuses are instantaneous, so they cannot be dispelled. \textit{Note:} An inherent bonus may not exceed +5 for a single ability score, and inherent bonuses to a particular ability score do not stack, so only the best one applies.
	\item Remove injuries and afflictions. A single \emph{wish} can aid one creature per caster level, and all subjects are cured of the same kind of affliction. For example, you could heal all the damage you and your companions have taken, or remove all poison effects from everyone in the party, but not do both with the same \emph{wish}. A \emph{wish} can never restore the experience point loss from casting a spell or the level or Constitution loss from being raised from the dead.
	\item Revive the dead. A \emph{wish} can bring a dead creature back to life by duplicating a resurrection spell. A \emph{wish} can revive a dead creature whose body has been destroyed, but the task takes two \emph{wish}es, one to recreate the body and another to infuse the body with life again. A \emph{wish} cannot prevent a character who was brought back to life from losing an experience level.
	\item Transport travelers. A \emph{wish} can lift one creature per caster level from anywhere on any plane and place those creatures anywhere else on any plane regardless of local conditions. An unwilling target gets a Will save to negate the effect, and spell resistance (if any) applies.
	\item Undo misfortune. A \emph{wish} can undo a single recent event. The \emph{wish} forces a reroll of any roll made within the last round (including your last turn). Reality reshapes itself to accommodate the new result. For example, a \emph{wish} could undo an opponent's successful save, a foe's successful critical hit (either the attack roll or the critical roll), a friend's failed save, and so on. The reroll, however, may be as bad as or worse than the original roll. An unwilling target gets a Will save to negate the effect, and spell resistance (if any) applies.
	\end{itemize*}

	You may try to use a \emph{wish} to produce greater effects than these, but doing so is dangerous. (The \emph{wish} may pervert your intent into a literal but undesirable fulfillment or only a partial fulfillment.)

	Duplicated spells allow saves and spell resistance as normal (but save DCs are for 9th-level spells).

	\textit{Material Component}:
	When a \emph{wish} duplicates a spell with a material component that costs more than 10,000 cp, you must provide that component.

	\textit{XP Cost}:
	The minimum XP cost for casting \emph{wish} is 5,000 XP. When a \emph{wish} duplicates a spell that has an XP cost, you must pay 5,000 XP or that cost, whichever is more. When a \emph{wish} creates or improves a magic item, you must pay twice the normal XP cost for crafting or improving the item, plus an additional 5,000 XP.

}

\input{subsections/spells/wood-shape.tex}
\Spell{Word of Chaos}{word of chaos}
{Evocation [Chaotic, Sonic]}
{
	\textbf{Level:}
	Chaos 7, Clr 7\\
	\textbf{Components:}
	V\\
	\textbf{Casting Time:}
	1 standard action\\
	\textbf{Range:}
	12 m\\
	\textbf{Area:}
	Nonchaotic creatures in a 12-m-radius spread centered on you\\
	\textbf{Duration:}
	Instantaneous\\
	\textbf{Saving Throw:}
	None or Will negates; see text\\
	\textbf{Spell Resistance:}
	Yes\\
}
{
	Any nonchaotic creature within the area who hears the word of chaos suffers the following ill effects.

	The effects are cumulative and concurrent. No saving throw is allowed against these effects.

\Table{}{XX}{
\tableheader HD & \tableheader Effect\\
	Equal to caster level & Deafened\\
	Up to caster level $-1$ & Stunned, deafened\\
	Up to caster level $-5$ & Confused, stunned, deafened\\
	Up to caster level $-10$ & Killed, confused, stunned, deafened\\
}

	\textit{Deafened}:
	The creature is deafened for 1d4 rounds.

	\textit{Stunned}:
	The creature is stunned for 1 round.

	\textit{Confused}:
	The creature is confused, as by the confusion spell, for 1d10 minutes. This is a mind-affecting enchantment effect.

	\textit{Killed}:
	Living creatures die. Undead creatures are destroyed.

	Furthermore, if you are on your home plane when you cast this spell, nonchaotic extraplanar creatures within the area are instantly banished back to their home planes. Creatures so banished cannot return for at least 24 hours. This effect takes place regardless of whether the creatures hear the word of chaos. The banishment effect allows a Will save (at a $-4$ penalty) to negate.

	Creatures whose HD exceed your caster level are unaffected by word of chaos.

}

\input{subsections/spells/word-of-recall.tex}
\input{subsections/spells/zone-of-silence.tex}
\input{subsections/spells/zone-of-truth.tex}
