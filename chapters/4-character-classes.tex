\section{The Classes}

\Capitalize{Y}{ou} may notice that there are some classes not described here. Some of these are core classes that have been deemed inappropriate to the feel of the Dark Sun campaign setting. Others are classes from previous editions of Dark Sun that don't fit in the 3e.

\textbf{Monk:} There are several monasteries on Athas, though little evidence in previous material supports the martial artist variety of monk. Monks are too few in number to warrant a core class.

\textbf{Paladin:} The idea of doing good for its own sake runs contrary to the tone and theme of the setting. There are no gods to reward selfless acts, and no grand traditions of chivalry and nobility to promote. In essence, Athas is a world where evil behavior is the norm.

\textbf{Sorcerer:} Mechanically, a sorcerer's spontaneous casting and a psion's manifesting are similar, thus including the sorcerer removes some of the uniqueness of the psion. Some also feel that an arcane spellcaster without a spellbook violates the flavor of the setting.

\textbf{Soulknife:} There is no precedent of a concept such as the soulknife in any previous material. However, in a metal‐poor, high psionic world, the ability to manifest a weapon using the mind has its place. There would probably not be enough soulknives to warrant a core class, which would need to be shoehorned into the existing campaign world.

\textbf{Trader:} The trader class, present in previous editions of the setting is not included here because it's benefits and traits are nearly all encompassed in the standard set of 3rd edition skills. Reproducing the class is easily done using a standard skill‐focused class, like the rogue or bard, or using the expert NPC class.

Some DMs choose to run Dark Sun as a low‐magic, low treasure campaign. In such games, the monk and soulknife could become unbalanced because of their lack of dependence on treasure.

DMs are free to include any of the above core classes in their games, but these classes will not appear in any official releases.

\section{Barbarian}

\Quote[-2em]{Gith's blood! I will hunt that wizard down and skin him alive.}{Borac, mul barbarian}

Brutality is a way of life in Athas, as much in some of the cities as in the dwindling tribes of Athas' harsh wastes. Cannibal headhunting halflings (who occasionally visit Urik from the Forest Ridge) sometimes express shock at the savagery and bloodshed of the folk that call themselves ``civilized'' and live between walls of stone. They would be more horrified if they were to see the skull piles of Draj, experience the Red Moon Hunt in Gulg, or watch a seemingly docile house slave in Eldaarich rage as she finally ``goes feral'', taking every frustration of her short cruel life out on whoever happens to be closest to hand. Nibenese sages claim that the potential for savagery is in every sentient race, and the history of Athas seems to support their claim.

Some on Athas have turned their brutality into an art of war. They are known as ``brutes'', ``barbarians'' or ``feral warriors'' and they wear the name with pride. Impious but superstitious, cunning and merciless, fearless and persistent, they have carved a name for their martial traditions out of fear and blood.

\WarriorTable{The Barbarian}{
1 & +1 & +2 & +0 & +0 & Fast movement, rage 1/day \\
2 & +2 & +3 & +0 & +0 & Uncanny dodge \\
3 & +3 & +3 & +1 & +1 & Wasteland trap sense +1 \\
4 & +4 & +4 & +1 & +1 & Rage 2/day \\
5 & +5 & +4 & +1 & +1 & Improved uncanny dodge \\
6 & +6/+1 & +5 & +2 & +2 & Wasteland trap sense +2 \\
7 & +7/+2 & +5 & +2 & +2 & Damage reduction 1/-- \\
8 & +8/+3 & +6 & +2 & +2 & Rage 3/day \\
9 & +9/+4 & +6 & +3 & +3 & Wasteland trap sense +3 \\
10 & +10/+5 & +7 & +3 & +3 & Damage reduction 2/-- \\
11 & +11/+6/+1 & +7 & +3 & +3 & Greater rage \\
12 & +12/+7/+2 & +8 & +4 & +4 & Rage 4/day, wasteland trap sense +4 \\
13 & +13/+8/+3 & +8 & +4 & +4 & Damage reduction 3/-- \\
14 & +14/+9/+4 & +9 & +4 & +4 & Indomitable will \\
15 & +15/+10/+5 & +9 & +5 & +5 & Wasteland trap sense +5 \\
16 & +16/+11/+6/+1 & +10 & +5 & +5 & Damage reduction 4/--, rage 5/day \\
17 & +17/+12/+7/+2 & +10 & +5 & +5 & Tireless rage \\
18 & +18/+13/+8/+3 & +11 & +6 & +6 & Wasteland trap sense +6 \\
19 & +19/+14/+9/+4 & +11 & +6 & +6 & Damage reduction 5/-- \\
20 & +20/+15/+10/+5 & +12 & +6 & +6 & Mighty rage, rage 6/day}

\subsection{Making a Barbarian}

The barbarian is a fearsome warrior, compensating for lack of training and discipline with bouts of powerful rage. While in this berserk fury, barbarians become stronger and tougher, better able to defeat their foes and withstand attacks. These rages leave barbarians winded; at first they only have the energy for a few such spectacular displays per day, but those few rages are usually sufficient.

\textbf{Races:} Humans are often barbarians, many having been raised in the wastes or escaped from slavery. Half‐elves sometimes become barbarians, having been abandoned by their elven parents to the desert to survive on their own; if more of them survived they would be quite numerous. Dwarves are very rarely barbarians, but their mul half‐children take to brutishness like a bird takes to flight, living by their wits and strengths in the wastes. Muls have a particular inclination this way of life, and very often ``go feral'' in the wilderness after escaping slavery in the city. Elves rarely take to the barbarian class; those that do are usually from raiding tribes such as the Silt Stalkers. Half‐giants readily take the barbarian class. Despite their feral reputations, halflings rarely become barbarians; their small statures and weak strength adapts them better for the ranger class. Likewise, despite their wild nature, thri‐kreen are rarely barbarians, since their innate memories allow them to gain more specialized classes such as ranger and psychic warrior without training. Pterrans of the Forest Ridge occasionally become barbarians, but like halflings they more often favor the ranger class.

\textbf{Alignment:} Barbarians are never lawful — their characteristic rage is anything but disciplined and controlled. Many barbarians in the cities are often rejects from the regular army, unable to bear regular discipline or training. Some may be honorable, but at heart they are wild. At best, chaotic barbarians are free and expressive. At worst, they are thoughtlessly destructive.

\subsection{Game Rule Information}
\textbf{Alignment:} Any nonlawful.

\textbf{Hit Die:} d12.

\subsubsection{Class Skills}
Climb (Str), Craft (Int), Escape Artist (Des), Handle Animal (Cha), Intimidate (Cha), Jump (Str), Listen (Wis), Profession (Wis), Ride (Dex), and Survival (Wis).

\textbf{Skill Points per Level:} 4 + Int modifier ($\times4$ at 1st level).

\subsubsection{Class Features}

\textbf{Weapon and Armor Proficiency:} A barbarian is proficient with all simple and martial weapons, light armor, medium armor, and shields (except tower shields).

\textbf{Fast Movement (Ex):} A barbarian's land speed is faster than the norm for his race by +10 feet. This benefit applies only when he is wearing no armor, light armor, or medium armor and not carrying a heavy load. Apply this bonus before modifying the barbarian's speed because of any load carried or armor worn.

\textbf{Rage (Ex):} A barbarian can fly into a rage a certain number of times per day. In a rage, a barbarian temporarily gains a +4 bonus to Strength, a +4 bonus to Constitution, and a +2 morale bonus on Will saves, but he takes a $-2$ penalty to Armor Class. The increase in Constitution increases the barbarian's hit points by 2 points per level, but these hit points go away at the end of the rage when his Constitution  score drops back to normal. (These extra hit points are not lost first the way temporary hit points are.) While raging, a barbarian cannot use any Charisma-, Dexterity-, or Intelligence-based skills (except for Balance, Escape Artist, Intimidate, and Ride), the Concentration skill, or any abilities that require patience or concentration, nor can he cast spells or activate magic items that require a command word, a spell trigger (such as a wand), or spell completion (such as a scroll) to function. He can use any feat he has except Combat Expertise, item creation feats, and metamagic feats. A fit of rage lasts for a number of rounds equal to 3 + the character's (newly improved) Constitution modifier. A barbarian may prematurely end his rage. At the end of the rage, the barbarian loses the rage modifiers and restrictions and becomes fatigued ($-2$ penalty to Strength, $-2$ penalty to Dexterity, can't charge or run) for the duration of the current encounter (unless he is a 17th-level barbarian, at which point this limitation no longer applies).

A barbarian can fly into a rage only once per encounter. At 1st level he can use his rage ability once per day. At 4th level and every four levels thereafter, he can use it one additional time per day (to a maximum of six times per day at 20th level). Entering a rage takes no time itself, but a barbarian can do it only during his action, not in response to someone else's action. 

\textbf{Uncanny Dodge (Ex):} At 2nd level, a barbarian retains his Dexterity bonus to AC (if any) even if he is caught flat-footed or struck by an invisible attacker. However, he still loses his Dexterity bonus to AC if immobilized. If a barbarian already has uncanny dodge from a different class, he automatically gains improved uncanny dodge instead.

\textbf{Wasteland Trap Sense (Ex):} Starting at 3rd level, a barbarian gains a +1 bonus on Reflex saves made to avoid traps and natural hazards, and a +1 dodge bonus to AC against attacks made by traps and natural hazards. These bonuses rise by +1 every three barbarian levels thereafter (6th, 9th, 12th, 15th, and 18th level). Trap sense bonuses gained from multiple classes stack.

\textbf{Improved Uncanny Dodge (Ex):} At 5th level and higher, a barbarian can no longer be flanked. This defense denies a rogue the ability to sneak attack the barbarian by flanking him, unless the attacker has at least four more rogue levels than the target has barbarian levels. If a character already has uncanny dodge from a second class, the character automatically gains improved uncanny dodge instead, and the levels from the classes that grant uncanny dodge stack to determine the minimum level a rogue must be to flank the character.

\textbf{Damage Reduction (Ex):} At 7th level, a barbarian gains Damage Reduction. Subtract 1 from the damage the barbarian takes each time he is dealt damage from a weapon or a natural attack. At 10th level, and every three barbarian levels thereafter (13th, 16th, and 19th level), this damage reduction rises by 1 point. Damage reduction can reduce damage to 0 but not below 0.

\textbf{Greater Rage (Ex):} At 11th level, a barbarian's bonuses to Strength and Constitution during his rage each increase to +6, and his morale bonus on Will saves increases to +3. The penalty to AC remains at $-2$.

\textbf{Indomitable Will (Ex):} While in a rage, a barbarian of 14th level or higher gains a +4 bonus on Will saves to resist enchantment spells. This bonus stacks with all other modifiers, including the morale bonus on Will saves he also receives during his rage.

\textbf{Tireless Rage (Ex):} At 17th level and higher, a barbarian no longer becomes fatigued at the end of his rage.

\textbf{Mighty Rage (Ex):} At 20th level, a barbarian's bonuses to Strength and Constitution during his rage each increase to +8, and his morale bonus on Will saves increases to +4. The penalty to AC remains at $-2$.

\subsubsection{Ex-Barbarians}

A barbarian who becomes lawful loses the ability to rage and cannot gain more levels as a barbarian. He retains all the other benefits of the class (damage reduction, fast movement, wasteland trap sense, and uncanny dodge).

\subsection{Playing a Barbarian}

All cower and stand in awe at the fury you can tap, enhancing your strength and toughness. But what do these people know of the burnt wastes of Athas, the hellish jungles of the Forest Ridge? The cruel vicissitudes of growing up in the wastes of Athas were nothing but normal to you. When your family was lost in a tembo attack, or when your entire village was either murdered or forced into slavery, how could you not know they might not had to die? These and many other brutal experiences marked you, and you now stand apart from those born into the “comforts” of the city‐states.

\subsubsection{Religion}

Although most are profoundly superstitious, barbarians distrust the established elemental temples of the cities. Some worship the elements of fire or air or devote themselves to a famous figure. Most barbarians truly believe the sorcerer‐kings to be gods, because of their undeniable power, and a few actually worship a sorcerer‐king, usually the one that conquered their tribe. Such barbarians often escape menial slavery by joining an elite unit of barbarians in the service of an aggressive city‐state such as Urik, Draj or Gulg.

\subsubsection{Other Classes}

Barbarians are most comfortable in the company of gladiators, and of clerics of Air and Fire. Enthusiastic lovers of music and dance, barbarians admire bardic talent, and some barbarians also express fascination with bardic poisons, antidotes and alchemical concoctions. With some justification, barbarians do not trust wizardry. Even though many barbarians manifest a wild talent, they tend to be wary of psions and Tarandan psionicists. Psychic warriors, on the other hand, are creatures after the barbarian's own heart, loving battle for its own sake. Barbarians have no special attitudes toward fighters or rogues. Barbarians admire gladiators and will ask about their tattoos and exploits, but will quickly grow bored if the gladiator does not respond boastfully.

\subsubsection{Combat}

You know that half the battle occurs before the fight even begins. You prefer to choose your battleground when you can, stalking your opponent into terrain that best suits your abilities. Once battle is joined, you become a wild frenzy of motion, striking quickly and powerfully until all your opponents are crushed. While you lack the training of the fighter, or the cunning of the gladiator, you more than compensate them through sheer power and resilience.

\subsubsection{Advancement}

Becoming a barbarian let you further tap into your feral nature, letting you become one with the savage beast in your hear, and through your training, you have learned what you must do to unlock it.

To fully utilize your barbarian abilities, you will want to focus on feats that take advantage of your superior strength and speed, such as Power Attack and Whirlwind Attack.

\subsection{Starting Packages}

\subsubsection{The Survivor}

Human Barbarian

\textbf{Ability Scores:} Str 15, Dex 13, Con 14, Int 10, Wis 12, Cha 8.

\textbf{Skills:} Climb, Escape Artist, Listen, Survival.

\textbf{Languages:} Common.

\textbf{Feat:} Great Fortitude, Wastelander.

\textbf{Weapons:} Carrikal (1d8/x3)

Atlatl with 10 javelins (1d6/x3, 40 ft.).

\textbf{Armor:} Scale mail (+4 AC).

\textbf{Gear:} Standard adventurer's kit, 13 Cp.

\subsubsection{The Crusher}

Half‐giant Barbarian

\textbf{Ability Scores:} Str 23, Dex 10, Con 18, Int 6, Wis 9, Cha 4.

\textbf{Skills:} Climb, Intimidate, Jump.

\textbf{Languages:} Common.

\textbf{Feat:} Exotic Weapon Proficiency (swatter).

\textbf{Weapons:} Swatter (3d8/x4).

\textbf{Armor:} Leather (+2 AC).

\textbf{Gear:} Standard adventurer's kit, 0 Cp.

\subsubsection{The Hunter}

Thri‐kreen Barbarian

\textbf{Ability Scores:} Str 15, Dex 14, Con 12, Int 10, Wis 13, Cha 8.

\textbf{Skills:} Jump, Knowledge (nature), Search, Survival.

\textbf{Languages:} Kreen.

\textbf{Feat:} Track.

\textbf{Weapons:} Four chatkchas (1d6, 20 ft.).

\textbf{Armor:} Heavy wooden shield (+2 AC).

\textbf{Gear:} Standard adventurer's kit, 13 Cp.

\subsection{Barbarians on Athas}
\Quote{Don't make my friend angry. You won't like him when he's angry.}{Cabal, half‐elven bard}

In a savage world like Athas, is only natural that some of its inhabitants have turned into barbarians. They are fierce combatants without the army training fighters receive or wild rangers without the hunting skills.

\subsubsection{Daily Life}

A barbarian is a passionate adventurer. As a survivalist, he often sees his involvement in a particular enterprise as a validation of his superior strength and resilience. In his mind, his presence alone is enough to ensure the success of a quest, adventure, or ruin raid. Even simple tasks are additional opportunities to prove his own worth by accomplishing the task with might and alacrity. Barbarians are typically hardheaded and unforgiving because of the rigors of his previous life.

\subsubsection{Notables}

It is rare for a barbarian to live long enough, or close enough to civilization, in order to become famous, but a few examples exist. Korno, a Raamite gladiator, became the leader of a group of slaves, and Korno's furious rage known from the arenas has only increased after losing everything in the Raam invasion by Dregoth. The leader of Pillage, Chilod, is a tarek know for his outbursts of rage and cruelty, being one of the most feared chiefs of the Bandit States.

\subsubsection{Organizations}

Because of their independent and sometimes downright chaotic natures, many barbarians refuse to join organizations of any kind, though they usually maintain relationships with trading houses and raiding tribes. There is no specific organization that binds barbarians together.

\subsubsection{NPC Reactions}

Many lay people cannot tell a barbarian from a ranger or a fighter until his rage overcomes him and he starts screaming and bashing. Most authority figures and templars do not appreciate barbarians since they are prone to losing control and cannot be truly trusted. Thus, they generally treat barbarians with a great deal of caution.

\subsubsection{Barbarian Lore}

Characters with ranks in Knowledge (nature) can research barbarians to learn more about them. When a character makes a skill check, read or paraphrase the following, including the information from lower DCs.

\textbf{DC 10:} Barbarians are hot‐blooded combatants who fight with great brutality and savagery.

\textbf{DC 15:} Barbarians become stronger and more resilient when they lose control.

\textbf{DC 20:} Barbarians can stand up to punishment that no other individual can endure, and their reflexes are as quick as a rogue's.
\input{sections/4.2-bard.tex}
\vskip10em
\section{Cleric}
\Quote{Without destruction, there is nothing to build.}{Credo of the fire cleric}

In a world without gods, spiritualism on Athas has unlocked the secrets of the raw forces of which the very planet is comprised: earth, air, fire, and water. However, other forces exist which seek to supplant them and rise to ascendancy in their place. These forces have taken up battle against the elements of creation on the element's own ground in the form of entropic perversions of the elements themselves: magma, rain, silt and sun.

\SpellcasterTable{The Cleric}{.5cm}{
1 & +0 & +2 & +0 & +2 & Turn or rebuke undead & 3 & 1+1 &  &  &  &  &  &  &  &  \\
2 & +1 & +3 & +0 & +3 &  & 4 & 2+1 &  &  &  &  &  &  &  &  \\
3 & +2 & +3 & +1 & +3 &  & 4 & 2+1 & 1+1 &  &  &  &  &  &  &  \\
4 & +3 & +4 & +1 & +4 &  & 5 & 3+1 & 2+1 &  &  &  &  &  &  &  \\
5 & +3 & +4 & +1 & +4 &  & 5 & 3+1 & 2+1 & 1+1 &  &  &  &  &  &  \\
6 & +4 & +5 & +2 & +5 &  & 5 & 3+1 & 3+1 & 2+1 &  &  &  &  &  &  \\
7 & +5 & +5 & +2 & +5 &  & 6 & 4+1 & 3+1 & 2+1 & 1+1 &  &  &  &  &  \\
8 & +6/+1 & +6 & +2 & +6 &  & 6 & 4+1 & 3+1 & 3+1 & 2+1 &  &  &  &  &  \\
9 & +6/+1 & +6 & +3 & +6 &  & 6 & 4+1 & 4+1 & 3+1 & 2+1 & 1+1 &  &  &  &  \\
10 & +7/+2 & +7 & +3 & +7 &  & 6 & 4+1 & 4+1 & 3+1 & 3+1 & 2+1 &  &  &  &  \\
11 & +8/+3 & +7 & +3 & +7 &  & 6 & 5+1 & 4+1 & 4+1 & 3+1 & 2+1 & 1+1 &  &  &  \\
12 & +9/+4 & +8 & +4 & +8 &  & 6 & 5+1 & 4+1 & 4+1 & 3+1 & 3+1 & 2+1 &  &  &  \\
13 & +9/+4 & +8 & +4 & +8 &  & 6 & 5+1 & 5+1 & 4+1 & 4+1 & 3+1 & 2+1 & 1+1 &  &  \\
14 & +10/+5 & +9 & +4 & +9 &  & 6 & 5+1 & 5+1 & 4+1 & 4+1 & 3+1 & 3+1 & 2+1 &  &  \\
15 & +11/+6/+1 & +9 & +5 & +9 &  & 6 & 5+1 & 5+1 & 5+1 & 4+1 & 4+1 & 3+1 & 2+1 & 1+1 &  \\
16 & +12/+7/+2 & +10 & +5 & +10 &  & 6 & 5+1 & 5+1 & 5+1 & 4+1 & 4+1 & 3+1 & 3+1 & 2+1 &  \\
17 & +12/+7/+2 & +10 & +5 & +10 &  & 6 & 5+1 & 5+1 & 5+1 & 5+1 & 4+1 & 4+1 & 3+1 & 2+1 & 1+1 \\
18 & +13/+8/+3 & +11 & +6 & +11 &  & 6 & 5+1 & 5+1 & 5+1 & 5+1 & 4+1 & 4+1 & 3+1 & 3+1 & 2+1 \\
19 & +14/+9/+4 & +11 & +6 & +11 &  & 6 & 5+1 & 5+1 & 5+1 & 5+1 & 5+1 & 4+1 & 4+1 & 3+1 & 2+1 \\
20 & +15/+10/+5 & +12 & +6 & +12 &  & 6 & 5+1 & 5+1 & 5+1 & 5+1 & 5+1 & 4+1 & 4+1 & 3+1 & 3+1}


\subsection{Making a Cleric}

Clerics are the masters of elemental forces; they possess unique supernatural abilities to direct and harness elemental energy, and cast elemental spells. All things are comprised of the four elements in some degree, thus clerics can use their elemental powers to heal or harm others. Due to their affinities with the elements, clerics possess a number of supernatural elemental abilities. Though dimly understood, there exists a connection between elemental forces and the nature of undeath. Clerics can turn away, control, or even destroy undead creatures. Athas is a dangerous world; this practicality dictates that clerics must be able to defend themselves capably. Clerics are trained to use simple weapons and, in some cases, martial weapons; they are also taught to wear and use armor, since wearing armor does not interfere with elemental spells as it does arcane spells.

\textbf{Races:} All races include clerics in their societies, though each race possesses different perspectives regarding what a cleric's role involves. As masters of myth and the elemental mysteries, most clerics hold a place of reverence within their respective societies. However, more than a few races have varying affinities for one element over another. Dwarves almost always become earth clerics, a connection they've shared since before they were driven from their halls under the mountains. Dwarven determination and obsessive dedication matches perfectly with the enduring earth. Elves most often revere water, fire, or the winds; as nomads, they seldom feel a deep-seated affinity for the land. Thri-kreen are known to ally with all elements to the exclusion of fire. This seems to stem from a mistrust of flame, which is common in many kreen.

\textbf{Alignment:} Attaining the abilities of a true servant of the elements requires a deep understanding of the chosen kind of element of paraelement. An aspiring cleric must make a study of the element's typical personality and role; opens the door to the element's power. Thus, Athasians clerics align their morals to suit the traits of the element to which they dedicate themselves.

\subsection{Game Rule Information}

\textbf{Hit Die:} d8.

\subsubsection{Class Skills}
\skill{Concentration} (Con), \skill{Craft} (Int), \skill{Diplomacy} (Cha), \skill{Heal} (Wis), \skill{Knowledge} (arcana) (Int), \skill{Knowledge} (history) (Int), \skill{Knowledge} (religion) (Int), \skill{Knowledge} (the planes) (Int), \skill{Profession} (Wis), and \skill{Spellcraft} (Int).

\textbf{Skill Points per Level:} 2 + Int modifier ($\times4$ at 1st level).

\subsubsection{Class Features}
\textbf{Weapon and Armor Proficiency:} Clerics are proficient with light armor and all simple weapons.

\textbf{Aura (Ex):} A cleric has a particularly powerful aura corresponding to the her alignment (see the detect evil spell for details).

\textbf{Spells:} A cleric casts divine spells, which are drawn from the cleric spell list. However, his alignment may restrict him from casting certain spells opposed to his moral or ethical beliefs; see Chaotic, Evil, Good, and Lawful Spells, below. A cleric must choose and prepare his spells in advance (see below).

To prepare or cast a spell, a cleric must have a Wisdom score equal to at least 10 + the spell level. The Difficulty Class for a saving throw against a cleric's spell is 10 + the spell level + the cleric's Wisdom modifier.

Like other spellcasters, a cleric can cast only a certain number of spells of each spell level per day. His base daily spell allotment is given on \tabref{The Cleric}. In addition, he receives bonus spells per day if he has a high Wisdom score. A cleric also gets one domain spell of each spell level he can cast, starting at 1st level. When a cleric prepares a spell in a domain spell slot, it must come from one of his two domains (see Elements, Domains, and Domain Spells, below).

Clerics meditate or pray for their spells. Each cleric must choose a time at which he must spend 1 hour each day in quiet contemplation or supplication to regain his daily allotment of spells. Time spent resting has no effect on whether a cleric can prepare spells. A cleric may prepare and cast any spell on the cleric spell list, provided that he can cast spells of that level, but he must choose which spells to prepare during his daily meditation.

\BigTablePair{Athasian Elements}{b{1.2cm} b{1.3cm} l X} {
\tableheader Element & \tableheader Energy Type & \tableheader Domains & \tableheader Worshipers\\
Air & Sonic & Sun Flare, Furious Storm, Ill Wind, Rolling Thunder, Soaring Spirit & Aarakocra, elves\\
Earth & Acid & Decaying Touch, Earthen Embrace, Forged Stone, Ruinous Swarm, Mountain's Fury & Dwarves, muls\\
Fire & Fire & Burning Eyes, Sky Blitz, Mountain's Fury, Smoldering Spirit, Fiery Wrath & Dwarves, ssurrans\\
Magma & Fire & Broken Sands, Dead Heart, Ill Wind, Mountain's Fury & Ssurrans\\
Rain & Electricity & Cold Malice, Decaying Touch, Furious Storm, Refreshing Storm & Drajis\\
Silt & Acid & Broken Sands, Decaying Touch, Dead Heart, Soul Slayer & Giants, silt runners\\
Sun & Fire & Sun Flare, Light's Revelation, Desert Mirage, Fiery Wrath & Aarakocra\\
Water & Acid & Desert Mirage, Drowning Despair, Sky Blitz, Living Waters & Half-elves, lizardfolk
}

\textbf{Elements, Domains, and Domain Spells:} A cleric's element influences what magic he can perform, his values, and how others see him. A cleric chooses two domains from among those belonging to his element.

Each domain gives the cleric access to a domain spell at each spell level he can cast, from 1st on up, as well as a granted power. The cleric gets the granted powers of both the domains selected.

With access to two domain spells at a given spell level, a cleric prepares one or the other each day in his domain spell slot. If a domain spell is not on the cleric spell list, a cleric can prepare it only in his domain spell slot.

\textbf{Spontaneous Casting:} A good cleric can channel stored spell energy into healing spells that the cleric did not prepare ahead of time. The cleric can ``lose'' any prepared spell that is not a domain spell in order to cast any cure spell of the same spell level or lower (a cure spell is any spell with ``cure'' in its name).

An evil cleric, can't convert prepared spells to cure spells but can convert them to inflict spells (an inflict spell is one with ``inflict'' in its name).

A cleric who is neither good nor evil can convert spells to either cure spells or inflict spells (player's choice). Once the player makes this choice, it cannot be reversed. This choice also determines whether the cleric turns or commands undead.

\textbf{Chaotic, Evil, Good, and Lawful Spells:} A cleric can't cast spells of an alignment opposed to his own. Spells associated with particular alignments are indicated by the chaos, evil, good, and law descriptors in their spell descriptions.

\textbf{Turn or Rebuke Undead (Su):} Any cleric, regardless of alignment, has the power to affect undead creatures by channeling the power of his faith through his holy (or unholy) symbol.

A good cleric can turn or destroy undead creatures. An evil cleric instead rebukes or commands such creatures. A neutral cleric must choose whether his turning ability functions as that of a good cleric or an evil cleric. Once this choice is made, it cannot be reversed. This decision also determines whether the cleric can cast spontaneous cure or inflict spells.

A cleric's worshiped element or paraelement has no impact on your ability to turn or rebuke undead. However, all elements and paraelements consider the undead to be a violation of the natural order of things. While evil clerics are free to control undead, they are expected to eventually destroy them.

A cleric may attempt to turn undead a number of times per day equal to 3 + his Charisma modifier. A cleric with 5 or more ranks in Knowledge (religion) gets a +2 bonus on turning checks against undead.

\subsection{Playing a Cleric}
The clerics of Athas are like the rare snows that blanket the highest peaks of the Ringing Mountains. Though the cascading flakes all seem the same, the pattern of each is as different as the faces of men are from muls. Indeed, clerics are like snowflakes, each preaching about preservation and the elements, but no two of them do it for the same reason. This makes these environmental warriors an extremely diverse and interesting class to play. Some are merely power-hungry, some seek revenge, and some are honestly struggling to save their dying planet and reverse the ancient environmental disaster.

You are a servant of your element, your goal in life is to expand its presence in Athas, and find your element's foes and destroy them with your cleansing element.

You adventure out of a desire to preach the words of your element, prove your worth and to destroy infidels who worship opposed elements.

\subsubsection{Religion}
Unlike clerics found on other worlds, elemental clerics do not generally congregate at temples or churches, nor do they participate in a uniform, organized religion. Each cleric's calling to the raw energy of the elements is personal, individual. Some clerics believe that, upon their initiation, they enter pacts with powerful beings, elemental lords, who grant powers to those who contract with them. Others believe that the elements are neither malevolent nor benevolent, but a tool to be used, or a force to be harnessed. Regardless, all clerics desire the preservation of their patron element, though the reasons for this are many and varied.

Clerics are found everywhere on Athas. Most common clerics are wanderers, who preach the concept of preservation with the hope of restoring Athas to a greener state. Wanderers are generally well received by those that dwell in the desert, such as villagers and slave tribes. They cure the sick and heal the wounded, sometimes even aiding in defeating local threats. Other clerics act as wardens of small, hidden shrines, which they hope creates a clearer channel to the elemental plane of worship, and fortifies their powers and spells. Tribal and primitive societies include shamans, who see to the spiritual needs of their groups, offering advice to the leaders and providing supernatural protection and offense. Lastly, some clerics stay in the cities, where they most commonly work against the sorcerer-kings and their templars. There they quietly preach the message of preservation to the citizenry, and even sometimes work with the Veiled Alliance.

\subsubsection{Other Classes}
In an adventuring party, the cleric often fills the role of advisor and protector. Clerics often possess an unshakable distrust of wizards and their arcane spells. Most clerics are well aware of the danger that sorcery represents to the dying planet, and watch those who wield such power carefully. Generally speaking, the elemental clerics are all on friendly terms with each other, recognizing an ancient pact made by their ancestors to put aside their differences in the opposition of Athas' destruction. However, clerics whose elements are diametrically opposed often clash regarding the means used in furthering their goals, and at times this has led to bloodshed.

\subsubsection{Combat}
Athasian clerics make use of the same general combat tactics as those described in the Player's Handbook---that is, stay back from melee and use your spells to either destroy your enemies or enhance your allies' abilities.

Your tactics on the battlefield depend largely on your element and domains chosen. Air clerics are not very offensive, but when needed they usually employ sonic attacks from the heights. Earth clerics believe the best defense is a good offense, but they also employ the strongest of metal weapons. Fire clerics are feared and unpredictable, appearing to thrive only when everything around them is being devoured by the fiery appetites of their patrons. Water clerics are usually healers, but they can be known to be meticulous in the cruelty of their vengeance when someone wantonly wastes water.

Don't neglect your ability to heal yourself or your allies, but don't burn through your spells early in an attempt to do so; make the most efficient use of your spells in battle, saving the healing until combat is over or it becomes absolutely necessary.

\subsubsection{Advancement}
Your first steps towards becoming a cleric were witnessing your element in action. After learning what your element could do, and that they could grant such powers into you, you dedicated yourself into serving your element. Your elemental pact marked the beginning of your journey and unlocked the first of many new abilities other creatures can only dream about.

You have only just begun your quest to become worthy of your element, and a lifetime of striving still lies ahead of you. If you truly want to serve your element the best you can, consider taking the elementalist prestige class (page 93).

\subsection{Starting Packages}
\subsubsection{The Defender}

Dwarf Earth Cleric

\textbf{Ability Scores:} Str 13, Dex 8, Con 16, Int 12, Wis 15, Cha 8.

\textbf{Skills:} \skill{Concentration}, \skill{Knowledge} (religion).

\textbf{Languages:} Common, Dwarven, Terran.

\textbf{Feat:} \feat{Disciplined}.

\textbf{Weapons:} Maul (1d12)

Bolas (1d4, 10 ft.).

\textbf{Armor:} Scale mail (+6 AC).

\textbf{Gear:} Spell component pouch, standard adventurer's kit, 45 Cp.

\textbf{Class Features:} Channels positive energy; Earthen Embrace and Mountain's Fury domains.

\textbf{Spells Prepared:} 1st---\spell{magic stone}$^D$, \spell{protection from evil}, \spell{shield of faith}; 0---\spell{create element}, \spell{detect element}, \spell{resistance}.

D: Domain spell.

\subsubsection{The Destroyer}

Human Magma Cleric

\textbf{Ability Scores:} Str 14, Dex 8, Con 13, Int 10, Wis 15, Cha 12.

\textbf{Skills:} Concentration.

\textbf{Languages:} Common.

\textbf{Feat:} Combat Casting, Elemental Might.

\textbf{Weapons:} Heartpick (1d8/x4).

\textbf{Armor:} Scale mail (+4 AC), heavy wooden shield (+2 AC).

\textbf{Gear:} Spell component pouch, standard adventurer's kit, 59 Cp.

\textbf{Class Features:} Channels positive energy; Broken Sands and Mountain's Fury domains.

\textbf{Spells Prepared:} 1st---\spell{bless}, \spell{divine favor}, \spell{sand pit}$^{D}$; 0---\spell{create element}, \spell{resistance}, \spell{virtue}.

D: Domain spell.

\subsubsection{The Healer}

Pterran Water Cleric

\textbf{Ability Scores:} Str 14, Dex 10, Con 10, Int 8, Wis 17, Cha 15.

\textbf{Skills:} Concentration, Diplomacy, Heal.

\textbf{Languages:} Saurian.

\textbf{Feat:} Skill Focus (Heal).

\textbf{Weapons:} Longspear (1d8/x3)

Net (10 ft.).

\textbf{Armor:} Scale mail (+4 AC).

\textbf{Gear:} Spell component pouch, standard adventurer's kit, 50 Cp.

\textbf{Class Features:} Channels positive energy; Drowning Despair and Living Waters domains.

\textbf{Spells Prepared:} 1st---\spell{clear water}$^{D}$, \spell{protection from evil}, \spell{sanctuary}; 0---\spell{create element}, \spell{detect poison}, \spell{purify food and drink}.

D: Domain spell.

\subsection{Clerics on Athas}
\Quote{As for the elemental clerics, some say we are mad ― driven insane by the chaotic beings we serve. But others see the gleam of patience in our eyes, and know that one day the clerics of Athas will throw off the yoke of oppression and return the flowing rivers and the sprawling forests to our withered lands.}{Jurgan, Urikite earth cleric}

Like the Athasian deserts, the elemental powers are neither benevolent nor malevolent, caring only that their natural forms are preserved in the material world. This is the source of their power, and the impending ecological collapse in Athas has created an unusual and dynamic power struggle on the elemental planes. The clerics of Athas are nothing but the pawns of this titanic struggle.

\subsubsection{Daily Life}

A cleric typically begins his day by finding a suitable locale where he can commune with his element and pray for the spells he desires. He then spends the rest of the day engaged in whatever task seems most important for advancing his element's goals while trying to avoid too much trouble. When not adventuring, clerics often spend their time seeking out scraps of information about the elemental planes and other clerics. The pursuit of such knowledge is often quite dangerous and can result in the cleric undertaking additional adventures.

\subsubsection{Notables}

The pursuit of his element's goals garners notoriety for a cleric, but it also can bring about his death of force him into exile. The Wanderer, famous for compiling the history and geography of Athas, is said to be an earth cleric. The sun cleric Caelum (page 285) became famous for leading his Dwarven army in their metal armor against the sorcerer-kings and helping re-imprisoning Rajaat back in to the Hollow.

\subsubsection{Organizations}

A cleric usually finds a role in an adventuring party or other organization that allows his free time to explore his divine abilities freely. Since no organization specifically caters to Athasian clerics, many find themselves in drastically different circumstances from those of their comrades.

Within the ranks of elemental clerics, prestige and influence is measured by the depth of their devotion to their element. The most highly admired are those who have further accomplished their element's pact and those who most wield elemental power. When two or more clerics come into conflict, they usually defer to the one with a greater knowledge of their element, relying on wisdom and experiences to provide a reasonable solution.

The elemental clerics are much more tightly tied to their temples than paraelemental ones. Because the elements are losing the battle against the paraelements, they cannot afford to be without staunch allies.

\subsubsection{NPC Reactions}

The reactions clerics receive from communities are directly tied to how those cultures regard their specific element. A silt cleric is viewed in a much friendlier manner near to the Sea of Silt than near the Forest Ridge, for example.

As a general rule of thumb, an NPC's attitude is one step nearer helpful for elemental clerics and one step nearer hostile for paraelemental clerics.

\subsubsection{Cleric Lore}

Characters with ranks in Knowledge (religion) can research clerics to learn more about them. When a character makes a skill check, read or paraphrase the following, including the information from lower DCs.

\textbf{DC 10:} Clerics are divine spellcasters that serve the elemental powers.

\textbf{DC 15:} A cleric devotes himself to a particular kind of element gains power based on the element chosen. They can easily heal of harm those around him by channeling divine energy.

\textbf{DC 20:} Elemental clerics have forged a pact of sorts in order to fight the paraelement clerics and their quick expansion over Athas.

\input{sections/4.4-druid.tex}
\input{sections/4.5-fighter.tex}
\input{sections/4.6-gladiator.tex}
\section{Psion}
\Quote{Resist all you like. I have ways of making you think.}{Dechares, Dwarven inquisitor}

The psion learns the Way, a philosophy of mental discipline, to become master of his will, or innate mental power. Most aspiring psions seek out an instructor, a master of the Way. Most Athasian cities contain psionic academies where students receive instructions in exchange for money or loyal service.

\PsychicTable{The Psion}{
1 & +0 & +0 & +0 & +2 & Bonus feat, discipline & 2 & 3 & 1st \\
2 & +1 & +0 & +0 & +3 &  & 6 & 5 & 1st \\
3 & +1 & +1 & +1 & +3 &  & 11 & 7 & 2nd \\
4 & +2 & +1 & +1 & +4 &  & 17 & 8 & 2nd \\
5 & +2 & +1 & +1 & +4 & Bonus feat & 25 & 11 & 3rd \\
6 & +3 & +2 & +2 & +5 &  & 35 & 13 & 3rd \\
7 & +3 & +2 & +2 & +5 &  & 46 & 15 & 4th \\
8 & +4 & +2 & +2 & +6 &  & 58 & 17 & 4th \\
9 & +4 & +3 & +3 & +6 &  & 72 & 19 & 5th \\
10 & +5 & +3 & +3 & +7 & Bonus feat & 88 & 21 & 5th \\
11 & +5 & +3 & +3 & +7 &  & 106 & 22 & 6th \\
12 & +6/+1 & +4 & +4 & +8 &  & 126 & 24 & 6th \\
13 & +6/+1 & +4 & +4 & +8 &  & 147 & 25 & 7th \\
14 & +7/+2 & +4 & +4 & +9 &  & 170 & 27 & 7th \\
15 & +7/+2 & +5 & +5 & +9 & Bonus feat & 195 & 28 & 8th \\
16 & +8/+3 & +5 & +5 & +10 &  & 221 & 30 & 8th \\
17 & +8/+3 & +5 & +5 & +10 &  & 250 & 31 & 9th \\
18 & +9/+4 & +6 & +6 & +11 &  & 280 & 33 & 9th \\
19 & +9/+4 & +6 & +6 & +11 &  & 311 & 34 & 9th \\
20 & +10/+5 & +6 & +6 & +12 & Bonus feat & 343 & 36 & 9th}

\subsection{Making a Psion}

The psion learns the Way in order to shape his Will. The psion uses, through study called the Way, how to manifest the power inherent in his inner self. The psion is able to project this power, the Will, into creating all sorts of supernatural effects. The psion may know a limited number of ways to shape his will, but he enjoys great flexibility in how he uses his known powers.

\textbf{Races:} Nearly all living creatures have a latent psionic capacity, and psions are found among all sentient races of the Tablelands, and even among some creatures that are not ordinarily considered sentient.

\textbf{Alignment:} The search for refinement of the Way tends to draw many psions into a neutral view of the world, so most psions have one part of their alignment that is neutral. Good psions may spend their time in search of new powers, or help their village defend itself against predators, or maybe join the ranks of Merchant Houses. Evil psions may serve as agents in service of the sorcerer‐kings, or as more shady agents of Merchant Houses, or simply work as mercenaries and offer their specialized services to the highest bidder. Even though many psions tend to have a neutral view of the world, they can be of any alignment.

\subsection{Game Rule Information}

\textbf{Hit Die:} d4.

\subsection{Class Skills}

\textbf{Class Skills:} Concentration (Con), Craft (Int), Knowledge (all skills, taken individually) (Int), Profession (Wis), and Psicraft (Int). In addition, a psion gains access to additional class skills based on his discipline:

\textbf{Seer (Clairsentience):} Gather Information (Cha), Listen (Wis), and Spot (Wis).

\textbf{Shaper (Metacreativity):} Bluff (Cha), Disguise (Cha), and Use Psionic Device (Cha).

\textbf{Kineticist (Psychokinesis):} Autohypnosis (Wis), Disable Device (Dex), and Intimidate (Cha).

\textbf{Egoist (Psychometabolism):} Autohypnosis (Wis), Balance (Dex) and Heal (Wis).

\textbf{Nomad (Psychoportation):} Climb (Str), Jump (Str), Ride (Dex), Survival (Wis), and Swim (Str).

\textbf{Telepath (Telepathy):} Bluff (Cha), Diplomacy (Cha), Gather Information (Cha), and Sense Motive (Wis).

\textbf{Skill Points per Level:} 2 + Int modifier ($\times 4$ at 1st level).

\subsection{Class Features}

\textbf{Weapon and Armor Proficiency:} Psions are proficient with the club, dagger, heavy crossbow, light crossbow, quarterstaff, and shortspear. They are not proficient with any type of armor or shield. Armor does not, however, interfere with the manifestation of powers.

\textbf{Power Points per Day:} A psion’s ability to manifest powers is limited by the power points he has available. His base daily allotment of power points is given on Table: The Psion. In addition, he receives bonus power points per day if he has a high Intelligence score (see Table: Ability Modifiers and Bonus Power Points). His race may also provide bonus power points per  day, as may certain feats and items.

\textbf{Discipline:} Every psion must decide at 1st level which psionic discipline he will specialize in. Choosing a discipline provides a psion with access to the class skills associated with that discipline (see above), as well as the powers restricted to that discipline. However, choosing a discipline also means that the psion cannot learn powers that are restricted to other disciplines. He can’t even use such powers by employing psionic items.

\textbf{Powers Known:} A psion begins play knowing three psion powers of your choice. Each time he achieves a new level, he unlocks the knowledge of new powers.

Choose the powers known from the psion power list, or from the list of powers of your chosen discipline. You cannot choose powers from restricted discipline lists other than your own discipline list. You can choose powers from disciplines other than your own if they are not on a restricted discipline list. (Exception: The feats Expanded Knowledge and Epic Expanded Knowledge do allow a psion to learn powers from the lists of other disciplines or even other classes.) A psion can manifest any power that has a power point cost equal to or lower than his manifester level.

The number of times a psion can manifest powers in a day is limited only by his daily power points.

A psion simply knows his powers; they are ingrained in his mind. He does not need to prepare them (in the way that some spellcasters prepare their spells), though he must get a good night’s sleep each day to regain all his spent power points.

The Difficulty Class for saving throws against psion powers is 10 + the power’s level + the psion’s Intelligence modifier. Maximum Power Level Known: A psion begins play with the ability to learn 1st-level powers. As he attains higher levels, a psion may gain the ability to master more complex powers.

To learn or manifest a power, a psion must have an Intelligence score of at least 10 + the power’s level.

\textbf{Bonus Feats:} A psion gains a bonus feat at 1st level, 5th level, 10th level, 15th level, and 20th level. This feat must be a psionic feat, a metapsionic feat, or a psionic item creation feat.

These bonus feats are in addition to the feats that a character of any class gains every three levels. A psion is not limited to psionic feats, metapsionic feats, and psionic item creation feats when choosing these other feats.

Psionic Disciplines

A discipline is one of six groupings of powers, each defined by a common theme. The six disciplines are clairsentience, metacreativity, psychokinesis, psychometabolism, psychoportation, and telepathy.

\textbf{Clairsentience:} A psion who chooses clairsentience is known as a seer. Seers can learn precognitive powers to aid their comrades in combat, as well as powers that permit them to gather information in many different ways.

\textbf{Metacreativity:} A psion specializing in metacreativity is known as a shaper. This discipline includes powers that draw ectoplasm or matter from the Astral Plane, creating semisolid and solid items such as armor, weapons, or animated constructs to do battle at the shaper’s command.

\textbf{Psychokinesis:} Psions who specialize in psychokinesis are known as kineticists. They are the masters of powers that manipulate and transform matter and energy. Kineticists can attack with devastating blasts of energy.

\textbf{Psychometabolism:} A psion who specializes in psychometabolism is known as an egoist. This discipline consists of powers that alter the psion’s psychobiology, or that of creatures near him. An egoist can both heal and transform himself into a fearsome fighter.

\textbf{Psychoportation:} A psion who relies on psychoportation powers is known as a nomad. Nomads can wield powers that propel or displace objects in space or time.

\textbf{Telepathy:} A psion who chooses the discipline of telepathy is known as a telepath. He is the master of powers that allow mental contact and control of other sentient creatures. A telepath can deceive or destroy the minds of his enemies with ease.

\subsection{Playing a Psion}

When you first learned to use psionics, you were taught to create a nexus ― a point in the center of your being where physical, mental, and spiritual energy can be harnessed. It is the union of these powers that allows you to perform the remarkable feats you’re capable of.

As a psion, your choice of discipline is all‐important to you. Seers are not very powerful, if one defines power as the ability to cause immediate harm to one’s foes, but they are the most capable information gatherers of Athas. Shapers are tinkerers, creating toys and monsters out of thin air, just to dismiss them and build another. Kineticists are battlefield psionicists who are actively sought out as military auxiliaries, and is almost as good as a wizard for creating mayhem in a fight. Egoists have a wide range of useful powers: they can fight as well as a fighter, become stealthier than a thief, heal like a cleric, or change shape like a wizard. Nomads possess an array of valuable powers that can bypass almost any obstacle and confound any enemies, working with the very fabric of space, time, and reality itself to achieve his goals. Telepaths are considered by some to the most powerful psions, and most Athasians are terrified of a telepath’s ability to manipulate their very thoughts.

\subsection{Religion}

Psions use the Way to manifest their inner powers; through long hours of meditation and extremes of the senses, they seek knowledge inward. Their power comes from inside them, so only psions from the most animistic cultures look to outside beings or religions for spiritual fulfillment.

\subsection{Other Classes}

Psions tend to be drawn to those like themselves. Lower‐level psions tend to towards a nearly worshipful attitude towards higher level psions, curious about their mysterious training and knowledge.

Higher‐level psions tend to either stay to themselves, or to try to befriend almost everyone, pressing for party leadership. Most psions tolerate priests and druids (although some psions make needling remarks about “foolish superstition”), but most psions are uneasy with wizards. Psions view wilders much in the same way that a fighter views a barbarian―untrained, erratic, and as much a danger to his companions as to his enemies.

\subsection{Combat}

You usually disdain combat and other primitive displays of force, but when needed, you use your impressive array of psionic powers for both attack and defense against your enemies, just as any other psionic character would.

\subsection{Advancement}

Most psions were strongly inclined towards a specific discipline before their ever realized they had any psionic talent. Once you have undergone your initial training, you can continue your studies on your own, much the way a wizard learn new spells.

As you attain more levels in the psion class, the most important choice you face is which powers to learn. A psion has access to much fewer spells than a wizard, so he has to chose carefully in order to find a good mix of offensive, defensive, and utility powers.

\subsection{Starting Packages}
\subsubsection{The Blaster}

Aarakocra Psion (Kineticist)

\textbf{Ability Scores:} Str 8, Dex 18, Con 13, Int 15, Wis 12, Cha 6.

\textbf{Skills:} Concentration, Intimidate, Knowledge (psionics), Psicraft.

\textbf{Languages:} Auran, Common.

\textbf{Feat:} Overchannel.

\textbf{Weapons:} Shortspear (1d6, 20 ft.)

Light crossbow with 20 bolts (1d6/19–20, 80 ft.).

\textbf{Armor:} None.

\textbf{Other Gear:} Standard adventurer’s kit, 62 Cp.

The Mindbender

Human Psion (Telepath)

\textbf{Ability Scores:} Str 8, Dex 10, Con 12, Int 15, Wis 13, Cha 14.

\textbf{Skills:} Bluff, Concentration, Gather Information, Knowledge (local), Sense Motive.

\textbf{Languages:} Common.

\textbf{Feat:} Inquisitor, Psionic Endowment.

\textbf{Weapons:} Club (1d4)

Light crossbow with 20 bolts (1d6/19–20, 80 ft.).

\textbf{Armor:} None.

\textbf{Other Gear:} Standard adventurer’s kit, 63 Cp.

The Teleporter

Elf Psion (Nomad)

\textbf{Ability Scores:} Str 10, Dex 16, Con 10, Int 15, Wis 13, Cha 8.

\textbf{Skills:} Concentration, Jump, Psicraft, Survival.

\textbf{Languages:} Elven, Common.

\textbf{Feat:} Speed of Thought.

\textbf{Weapons:} Quarterstaff (1d6)

Dagger (1d4/19–20, 10 ft.)

Shortbow with 20 arrows (1d6/x3. 60 ft.).

\textbf{Armor:} None.

\textbf{Other Gear:} Standard adventurer’s kit, 64 Cp.

\subsection{Psions on Athas}
\Quote{Once, I encountered a shattered tribe of elves wandering aimlessly through the desert. Lost and unprovisioned, they clearly had no hope of survival beyond the next few days. I later learned that they had made the mistake of disturbing a psionic master’s trance as they attempted to rob his home.}{The Wanderer’s Journal}

Nearly every level of Athasian society is permeated with psionics. Even the humblest slave may possess an unusual talent or ability, while the most powerful enchantments of the sorcerer‐monarchs include psionic elements. Mental powers are used on an everyday basis in Athasian culture.

Telepaths allow instantaneous communication across hundreds of miles. Draft animals and slaves are kept under control by psionic overseers. Prophets use their visionary powers to forecast the fortunes of kings and peasants, find missing objects, and solve crimes. Kineticists and egoists use their potent abilities in all manner of enterprises, both legitimate and otherwise.

\subsubsection{Daily Life}

The study of the Way is very similar to the study of magic. Just as wizards strive to master more advanced and difficult spells, psionicists must constantly seek to unlock new and more powerful abilities. Unlike wizardry, there is no single formula that will reproduce an effect of the Way that will work the same for each individual. Students must independently develop the command of their powers.

High‐level psions tend to become contemplative masters, so they can make good patrons for lower‐level PCs. Such psions often hire adventurers to gather rare psionic items for study or to recover lost knowledge of the ancient ages in their stead.

\subsubsection{Notables}

The human psion known as Pharistes brought chaos over the Tyr Region when he activated a powerful artifact that dampened all psionic power in the region and drove all thri‐kreen mad because he thought the abuse of psionics was the cause of all the evil under the dark sun. Agis of Asticles was an accomplished telepath and politician, who fought to bring freedom to the city‐stated of Tyr and helped to remove Athas from the menace of the Dragon of Tyr.

\subsubsection{Organizations}

Psions don’t organize together, but they often join other organizations, specially psionic academies and monasteries. Psions who dedicate themselves into extensive studies in such organizations in order to master the Way often become psiologists (page 104).

\subsubsection{NPC Reactions}

The common people usually react to a psion exactly as they would to any other psionicists in their community. Because trained psionicists are scarce and their skills are vital, they are highly valued by many elements of the Athasian society. Unlike wizards, psionicists are free of the taint of magic and need not disguise their calling. They owe no loyalty to the sorcerer‐kings, unlike the templars. Even clerics and druids have elemental powers and guarded lands that they must place before all other considerations. Psionicists are free of these patrons and responsibilities and may employ their powers as they see fit.

\subsubsection{Psion Lore}

Characters with ranks in Knowledge (psionics) can research psions to learn more about them. When a character makes a skill check, read or paraphrase the following, including the information from lower DCs.

\textbf{DC 10:} Psions are manifesters who use the forces of their own minds to affect their environment.

\textbf{DC 15:} Psionic powers do not draw upon magical energy that surrounds all things. Rather they are derived from within when the psionicist has his entire essence in coordination; his mind, body, and soul in perfect harmony.

\textbf{DC 20:} Psions choose one of the six psionic disciplines in which to focus their efforts.

\section{Psychic Warrior}
\Quote{The body is not bound to the forms and function you were born with. To master the art of delivering death, you must break your given mold.}{Tharlkar, psychic sense}

The term “psychic warrior” is a loose translation of the Thri‐kreen word “chakak,” which is better translated as “mind warrior.” In the Tablelands, non‐kreen psychic warriors have long been known as “mercenary psionicists.”

\PsychicTable{The Psychic Warrior}{
1 & +0 & +2 & +0 & +0 & Bonus feat & 0¹ & 1 & 1st \\
 2 & +1 & +3 & +0 & +0 & Bonus feat & 1 & 2 & 1st \\
 3 & +1 & +3 & +1 & +1 &  & 3 & 3 & 1st \\
 4 & +2 & +4 & +1 & +1 &  & 5 & 4 & 2nd \\
 5 & +2 & +4 & +1 & +1 & Bonus feat & 7 & 5 & 2nd \\
 6 & +3 & +5 & +2 & +2 &  & 11 & 6 & 2nd \\
 7 & +3 & +5 & +2 & +2 &  & 15 & 7 & 3rd \\
 8 & +4 & +6 & +2 & +2 & Bonus feat & 19 & 8 & 3rd \\
 9 & +4 & +6 & +3 & +3 &  & 23 & 9 & 3rd \\
 10 & +5 & +7 & +3 & +3 &  & 27 & 10 & 4th \\
 11 & +5 & +7 & +3 & +3 & Bonus feat & 35 & 11 & 4th \\
 12 & +6/+1 & +8 & +4 & +4 &  & 43 & 12 & 4th \\
 13 & +6/+1 & +8 & +4 & +4 &  & 51 & 13 & 5th \\
 14 & +7/+2 & +9 & +4 & +4 & Bonus feat & 59 & 14 & 5th \\
 15 & +7/+2 & +9 & +5 & +5 &  & 67 & 15 & 5th \\
 16 & +8/+3 & +10 & +5 & +5 &  & 79 & 16 & 6th \\
 17 & +8/+3 & +10 & +5 & +5 & Bonus feat & 91 & 17 & 6th \\
 18 & +9/+4 & +11 & +6 & +6 &  & 103 & 18 & 6th \\
 19 & +9/+4 & +11 & +6 & +6 &  & 115 & 19 & 6th \\
 20 & +10/+5 & +12 & +6 & +6 & Bonus feat & 127 & 20 & 6th
}

\subsection{Making a Psychic Warrior}
Despite his spectacular combat powers, a psychic warrior is not a typical front‐line combatant. Although a fighter, barbarian, or gladiator might swing a sword more accurately, or with greater force, a psychic warrior depends on his repertoire of power and feats. A psychic warrior is the psionic equivalent of an eldritch knight or a warmage from other settings. A psychic warrior’s role in the party isn’t easily defined, but his combination of physical might, the Way, and martial arts is useful in almost any encounter.

\textbf{Races:} Practicing psionics as part of hunting or combat comes as naturally to a Thri‐kreen, as running comes to an elf. The Thri‐kreen propensity to become “chakak” is rooted in the kreen ancestral memory. Becoming “chakak” is an almost unavoidable rite of kreen adulthood. Even kreen who focus their attentions in another class, such as the druid, tend to take at least one level as a psychic warrior. Nearly all pack‐leaders and clutchleaders are accomplished chakak. Because of the clutch‐mind, kreen chakak are far more cooperative, and infinitely less competitive with each other than the psychic warriors of other races.

Muls particularly excel as psychic warriors, as do humans, elves, and dwarves, to a lesser extent. Aarakocra and pterran psychic warriors are rare in those racial cultures, but individuals who take up the psychic warrior class tend to thrive. Halflings and jozhal psychic warriors are virtually unheard of.

\textbf{Alignment:} Psychic warriors tend towards neutrality with regards to good and evil, but they must be either lawful or chaotic. Chaotic psychic warriors, known commonly as “mercenary psionicists,” often work asattack thugs or assassins, though like bards, mercenary psionicists are notorious for switching allegiances according to the highest purse. Lawful psychic warriors, or “mindguards,” are the most sought‐after personal guards for nobles and merchant lords. Like the elite rogue servants of the nobles, mindguards serve loyally in exchange for lavish compensation. Any psychic warrior who ceases to be either lawful or chaotic, can no longer progress as a psychic warrior, although she keeps her current psychic warrior levels and abilities.

\subsection{Game Rule Information}

\textbf{Hit Die:} d8.

\subsection{Class Skills}

\textbf{Class Skills:} Autohypnosis (Wis), Climb (Str), Concentration (Con), Craft (Int), Intimidate (Cha), Jump (Str), Knowledge (psionics) (Int), Profession (Wis), Ride (Dex), and Search (Int).

\textbf{Skill Points per Level:} 2 + Int modifier ($\times 4$ at 1st level).

\subsection{Class Features}

\textbf{Weapon and Armor Proficiency:} Psychic warriors are proficient with all simple and martial weapons, with all types of armor (heavy, medium, and light), and with shields (except tower shields).

\textbf{Power Points/Day:} A psychic warrior’s ability to manifest powers is limited by the power points he has available. His base daily allotment of power points is given on Table: The Psychic Warrior. In addition, he receives bonus power points per day if he has a high Wisdom score (see Table: Ability Modifiers and Bonus Power Points). His race may also provide bonus power points per day, as may certain feats and items. A 1st-level psychic warrior gains no power points for his class level, but he gains bonus power points (if he is entitled to any), and can manifest the single power he knows with those power points.

\textbf{Powers Known:} A psychic warrior begins play knowing one psychic warrior power of your choice. Each time he achieves a new level, he unlocks the knowledge of a new power.

\textbf{Choose the powers known from the psychic warrior power list. (Exception:} The feats Expanded Knowledge and Epic Expanded Knowledge do allow a psychic warrior to learn powers from the lists of other classes.) A psychic warrior can manifest any power that has a power point cost equal to or lower than his manifester level.

The total number of powers a psychic warrior can manifest in a day is limited only by his daily power points.

A psychic warrior simply knows his powers; they are ingrained in his mind. He does not need to prepare them (in the way that some spellcasters prepare their spells), though he must get a good night’s sleep each day to regain all his spent power points.

The Difficulty Class for saving throws against psychic warrior powers is 10 + the power’s level + the psychic warrior’s Wisdom modifier.

\textbf{Maximum Power Level Known:} A psychic warrior begins play with the ability to learn 1st-level powers. As he attains higher levels, he may gain the ability to master more complex powers.

To learn or manifest a power, a psychic warrior must have a Wisdom score of at least 10 + the power’s level.

\textbf{Bonus Feats:} At 1st level, a psychic warrior gets a bonus combat-oriented feat in addition to the feat that any 1st level character gets and the bonus feat granted to a human character. The psychic warrior gains an additional bonus feat at 2nd level and every three levels thereafter (5th, 8th, 11th, 14th, 17th, and 20th). These bonus feats must be drawn from the feats noted as fighter bonus feats or psionic feats. The psychic warrior must still meet all prerequisites for the bonus feat, including ability score and base attack bonus minimums as well as class requirements. A psychic warrior cannot choose feats that specifically require levels in the fighter class unless he is a multiclass character with the requisite levels in the fighter class.

These bonus feats are in addition to the feats that a character of any class gains every three levels. A psychic warrior is not limited to fighter bonus feats and psionic feats when choosing these other feats.

\subsection{Playing a Psychic Warrior}

When you mold your body and mind with the same rigor as a dwarf tempers his steel, no feat or combat prowess is beyond you. Through it all, you seek to understand the secret knowledge of combat, and how to take your nexus ― a point in the center of your being where physical, mental, and spiritual energy can be harnessed ― to the next level. You know the exact extent of your abilities and how hard it was to achieve them, so you are prone to showing it off flamboyantly, and claim to fear nothing.

Psychic warriors adventure for a plethora of reasons. Neither the religious fervor of an elemental cleric nor the glory of the fighter causes you to travel the Tablelands. More than faith, more than glory, you seek martial perfection. Whether you find that perfection in the cannibal‐filled jungles of the Forest Ridge, in the choking silt of the Silt Sea, or in the den of the deadly braxat, you are driven to learn it and master it.

\subsection{Religion}

Religion might be entirely delusional to you, or you might find comfort in the elemental (or paraelemental) faiths, or even in the sorcerer‐monarch of your city‐state. If you are among the minority of psychic warriors who revere an element, you probably worship one associated with physical strength, such as Earth or Magma, or wisdom, such as Air or Sun.

\subsection{Other Classes}

Psychic warriors get along best with rogues, and to a lesser extent, fighters and bards. Generally, allies who show admiration for the psychic warriors’ talents tend to get along well with the psychic warrior. Gladiators tend to get suspicious and envious of the psychic warrior’s shows of unnatural and spectacular force, and many psychic warriors take a perverse pleasure in playing against the gladiator’s jealousy, showing up the gladiator with spectacular stunts. Psychic warriors pretend to be indifferent to wizards, and to a lesser extent, psions, but many secretly envy the spectacle of a fireball.

\subsection{Combat}

You use your sword skills to defeat your foes as well as the limited access to manifest melee‐oriented psionic powers. You have access to an amazing array of powerful combat feats. You have almost exclusive access to feats such as Deep Impact, Focused Sunder, and Wounding Attack, and you would do well to learn at least some of them. You have a limited selection of powers, so choose them carefully so you have a good mix of offensive, defensive and utility powers at your disposal.

\subsection{Advancement}

Your training began when you fought your way into an apprenticeship with a mentor―either a retired psychic warrior or an instructor in one of the many psionic academies dotting the Tablelands. You knew that finding that psychic warrior apprenticeship would not be that easy―that in fact, it would be an ordeal designed to test your body and mind to its fullest.

As a psychic warrior, your selection of psionic powers is paramount to your success. You might choose to focus on a specific psionic discipline, such as psychometabolism or psychokinesis, but learning a few powers from other disciplines is almost always advisable. True success in combat requires being ready for everything.

\subsection{Starting Packages}
\subsubsection{The Defender}

Mul Psychic Warrior

\textbf{Ability Scores:} Str 18, Dex 12, Con 15, Int 10, Wis 15, Cha 6.

\textbf{Skills:} Autohypnosis, Concentration, Intimidate.

\textbf{Languages:} Common.

\textbf{Feat:} Combat Manifestation.

\textbf{Weapons:} Great macahuitl (2d6/19–20)

Five javelins (1d6, 30 ft.).

\textbf{Armor:} Scale mail (+4 AC).

\textbf{Other Gear:} Standard adventurer’s kit, 45 Cp.

\subsubsection{The Destroyer}

Thri‐kreen Psychic Warrior

\textbf{Ability Scores:} Str 17, Dex 17, Con 12, Int 8, Wis 16, Cha 4.

\textbf{Skills:} Concentration, Intimidate, Jump.

\textbf{Languages:} Kreen.

\textbf{Feat:} Multiweapon Fighting.

\textbf{Weapons:} Gythka (1d8/1d8)

Four chatkchas (1d6, 20 ft.).

\textbf{Armor:} Leather (+2 AC).

\textbf{Other Gear:} Standard adventurer’s kit.

\subsubsection{The Skirmisher}

Human Psychic Warrior

\textbf{Ability Scores:} Str 14, Dex 13, Con 12, Int 10, Wis 15, Cha 8.

\textbf{Skills:} Concentration, Intimidate, Jump, Psicraft, Spot (cc).

\textbf{Languages:} Common.

\textbf{Feat:} Dodge, Weapon Focus (glaive).

\textbf{Weapons:} Gouge (1d10/x3)

Five javelins (1d6, 30 ft.).

\textbf{Armor:} Studded leather (+3 AC).

\textbf{Other Gear:} Standard adventurer’s kit, 100 Cp.

\subsection{Psychic Warriors on Athas}
\Quote{‘Your studies have gone well, Turek,’ he said quietly. ‘You have learned the basics of psychic defense. It is time to practice your lessons.’

Turek nodded, his palms wet with sweat. He had known this was coming; he was one of the older students and it was time to begin his final studies before leaving the academy.

His master watched him without expression. Suddenly Turek found his attention ripped away from the patio and the master’s physical form, being drawn inward. In his mind’s eye a glowing sword appeared, poised to strike. ‘I am the Sword’, his master whispered. I pierce barriers and rend armor.’ Turek swallowed nervously and summoned his defense. ‘I am the Void, he thought over and over again. I cannot be found, I cannot be harmed.’

The Sword lunged forward, driving through the heart of the nothingness that cloaked Turek’s presence…}{}

No place on Athas is safe from psionics. Armies and fortresses mean nothing to a master of the Way. To answer the threat of psionic attack, nobles and merchants retain the services of mercenary psionicists to guard against other users of the Way.

With a potential to advance in a number of different directions ― offensive, defensive, support, and quick strike ― psychic warriors make excellent additions to adventuring parties.

\subsubsection{Daily Life}

A psychic warrior spends the majority of his time perfecting his mind and body. The mental and spiritual demands of the Way require constant attention, so he can spare little time for carousing.

A psychic warrior with an apprentice spends much of his time training his student. A psychic warrior without one might or might not spend time seeking out one, according to his whims.

\subsubsection{Notables}

Hurgen Vurst, the half‐giant garrison chief for Fort Harbeth is considered to be one of the most deadly specimens of his race, combining massive strength and a cleverness rarely found on half‐giants. Chukaka the thri‐kreen, was one of the first to be coin the term Kiltektet (the‐learning‐pack‐who‐enlightens), was a psychic warrior. Known as much for her wisdom, her teachings, as for her chatkchas, she is regarded by many the prototypical psychic warrior ― serene, poised, and deadly.

\subsubsection{Organizations}

There is no specific organization that caters to psychic warriors. The Exalted Path (for males) and Serene Bliss (for females) orders in the city‐state of Nibenay keep the city’s ancient monastic tradition and they usually have several psychic warriors in their milieu. Villichi communities, female humans born with amazing psionic abilities, lie hidden in the deserts, harboring powerful psychic warriors.

\subsubsection{NPC Reactions}

As with fighters, individuals react to psychic warriors based on their previous interactions with other members of the class.

Gladiators have mixed feelings towards psychic warriors, they abilities can be of great value in the arena, but sometimes they feel a bit jealous of those abilities themselves, and they do not like other show offs competing for attention during gladiatorial matches. The only characters that psychic warriors as a rule will have an extremely hard time getting along with are other psychic warriors. Any party unfortunate enough to include more than one psychic warrior will be wrought with petty bickering, snide remarks, and endless competitions of spectacular force.

Merchants and nobles, on the other hand, greatly appreciate psychic warriors. They can always find ready employment as an elite mercenary, in the permanent guard of a noble family, or a merchant house sentry cadre.

\subsubsection{Psychic Warrior Lore}

Characters with ranks in Knowledge (psionics) can research psychic warriors to learn more about them. When a character makes a skill check, read or paraphrase the following, including the information from lower DCs.

\textbf{DC 10:} A psychic warrior is a psionic sword‐swinger who thinks he knows more about swordplay than anyone else.

\textbf{DC 15:} Like psions and wilders, psychic warrior walk the Unseen Way. Unlike them, psychic warriors train their bodies with the same rigor that they train their minds.

\textbf{DC 20:} Psychic warriors are strong, calm, and lethal. They gain the most psychic might of all those who study the Way.

\section{Ranger}
\Quote{What you call monsters and beasts are simply other beings trying to survive in the wastelands. Some of them are just as desperate, lost, and confused as you are.}{Sudatu, elven scout}

The wastes of Athas are home to fierce and cunning creatures, from the bloodthirsty tembo to the malicious gaj. Because of that, Athasians have long learned how to adapt and survive even in the most inhospitable and savage environments.

One of the most cunning and powerful creatures of the wastes is the ranger, a skilled hunter and stalker. He knows his lands as if they were his home (as indeed they are); he knows his prey in deadly detail.

\HalfSpellcasterTable{The Ranger}{1cm}{
1 & +1 & +2 & +2 & +0 & 1st favored enemy, Track, wild empathy &&&&\\
2 & +2 & +3 & +3 & +0 & Combat style &&&&\\
3 & +3 & +3 & +3 & +1 & Endurance &&&&\\
4 & +4 & +4 & +4 & +1 & Animal companion & 0 &&&\\
5 & +5 & +4 & +4 & +1 & 2nd favored enemy & 0 &&&\\
6 & +6/+1 & +5 & +5 & +2 & Improved combat style & 1 &&&\\
7 & +7/+2 & +5 & +5 & +2 & Woodland stride & 1 &&&\\
8 & +8/+3 & +6 & +6 & +2 & Swift tracker & 1 & 0 &&\\
9 & +9/+4 & +6 & +6 & +3 & Evasion & 1 & 0 &&\\
10 & +10/+5 & +7 & +7 & +3 & 3rd favored enemy & 1 & 1 &&\\
11 & +11/+6/+1 & +7 & +7 & +3 & Combat style mastery & 1 & 1 & 0 &\\
12 & +12/+7/+2 & +8 & +8 & +4 & & 1 & 1 & 1 &\\
13 & +13/+8/+3 & +8 & +8 & +4 & Camouflage & 1 & 1 & 1 &\\
14 & +14/+9/+4 & +9 & +9 & +4 & & 2 & 1 & 1 & 0 \\
15 & +15/+10/+5 & +9 & +9 & +5 & 4th favored enemy & 2 & 1 & 1 & 1 \\
16 & +16/+11/+6/+1 & +10 & +10 & +5 & & 2 & 2 & 1 & 1 \\
17 & +17/+12/+7/+2 & +10 & +10 & +5 & Hide in plain sight & 2 & 2 & 2 & 1 \\
18 & +18/+13/+8/+3 & +11 & +11 & +6 & & 3 & 2 & 2 & 1 \\
19 & +19/+14/+9/+4 & +11 & +11 & +6 & & 3 & 3 & 3 & 2 \\
20 & +20/+15/+10/+5 & +12 & +12 & +6 & 5th favored enemy & 3 & 3 & 3 & 3}

\subsection{Making a Ranger}

Rangers are capable in combat, although less so in open melee than the fighter, gladiator, or barbarian. His skills allow him to survive in the wilderness, to find his prey and to avoid detection. The ranger has the ability to gain special knowledge of certain types of creatures or lands. Knowledge of his enemies makes him more capable of finding and defeating those foes. Knowledge of terrain types or of specific favored lands makes it easier for him to live off the land, and makes it easier for him to take advantage of less knowledgeable foes. Rangers eventually learn to use the lesser spirits that inhabit Athas in order to produce spell-like effects. These lesser spirits inhabit small features of the land – rocks, trees, cacti and the like.

These spirits are relatively powerless, and cannot manifest themselves. Their awareness is low, and their instincts are of the most primitive sort. The relationship between these lesser spirits and the creatures known as Spirits of the Land is unknown.

\textbf{Races:} As the race that carries the most fear and hatred of other races, and as the people with the richest land to protect, Halflings become rangers more commonly than any other race except for half-elves. Halflings are at home in their terrain (typically Forest Ridge or the Jagged Cliffs) and the ranger class teaches them the grace to move without detection, often to deadly effect. Their practice of cannibalism to emphasize their superiority over other sentient beings puts the ranger's tracking abilities to deadly use. Halfling rangers tend to take favored lands primarily, followed by favored enemy benefits. In the Forest Ridge, halfling rangers tend to pick pterrans and other neighboring races as favored enemies; rangers of the Jagged Cliffs tend to focus on bvanen, and kreen.

Elves frequently become rangers, serving as scouts and hunters for their tribes, but elves are not as naturally drawn to the wilderness as they are to magic. Half-elves are the race most compellingly drawn to the ranger class, since their isolation and natural gift with animals gives them a head start above rangers of other races. Half-Elven rangers sometimes seek to impress their Elven cousins with their desert skills, and when they are rejected, the wilderness often becomes the half-elf's only solace. A few half-elves turn to bitter hatred of the parent races that rejected them, and become merciless slave–hunters.

Although ranger skills do not come to naturally humans, their famous adaptability wins out in the end, and many humans make fine rangers. A few muls take up the ranger class while surviving in the wilderness after escaping slavery. Dwarves who become rangers find that their focus ability combines powerfully with the abilities of favored enemy and favored lands, but such characters rarely become adventurers since they tend to master wilderness skills in order to guard Dwarven communities.

Pterran rangers are common since rangers get along so well with the druidic and psionic leaders of the pterran villages. Aarakocra are similarly drawn to the ranger class to protect their villages from predators and enemies. Rangers are not unusual among the most hated humanoid races of Athas, such as gith, belgoi, and braxat. Among the various and dwindling communities of the wastes rangers are the most common character class.

\textbf{Alignment:} Rangers can be of any alignment, although they tend not to be lawful, preferring nature to civilization, silence to casual conversation, and ambush to meeting a foe boldly on the battlefield. Good rangers often serve as protectors of a village or of a wild area. In this capacity, rangers try to exterminate or drive off evil creatures that threaten the rangers' lands. Good rangers sometimes protect those who travel through the wilderness, serving sometimes as paid guides, but sometimes as unseen guardians. Neutral rangers tend to be wanderers and mercenaries, rarely tying themselves down to favored lands. The tracking and animal skills of rangers are well known in the World; virtually every trade caravan has at least one ranger scout or mekillot handler. Sometimes they stalk the land for vengeance, either for themselves or for an employer. Generally only evil rangers ply their skills in the slave trade. Other evil rangers seek to emulate nature's most fearsome predators, and take pride and pleasure in the terror that strangers take in their names.

\subsection{Game Rule Information}

\textbf{Hit Die:} d8.

\subsubsection{Class Skills}
\skill{Climb} (Str), \skill{Concentration} (Con), \skill{Craft} (Int), Handle Animal (Cha), \skill{Heal} (Wis), \skill{Hide} (Dex), \skill{Jump} (Str), \skill{Knowledge} (dungeoneering) (Int), \skill{Knowledge} (geography) (Int), \skill{Knowledge} (nature) (Int), \skill{Listen} (Wis), \skill{Move Silently} (Dex), \skill{Profession} (Wis), \skill{Ride} (Dex), \skill{Search} (Int), \skill{Spot} (Wis), \skill{Survival} (Wis), and \skill{Use Rope} (Dex).

\textbf{Skill Points per Level:} 6 + Int modifier ($\times 4$ at 1st level).

\subsubsection{Class Features}

\textbf{Weapon and Armor Proficiency:} A ranger is proficient with all simple and martial weapons, and with light armor and shields (except tower shields).

\textbf{Favored Enemy (Ex):} At 1st level, a ranger may select a type of creature from among those given on \tabref{Athasian Favored Enemies}. The ranger gains a +2 bonus on \skill{Bluff}, \skill{Listen}, \skill{Sense Motive}, \skill{Spot}, and \skill{Survival} checks when using these skills against creatures of this type. Likewise, he gets a +2 bonus on weapon damage rolls against such creatures.

At 5th level and every five levels thereafter (10th, 15th, and 20th level), the ranger may select an additional favored enemy from those given on the table. In addition, at each such interval, the bonus against any one favored enemy (including the one just selected, if so desired) increases by 2.

If the ranger chooses humanoids or outsiders as a favored enemy, he must also choose an associated subtype, as indicated on the table. If a specific creature falls into more than one category of favored enemy, the ranger's bonuses do not stack; he simply uses whichever bonus is higher.

\Table{Athasian Favored Enemies}{X X}{
\tableheader Type (Subtype) & \tableheader Example\\
Aberration & gaj \\
Animal & lion \\
Construct & golem \\
Elemental (air) & air elemental beast \\
Elemental (earth) & crystal spider \\
Elemental (fire) & fire incarnation \\
Elemental (water) & rain paraelemental beast \\
Giant & beasthead giant \\
Humanoid (dwarf) & dwarf \\
Humanoid (elf) & elf \\
Humanoid (gith) & gith \\
Humanoid (halfling) & halfling \\
Humanoid (human) & human \\
Humanoid (jozhal) & jozhal \\
Humanoid (nikaal) & nikaal \\
Humanoid (psionic) & villichi \\
Humanoid (pterran) & pterran \\
Humanoid (reptilian) & silt runner \\
Humanoid (tarek) & tarek \\
Humanoid (tari) & tari \\
Magical beast & kirre \\
Monstrous humanoid & Thri-kreen \\
Outsider & silt half-elemental \\
Plant & hunting cactus \\
Undead & kaisharga \\
Vermin & kank}


\textbf{Favored Terrain (Ex):} At any time when you could normally select a favored enemy, you may instead choose to select a favored terrain given on \tabref{Athasian Terrains}. You receive a +2 bonus to \skill{Hide}, \skill{Knowledge} (nature), \skill{Move Silently}, \skill{Spot} and \skill{Survival} checks made within your favored terrain.

This ability uses the same graduated progression that the favored enemy ability receives.

For example, at first level Sudatu selects monstrous humanoids as a favored enemy, receiving a +2 bonus when combating them. At fifth level, instead of taking a new favored enemy, he selects a Rocky Badlands as his favored terrain, and chooses to increase his favored enemy bonus to +4. At 10th level, Sudatu may again choose a new Favored Enemy, and may also choose between raising his favored enemy or favored terrain bonus by +2.

\Table{Athasian Terrains}{X X}{
\tableheader Terrain Type & \tableheader Terrain Type\\
Boulder Field & Rocky Badland \\
Forest & Salt Flat \\
Jagged Cliffs & Sandy Waste \\
Mountain & Sea of Silt \\
Mud Flat & Stony Barren \\
Obsidian Waste & Swamp \\
Ocean & Verdant Belt}

\textbf{Track:} A ranger gains \feat{Track} as a bonus feat.

\textbf{Wild Empathy (Ex):} A ranger can improve the attitude of an animal. This ability functions just like a \skill{Diplomacy} check to improve the attitude of a person. The ranger rolls 1d20 and adds his ranger level and his Charisma modifier to determine the wild empathy check result. The typical domestic animal has a starting attitude of indifferent, while wild animals are usually unfriendly.

To use wild empathy, the ranger and the animal must be able to study each other, which means that they must be within 30 feet of one another under normal visibility conditions. Generally, influencing an animal in this way takes 1 minute, but, as with influencing people, it might take more or less time.

The ranger can also use this ability to influence a magical beast with an Intelligence score of 1 or 2, but he takes a $-4$ penalty on the check.

\textbf{Combat Style (Ex):} At 2nd level, a ranger must select one of two combat styles to pursue: archery or two-weapon combat. This choice affects the character's class features but does not restrict his selection of feats or special abilities in any way.

If the ranger selects archery, he is treated as having the \feat{Rapid Shot} feat, even if he does not have the normal prerequisites for that feat.

If the ranger selects two-weapon combat, he is treated as having the \feat{Two-Weapon Fighting} feat, even if he does not have the normal prerequisites for that feat.

The benefits of the ranger's chosen style apply only when he wears light or no armor. He loses all benefits of his combat style when wearing medium or heavy armor.

\textbf{Endurance:} A ranger gains \feat{Endurance} as a bonus feat at 3rd level.

\textbf{Animal Companion (Ex):} At 4th level, a ranger gains an animal companion selected from the following list: badger, camel, dire rat, dog, riding dog, eagle, hawk, horse (light or heavy), owl, pony, snake (Small or Medium viper), or wolf. If the campaign takes place wholly or partly in an aquatic environment, the following creatures may be added to the ranger's list of options: manta ray, porpoise, Medium shark, and squid. This animal is a loyal companion that accompanies the ranger on his adventures as appropriate for its kind.

This ability functions like the druid ability of the same name, except that the ranger's effective druid level is one-half his ranger level. A ranger may select from the alternative lists of animal companions just as a druid can, though again his effective druid level is half his ranger level. Like a druid, a ranger cannot select an alternative animal if the choice would reduce his effective druid level below 1st.

\textbf{Spells:} Beginning at 4th level, a ranger gains the ability to cast a small number of divine spells, which are drawn from the ranger spell list. A ranger must choose and prepare his spells in advance (see below).

To prepare or cast a spell, a ranger must have a Wisdom score equal to at least 10 + the spell level. The Difficulty Class for a saving throw against a ranger's spell is 10 + the spell level + the ranger's Wisdom modifier.

Like other spellcasters, a ranger can cast only a certain number of spells of each spell level per day. His base daily spell allotment is given on \tabref{The Ranger}. In addition, he receives bonus spells per day if he has a high Wisdom score. When \tabref{The Ranger} indicates that the ranger gets 0 spells per day of a given spell level, he gains only the bonus spells he would be entitled to based on his Wisdom score for that spell level. The ranger does not have access to any domain spells or granted powers, as a cleric does.

A ranger prepares and casts spells the way a cleric does, though he cannot lose a prepared spell to cast a cure spell in its place. A ranger may prepare and cast any spell on the ranger spell list, provided that he can cast spells of that level, but he must choose which spells to prepare during his daily meditation.

Through 3rd level, a ranger has no caster level. At 4th level and higher, his caster level is one-half his ranger level.

\textbf{Improved Combat Style (Ex):} At 6th level, a ranger's aptitude in his chosen combat style (archery or two-weapon combat) improves. If he selected archery at 2nd level, he is treated as having the \feat{Manyshot} feat, even if he does not have the normal prerequisites for that feat.

If the ranger selected two-weapon combat at 2nd level, he is treated as having the \feat{Improved Two-Weapon Fighting} feat, even if he does not have the normal prerequisites for that feat.

As before, the benefits of the ranger's chosen style apply only when he wears light or no armor. He loses all benefits of his combat style when wearing medium or heavy armor.

\textbf{Woodland Stride (Ex):} Starting at 7th level, a ranger may move through any sort of undergrowth (such as natural thorns, briars, overgrown areas, and similar terrain) at his normal speed and without taking damage or suffering any other impairment.

However, thorns, briars, and overgrown areas that are enchanted or magically manipulated to impede motion still affect him.

\textbf{Swift Tracker (Ex):} Beginning at 8th level, a ranger can move at his normal speed while following tracks without taking the normal $-5$ penalty. He takes only a $-10$ penalty (instead of the normal $-20$) when moving at up to twice normal speed while tracking.

\textbf{Evasion (Ex):} At 9th level, a ranger can avoid even magical and unusual attacks with great agility. If he makes a successful Reflex saving throw against an attack that normally deals half damage on a successful save, he instead takes no damage. Evasion can be used only if the ranger is wearing light armor or no armor. A helpless ranger does not gain the benefit of evasion.

\textbf{Combat Style Mastery (Ex):} At 11th level, a ranger's aptitude in his chosen combat style (archery or two-weapon combat) improves again. If he selected archery at 2nd level, he is treated as having the \feat{Improved Precise Shot} feat, even if he does not have the normal prerequisites for that feat.

If the ranger selected two-weapon combat at 2nd level, he is treated as having the \feat{Greater Two-Weapon Fighting} feat, even if he does not have the normal prerequisites for that feat.

As before, the benefits of the ranger's chosen style apply only when he wears light or no armor. He loses all benefits of his combat style when wearing medium or heavy armor.

\textbf{Camouflage (Ex):} A ranger of 13th level or higher can use the \skill{Hide} skill in any sort of natural terrain, even if the terrain doesn't grant cover or concealment.

\textbf{Hide in Plain Sight (Ex):} While in any sort of natural terrain, a ranger of 17th level or higher can use the \skill{Hide} skill even while being observed.

\subsection{Playing a Ranger}

As a ranger, you nurture a close, almost mystical connection to the deadly terrain of Athas. To you, the burnt landscape is not a friend, but a well-respected adversary. Danger is always present, yet you understand it and even find a certain succor in living alongside it.

\subsubsection{Religion}

Many rangers pay homage to the elements, but a greater number honor the moons and the stars that guide them in the night---even though these celestial bodies do not have priests. In several city-states, particularly Gulg,
Kurn, and Eldaarich, many rangers owe fealty to the sorcerer-kings---virtually the entire noble caste of Gulg is comprised of rangers called judaga. Some rangers pay patronage to the Spirits of the Land, although these spirits do not bestow spells on rangers except those that multi-class as druid.

\subsubsection{Other Classes}

Rangers are slow to make friends with anyone, but have a particular affinity to druids, and to a lesser extent, barbarians and psions. Rangers tend not to lean on others for support and friendship, and often find it difficult to tolerate others who are quite different from themselves, such as talkative traders or controlling templars. Good rangers might simply try to avoid sharing a watch with a character that annoys them; neutral rangers tend to abandon annoying companions or just let them die; while evil rangers act friendly to the annoying companion and then slit their throat in their sleep.

Good rangers tend to hate defilers, although many rangers are ignorant of the distinction between preserving and defiling and hate wizards of all stripes. Strangely, many rangers have little objection to taking a companion who is of a favored enemy race, so long as that they are convinced that the companion is trustworthy and loyal.

\subsubsection{Combat}

Although you are a formidable warrior, you usually prefer not to stand against the sheer might of Athas' fighter, barbarians and gladiators. Your greatest ally is the environment itself. While in you favored terrain, you have a clear advantage over your adversaries. Try choosing favored enemies that are more common in your favored terrain.

As you advance, you are well served to invest in spells that have an effect other than dealing damage. If you can't drop a foe in one or two attacks, you can use entangle, snare, sting of the gold scorpion, or the like to make your opponents less dangerous in a prolonged fight.

\subsubsection{Advancement}

Perhaps the most dangerous place in Athas is inside a city-state: an environment rife with political intrigue, diseases, and assassination. To escape these noxious environs, you sought refuge in the wild where even the foulest elements of a society fear to tread. By gaining an intimate knowledge of this hazardous realm, you buy some breathing room and security from the urban madness.

As your ranger abilities increase, you find the Athasian wilderness a more and more inviting place (if a place with such constant peril can be called inviting). You can use your skills to establish safe havens for yourself or to gain employment opportunities---perhaps guiding a group of recently caught slaves through the Tyr valley or some noble into distant dangerous, location. You can also find that continuing to advance as a ranger or barbarian augments your already impressive abilities in the Athasian lands.

Continue to focus on skills such as Hive, Move Silently, and Survival. Spend discovered treasure on poison, magic weapons, and protective magic. The Mobility feat is good to consider, as is Nature's Child or Wastelander.

\subsection{Starting Packages}
\subsubsection{The Archer}
Elf Ranger

\textbf{Ability Scores:} Str 14, Dex 17, Con 10, Int 10, Wis 13, Cha 8.

\textbf{Skills:} \skill{Hide}, \skill{Listen}, \skill{Move Silently}, \skill{Spot}, \skill{Survival}.

\textbf{Languages:} Common, Elven.

\textbf{Feat:} \feat{Point Blank Shot}, \feat{Track}.

\textbf{Weapons:} Macahuitl (1d8/19–20)

Longbow with 20 arrows (1d8/x3, 100 ft.).

\textbf{Armor:} Studded leather (+3 AC).

\textbf{Other Gear:} Standard adventurer's kit, 19 Cp.

\subsubsection{The Scout}
Halfling Ranger

\textbf{Ability Scores:} Str 11, Dex 17, Con 12, Int 10, Wis 14, Cha 8.

\textbf{Skills:} \skill{Hide}, \skill{Knowledge} (nature), \skill{Listen}, \skill{Move Silently}, \skill{Spot}, \skill{Survival}.

\textbf{Languages:} Halfling.

\textbf{Feat:} \feat{Stealthy}, \feat{Track}.

\textbf{Weapons:} Macahuitl (1d6/19–20)

Small macahuitl (1d3/19–20)

Five javelins (1d4, 30 ft.).

\textbf{Armor:} Studded leather (+3 AC).

\textbf{Other Gear:} Standard adventurer's kit, 65 Cp.

\subsubsection{The Wastelander}
Thri-kreen Ranger

\textbf{Ability Scores:} Str 14, Dex 19, Con 14, Int 8, Wis 15, Cha 4.

\textbf{Skills:} \skill{Hide}, \skill{Knowledge} (nature), \skill{Listen}, \skill{Move Silently}, \skill{Spot}, \skill{Survival}.

\textbf{Languages:} Kreen.

\textbf{Feat:} \feat{Track}, \feat{Wastelander}.

\textbf{Weapons:} Gythka (1d8/1d8)

Three chatkchas (1d6, 20 ft.).

\textbf{Armor:} Studded leather (+3 AC).

\textbf{Other Gear:} Standard adventurer's kit, 5 Cp.

\subsection{Rangers on Athas}
\Quote{Trust me. He might not talk a lot and smell funnier than the rest of your men, but there is no other one I would bring along with me around the Great Ivory Plains.}{Waltian Inika, Gulg dune trader}

The Athasian wilderness is harsh and unforgiving, calling for skilled and capable men to master its ways---the ranger answers that challenge, living a rugged life through clever mastery of his surroundings. The ranger has a potent combination of stealth, woodcraft, magic, and fighting skill, making him the master of the wilderness.

\subsubsection{Daily Life}

A ranger adventures to learn about Athas, to protect nature, and to prove his superior hunting skills. Rangers spend their days in contemplation of nature, and tending their animal companions.

The Athasian ranger is a wanderer who hunts down a defiler to avenge himself for having his village destroyed, or a mercenary hunter for both monsters and humanoid creatures, or even a loner who simply prefers the company of animals.

\subsubsection{Notables}

Tales of halfling snipers are among the common Athasian legends. Any traveler to the Forest Ridge should rightfully fear the cannibals that move without a sound and strike without being seen. Thri-kreen are fabled for their rangers, as they are fast-moving relentless natural hunters, and their unarmed combat abilities become even more deadly when applied to subduing a quarry.

\subsubsection{Organizations}

There is no organized ranger organization; you are most likely to be a loner---or at best the leader of a group of raiders or renegades―than you are to gather with other rangers.

Often merchant houses are eager to employ you as a caravan guide through the most dangerous trade routes, or a city-state's templarate might hire you to provide a safe path to a templar patrol.

\subsubsection{NPC Reactions}

Within a city-state or large settlement, you find that you are either ignored or regarded with some small amount of curiosity. It is only after a city-dweller find himself outside the boundaries of his city-state that he truly appreciates you. Indeed, he holds you in the highest of regards, knowing that you are all that stands between him and a horrible death in the wastes.

\subsubsection{Ranger Lore}

Characters with ranks in \skill{Knowledge} (nature) can research rangers to learn more about them. When a character makes a skill check, read or paraphrase the following, including the information from lower DCs.

\textbf{DC 10:} Only those assisted by a ranger can hope to survive in the Athasian wilderness for long.

\textbf{DC 15:} Rangers move with ease through the harsh terrains that others find dangerous or impassable. They make of this aptitude to specialize in battling specific creatures of the wild.

\textbf{DC 20:} As a ranger advances in knowledge and skill, he grows more and more connected to the land, and eventually manages to draw spells from it.

\section{Rogue}
\Quote{Marek, always helpful, said that the UnderTyr catacombs are supposed to be haunted. Think I'll go make some inquiries about where a 'heretic' like me can get some holy earth. Always go prepared...}{Janos, human rogue}

Dark Sun offers a world of intrigue, manipulation, secret deals, and subtle treachery—in short, a rogue's playground. Rather than eking out their living at the
borders of society, many Athasian rogues dominate the action in many of the most powerful political factions in the Seven Cities: the Noble Houses, the templars, and the Merchant Houses. Often rogues themselves, the wealthy and powerful deploy lesser rogues as pawns in their endless games of acquisition, espionage, and deceit.

Individual rogues run the gamut of Athasian society, from the street rats of the cities to the vagabonds of the outlands, to the prosperous and respectable dune traders, to the low‐ranking templars that search their caravans at the gates. Accomplished rogues are often sought by the nobility as agents, and can earn both wealth and honor in such positions --- or earn a quick death should they be caught contemplating treachery against their masters.

\subsection{Making a Rogue}

A rogue can't stand up face to face with a mul warrior as well as a fighter or gladiator can. With his cunning and your various skills, however, he excels at taking the slightest opportunity and turning to his advantage. His ability to slip under the notice of an observer makes him a capable lone hunter, but his greatest strength are found through interaction with allies and foes, inside or outside, a battle—he can use his enemy`s slightest distraction to deliver a lethal blow, or ensure his party`s safe passage through a templar patrol.

\textbf{Races:} Elves, half‐elves, and humans take to the rogue's skills and lifestyle with the greatest ease. Halflings, dwarves, and muls, while not commonly rogues, adapt to the class remarkably well when they take to it. Thri‐kreen, pterrans, and aarakocra are usually quite adverse to the rogue class, and tend to do poorly. Half‐
giant rogues are unheard of except as fictional figures in comical tales around the fireside.

\textbf{Alignment:} Athasian rogues follow opportunity rather than ideals, but as many of them are lawful as chaotic. Lawful rogues tend to seek security and advancement in the service of nobles or in the ranks of the templars.

\subsection{Game Rule Information}
\textbf{Hit Die:} d6.

\subsection{Class Skills}
\textbf{Class Skills:} Appraise (Int), Balance (Dex), Bluff (Cha), Climb (Str), Craft (Int), Decipher Script (Int), Diplomacy (Cha), Disable Device (Int), Disguise (Cha), Escape Artist (Dex), Forgery (Int), Gather Information (Cha), Hide (Dex), Intimidate (Cha), Jump (Str), Knowledge (local) (Int), Listen (Wis), Move Silently (Dex), Open Lock (Dex), Perform (Cha), Profession (Wis), Search (Int), Sense Motive (Wis), Sleight of Hand (Dex), Spot (Wis), Tumble (Dex), Use Magic Device (Cha), Use Psionic Device (Cha), and Use Rope (Dex).
\textbf{Skill Points per Level:} 8 + Int modifier ($\times4$ at 1st level).

\subection{Class Features}

\textbf{Class Skills:} Use Psionic Device is a class skill for Athasian rogues. Swim is a cross‐class skill for Athasian rogues.

\textbf{Weapon and Armor Proficiency:} In addition to those presented on the Player's Handbook, Athasian rogues are proficient with the bard's friend, blowgun, garrote, small
macahuitl, tonfa, widow's knife, and wrist razor.

\textbf{Special Abilities:} In addition to those presented on the Player's Handbook, Athasian rogues may choose from the following abilities.