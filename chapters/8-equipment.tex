\Chapter{Equipment}
{I have heard tales that suits of clothing fashioned from metal have even been found from time to time. It is generally agreed that these were worn by warriors to protect against the blows of enemy weapons. I can only speculate that the climate must have been far cooler in those ancient days. Any fool that would wear such clothing now would die faster from heat stroke than he would have from the weapons of his foes. Still, the idea that there was once enough metal in the world to allow such a garment to have been manufactured astounds me.

There are even rumors that mounds of steel, silver, and gold lie hidden in the deepest tunnels of certain forlorn cities. I have never seen such a thing myself, but if such treasures exist, they will reward those who find them most handsomely. Those who control such stores of metal can buy food, power, influence, and sometimes even the sorcerer-king's protection.}
{The Wanderer's Journal}

Dark Sun characters must be well equipped in order to endure the rigors of Athas. This chapter covers a variety of topics related to mundane equipment that every hero needs to survive and prosper.

\section{Equipping a Character}
The world of Athas has a very specific feel to it; many things that are taken for granted in other campaign worlds, like the availability of metal and water, are very different on this heat-wracked planet. To maintain this feel, the equipment available to characters should reflect these differences.

\section{Wealth and Money}
\subsection{Coins}
All prices in Dark Sun are given in terms of ceramic pieces, the most common coin. Ceramics are made from glazed clay and baked in batches once a year in a secure process supervised by the high templar that supervises the city's treasury. Bits are literally one‐tenth parts of a ceramic piece---the ceramic pieces break easily into ten bits. Some cities' ceramic pieces have small holes that can be threaded onto a bracelet or necklace. The lowest unit of Athasian trade is the lead bead (bd).

Each of the city-states of the Tablelands produces its own currency. All cities use ceramic pieces as the most common coin, but also mint silver coins and, in some cases, rare and highly prized gold coins.

\Table[\scriptsize]{Currency Conversions}{l C C C C}{
& \multicolumn{4}{c}{\tableheader Exchange Value}\\
& \tableheader bd & \tableheader bit & \tableheader cp & \tableheader sp\\
Lead bead (bd) & 1 & 1/10 & 1/100 & 1/1,000\\
Ceramic bit (bit) & 10 & 1 & 1/10 & 1/100\\
Ceramic piece (cp) & 100 & 10 & 1 & 1/10\\
Silver piece (sp) & 1,000 & 100 & 10 & 1\\
Gold piece (gp) & 10,000 & 1000 & 100 & 10\\
}

\subsubsection{Moneychangers}
Adventurers that travel between cities will need to change their currency for local currency at each city they visit. With a couple of exceptions, the city-states have moneychangers available for incoming visitors. Located near the city gates and in large market places, moneychangers denote their business by hanging a large purple banner from their shop. The banners are always purple, but the moneychangers in each city-state display a different emblem on the banner, based on their city's standard.

Moneychangers charge each customer a fee to change coins. The fees differ by city and are summarized on \tabref{Moneychangers}.

\Table{Moneychangers}{C C}{
\tableheader City & \tableheader Exchange Rate\\
Balic & 6\%\\
Draj & 8\%\\
Gulg & 10\%\\
Kurn & 16\%\\
Nibenay & 14\%\\
Raam & 12\%\\
Tyr & 12\%\\
Urik & 9\%\\
}

These fees are averages and may vary slightly. There are of course many unscrupulous money merchants who will charge as much as they can get away with. Moneychangers in Kurn are rare but a couple do exist. Since metal coins from any city-state are readily accepted by local merchants and no corresponding Kurnish coins exist there is little need to exchange such coins. There are, however, a few moneychangers willing to exchange ceramic pieces.

There are two cities that do not have moneychangers. Visitors to Celik have no need of a moneychangers, as merchants in that city take coins of all types. Nor are there any moneychangers in Eldaarich. Since Eldaarich has been shut off from the rest of the land for so long, visitors needing to exchange money have been nonexistent, so no moneychangers have set up business.

\subsection{Trade}
In general, the Athasian economy in the cities is relatively stable thanks to the Merchant Houses. Under normal conditions, supply is ample thanks to the caravans traveling back and forth between the cities. However, for smaller communities and trade outposts the price situation on certain goods can sway drastically. A raider attack or sandstorm can result in lack of necessities such as food and water, for which people will pay almost any amount of coin. Coins are not the only means of exchange. Barter and trade in commodities is widespread.

Dune traders commonly exchange trade goods without using currency, instead relying on a basic bartering system.


\Table{Trade Goods}{l X}{
\tableheader Cost & \tableheader Item\\
2 bits & One pound of salt\\
3 bits & One pound of grain or faro\\
5 bits & One pound of lead\\
1 cp & One pound of nuts, or one pound of kank nectar\\
4 cp & One square yard of cotton (cloth)\\
5 cp & One pound of obsidian\\
10 cp & One square yard of silk, or one tun of water, or one erdlu\\
50 cp & One pound of copper, or one herding kank, or one aprig\\
100 cp & One pound of iron, or one male carru, or one inix\\
200 cp & One mekillot\\
300 cp & One female carru\\
500 cp & One pound of silver\\
5,000 cp & One pound of gold\\
}

\subsection{Selling Loot}
In general, a character can sell something for half its listed price.

Trade goods are the exception to the half-price rule. A trade good, in this sense, is a valuable good that can be easily exchanged almost as if it were cash itself.

\section{Weapons}
Characters in a Dark Sun game use a variety of weapons: some with direct counterparts in the real world, some without.

\subsection{Weapon Categories}
Weapons are grouped into several interlocking sets of categories.

These categories pertain to what training is needed to become proficient in a weapon's use (simple, martial, or exotic), the weapon's usefulness either in close combat (melee) or at a distance (ranged, which includes both thrown and projectile weapons), its relative encumbrance (light, one-handed, or two-handed), and its size (Small, Medium, or Large).

\textbf{Simple, Martial, and Exotic Weapons:} Anybody but a druid, monk, or wizard is proficient with all simple weapons. Barbarians, fighters, paladins, and rangers are proficient with all simple and all martial weapons. Characters of other classes are proficient with an assortment of mainly simple weapons and possibly also some martial or even exotic weapons. A character who uses a weapon with which he or she is not proficient takes a -4 penalty on attack rolls.

\textbf{Melee and Ranged Weapons:} Melee weapons are used for making melee attacks, though some of them can be thrown as well. Ranged weapons are thrown weapons or projectile weapons that are not effective in melee.

\textit{Reach Weapons:} Glaives, guisarmes, lances, longspears, ranseurs, spiked chains, and whips are reach weapons. A reach weapon is a melee weapon that allows its wielder to strike at targets that aren't adjacent to him or her. Most reach weapons double the wielder's natural reach, meaning that a typical Small or Medium wielder of such a weapon can attack a creature 10 feet away, but not a creature in an adjacent square. A typical Large character wielding a reach weapon of the appropriate size can attack a creature 15 or 20 feet away, but not adjacent creatures or creatures up to 10 feet away.

Note: Small and Medium creatures wielding reach weapons threaten all squares 10 feet (2 squares) away, even diagonally. (This is an exception to the rule that 2 squares of diagonal distance is measured as 15 feet.)

\textit{Double Weapons:} Dire flails, dwarven urgroshes, gnome hooked hammers, orc double axes, quarterstaffs, and two-bladed swords are double weapons. A character can fight with both ends of a double weapon as if fighting with two weapons, but he or she incurs all the normal attack penalties associated with two-weapon combat, just as though the character were wielding a one-handed weapon and a light weapon.

The character can also choose to use a double weapon two handed, attacking with only one end of it. A creature wielding a double weapon in one hand can't use it as a double weapon—only one end of the weapon can be used in any given round.

\textit{Thrown Weapons:} Daggers, clubs, shortspears, spears, darts, javelins, throwing axes, light hammers, tridents, shuriken, and nets are thrown weapons. The wielder applies his or her Strength modifier to damage dealt by thrown weapons (except for splash weapons). It is possible to throw a weapon that isn't designed to be thrown (that is, a melee weapon that doesn't have a numeric entry in the Range Increment column on Table: Weapons), but a character who does so takes a -4 penalty on the attack roll. Throwing a light or one-handed weapon is a standard action, while throwing a two-handed weapon is a full-round action. Regardless of the type of weapon, such an attack scores a threat only on a natural roll of 20 and deals double damage on a critical hit. Such a weapon has a range increment of 10 feet.

\textit{Projectile Weapons:} Light crossbows, slings, heavy crossbows, shortbows, composite shortbows, longbows, composite longbows, hand crossbows, and repeating crossbows are projectile weapons. Most projectile weapons require two hands to use (see specific weapon descriptions). A character gets no Strength bonus on damage rolls with a projectile weapon unless it's a specially built composite shortbow, specially built composite longbow, or sling. If the character has a penalty for low Strength, apply it to damage rolls when he or she uses a bow or a sling.

\textit{Ammunition:} Projectile weapons use ammunition: arrows (for bows), bolts (for crossbows), or sling bullets (for slings). When using a bow, a character can draw ammunition as a free action; crossbows and slings require an action for reloading. Generally speaking, ammunition that hits its target is destroyed or rendered useless, while normal ammunition that misses has a 50\% chance of being destroyed or lost.

Although they are thrown weapons, shuriken are treated as ammunition for the purposes of drawing them, crafting masterwork or otherwise special versions of them (see Masterwork Weapons), and what happens to them after they are thrown.

\textbf{Light, One-Handed, and Two-Handed Melee Weapons:} This designation is a measure of how much effort it takes to wield a weapon in combat. It indicates whether a melee weapon, when wielded by a character of the weapon's size category, is considered a light weapon, a one-handed weapon, or a two-handed weapon.

\textit{Light:} A light weapon is easier to use in one's off hand than a one-handed weapon is, and it can be used while grappling. A light weapon is used in one hand. Add the wielder's Strength bonus (if any) to damage rolls for melee attacks with a light weapon if it's used in the primary hand, or one-half the wielder's Strength bonus if it's used in the off hand. Using two hands to wield a light weapon gives no advantage on damage; the Strength bonus applies as though the weapon were held in the wielder's primary hand only.

An unarmed strike is always considered a light weapon.

\textit{One-Handed:} A one-handed weapon can be used in either the primary hand or the off hand. Add the wielder's Strength bonus to damage rolls for melee attacks with a one-handed weapon if it's used in the primary hand, or ½ his or her Strength bonus if it's used in the off hand. If a one-handed weapon is wielded with two hands during melee combat, add 1½ times the character's Strength bonus to damage rolls.

\textit{Two-Handed:} Two hands are required to use a two-handed melee weapon effectively. Apply 1½ times the character's Strength bonus to damage rolls for melee attacks with such a weapon.

\textbf{Weapon Size:} Every weapon has a size category. This designation indicates the size of the creature for which the weapon was designed.

\Table{Larger and Smaller Weapon Damage}{l C C C C C}{
\tableheader Example Weapon & \tableheader Tiny & \tableheader Small & \tableheader Medium & \tableheader Large & \tableheader Huge \\
Shuriken & — & 1 & 1d2 & 1d3 & 1d4\\
Gauntlet & 1 & 1d2 & 1d3 & 1d4 & 1d6\\
Dagger & 1d2 & 1d3 & 1d4 & 1d6 & 1d8\\
Shortspear & 1d3 & 1d4 & 1d6 & 1d8 & 2d6\\
Falchion & 1d4 & 1d6 & 2d4 & 2d6 & 3d6\\
Longsword & 1d4 & 1d6 & 1d8 & 2d6 & 3d6\\
Bastard Sword & 1d6 & 1d8 & 1d10 & 2d8 & 3d8\\
Greataxe & 1d8 & 1d10 & 1d12 & 3d6 & 4d6\\
Greatsword & 1d8 & 1d10 & 2d6 & 3d6 & 4d6\\
}

A weapon's size category isn't the same as its size as an object. Instead, a weapon's size category is keyed to the size of the intended wielder. In general, a light weapon is an object two size categories smaller than the wielder, a one-handed weapon is an object one size category smaller than the wielder, and a two-handed weapon is an object of the same size category as the wielder.

\textit{Inappropriately Sized Weapons:} A creature can't make optimum use of a weapon that isn't properly sized for it. A cumulative -2 penalty applies on attack rolls for each size category of difference between the size of its intended wielder and the size of its actual wielder. If the creature isn't proficient with the weapon a -4 nonproficiency penalty also applies.

The measure of how much effort it takes to use a weapon (whether the weapon is designated as a light, one-handed, or two-handed weapon for a particular wielder) is altered by one step for each size category of difference between the wielder's size and the size of the creature for which the weapon was designed. If a weapon's designation would be changed to something other than light, one-handed, or two-handed by this alteration, the creature can't wield the weapon at all.

\textbf{Improvised Weapons:} Sometimes objects not crafted to be weapons nonetheless see use in combat. Because such objects are not designed for this use, any creature that uses one in combat is considered to be nonproficient with it and takes a -4 penalty on attack rolls made with that object. To determine the size category and appropriate damage for an improvised weapon, compare its relative size and damage potential to the weapon list to find a reasonable match. An improvised weapon scores a threat on a natural roll of 20 and deals double damage on a critical hit. An improvised thrown weapon has a range increment of 10 feet.

\subsubsection{Metal Weapons}
Metal is rare on Athas, and many weapons ordinarily crafted using metal components are extremely expensive. Unworked iron is worth 100 Cp per pound on average, but can cost much, much more in some places. Worked metal is even more expensive, as craftsmen who actually know how to craft metal items are rare at best. Most metal weapons are items dating back to the Green Age, or have been crafted from the meager resources of Tyr's iron mines.

Due to the rarity of metal, weapons and other items constructed primarily from metal are priced at in gp, e.g., a metal longsword costs 15 gp (or 1,500 Cp). Due to the extremely high cost of metal weaponry, most weapons are constructed from inferior, but functional, materials instead on Athas. Most common are bone and stone such as flint or obsidian, but treated wood is sometimes used as well. Metal weapons constructed from inferior materials, such as bone longsword or an axe with a head made from stone, suffer a $-1$ penalty to attack and damage rolls. This penalty cannot reduce damage dealt below 1.

Furthermore, due to the rarity of metal, Athas has its share of unique weapons designed to be constructed from non–metal materials; as such, they do not suffer from the inferior materials penalties described above.

\NamedWeaponTable{Simple Weapons}{
\WeaponType{Unarmed Attacks}\\
Gauntlet
	& 2 cp & 4 cp & 1d2 & 1d3 & 1d4 & $\times2$ & & 0.25 kg& 0.5 kg& 1 kg & Bludgeoning\\
Unarmed strike
	& & & 1d2 & 1d3 & 1d4 & $\times2$ & & & & & Blud. non-lethal\\

\WeaponType{Light Melee Weapons}\\
Dagger
	& 2 cp & 4 cp & 1d3 & 1d4 & 1d6 & 19--20/$\times2$ & 3 m & 0.25 kg & 0.5 kg & 1 kg & Pierc. or slash.\\
Pushik
	& 4 cp & 8 cp & 1d3 & 1d4 & 1d6 & $\times3$ & & 0.25 kg & 0.5 kg & 1 kg & Piercing\\

\WeaponType{One-Handed Melee Weapons}\\
Club
	& & & 1d4 & 1d6 & 1d8 & $\times2$ & 3 m & 0.75 kg & 1.5 kg & 3 kg & Bludgeoning\\
Quabone
	& 3 cp & 6 cp & 1d4 & 1d6 & 1d8 & $\times2$ & & 1 kg & 2 kg & 4 kg & Bludgeoning\\
Shortspear
	& 1 cp & 2 cp & 1d4 & 1d6 & 1d8 & $\times2$ & 6 m & 0.75 kg & 1.5 kg & 3 kg & Piercing\\
Tonfa
	& 5 cp & 10 cp & 1d3 & 1d4 & 1d6 & $\times2$ & & 0.5 kg & 1 kg & 2 kg & Bludgeoning\\

\WeaponType{Two-Handed Melee Weapons}\\
Great tonfa
	& 10 cp & 20 cp & 1d4 & 1d6 & 1d8 & $\times2$ & & 1.25 kg & 2.5 kg & 5 kg & Bludgeoning\\
Longspear\footnotemark[1]
	& 5 cp & 10 cp & 1d6 & 1d8 & 2d6 & $\times3$ &  & 2.25 kg & 4.5 kg & 9 kg & Piercing\\
Quarterstaff\footnotemark[2]
	& & & 1d4 1d4 & 1d6 1d6 & 1d8 1d8 & $\times2$ & & 1 kg & 2 kg & 4 kg & Bludgeoning\\
Spear
	& 2 cp & 4 cp & 1d6 & 1d8 & 2d6 & $\times3$ & 6 m & 1.5 kg & 3 kg & 6 kg & Piercing\\

\WeaponType{Ranged Weapons}\\
Blowgun
	& 5 cp & 10 cp & 1 & 1d2 & 1d3 & $\times2$ & 3 m & 1 kg & 2 kg & 4 kg & Piercing\\
\Projectile{Needles, blowgun (20)}{1 cp}{2 cp}{}{}{}\\
Crossbow, heavy
	& 50 cp & 100 cp & 1d8 & 1d10 & 2d8 & 19--20/$\times2$ & 36 m & 2 kg & 4 kg & 8 kg & Piercing\\
\Projectile{Bolts, crossbow (10)}{1 cp}{2 cp}{0.25 kg}{0.5 kg}{1 kg}\\
Crossbow, light
	& 35 cp & 70 cp & 1d6 & 1d8 & 2d6 & 19--20/$\times2$ & 24 m & 1 kg & 2 kg & 4 kg & Piercing\\
\Projectile{Bolts, crossbow (10)}{1 cp}{2 cp}{0.25 kg}{0.5 kg}{1 kg}\\
Dart
	& 5 bits & 1 cp & 1d3 & 1d4 & 1d6 & $\times2$ & 6 m & 0.12 kg & 0.25 kg & 0.5 kg & Piercing\\
Javelin
	& 1 cp & 2 cp & 1d4 & 1d6 & 1d8 & $\times2$ & 9 m & 0.5 kg & 1 kg & 2 kg & Piercing\\
Pelota
	& 2 cp & 4 cp & 1d3 & 1d4 & 1d6 & $\times2$ & 3 m & 0.25 kg & 0.5 kg & 1 kg & Blud. and pierc.\\
Sling
	& & & 1d3 & 1d4 & 1d6 & $\times2$ & 50 ft. & & & & Bludgeoning\\
\Projectile{Bullets, sling (10)}{1 bit}{2 bits}{0.25 kg}{0.5 kg}{1 kg}\\

\rowcolor{white}
\multicolumn{12}{l}{1 Reach weapon}\\
\rowcolor{white}
\multicolumn{12}{l}{2 Double weapon}\\
}

\NamedWeaponTable{Martial Weapons}{
\WeaponType{Light Melee Weapons}\\
Forearm Axe
	& 30 cp & 60 cp & 1d3 & 1d4 & 1d6 & $\times3$ & & 1.5 kg & 3 kg & 6 kg & Slashing\\
Macahuitl, small
	& 20 cp & 40 cp & 1d4 & 1d6 & 1d8 & 19--20/$\times2$ & & 0.5 kg & 1 kg & 2 kg & Slashing\\
Sap
	& 1 cp & 2 cp & 1d4 & 1d6 & 1d8 & $\times2$ & & 0.5 kg & 1 kg & 2 kg & Blud. non-lethal\\
Shield, light
	& $\star$ & $\star$ & 1d2 & 1d3 & 1d4 & $\times2$ & & $\star$ & $\star$ & $\star$ & Bludgeoning\\
Slodak
	& 18 cp & 36 cp & 1d4 & 1d6 & 1d8 & 19--20/$\times2$ & & 1 kg & 2 kg & 4 kg & Slashing\\
Spiked armor
	& $\star$ & $\star$ & 1d4 & 1d6 & 1d8 & $\times2$ & & $\star$ & $\star$ & $\star$ & Piercing\\
Spiked shield, light
	& $\star$ & $\star$ & 1d3 & 1d4 & 1d6 & $\times2$ & & $\star$ & $\star$ & $\star$ & Piercing\\
Tortoise blade
	& 20 cp & 40 cp & 1d3 & 1d4 & 1d6 & 20/$\times2$ & & 0.5 kg & 1 kg & 2 kg & Piercing\\

\WeaponType{One-Handed Melee Weapons}\\
Alak
	& 7 cp & 14 cp & 1d4 & 1d6 & 1d8 & $\times3$ & & 1.5 kg & 3 kg & 6 kg & Piercing\\
Alhulak\footnotemark[1]
	& 40 cp & 80 cp & 1d4 & 1d6 & 1d8 & $\times3$ & & 2.25 kg & 4.5 kg & 9 kg & Piercing\\
Carrikal
	& 10 cp & 20 cp & 1d6 & 1d8 & 2d6 & $\times3$ & & 1.5 kg & 3 kg & 6 kg & Slashing\\
Impaler
	& 8 cp & 16 cp & 1d4 & 1d6 & 1d8 & $\times4$ & & 1.25 kg & 2.5 kg & 5 kg & Piercing\\
Macahuitl
	& 35 cp & 70 cp & 1d6 & 1d8 & 2d6 & 19--20/$\times2$ & & 1.25 kg & 2.5 kg & 5 kg & Slashing\\
Shield, heavy
	& $\star$ & $\star$ & 1d3 & 1d4 & 1d6 & $\times2$ & & $\star$ & $\star$ & $\star$ & Bludgeoning\\
Spiked shield, heavy
	& $\star$ & $\star$ & 1d4 & 1d6 & 1d8 & $\times2$ & & $\star$ & $\star$ & $\star$ & Piercing\\

\WeaponType{Two-Handed Melee Weapons}\\
Club, datchi\footnotemark[1]
	& 5 cp & 10 cp & 1d6 & 1d8 & 2d6 & $\times3$ & & 2.5 kg & 5 kg & 10 kg & Bludgeoning\\
Crusher, Fixed\footnotemark[1]
	& 60 cp & 120 cp & 1d6 & 1d8 & 2d6 & $\times2$ & & 3 kg & 6 kg & 12 kg & Bludgeoning\\
Gouge
	& 20 cp & 40 cp & 1d8 & 1d10 & 2d8 & $\times3$ & & 3 kg & 6 kg & 12 kg & Piercing\\
Greatclub
	& 5 cp & 10 cp & 1d8 & 1d10 & 2d8 & $\times2$ & & 2 kg & 4 kg & 8 kg & Bludgeoning\\
Lance
	& 10 cp & 20 cp & 1d6 & 1d8 & 2d6 & $\times3$ & & 2.5 kg & 5 kg & 10 kg & Piercing\\
Macahuitl, Great
	& 50 cp & 100 cp & 1d10 & 2d6 & 3d6 & 19--20/$\times2$ & & 3 kg & 6 kg & 12 kg & Slashing\\
Maul
	& 25 cp & 50 cp & 1d10 & 1d12 & 3d6 & $\times2$ & & 2.5 kg & 5 kg & 10 kg & Bludgeoning\\
Shield, long
	& $\star$ & $\star$ & 1d4 & 1d6 & 1d8 & $\times2$ & & $\star$ & $\star$ & $\star$ & Bludgeoning\\
Spiked shield, long
	& $\star$ & $\star$ & 1d6 & 1d8 & 2d6 & $\times2$ & & $\star$ & $\star$ & $\star$ & Piercing\\
Tkaesali\footnotemark[1]
	& 8 cp & 16 cp & 1d8 & 1d10 & 2d8 & $\times3$ & & 3.75 kg & 7.5 kg & 15 kg & Slashing\\
Trikal
	& 10 cp & 20 cp & 1d6 & 1d8 & 2d6 & $\times3$ & & 1.75 kg & 3.5 kg & 7 kg & Slashing\\

\WeaponType{Ranged Weapons}\\
Atlatl
	& 25 cp & 50 cp & 1d4 & 1d6 & 1d8 & $\times3$ & 40 ft. & 1.5 kg & 3 kg & 6 kg & Piercing\\
\Projectile{Javeli, atlatl}{2 cp}{4 cp}{0.5 kg}{1 kg}{2 kg}\\
Crossbow, fixed
	& 200 cp & 400 cp & 1d12 & 2d8 & 3d8 & 19--20/$\times2$ & 45 m & 25 kg & 50 kg & 100 kg & Piercing\\
\Projectile{Bolts (10)}{3 cp}{6 cp}{0.75 kg}{1.5 kg}{3 kg}\\
Longbow
	& 75 cp & 150 cp & 1d6 & 1d8 & 2d6 & $\times3$ & 30 m & 0.75 kg & 1.5 kg & 3 kg & Piercing\\
\Projectile{Arrows (10)}{1 cp}{2 cp}{0.75 kg}{1.5 kg}{3 kg}\\
Longbow, composite
	& 100 cp & 200 cp & 1d6 & 1d8 & 2d6 & $\times3$ & 33 m & 0.75 kg & 1.5 kg & 3 kg & Piercing\\
\Projectile{Arrows (10)}{1 cp}{2 cp}{0.75 kg}{1.5 kg}{3 kg}\\
Shortbow
	& 30 cp & 60 cp & 1d4 & 1d6 & 1d8 & $\times3$ & 18 m & 0.5 kg & 1 kg & 2 kg & Piercing\\
\Projectile{Arrows (10)}{1 cp}{2 cp}{0.75 kg}{1.5 kg}{3 kg}\\
Shortbow, composite
	& 75 cp & 150 cp & 1d4 & 1d6 & 1d8 & $\times3$ & 21 m & 0.5 kg & 1 kg & 2 kg & Piercing\\
\Projectile{Arrows (10)}{1 cp}{2 cp}{0.75 kg}{1.5 kg}{3 kg}\\

\rowcolor{white}
\multicolumn{12}{l}{1 Reach weapon}\\
\rowcolor{white}
\multicolumn{12}{l}{$\star$ Inherent properties of the armor or shield. See Armor for details.}\\
}

\NamedWeaponTable{Exotic Weapons}{
\WeaponType{Light Melee Weapons}\\
Bard's Friend
	& 20 cp & 40 cp & 1d3 & 1d4 & 1d6 & 18--20/$\times2$ & & 0.25 kg & 0.5 kg & 1 kg & Piercing\\
Bard's Garrote
	& 200 cp & 400 cp & 1d6 & 2d4 & 2d6 & $\times2$ & & 0.25 kg & 0.5 kg & 1 kg & Blud. non-lethal\\
Ko
	& 1 cp & 2 cp & 1d3 & 1d4 & 1d6 & $\times4$ & & 0.75 kg & 1.5 kg & 3 kg & Piercing\\
Handfork
	& 20 cp & 40 cp & 1d3 & 1d4 & 1d6 & $\times2$ & & 0.5 kg & 1 kg & 2 kg & Slashing\\
Lajav
	& 8 cp & 16 cp & 1d3 & 1d4 & 1d6 & $\times4$ & & 2 kg & 4 kg & 8 kg & Bludgeoning\\
Nunchaku
	& 2 cp & 4 cp & 1d4 & 1d6 & 1d8 & $\times2$ & & 0.5 kg & 1 kg & 2 kg & Bludgeoning\\
Sai
	& 1 cp & 2 cp & 1d3 & 1d4 & 1d6 & $\times2$ & 3 m & 0.25 kg & 0.5 kg & 1 kg & Bludgeoning\\
Singing Sticks
	& 10 cp & 20 cp & 1d4 & 1d6 & 1d8 & $\times2$ & & 0.25 kg & 0.5 kg & 1 kg & Bludgeoning\\
Talid
	& 40 cp & 80 cp & 1d4 & 1d6 & 1d8 & 19--20/$\times2$ & & 1 kg & 2 kg & 4 kg & Piercing\\
Widow's Knife
	& 50 cp & 100 cp & 1d3 & 1d4 & 1d6 & $\times3$ & & 0.5 kg & 1 kg & 2 kg & Piercing\\
Wrist Razor
	& 15 cp & 30 cp & 1d4 & 1d6 & 1d8 & 18--20/$\times2$ & & 0.5 kg & 1 kg & 2 kg & Slashing\\

\WeaponType{One-Handed Melee Weapons}\\
Longblade, elven
	& 100 Cp & 200 Cp & 1d6 & 1d8 & 2d6 & 18--20/$\times2$ & & 0.75 kg & 1.5 kg & 3 kg & Slashing\\
Heartpick
	& 9 Cp & 18 Cp & 1d6 & 1d8 & 2d6 & $\times4$ & & 0.5 kg & 1 kg & 2 kg & Piercing\\
Master's Whip\footnotemark[1]
	& 25 Cp & 50 Cp & 1d2 & 1d3 & 1d4 & $\times2$ & & 1.25 kg & 2.5 kg & 5 kg & Slashing\\
Whip\footnotemark[1]
	& 1 cp & 2 cp & 1d2 & 1d3 & 1d4 & $\times2$ & & 0.5 kg & 1 kg & 2 kg & Slash. non-lethal\\

\WeaponType{Two-Handed Melee Weapons}\\
Cahulak\footnotemark[2]
	& 120 cp & 240 cp & 1d4 1d4 & 1d6 1d6 & 1d8 1d8 & $\times3$ & & 3 kg & 6 kg & 12 kg & Piercing\\
Crusher, free\footnotemark[1]
	& 18 cp & 36 cp & 1d8 & 1d10 & 2d8 & $\times2$ & & 3 kg & 6 kg & 12 kg & Bludgeoning\\
Dragon's paw\footnotemark[2]
	& 80 cp & 160 cp & 1d4 1d4 & 1d6 1d6 & 1d8 1d8 & 19--20/$\times2$ & & 2.25 kg & 4.5 kg & 9 kg & Piercing\\
Gythka\footnotemark[2]
	& 60 cp & 120 cp & 1d6 1d6 & 1d8 1d8 & 2d6 2d6 & $\times2$ & & 6.25 kg & 12.5 kg & 25 kg & Slashing\\
Lotulis\footnotemark[2]
	& 115 cp & 230 cp & 1d6 1d6 & 1d8 1d8 & 2d6 2d6 & 19--20/$\times2$ & & 2.25 kg & 4.5 kg & 9 kg & Slashing\\
Pike, weighted\footnotemark[2]
	& 75 cp & 150 cp & 1d6 1d4 & 1d8 1d6 & 2d6 1d8 & 19--20/$\times2$ & & 3.75 kg & 7.5 kg & 15 kg & Bludgeoning Piercing\\
Spear, double-tipped\footnotemark[2]
	& 20 cp & 40 cp & 1d6 1d6 & 1d8 1d8 & 2d6 2d6 & $\times3$ & 6 m & 1.5 kg & 3 kg & 6 kg & Piercing\\
Thanak
	& 20 cp & 40 cp & 1d10 & 2d6 & 3d6 & $\times3$ & & 2.5 kg & 5 kg & 10 kg & Slashing\\
Swatter
	& 100 cp & 200 cp & 1d12 & 2d8 & 3d8 & $\times4$ & & 8.75 kg & 17.5 kg & 35 kg & Bludgeoning\\
Mekillot sap\footnotemark[1]
	& 25 cp & 50 cp & 1d12 & 2d8 & 3d8 & $\times2$ & 3 m & 7.5 kg & 15 kg & 30 kg & Blud. non-lethal\\

\WeaponType{Ranged Weapons}\\
Blowgun, greater
	& 10 cp & 20 cp & 1d3 & 1d4 & 1d6 & $\times2$ & 3 m & 1 kg & 2 kg & 4 kg & Piercing\\
\Projectile{Darts, blowgun (10)}{1 cp}{2 cp}{0.25 kg}{0.5 kg}{1 kg}\\
Bolas
	& 5 cp & 10 cp & 1d3 & 1d4 & 1d6 & $\times2$ & 3 m & 0.5 kg & 1 kg & 2 kg & Blud. non-lethal\\
Chatkcha
	& 20 cp & 40 cp & 1d4 & 1d6 & 1d8 & $\times2$ & 6 m & 0.75 kg & 1.5 kg & 3 kg & Piercing\\
Crossbow, hand
	& 100 cp & 200 cp & 1d3 & 1d4 & 1d6 & 19--20/$\times2$ & 9 m & 0.5 kg & 1 kg & 2 kg & Piercing\\
\Projectile{Bolts (10)}{1 cp}{2 cp}{0.25 kg}{0.5 kg}{1 kg}\\
Crossbow, repeat. heavy
	& 400 cp & 800 cp & 1d8 & 1d10 & 2d8 & 19--20/$\times2$ & 36 m & 3 kg & 6 kg & 12 kg & Piercing\\
\Projectile{Bolts (5)}{1 cp}{2 cp}{0.25 kg}{0.5 kg}{1 kg}\\
Crossbow, repeating light
	& 250 cp & 500 cp & 1d6 & 1d8 & 2d6 & 19--20/$\times2$ & 24 m & 1.5 kg & 3 kg & 6 kg & Piercing\\
\Projectile{Bolts (5)}{1 cp}{2 cp}{0.25 kg}{0.5 kg}{1 kg}\\
Dejada
	& 20 cp & 40 cp & 1d4 & 1d6 & 1d8 & $\times2$ & 9 m & 0.5 kg & 1 kg & 2 kg & Piercing\\
\Projectile{Pelota, dejada}{2 cp}{4 cp}{0.25 kg}{0.5 kg}{1 kg}\\
Lasso
	& 2 cp & 4 cp & & & & $\times2$ & 3 m & 0.5 kg & 1 kg & 2 kg & Bludgeoning\\
Net
	& 20 cp & 40 cp & & & & & 3 m & 1.5 kg & 3 kg & 6 kg & \\
Skyhammer
	& 50 cp & 100 cp & 1d8 & 1d10 & 2d8 & $\times2$ & 4.5 m & 1.5 kg & 3 kg & 6 kg & Bludgeoning\\
Splashbow
	& 300 cp & 600 cp & 1d3 & 1d4 & 1d6 & $\times2$ & 18 m & 15 kg & 30 kg & 60 kg & Bludgeoning\\
~ Pelota, hinged
	& 5 cp & 10 cp & & & & $\times2$ & 4.5 m & 0.5 kg & 1 kg & 2 kg & Bludgeoning\\
Zerka
	& 30 cp & 60 cp & 1d6 & 1d8 & 2d6 & 18--20/$\times2$ & 9 m & 2.25 kg & 4.5 kg & 9 kg & Piercing\\

\rowcolor{white}
\multicolumn{12}{l}{1 Reach weapon}\\
\rowcolor{white}
\multicolumn{12}{l}{2 Double weapon}\\
}

\WeaponTable{Metal Simple Weapons}{
\WeaponType{Light Melee Weapons}\\
Dagger, punching
	& 2 gp & 4 gp & 1d3 & 1d4 & 1d6 & $\times3$ & & 0.25 kg & 0.5 kg & 1 kg & Piercing\\
\InferiorWeapon{(bone or wood)}{1 cp}{2 cp}{0.12 kg}{0.25 kg}{0.5 kg}\\
\InferiorWeapon{(stone)}{1 cp}{2 cp}{0.5 kg}{1 kg}{2 kg}\\
Gauntlet, spiked
	& 25 sp & 5 gp & 1d3 & 1d4 & 1d6 & $\times2$ & & 0.25 kg & 0.5 kg & 1 kg & Piercing\\
\InferiorWeapon{(bone or wood)}{25 bits}{5 cp}{0.12 kg}{0.25 kg}{0.5 kg}\\
\InferiorWeapon{(stone)}{25 bits}{5 cp}{0.5 kg}{1 kg}{2 kg}\\
Mace, light
	& 5 gp & 10 gp & 1d4 & 1d6 & 1d8 & $\times2$ & & 1 kg & 2 kg & 4 kg & Bludgeoning\\
\InferiorWeapon{(bone or wood)}{25 bits}{5 cp}{0.5 kg}{1 kg}{2 kg}\\
\InferiorWeapon{(stone)}{25 bits}{5 cp}{2 kg}{4 kg}{8 kg}\\
Sickle
	& 6 gp & 12 gp & 1d4 & 1d6 & 1d8 & $\times2$ & & 0.5 kg & 1 kg & 2 kg & Slashing\\
\InferiorWeapon{(bone or wood)}{1 cp}{2 cp}{0.25 kg}{0.5 kg}{1 kg}\\
\InferiorWeapon{(stone)}{1 cp}{2 cp}{1 kg}{2 kg}{4 kg}\\

\WeaponType{One-Handed Melee Weapons}\\
Mace, heavy
	& 12 cp & 24 cp & 1d6 & 1d8 & 2d6 & $\times2$ & & 2 kg & 4 kg & 8 kg & Bludgeoning\\
\InferiorWeapon{(bone or wood)}{6 cp}{12 cp}{1 kg}{2 kg}{4 kg}\\
\InferiorWeapon{(stone)}{6 cp}{12 cp}{4 kg}{8 kg}{16 kg}\\
Morningstar
	& 8 gp & 16 gp & 1d6 & 1d8 & 2d6 & $\times2$ & & 1.5 kg & 3 kg & 6 kg & Blud. and pierc.\\
\InferiorWeapon{(bone or wood)}{4 cp}{8 cp}{0.75 kg}{1.5 kg}{3 kg}\\
\InferiorWeapon{(stone)}{4 cp}{8 cp}{3 kg}{6 kg}{12 kg}%\\

% \rowcolor{white}
% \multicolumn{12}{l}{$\diamond$ Inferior material weapons suffer -1 penalty to attack and damage rolls.}\\
}

\WeaponTable{Metal Martial Weapons}{
\WeaponType{Light Melee Weapons}\\
Axe, throwing
	& 8 gp & 16 gp & 1d4 & 1d6 & 1d6 & $\times2$ & 3 m & 0.5 kg & 1 kg & 2 kg & Slashing\\
\InferiorWeapon{(bone or wood)}{4 cp}{8 cp}{0.25 kg}{0.5 kg}{1 kg}\\
\InferiorWeapon{(stone)}{4 cp}{8 cp}{1 kg}{2 kg}{4 kg}\\
Hammer, light
	& 1 gp & 2 gp & 1d3 & 1d4 & 1d4 & $\times2$ & 6 m & 0.5 kg & 1 kg & 2 kg & Bludgeoning\\
\InferiorWeapon{(bone or wood)}{5 bits}{1 cp}{0.25 kg}{0.5 kg}{1 kg}\\
\InferiorWeapon{(stone)}{5 bits}{1 cp}{1 kg}{2 kg}{4 kg}\\
Handaxe
	& 6 gp & 12 gp & 1d4 & 1d6 & 1d6 & $\times3$ & & 0.75 kg & 1.5 kg & 3 kg & Slashing\\
\InferiorWeapon{(bone or wood)}{3 cp}{6 cp}{0.37 kg}{0.75 kg}{1.5 kg}\\
\InferiorWeapon{(stone)}{3 cp}{6 cp}{1.5 kg}{3 kg}{6 kg}\\
Kukri
	& 8 gp & 16 gp & 1d3 & 1d4 & 1d6 & 18--20/$\times2$ & & 0.5 kg & 1 kg & 2 kg & Slashing\\
\InferiorWeapon{(bone or wood)}{4 cp}{8 cp}{0.25 kg}{0.5 kg}{1 kg}\\
\InferiorWeapon{(stone)}{4 cp}{8 cp}{1 kg}{2 kg}{4 kg}\\
Pick, light
	& 4 gp & 8 gp & 1d3 & 1d4 & 1d6 & $\times4$ & & 0.75 kg & 1.5 kg & 3 kg & Piercing\\
\InferiorWeapon{(bone or wood)}{2 cp}{4 cp}{0.37 kg}{0.75 kg}{1.5 kg}\\
\InferiorWeapon{(stone)}{2 cp}{4 cp}{1.5 kg}{3 kg}{6 kg}\\
Sword, short
	& 10 gp & 20 gp & 1d4 & 1d6 & 1d8 & 19--20/$\times2$ & & 0.5 kg & 1 kg & 2 kg & Piercing\\
\InferiorWeapon{(bone or wood)}{5 cp}{10 cp}{0.25 kg}{0.5 kg}{1 kg}\\
\InferiorWeapon{(stone)}{5 cp}{10 cp}{1 kg}{2 kg}{4 kg}\\

\WeaponType{One-Handed Melee Weapons}\\
Battleaxe
	& 10 gp & 20 gp & 1d6 & 1d8 & 2d6 & $\times3$ & & 1.5 kg & 3 kg & 6 kg & Slashing\\
\InferiorWeapon{(bone or wood)}{5 cp}{10 cp}{0.75 kg}{1.5 kg}{3 kg}\\
\InferiorWeapon{(stone)}{5 cp}{10 cp}{3 kg}{6 kg}{12 kg}\\
Flail
	& 8 gp & 16 gp & 1d6 & 1d8 & 2d6 & $\times2$ & & 1.25 kg & 2.5 kg & 5 kg & Bludgeoning\\
\InferiorWeapon{(bone or wood)}{4 cp}{8 cp}{0.62 kg}{1.25 kg}{2.5 kg}\\
\InferiorWeapon{(stone)}{4 cp}{8 cp}{2.5 kg}{5 kg}{10 kg}\\
Longsword
	& 15 gp & 30 gp & 1d6 & 1d8 & 2d6 & 19--20/$\times2$ & & 1 kg & 2 kg & 4 kg & Slashing\\
\InferiorWeapon{(bone or wood)}{75 bits}{15 cp}{0.5 kg}{1 kg}{2 kg}\\
\InferiorWeapon{(stone)}{75 bits}{15 cp}{2 kg}{4 kg}{8 kg}\\
Pick, heavy
	& 8 gp & 16 gp & 1d4 & 1d6 & 1d8 & $\times4$ & & 1.5 kg & 3 kg & 6 kg & Piercing\\
\InferiorWeapon{(bone or wood)}{4 cp}{8 cp}{0.75 kg}{1.5 kg}{3 kg}\\
\InferiorWeapon{(stone)}{4 cp}{8 cp}{3 kg}{6 kg}{12 kg}\\
Rapier
	& 20 gp & 40 gp & 1d4 & 1d6 & 1d8 & 18--20/$\times2$ & & 0.5 kg & 1 kg & 2 kg & Piercing\\
\InferiorWeapon{(bone or wood)}{10 cp}{20 cp}{0.25 kg}{0.5 kg}{1 kg}\\
\InferiorWeapon{(stone)}{10 cp}{20 cp}{1 kg}{2 kg}{4 kg}\\
Scimitar
	& 15 gp & 30 gp & 1d4 & 1d6 & 1d8 & 18--20/$\times2$ & & 1 kg & 2 kg & 4 kg & Slashing\\
\InferiorWeapon{(bone or wood)}{75 bits}{15 cp}{0.5 kg}{1 kg}{2 kg}\\
\InferiorWeapon{(stone)}{75 bits}{15 cp}{2 kg}{4 kg}{8 kg}\\
Trident
	& 15 gp & 30 gp & 1d6 & 1d8 & 2d6 & $\times2$ & 3 m & 1 kg & 2 kg & 4 kg & Piercing\\
\InferiorWeapon{(bone or wood)}{75 bits}{15 cp}{0.5 kg}{1 kg}{2 kg}\\
\InferiorWeapon{(stone)}{75 bits}{15 cp}{2 kg}{4 kg}{8 kg}\\
Warhammer
	& 12 gp & 24 gp & 1d6 & 1d8 & 2d6 & $\times3$ & & 1.25 kg & 2.5 kg & 5 kg & Bludgeoning\\
\InferiorWeapon{(bone or wood)}{6 cp}{12 cp}{0.62 kg}{1.25 kg}{2.5 kg}\\
\InferiorWeapon{(stone)}{6 cp}{12 cp}{2.5 kg}{5 kg}{10 kg}\\

\rowcolor{white}
\multicolumn{12}{l}{$\diamond$ Inferior material weapons suffer -1 penalty to attack and damage rolls.}\\
}

\WeaponTable{Metal Martial Weapons}{
\WeaponType{Two-Handed Melee Weapons}\\
Falchion
	& 75 gp & 150 gp & 1d6 & 2d4 & 2d6 & 18--20/$\times2$ & & 2 kg & 4 kg & 8 kg & Slashing\\
\InferiorWeapon{(bone or wood)}{32.5 cp}{75 cp}{1 kg}{2 kg}{4 kg}\\
\InferiorWeapon{(stone)}{32.5 cp}{75 cp}{4 kg}{8 kg}{16 kg}\\
Glaive\footnotemark[1]
	& 8 gp & 16 gp & 1d8 & 1d10 & 2d8 & $\times3$ & & 2.5 kg & 5 kg & 10 kg & Slashing\\
\InferiorWeapon{(bone or wood)}{4 cp}{8 cp}{1.25 kg}{2.5 kg}{5 kg}\\
\InferiorWeapon{(stone)}{4 cp}{8 cp}{5 kg}{10 kg}{20 kg}\\
Greataxe
	& 20 gp & 40 gp & 1d10 & 1d12 & 3d6 & $\times3$ & & 3 kg & 6 kg & 12 kg & Slashing\\
\InferiorWeapon{(bone or wood)}{10 cp}{20 cp}{1.5 kg}{3 kg}{6 kg}\\
\InferiorWeapon{(stone)}{10 cp}{20 cp}{6 kg}{12 kg}{24 kg}\\
Flail, heavy
	& 15 gp & 30 gp & 1d8 & 1d10 & 2d8 & 19--20/$\times2$ & & 2.5 kg & 5 kg & 10 kg & Bludgeoning\\
\InferiorWeapon{(bone or wood)}{75 bits}{15 cp}{1.25 kg}{2.5 kg}{5 kg}\\
\InferiorWeapon{(stone)}{75 bits}{15 cp}{5 kg}{10 kg}{20 kg}\\
Greatsword
	& 50 gp & 100 gp & 1d10 & 2d6 & 3d6 & 19--20/$\times2$ & & 2 kg & 4 kg & 8 kg & Slashing\\
\InferiorWeapon{(bone or wood)}{25 cp}{50 cp}{1 kg}{2 kg}{4 kg}\\
\InferiorWeapon{(stone)}{25 cp}{50 cp}{4 kg}{8 kg}{16 kg}\\
Guisarme\footnotemark[1]
	& 45 sp & 9 gp & 1d6 & 2d4 & 2d6 & $\times3$ & & 3 kg & 6 kg & 12 kg & Slashing\\
\InferiorWeapon{(bone or wood)}{25 bits}{45 bits}{0.25 kg}{0.5 kg}{1 kg}\\
\InferiorWeapon{(stone)}{25 bits}{45 bits}{1 kg}{2 kg}{4 kg}\\
Halberd
	& 10 gp & 20 gp & 1d8 & 1d10 & 2d8 & $\times3$ & & 3 kg & 6 kg & 12 kg & Pierc. or slash.\\
\InferiorWeapon{(bone or wood)}{5 cp}{10 cp}{1.5 kg}{3 kg}{6 kg}\\
\InferiorWeapon{(stone)}{5 cp}{10 cp}{6 kg}{12 kg}{24 kg}\\
Ranseur\footnotemark[1]
	& 10 gp & 20 gp & 1d6 & 2d4 & 2d6 & $\times3$ & & 3 kg & 6 kg & 12 kg & Piercing\\
\InferiorWeapon{(bone or wood)}{5 cp}{10 cp}{1.5 kg}{3 kg}{6 kg}\\
\InferiorWeapon{(stone)}{5 cp}{10 cp}{6 kg}{12 kg}{24 kg}\\
Scythe
	& 18 gp & 36 gp & 1d6 & 2d4 & 2d6 & $\times4$ & & 2.5 kg & 5 kg & 10 kg & Pierc. or slash.\\
\InferiorWeapon{(bone or wood)}{9 cp}{18 cp}{1.25 kg}{2.5 kg}{5 kg}\\
\InferiorWeapon{(stone)}{9 cp}{18 cp}{5 kg}{10 kg}{20 kg}\\

\WeaponType{Ranged Weapons}\\
\Projectile{Arrows, metal (20)}{1 gp}{2 gp}{0.75 kg}{1.5 kg}{3 kg}\\


% \rowcolor{white}
% \multicolumn{12}{l}{1 Reach weapon}\\
% \rowcolor{white}
% \multicolumn{12}{l}{$\diamond$ Inferior material weapons suffer -1 penalty to attack and damage rolls.}\\
}

\WeaponTable{Metal Exotic Weapons}{
\WeaponType{Light Melee Weapons}\\
Kama
	& 2 gp & 4 gp & 1d4 & 1d6 & 1d8 & $\times2$ & & 0.5 kg & 1 kg & 2 kg & Slashing\\
\InferiorWeapon{(bone or wood)}{1 cp}{2 cp}{0.25 kg}{0.5 kg}{1 kg}\\
\InferiorWeapon{(stone)}{1 cp}{2 cp}{1 kg}{2 kg}{4 kg}\\
\WeaponType{One-Handed Melee Weapons}\\
Sword, bastard
	& 35 gp & 70 gp & 1d8 & 1d10 & 2d8 & 19--20/$\times2$ & & 1.5 kg & 3 kg & 6 kg & Slashing\\
\InferiorWeapon{(bone or wood)}{18 cp}{35 cp}{0.75 kg}{1.5 kg}{3 kg}\\
\InferiorWeapon{(stone)}{18 cp}{35 cp}{3 kg}{6 kg}{12 kg}\\
Waraxe, dwarven
	& 30 gp & 60 gp & 1d8 & 1d10 & 2d8 & $\times3$ & & 2 kg & 4 kg & 8 kg & Slashing\\
\InferiorWeapon{(bone or wood)}{15 cp}{30 cp}{1 kg}{2 kg}{4 kg}\\
\InferiorWeapon{(stone)}{15 cp}{30 cp}{4 kg}{8 kg}{16 kg}\\

\WeaponType{Two-Handed Melee Weapons}\\
Chain, spiked\footnotemark[1]
	& 25 gp & 50 gp & 1d6 & 2d4 & 2d6 & $\times2$ & & 2.5 kg & 5 kg & 10 kg & Piercing\\
\InferiorWeapon{(bone or wood)}{13 cp}{25 cp}{1.25 kg}{2.5 kg}{5 kg}\\
\InferiorWeapon{(stone)}{13 cp}{25 cp}{5 kg}{10 kg}{20 kg}\\
Flail, dire
	& 90 gp & 180 gp & 1d6 1d6 & 1d8 1d8 & 2d6 2d6 & $\times2$ & & 2.5 kg & 5 kg & 10 kg & Bludgeoning\\
\InferiorWeapon{(bone or wood)}{45 cp}{90 cp}{1.25 kg}{2.5 kg}{5 kg}\\
\InferiorWeapon{(stone)}{45 cp}{90 cp}{5 kg}{10 kg}{20 kg}\\
Sword, two-bladed\footnotemark[2]
	& 100 gp & 200 gp & 1d6 1d6 & 1d8 1d8 & 2d6 2d6 & 19--20/$\times2$ & & 2.5 kg & 5 kg & 10 kg & Slashing\\
\InferiorWeapon{(bone or wood)}{45 cp}{90 cp}{1.25 kg}{2.5 kg}{5 kg}\\
\InferiorWeapon{(stone)}{45 cp}{90 cp}{5 kg}{10 kg}{20 kg}\\
Urgrosh, dwarven\footnotemark[2]
	& 50 gp & 100 gp & 1d6 1d4 & 1d8 1d6 & 2d6 1d8 & $\times3$ & & 3 kg & 6 kg & 12 kg & Slashing Piercing\\
\InferiorWeapon{(bone or wood)}{25 cp}{50 cp}{1.5 kg}{3 kg}{6 kg}\\
\InferiorWeapon{(stone)}{25 cp}{50 cp}{6 kg}{12 kg}{24 kg}\\

\WeaponType{Ranged Weapons}\\
Shuriken (5)
	& 1 gp & 2 gp & 1 & 1d2 & 1d3 & $\times2$ & 3 m & 0.25 kg & 0.5 kg & 1 kg & Piercing\\
\InferiorWeapon{(bone or wood)}{5 bits}{1 cp}{0.12 kg}{0.25 kg}{0.5 kg}\\
\InferiorWeapon{(stone)}{5 bits}{1 cp}{0.5 kg}{1 kg}{2 kg}\\

\rowcolor{white}
\multicolumn{12}{l}{1 Reach weapon}\\
\rowcolor{white}
\multicolumn{12}{l}{2 Double weapon}\\
\rowcolor{white}
\multicolumn{12}{l}{$\diamond$ Inferior material weapons suffer -1 penalty to attack and damage rolls.}\\
}



\subsection{Weapon Qualities}
Here is the format for weapon entries (given as column headings on each weapon table).

\textbf{Cost:} This value is the weapon's cost in gold pieces (gp), silver pieces (sp), ceramic pieces (cp) or ceramic bits (bits). The cost includes miscellaneous gear that goes with the weapon.

This cost is the same for a Small or Medium version of the weapon. A Large version costs twice the listed price.

\textbf{Damage:} The Damage columns give the damage dealt by the weapon on a successful hit. There are three columns for different weapon sizes: ``S'' for Small weapons, ``M'' for Medium weapons, and ``L'' for Large weapons. If two damage ranges are given then the weapon is a double weapon. Use the second damage figure given for the double weapon's extra attack. \tabref{Larger and Smaller Weapon Damage} gives weapon damage values for weapons of various sizes.

\textbf{Critical:} The entry in this column notes how the weapon is used with the rules for critical hits. When your character scores a critical hit, roll the damage two, three, or four times, as indicated by its critical multiplier (using all applicable modifiers on each roll), and add all the results together.

\textit{Exception:} Extra damage over and above a weapon's normal damage is not multiplied when you score a critical hit.

\textit{×2:} The weapon deals double damage on a critical hit.

\textit{×3:} The weapon deals triple damage on a critical hit.

\textit{×4:} The weapon deals quadruple damage on a critical hit.

\textit{19--20/×2:} The weapon scores a threat on a natural roll of 19 or 20 (instead of just 20) and deals double damage on a critical hit. (The weapon has a threat range of 19--20.)

\textit{18--20/×2:} The weapon scores a threat on a natural roll of 18, 19, or 20 (instead of just 20) and deals double damage on a critical hit. (The weapon has a threat range of 18--20.)

\textbf{Range Increment:} Any attack at less than this distance is not penalized for range. However, each full range increment imposes a cumulative -2 penalty on the attack roll. A thrown weapon has a maximum range of five range increments. A projectile weapon can shoot out to ten range increments.

\textbf{Weight:} There are three columns for the weight of a weapon: ``S'' for Small weapons, ``M'' for Medium weapons, and ``L'' for Large weapons.

\textbf{Type:} Weapons are classified according to the type of damage they deal: bludgeoning, piercing, or slashing. Some monsters may be resistant or immune to attacks from certain types of weapons.

Some weapons deal damage of multiple types. If a weapon is of two types, the damage it deals is not half one type and half another; all of it is both types. Therefore, a creature would have to be immune to both types of damage to ignore any of the damage from such a weapon.

In other cases, a weapon can deal either of two types of damage. In a situation when the damage type is significant, the wielder can choose which type of damage to deal with such a weapon.

\textbf{Special:} Some weapons have special features. See the weapon descriptions for details.

\subsection{Weapon Descriptions}
Weapons found on \tabref{Weapons} that have special options for the wielder ("you") are described below. Splash weapons are described under Special Substances and Items.

\textbf{Alak:} An alak consists of a 60 centemeters long shaft of bone or wood, with four serrated bones tied to the sharp end, like the four prongs of a grappling hook.

When using an alak, you get a +2 bonus on the opposed attack roll when attempting to disarm an opponent (including the roll to avoid being disarmed if you fail to disarm your opponent). 

\textbf{Alhulak:} The alhulak consists of an alak tied to a 1.5 meter long leather cord, which wraps around your wrist at the other end. An alhulak has reach. You can strike opponents 3 meters away with it. In you getition, you can use it against an adjacent foe.

When using an alhulak, you get a +2 bonus on the opposed attack roll when attempting to disarm an opponent (including the roll to avoid being disarmed if the character fails to disarm his or her opponent). 

\textbf{Arrows:} An arrow used as a melee weapon is treated as a light improvised weapon (-4 penalty on attack rolls) and deals damage as a dagger of its size (critical multiplier ×2). Arrows come in a leather quiver that holds 20 arrows. An arrow that hits its target is destroyed; one that misses has a 50\% chance of being destroyed or lost. 

\textbf{Atlatl:} The atlatl, sometimes called a ``staff-sling,'' is a javelin-throwing device that is swung over the shoulder, using both hands. Javelins flung with an atlatl gain greater range than those thrown by hand. 

\textbf{Bard's Friend:} This weapon is crafted with several obsidian blades and wooden prongs, which are fastened to a handle. Several small spikes jut out from where the knuckles hold the weapon. Bards are known for smearing these spikes with injury poison. The bard's friend can be coated with three charges of poison, but only one may be delivered per attack made with the weapon. 

\textbf{Blowgun, Greater:} The greater blowgun fires blowgun darts, which are slightly smaller than thrown darts, and are capable of delivering poison as well. 

\textbf{Blowgun:} The blowgun is a long tube through which you blow air to fire needles. The needles don't deal much damage, but are often coated in poison. 

\textbf{Bolas:} You can use this weapon to make a ranged trip attack against an opponent. You can't be tripped during your own trip attempt when using a set of bolas. 

\textbf{Bolts:} A crossbow bolt used as a melee weapon is treated as a light improvised weapon (-4 penalty on attack rolls) and deals damage as a dagger of its size (crit ×2). Bolts come in a wooden case that holds 10 bolts (or 5, for a repeating crossbow). A bolt that hits its target is destroyed; one that misses has a 50\% chance of being destroyed or lost. 

\textbf{Bullets, Sling:} Bullets come in a leather pouch that holds 10 bullets. A bullet that hits its target is destroyed; one that misses has a 50\% chance of being destroyed or lost. 

\textbf{Cahulak:} A cahulak consists of two alaks (see above) joined by a 1.5 meter rope. You may fight as if fighting with two weapons, but if you do, you incur all the normal attack penalties associated with fighting with a light offhand weapon. A creature using a double weapon in one hand, such as a half-giant using a set of cahulaks can't use it as a double weapon.

When using a cahulak, you get a +2 bonus on the opposed attack roll when attempting to disarm an opponent (including the roll to avoid being disarmed if the character fails to disarm his or her opponent).

Because the cahulak can wrap around an enemy's leg or other limb, you can make trip attacks with it. If you are tripped during your own trip attempt, you can drop the cahulak to avoid being tripped.

If you strike at an opponent 3 meters away, you cannot use the cahulak as a double weapon unless you possess natural reach. 

\textbf{Carrikal:} The sharpened jawbone of a large creature is lashed to a haft. The jagged edges are sharpened, forming a sort of battleaxe with two forward-facing heads.

\textbf{Chain, Spiked:} A spiked chain has reach, so you can strike opponents 3 meters away with it. In addition, unlike most other weapons with reach, it can be used against an adjacent foe.

You can make trip attacks with the chain. If you are tripped during your own trip attempt, you can drop the chain to avoid being tripped.

When using a spiked chain, you get a +2 bonus on opposed attack rolls made to disarm an opponent (including the roll to avoid being disarmed if such an attempt fails).

You can use the \feat{Weapon Finesse} feat to apply your Dexterity modifier instead of your Strength modifier to attack rolls with a spiked chain sized for you, even though it isn't a light weapon for you. 

\textbf{Chatkcha:} The chatkcha returns to a proficient thrower on a missed attack roll. To catch it, the character must make an attack roll against AC 10 using the same bonus they threw the chatkcha with. Failure indicates the weapon falls to the ground 3 meters in a random direction from the thrower. Catching the chatkcha is part of the attack and does not count as a separate attack.

\textbf{Crossbow, Fixed:} This version of the crossbow can be fired by any capable of using it, but cannot be carried like a conventional crossbow. It is fixed in place, i.e. mounted on top of a wall, pole, or vehicle, and swivels so that you can aim the shot. Crossbows at the edge of a caravan, cart, or wall tend to offer cover, but limit your range of firing to a cone shape directly in front of the weapon.

It is possible to mount a fixed crossbow on top of a pole but inside a shallow pit, giving you a 360-degree range of motion, while giving you cover. In any case, it is impossible to swivel a fixed crossbow in order to attack upwards (your upward angle is limited to 45 degrees). Reloading a fixed crossbow is a full-round action.

\textbf{Crossbow, Hand:} You can draw a hand crossbow back by hand. Loading a hand crossbow is a move action that provokes attacks of opportunity.

You can shoot, but not load, a hand crossbow with one hand at no penalty. You can shoot a hand crossbow with each hand, but you take a penalty on attack rolls as if attacking with two light weapons. 

\textbf{Crossbow, Heavy:} You draw a heavy crossbow back by turning a small winch. Loading a heavy crossbow is a full-round action that provokes attacks of opportunity.

Normally, operating a heavy crossbow requires two hands. However, you can shoot, but not load, a heavy crossbow with one hand at a -4 penalty on attack rolls. You can shoot a heavy crossbow with each hand, but you take a penalty on attack rolls as if attacking with two one-handed weapons. This penalty is cumulative with the penalty for one-handed firing. 

\textbf{Crossbow, Light:} You draw a light crossbow back by pulling a lever. Loading a light crossbow is a move action that provokes attacks of opportunity.

Normally, operating a light crossbow requires two hands. However, you can shoot, but not load, a light crossbow with one hand at a -2 penalty on attack rolls. You can shoot a light crossbow with each hand, but you take a penalty on attack rolls as if attacking with two light weapons. This penalty is cumulative with the penalty for one-handed firing. 

\textbf{Crossbow, Repeating:} The repeating crossbow (whether heavy or light) holds 5 crossbow bolts. As long as it holds bolts, you can reload it by pulling the reloading lever (a free action). Loading a new case of 5 bolts is a full-round action that provokes attacks of opportunity.

You can fire a repeating crossbow with one hand or fire a repeating crossbow in each hand in the same manner as you would a normal crossbow of the same size. However, you must fire the weapon with two hands in order to use the reloading lever, and you must use two hands to load a new case of bolts. 

\textbf{Crusher:} The crusher is made from a large stone or metal weight, mounted at the end of a 4.5 meters long shaft of springy wood. The weight is whipped back and forth. The crusher is a reach weapon. You can strike opponents 3 meters away with it, but you cannot use it against an adjacent foe. You need a 4.5 meters ceiling to use the weapon, but it can reach over cover.

Crushers come in two varieties, fixed and free. A fixed crusher requires a base to use. The fixed crusher's base is enormously heavy, usually consisting of a thick slab of stone with a hole drilled through it to support the crusher's pole. The base is transported separately from the pole, and it takes one full minute to set the fixed crusher up for battle. The fixed crusher is a martial weapon, finding most use in infantry units.

It is possible to use the crusher pole without the base as a free crusher, but this requires considerable expertise. You need an exotic weapon proficiency in the free crusher to accomplish this feat without the -4 proficiency penalty, even if you are proficient in the fixed crusher.

\textbf{Dagger:} You get a +2 bonus on Sleight of Hand checks made to conceal a dagger on your body (see the Sleight of Hand skill). 

\textbf{Datchi Club:} A datchi club has reach. You can strike opponents 3 meters away with it, but you cannot use it against an adjacent foe. This weapon, generally found in the arenas, is made by affixing a 1.2--1.5 meter length of dried insect hive or roots to a 90 centimeters long shaft. Teeth, claws, or obsidian shards are embedded into the head of the weapon.

\textbf{Dejada:} The dejada allows the wielder to throw pelota (see the pelota description for details). These pelotas deal more damage than those thrown by hand, due to the great speed at which they are thrown from a dejada.

\textbf{Dragon's Paw:} Popular in the arenas, the dragon's paw consists of a 1.5--1.8 meter long pole, with a blade on either end. A basket guards your hands from attack, granting a +2 bonus on all attempts to defend against being disarmed.

A dragon's paw is a double weapon. You may fight as if fighting with two weapons, but if you do, you incur all the normal attack penalties associated with fighting with a light off-hand weapon. A creature using a double weapon in one hand, such as a half-giant using a dragon's paw can't use it as a double weapon.

\textbf{Flail or Heavy Flail:} With a flail, you get a +2 bonus on opposed attack rolls made to disarm an enemy (including the roll to avoid being disarmed if such an attempt fails).

You can also use this weapon to make trip attacks. If you are tripped during your own trip attempt, you can drop the flail to avoid being tripped. 

\textbf{Flail, Dire:} A dire flail is a double weapon. You can fight with it as if fighting with two weapons, but if you do, you incur all the normal attack penalties associated with fighting with two weapons, just as if you were using a one-handed weapon and a light weapon. A creature wielding a dire flail in one hand can't use it as a double weapon---only one end of the weapon can be used in any given round.

When using a dire flail, you get a +2 bonus on opposed attack rolls made to disarm an enemy (including the opposed attack roll to avoid being disarmed if such an attempt fails).

You can also use this weapon to make trip attacks. If you are tripped during your own trip attempt, you can drop the dire flail to avoid being tripped. 

\textbf{Forearm Axe:} Strapped to the forearm like a buckler, the forearm axe resembles a double-headed battleaxe, with the wearer's arm serving as the haft of the axe. You may continue to use your hand normally, but you cannot attack with the forearm axe and a wielded weapon in the same hand in one round. Your opponent cannot use a disarm action to disarm you of a forearm axe.

\textbf{Garrote, Bard's:} This exotic weapon is made from giant hair. A bard's garrote can only be used as part of a grapple attack, and you must wield it with both hands regardless of your size. As part of a grapple attack, using a garrote subjects you to attacks of opportunity and all other limitations described in the grappling rules, except that as follows: The garrote inflicts 2d4 points of nonlethal damage plus 1.5 times your Strength bonus. You can use a bard's garrote to deliver a coup de grace.

\textbf{Gauntlet, Spiked:} Your opponent cannot use a disarm action to disarm you of spiked gauntlets. The cost and weight given are for a single gauntlet. An attack with a spiked gauntlet is considered an armed attack. 

\textbf{Gauntlet:} This metal glove lets you deal lethal damage rather than nonlethal damage with unarmed strikes. A strike with a gauntlet is otherwise considered an unarmed attack. The cost and weight given are for a single gauntlet. Medium and heavy armors (except breastplate) come with gauntlets. 

\textbf{Glaive:} A glaive has reach. You can strike opponents 3 meters away with it, but you can't use it against an adjacent foe. 

\textbf{Gouge:} Worn in an over-the-shoulder harness, the gouge is commonly found in the Nibenese infantry. A wide blade of bone, obsidian or chitin is mounted to a 90 centimeters long shaft of wood. Your opponent cannot use a disarm action to disarm you of a gouge while you are wearing the harness. Donning the harness is a full-round action. Removing it is a move action.

\textbf{Guisarme:} A guisarme has reach. You can strike opponents 3 meters away with it, but you can't use it against an adjacent foe.  You can also use it to make trip attacks. If you are tripped during your own trip attempt, you can drop the guisarme to avoid being tripped. 

\textbf{Gythka:} A gythka is a double weapon. You may fight as if fighting with two weapons, but if you do, you incur all the normal attack penalties associated with fighting with a light off-hand weapon. A creature using a double weapon in one hand, such as a half-giant using a gythka can't use it as a double weapon.

\textbf{Halberd:} If you use a ready action to set a halberd against a charge, you deal double damage on a successful hit against a charging character.

You can use a halberd to make trip attacks. If you are tripped during your own trip attempt, you can drop the halberd to avoid being tripped. 

\textbf{Handfork:} The handfork, most popular among tareks, is a slicing weapon with a handle-grip and obsidian blades that join above the knuckles in an ``M'' shape.

\textbf{Heartpick:} The name of this weapon expresses its simple intent. Usually made of bone, the heartpick is a hammer like weapon with a serrated pick on the front, and a heavy, flat head on the back.

\textbf{Impaler:} Like many Athasian weapons, the impaler was developed for the arenas. Two blades are mounted parallel to the end of a 1.2 meter long shaft, forming a bladed ‘T'. The impaler is swung horizontally or vertically with great force.

\textbf{Javelin:} Since it is not designed for melee, you are treated as nonproficient with it and take a -4 penalty on attack rolls if you use a javelin as a melee weapon. 

\textbf{Kama:} You can use a kama to make trip attacks. If you are tripped during your own trip attempt, you can drop the kama to avoid being tripped. 

\textbf{Ko:} The Ko combines a jagged blade that has been carved from a roughly oval stone. This exotic weapon of kreen manufacture is typically used in matching pairs. The ko is designed to pierce chitin, shells and tough skin. If a ko is used against a creature with natural armor, the attacker gets a +1 bonus to attack rolls.

\textbf{Kyorkcha:} The kyorkcha is a more dangerous variant of the chatkcha. This tohr-kreen weapon consists of a curved blade, much like a boomerang, with several protrusions along the edge, as well as jutting spikes near each end.

\textbf{Lajav:} The lajav is a kreen weapon designed to capture opponents. It incorporates two flattened bones, joined in a hinge about 60 centimeters from the end. The result looks something like a nutcracker, and is used roughly in the same crushing way.

If you hit an opponent at least one size category smaller than yourself with a lajav, you can immediately initiate a grapple (as a free action) without provoking an attack of opportunity.

Regardless of your size, you need two hands to use a lajav, since a second hand is required to catch the other end of the lajav. As with the gythka, kreen are able to wield two lajav at a time because of their four arms.

\textbf{Lance:} A lance deals double damage when used from the back of a charging mount. It has reach, so you can strike opponents 3 meters away with it, but you can't use it against an adjacent foe.

While mounted, you can wield a lance with one hand. 

\textbf{Lasso:} This weapon consists of a rope that you can throw and then draw closed. The total range of your lasso depends on the length of the rope. Throwing a lasso is a ranged touch attack. If you successfully hit your opponent, make a grapple check. If you succeed at the grapple check, then your opponent is grappled, and you can continue the grapple contest by continuing to pull on the rope. You can make trip attacks with a lasso against a grappling opponent. If you are tripped during your own trip attempt, you can drop the lasso to avoid being tripped.

\textbf{Longblade, elven:} You can use the \feat{Weapon Finesse} feat to apply your Dexterity modifier, rather than your Strength modifier, to all attack rolls made with the elven longblade.

\textbf{Longbow, Composite:} You need at least two hands to use a bow, regardless of its size. You can use a composite longbow while mounted. All composite bows are made with a particular strength rating (that is, each requires a minimum Strength modifier to use with proficiency). If your Strength bonus is less than the strength rating of the composite bow, you can't effectively use it, so you take a -2 penalty on attacks with it. The default composite longbow requires a Strength modifier of +0 or higher to use with proficiency. A composite longbow can be made with a high strength rating to take advantage of an above-average Strength score; this feature allows you to add your Strength bonus to damage, up to the maximum bonus indicated for the bow. Each point of Strength bonus granted by the bow adds 100 gp to its cost.

For purposes of weapon proficiency and similar feats, a composite longbow is treated as if it were a longbow. 

\textbf{Longbow:} You need at least two hands to use a bow, regardless of its size. A longbow is too unwieldy to use while you are mounted. If you have a penalty for low Strength, apply it to damage rolls when you use a longbow. If you have a bonus for high Strength, you can apply it to damage rolls when you use a composite longbow (see below) but not a regular longbow. 

\textbf{Longspear:} A longspear has reach. You can strike opponents 3 meters away with it, but you can't use it against an adjacent foe. If you use a ready action to set a longspear against a charge, you deal double damage on a successful hit against a charging character. 

\textbf{Lotulis:} Two barbed, crescent shaped blades adorn either end of the lotulis, a double weapon once popular in the arena of Tyr. You may fight as if fighting with two weapons, but if you do, you incur all the normal attack penalties associated with fighting with a light off-hand weapon. A creature using a double weapon in one hand, such as a half-giant using a lotulis can't use it as a double weapon.

\textbf{Macahuitl:} A macahuitl is a sword painstakingly crafted using a core of solid wood, with small, sharp shards of obsidian embedded into the wood to form an edge on two opposite sides of the weapon. These weapons are swung like the scimitar, though macahuitls tend to require more maintenance. The macahuitl is especially popular among the Draji, who seem to be the only ones who can easily pronounce this weapon's Draji name (``ma-ka-wheet-luh''). Non-Draji simply refer to it as the ``obsidian sword'' or the ``Draji sword.''

\textbf{Master's Whip:} The master's whip is usually braided from giant hair or leather, and has shards of chitin, obsidian or bone braided into the end of the whip. Unlike normal whips, the master's whip deals damage normally, has only a 3 meters range, and you apply your Strength modifier to damage dealt. In all other respects, it is treated as a normal whip.

\textbf{Maul:} A maul is effectively a very large sledgehammer that crushes opponents to death. This weapon is commonly used by dwarves, muls, half-giants and other creatures that value great strength.

\textbf{Mekillot Sap:} The mekillot sap is a soft but tough large leather bag filled with fine gravel or sand, stitched together with giant's hair, and tied to the end of a 1.5 meter rope. The throwing sap is swung overhead with both hands.

A mekillot sap has reach, so you can strike opponents 3 meters away with it. In addition, unlike other weapons with reach, you can grip the rope higher, and use the mekillot sap against an adjacent foe. You can make trip attacks with the mekillot sap. If you are tripped during your own trip attempt, you can drop the sap to avoid being tripped. You get a +2 bonus to your opposed Strength check when attempting to trip your opponent.

\textbf{Net:} A net is used to entangle enemies. When you throw a net, you make a ranged touch attack against your target. A net's maximum range is 3 meters. If you hit, the target is entangled. An entangled creature takes a -2 penalty on attack rolls and a -4 penalty on Dexterity, can move at only half speed, and cannot charge or run. If you control the trailing rope by succeeding on an opposed Strength check while holding it, the entangled creature can move only within the limits that the rope allows. If the entangled creature attempts to cast a spell, it must make a DC 15 Concentration check or be unable to cast the spell.

An entangled creature can escape with a DC 20 Escape Artist check (a full-round action). The net has 5 hit points and can be burst with a DC 25 Strength check (also a full-round action).

A net is useful only against creatures within one size category of you.

A net must be folded to be thrown effectively. The first time you throw your net in a fight, you make a normal ranged touch attack roll. After the net is unfolded, you take a -4 penalty on attack rolls with it. It takes 2 rounds for a proficient user to fold a net and twice that long for a nonproficient one to do so. 

\textbf{Nunchaku:} With a nunchaku, you get a +2 bonus on opposed attack rolls made to disarm an enemy (including the roll to avoid being disarmed if such an attempt fails). 

\textbf{Pelota, Hinged:} To the careless eye a hinged pelota looks like an ordinary pelota without obsidian spikes. Hinged pelota can be twisted open like a small jar. Bards and assassins often use this feature to insert a splash-globe---a thin crystal sphere that contains acid, injury poison, contact poison, alchemical fire, or some other liquid. When the pelota strikes, the globe breaks, spilling the liquid through the holes of the pelota. Like pelota, hinged pelota can be thrown with a dejada. Hinged pelotas are also used as ammunition for the splashbow.

\textbf{Pelota:} Popular in arena games and increasingly popular in the street games of some city-states, pelota are hollow leaden spheres with small holes that cause the sphere to whistle as it flies through the air. The surface of most pelota is studded with obsidian shards. You can use the dejada throwing glove to cast pelota at much higher speed and with greater accuracy, dealing more damage than a pelota thrown by hand.

\textbf{Puchik:} A bone or obsidian punching dagger.

\textbf{Quabone:} Four jawbones are fastened around a central haft, at right angles to one another. The quabone is often used in the arenas. The wounds it inflicts are non-lethal, yet have entertainment value, as the quabone tends to open up many small cuts that bleed freely---for a brief time.

\textbf{Quarterstaff:} A quarterstaff is a double weapon. You can fight with it as if fighting with two weapons, but if you do, you incur all the normal attack penalties associated with fighting with two weapons, just as if you were using a one-handed weapon and a light weapon. A creature wielding a quarterstaff in one hand can't use it as a double weapon---only one end of the weapon can be used in any given round. 

\textbf{Ranseur:} A ranseur has reach. You can strike opponents 3 meters away with it, but you can't use it against an adjacent foe.

With a ranseur, you get a +2 bonus on opposed attack rolls made to disarm an opponent (including the roll to avoid being disarmed if such an attempt fails). 

\textbf{Rapier:} You can use the \feat{Weapon Finesse} feat to apply your Dexterity modifier instead of your Strength modifier to attack rolls with a rapier sized for you, even though it isn't a light weapon for you. You can't wield a rapier in two hands in order to apply 1½ times your Strength bonus to damage. 

\textbf{Sai:} With a sai, you get a +4 bonus on opposed attack rolls made to disarm an enemy (including the roll to avoid being disarmed if such an attempt fails). 

\textbf{Scythe:} A scythe can be used to make trip attacks. If you are tripped during your own trip attempt, you can drop the scythe to avoid being tripped. 

\textbf{Shield, Heavy or Light or Long:} You can bash with a shield instead of using it for defense. See Armor for details. 

\textbf{Shortbow, Composite:} You need at least two hands to use a bow, regardless of its size. You can use a composite shortbow while mounted. All composite bows are made with a particular strength rating (that is, each requires a minimum Strength modifier to use with proficiency). If your Strength bonus is lower than the strength rating of the composite bow, you can't effectively use it, so you take a -2 penalty on attacks with it. The default composite shortbow requires a Strength modifier of +0 or higher to use with proficiency. A composite shortbow can be made with a high strength rating to take advantage of an above-average Strength score; this feature allows you to add your Strength bonus to damage, up to the maximum bonus indicated for the bow. Each point of Strength bonus granted by the bow adds 75 gp to its cost.

For purposes of weapon proficiency and similar feats, a composite shortbow is treated as if it were a shortbow. 

\textbf{Shortbow:} You need at least two hands to use a bow, regardless of its size. You can use a shortbow while mounted. If you have a penalty for low Strength, apply it to damage rolls when you use a shortbow. If you have a bonus for high Strength, you can apply it to damage rolls when you use a composite shortbow (see below) but not a regular shortbow. 

\textbf{Shortspear:} A shortspear is small enough to wield one-handed. It may also be thrown. 

\textbf{Shuriken:} A shuriken can't be used as a melee weapon. Although they are thrown weapons, shuriken are treated as ammunition for the purposes of drawing them, crafting masterwork or otherwise special versions of them and what happens to them after they are thrown. 

\textbf{Sickle:} A sickle can be used to make trip attacks. If you are tripped during your own trip attempt, you can drop the sickle to avoid being tripped. 

\textbf{Singing Stick:} A singing stick is a carefully crafted and polished club, often used in pairs. Singing sticks draw their name from the characteristic whistling sound they make when used. A character proficient with singing sticks may use a pair of singing sticks as if he had the Two-Weapon Fighting feat. In the hands of a nonproficient character, singing sticks are nothing more than light clubs.

\textbf{Skyhammer:} The sky hammer consists of a 3 meter length of rope with a large hammer-like object at one end. Its rope is coiled and swung around the body two-handedly until enough momentum is gained to hurl the hammer at a target. A successful hit grants a free trip attempt, and you receive a +4 bonus to your opposed Strength roll due to the momentum of the skyhammer.

\textbf{Sling:} Your Strength modifier applies to damage rolls when you use a sling, just as it does for thrown weapons. You can fire, but not load, a sling with one hand. Loading a sling is a move action that requires two hands and provokes attacks of opportunity.  You can hurl ordinary stones with a sling, but stones are not as dense or as round as bullets. Thus, such an attack deals damage as if the weapon were designed for a creature one size category smaller than you and you take a -1 penalty on attack rolls. 

\textbf{Slodak:} The slodak is a wooden short sword, carved from young hardwood trees and treated with a mixture of tree sap and id fiend blood. This treatment renders the blade of the weapon extremely strong, making it a deadly weapon.

\textbf{Spear, Double-Tipped:} A double-tipped spear is a double weapon. You can fight with it as if fighting with two weapons, but if you do, you incur all the normal attack penalties associated with fighting with two weapons, just as if you were using a one-handed weapon and a light weapon. A creature wielding a double-tipped spear in one hand can't use it as a double weapon---only one end of the weapon can be used in any given round.

\textbf{Spear:} A spear can be thrown. If you use a ready action to set a spear against a charge, you deal double damage on a successful hit against a charging character. 

\textbf{Spiked Armor:} You can outfit your armor with spikes, which can deal damage in a grapple or as a separate attack. See Armor for details. 

\textbf{Spiked Shield, Heavy or Light:} You can bash with a spiked shield instead of using it for defense. See Armor for details. 

\textbf{Splashbow:} This exotic weapon looks like a misshapen crossbow, only 90 centimeters long from bow to handle, but with a horizontal bow nearly 1.5 meter wide. Rather than bolts, the splashbow fires hinged pelotas, which can be filled with splash-globes of alchemical fire, contact poison, acids, or other interesting liquids. Splash-globes burst on impact, spraying their contents like a thrown grenade. The splashbow takes a full round to draw and load, assuming that the hinged pelotas have already been prepared.

\textbf{Swatter:} The swatter is a popular name for a half-giant weapon consisting of a heavy spiked club made from hardwood, with a bronze or lead core in the tip for added weight. The swatter got its name from the tales of a half- giant soldier who reputedly used a similar weapon to defeat an entire thri-kreen hunting party.

\textbf{Sword, Bastard:} A bastard sword is too large to use in one hand without special training; thus, it is an exotic weapon. A character can use a bastard sword two-handed as a martial weapon. 

\textbf{Sword, Two-Bladed:} A two-bladed sword is a double weapon. You can fight with it as if fighting with two weapons, but if you do, you incur all the normal attack penalties associated with fighting with two weapons, just as if you were using a one-handed weapon and a light weapon. A creature wielding a two-bladed sword in one hand can't use it as a double weapon---only one end of the weapon can be used in any given round. 

\textbf{Talid:} The talid, also known as the gladiator's gauntlet, is made of stiff leather with metal, chitin or bone plating on the hand cover and all along the forearm. Spikes protrude from each of the knuckles and along the back of the hand. A sharp blade runs along the thumb and there is a 6-inch spike on the elbow. A strike with a talid is considered an armed attack. The cost and weight given are for a single talid. An opponent cannot use a disarm action to disarm a character's talid.

\textbf{Thanak:} The thanak is a chopping weapon of pterran manufacture resembling a jagged sword or sawblade. It consists of a pair of hardwood strips bound together, with a row of pterrax teeth protruding from between them along one edge of the weapon particularly capable of slicing through muscle and sinew. On a critical hit, the thanak inflicts one point of Strength damage in addition to triple normal damage.

\textbf{Tkaesali:} This polearm, commonly used by the nikaal, consists of long wooden haft topped with a circular, jagged blade. A tkaesali has reach. You can strike opponents 3 meters away with it, but you can't use it against an adjacent foe.

\textbf{Tonfa:} The tonfa is a stick with a short handle, and is popular among street-patrolling Nibenese templars and their guards. You can deal nonlethal damage with a tonfa without taking the usual -4 penalty.

\textbf{Tortoise Blade:} The tortoise blade consists of a 30 centimeters dagger mounted to the center of a shell. The tortoise blade is strapped over the wearer's hand, preventing them from holding anything but the tortoise blade. The tortoise blade also functions as a buckler, granting a +1 armor bonus, inflicting a -1 armor check penalty and incurring a 5\% arcane spell failure chance. A masterwork tortoise blade either functions as a masterwork shield or a masterwork weapon (or both, for twice the normal masterwork cost).

\textbf{Trident:} This weapon can be thrown. If you use a ready action to set a trident against a charge, you deal double damage on a successful hit against a charging character. 

\textbf{Trikal:} Three blades project radially from the business end of a 1.8 meter long haft. A series of sharp serrated edges line the shaft below the 30 centimeters blades, while the far end of the weapon is weighted, in order to balance the weapon. Because of the trikal's curved blades on the top of the weapon, trip attacks can also be made with it. If a character is tripped during his or her trip attempt, the trikal can be dropped to avoid being tripped.

\textbf{Unarmed Strike:} A Medium character deals 1d3 points of nonlethal damage with an unarmed strike. A Small character deals 1d2 points of nonlethal damage. A monk or any character with the Improved Unarmed Strike feat can deal lethal or nonlethal damage with unarmed strikes, at her option. The damage from an unarmed strike is considered weapon damage for the purposes of effects that give you a bonus on weapon damage rolls.

An unarmed strike is always considered a light weapon. Therefore, you can use the \feat{Weapon Finesse} feat to apply your Dexterity modifier instead of your Strength modifier to attack rolls with an unarmed strike. 

\textbf{Urgrosh, Dwarven:} A dwarven urgrosh is a double weapon. You can fight with it as if fighting with two weapons, but if you do, you incur all the normal attack penalties associated with fighting with two weapons, just as if you were using a one-handed weapon and a light weapon. The urgrosh's axe head is a slashing weapon that deals 1d8 points of damage. Its spear head is a piercing weapon that deals 1d6 points of damage. You can use either head as the primary weapon. The other is the off-hand weapon. A creature wielding a dwarven urgrosh in one hand can't use it as a double weapon---only one end of the weapon can be used in any given round.

If you use a ready action to set an urgrosh against a charge, you deal double damage if you score a hit against a charging character. If you use an urgrosh against a charging character, the spear head is the part of the weapon that deals damage.

Dwarves treat dwarven urgroshes as martial weapons. 

\textbf{Waraxe, Dwarven:} A dwarven waraxe is too large to use in one hand without special training; thus, it is an exotic weapon. A Medium character can use a dwarven waraxe two-handed as a martial weapon, or a Large creature can use it one-handed in the same way. A dwarf treats a dwarven waraxe as a martial weapon even when using it in one hand. 

\textbf{Weighted Pike:} A solid head, generally stone or baked ceramic, is mounted on the end of a spear or a pike. A weighted pike is a double weapon. You may fight as if fighting with two weapons, but if you do, you incur all the normal attack penalties associated with fighting with a light off-hand weapon. A creature using a double weapon in one hand, such as a half-giant using a weighted pike can't use it as a double weapon.

\textbf{Whip:} A whip deals nonlethal damage. It deals no damage to any creature with an armor bonus of +1 or higher or a natural armor bonus of +3 or higher. The whip is treated as a melee weapon with 4.5 meters reach, though you don't threaten the area into which you can make an attack. In addition, unlike most other weapons with reach, you can use it against foes anywhere within your reach (including adjacent foes).

Using a whip provokes an attack of opportunity, just as if you had used a ranged weapon.

You can make trip attacks with a whip. If you are tripped during your own trip attempt, you can drop the whip to avoid being tripped.

When using a whip, you get a +2 bonus on opposed attack rolls made to disarm an opponent (including the roll to keep from being disarmed if the attack fails).

You can use the \feat{Weapon Finesse} feat to apply your Dexterity modifier instead of your Strength modifier to attack rolls with a whip sized for you, even though it isn't a light weapon for you. 

\textbf{Widow's Knife:} Two prongs are hidden within the hilt of a widow's knife. On a successful hit, you may trigger the prongs by releasing a catch in the hilt as a free action. The prongs do an additional 1d3 points of damage (1d2 for a Small widow's knife, 1d4 for a Large widow's knife) when sprung, and take a standard action to reload.

\textbf{Wrist Razor:} Several shards of obsidian or bone are fastened to a strip of leather or other binding material, or are lashed onto the forearm of the wielder. Wrist razors are hard to disarm, granting you a +2 bonus when opposing a disarm attempt.

\textbf{Zerka:} The zerka is a javelin with short barbs that cover 60 centimeters of the bone shaft. These barbs point away from the zerka's tip, causing the weapon's head to snag against its target's flesh and bone as it is removed. If a zerka hits, it lodges in the victim if he fails a Reflex save (DC equal to 5 + damage inflicted). A failed check means the zerka is stuck and the victim moves at half-speed, cannot charge or run, and must make a Concentration check (DC 10 + spell level) in order to cast a spell with somatic components. The victim can pull the zerka from his wound with a move action if he has at least one hand free, but suffers an additional 1d4 damage. A Heal check DC 13 allows the zerka to be removed without further injury.


\subsection{Masterwork Weapons}
A masterwork weapon is a finely crafted version of a normal weapon. Wielding it provides a +1 enhancement bonus on attack rolls.

You can't add the masterwork quality to a weapon after it is created; it must be crafted as a masterwork weapon (see the \skill{Craft} skill). The masterwork quality adds 300 cp to the cost of a normal weapon (or 6 cp to the cost of a single unit of ammunition). Adding the masterwork quality to a double weapon costs twice the normal increase (+600 cp).

Masterwork ammunition is damaged (effectively destroyed) when used. The enhancement bonus of masterwork ammunition does not stack with any enhancement bonus of the projectile weapon firing it.

All magic weapons are automatically considered to be of masterwork quality. The enhancement bonus granted by the masterwork quality doesn't stack with the enhancement bonus provided by the weapon's magic.

Even though some types of armor and shields can be used as weapons, you can't create a masterwork version of such an item that confers an enhancement bonus on attack rolls. Instead, masterwork armor and shields have lessened armor check penalties.

\section{Armor}

While Athasian characters use all the varieties of armor, the armor they use incorporates materials commonly found in the world around them. Though most of the armors are made using various parts of common Athasian animals, the armor construction process makes use of several different reinforcement methods developed over time. Many of the armors are highly composite, made using the pieces of several different animals---no two suits of armor look quite alike. Through the use of hardening resins, shaped chitin and stiff leather backings, Athasian armorers can craft remarkably durable armors from the material at hand.

Thousands of years of tortuous heat have lead Athasian armorers to develop ingenious air ventilation and air circulation methods. This allows medium and heavy armors to be worn in the Athasian heat.


\begin{table*}[t!]
\caption{\label{tab:Armor and Shields}Armor and Shields}
\rowcolors{0}{TableColor}{}
\scriptsize
\begin{tabularx}{\textwidth}{X r{9mm}r{11mm}z{7mm}z{9mm}z{8mm}z{9mm}z{7mm}z{7mm}z{7mm}r{11mm}r{11mm}r{11mm}}
	\rowcolor{white}
& \multicolumn{2}{c}{\tableheader Cost}
& \multirow[b]{2}{7mm}{\centering \tableheader AC Bonus}
& \multirow[b]{2}{1cm}{\centering \tableheader Max. Dex Bonus}
& \multirow[b]{2}{8mm}{\centering \tableheader Armor Penalty}
& \multirow[b]{2}{9mm}{\centering \tableheader Failure Chance}
& \multicolumn{3}{c}{\tableheader Speed}
& \multicolumn{3}{c}{\tableheader Weight}\\
\cmidrule[0.5pt]{2-3}
\cmidrule[0.5pt]{8-13}

\footnotesize \tableheader Armor & \tableheader S/M & \tableheader L & & & & & \tableheader 6 m & \tableheader 9 m & \tableheader 12 m & \tableheader S & \tableheader M & \tableheader L\\

\multicolumn{13}{l}{\textit{Light Armor}}\\
Padded & 5 cp & 10 cp & +1 & +8 & +0 & 5\% & 6 m & 9 m & 12 m & 2.5 kg & 5 kg & 10 kg\\
Leather & 10 cp & 20 cp & +2 & +6 & +0 & 10\% & 6 m & 9 m & 12 m & 3.75 kg & 7.5 kg & 15 kg\\
Studded leather & 25 cp & 50 cp & +3 & +5 & $-1$ & 15\% & 6 m & 9 m & 12 m & 5 kg & 10 kg & 20 kg\\
Gladiator, light & 50 cp & 100 cp & +3 & +6 & $-1$ & 15\% & 6 m & 9 m & 12 m & 3 kg & 6 kg & 12 kg\\
Caravan & 75 cp & 150 cp & +4 & +3 & $-2$ & 20\% & 6 m & 9 m & 12 m & 4 kg & 8 kg & 16 kg\\
Chitin & 100 cp & 200 cp & +4 & +4 & $-2$ & 20\% & 6 m & 9 m & 12 m & 6.25 kg & 12.5 kg & 25 kg\\

\multicolumn{13}{l}{\textit{Medium Armor}}\\
Hide & 15 cp & 30 cp & +3 & +4 & $-3$ & 20\% & 4.5 m & 6 m & 9 m & 6.25 kg & 12.5 kg & 25 kg\\
Scale mail & 50 cp & 100 cp & +4 & +3 & $-4$ & 25\% & 4.5 m & 6 m & 9 m & 7.5 kg & 15 kg & 30 kg\\
Gladiator, medium & 100 cp & 200 cp & +4 & +5 & $-3$ & 20\% & 4.5 m & 6 m & 9 m & 4 kg & 8 kg & 16 kg\\
Shell & 150 cp & 300 cp & +5 & +2 & $-5$ & 30\% & 4.5 m & 6 m & 9 m & 10 kg & 20 kg & 40 kg\\
Breastplate & 200 cp & 400 cp & +5 & +3 & $-4$ & 25\% & 4.5 m & 6 m & 9 m & 7.5 kg & 15 kg & 30 kg\\

\multicolumn{13}{l}{\textit{Heavy armor}\footnotemark[1]}\\
Chitin warsuit & 165 cp & 330 cp & +5 & +1 & $-5$ & 40\% & 4.5 m & 6 m & 9 m & 6.25 kg & 12.5 kg & 25 kg\\
Splint mail & 200 cp & 400 cp & +6 & +0 & $-7$ & 40\% & 4.5 m & 6 m & 9 m & 11.25 kg & 22.5 kg & 45 kg\\
Banded mail & 250 cp & 500 cp & +6 & +1 & $-6$ & 35\% & 4.5 m & 6 m & 9 m & 8.75 kg & 17.5 kg & 35 kg\\
Tyrian warsuit & 410 cp & 810 cp & +6 & +1 & $-5$ & 30\% & 4.5 m & 6 m & 9 m & 10 kg & 20 kg & 40 kg\\
Half-plate & 6 gp & 12 gp & +7 & +0 & $-7$ & 40\% & 4.5 m & 6 m & 9 m & 12.5 kg & 25 kg & 50 kg\\
Full plate & 15 gp & 30 gp & +8 & +1 & $-6$ & 35\% & 4.5 m & 6 m & 9 m & 12.5 kg & 25 kg & 50 kg\\

\multicolumn{13}{l}{\textit{Shields}}\\
Buckler & 15 cp & 30 cp & +1 && $-1$ & 5\% &&&& 1.25 kg & 2.5 kg & 5 kg\\
Shield, light wooden & 3 cp & 6 cp & +1 && $-1$ & 5\% &&&& 1.25 kg & 2.5 kg & 5 kg\\
Shield, light steel & 9 gp & 18 gp & +1 && $-1$ & 5\% &&&& 1.5 kg & 3 kg & 6 kg\\
Shield, heavy wooden & 7 cp & 14 cp & +2 && $-2$ & 15\% &&&& 2.5 kg & 5 kg & 10 kg\\
Shield, heavy steel & 20 gp & 40 gp & +2 && $-2$ & 15\% &&&& 3.75 kg & 7.5 kg & 15 kg\\
Shield, long & 20 cp & 40 cp & +3 && $-4$ & 20\% &&&& 2.5 kg & 5 kg & 10 kg\\
Shield, tower & 30 cp & 60 cp & +4\footnotemark[2] & +2 & $-10$ & 50\% &&&& 11.25 kg & 22.5 kg & 
45 kg\\

\multicolumn{13}{l}{\textit{Extras}}\\
Armor spikes & +50 cp & +100 cp &&&& &&&& +2.5 kg & +5 kg & +10 kg\\
Gauntlet, locked & 8 cp & 16 cp &&& $\star$ & $\star$ &&&& +1.25 kg & +2.5 kg & +5 kg\\
Shield spikes & +10 cp & +20 cp &&&& &&&& +1.25 kg & +2.5 kg & +5 kg\\
\rowcolor{white}
\multicolumn{13}{l}{1 When running in heavy armor, you move only triple your speed, not quadruple.}\\
\rowcolor{white}
\multicolumn{13}{l}{2 A tower shield can instead grant you cover.}\\
\rowcolor{white}
\multicolumn{13}{l}{$\star$ Hand not free to cast spells or employ skills.}\\
\end{tabularx}
\end{table*}

\subsection{Armor Qualities}
To wear heavier armor effectively, a character can select the Armor Proficiency feats, but most classes are automatically proficient with the armors that work best for them.

Armor and shields can take damage from some types of attacks.

Here is the format for armor entries (given as column headings on \tabref{Armor and Shields}, below).

\textbf{Cost:} The cost of the armor for Small, Medium, or Large humanoid creatures. See Armor for Unusual Creatures, below, for armor prices for other creatures.

\textbf{Armor/Shield Bonus:} Each armor grants an armor bonus to AC, while shields grant a shield bonus to AC. The armor bonus from a suit of armor doesn't stack with other effects or items that grant an armor bonus. Similarly, the shield bonus from a shield doesn't stack with other effects that grant a shield bonus.

\textbf{Maximum Dex Bonus:} This number is the maximum Dexterity bonus to AC that this type of armor allows. Heavier armors limit mobility, reducing the wearer's ability to dodge blows. This restriction doesn't affect any other Dexterity-related abilities.

Even if a character's Dexterity bonus to AC drops to 0 because of armor, this situation does not count as losing a Dexterity bonus to AC.

Your character's encumbrance (the amount of gear he or she carries) may also restrict the maximum Dexterity bonus that can be applied to his or her Armor Class.

\textit{Shields:} Shields do not affect a character's maximum Dexterity bonus.

\textbf{Armor Check Penalty:} Any armor heavier than leather hurts a character's ability to use some skills. An armor check penalty number is the penalty that applies to \skill{Balance}, \skill{Climb}, \skill{Escape Artist}, \skill{Hide}, \skill{Jump}, \skill{Move Silently}, \skill{Sleight of Hand}, and \skill{Tumble} checks by a character wearing a certain kind of armor. Double the normal armor check penalty is applied to Swim checks. A character's encumbrance (the amount of gear carried, including armor) may also apply an armor check penalty.

\textit{Shields:} If a character is wearing armor and using a shield, both armor check penalties apply.

\textit{Nonproficient with Armor Worn:} A character who wears armor and/or uses a shield with which he or she is not proficient takes the armor's (and/or shield's) armor check penalty on attack rolls and on all Strength-based and Dexterity-based ability and skill checks. The penalty for nonproficiency with armor stacks with the penalty for nonproficiency with shields.

\textit{Sleeping in Armor:} A character who sleeps in medium or heavy armor is automatically fatigued the next day. He or she takes a -2 penalty on Strength and Dexterity and can't charge or run. Sleeping in light armor does not cause fatigue.

\textbf{Arcane Spell Failure:} Armor interferes with the gestures that a spellcaster must make to cast an arcane spell that has a somatic component. Arcane spellcasters face the possibility of arcane spell failure if they're wearing armor. Bards can wear light armor without incurring any arcane spell failure chance for their bard spells.

\textit{Casting an Arcane Spell in Armor:} A character who casts an arcane spell while wearing armor must usually make an arcane spell failure roll. The number in the Arcane Spell Failure Chance column on Table: Armor and Shields is the chance that the spell fails and is ruined. If the spell lacks a somatic component, however, it can be cast with no chance of arcane spell failure.

\textit{Shields:} If a character is wearing armor and using a shield, add the two numbers together to get a single arcane spell failure chance.

\textbf{Speed:} Medium or heavy armor slows the wearer down. The number on \tabref{Armor and Shields} is the character's speed while wearing the armor. When running in heavy armor, you move only triple your speed, not quadruple.

Elves, half-giants, and thri-kreen have an unencumbered speed of 12 meters. Humans, half-elves, muls, and pterrans have an unencumbered speed of 9 meters.

They use the first column. Aarakocras, dwarves, and halflings have an unencumbered speed of 6 meters. They use the second column. Remember, however, that a dwarf's land speed remains 6 meters even in medium or heavy armor or when carrying a medium or heavy load.

\textit{Shields:} Shields do not affect a character's speed.

\textbf{Weight:} This column gives the weight of the armor sized for a Medium wearer. Armor fitted for Small characters weighs half as much, and armor for Large characters weighs twice as much.

\subsection{Getting Into And Out Of Armor}
The time required to don armor depends on its type; see \tabref{Donning Armor}.

\textbf{Don:} This column tells how long it takes a character to put the armor on. (One minute is 10 rounds.) Readying (strapping on) a shield is only a move action.

\textbf{Don Hastily:} This column tells how long it takes to put the armor on in a hurry. The armor check penalty and armor bonus for hastily donned armor are each 1 point worse than normal.

\textbf{Remove:} This column tells how long it takes to get the armor off. Loosing a shield (removing it from the arm and dropping it) is only a move action.

\TransparentTable{Donning Armor}{l CCC}{
\tableheader Armor Type & \tableheader Don & \tableheader Don Hastily & \tableheader Remove\\
\rowcolor{TableColor}
Shield (any) & 1 move action & n/a & 1 move action\\
Padded & \multirow{6}{*}{1 minute} & \multirow{6}{*}{5 rounds} & \multirow{6}{*}{1 minute\footnotemark[1]}\\
Leather &&&\\
Hide &&&\\
Studded leather &&&\\
Chitin &&&\\
Chitin warsuit &&&\\
\rowcolor{TableColor}
Breastplate &&&\\
\rowcolor{TableColor}
Caravan &&&\\
\rowcolor{TableColor}
Light gladiator &&&\\
\rowcolor{TableColor}
Medium gladiator &&&\\
\rowcolor{TableColor}
Scale mail &&&\\
\rowcolor{TableColor}
Shell &&&\\
\rowcolor{TableColor}
Banded mail &&&\\
\rowcolor{TableColor}
Splint mail & \multirow{-8}{*}{4 minutes\footnotemark[1]} & \multirow{-8}{*}{1 minute} & \multirow{-8}{*}{1 minute\footnotemark[1]}\\
Half-plate & \multirow{3}{*}{4 minutes\footnotemark[2]} & \multirow{3}{*}{4 minutes\footnotemark[1]} & \multirow{3}{1.5cm}{\centering 1d4+1 minutes\footnotemark[1]}\\
Full plate &&&\\
Tyrian warsuit &&&\\
}
\TransparentTable{}{l X}{1 & If the character has some help, cut this time in half. A single character doing nothing else can help one or two adjacent characters. Two characters can't help each other don armor at the same time.\\
2 & The wearer must have help to don this armor. Without help, it can be donned only hastily.
}

\subsection{Armor For Unusual Creatures}
Armor and shields for unusually big creatures, unusually little creatures, and nonhumanoid creatures have different costs and weights from those given on \tabref{Armor and Shields}. Refer to the appropriate line on the table below and apply the multipliers to cost and weight for the armor type in question.

For creatures of size Tiny or smaller, divide armor bonus by 2.

\Table{}{l CCCC}{
 & \multicolumn{2}{c}{\tableheader Humanoid} & \multicolumn{2}{c}{\tableheader Nonhumanoid} \\
\cmidrule[0.5pt]{2-5}
\tableheader Size & \tableheader Cost & \tableheader Weight & \tableheader Cost & \tableheader Weight\\
Tiny or smaller & $\times$\onehalf & $\times$1/10 & $\times$1 & $\times$1/10\\
Small & $\times$1 & $\times$\onehalf & $\times$2 & $\times$\onehalf\\
Medium & $\times$1 & $\times$1 & $\times$2 & $\times$1\\
Large & $\times$2 & $\times$2 & $\times$4 & $\times$2\\
Huge & $\times$4 & $\times$5 & $\times$8 & $\times$5\\
Gargantuan & $\times$8 & $\times$8 & $\times$16 & $\times$8\\
Colossal & $\times$16 & $\times$12 & $\times$32 & $\times$12\\
}

\subsection{Armor Descriptions}
Any special benefits or accessories to the types of armor found on \tabref{Armor and Shields} are described below.

\textbf{Armor Spikes:} You can have spikes added to your armor, which allow you to deal extra piercing damage on a successful grapple attack. The spikes count as a martial weapon. If you are not proficient with them, you take a -4 penalty on grapple checks when you try to use them. You can also make a regular melee attack (or off-hand attack) with the spikes, and they count as a light weapon in this case. (You can't also make an attack with armor spikes if you have already made an attack with another off-hand weapon, and vice versa.)\\An enhancement bonus to a suit of armor does not improve the spikes' effectiveness, but the spikes can be made into magic weapons in their own right.

\textbf{Banded Mail:} The suit includes gauntlets.

\textbf{Breastplate:} It comes with a helmet and greaves.

\textbf{Buckler:} This small metal shield is worn strapped to your forearm. You can use a bow or crossbow without penalty while carrying it. You can also use your shield arm to wield a weapon (whether you are using an off-hand weapon or using your off hand to help wield a two-handed weapon), but you take a -1 penalty on attack rolls while doing so. This penalty stacks with those that may apply for fighting with your off hand and for fighting with two weapons. In any case, if you use a weapon in your off hand, you don't get the buckler's AC bonus for the rest of the round.

You can't bash someone with a buckler.

\textbf{Caravan Armor:} This suit of armor is a combination of several different materials. Thick chitin bracers provide efficient protection to the forearms, while thick leather protects the shins and knees. A leather kilt and and shirt of thick cord layers protect the body and provide sufficient cooling. This armor is so named because it's mostly used by caravan guards who need decent protection while not being slowed down by their armor. Light caravan armor comes with a turban made of thick cord.

This armor doesn't provide full-body protection and thus the wearer is more prone to critical hits; the AC against rolls to confirm a critical hit is reduced by 1.

\textbf{Chitin Warsuit:} This suit of armor comes with padded armor, which is worn beneath the actual armor, to prevent abrasions. A long shell shirt covers the torso and the waist, chitin sleeves over both arms and shoulders and end in chitin gauntlets, long chitin pants cover the legs, and a bone or chitin helmet, usually made of a creature's skull or head exoskeleton, covers the head. This armor offers good protection, but brings the usual problems with heat accumulation.

\textbf{Chitin:} This armor is skillfully made by interlocking hexagonal bits of chitin (usually carved from a kank's carapace).

A chitin armor comes with a chitin cap.

\textbf{Full Plate:} The suit includes gauntlets, heavy leather boots, a visored helmet, and a thick layer of padding that is worn underneath the armor. Each suit of full plate must be individually fitted to its owner by a master armorsmith, although a captured suit can be resized to fit a new owner at a cost of 200 to 800 (2d4$\times$100) ceramic pieces.

\textbf{Gauntlet, Locked:} This armored gauntlet has small chains and braces that allow the wearer to attach a weapon to the gauntlet so that it cannot be dropped easily. It provides a +10 bonus on any roll made to keep from being disarmed in combat. Removing a weapon from a locked gauntlet or attaching a weapon to a locked gauntlet is a full-round action that provokes attacks of opportunity.

The price given is for a single locked gauntlet. The weight given applies only if you're wearing a breastplate, light armor, or no armor. Otherwise, the locked gauntlet replaces a gauntlet you already have as part of the armor.

While the gauntlet is locked, you can't use the hand wearing it for casting spells or employing skills. (You can still cast spells with somatic components, provided that your other hand is free.)

Like a normal gauntlet, a locked gauntlet lets you deal lethal damage rather than nonlethal damage with an unarmed strike.

\textbf{Half-Plate:} The suit includes gauntlets.

\textbf{Light Gladiator Armor:} This suit of armor combines leather and bone to provide the gladiator with decent protection and minimal hindrance. Thick leather shinpads provide leg protection whithout hampering movement, while a breastplate of bone and leather skirt or loincoth protects the  gladiator's torso. A bone helmet protect the head and face, and a cuff of thick leather protects the gladiator's weapon hand. Gladiators that rely on high maneuverability prefer this kind of of armor; masterwork suits are highly desired and respected.

This armor's lightweight and area-specific coverage provides many openings for critical hits; thus, the AC against rolls to confirm critical hits is reduced by 2.

\textbf{Medium Gladiator Armor:} This suit of armor combines leather and chitin to provide the gladiator with good protection without hampering his freedom of movement too much. A vambrace made of chitin covers the gladiator's weapon arm and is held in place by a leather corselet, while a shoulder plate of chitin covers the off-hand shoulder. A thick leather skirt protects the gladiator's haunch and chitin shinpads protect his tibia. This suit of picemeal armor comes with a chitin helm that usually resembles a beast's head.

This armor provides many openings for critical hits; therefore, the AC against rolls to confirm critical hits is reduced by 2.

\textbf{Scale Mail:} Scale mail is usually made from the scales of an erdlu, inix or other naturally scaled creatures.

The suit includes gauntlets.

\textbf{Shell:} Shell armor is made by weaving giant's hair around the shells of various small creatures such as an aprig.

The suit includes gauntlets.

\textbf{Shield, Heavy:} You strap a shield to your forearm and grip it with your hand. A heavy shield is so heavy that you can't use your shield hand for anything else.

\textit{Wooden or Steel:} Wooden and steel shields offer the same basic protection, though they respond differently to special attacks.

\textit{Shield Bash Attacks:} You can bash an opponent with a heavy shield, using it as an off-hand weapon. See \tabref{Martial Weapons} for the damage dealt by a shield bash. Used this way, a heavy shield is a martial bludgeoning weapon. For the purpose of penalties on attack rolls, treat a heavy shield as a one-handed weapon. If you use your shield as a weapon, you lose its AC bonus until your next action (usually until the next round). An enhancement bonus on a shield does not improve the effectiveness of a shield bash made with it, but the shield can be made into a magic weapon in its own right.

\textbf{Shield, Light:} You strap a shield to your forearm and grip it with your hand. A light shield's weight lets you carry other items in that hand, although you cannot use weapons with it.

\textit{Wooden or Steel:} Wooden and steel shields offer the same basic protection, though they respond differently to special attacks.

\textit{Shield Bash Attacks:} You can bash an opponent with a light shield, using it as an off-hand weapon. See \tabref{Martial Weapons} for the damage dealt by a shield bash. Used this way, a light shield is a martial bludgeoning weapon. For the purpose of penalties on attack rolls, treat a light shield as a light weapon. If you use your shield as a weapon, you lose its AC bonus until your next action (usually until the next round). An enhancement bonus on a shield does not improve the effectiveness of a shield bash made with it, but the shield can be made into a magic weapon in its own right.

\textbf{Shield, Long:} This is a slim, two-handed shield commonly used by the kreen races of the northern kreen Empire; it is extremely rare to find a long shield in the hands of a nomadic kreen of the Tablelands, although they are occasionally spotted in the arena. Kreen usually hold the long shield with two arms from the same side. Long shields are made of bone, chitin, hide, or wood.

You need two hands to use a long shield. Two handed humanoids who use a long shield can do so by using it horizontally, but by doing so you cannot wield a weapon.

\textit{Shield Bash Attacks:} You can bash an opponent with a long shield, using it as an off-hand weapon. See \tabref{Martial Weapons} in the Player's Handbook for the damage dealt by a shield bash. Used this way, a long shield is a martial bludgeoning weapon. For the purpose of penalties on attack rolls, treat a long shield as a two-handed weapon. If you use your shield as a weapon, you lose its AC bonus until your next action (usually until the next round). An enhancement bonus on a shield does not improve the effectiveness of a shield bash made with it, but the shield can be made into a magic weapon in its own right.

\textbf{Shield, Tower:} This massive wooden shield is nearly as tall as you are. In most situations, it provides the indicated shield bonus to your AC. However, you can instead use it as total cover, though you must give up your attacks to do so. The shield does not, however, provide cover against targeted spells; a spellcaster can cast a spell on you by targeting the shield you are holding. You cannot bash with a tower shield, nor can you use your shield hand for anything else.

When employing a tower shield in combat, you take a -2 penalty on attack rolls because of the shield's encumbrance.

\textbf{Shield Spikes:} When added to your shield, these spikes turn it into a martial piercing weapon that increases the damage dealt by a shield bash as if the shield were designed for a creature one size category larger than you. You can't put spikes on a buckler or a tower shield. Otherwise, attacking with a spiked shield is like making a shield bash attack.

An enhancement bonus on a spiked shield does not improve the effectiveness of a shield bash made with it, but a spiked shield can be made into a magic weapon in its own right.

\textbf{Splint Mail:} The suit includes gauntlets.

\textbf{Tyrian Warsuit:} This armor combines metal and chitin. A chitin breatplate covers the front, back, shoulders and upper arms, while a long shell skirt protects the haunch. Metal shinpads, padded on the inside, are worn over leather boots, to avoid burns, while metal gauntlets, also padded on the inside, are worn over leather cuffs. This suit of armor comes with a full chitin helmet.

\subsection{Masterwork Armor}
Just as with weapons, you can purchase or craft masterwork versions of armor or shields. Such a well-made item functions like the normal version, except that its armor check penalty is lessened by 1.

A masterwork suit of armor or shield costs an extra 150 gp over and above the normal cost for that type of armor or shield.

The masterwork quality of a suit of armor or shield never provides a bonus on attack or damage rolls, even if the armor or shield is used as a weapon.

All magic armors and shields are automatically considered to be of masterwork quality.

You can't add the masterwork quality to armor or a shield after it is created; it must be crafted as a masterwork item.
