\Chapter{Equipment}
{I have heard tales that suits of clothing fashioned from metal have even been found from time to time. It is generally agreed that these were worn by warriors to protect against the blows of enemy weapons. I can only speculate that the climate must have been far cooler in those ancient days. Any fool that would wear such clothing now would die faster from heat stroke than he would have from the weapons of his foes. Still, the idea that there was once enough metal in the world to allow such a garment to have been manufactured astounds me.

There are even rumors that mounds of steel, silver, and gold lie hidden in the deepest tunnels of certain forlorn cities. I have never seen such a thing myself, but if such treasures exist, they will reward those who find them most handsomely. Those who control such stores of metal can buy food, power, influence, and sometimes even the sorcerer-king's protection.}
{The Wanderer's Journal}

Dark Sun characters must be well equipped in order to endure the rigors of Athas. This chapter covers a variety of topics related to mundane equipment that every hero needs to survive and prosper.

\section{Equipping a Character}
The world of Athas has a very specific feel to it; many things that are taken for granted in other campaign worlds, like the availability of metal and water, are very different on this heat-wracked planet. To maintain this feel, the equipment available to characters should reflect these differences.

\section{Wealth and Money}
\subsection{Coins}
All prices in Dark Sun are given in terms of ceramic pieces, the most common coin. Ceramics are made from glazed clay and baked in batches once a year in a secure process supervised by the high templar that supervises the city's treasury. Bits are literally one‐tenth parts of a ceramic piece---the ceramic pieces break easily into ten bits. Some cities' ceramic pieces have small holes that can be threaded onto a bracelet or necklace. The lowest unit of Athasian trade is the lead bead (bd).

Each of the city-states of the Tablelands produces its own currency. All cities use ceramic pieces as the most common coin, but also mint silver coins and, in some cases, rare and highly prized gold coins.

\Table[\scriptsize]{Currency Conversions}{l C C C C}{
& \multicolumn{4}{c}{\tableheader Exchange Value}\\
& \tableheader bd & \tableheader bit & \tableheader cp & \tableheader sp\\
Lead bead (bd) & 1 & 1/10 & 1/100 & 1/1,000\\
Ceramic bit (bit) & 10 & 1 & 1/10 & 1/100\\
Ceramic piece (cp) & 100 & 10 & 1 & 1/10\\
Silver piece (sp) & 1,000 & 100 & 10 & 1\\
Gold piece (gp) & 10,000 & 1000 & 100 & 10\\
}

\subsubsection{Moneychangers}
Adventurers that travel between cities will need to change their currency for local currency at each city they visit. With a couple of exceptions, the city-states have moneychangers available for incoming visitors. Located near the city gates and in large market places, moneychangers denote their business by hanging a large purple banner from their shop. The banners are always purple, but the moneychangers in each city-state display a different emblem on the banner, based on their city's standard.

Moneychangers charge each customer a fee to change coins. The fees differ by city and are summarized on \tabref{Moneychangers}.

\Table{Moneychangers}{C C}{
\tableheader City & \tableheader Exchange Rate\\
Balic & 6\%\\
Draj & 8\%\\
Gulg & 10\%\\
Kurn & 16\%\\
Nibenay & 14\%\\
Raam & 12\%\\
Tyr & 12\%\\
Urik & 9\%\\
}

These fees are averages and may vary slightly. There are of course many unscrupulous money merchants who will charge as much as they can get away with. Moneychangers in Kurn are rare but a couple do exist. Since metal coins from any city-state are readily accepted by local merchants and no corresponding Kurnish coins exist there is little need to exchange such coins. There are, however, a few moneychangers willing to exchange ceramic pieces.

There are two cities that do not have moneychangers. Visitors to Celik have no need of a moneychangers, as merchants in that city take coins of all types. Nor are there any moneychangers in Eldaarich. Since Eldaarich has been shut off from the rest of the land for so long, visitors needing to exchange money have been nonexistent, so no moneychangers have set up business.

\subsection{Trade}
In general, the Athasian economy in the cities is relatively stable thanks to the Merchant Houses. Under normal conditions, supply is ample thanks to the caravans traveling back and forth between the cities. However, for smaller communities and trade outposts the price situation on certain goods can sway drastically. A raider attack or sandstorm can result in lack of necessities such as food and water, for which people will pay almost any amount of coin. Coins are not the only means of exchange. Barter and trade in commodities is widespread.

Dune traders commonly exchange trade goods without using currency, instead relying on a basic bartering system.


\Table{Trade Goods}{l X}{
\tableheader Cost & \tableheader Item\\
2 bits & One pound of salt\\
3 bits & One pound of grain or faro\\
5 bits & One pound of lead\\
1 cp & One pound of nuts, or one pound of kank nectar\\
4 cp & One square yard of cotton (cloth)\\
5 cp & One pound of obsidian\\
10 cp & One square yard of silk, or one tun of water, or one erdlu\\
50 cp & One pound of copper, or one herding kank, or one aprig\\
100 cp & One pound of iron, or one male carru, or one inix\\
200 cp & One mekillot\\
300 cp & One female carru\\
500 cp & One pound of silver\\
5,000 cp & One pound of gold\\
}

\subsection{Selling Loot}
In general, a character can sell something for half its listed price.

Trade goods are the exception to the half-price rule. A trade good, in this sense, is a valuable good that can be easily exchanged almost as if it were cash itself.

\section{Weapons}
Characters in a Dark Sun game use a variety of weapons: some with direct counterparts in the real world, some without.

\subsection{Weapon Categories}
Weapons are grouped into several interlocking sets of categories.

These categories pertain to what training is needed to become proficient in a weapon's use (simple, martial, or exotic), the weapon's usefulness either in close combat (melee) or at a distance (ranged, which includes both thrown and projectile weapons), its relative encumbrance (light, one-handed, or two-handed), and its size (Small, Medium, or Large).

\textbf{Simple, Martial, and Exotic Weapons:} Anybody but a druid, monk, or wizard is proficient with all simple weapons. Barbarians, fighters, paladins, and rangers are proficient with all simple and all martial weapons. Characters of other classes are proficient with an assortment of mainly simple weapons and possibly also some martial or even exotic weapons. A character who uses a weapon with which he or she is not proficient takes a -4 penalty on attack rolls.

\textbf{Melee and Ranged Weapons:} Melee weapons are used for making melee attacks, though some of them can be thrown as well. Ranged weapons are thrown weapons or projectile weapons that are not effective in melee.

\textit{Reach Weapons:} Glaives, guisarmes, lances, longspears, ranseurs, spiked chains, and whips are reach weapons. A reach weapon is a melee weapon that allows its wielder to strike at targets that aren't adjacent to him or her. Most reach weapons double the wielder's natural reach, meaning that a typical Small or Medium wielder of such a weapon can attack a creature 10 feet away, but not a creature in an adjacent square. A typical Large character wielding a reach weapon of the appropriate size can attack a creature 15 or 20 feet away, but not adjacent creatures or creatures up to 10 feet away.

Note: Small and Medium creatures wielding reach weapons threaten all squares 10 feet (2 squares) away, even diagonally. (This is an exception to the rule that 2 squares of diagonal distance is measured as 15 feet.)

\textit{Double Weapons:} Dire flails, dwarven urgroshes, gnome hooked hammers, orc double axes, quarterstaffs, and two-bladed swords are double weapons. A character can fight with both ends of a double weapon as if fighting with two weapons, but he or she incurs all the normal attack penalties associated with two-weapon combat, just as though the character were wielding a one-handed weapon and a light weapon.

The character can also choose to use a double weapon two handed, attacking with only one end of it. A creature wielding a double weapon in one hand can't use it as a double weapon—only one end of the weapon can be used in any given round.

\textit{Thrown Weapons:} Daggers, clubs, shortspears, spears, darts, javelins, throwing axes, light hammers, tridents, shuriken, and nets are thrown weapons. The wielder applies his or her Strength modifier to damage dealt by thrown weapons (except for splash weapons). It is possible to throw a weapon that isn't designed to be thrown (that is, a melee weapon that doesn't have a numeric entry in the Range Increment column on Table: Weapons), but a character who does so takes a -4 penalty on the attack roll. Throwing a light or one-handed weapon is a standard action, while throwing a two-handed weapon is a full-round action. Regardless of the type of weapon, such an attack scores a threat only on a natural roll of 20 and deals double damage on a critical hit. Such a weapon has a range increment of 10 feet.

\textit{Projectile Weapons:} Light crossbows, slings, heavy crossbows, shortbows, composite shortbows, longbows, composite longbows, hand crossbows, and repeating crossbows are projectile weapons. Most projectile weapons require two hands to use (see specific weapon descriptions). A character gets no Strength bonus on damage rolls with a projectile weapon unless it's a specially built composite shortbow, specially built composite longbow, or sling. If the character has a penalty for low Strength, apply it to damage rolls when he or she uses a bow or a sling.

\textit{Ammunition:} Projectile weapons use ammunition: arrows (for bows), bolts (for crossbows), or sling bullets (for slings). When using a bow, a character can draw ammunition as a free action; crossbows and slings require an action for reloading. Generally speaking, ammunition that hits its target is destroyed or rendered useless, while normal ammunition that misses has a 50\% chance of being destroyed or lost.

Although they are thrown weapons, shuriken are treated as ammunition for the purposes of drawing them, crafting masterwork or otherwise special versions of them (see Masterwork Weapons), and what happens to them after they are thrown.

\textbf{Light, One-Handed, and Two-Handed Melee Weapons:} This designation is a measure of how much effort it takes to wield a weapon in combat. It indicates whether a melee weapon, when wielded by a character of the weapon's size category, is considered a light weapon, a one-handed weapon, or a two-handed weapon.

\textit{Light:} A light weapon is easier to use in one's off hand than a one-handed weapon is, and it can be used while grappling. A light weapon is used in one hand. Add the wielder's Strength bonus (if any) to damage rolls for melee attacks with a light weapon if it's used in the primary hand, or one-half the wielder's Strength bonus if it's used in the off hand. Using two hands to wield a light weapon gives no advantage on damage; the Strength bonus applies as though the weapon were held in the wielder's primary hand only.

An unarmed strike is always considered a light weapon.

\textit{One-Handed:} A one-handed weapon can be used in either the primary hand or the off hand. Add the wielder's Strength bonus to damage rolls for melee attacks with a one-handed weapon if it's used in the primary hand, or ½ his or her Strength bonus if it's used in the off hand. If a one-handed weapon is wielded with two hands during melee combat, add 1½ times the character's Strength bonus to damage rolls.

\textit{Two-Handed:} Two hands are required to use a two-handed melee weapon effectively. Apply 1½ times the character's Strength bonus to damage rolls for melee attacks with such a weapon.

\textbf{Weapon Size:} Every weapon has a size category. This designation indicates the size of the creature for which the weapon was designed.

\Table{Larger and Smaller Weapon Damage}{l C C C C C}{
\tableheader Example Weapon & \tableheader Tiny & \tableheader Small & \tableheader Medium & \tableheader Large & \tableheader Huge \\
Shuriken & — & 1 & 1d2 & 1d3 & 1d4\\
Gauntlet & 1 & 1d2 & 1d3 & 1d4 & 1d6\\
Dagger & 1d2 & 1d3 & 1d4 & 1d6 & 1d8\\
Shortspear & 1d3 & 1d4 & 1d6 & 1d8 & 2d6\\
Falchion & 1d4 & 1d6 & 2d4 & 2d6 & 3d6\\
Longsword & 1d4 & 1d6 & 1d8 & 2d6 & 3d6\\
Bastard Sword & 1d6 & 1d8 & 1d10 & 2d8 & 3d8\\
Greataxe & 1d8 & 1d10 & 1d12 & 3d6 & 4d6\\
Greatsword & 1d8 & 1d10 & 2d6 & 3d6 & 4d6\\
}

A weapon's size category isn't the same as its size as an object. Instead, a weapon's size category is keyed to the size of the intended wielder. In general, a light weapon is an object two size categories smaller than the wielder, a one-handed weapon is an object one size category smaller than the wielder, and a two-handed weapon is an object of the same size category as the wielder.

\textit{Inappropriately Sized Weapons:} A creature can't make optimum use of a weapon that isn't properly sized for it. A cumulative -2 penalty applies on attack rolls for each size category of difference between the size of its intended wielder and the size of its actual wielder. If the creature isn't proficient with the weapon a -4 nonproficiency penalty also applies.

The measure of how much effort it takes to use a weapon (whether the weapon is designated as a light, one-handed, or two-handed weapon for a particular wielder) is altered by one step for each size category of difference between the wielder's size and the size of the creature for which the weapon was designed. If a weapon's designation would be changed to something other than light, one-handed, or two-handed by this alteration, the creature can't wield the weapon at all.

\textbf{Improvised Weapons:} Sometimes objects not crafted to be weapons nonetheless see use in combat. Because such objects are not designed for this use, any creature that uses one in combat is considered to be nonproficient with it and takes a -4 penalty on attack rolls made with that object. To determine the size category and appropriate damage for an improvised weapon, compare its relative size and damage potential to the weapon list to find a reasonable match. An improvised weapon scores a threat on a natural roll of 20 and deals double damage on a critical hit. An improvised thrown weapon has a range increment of 10 feet.

\subsubsection{Metal Weapons}
Metal is rare on Athas, and many weapons ordinarily crafted using metal components are extremely expensive. Unworked iron is worth 100 Cp per pound on average, but can cost much, much more in some places. Worked metal is even more expensive, as craftsmen who actually know how to craft metal items are rare at best. Most metal weapons are items dating back to the Green Age, or have been crafted from the meager resources of Tyr's iron mines.

Due to the rarity of metal, weapons and other items constructed primarily from metal are priced at in gp, e.g., a metal longsword costs 15 gp (or 1,500 Cp). Due to the extremely high cost of metal weaponry, most weapons are constructed from inferior, but functional, materials instead on Athas. Most common are bone and stone such as flint or obsidian, but treated wood is sometimes used as well. Metal weapons constructed from inferior materials, such as bone longsword or an axe with a head made from stone, suffer a $-1$ penalty to attack and damage rolls. This penalty cannot reduce damage dealt below 1.

Furthermore, due to the rarity of metal, Athas has its share of unique weapons designed to be constructed from non–metal materials; as such, they do not suffer from the inferior materials penalties described above.

\WeaponTable{Simple Weapons}{
\WeaponType{Unarmed Attacks}\\
Gauntlet
	& 2 cp & 4 cp & 1d2 & 1d3 & 1d4 & $\times2$ & & \onehalf lb& 1 lb& 2 lb & Bludgeoning\\
Unarmed strike
	& & & 1d2 & 1d3 & 1d4 & $\times2$ & & & & & Bludg. non-lethal\\

\WeaponType{Light Melee Weapons}\\
Dagger
	& 2 cp & 4 cp & 1d3 & 1d4 & 1d6 & 19--20/$\times2$ & 10 ft. & \onehalf lb & 1 lb & 2 lb & Pierc. or slash.\\
Pushik
	& 4 cp & 8 cp & 1d3 & 1d4 & 1d6 & $\times3$ & & \onehalf lb & 1 lb & 2 lb & Piercing\\
Mace, light
	& 5 cp & 10 cp & 1d4 & 1d6 & 1d8 & $\times2$ & & 2 lb & 4 lb & 8 lb & Bludgeoning\\

\WeaponType{One-Handed Melee Weapons}\\
Club
	& & & 1d4 & 1d6 & 1d8 & $\times2$ & 10 ft. & 1\onehalf lb & 3 lb & 6 lb & Bludgeoning\\
Mace, heavy
	& 12 cp & 24 cp & 1d6 & 1d8 & 2d6 & $\times2$ & & 4 lb & 8 lb & 16 lb & Bludgeoning\\
Quabone
	& 3 cp & 6 cp & 1d4 & 1d6 & 1d8 & $\times2$ & & 2 lb & 4 lb & 8 lb & Bludgeoning\\
Shortspear
	& 1 cp & 2 cp & 1d4 & 1d6 & 1d8 & $\times2$ & 20 ft. & 1\onehalf lb & 3 lb & 6 lb & Piercing\\
Tonfa
	& 5 cp & 10 cp & 1d3 & 1d4 & 1d6 & $\times2$ & & 1 lb & 2 lb & 4 lb & Bludgeoning\\

\WeaponType{Two-Handed Melee Weapons}\\
Great tonfa
	& 10 cp & 20 cp & 1d4 & 1d6 & 1d8 & $\times2$ & & 2\onehalf lb & 5 lb & 10 lb & Bludgeoning\\
Longspear\footnotemark[1]
	& 5 cp & 10 cp & 1d6 & 1d8 & 2d6 & $\times3$ &  & 4\onehalf lb & 9 lb & 18 lb & Piercing\\
Quarterstaff
	& & & 1d4 1d4 & 1d6 1d6 & 1d8 1d8 & $\times2$ & & 2 lb & 4 lb & 8 lb & Bludgeoning\\
Spear
	& 2 cp & 4 cp & 1d6 & 1d8 & 2d6 & $\times3$ & 20 ft. & 3 lb & 6 lb & 12 lb & Piercing\\

\WeaponType{Ranged Weapons}\\
Blowgun
	& 5 cp & 10 cp & 1 & 1d2 & 1d3 & $\times2$ & 10 ft. & 2 lb & 4 lb & 8 lb & Piercing\\
\Projectile{Needles, blowgun (20)}{1 cp}{2 cp}{}{}{}\\
Crossbow, heavy
	& 50 cp & 100 cp & 1d8 & 1d10 & 2d8 & 19--20/$\times2$ & 120 ft. & 4 lb & 8 lb & 16 lb & Piercing\\
\Projectile{Bolts, crossbow (10)}{1 cp}{2 cp}{\onehalf lb}{1 lb}{2 lb}\\
Crossbow, light
	& 35 cp & 70 cp & 1d6 & 1d8 & 2d6 & 19--20/$\times2$ & 80 ft. & 2 lb & 4 lb & 8 lb & Piercing\\
\Projectile{Bolts, crossbow (10)}{1 cp}{2 cp}{\onehalf lb}{1 lb}{2 lb}\\
Dart
	& 5 bits & 1 cp & 1d3 & 1d4 & 1d6 & $\times2$ & 20 ft. & \onequarter lb & \onehalf lb & 1 lb & Piercing\\
Javelin
	& 1 cp & 2 cp & 1d4 & 1d6 & 1d8 & $\times2$ & 30 ft. & 1 lb & 2 lb & 4 lb & Piercing\\
Pelota
	& 2 cp & 4 cp & 1d3 & 1d4 & 1d6 & $\times2$ & 10 ft. & \onehalf lb & 1 lb & 2 lb & Bludg. and pierc.\\
Sling
	& & & 1d3 & 1d4 & 1d6 & $\times2$ & 50 ft. & & & & Bludgeoning\\
\Projectile{Bullets, sling (10)}{1 bit}{2 bits}{\onehalf lb}{1 lb}{2 lb}\\
}

\WeaponTable{Martial Weapons}{
\WeaponType{Light Melee Weapons}\\
Forearm Axe
	& 30 cp & 60 cp & 1d3 & 1d4 & 1d6 & $\times3$ & & 3 lb & 6 lb & 12 lb & Slashing\\
Macahuitl, small
	& 20 cp & 40 cp & 1d4 & 1d6 & 1d8 & 19--20/$\times2$ & & 1 lb & 2 lb & 4 lb & Slashing\\
Sap
	& 1 cp & 2 cp & 1d4 & 1d6 & 1d8 & $\times2$ & & 1 lb & 2 lb & 4 lb & Bludg. non-lethal\\
Shield, light
	& $\star$ & $\star$ & 1d2 & 1d3 & 1d4 & $\times2$ & & $\star$ & $\star$ & $\star$ & Bludgeoning\\
Slodak
	& 18 cp & 36 cp & 1d4 & 1d6 & 1d8 & 19--20/$\times2$ & & 2 lb & 4 lb & 8 lb & Slashing\\
Spiked armor
	& $\star$ & $\star$ & 1d4 & 1d6 & 1d8 & $\times2$ & & $\star$ & $\star$ & $\star$ & Piercing\\
Spiked shield, light
	& $\star$ & $\star$ & 1d3 & 1d4 & 1d6 & $\times2$ & & $\star$ & $\star$ & $\star$ & Piercing\\
Tortoise blade
	& 20 cp & 40 cp & 1d3 & 1d4 & 1d6 & 20/$\times2$ & & 1 lb & 2 lb & 4 lb & Piercing\\

\WeaponType{One-Handed Melee Weapons}\\
Alak
	& 7 cp & 14 cp & 1d4 & 1d6 & 1d8 & $\times3$ & & 3 lb & 6 lb & 12 lb & Piercing\\
Alhulak\footnotemark[1]
	& 40 cp & 80 cp & 1d4 & 1d6 & 1d8 & $\times3$ & & 4\onehalf lb & 9 lb & 18 lb & Piercing\\
Carrikal
	& 10 cp & 20 cp & 1d6 & 1d8 & 2d6 & $\times3$ & & 3 lb & 6 lb & 12 lb & Slashing\\
Impaler
	& 8 cp & 16 cp & 1d4 & 1d6 & 1d8 & $\times4$ & & 2\onehalf lb & 5 lb & 10 lb & Piercing\\
Macahuitl
	& 35 cp & 70 cp & 1d6 & 1d8 & 2d6 & 19--20/$\times2$ & & 2\onehalf lb & 5 lb & 10 lb & Slashing\\
Shield, heavy
	& $\star$ & $\star$ & 1d3 & 1d4 & 1d6 & $\times2$ & & $\star$ & $\star$ & $\star$ & Bludgeoning\\
Spiked shield, heavy
	& $\star$ & $\star$ & 1d4 & 1d6 & 1d8 & $\times2$ & & $\star$ & $\star$ & $\star$ & Piercing\\

\WeaponType{Two-Handed Melee Weapons}\\
Crusher, Fixed\footnotemark[1]
	& 60 cp & 120 cp & 1d6 & 1d8 & 2d6 & $\times2$ & & 6 lb & 12 lb & 24 lb & Bludgeoning\\
Datchi Club\footnotemark[1]
	& 5 cp & 10 cp & 1d6 & 1d8 & 2d6 & $\times3$ & & 5 lb & 10 lb & 20 lb & Bludgeoning\\
Gouge
	& 20 cp & 40 cp & 1d8 & 1d10 & 2d8 & $\times3$ & & 6 lb & 12 lb & 24 lb & Piercing\\
Greatclub
	& 5 cp & 10 cp & 1d8 & 1d10 & 2d8 & $\times2$ & & 4 lb & 8 lb & 16 lb & Bludgeoning\\
Lance
	& 10 cp & 20 cp & 1d6 & 1d8 & 2d6 & $\times3$ & & 5 lb & 10 lb & 20 lb & Piercing\\
Macahuitl, Great
	& 50 cp & 100 cp & 1d10 & 2d6 & 3d6 & 19--20/$\times2$ & & 6 lb & 12 lb & 24 lb & Slashing\\
Maul
	& 25 cp & 50 cp & 1d10 & 1d12 & 3d6 & $\times2$ & & 5 lb & 10 lb & 20 lb & Bludgeoning\\
Tkaesali\footnotemark[1]
	& 8 cp & 16 cp & 1d8 & 1d10 & 2d8 & $\times3$ & & 7\onehalf lb & 15 lb & 30 lb & Slashing\\
Trikal
	& 10 cp & 20 cp & 1d6 & 1d8 & 2d6 & $\times3$ & & 3\onehalf lb & 7 lb & 14 lb & Slashing\\

\WeaponType{Ranged Weapons}\\
Atlatl
	& 25 cp & 50 cp & 1d4 & 1d6 & 1d8 & $\times3$ & 40 ft. & 3 lb & 6 lb & 12 lb & Piercing\\
\Projectile{Javeli, atlatl}{2 cp}{4 cp}{1 lb}{2 lb}{4 lb}\\
Crossbow, fixed
	& 200 cp & 400 cp & 1d12 & 2d8 & 3d8 & 19--20/$\times2$ & 150 ft. & 50 lb & 100 lb & 200 lb & Piercing\\
\Projectile{Bolts (10)}{3 cp}{6 cp}{1\onehalf lb}{3 lb}{6 lb}\\
Longbow
	& 75 cp & 150 cp & 1d6 & 1d8 & 2d6 & $\times3$ & 100 ft. & 1\onehalf lb & 3 lb & 6 lb & Piercing\\
\Projectile{Arrows (10)}{1 cp}{2 cp}{1\onehalf lb}{3 lb}{6 lb}\\
Longbow, composite
	& 100 cp & 200 cp & 1d6 & 1d8 & 2d6 & $\times3$ & 110 ft. & 1\onehalf lb & 3 lb & 6 lb & Piercing\\
\Projectile{Arrows (10)}{1 cp}{2 cp}{1\onehalf lb}{3 lb}{6 lb}\\
Shortbow
	& 30 cp & 60 cp & 1d4 & 1d6 & 1d8 & $\times3$ & 60 ft. & 1 lb & 2 lb & 4 lb & Piercing\\
\Projectile{Arrows (10)}{1 cp}{2 cp}{1\onehalf lb}{3 lb}{6 lb}\\
Shortbow, composite
	& 75 cp & 150 cp & 1d4 & 1d6 & 1d8 & $\times3$ & 70 ft. & 1 lb & 2 lb & 4 lb & Piercing\\
\Projectile{Arrows (10)}{1 cp}{2 cp}{1\onehalf lb}{3 lb}{6 lb}\\
}

\WeaponTable{Exotic Weapons}{
\WeaponType{Light Melee Weapons}\\
Nunchaku
	& 2 cp & 4 cp & 1d4 & 1d6 & 1d8 & $\times2$ & & 1 lb & 2 lb & 4 lb & Bludgeoning\\
Sai
	& 1 cp & 2 cp & 1d3 & 1d4 & 1d6 & $\times2$ & 10 ft. & \onehalf lb & 1 lb & 2 lb & Bludgeoning\\
\WeaponType{One-Handed Melee Weapons}\\
Whip\footnotemark[1]
	& 1 cp & 2 cp & 1d2 & 1d3 & 1d4 & $\times2$ & & 1 lb & 2 lb & 4 lb & Slash. non-lethal\\
\WeaponType{Two-Handed Melee Weapons}\\

\WeaponType{Ranged Weapons}\\
Bolas
	& 5 cp & 10 cp & 1d3 & 1d4 & 1d6 & $\times2$ & 10 ft. & 1 lb & 2 lb & 4 lb & Bludg. non-lethal\\
Crossbow, hand
	& 100 cp & 200 cp & 1d3 & 1d4 & 1d6 & 19--20/$\times2$ & 30 ft. & 1 lb & 2 lb & 4 lb & Piercing\\
\Projectile{Bolts (10)}{1 cp}{2 cp}{\onehalf lb}{1 lb}{2 lb}\\
Crossbow, repeating heavy
	& 400 cp & 800 cp & 1d8 & 1d10 & 2d8 & 19--20/$\times2$ & 120 ft. & 6 lb & 12 lb & 24 lb & Piercing\\
\Projectile{Bolts (5)}{1 cp}{2 cp}{\onehalf lb}{1 lb}{2 lb}\\
Crossbow, repeating light
	& 250 cp & 500 cp & 1d6 & 1d8 & 2d6 & 19--20/$\times2$ & 80 ft. & 3 lb & 6 lb & 12 lb & Piercing\\
\Projectile{Bolts (5)}{1 cp}{2 cp}{\onehalf lb}{1 lb}{2 lb}\\
Net
	& 20 cp & 40 cp & & & & & 10 ft. & 3 lb & 6 lb & 12 lb & \\
}

\WeaponTable{Metal Simple Weapons}{
\WeaponType{Light Melee Weapons}\\
Dagger, punching
	& 2 gp & 4 gp & 1d3 & 1d4 & 1d6 & $\times3$ & & \onehalf lb & 1 lb & 2 lb & Piercing\\
\InferiorWeapon{(bone or wood)}{1 cp}{2 cp}{\onequarter lb}{\onehalf lb}{1 lb}\\
\InferiorWeapon{(stone)}{1 cp}{2 cp}{1 lb}{2 lb}{4 lb}\\
Gauntlet, spiked
	& 25 sp & 5 gp & 1d3 & 1d4 & 1d6 & $\times2$ & & \onehalf lb & 1 lb & 2 lb & Piercing\\
\InferiorWeapon{(bone or wood)}{25 bits}{5 cp}{\onequarter lb}{\onehalf lb}{1 lb}\\
\InferiorWeapon{(stone)}{25 bits}{5 cp}{1 lb}{2 lb}{4 lb}\\
Sickle
	& 6 gp & 12 gp & 1d4 & 1d6 & 1d8 & $\times2$ & & 1 lb & 2 lb & 4 lb & Slashing\\
\InferiorWeapon{(bone or wood)}{1 cp}{2 cp}{\onehalf lb}{1 lb}{2 lb}\\
\InferiorWeapon{(stone)}{1 cp}{2 cp}{2 lb}{4 lb}{8 lb}\\

\WeaponType{One-Handed Melee Weapons}\\
Morningstar
	& 8 gp & 16 gp & 1d6 & 1d8 & 2d6 & $\times2$ & & 3 lb & 6 lb & 12 lb & Bludg. and pierc.\\
\InferiorWeapon{(bone or wood)}{4 cp}{8 cp}{1\onehalf lb}{3 lb}{6 lb}\\
\InferiorWeapon{(stone)}{4 cp}{8 cp}{6 lb}{12 lb}{24 lb}\\

% \rowcolor{white}
% \multicolumn{12}{l}{$\diamond$ Inferior material weapons suffer -1 penalty to attack and damage rolls.}\\
}

\WeaponTable{Metal Martial Weapons}{
\WeaponType{Light Melee Weapons}\\
Axe, throwing
	& 8 gp & 16 gp & 1d4 & 1d6 & 1d6 & $\times2$ & 10 ft. & 1 lb & 2 lb & 4 lb & Slashing\\
\InferiorWeapon{(bone or wood)}{4 cp}{8 cp}{\onehalf lb}{1 lb}{2 lb}\\
\InferiorWeapon{(stone)}{4 cp}{8 cp}{2 lb}{4 lb}{8 lb}\\
Hammer, light
	& 1 gp & 2 gp & 1d3 & 1d4 & 1d4 & $\times2$ & 20 ft. & 1 lb & 2 lb & 4 lb & Bludgeoning\\
\InferiorWeapon{(bone or wood)}{5 bits}{1 cp}{\onehalf lb}{1 lb}{2 lb}\\
\InferiorWeapon{(stone)}{5 bits}{1 cp}{2 lb}{4 lb}{8 lb}\\
Handaxe
	& 6 gp & 12 gp & 1d4 & 1d6 & 1d6 & $\times3$ & & 1\onehalf lb & 3 lb & 6 lb & Slashing\\
\InferiorWeapon{(bone or wood)}{3 cp}{6 cp}{\threequarters lb}{1\onehalf lb}{3 lb}\\
\InferiorWeapon{(stone)}{3 cp}{6 cp}{3 lb}{6 lb}{12 lb}\\
Kukri
	& 8 gp & 16 gp & 1d3 & 1d4 & 1d6 & 18--20/$\times2$ & & 1 lb & 2 lb & 4 lb & Slashing\\
\InferiorWeapon{(bone or wood)}{4 cp}{8 cp}{\onehalf lb}{1 lb}{2 lb}\\
\InferiorWeapon{(stone)}{4 cp}{8 cp}{2 lb}{4 lb}{8 lb}\\
Pick, light
	& 4 gp & 8 gp & 1d3 & 1d4 & 1d6 & $\times4$ & & 1\onehalf lb & 3 lb & 6 lb & Piercing\\
\InferiorWeapon{(bone or wood)}{2 cp}{4 cp}{\threequarters lb}{1\onehalf lb}{3 lb}\\
\InferiorWeapon{(stone)}{2 cp}{4 cp}{3 lb}{6 lb}{12 lb}\\
Sword, short
	& 10 gp & 20 gp & 1d4 & 1d6 & 1d8 & 19--20/$\times2$ & & 1 lb & 2 lb & 4 lb & Piercing\\
\InferiorWeapon{(bone or wood)}{5 cp}{10 cp}{\onehalf lb}{1 lb}{2 lb}\\
\InferiorWeapon{(stone)}{5 cp}{10 cp}{2 lb}{4 lb}{8 lb}\\

\WeaponType{One-Handed Melee Weapons}\\
Battleaxe
	& 10 gp & 20 gp & 1d6 & 1d8 & 2d6 & $\times3$ & & 3 lb & 6 lb & 12 lb & Slashing\\
\InferiorWeapon{(bone or wood)}{5 cp}{10 cp}{1\onehalf lb}{3 lb}{6 lb}\\
\InferiorWeapon{(stone)}{5 cp}{10 cp}{6 lb}{12 lb}{24 lb}\\
Flail
	& 8 gp & 16 gp & 1d6 & 1d8 & 2d6 & $\times2$ & & 2\onehalf lb & 5 lb & 10 lb & Bludgeoning\\
\InferiorWeapon{(bone or wood)}{4 cp}{8 cp}{1\onequarter lb}{2\onehalf lb}{5 lb}\\
\InferiorWeapon{(stone)}{4 cp}{8 cp}{5 lb}{10 lb}{20 lb}\\
Longsword
	& 15 gp & 30 gp & 1d6 & 1d8 & 2d6 & 19--20/$\times2$ & & 2 lb & 4 lb & 8 lb & Slashing\\
\InferiorWeapon{(bone or wood)}{75 bits}{15 cp}{1 lb}{2 lb}{4 lb}\\
\InferiorWeapon{(stone)}{75 bits}{15 cp}{4 lb}{8 lb}{16 lb}\\
Pick, heavy
	& 8 gp & 16 gp & 1d4 & 1d6 & 1d8 & $\times4$ & & 3 lb & 6 lb & 12 lb & Piercing\\
\InferiorWeapon{(bone or wood)}{5 cp}{10 cp}{\onehalf lb}{1 lb}{2 lb}\\
\InferiorWeapon{(stone)}{5 cp}{10 cp}{2 lb}{4 lb}{8 lb}\\
Rapier
	& 20 gp & 40 gp & 1d4 & 1d6 & 1d8 & 18--20/$\times2$ & & 1 lb & 2 lb & 4 lb & Piercing\\
\InferiorWeapon{(bone or wood)}{5 cp}{10 cp}{\onehalf lb}{1 lb}{2 lb}\\
\InferiorWeapon{(stone)}{5 cp}{10 cp}{2 lb}{4 lb}{8 lb}\\
Scimitar
	& 15 gp & 30 gp & 1d4 & 1d6 & 1d8 & 18--20/$\times2$ & & 2 lb & 4 lb & 8 lb & Slashing\\
\InferiorWeapon{(bone or wood)}{5 cp}{10 cp}{\onehalf lb}{1 lb}{2 lb}\\
\InferiorWeapon{(stone)}{5 cp}{10 cp}{2 lb}{4 lb}{8 lb}\\
Trident
	& 15 gp & 30 gp & 1d6 & 1d8 & 2d6 & $\times2$ & 10 ft. & 2 lb & 4 lb & 8 lb & Piercing\\
\InferiorWeapon{(bone or wood)}{5 cp}{10 cp}{\onehalf lb}{1 lb}{2 lb}\\
\InferiorWeapon{(stone)}{5 cp}{10 cp}{2 lb}{4 lb}{8 lb}\\
Warhammer
	& 12 gp & 24 gp & 1d6 & 1d8 & 2d6 & $\times3$ & & 2\onehalf lb & 5 lb & 10 lb & Bludgeoning\\
\InferiorWeapon{(bone or wood)}{5 cp}{10 cp}{\onehalf lb}{1 lb}{2 lb}\\
\InferiorWeapon{(stone)}{5 cp}{10 cp}{2 lb}{4 lb}{8 lb}\\
}

\WeaponTable{Metal Martial Weapons}{
\WeaponType{Two-Handed Melee Weapons}\\
Falchion
	& 75 gp & 150 gp & 1d6 & 2d4 & 2d6 & 18--20/$\times2$ & & 4 lb & 8 lb & 16 lb & Slashing\\
\InferiorWeapon{(bone or wood)}{5 cp}{10 cp}{\onehalf lb}{1 lb}{2 lb}\\
\InferiorWeapon{(stone)}{5 cp}{10 cp}{2 lb}{4 lb}{8 lb}\\
Glaive\footnotemark[1]
	& 8 gp & 16 gp & 1d8 & 1d10 & 2d8 & $\times3$ & & 10 lb & 10 lb & 10 lb & Slashing\\
\InferiorWeapon{(bone or wood)}{5 cp}{10 cp}{\onehalf lb}{1 lb}{2 lb}\\
\InferiorWeapon{(stone)}{5 cp}{10 cp}{2 lb}{4 lb}{8 lb}\\
Greataxe
	& 20 gp & 40 gp & 1d10 & 1d12 & 3d6 & $\times3$ & & 12 lb & 12 lb & 12 lb & Slashing\\
\InferiorWeapon{(bone or wood)}{5 cp}{10 cp}{\onehalf lb}{1 lb}{2 lb}\\
\InferiorWeapon{(stone)}{5 cp}{10 cp}{2 lb}{4 lb}{8 lb}\\
Flail, heavy
	& 15 gp & 30 gp & 1d8 & 1d10 & 2d8 & 19--20/$\times2$ & & 10 lb & 10 lb & 10 lb & Bludgeoning\\
\InferiorWeapon{(bone or wood)}{5 cp}{10 cp}{\onehalf lb}{1 lb}{2 lb}\\
\InferiorWeapon{(stone)}{5 cp}{10 cp}{2 lb}{4 lb}{8 lb}\\
Greatsword
	& 50 gp & 100 gp & 1d10 & 2d6 & 3d6 & 19--20/$\times2$ & & 8 lb & 8 lb & 8 lb & Slashing\\
\InferiorWeapon{(bone or wood)}{5 cp}{10 cp}{\onehalf lb}{1 lb}{2 lb}\\
\InferiorWeapon{(stone)}{5 cp}{10 cp}{2 lb}{4 lb}{8 lb}\\
Guisarme\footnotemark[1]
	& 45 sp & 9 gp & 1d6 & 2d4 & 2d6 & $\times3$ & & 12 lb & 12 lb & 12 lb & Slashing\\
\InferiorWeapon{(bone or wood)}{5 cp}{10 cp}{\onehalf lb}{1 lb}{2 lb}\\
\InferiorWeapon{(stone)}{5 cp}{10 cp}{2 lb}{4 lb}{8 lb}\\
Halberd
	& 10 gp & 20 gp & 1d8 & 1d10 & 2d8 & $\times3$ & & 12 lb & 12 lb & 12 lb & Pierc. or slash.\\
\InferiorWeapon{(bone or wood)}{5 cp}{10 cp}{\onehalf lb}{1 lb}{2 lb}\\
\InferiorWeapon{(stone)}{5 cp}{10 cp}{2 lb}{4 lb}{8 lb}\\
Ranseur\footnotemark[1]
	& 10 gp & 20 gp & 1d6 & 2d4 & 2d6 & $\times3$ & & 12 lb & 12 lb & 12 lb & Piercing\\
\InferiorWeapon{(bone or wood)}{5 cp}{10 cp}{\onehalf lb}{1 lb}{2 lb}\\
\InferiorWeapon{(stone)}{5 cp}{10 cp}{2 lb}{4 lb}{8 lb}\\
Scythe
	& 18 gp & 36 gp & 1d6 & 2d4 & 2d6 & $\times4$ & & 10 lb & 10 lb & 10 lb & Pierc. or slash.\\
\InferiorWeapon{(bone or wood)}{5 cp}{10 cp}{\onehalf lb}{1 lb}{2 lb}\\
\InferiorWeapon{(stone)}{5 cp}{10 cp}{2 lb}{4 lb}{8 lb}\\

\WeaponType{Ranged Weapons}\\
\Projectile{Arrows, metal (20)}{1 gp}{2 gp}{1\onehalf lb}{3 lb}{6 lb}\\
}


% Metal Exotic Weapons	Cost	Dmg (S)	Dmg (M)	Critical	Range Increment	Weight1	Type2
% Light Melee Weapons
% Kama	2 gp	1d4	1d6	×2	—	2 lb	Slashing
% Siangham	3 gp	1d4	1d6	×2	—	1 lb	Piercing
% One-Handed Melee Weapons
% Sword, bastard	35 gp	1d8	1d10	19-20/×2	—	6 lb	Slashing
% Waraxe, dwarven	30 gp	1d8	1d10	×3	—	8 lb	Slashing
% Two-Handed Melee Weapons
% Chain, spiked4	25 gp	1d6	2d4	×2	—	10 lb	Piercing
% Flail, dire	90 gp	1d6/1d6	1d8/1d8	×2	—	10 lb	Bludgeoning
% Sword, two-bladed	100 gp	1d6/1d6	1d8/1d8	19-20/×2	—	10 lb	Slashing
% Urgrosh, dwarven	50 cp	1d6 1d4	1d8 1d6	×3	—	12 lb	Slashing Piercing
% Ranged Weapons
% Shuriken (5)	1 gp	1	1d2	×2	10 ft.	\onehalf lb	Piercing

\section{Weapon Qualities}
Here is the format for weapon entries (given as column headings on Table: Weapons).

\textbf{Cost:} This value is the weapon’s cost in gold pieces (gp) or silver pieces (sp). The cost includes miscellaneous gear that goes with the weapon.

This cost is the same for a Small or Medium version of the weapon. A Large version costs twice the listed price.

\textbf{Damage:} The Damage columns give the damage dealt by the weapon on a successful hit. The column labeled ``Dmg (S)'' is for Small weapons. The column labeled "Dmg (M)" is for Medium weapons. If two damage ranges are given then the weapon is a double weapon. Use the second damage figure given for the double weapon’s extra attack. Table: Larger and Smaller Weapon Damage gives weapon damage values for weapons of various sizes.

\textbf{Critical:} The entry in this column notes how the weapon is used with the rules for critical hits. When your character scores a critical hit, roll the damage two, three, or four times, as indicated by its critical multiplier (using all applicable modifiers on each roll), and add all the results together.

\textit{Exception:} Extra damage over and above a weapon’s normal damage is not multiplied when you score a critical hit.

\textit{×2:} The weapon deals double damage on a critical hit.

\textit{×3:} The weapon deals triple damage on a critical hit.

\textit{×3/×4:} One head of this double weapon deals triple damage on a critical hit. The other head deals quadruple damage on a critical hit.

\textit{×4:} The weapon deals quadruple damage on a critical hit.

\textit{19-20/×2:} The weapon scores a threat on a natural roll of 19 or 20 (instead of just 20) and deals double damage on a critical hit. (The weapon has a threat range of 19-20.)

\textit{18-20/×2:} The weapon scores a threat on a natural roll of 18, 19, or 20 (instead of just 20) and deals double damage on a critical hit. (The weapon has a threat range of 18-20.)

\textbf{Range Increment:} Any attack at less than this distance is not penalized for range. However, each full range increment imposes a cumulative -2 penalty on the attack roll. A thrown weapon has a maximum range of five range increments. A projectile weapon can shoot out to ten range increments.

\textbf{Weight:} This column gives the weight of a Medium version of the weapon. Halve this number for Small weapons and double it for Large weapons.

\textbf{Type:} Weapons are classified according to the type of damage they deal: bludgeoning, piercing, or slashing. Some monsters may be resistant or immune to attacks from certain types of weapons.

Some weapons deal damage of multiple types. If a weapon is of two types, the damage it deals is not half one type and half another; all of it is both types. Therefore, a creature would have to be immune to both types of damage to ignore any of the damage from such a weapon.

In other cases, a weapon can deal either of two types of damage. In a situation when the damage type is significant, the wielder can choose which type of damage to deal with such a weapon.

\textbf{Special:} Some weapons have special features. See the weapon descriptions for details.

