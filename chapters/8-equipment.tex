\Chapter{Equipment}
{I have heard tales that suits of clothing fashioned from metal have even been found from time to time. It is generally agreed that these were worn by warriors to protect against the blows of enemy weapons. I can only speculate that the climate must have been far cooler in those ancient days. Any fool that would wear such clothing now would die faster from heat stroke than he would have from the weapons of his foes. Still, the idea that there was once enough metal in the world to allow such a garment to have been manufactured astounds me.

There are even rumors that mounds of steel, silver, and gold lie hidden in the deepest tunnels of certain forlorn cities. I have never seen such a thing myself, but if such treasures exist, they will reward those who find them most handsomely. Those who control such stores of metal can buy food, power, influence, and sometimes even the sorcerer-king's protection.}
{The Wanderer's Journal}

Dark Sun characters must be well equipped in order to endure the rigors of Athas. This chapter covers a variety of topics related to mundane equipment that every hero needs to survive and prosper.

% \section{Equipping a Character}
% The world of Athas has a very specific feel to it; many things that are taken for granted in other campaign worlds, like the availability of metal and water, are very different on this heat-wracked planet. To maintain this feel, the equipment available to characters should reflect these differences.

\section{Wealth and Money}
\subsection{Coins}
All prices in Dark Sun are given in terms of ceramic pieces, the most common coin. Ceramics are made from glazed clay and baked in batches once a year in a secure process supervised by the high templar that supervises the city's treasury. Bits are literally one‐tenth parts of a ceramic piece---the ceramic pieces break easily into ten bits. Some cities' ceramic pieces have small holes that can be threaded onto a bracelet or necklace. The lowest unit of Athasian trade is the lead bead (bd).

Each of the city-states of the Tablelands produces its own currency. All cities use ceramic pieces as the most common coin, but also mint silver coins and, in some cases, rare and highly prized gold coins.

\Table[\scriptsize]{Currency Conversions}{l C C C C}{
& \multicolumn{4}{c}{\tableheader Exchange Value}\\
& \tableheader bd & \tableheader bit & \tableheader cp & \tableheader sp\\
Lead bead (bd) & 1 & 1/10 & 1/100 & 1/1,000\\
Ceramic bit (bit) & 10 & 1 & 1/10 & 1/100\\
Ceramic piece (cp) & 100 & 10 & 1 & 1/10\\
Silver piece (sp) & 1,000 & 100 & 10 & 1\\
Gold piece (gp) & 10,000 & 1000 & 100 & 10\\
}

\subsubsection{Moneychangers}
Adventurers that travel between cities will need to change their currency for local currency at each city they visit. With a couple of exceptions, the city-states have moneychangers available for incoming visitors. Located near the city gates and in large market places, moneychangers denote their business by hanging a large purple banner from their shop. The banners are always purple, but the moneychangers in each city-state display a different emblem on the banner, based on their city's standard.

Moneychangers charge each customer a fee to change coins. The fees differ by city and are summarized on \tabref{Moneychangers}.

\Table{Moneychangers}{C C}{
\tableheader City & \tableheader Exchange Rate\\
Balic & 6\%\\
Draj & 8\%\\
Gulg & 10\%\\
Kurn & 16\%\\
Nibenay & 14\%\\
Raam & 12\%\\
Tyr & 12\%\\
Urik & 9\%\\
}

These fees are averages and may vary slightly. There are of course many unscrupulous money merchants who will charge as much as they can get away with. Moneychangers in Kurn are rare but a couple do exist. Since metal coins from any city-state are readily accepted by local merchants and no corresponding Kurnish coins exist there is little need to exchange such coins. There are, however, a few moneychangers willing to exchange ceramic pieces.

There are two cities that do not have moneychangers. Visitors to Celik have no need of a moneychangers, as merchants in that city take coins of all types. Nor are there any moneychangers in Eldaarich. Since Eldaarich has been shut off from the rest of the land for so long, visitors needing to exchange money have been nonexistent, so no moneychangers have set up business.

\subsection{Trade}
In general, the Athasian economy in the cities is relatively stable thanks to the Merchant Houses. Under normal conditions, supply is ample thanks to the caravans traveling back and forth between the cities. However, for smaller communities and trade outposts the price situation on certain goods can sway drastically. A raider attack or sandstorm can result in lack of necessities such as food and water, for which people will pay almost any amount of coin. Coins are not the only means of exchange. Barter and trade in commodities is widespread.

Dune traders commonly exchange trade goods without using currency, instead relying on a basic bartering system.


\Table{Trade Goods}{l X}{
\tableheader Cost & \tableheader Item\\
4 bits & One kilogram of salt\\
6 bits & One kilogram of grain or faro\\
1 cp & One kilogram of lead\\
2 cp & One kilogram of nuts, or one kilogram of kank nectar\\
4 cp & One square meter of cotton (cloth)\\
10 cp & One kilogram of obsidian, or one square meter of silk, or one metric ton of water, or one erdlu\\
50 cp & One herding kank, or one aprig\\
100 cp & One kilogram of copper, or one male carru, or one inix\\
200 cp &  One kilogram of iron, or one mekillot\\
300 cp & One female carru\\
1,000 cp & One kilogram of silver\\
10,000 cp & One kilogram of gold\\
}

\subsection{Selling Loot}
In general, a character can sell something for half its listed price.

Trade goods are the exception to the half-price rule. A trade good, in this sense, is a valuable good that can be easily exchanged almost as if it were cash itself.

\input{sections/8.2-weapons.tex}
\section{Armor}

While Athasian characters use all the varieties of armor, the armor they use incorporates materials commonly found in the world around them. Though most of the armors are made using various parts of common Athasian animals, the armor construction process makes use of several different reinforcement methods developed over time. Many of the armors are highly composite, made using the pieces of several different animals---no two suits of armor look quite alike. Through the use of hardening resins, shaped chitin and stiff leather backings, Athasian armorers can craft remarkably durable armors from the material at hand.

Thousands of years of tortuous heat have lead Athasian armorers to develop ingenious air ventilation and air circulation methods. This allows medium and heavy armors to be worn in the Athasian heat.


\begin{table*}[t!]
\caption{\label{tab:Armor and Shields}Armor and Shields}
\rowcolors{0}{TableColor}{}
\scriptsize
\begin{tabularx}{\textwidth}{X z{9mm}z{11mm}z{7mm}z{9mm}z{8mm}z{9mm}z{7mm}z{7mm}z{7mm}z{11mm}z{11mm}z{11mm}}
	\rowcolor{white}
& \multicolumn{2}{c}{\tableheader Cost}
& \multirow[b]{2}{7mm}{\centering \tableheader AC Bonus}
& \multirow[b]{2}{1cm}{\centering \tableheader Max. Dex Bonus}
& \multirow[b]{2}{8mm}{\centering \tableheader Armor Penalty}
& \multirow[b]{2}{9mm}{\centering \tableheader Failure Chance}
& \multicolumn{3}{c}{\tableheader Speed}
& \multicolumn{3}{c}{\tableheader Weight}\\
\cmidrule[0.5pt]{2-3}
\cmidrule[0.5pt]{8-13}

\footnotesize \tableheader Armor & \tableheader S/M & \tableheader L & & & & & \tableheader 6 m & \tableheader 9 m & \tableheader 12 m & \tableheader S & \tableheader M & \tableheader L\\

\multicolumn{13}{l}{\textit{Light Armor}}\\
Padded & 5 cp & 10 cp & +1 & +8 & +0 & 5\% & 6 m & 9 m & 12 m & 2.5 kg & 5 kg & 10 kg\\
Leather & 10 cp & 20 cp & +2 & +6 & +0 & 10\% & 6 m & 9 m & 12 m & 3.75 kg & 7.5 kg & 15 kg\\
Studded leather & 25 cp & 50 cp & +3 & +5 & $-1$ & 15\% & 6 m & 9 m & 12 m & 5 kg & 10 kg & 20 kg\\
Gladiator, light & 50 cp & 100 cp & +3 & +6 & $-1$ & 15\% & 6 m & 9 m & 12 m & 3 kg & 6 kg & 12 kg\\
Caravan & 75 cp & 150 cp & +4 & +3 & $-2$ & 20\% & 6 m & 9 m & 12 m & 4 kg & 8 kg & 16 kg\\
Chitin & 100 cp & 200 cp & +4 & +4 & $-2$ & 20\% & 6 m & 9 m & 12 m & 6.25 kg & 12.5 kg & 25 kg\\

\multicolumn{13}{l}{\textit{Medium Armor}}\\
Hide & 15 cp & 30 cp & +3 & +4 & $-3$ & 20\% & 4.5 m & 6 m & 9 m & 6.25 kg & 12.5 kg & 25 kg\\
Scale mail & 50 cp & 100 cp & +4 & +3 & $-4$ & 25\% & 4.5 m & 6 m & 9 m & 7.5 kg & 15 kg & 30 kg\\
Gladiator, medium & 100 cp & 200 cp & +4 & +5 & $-3$ & 20\% & 4.5 m & 6 m & 9 m & 4 kg & 8 kg & 16 kg\\
Shell & 150 cp & 300 cp & +5 & +2 & $-5$ & 30\% & 4.5 m & 6 m & 9 m & 10 kg & 20 kg & 40 kg\\
Breastplate & 200 cp & 400 cp & +5 & +3 & $-4$ & 25\% & 4.5 m & 6 m & 9 m & 7.5 kg & 15 kg & 30 kg\\

\multicolumn{13}{l}{\textit{Heavy armor}\footnotemark[1]}\\
Chitin warsuit & 165 cp & 330 cp & +5 & +1 & $-5$ & 40\% & 4.5 m & 6 m & 9 m & 6.25 kg & 12.5 kg & 25 kg\\
Splint mail & 200 cp & 400 cp & +6 & +0 & $-7$ & 40\% & 4.5 m & 6 m & 9 m & 11.25 kg & 22.5 kg & 45 kg\\
Banded mail & 250 cp & 500 cp & +6 & +1 & $-6$ & 35\% & 4.5 m & 6 m & 9 m & 8.75 kg & 17.5 kg & 35 kg\\
Tyrian warsuit & 410 cp & 810 cp & +6 & +1 & $-5$ & 30\% & 4.5 m & 6 m & 9 m & 10 kg & 20 kg & 40 kg\\
Half-plate & 6 gp & 12 gp & +7 & +0 & $-7$ & 40\% & 4.5 m & 6 m & 9 m & 12.5 kg & 25 kg & 50 kg\\
Full plate & 15 gp & 30 gp & +8 & +1 & $-6$ & 35\% & 4.5 m & 6 m & 9 m & 12.5 kg & 25 kg & 50 kg\\

\multicolumn{13}{l}{\textit{Shields}}\\
Buckler & 15 cp & 30 cp & +1 && $-1$ & 5\% &&&& 1.25 kg & 2.5 kg & 5 kg\\
Shield, light wooden & 3 cp & 6 cp & +1 && $-1$ & 5\% &&&& 1.25 kg & 2.5 kg & 5 kg\\
Shield, light steel & 9 gp & 18 gp & +1 && $-1$ & 5\% &&&& 1.5 kg & 3 kg & 6 kg\\
Shield, heavy wooden & 7 cp & 14 cp & +2 && $-2$ & 15\% &&&& 2.5 kg & 5 kg & 10 kg\\
Shield, heavy steel & 20 gp & 40 gp & +2 && $-2$ & 15\% &&&& 3.75 kg & 7.5 kg & 15 kg\\
Shield, long & 20 cp & 40 cp & +3 && $-4$ & 20\% &&&& 2.5 kg & 5 kg & 10 kg\\
Shield, tower & 30 cp & 60 cp & +4\footnotemark[2] & +2 & $-10$ & 50\% &&&& 11.25 kg & 22.5 kg & 
45 kg\\

\multicolumn{13}{l}{\textit{Extras}}\\
Armor spikes & +50 cp & +100 cp &&&& &&&& +2.5 kg & +5 kg & +10 kg\\
Gauntlet, locked & 8 cp & 16 cp &&& $\star$ & $\star$ &&&& +1.25 kg & +2.5 kg & +5 kg\\
Shield spikes & +10 cp & +20 cp &&&& &&&& +1.25 kg & +2.5 kg & +5 kg\\
\rowcolor{white}
\multicolumn{13}{l}{1 When running in heavy armor, you move only triple your speed, not quadruple.}\\
\rowcolor{white}
\multicolumn{13}{l}{2 A tower shield can instead grant you cover.}\\
\rowcolor{white}
\multicolumn{13}{l}{$\star$ Hand not free to cast spells or employ skills.}\\
\end{tabularx}
\end{table*}

\subsection{Armor Qualities}
To wear heavier armor effectively, a character can select the Armor Proficiency feats, but most classes are automatically proficient with the armors that work best for them.

Armor and shields can take damage from some types of attacks.

Here is the format for armor entries (given as column headings on \tabref{Armor and Shields}, below).

\textbf{Cost:} The cost of the armor for Small, Medium, or Large humanoid creatures. See Armor for Unusual Creatures, below, for armor prices for other creatures.

\textbf{Armor/Shield Bonus:} Each armor grants an armor bonus to AC, while shields grant a shield bonus to AC. The armor bonus from a suit of armor doesn't stack with other effects or items that grant an armor bonus. Similarly, the shield bonus from a shield doesn't stack with other effects that grant a shield bonus.

\textbf{Maximum Dex Bonus:} This number is the maximum Dexterity bonus to AC that this type of armor allows. Heavier armors limit mobility, reducing the wearer's ability to dodge blows. This restriction doesn't affect any other Dexterity-related abilities.

Even if a character's Dexterity bonus to AC drops to 0 because of armor, this situation does not count as losing a Dexterity bonus to AC.

Your character's encumbrance (the amount of gear he or she carries) may also restrict the maximum Dexterity bonus that can be applied to his or her Armor Class.

\textit{Shields:} Shields do not affect a character's maximum Dexterity bonus.

\textbf{Armor Check Penalty:} Any armor heavier than leather hurts a character's ability to use some skills. An armor check penalty number is the penalty that applies to \skill{Balance}, \skill{Climb}, \skill{Escape Artist}, \skill{Hide}, \skill{Jump}, \skill{Move Silently}, \skill{Sleight of Hand}, and \skill{Tumble} checks by a character wearing a certain kind of armor. Double the normal armor check penalty is applied to Swim checks. A character's encumbrance (the amount of gear carried, including armor) may also apply an armor check penalty.

\textit{Shields:} If a character is wearing armor and using a shield, both armor check penalties apply.

\textit{Nonproficient with Armor Worn:} A character who wears armor and/or uses a shield with which he or she is not proficient takes the armor's (and/or shield's) armor check penalty on attack rolls and on all Strength-based and Dexterity-based ability and skill checks. The penalty for nonproficiency with armor stacks with the penalty for nonproficiency with shields.

\textit{Sleeping in Armor:} A character who sleeps in medium or heavy armor is automatically fatigued the next day. He or she takes a -2 penalty on Strength and Dexterity and can't charge or run. Sleeping in light armor does not cause fatigue.

\textbf{Arcane Spell Failure:} Armor interferes with the gestures that a spellcaster must make to cast an arcane spell that has a somatic component. Arcane spellcasters face the possibility of arcane spell failure if they're wearing armor. Bards can wear light armor without incurring any arcane spell failure chance for their bard spells.

\textit{Casting an Arcane Spell in Armor:} A character who casts an arcane spell while wearing armor must usually make an arcane spell failure roll. The number in the Arcane Spell Failure Chance column on Table: Armor and Shields is the chance that the spell fails and is ruined. If the spell lacks a somatic component, however, it can be cast with no chance of arcane spell failure.

\textit{Shields:} If a character is wearing armor and using a shield, add the two numbers together to get a single arcane spell failure chance.

\textbf{Speed:} Medium or heavy armor slows the wearer down. The number on \tabref{Armor and Shields} is the character's speed while wearing the armor. When running in heavy armor, you move only triple your speed, not quadruple.

Elves, half-giants, and thri-kreen have an unencumbered speed of 12 meters. Humans, half-elves, muls, and pterrans have an unencumbered speed of 9 meters.

They use the first column. Aarakocras, dwarves, and halflings have an unencumbered speed of 6 meters. They use the second column. Remember, however, that a dwarf's land speed remains 6 meters even in medium or heavy armor or when carrying a medium or heavy load.

\textit{Shields:} Shields do not affect a character's speed.

\textbf{Weight:} This column gives the weight of the armor sized for a Medium wearer. Armor fitted for Small characters weighs half as much, and armor for Large characters weighs twice as much.

\subsection{Getting Into And Out Of Armor}
The time required to don armor depends on its type; see \tabref{Donning Armor}.

\textbf{Don:} This column tells how long it takes a character to put the armor on. (One minute is 10 rounds.) Readying (strapping on) a shield is only a move action.

\textbf{Don Hastily:} This column tells how long it takes to put the armor on in a hurry. The armor check penalty and armor bonus for hastily donned armor are each 1 point worse than normal.

\textbf{Remove:} This column tells how long it takes to get the armor off. Loosing a shield (removing it from the arm and dropping it) is only a move action.

\TransparentTable{Donning Armor}{l CCC}{
\tableheader Armor Type & \tableheader Don & \tableheader Don Hastily & \tableheader Remove\\
\rowcolor{TableColor}
Shield (any) & 1 move action & n/a & 1 move action\\
Padded & \multirow{6}{*}{1 minute} & \multirow{6}{*}{5 rounds} & \multirow{6}{*}{1 minute\footnotemark[1]}\\
Leather &&&\\
Hide &&&\\
Studded leather &&&\\
Chitin &&&\\
Chitin warsuit &&&\\
\rowcolor{TableColor}
Breastplate &&&\\
\rowcolor{TableColor}
Caravan &&&\\
\rowcolor{TableColor}
Light gladiator &&&\\
\rowcolor{TableColor}
Medium gladiator &&&\\
\rowcolor{TableColor}
Scale mail &&&\\
\rowcolor{TableColor}
Shell &&&\\
\rowcolor{TableColor}
Banded mail &&&\\
\rowcolor{TableColor}
Splint mail & \multirow{-8}{*}{4 minutes\footnotemark[1]} & \multirow{-8}{*}{1 minute} & \multirow{-8}{*}{1 minute\footnotemark[1]}\\
Half-plate & \multirow{3}{*}{4 minutes\footnotemark[2]} & \multirow{3}{*}{4 minutes\footnotemark[1]} & \multirow{3}{1.5cm}{\centering 1d4+1 minutes\footnotemark[1]}\\
Full plate &&&\\
Tyrian warsuit &&&\\
}
\TransparentTable{}{l X}{1 & If the character has some help, cut this time in half. A single character doing nothing else can help one or two adjacent characters. Two characters can't help each other don armor at the same time.\\
2 & The wearer must have help to don this armor. Without help, it can be donned only hastily.
}

\subsection{Armor For Unusual Creatures}
Armor and shields for unusually big creatures, unusually little creatures, and nonhumanoid creatures have different costs and weights from those given on \tabref{Armor and Shields}. Refer to the appropriate line on the table below and apply the multipliers to cost and weight for the armor type in question.

For creatures of size Tiny or smaller, divide armor bonus by 2.

\Table{}{l CCCC}{
 & \multicolumn{2}{c}{\tableheader Humanoid} & \multicolumn{2}{c}{\tableheader Nonhumanoid} \\
\cmidrule[0.5pt]{2-5}
\tableheader Size & \tableheader Cost & \tableheader Weight & \tableheader Cost & \tableheader Weight\\
Tiny or smaller & $\times$\onehalf & $\times$1/10 & $\times$1 & $\times$1/10\\
Small & $\times$1 & $\times$\onehalf & $\times$2 & $\times$\onehalf\\
Medium & $\times$1 & $\times$1 & $\times$2 & $\times$1\\
Large & $\times$2 & $\times$2 & $\times$4 & $\times$2\\
Huge & $\times$4 & $\times$5 & $\times$8 & $\times$5\\
Gargantuan & $\times$8 & $\times$8 & $\times$16 & $\times$8\\
Colossal & $\times$16 & $\times$12 & $\times$32 & $\times$12\\
}

\subsection{Armor Descriptions}
Any special benefits or accessories to the types of armor found on \tabref{Armor and Shields} are described below.

\textbf{Armor Spikes:} You can have spikes added to your armor, which allow you to deal extra piercing damage on a successful grapple attack. The spikes count as a martial weapon. If you are not proficient with them, you take a -4 penalty on grapple checks when you try to use them. You can also make a regular melee attack (or off-hand attack) with the spikes, and they count as a light weapon in this case. (You can't also make an attack with armor spikes if you have already made an attack with another off-hand weapon, and vice versa.)\\An enhancement bonus to a suit of armor does not improve the spikes' effectiveness, but the spikes can be made into magic weapons in their own right.

\textbf{Banded Mail:} The suit includes gauntlets.

\textbf{Breastplate:} It comes with a helmet and greaves.

\textbf{Buckler:} This small metal shield is worn strapped to your forearm. You can use a bow or crossbow without penalty while carrying it. You can also use your shield arm to wield a weapon (whether you are using an off-hand weapon or using your off hand to help wield a two-handed weapon), but you take a -1 penalty on attack rolls while doing so. This penalty stacks with those that may apply for fighting with your off hand and for fighting with two weapons. In any case, if you use a weapon in your off hand, you don't get the buckler's AC bonus for the rest of the round.

You can't bash someone with a buckler.

\textbf{Caravan Armor:} This suit of armor is a combination of several different materials. Thick chitin bracers provide efficient protection to the forearms, while thick leather protects the shins and knees. A leather kilt and and shirt of thick cord layers protect the body and provide sufficient cooling. This armor is so named because it's mostly used by caravan guards who need decent protection while not being slowed down by their armor. Light caravan armor comes with a turban made of thick cord.

This armor doesn't provide full-body protection and thus the wearer is more prone to critical hits; the AC against rolls to confirm a critical hit is reduced by 1.

\textbf{Chitin Warsuit:} This suit of armor comes with padded armor, which is worn beneath the actual armor, to prevent abrasions. A long shell shirt covers the torso and the waist, chitin sleeves over both arms and shoulders and end in chitin gauntlets, long chitin pants cover the legs, and a bone or chitin helmet, usually made of a creature's skull or head exoskeleton, covers the head. This armor offers good protection, but brings the usual problems with heat accumulation.

\textbf{Chitin:} This armor is skillfully made by interlocking hexagonal bits of chitin (usually carved from a kank's carapace).

A chitin armor comes with a chitin cap.

\textbf{Full Plate:} The suit includes gauntlets, heavy leather boots, a visored helmet, and a thick layer of padding that is worn underneath the armor. Each suit of full plate must be individually fitted to its owner by a master armorsmith, although a captured suit can be resized to fit a new owner at a cost of 200 to 800 (2d4$\times$100) ceramic pieces.

\textbf{Gauntlet, Locked:} This armored gauntlet has small chains and braces that allow the wearer to attach a weapon to the gauntlet so that it cannot be dropped easily. It provides a +10 bonus on any roll made to keep from being disarmed in combat. Removing a weapon from a locked gauntlet or attaching a weapon to a locked gauntlet is a full-round action that provokes attacks of opportunity.

The price given is for a single locked gauntlet. The weight given applies only if you're wearing a breastplate, light armor, or no armor. Otherwise, the locked gauntlet replaces a gauntlet you already have as part of the armor.

While the gauntlet is locked, you can't use the hand wearing it for casting spells or employing skills. (You can still cast spells with somatic components, provided that your other hand is free.)

Like a normal gauntlet, a locked gauntlet lets you deal lethal damage rather than nonlethal damage with an unarmed strike.

\textbf{Half-Plate:} The suit includes gauntlets.

\textbf{Light Gladiator Armor:} This suit of armor combines leather and bone to provide the gladiator with decent protection and minimal hindrance. Thick leather shinpads provide leg protection whithout hampering movement, while a breastplate of bone and leather skirt or loincoth protects the  gladiator's torso. A bone helmet protect the head and face, and a cuff of thick leather protects the gladiator's weapon hand. Gladiators that rely on high maneuverability prefer this kind of of armor; masterwork suits are highly desired and respected.

This armor's lightweight and area-specific coverage provides many openings for critical hits; thus, the AC against rolls to confirm critical hits is reduced by 2.

\textbf{Medium Gladiator Armor:} This suit of armor combines leather and chitin to provide the gladiator with good protection without hampering his freedom of movement too much. A vambrace made of chitin covers the gladiator's weapon arm and is held in place by a leather corselet, while a shoulder plate of chitin covers the off-hand shoulder. A thick leather skirt protects the gladiator's haunch and chitin shinpads protect his tibia. This suit of picemeal armor comes with a chitin helm that usually resembles a beast's head.

This armor provides many openings for critical hits; therefore, the AC against rolls to confirm critical hits is reduced by 2.

\textbf{Scale Mail:} Scale mail is usually made from the scales of an erdlu, inix or other naturally scaled creatures.

The suit includes gauntlets.

\textbf{Shell:} Shell armor is made by weaving giant's hair around the shells of various small creatures such as an aprig.

The suit includes gauntlets.

\textbf{Shield, Heavy:} You strap a shield to your forearm and grip it with your hand. A heavy shield is so heavy that you can't use your shield hand for anything else.

\textit{Wooden or Steel:} Wooden and steel shields offer the same basic protection, though they respond differently to special attacks.

\textit{Shield Bash Attacks:} You can bash an opponent with a heavy shield, using it as an off-hand weapon. See \tabref{Martial Weapons} for the damage dealt by a shield bash. Used this way, a heavy shield is a martial bludgeoning weapon. For the purpose of penalties on attack rolls, treat a heavy shield as a one-handed weapon. If you use your shield as a weapon, you lose its AC bonus until your next action (usually until the next round). An enhancement bonus on a shield does not improve the effectiveness of a shield bash made with it, but the shield can be made into a magic weapon in its own right.

\textbf{Shield, Light:} You strap a shield to your forearm and grip it with your hand. A light shield's weight lets you carry other items in that hand, although you cannot use weapons with it.

\textit{Wooden or Steel:} Wooden and steel shields offer the same basic protection, though they respond differently to special attacks.

\textit{Shield Bash Attacks:} You can bash an opponent with a light shield, using it as an off-hand weapon. See \tabref{Martial Weapons} for the damage dealt by a shield bash. Used this way, a light shield is a martial bludgeoning weapon. For the purpose of penalties on attack rolls, treat a light shield as a light weapon. If you use your shield as a weapon, you lose its AC bonus until your next action (usually until the next round). An enhancement bonus on a shield does not improve the effectiveness of a shield bash made with it, but the shield can be made into a magic weapon in its own right.

\textbf{Shield, Long:} This is a slim, two-handed shield commonly used by the kreen races of the northern kreen Empire; it is extremely rare to find a long shield in the hands of a nomadic kreen of the Tablelands, although they are occasionally spotted in the arena. Kreen usually hold the long shield with two arms from the same side. Long shields are made of bone, chitin, hide, or wood.

You need two hands to use a long shield. Two handed humanoids who use a long shield can do so by using it horizontally, but by doing so you cannot wield a weapon.

\textit{Shield Bash Attacks:} You can bash an opponent with a long shield, using it as an off-hand weapon. See \tabref{Martial Weapons} in the Player's Handbook for the damage dealt by a shield bash. Used this way, a long shield is a martial bludgeoning weapon. For the purpose of penalties on attack rolls, treat a long shield as a two-handed weapon. If you use your shield as a weapon, you lose its AC bonus until your next action (usually until the next round). An enhancement bonus on a shield does not improve the effectiveness of a shield bash made with it, but the shield can be made into a magic weapon in its own right.

\textbf{Shield, Tower:} This massive wooden shield is nearly as tall as you are. In most situations, it provides the indicated shield bonus to your AC. However, you can instead use it as total cover, though you must give up your attacks to do so. The shield does not, however, provide cover against targeted spells; a spellcaster can cast a spell on you by targeting the shield you are holding. You cannot bash with a tower shield, nor can you use your shield hand for anything else.

When employing a tower shield in combat, you take a -2 penalty on attack rolls because of the shield's encumbrance.

\textbf{Shield Spikes:} When added to your shield, these spikes turn it into a martial piercing weapon that increases the damage dealt by a shield bash as if the shield were designed for a creature one size category larger than you. You can't put spikes on a buckler or a tower shield. Otherwise, attacking with a spiked shield is like making a shield bash attack.

An enhancement bonus on a spiked shield does not improve the effectiveness of a shield bash made with it, but a spiked shield can be made into a magic weapon in its own right.

\textbf{Splint Mail:} The suit includes gauntlets.

\textbf{Tyrian Warsuit:} This armor combines metal and chitin. A chitin breatplate covers the front, back, shoulders and upper arms, while a long shell skirt protects the haunch. Metal shinpads, padded on the inside, are worn over leather boots, to avoid burns, while metal gauntlets, also padded on the inside, are worn over leather cuffs. This suit of armor comes with a full chitin helmet.

\subsection{Masterwork Armor}
Just as with weapons, you can purchase or craft masterwork versions of armor or shields. Such a well-made item functions like the normal version, except that its armor check penalty is lessened by 1.

A masterwork suit of armor or shield costs an extra 150 gp over and above the normal cost for that type of armor or shield.

The masterwork quality of a suit of armor or shield never provides a bonus on attack or damage rolls, even if the armor or shield is used as a weapon.

All magic armors and shields are automatically considered to be of masterwork quality.

You can't add the masterwork quality to armor or a shield after it is created; it must be crafted as a masterwork item.
\section{Goods and Services}
The world of Athas has a very specific feel to it; many things that are taken for granted in other campaign worlds, like the availability of metal and water, are very different on this heat-wracked planet. To maintain this feel, the equipment available to characters should reflect these differences.

\input{subsections/8.4-adventuring-gear.tex}
\subsection{Special Substances And Items}

\ItemTable{Special Substances and Items}{
Acid (flask) & 10 cp & 0.5 kg\\
Alchemist's fire (flask) & 20 cp & 0.5 kg\\
Antitoxin (vial) & 50 cp &\\
Balican sting & 5 Cp & 0.5 kg\\
Chitin ointment & 40 Cp & 0.5 kg\\
Draxia ointment & 20 Cp & 0.5 kg\\
Esperweed & 250 cp &\\
Everburning torch & 110 cp & 0.5 kg\\
Holy water (flask) & 25 cp & 0.5 kg\\
Hypnotic brew & 30 cp & 0.5 kg\\
Ignan tallgrass & 100 Cp &\\
Kuzza powder & 20 Cp &\\
Ranike sap (1 liter) & 2 Cp & 0.5 kg\\
Smokestick & 20 cp & 0.25 kg\\
\multicolumn{3}{l}{\textit{Splash-globe}}\\
~ Acid & 10 cp &\\
~ Kip pheromones & 30 cp &\\
~ Liquid darkness & 10 cp &\\
~ Liquid dust & 10 cp &\\
~ Liquid fire & 10 cp &\\
~ Liquid light & 10 cp &\\
~ Poison & Poison cost $\times$ 1.5 &\\
~ Ranike sap smoke & 10 cp &\\
~ Stench cloud & 50 cp &\\
~ Stun cloud & 35 cp &\\
Sunrod & 2 cp & 0.5 kg\\
Tanglefoot bag & 50 cp & 2 kg\\
Thunderstone & 30 cp & 0.5 kg\\
Tindertwig & 1 cp &\\
}

The following items are often, but not always available for sale in the Bard's Quarter of most city‐states. Contacting someone willing to sell these and other associated goods usually requires proficient use of the \skill{Bluff}, \skill{Diplomacy}, and/or \skill{Gather Information} skills.

Any of these substances except for the everburning torch and holy water can be made by a character with the \skill{Craft} (alchemy) skill.


\textbf{Acid:} You can throw a flask of acid as a splash weapon. Treat this attack as a ranged touch attack with a range increment of 10 feet. A direct hit deals 1d6 points of acid damage. Every creature within 5 feet of the point where the acid hits takes 1 point of acid damage from the splash.

\textbf{Alchemist's Fire:} You can throw a flask of alchemist's fire as a splash weapon. Treat this attack as a ranged touch attack with a range increment of 10 feet.

A direct hit deals 1d6 points of fire damage. Every creature within 5 feet of the point where the flask hits takes 1 point of fire damage from the splash. On the round following a direct hit, the target takes an additional 1d6 points of damage. If desired, the target can use a full-round action to attempt to extinguish the flames before taking this additional damage. Extinguishing the flames requires a DC 15 Reflex save. Rolling on the ground provides the target a +2 bonus on the save. Leaping into a lake or magically extinguishing the flames automatically smothers the fire.

\textbf{Antitoxin:} If you drink antitoxin, you get a +5 alchemical bonus on Fortitude saving throws against poison for 1 hour.

\textbf{Balican Sting:} This mixture of many vegetal irritants is used in conjunction with the flint-tipped javelin of the Balican fleet. Bards working for the late king Andropinis developed the substance to improve the damage done by his warriors fighting against the thick-skinned giants. This mixture, which is only effective against giants of the beasthead, crag, desert, or plains variety, causes the wound made by a balican javelin that breaks within it to itch. Unless a DC 15 Wisdom check is made by the giant on each of the following 1d4 rounds, he will scratch and inadvertently rub the shallow shards deeper, causing an additional 1d4 points of damage for each failed check.

\textbf{Chitin Ointment:} This salve is used to cure damaged chitin on kreens and other insectoid creatures. Once applied, as a standard action, this substance mends brittle or broken chitin, effectively stabilizing the creature if it had less than 0 hit points. Applying this substance to non-chitinous creatures produces no effects.

\textbf{Draxia Ointment:} The draxia weed grows on the islands of the Sea of Silt. It can be turned into an ointment that repels silt spawn by mixing the plant's juices with oil or fat. The ointment, when applied to the skin, emits a smell that repels silt spawn for two hours. Silt spawn will not come within 10 feet of a creature or object coated with draxia ointment. Although adult silt horrors find the smell irritating, they are usually unaffected by it. Sometimes silt horrors are irritated to such a level, however, that they may attack the creature or object giving off the smell. There is a 60\% chance that a silt horror will ignore all other targets and instead attack a character or object that smells of draxia weed.

\textbf{Everburning Torch:} This otherwise normal torch has a continual flame spell cast upon it. An everburning torch clearly illuminates a 20-foot radius and provides shadowy illumination out to a 40-foot radius.

\textbf{Holy Water:} Holy water damages undead creatures and evil outsiders almost as if it were acid. A flask of holy water can be thrown as a splash weapon.

Treat this attack as a ranged touch attack with a range increment of 3 meters. A flask breaks if thrown against the body of a corporeal creature, but to use it against an incorporeal creature, you must open the flask and pour the holy water out onto the target. Thus, you can douse an incorporeal creature with holy water only if you are adjacent to it. Doing so is a ranged touch attack that does not provoke attacks of opportunity.

A direct hit by a flask of holy water deals 2d4 points of damage to an undead creature or an evil outsider. Each such creature within 1.5 meter of the point where the flask hits takes 1 point of damage from the splash.

Temples to good deities sell holy water at cost (making no profit).

\textbf{Ignan Tallgrass:} A redish plant that grows in the Burning Plains near the Last Sea, ignan tallgrass can be harvested from the plains after flashfires, when they are easily spotted in small clumps untouched by the fires. Ignan tallgrass is tough and can be used to make mats and roofs of twinned fibers that stay fireproof for several months, if the harvesters are brave enough to face the flashfires to get to it, as the plant cannot be cultivated. If ignan tallgrass is sun-dried, crushed, and ingested within a week of it being picked, unless somehow magically kept fresh (as through the nurturing seeds spell), it confers resistance to fire 1 for one hour.

\textbf{Kuzza Powder:} Kuzza peppers are very hot. Typically, these vivid red peppers, when ripe, measure 2 to 2 1/2 inches long. These peppers are sometimes dried and ground into a powder by unscrupulous gladiators who use a blowpipe to blow the powder on a target, causing sever irritation. Treat this blowpipe as a blowgun with half the range increment. Filling a blowpipe is a move action that provokes attacks of opportunity. A direct hit blinds a creature for 1 round unless it makes a Fortitude DC 15. Every creature within 5 feet of the target takes a -2 penalty to Search and Spot checks for 5 rounds.

\textbf{Ranike Sap:} The sap of the ranike tree, which constantly runs down its bark, is toxic to insects. Gulg posesses the secret of safely extracting large quantities of sap from this tree, effectively milking the tree in a process called “bleeding”. If a liter of the sap is poured in a large receptacle, such as a brazier, and lit afire, a clear smoke that impairs neither vision or breathing forms, filling a 50-foot cube (a moderate or stronger wind dissipates the smoke in 5 rounds). The smoke repels mundane insects, while giant insects, or those creatures that can be categorised as insect-like (such as antloids, kanks, and thri-kreen), that breathe or contact the smoke must make a DC 15 Fortitude save each round for one minute; failure indicates that they are sickened for that round. The sap burns 1 hour for each liter of sap in the receptacle, after which the smoke dissipates naturally.

A shallow depression in the ground several feet wide can replace the need for a receptacle. The sap can also be used to deliniate an area---each liter poured on the ground can create a line a few inches wide and 10 feet long. When such a line is set afire, it burns for 1 minute and creates smoke in an area 10 feet long by 5 feet wide and high.

\textbf{Smokestick:} This alchemically treated wooden stick instantly creates thick, opaque smoke when ignited. The smoke fills a 10-foot cube (treat the effect as a fog cloud spell, except that a moderate or stronger wind dissipates the smoke in 1 round). The stick is consumed after 1 round, and the smoke dissipates naturally.

\textbf{Splash-globes:} Splash-globes are spherical glass jars containing contact poison or up to half a pint of some alchemical fluid. In addition to bursting on impact like any grenade, splash-globes can be placed in hinged pelota, thus giving the grenade additional range when fired through a splash-bow or dejada. The following types of splash-globes are available:

 \textit{Acid:} Standard flask acid can be placed in splash-globes.

 \textit{Contact Poison:} Any contact poison can be placed in a splash-globe.

 \textit{Kip Pheromones:} This splash-globe is commonly crafted by bards using kip pheromones collected by dwarven kip herders. The liquid contained within the globe is an alchemical mixture that turns into smoke on contact with air. The smoke produced is clear and does not impair vision or breathing, filling a 10-foot cube for one minute (a moderate or stronger wind dissipates the smoke in 1 round). Those within the smoke must make a DC 15 Fortitude save each round they are in contact with it or become fascinated for the as long as the smoke remains. Dwarves gain a +4 racial bonus on their Fortitude save against kip pheromones.

 \textit{Liquid Darkness:} Anyone struck directly by liquid darkness must make a Reflex save (DC 15) or be blinded for one minute. Those splashed with liquid darkness have their vision blurred for one minute if they fail a DC 15 Reflex save, granting their opponents concealment. In addition, all natural fires within the splash area are instantly extinguished. Liquid darkness immediately extinguishes liquid light.

 \textit{Liquid Dust:} The liquid from this splash-globe turns into dust on contact with the air. You can use this liquid to cover up to 20 1.5-meter squares of tracks. On impact, liquid dust forms a 4.5-meter diameter cloud, ten feet high that lasts one round. Alternately, liquid dust can be launched via slash-globes. Anyone struck directly by liquid dust must make a DC 15 Fortitude save each round for one minute; failure dictates that they are nauseated for that round. Those splashed with liquid dust suffer the same effect for one round if they fail a DC 15 Fortitude save.

 \textit{Liquid Fire:} Alchemist's fire can be placed in splash-globes.

 \textit{Liquid Light:} This splash-globe contains two liquids that mix together when the splash-globe is ruptured. The resulting mixture glows for eight hours. If you break the liquid light globe while it is still in its pouch, the pouch can serve as a light source just like a sunrod. Anyone struck directly by liquid light must make a DC 20 Fortitude save or be temporarily dazzled (–1 on all attack rolls) for 1 minute, and will glow in darkness for eight hours unless they somehow cover the affected areas. Creatures splashed with liquid light (see grenade rules) also glow in the darkness, but are not blinded.

 \textit{Ranike Sap Smoke:} The liquid from this splash-globe is an alchemical mixture of ranike sap that turns into smoke on contact with air. The smoke produced is clear and does not impair vision or breathing, filling a 10-foot cube (a moderate or stronger wind dissipates the smoke at the end of the character's action). The smoke repels mundane insects, while giant insects, or those creatures that can be categorised as insect-like (such as antloids, kanks, and thri-kreens), that breath or enter in contact with the smoke must make a DC 15 Fortitude save each round for one minute; failure indicates that they are sickened for that round. This small quantity of sap only reacts with the air for 1 round, after which the smoke dissipates naturally.

 \textit{Stench Cloud:} The liquid inside this splash-globe is crafted from fordorran musk and stinkweed extract. The foul liquid turns into smoke on contact with air. The smoke produced is clear and does not impair vision or breathing, filling a 10-foot cube for one minute (a moderate or stronger wind dissipates the smoke in 1 round). Those within the smoke must make a DC 15 Fortitude save each round they are in contact with it or become nauseated for as long as they remain in contact with the cloud.

 \textit{Stun Cloud:} The liquid inside this splash-globe is crafted from boiled floater jelly combined with the pulped spines from a poisonous cactus. The liquid turns into smoke on contact with air. The smoke produced is clear and does not impair vision or breathing, filling a 10-foot cube for one minute (a moderate or stronger wind dissipates the smoke in 1 round). Those within the smoke must make a DC 15 Fortitude save each round they are in contact with it or become stunned for as long as they remain in contact with the cloud.

\textbf{Sunrod:} This 1-foot-long, gold-tipped, iron rod glows brightly when struck. It clearly illuminates a 30-foot radius and provides shadowy illumination in a 60-foot radius. It glows for 6 hours, after which the gold tip is burned out and worthless.

\textbf{Tanglefoot Bag:} When you throw a tanglefoot bag at a creature (as a ranged touch attack with a range increment of 10 feet), the bag comes apart and the goo bursts out, entangling the target and then becoming tough and resilient upon exposure to air. An entangled creature takes a -2 penalty on attack rolls and a -4 penalty to Dexterity and must make a DC 15 Reflex save or be glued to the floor, unable to move. Even on a successful save, it can move only at half speed. Huge or larger creatures are unaffected by a tanglefoot bag. A flying creature is not stuck to the floor, but it must make a DC 15 Reflex save or be unable to fly (assuming it uses its wings to fly) and fall to the ground. A tanglefoot bag does not function underwater.

A creature that is glued to the floor (or unable to fly) can break free by making a DC 17 Strength check or by dealing 15 points of damage to the goo with a slashing weapon. A creature trying to scrape goo off itself, or another creature assisting, does not need to make an attack roll; hitting the goo is automatic, after which the creature that hit makes a damage roll to see how much of the goo was scraped off. Once free, the creature can move (including flying) at half speed. A character capable of spellcasting who is bound by the goo must make a DC 15 Concentration check to cast a spell. The goo becomes brittle and fragile after 2d4 rounds, cracking apart and losing its effectiveness. An application of universal solvent to a stuck creature dissolves the alchemical goo immediately.

\textbf{Thunderstone:} You can throw this stone as a ranged attack with a range increment of 20 feet. When it strikes a hard surface (or is struck hard), it creates a deafening bang that is treated as a sonic attack. Each creature within a 10-foot-radius spread must make a DC 15 Fortitude save or be deafened for 1 hour. A deafened creature, in addition to the obvious effects, takes a -4 penalty on initiative and has a 20\% chance to miscast and lose any spell with a verbal component that it tries to cast.

Since you don't need to hit a specific target, you can simply aim at a particular 5-foot square. Treat the target square as AC 5.

\textbf{Tindertwig:} The alchemical substance on the end of this small, wooden stick ignites when struck against a rough surface. Creating a flame with a tindertwig is much faster than creating a flame with flint and steel (or a magnifying glass) and tinder. Lighting a torch with a tindertwig is a standard action (rather than a full-round action), and lighting any other fire with one is at least a standard action.
\input{subsections/8.4-metaempiric-components.tex}
\input{subsections/8.4-psychoactive-components.tex}
\input{subsections/8.4-tools-and-kits.tex}
\input{subsections/8.4-clothing.tex}

% These items weigh one-quarter this amount when made for Small characters. Containers for Small characters also carry one-quarter the normal amount.
