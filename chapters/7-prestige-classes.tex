Prestige classes offer a new form of multiclassing. Unlike the basic classes, characters must meet Requirements before they can take their first level of a prestige class. The rules for level advancement apply to this system, meaning the first step of advancement is always choosing a class. If a character does not meet the Requirements for a prestige class before that first step, that character cannot take the first level of that prestige class. Taking a prestige class does not incur the experience point penalties normally associated with multiclassing.

% The prestige classes in this chapter are designed for characters in an Athasian campaign. They make use of defiling rules, the characters classes described in Chapter 2, or allow characters to further develop their elemental powers.

In addition to the prestige classes presented in this chapter, characters in a {\tableheader Dark Sun} campaign can adopt any prestige classes---with the DM's permission---from other sources as well. This chapter is not a comprehensive guide to the prestige classes on Athas. Check the \href{http://athas.org/products/prc1}{Prestige Class Appendix I} and \href{http://athas.org/products/prc2}{Prestige Class Appendix II} for more Athasian prestige classes.

\subsection{Definitions of Terms}
Here are definitions of some terms used in this section.

\textbf{Base Class:} One of the standard classes.

\textbf{Caster Level:} Generally equal to the number of class levels (see below) in a spellcasting class. Some prestige classes add caster levels to an existing class.

\textbf{Character Level:} The total level of the character, which is the sum of all class levels held by that character.

\textbf{Class Level:} The level of a character in a particular class. For a character with levels in only one class, class level and character level are the same.

% \textbf{Arcane Trickster:} 

% \textbf{Arch Defiler:} 

% \textbf{Archmage:} 

% \textbf{Arena Champion:} 

% \textbf{Executioner:} 


% \input{sections/7-arcane-archer.tex}
\PrestigeClass{Arcane Trickster}
{We all got tricks up in our sleves, don't we?}{Ulaam, elf arcane trickster}
{Arcane tricksters are magical spies and infiltrators who try to find a balance between their skills and their spellcasting, using their magic to hide, steal and murder more effectively.}
{d4}
{an}
{Arcane tricksters are mostly bards or rogues that have learned wizardry to get better at their jobs, but it is not uncommon for assassins to become arcane tricksters as well.}
{
\textbf{Alignment:} Any nonlawful.

\textbf{Skills:} \skill{Decipher Script} 7 ranks, \skill{Disable Device} 7 ranks, \skill{Escape Artist} 7 ranks, \skill{Knowledge} (arcana) 4 ranks.

\textbf{Spells:} Ability to cast \spell{mage hand} and at least one arcane spell of 3rd level or higher.

\textbf{Special:} Sneak attack +2d6.
}
{\skill{Appraise} (Int), \skill{Balance} (Dex), \skill{Bluff} (Cha), \skill{Climb} (Str), \skill{Concentration} (Con), \skill{Craft} (Int), \skill{Decipher Script} (Int), \skill{Diplomacy} (Cha), \skill{Disable Device} (Int), \skill{Disguise} (Cha), \skill{Escape Artist} (Dex), \skill{Gather Information} (Cha), \skill{Hide} (Dex), \skill{Jump} (Str), \skill{Knowledge} (all skills taken individually) (Int), \skill{Listen} (Wis), \skill{Move Silently} (Dex), \skill{Open Lock} (Dex), \skill{Profession} (Wis), \skill{Search} (Int), \skill{Sense Motive} (Wis), \skill{Sleight of Hand} (Dex), \skill{Speak Language} (None), \skill{Spellcraft} (Int), \skill{Spot} (Wis), \skill{Swim} (Str), \skill{Tumble} (Dex), and \skill{Use Rope} (Dex).
}
{4}
{\PrestigeSpellTable}{
1 & +0 & +0 & +2 & +2 & Ranged legerdemain 1/day & +1 level of existing class\\
2 & +1 & +0 & +3 & +3 & Sneak attack +1d6 & +1 level of existing class\\
3 & +1 & +1 & +3 & +3 & Impromptu sneak attack 1/day & +1 level of existing class\\
4 & +2 & +1 & +4 & +4 & Sneak attack +2d6 & +1 level of existing class\\
5 & +2 & +1 & +4 & +4 & Ranged legerdemain 2/day & +1 level of existing class\\
6 & +3 & +2 & +5 & +5 & Sneak attack +3d6 & +1 level of existing class\\
7 & +3 & +2 & +5 & +5 & Impromptu sneak attack 2/day & +1 level of existing class\\
8 & +4 & +2 & +6 & +6 & Sneak attack +4d6 & +1 level of existing class\\
9 & +4 & +3 & +6 & +6 & Ranged legerdemain 3/day & +1 level of existing class\\
10 & +5 & +3 & +7 & +7 & Sneak attack +5d6 & +1 level of existing class\\
}

\textbf{Weapon and Armor Proficiency:} Arcane tricksters gain no proficiency with any weapon or armor.

\textbf{Spells per Day:} When a new arcane trickster level is gained, the character gains new spells per day as if he had also gained a level in a spellcasting class he belonged to before adding the prestige class. He does not, however, gain any other benefit a character of that class would have gained, except for an increased effective level of spellcasting. If a character had more than one spellcasting class before becoming an arcane trickster, he must decide to which class he adds the new level for purposes of determining spells per day.

\textbf{Ranged Legerdemain:} An arcane trickster can perform one of the following class skills at a range of 30 feet: \skill{Disable Device}, \skill{Open Lock}, or \skill{Sleight of Hand}. Working at a distance increases the normal skill check DC by 5, and an arcane trickster cannot take 10 on this check. Any object to be manipulated must weigh 5 pounds or less.

An arcane trickster can use ranged legerdemain once per day initially, twice per day upon attaining 5th level, and three times per day at 9th level or higher. He can make only one ranged legerdemain skill check each day, and only if he has at least 1 rank in the skill being used.

\textbf{Sneak Attack:} This is exactly like the rogue ability of the same name. The extra damage dealt increases by +1d6 every other level (2nd, 4th, 6th, 8th, and 10th). If an arcane trickster gets a sneak attack bonus from another source the bonuses on damage stack.

\textbf{Impromptu Sneak Attack:} Beginning at 3rd level, once per day an arcane trickster can declare one melee or ranged attack he makes to be a sneak attack (the target can be no more than 30 feet distant if the impromptu sneak attack is a ranged attack). The target of an impromptu sneak attack loses any Dexterity bonus to AC, but only against that attack. The power can be used against any target, but creatures that are not subject to critical hits take no extra damage (though they still lose any Dexterity bonus to AC against the attack).

At 7th level, an arcane trickster can use this ability twice per day.

% \subsection{Playing an Arcane Trickster}
% You are a jack-of-all-trades, able to cast arcane spells and having a high skill count. Stealth? Diplomacy? Pure arcane power? You got it all. 

% \subsubsection{Combat}
% Your one-to-one combat skills probably haven't got you so far, so you need to depend on your spells so you can deal the final blow. In combat, your spellcasting is best used for protection, diversion, and mobility. You need to be at the right place at the right time.

% \subsubsection{Advancement}
% Arcane tricksters have all shapes and stories. Defilers, preservers, assassins. Elfs, humans, muls. While they all don't come from the same background, the need (or desire) to use arcane spells to improve thier skills is what is common between all arcane tricksters. They can use their magic to become stealthier and infiltrate better, or become stronger and fight better, or become faster and flee better.

% \subsubsection{Resources}
% In order to improve your spellcasting, you search arcane knowledge while also 


% \subsection{Arcane Tricksters on Athas}
% \Quote{}{}


% \subsubsection{Organization}

% \subsubsection{NPC Reactions}

% \subsubsection{Arcane Trickster Lore}
\subsection{Arcane Trickster Lore}
Characters with ranks in \skill{Knowledge} (arcana) or \skill{Knowledge} (local) can research arcane tricksters to learn more about them. When a character makes a skill check, read or paraphrase the following, including the information from lower DCs.

\textbf{DC 15:} The arcane trickster is a rogue who became an arcane caster.

\textbf{DC 20:} Arcane tricksters are masters of using \spell{mage hand}, able to open locked doors at a distance.

\textbf{DC 30:} Arcane tricksters learn to distract their foes using their magic, leaving them prone for a fatal attack.
\PrestigeClass{Arch Defiler}
{Power comes at a price. I am willing to pay it.}{Marakesh, human arch defiler}
{Arch defilers are defilers who seek to increase the power of their magic at the cost of a greater taint of defilement. So foul is the magic commanded by these defilers, that their very souls are scarred. Animals become nervous, people feel uncomfortable, and yet the arch defiler demands obedience. To command their foul magics, arch defilers need physical stamina to resist the vast energies they manipulate.}
{d4}
{an}
{Almost all arch defilers are wizards corrupted by their desire for power. Sometimes an adept becomes powerful enough to become one, but this is rare to say the best.}
{\textbf{Skills:} \skill{Knowledge} (arcana) 8 ranks, \skill{Spellcraft} 8 ranks.

\textbf{Feats:} \feat{Agonizing Radius}, \feat{Great Fortitude}, any metamagic feat.

\textbf{Spells:} Able to cast 3rd level arcane spells.

\textbf{Special:} Must be a defiler.}
{\skill{Bluff} (Cha), \skill{Concentration} (Con), \skill{Craft} (Int), \skill{Decipher Script} (Int), \skill{Disguise} (Cha), \skill{Intimidate} (Cha), \skill{Knowledge} (all skills individually) (Int), \skill{Profession} (Wiz), \skill{Spellcraft} (Int).}
{2}
{\PrestigeSpellTable}{
	% 1 & +0 & +0 & +0 & +2 & Tainted aura, defiler feat & +1 level of existing arcane spellcasting class\\
	% 2 & +1 & +0 & +0 & +3 &  & +1 level of existing arcane spellcasting class\\
	% 3 & +1 & +1 & +1 & +3 & Casting time metamagic 1/day & +1 level of existing arcane spellcasting class\\
	% 4 & +2 & +1 & +1 & +4 &  & +1 level of existing arcane spellcasting class\\
	% 5 & +2 & +1 & +1 & +4 & Painful radius, defiler feat & +1 level of existing arcane spellcasting class\\
	% 6 & +3 & +2 & +2 & +5 &  & +1 level of existing arcane spellcasting class\\
	% 7 & +3 & +2 & +2 & +5 & Casting time metamagic 2/day & +1 level of existing arcane spellcasting class\\
	% 8 & +4 & +2 & +2 & +6 &  & +1 level of existing arcane spellcasting class\\
	% 9 & +4 & +3 & +3 & +6 & Defiler feat & +1 level of existing arcane spellcasting class\\
	% 10 & +5 & +3 & +3 & +7 & Metamagic raze & +1 level of existing arcane spellcasting class\\

	1 & +0 & +0 & +0 & +2 & Tainted aura, defiler feat & +1 level of existing arcane spellcasting class\\
	2 & +1 & +0 & +0 & +3 & Casting time metamagic 1/day & +1 level of existing arcane spellcasting class\\
	3 & +1 & +1 & +1 & +3 & Painful radius, defiler feat & +1 level of existing arcane spellcasting class\\
	4 & +2 & +1 & +1 & +4 & Casting time metamagic 2/day & +1 level of existing arcane spellcasting class\\
	5 & +2 & +1 & +1 & +4 & Metamagic raze, defiler feat & +1 level of existing arcane spellcasting class\\
}
{
% You study how to extract every possible amount of energy from your surroundings, no matter what the cost. You understand the adverse effects of defiling and use them to the fullest.
\textbf{Weapon and Armor Proficiency:} Arch defilers gain no proficiency with any weapon or armor.

\textbf{Spellcasting:} When a new arch defiler level is gained, you gain new spells per day as if you had also gained a level in whatever arcane spellcasting class you belonged to before you added the prestige class. You do not, however, gain any other benefit a character of that class would have gained. This essentially means that you add the level of arch defiler to the level of whatever other arcane spellcasting class you have, and then determines spells per day and caster level accordingly. If you had more than one arcane spellcasting class before you became an arch defiler, you must decide to which class you add each level of arch defiler for the purpose of determining spells per day.

\textbf{Tainted Aura (Su):} You are tainted by your arcane ways in such a matter that it is noticeable. People feel uncomfortable and wary when you are present and animals whimper when you approach. You suffer a $-1$ circumstance penalty to \skill{Bluff}, \skill{Diplomacy}, \skill{Gather Information} and \skill{Handle Animal} checks for every level of arch defiler you have. Likewise, you receive a +2 circumstance bonus to \skill{Intimidate} checks for every level of arch defiler gained. The tainted aura has a range of 5 feet
per arch defiler level.

\textbf{Defiler Feat:} At 1st, 3rd and 5th levels, you gain a bonus feat from the following list: \feat{Controlled Raze}, \feat{Destructive Raze}, \feat{Distance Raze}, \feat{Efficient Raze}, \feat{Exterminating Raze}, \feat{Fast Raze}, \feat{Path Sinister}, \feat{Sickening Raze}. You must qualify for any feat requirements.

\textbf{Casting Time Metamagic:} An arch defiler of 2nd level or higher can apply a metamagic feat he knows to a spell at casting time, once per day. This does not increase the spell's level or require a higher level spell slot. Casting time metamagic doubles the casting time of the spell (a casting time of 1 action becomes 1 full round). At 4th level, you can use casting time metamagic twice per day, but only once per spell. Only metamagic feats that would increase a spell slot by 3 or less may be applied with casting time metamagic.

\textbf{Painful Radius:} The penalties suffered to attacks, saves and skill checks for being caught in your defiling radius increase by one. This effect stacks with the \feat{Agonizing Radius} feat, bringing the modifier to a total of $-3$.

\textbf{Metamagic Raze:} You can gather energy during spell preparation to improve your metamagic capacity. Spell slot level adjustments from metamagic feats are reduced by one (to a minimum of one), but only once per spell. For each use of metamagic raze, you defile a 5-foot-radius where the spell is prepared. Preparation of multiple spells increase the radius by 5 foot for each spell prepared with this ability.

% \subsection{Playing an Arch Defiler}
% Your years of reckless defiling and delving into forbidden secrets have taken a toll on your soul. Your presence has become uncomfortable for most creatures. So what if you're doomed to spend your last days on Athas as a lonely, and sometimes hunted, wanderer? The power and knowledge you seek is worth any price.

% \subsubsection{Combat}
% Spellcasting remains your greatest strength and progresses at its full rate as you advance. Thus, your place in combat is not likely to change much---if you are like most wizards, you will hang back from melee in order to blast opponents with the most potent evocation spells you have available.

% All of your abilities revolve around defiling to make your spells stronger, so make sure to have a good mix of Raze feats---always good in a straightforward fight against a physically powerful foe.

% \subsubsection{Advancement}
% Arch defilers come from a variety of backgrounds. Human wizards corrupted by their desire for power often become arch defilers. So do nomadic elven mages without concern for the environment they leave behind them in the disappearing horizon, or those who want to increase their spellcasting powers to gain status in their tribes. What they all have in common is a fascination with the interweaving of magic, life, and power. With that fascination comes a lust for more knowledge, a lust that quickly overpowers any concerns about morality.

% As an arch defiler, you spend much of your time seeking out ancient scrolls containing forbidden secrets, ruins with arcane inscriptions, undead wizards and the vanished arts they might have preserved. Between adventures, you pore over the lore you have uncovered, looking for to one day to become as powerful as a sorcerer-king, or better yet, the Dragon.

% \subsubsection{Resources}
% The one resource you covet above others is knowledge. If you have arch defilers in your acquaintance---or even other defilers or similar characters with arcane interests---the exchange of knowledge can be highly profitable for all of you, if you manage to trust such corrupt and amoral people like yourself. Besides, keeping in touch with your peers puts you in a good position to seize their knowledge should some sad fate befall them.

% \subsection{Arch Defilers on Athas}
% \Quote{I remember the first time I drew in the sweet power of magic. I felt it course through my veins as the land around me turned to ash. I felt powerful, almost complete. Now, I am powerful. Let others worry about the ash, I am as far above those concerns as I am above the common man.}{Friztroy Gelt, under interrogation by high templars of Hamanu for destruction of the western grain fields.}

% Arch defilers can be found anywhere. Some are loners, practicing their dark art in secrecy, while others seek employment and safety in organizations and groups without moral scruples. NPC arch defilers can typically be found in the ranks of merchant houses, raiding tribes, or operating on their own.
% \subsubsection{Organization}
% There is no general organization of arch defilers. A very few arch defiler cabals have been created, but they were quickly destroyed or disbanded since arch defilers, like most defilers, do not usually play well with others.
% \subsubsection{NPC Reactions}
% Arch defilers are typically greeted with nervousness, dislike, or outright disgust. Their taint might be invisible, but their corruption is sometimes abundantly clear even to casual acquaintances. As a result, most NPCs have an initial reaction of unfriendly when encountering an arch defiler, even if they can't put a finger on the reason for their dislike.

% Druids, good elemental clerics, and members of the Veiled Alliance are natural enemies of arch defilers, and will do almost anything to send them to a swift death.
}
% \subsubsection{Arch Defiler Lore}
{}
{arcana}
{The arch defiler is an arcane caster who focuses on defiling as an energy source.}
{The arch defiler embraces the taint of defiling so much his very presence becomes tainted with it.}
{Arch defilers learn defiling secrets that improves their spellcasting, weakens their foes, and grants them unearthly knowledge.}

Alternatively, similar information might be learned through bardic knowledge checks, or \skill{Gather Information} checks made with Veiled Alliance members.
\PrestigeClass{Archmage}
{What I've seen and what I've become are beyond your reach, defiler.}{Raek'n, halfling archmage}
{Archmages are advanced arcane practitioners who bend their spell power to acquire abilities other arcane spellcaster can't. They sacrifice their spell capability to apply effects to spells they cast and to gain exquisite abilities.}
{d4}
{an}
{High level wizards motivated either by lust for power or urge to protect become archmages. The arcane mastery does not distinguish preservers from defilers.}
{\textbf{Skills:} \skill{Knowledge} (arcana) 15 ranks, \skill{Spellcraft} 15 ranks.

\textbf{Feats:} \feat{Skill Focus} (Spellcraft), \feat{Spell Focus} in two schools of magic.

\textbf{Spells:} Ability to cast 7th-level arcane spells, knowledge of 5th-level or higher spells from at least five schools.
}
{\skill{Concentration} (Con), \skill{Craft} (alchemy) (Int), \skill{Knowledge} (all skills taken individually) (Int), \skill{Profession} (Wis), \skill{Search} (Int), and \skill{Spellcraft} (Int).}
{2}
{\PrestigeSpellTable}{
1 & +0 & +0 & +0 & +2 & High arcana & +1 level of existing arcane spellcasting class\\
2 & +1 & +0 & +0 & +3 & High arcana & +1 level of existing arcane spellcasting class\\
3 & +1 & +1 & +1 & +3 & High arcana & +1 level of existing arcane spellcasting class\\
4 & +2 & +1 & +1 & +4 & High arcana & +1 level of existing arcane spellcasting class\\
5 & +2 & +1 & +1 & +4 & High arcana & +1 level of existing arcane spellcasting class\\
}

\textbf{Weapon and Armor Proficiency:} Archmages gain no proficiency with any weapon or armor.

\textbf{Spells per Day/Spells Known:} When a new archmage level is gained, the character gains new spells per day (and spells known, if applicable) as if he had also gained a level in whatever arcane spellcasting class in which he could cast 7th-level spells before he added the prestige class level. He does not, however, gain any other benefit a character of that class would have gained. If a character had more than one arcane spellcasting class in which he could cast 7th-level spells before he became an archmage, he must decide to which class he adds each level of archmage for the purpose of determining spells per day.

\textbf{High Arcana:} An archmage gains the opportunity to select a special ability from among those described below by permanently eliminating one existing spell slot (she cannot eliminate a spell slot of higher level than the highest-level spell she can cast). Each special ability has a minimum required spell slot level, as specified in its description.

An archmage may choose to eliminate a spell slot of a higher level than that required to gain a type of high arcana.

\textit{Arcane Fire (Su):} The archmage gains the ability to change arcane spell energy into arcane fire, manifesting it as a bolt of raw magical energy. The bolt is a ranged touch attack with long range (400 feet + 40 feet/level of archmage) that deals 1d6 points of damage per class level of the archmage plus 1d6 points of damage per level of the spell used to create the effect. This ability costs one 9th-level spell slot.

\textit{Arcane Reach (Su):} The archmage can use spells with a range of touch on a target up to 30 feet away. The archmage must make a ranged touch attack. Arcane reach can be selected a second time as a special ability, in which case the range increases to 60 feet. This ability costs one 7th-level spell slot.

\textit{Mastery of Counterspelling:} When the archmage counterspells a spell, it is turned back upon the caster as if it were fully affected by a spell turning spell. If the spell cannot be affected by spell turning, then it is merely counterspelled. This ability costs one 7th-level spell slot.

\textit{Mastery of Elements:} The archmage can alter an arcane spell when cast so that it utilizes a different element from the one it normally uses. This ability can only alter a spell with the acid, cold, fire, electricity, or sonic descriptor. The spell's casting time is unaffected. The caster decides whether to alter the spell's energy type and chooses the new energy type when he begins casting. This ability costs one 8th-level spell slot.

\textit{Mastery of Shaping:} The archmage can alter area and effect spells that use one of the following shapes: burst, cone, cylinder, emanation, or spread. The alteration consists of creating spaces within the spell's area or effect that are not subject to the spell. The minimum dimension for these spaces is a 5-foot cube. Furthermore, any shapeable spells have a minimum dimension of 5 feet instead of 10 feet. This ability costs one 6th-level spell slot.

\textit{Spell Power:} This ability increases the archmage's effective caster level by +1 (for purposes of determining level-dependent spell variables such as damage dice or range, and caster level checks only). This ability costs one 5th-level spell slot.

\textit{Spell-Like Ability:} An archmage who selects this type of high arcana can use one of her arcane spell slots (other than a slot expended to learn this or any other type of high arcana) to permanently prepare one of her arcane spells as a spell-like ability that can be used twice per day. The archmage does not use any components when casting the spell, although a spell that costs XP to cast still does so and a spell with a costly material component instead costs her 10 times that amount in XP. This ability costs one 5th-level spell slot.

The spell-like ability normally uses a spell slot of the spell's level, although the archmage can choose to make a spell modified by a metamagic feat into a spell-like ability at the appropriate spell level.

The archmage may use an available higher-level spell slot in order to use the spell-like ability more often. Using a slot three levels higher than the chosen spell allows her to use the spell-like ability four times per day, and a slot six levels higher lets her use it six times per day.

If spell-like ability is selected more than one time as a high arcana choice, this ability can apply to the same spell chosen the first time (increasing the number of times per day it can be used) or to a different spell.

\subsection{Archmage Lore}
Characters with ranks in \skill{Knowledge} (arcana) can research archmages to learn more about them. When a character makes a skill check, read or paraphrase the following, including the information from lower DCs.

\textbf{DC 15:} Archmages are among the most powerful arcane users on the whole Athas.

\textbf{DC 20:} Archmages gave up the frequency of their spells for powerful abilities.

\textbf{DC 30:} No archmage is equal, their abilites vary from person to person. Some become masters of counterspell, others become masters of damage.

This information can also be learned through bardic knowledge checks or \skill{Gather Information} checks made with Veiled Alliance members.
\PrestigeClass{Arena Champion}
{You fought like the Dragon.}{Jarek, half-elf arena champion}
{Arena champions are gladiatorial combatants who aspire to greatness in their blood sports. They dream of performing in arenas filled with thousands of frenzied spectators. Risking their lives for fame, wealth and adoration, arena champions are the heroes to commoners of all ages. Many hail from local neighborhoods or nearby communities.}
{d12}
{an}
{Gladiators make up the majority of arena champions, since their abilities already so closely mimic those of the prestige class. Most of the rest are fighters, along with the occasional psychic warrior or barbarian with a knack for flare and extravagance.}
{
\textbf{Base Attack Bonus:} +5.

\textbf{Skills:} \skill{Perform} (any) 6 ranks.

\textbf{Feats:} \feat{Weapon Focus} (any weapon), \feat{Toughness}.}
{\skill{Balance} (Dex), \skill{Bluff} (Cha), \skill{Climb} (Str), \skill{Craft} (Int), \skill{Intimidate} (Cha), \skill{Jump} (Str), \skill{Perform} (Cha), \skill{Profession} (Wis), \skill{Sense Motive} (Wis), \skill{Tumble} (Dex).}
{4}
{\PrestigeWarriorTable}{
1 & +1 & +2 & +0 & +0 & Crowd support +1 (10)\\
2 & +2 & +3 & +0 & +0 & Reputation\\
3 & +3 & +3 & +1 & +1 & Weapon mastery\\
4 & +4 & +4 & +1 & +1 & Signature move\\
5 & +5 & +4 & +1 & +1 & Crowd support +2 (50)\\
6 & +6 & +5 & +2 & +2 & Fame\\
7 & +7 & +5 & +2 & +2 & Improved signature move\\
8 & +8 & +6 & +2 & +2 & Roar of the crowd\\
9 & +9 & +6 & +3 & +3 & Crowd support +3 (100)\\
10 & +10 & +7 & +3 & +3 & Legend, finishing move
}
\textbf{Weapon and Armor Proficiency:} Arena champions gain no proficiency with any weapon or armor.

\textbf{Gladiatorial Performance:} Your gladiator levels stack with your arena champion levels for the purpose of determining your gladiatorial performance special abilities.

\textbf{Crowd Support (Ex):} The crowd may love or loathe you, but regardless their presence motivates you, who enjoys a +1 morale bonus to attack and damage rolls whenever there are ten or more non‐combatant spectators present. At 5th and 9th level this bonus increases, but the minimum amount of spectators must be fifty and one hundred respectively for the bonuses to take effect.

\textbf{Reputation:} You enjoy respect and admiration. You receive a +1 circumstance bonus to non‐combat uses of the \skill{Bluff}, \skill{Diplomacy}, \skill{Gather Information} and \skill{Intimidate} skills. These benefits do not apply when dealing with the devoted fans of rival gladiators. If you have or select the \feat{Leadership} feat, you get a +1 bonus to your leadership score.

\textbf{Weapon Mastery:} You get a +2 bonus to damage rolls with a chosen weapon. The weapon must be one for which you already have selected the \feat{Weapon Focus} feat.

\textbf{Signature Move (Ex):} You have developed a signature move. The exact technical nature of the move is up to the individual arena champion to develop, and it grants one of the following benefits:
\begin{itemize*}
\item A +2 competence bonus on opposed rolls for disarm attempts.
\item A +2 competence bonus on opposed rolls for trip attempts.
\item A +2 competence bonus on damage rolls for sunder attempts.
\item A +2 competence bonus on opposed Bluff and Sense Motive checks in combat.
\item A +1 dodge bonus to AC when fighting defensively or using total defense.
\end{itemize*}

\textbf{Fame:} As your reputation grows, so does your ability to influence others. Your bonuses to non‐combat uses of the \skill{Bluff}, \skill{Diplomacy}, \skill{Gather Information} and \skill{Intimidate} skills increase to +2, and the bonus to your leadership score increases to +2.

\textbf{Improved Signature Move (Ex):} Through specialization the benefit associated with your chosen signature move is doubled, for example, a +2 competence bonus on opposed rolls for disarm attempts increases to +4.

Alternatively, you may select a second signature move instead to expand your repertoire.

\textbf{Roar of the Crowd (Ex):} At 8th level, you can use the cheering of the crowd to your advantage. In any round when you successfully deal damage to an opponent, you gain a +1 morale bonus on the next attack roll you make in the same round.

\textbf{Legend:} You are a legend known to all. Treat general NPC initial attitudes as one category better. This benefit does not extend to the devoted fans of rival gladiators. Your bonus to your leadership score increases to +3.

\textbf{Finishing Move (Ex):} If an attack reduces an opponent below 0 hit points, you can attempt a coup de grace as a free action.

\subsection{Arena Champion Lore}
Characters with ranks in \skill{Knowledge} (history) or \skill{Knowledge} (local) can research arena champions to learn more about them. When a character makes a skill check, read or paraphrase the following, including the information from lower DCs.

\textbf{DC 15:} Arena champions make the best fights in an arena. Always bet on them.

\textbf{DC 20:} Not every arena champion is a pompous gladiator---some of them are also raging barbarians or powerful fighters. They simply want the prestige and fame the profession brings.

\textbf{DC 30:} Characters who achieve this level of success can learn important details about specific arena champions in your campaign, including a notable individual, the area in which he operates, and the kinds of activities he undertakes.

Because arena champions have great reputations and a vast number of fans, the easiest way to find one is to contact the local arena manager or gladiatorial match agent and inquire about individuals with unique fighting talents.
\PrestigeClass{Assassin}
{Oh, don't worry. It's not the stab that will kill you. Nor will it be the poison taking effect right now. What killed you, mage, was only your Alliance. You should have chosen better friends.}{Kalar, half-elf assassin}
{Assassins are stealthy warriors specialized in dealing a single fatal attack after studying the target. The methods they take before dealing this attack vary from assassin to assassin. They may be master infiltrators, attacking the target while disguised. Many are killers for hire, and most are spies for the sorcerer-monarchs. They train in the use of poison and the forbidden arcane arts in order to finish their jobs effectively.}
{d6}
{an}
{The assassins are predominantly bards and rogues that enjoy killing, one way or another. Rangers become assassins specialized on targets of a race or killing in a terrain.}
{
\textbf{Alignment:} Any evil.

\textbf{Skills:} \skill{Disguise} 4 ranks, \skill{Hide} 8 ranks, \skill{Move Silently} 8 ranks.

\textbf{Special:} The character must kill someone for no other reason than to join the assassins.
}
{
\skill{Balance} (Dex), \skill{Bluff} (Cha), \skill{Climb} (Str), \skill{Craft} (Int), \skill{Decipher Script} (Int), \skill{Diplomacy} (Cha), \skill{Disable Device} (Int), \skill{Disguise} (Cha), \skill{Escape Artist} (Dex), \skill{Forgery} (Int), \skill{Gather Information} (Cha), \skill{Hide} (Dex), \skill{Intimidate} (Cha), \skill{Jump} (Str), \skill{Listen} (Wis), \skill{Move Silently} (Dex), \skill{Open Lock} (Dex), \skill{Search} (Int), \skill{Sense Motive} (Wis), \skill{Sleight of Hand} (Dex), \skill{Spot} (Wis), \skill{Swim} (Str), \skill{Tumble} (Dex), \skill{Use Magic Device} (Cha), and \skill{Use Rope} (Dex).
}
{4}
{\HalfSpellcasterTable[.6cm]}{
1 & +0 & +0 & +2 & +0 & Sneak attack +1d6, death attack, poison use, spells & 0 &&&\\
2 & +1 & +0 & +3 & +0 & +1 save against poison, uncanny dodge & 1 &&&\\
3 & +2 & +1 & +3 & +1 & Sneak attack +2d6 & 2 & 0 &&\\
4 & +3 & +1 & +4 & +1 & +2 save against poison & 3 & 1 &&\\
5 & +3 & +1 & +4 & +1 & Improved uncanny dodge, sneak attack +3d6 & 3 & 2 & 0 &\\
6 & +4 & +2 & +5 & +2 & +3 save against poison & 3 & 3 & 1 &\\
7 & +5 & +2 & +5 & +2 & Sneak attack +4d6 & 3 & 3 & 2 & 0\\
8 & +6 & +2 & +6 & +2 & +4 save against poison, hide in plain sight & 3 & 3 & 3 & 1\\
9 & +6 & +3 & +6 & +3 & Sneak attack +5d6 & 3 & 3 & 3 & 2\\
10 & +7 & +3 & +7 & +3 & +5 save against poison & 3 & 3 & 3 & 3	\\
}
{
\textbf{Weapon and Armor Proficiency:} Assassins are proficient with the crossbow (hand, light, or heavy), dagger (any type), dart, rapier, sap, shortbow (normal and composite), and short sword. Assassins are proficient with light armor but not with shields.

\textbf{Sneak Attack:} This is exactly like the rogue ability of the same name. The extra damage dealt increases by +1d6 every other level (1st, 3rd, 5th, 7th, and 9th). If an assassin gets a sneak attack bonus from another source the bonuses on damage stack.

\textbf{Death Attack:} If an assassin studies his victim for 3 rounds and then makes a sneak attack with a melee weapon that successfully deals damage, the sneak attack has the additional effect of possibly either paralyzing or killing the target (assassin's choice). While studying the victim, the assassin can undertake other actions so long as his attention stays focused on the target and the target does not detect the assassin or recognize the assassin as an enemy. If the victim of such an attack fails a Fortitude save (DC 10 + the assassin's class level + the assassin's Int modifier) against the kill effect, she dies. If the saving throw fails against the paralysis effect, the victim is rendered helpless and unable to act for 1d6 rounds plus 1 round per level of the assassin. If the victim's saving throw succeeds, the attack is just a normal sneak attack. Once the assassin has completed the 3 rounds of study, he must make the death attack within the next 3 rounds.

If a death attack is attempted and fails (the victim makes her save) or if the assassin does not launch the attack within 3 rounds of completing the study, 3 new rounds of study are required before he can attempt another death attack.

\textbf{Poison Use:} Assassins are trained in the use of poison and never risk accidentally poisoning themselves when applying poison to a blade.

\textbf{Spells:} Beginning at 1st level, an assassin gains the ability to cast a number of arcane spells. To cast a spell, an assassin must have an Intelligence score of at least 10 + the spell's level, so an assassin with an Intelligence of 10 or lower cannot cast these spells. Assassin bonus spells are based on Intelligence, and saving throws against these spells have a DC of 10 + spell level + the assassin's Intelligence bonus. When the assassin gets 0 spells per day of a given spell level he gains only the bonus spells he would be entitled to based on his Intelligence score for that spell level.

An assassin casts arcane spells, which are drawn from the assassin spell list (shown below). He can cast any spell he knows without preparing it ahead of time. Every assassin spell has a verbal component. An assassin can cast assassin spells while wearing light armor without incurring the normal arcane spell failure chance. However, like any other arcane spellcaster, an assassin wearing medium or heavy armor or using a shield incurs a chance of arcane spell failure if the spell in question has a somatic component (most do).

\Table{Assassin Spells Known}{b{1cm} C C C C}{
\tableheader Level & \multicolumn{4}{c}{\tableheader Spells Known}\\
\cmidrule[0.5pt]{2-5}
& \tableheader 1st & \tableheader 2nd & \tableheader 3rd & \tableheader 4th\\
1 & 2 &&&\\
2 & 3 &&&\\
3 & 3 & 2 &&\\
4 & 4 & 3 &&\\
5 & 4 & 3 & 2 &\\
6 & 4 & 4 & 3 &\\
7 & 4 & 4 & 3 & 2\\
8 & 4 & 4 & 4 & 3\\
9 & 4 & 4 & 4 & 3\\
10 & 4 & 4 & 4 & 4\\
}

An assassin can only learn spells of a certain level if he has at least one spell slot per day at that level. This means that when the number of slots available at that level is zero, he needs to have sufficient Intelligence to have a bonus spell at that level in order to know any spells of that level.

Upon reaching 6th level, at every even-numbered level after that (8th and 10th), an assassin can choose to learn a new spell in place of one he already knows. The new spell's level must be the same as that of the spell being exchanged, and it must be at least two levels lower than the highest-level assassin spell the assassin can cast. An assassin may swap only a single spell at any given level, and must choose whether or not to swap the spell at the same time that he gains new spells known for that level.

As noted above, an assassin need not prepare his spells in advance. He can cast any spell he knows at any time, assuming he has not yet used up his allotment of spells per day for the spell’s level.

\textbf{Save Bonus against Poison:} The assassin gains a natural saving throw bonus to all poisons gained at 2nd level that increases by +1 for every two additional levels the assassin gains.

\textbf{Uncanny Dodge (Ex):} Starting at 2nd level, an assassin retains his Dexterity bonus to AC (if any) regardless of being caught flat-footed or struck by an invisible attacker. (He still loses any Dexterity bonus to AC if immobilized.)

If a character gains uncanny dodge from a second class the character automatically gains improved uncanny dodge (see below).

\textbf{Improved Uncanny Dodge (Ex):} At 5th level, an assassin can no longer be flanked, since he can react to opponents on opposite sides of him as easily as he can react to a single attacker. This defense denies rogues the ability to use flank attacks to sneak attack the assassin. The exception to this defense is that a rogue at least four levels higher than the assassin can flank him (and thus sneak attack him).

If a character gains uncanny dodge (see above) from a second class the character automatically gains improved uncanny dodge, and the levels from those classes stack to determine the minimum rogue level required to flank the character.

\textbf{Hide in Plain Sight (Su):} At 8th level, an assassin can use the \skill{Hide} skill even while being observed. As long as he is within 10 feet of some sort of shadow, an assassin can hide himself from view in the open without having anything to actually hide behind. He cannot, however, hide in his own shadow.

\subsubsection{Assassin Spell List}
Assassins choose their spells from the following list:

\textbf{1st Level:} \spell{disguise self}, \spell{detect poison}, \spell{feather fall}, \spell{ghost sound}, \spell{jump}, \spell{obscuring mist}, \spell{slave scent}, \spell{sleep}, \spell{true strike}.

\textbf{2nd Level:} \spell{alter self}, \spell{cat's grace}, \spell{darkness}, \spell{death mark}, \spell{fox's cunning}, \spell{illusory script}, \spell{invisibility}, \spell{pass without trace}, \spell{spider climb}, \spell{undetectable alignment}.

\textbf{3rd Level:} \spell{boneclaw's cut}, \spell{death whip} \spell{deep slumber}, \spell{deeper darkness}, \spell{false life}, \spell{magic circle against good}, \spell{misdirection}, \spell{nondetection}.

\textbf{4th Level:} \spell{clairaudience/clairvoyance}, \spell{claws of the tembo}, \spell{dimension door}, \spell{freedom of movement}, \spell{glibness}, \spell{greater invisibility}, \spell{locate creature}, \spell{mage seeker}, \spell{modify memory}, \spell{poison}, \spell{rangeblade}, \spell{scapegoat}.
}
{}
{local}
{Assassins are spies that kill anyone for more than the right amount of money.}
{Assassins specialize in dealing a single fatal attack. Their victims are known for being poisoned.}
{Besides poison and their death attack, assassins can cast spells that enhance their abilities.}
\PrestigeClass{Executioner}
{To find Harmony, the people must understand that the King has all power and that the people have none. The spectacle of public execution teaches that the King has power over life and death. The executioner represents the King, while the perpetrator represents the people.}{excerpt from Finding Harmony in these Troubled Times, T'karei Khala, Haleban High Templar}
{While death for entertainment and public executions are far from uncommon in the Tyr region, only in Eldaarich has public execution become the predominant form of performance art. The Haleban encourages this trend, believing executions bring harmony to society by teaching proper fear of the King.

Executioners receive many privileges within Eldaarich: an honorary exception to the laws forbidding persons to wear armor and carry weapons in the city, and some are even allowed the privilege of disguising themselves as King Daskinor while carrying out an arena execution.}
{d12}
{an}
{Most executioners begin as either fighters or gladiators. Fighters are often drawn to the melee aspect of it. Most gladiators who become executioners do so because they just want the crowd around while they kill someone.}
{\textbf{Alignment:} Lawful evil.

\textbf{Skills:} \skill{Perform} (acting or oratory) 8 ranks, \skill{Heal} 4 ranks.

\textbf{Feats:} \feat{Weapon Finesse}.

\textbf{Special:} Must have composed and popularized a new style of execution; must have executed someone to entertain a crowd of at least 100 people. Must be approved by the Haleban Order.}
{
\skill{Bluff} (Cha), \skill{Craft} (Int), \skill{Diplomacy} (Cha), \skill{Disguise} (Cha), \skill{Escape Artist} (Dex), \skill{Handle Animal} (Cha), \skill{Heal} (Wis), \skill{Intimidate} (Cha), \skill{Jump} (Str), \skill{Perform} (Cha), \skill{Ride} (Dex), \skill{Sense Motive} (Wis), \skill{Sleight of Hand} (Dex), \skill{Spot} (Wis), \skill{Tumble} (Dex), and \skill{Use Rope} (Dex).
}
{4}
{\MiniWarriorTable}{
1 & +1 & +2 & +0 & +0 & Status, exact agony\\
2 & +2 & +3 & +0 & +0 & Gruesome trophy\\
3 & +3 & +3 & +1 & +1 & Crippling strike\\
4 & +4 & +4 & +1 & +1 & Die again\\
5 & +5 & +4 & +1 & +1 & Exact status, live to die another day
}

\textbf{Weapon and Armor Proficiency:} Executioners are proficient with all axes, bard’s garrote, net, and lasso.

\textbf{Status (Ex):} By making a move-equivalent action and a \skill{Spot} check (DC 15 or the creature’s \skill{Bluff} check, whichever is higher), you can discern the conditions affecting any one living creature within 10 feet of you. Executioners regard this ability as a sacred mystery, and would never share this information with any other person, including allies and superiors.

\textbf{Exact Agony (Su):} You can set a damage cap for the damage on your weapon damage rolls. Regardless of the die roll, the victim will not take more than the designated cap damage. Additionally, you can choose to deal nonlethal damage against one target rather than lethal damage, with either weapons, spells, or psionic powers, without any attack roll penalty, higher spell slot or additional power point expenditure (anyone other than the specified target that is affected by the attack takes lethal damage as normal).

\textbf{Gruesome Trophy (Sp):} Beginning at 2nd level, you learn the mysterious Eldaarish craft of shrinking heads, and you’re able to create magical shrunken heads from enemies you have personally executed, even if you’re not  a spellcaster. Additionally, while displaying the head of one of your victims, you gain a circumstance bonus on \skill{Intimidate} checks equal to you executioner level, against anyone who witnessed the execution or who knew the victim.

\textbf{Crippling Strike (Ex):} This is exactly like the rogue special ability of the same name, except no sneak attack is required and it only inflicts 1 Str damage. Hence, the executioner does not need to deny the target’s Dex bonus in order to make the crippling strike.

\textbf{Die Again (Sp):} You can animate any single deceased character within 30 feet as a swift action into a zombie. This ability otherwise works like the animate dead spell, except you can’t have more HD of undead than twice your executioner level and no material component is needed. Executioners often use this ability to give the appearance that the executed person has gotten back up and is attacking him from behind, so that the executioner can whirl around and hack the body’s head off right  before it the zombie strikes him, electrifying the crowd.

\textbf{Exact Status (Ex):} At 5th level, you can use your status ability to identify the exact hit points left and nonlethal damage of any one living creature. Executioners use this ability, in conjunction with the exact agony ability, to make it appear that they have absolute power over life and death. Obviously, executioners regard this ability as sacred as the exact agony one. Any executioner who discloses information learned through this ability loses all executioner abilities (but not weapon proficiencies). He may not progress any farther in levels as an executioner. He regains his abilities and advancement potential if he atones for his violations (see the atonement spell description) to a Haleban templar, as appropriate.

\textbf{Live to Die Another Day (Sp):} At 5th level, you can produce a \spell{raise dead} effect, as the spell, once per week. The recipient must have died within the last 5 minutes for the ability to be successful.

\subsection{Executioner Lore}
Characters with ranks in \skill{Knowledge} (local [Eldaarich]) can research executioners to learn more about them. When a character makes a skill check, read or paraphrase the following, including the information from lower DCs.

\textbf{DC 10:} Executioners are some kind of gladiator that always seem to know when his victims are about to die.

\textbf{DC 15:} This is a much-demanded occupation in Eldaarich and most dens in the city host executions.

\textbf{DC 20:} The most experienced executioners are able to bring their victims back from the Grey, only to ruthless execute them again.