\Chapter{Powers}
{}{}

\Capitalize{T}{his} chapter begins with the power lists for all manifesting classes and prestiges classes, as well as the description of psion's disciplines. An \textsuperscript{A} appearing at the end of a power's name in the power lists denotes an augmentable power. An \textsuperscript{X} denotes a power with an XP component paid by the manifester.

\textbf{Power Chains:} Some powers reference other powers that they are based upon. Only information in a power later in the power chain that is different from the base power is covered in the power being described. Header entries and other information that are the same as the base power are not repeated. The same holds true for powers that are the equivalents of spells, only the way the power varies from the spell is noted, such as power point cost.

\textbf{Order of Presentation:} In the power lists and the power descriptions that follow them, the powers are presented in alphabetical order by name---except for those belonging to certain power chains and those that are psionic equivalents of spells. When a power's name begins with ``lesser,'' ``greater,'' ``mass,'' or a similar kind of qualifier, the power description is alphabetized under the second word of the power description instead. When the effect of a power is essentially the same as that of a spell, the power's name is simply ``Psionic'' followed by the name of the spell, and it is alphabetized according to the spell name.

\textbf{Manifester Level:} A power's effect often depends on the manifester level, which is the manifester's psionic class level. A creature with no classes has a manifester level equal to its Hit Dice unless otherwise specified. The word ``level'' in the power lists always refers to manifester level.

\textbf{Creatures and Characters:} ``Creatures'' and ``characters'' are used synonymously in the power descriptions.

\textbf{Augment:} Many powers vary in strength depending on how many power points you put into them. The more power points you spend, the more powerful the manifestation. However, you can spend only a total number of points on a power equal to your manifester level, unless you have an ability that increases your effective manifester level.

Many powers can be augmented in more than one way. When the Augment section contains numbered paragraphs, you need to spend power points separately for each of the numbered options. When a paragraph in the Augment section begins with ``In addition,'' you gain the indicated benefit according to how many power points you have already decided to spend on manifesting the power.

\section{Psion/Wilder Powers}




\subsection{1st-Level Psion/Wilder Powers}

\psionicList{Astral Traveler}: Enable yourself or another to join an \psionic{astral caravan}-enabled trip.

\psionicList{Attraction}\textsuperscript{A}: Subject has an attraction you specify.

\psionicList{Aura Reading}: Reveal personal details about the target.

\psionicList{Bioflexibilty}: You gain +10 competence bonus to \skill{Escape Artist} checks.

\psionicList{Bolt}\textsuperscript{A}: You create a few enhanced short-lived bolts, arrows, or bullets.

\psionicList{Call to Mind}: Gain additional \skill{Knowledge} check with +4 competence bonus.

\psionicList{Cast Missiles}: You can launch missiles without a bow or other weapon.

\psionicList{Catfall}\textsuperscript{A}: Instantly save yourself from a fall.

\psionicList{Cause Sleep}\textsuperscript{A}: Puts 4 HD of creatures into deep slumber.

\psionicList{Conceal Thoughts}: You conceal your motives.

\psionicList{Control Flames}\textsuperscript{A}: Take control of nearby open flame.

\psionicList{Control Light}: Adjust ambient light levels.

\psionicList{Create Sound}: Create the sound you desire.

\psionicList{Cryokinesis}: You cool a creature or object.

\psionicList{Crystal Shard}\textsuperscript{A}: Ranged touch attack for 1d6 points of piercing damage.

\psionicList{Daze, Psionic}\textsuperscript{A}: Humanoid creature of 4 HD or less loses next action.

\psionicList{Deceleration}\textsuperscript{A}: Target's speed is halved.

\psionicList{Deflect Strike}: You psychokinetically deflect the next attack of a creature within range.

\noindent\textit{\hyperref[psionic:Deja Vu]{Déjà Vu}}\textsuperscript{A}: Your target repeats his last action.

\psionicList{Demoralize}\textsuperscript{A}: Enemies become shaken.

\psionicList{Detect Poison, Psionic}: Detects poison in one creature or object.

\psionicList{Detect Psionics}: You detect the presence of psionics.

\psionicList{Disable}\textsuperscript{A}: Subjects incorrectly believe they are disabled.

\psionicList{Dissipating Touch}\textsuperscript{A}: Touch deals 1d6 damage.

\psionicList{Distract}: Target gets $-4$ bonus on \skill{Listen}, \skill{Search}, \skill{Sense Motive}, and \skill{Spot} checks.

\psionicList{Ecto Protection}\textsuperscript{A}: An astral construct gains bonus against \psionic{dismiss ectoplasm}.

\psionicList{Empathy}\textsuperscript{A}: You know the subject's surface emotions.

\psionicList{Empty Mind}\textsuperscript{A}: You gain +2 on Will saves until your next action.

\psionicList{Energy Ray}\textsuperscript{A}: Deal 1d6 energy (cold, electricity, fire, or sonic) damage.

\psionicList{Entangling Ectoplasm}: You entangle a foe in sticky goo.

\psionicList{Far Hand}\textsuperscript{A}: Move small objects at a limited distance.

\psionicList{Float}: You buoy yourself in water or other liquid.

\psionicList{Force Screen}\textsuperscript{A}: Invisible disc provides +4 shield bonus to AC.

\psionicList{Ghost Writing}: Creatures writing on a distant surface or creature touched.

\psionicList{Grease, Psionic}: Makes 3-m square or one object slippery.

\psionicList{Hammer}\textsuperscript{A}: Melee touch attack deals 1d8/round.

\psionicList{Hush}\textsuperscript{A}: Subjects become utterly silent.

\psionicList{Inertial Armor}\textsuperscript{A}: Tangible field of force provides you with +4 armor bonus to AC.

\psionicList{Know Direction and Location}: You discover where you are and what direction you face.

\psionicList{Matter Agitation}: You heat a creature or object.

\psionicList{Mind Thrust}\textsuperscript{A}: Deal 1d10 damage.

\psionicList{Missive}\textsuperscript{A}: Send a one-way telepathic message to subject.

\psionicList{My Light}\textsuperscript{A}: Your eyes emit 6-m cone of light.

\psionicList{Photosynthesis}\textsuperscript{A}: Transform light into healing.
% \psionicList{Precognition, Defensive}\textsuperscript{A}: Gain +1 insight bonus to AC and saving throws.
% \psionicList{Precognition, Offensive}\textsuperscript{A}: Gain +1 insight bonus on your attack rolls.
% \psionicList{Prescience, Offensive}\textsuperscript{A}: Gain +2 insight bonus on your damage rolls.

\psionicList{Psionic Draw}: Instantly draw a weapon.

\psionicList{Psychic Tracking}: Track a creature using \skill{Psicraft}.

\psionicList{Sense Link}\textsuperscript{A}: You sense what the subject senses (single sense).

\psionicList{Skate}: Subject slides skillfully along the ground.

\psionicList{Synesthete}: You receive one kind of sense when another sense is stimulated.

\psionicList{Tattoo Animation}\textsuperscript{A}: Animates your tattoos or steals another's.

\psionicList{Telempathic Projection}: Alter the subject's mood.

\psionicList{Trail of Destruction}: Detects recent defiling.

\psionicList{Vigor}\textsuperscript{A}: Gain 5 temporary hit points.

\psionicList{Wild Leap}\textsuperscript{A}: Make an additional leap and gain a bonus to \skill{Jump} checks.




\subsection{2nd-Level Psion/Wilder Powers}

\psionicList{Alter Self, Psionic}\textsuperscript{A}: Assume form of a similar creature.

\psionicList{Bestow Power}\textsuperscript{A}: Subject receives 2 power points.

\psionicList{Biofeedback}\textsuperscript{A}: Gain damage reduction 2/--.

\psionicList{Body Equilibrium}: You can walk on nonsolid surfaces.

\psionicList{Calm Emotions, Psionic}: Calms creatures, negating emotion effects.

\psionicList{Cloud Mind}: You erase knowledge of your presence from target's mind.

\psionicList{Concealing Amorpha}: Quasi-real membrane grants you concealment.

\psionicList{Concentrate Water}: Collects water from surrounding area.

\psionicList{Concussion Blast}\textsuperscript{A}: Deal 1d6 force damage to target.

\psionicList{Control Sound}: Create very specific sounds.

\psionicList{Detect Hostile Intent}: You can detect hostile creatures within 9 m of you.

\psionicList{Detect Life}: Reveals living creatures.

\psionicList{Ego Whip}\textsuperscript{A}: Deal 1d4 Cha damage and daze for 1 round.

\psionicList{Elfsight}: Gain low-light vision, +2 bonus on Search and Spot checks, and notice secret doors.

\psionicList{Energy Adaptation, Specified}\textsuperscript{A}: Gain resistance 10 against one energy type.

\psionicList{Energy Push}\textsuperscript{A}: Deal 2d6 damage and knock subject back.

\psionicList{Energy Stun}\textsuperscript{A}: Deal 1d6 damage and stun target if it fails both saves.

\psionicList{Feat Leech}\textsuperscript{A}: Borrow another's psionic or metapsionic feats.

\psionicList{Id Insinuation}\textsuperscript{A}: Swift tendrils of thought disrupt and confuse your target.

\psionicList{Identify, Psionic}: Learn the properties of a psionic item.

\psionicList{Inflict Pain}\textsuperscript{A}: Telepathic stab gives your foe $-4$ on attack rolls, or $-2$ if he makes the save.

\psionicList{Knock, Psionic}: Opens locked or psionically sealed door.

\psionicList{Levitate, Psionic}: Subject moves up and down at your direction.

\psionicList{Mental Disruption}\textsuperscript{A}: Daze creatures within 3 meters for 1 round.

\psionicList{Missive, Mass}\textsuperscript{A}: You send a one-way telepathic message to an area.

\psionicList{Molecular Bonding}: Temporarily glue two surfaces together.

\psionicList{Pheromone Discharge}: Vermin react well to you.

\psionicList{Psionic Lock}: Secure a door, chest, or portal.

\psionicList{Recall Agony}\textsuperscript{A}: Foe takes 2d6 damage.

\psionicList{Return Missile}\textsuperscript{A}: Make one weapon return to you after thrown.

\psionicList{Sense Link, Forced}: Sense what subject senses.

\psionicList{Sensory Suppression}: Victim loses one sense---sight, hearing, smell.

\psionicList{Sever the Tie}\textsuperscript{A}: Disrupt an undead's tie to the Gray, damaging or destroying it.

\psionicList{Share Pain}: Willing subject takes some of your damage.

\psionicList{Sustenance}: Go without food and water for one day.

\psionicList{Swarm of Crystals}\textsuperscript{A}: Crystal shards are sprayed forth doing 3d4 slashing damage.

\psionicList{Thought Shield}\textsuperscript{A}: Gain PR 13 against mind-affecting powers.

\psionicList{Tongues, Psionic}: You can communicate with intelligent creatures.

\psionicList{Watcher Ward}\textsuperscript{A}: You are aware of creature within the warded area.

\psionicList{Weather Prediction}: Predicts weather for next 24 hours.




\subsection{3rd-Level Psion/Wilder Powers}

\psionicList{Antidote Simulation}\textsuperscript{A}: Detoxifies venom in your system.

\psionicList{Beacon}\textsuperscript{A}: Creates a ball on light that can become much larger with concentration.

\psionicList{Blink, Psionic}\textsuperscript{A}: You randomly vanish and reappear for 1 round/level.

\psionicList{Body Adjustment}\textsuperscript{A}: You heal 1d12 damage.

\psionicList{Body Purification}\textsuperscript{A}: You restore 2 points of ability damage.

\psionicList{Danger Sense}\textsuperscript{A}: You gain +4 bonus against traps.

\psionicList{Darkvision, Psionic}: See 18 m in total darkness.

\psionicList{Dismiss Ectoplasm}: Dissipates ectoplasmic targets and effects.

\psionicList{Dispel Psionics}\textsuperscript{A}: Cancels psionic powers and effects.

\psionicList{Energy Bolt}\textsuperscript{A}: Deal 5d6 energy damage in 36-m line.

\psionicList{Energy Burst}\textsuperscript{A}: Deal 5d6 energy damage in 12-m burst.

\psionicList{Energy Retort}\textsuperscript{A}: Ectoburst of energy automatically targets your attacker for 4d6 damage once each round.

\psionicList{Energy Wall}: Create wall of your chosen energy type.

\psionicList{Eradicate Invisibility}\textsuperscript{A}: Negate invisibility in 15-m burst.

\psionicList{Keen Edge, Psionic}: Doubles normal weapon's threat range.

\psionicList{Lighten Load, Psionic}: Increases Strength for carrying capacity only.

\psionicList{Mass Manipulation}\textsuperscript{A}: Alter the weight of a creature or object.

\psionicList{Mental Barrier}\textsuperscript{A}: Gain +4 deflection bonus to AC until your next action.

\psionicList{Mind Trap}\textsuperscript{A}: Drain 1d6 power points from anyone who attacks you with a telepathy power.

\psionicList{Nerve Manipulation}\textsuperscript{A}: Disrupts a creature nervous system.

\psionicList{Psionic Blast}\textsuperscript{A}: Stun creatures in 9-m cone for 1 round.

\psionicList{Psionic Sight}\textsuperscript{A}: Psionic auras become visible to you.

\psionicList{Share Pain, Forced}\textsuperscript{A}: Unwilling subject takes some of your damage.

\psionicList{Solicit Psicrystal}\textsuperscript{A}: Your psicrystal takes over your concentration power.

\psionicList{Telekinetic Force}\textsuperscript{A}: Move an object with the sustained force of your mind.

\psionicList{Telekinetic Thrust}\textsuperscript{A}: Hurl objects with the force of your mind.

\psionicList{Time Hop}\textsuperscript{A}: Subject hops forward in time 1 round/level.

\psionicList{Touchsight}\textsuperscript{A}: Your telekinetic field tells you where everything is.

\psionicList{Ubiquitous Vision}: You have all-around vision.




\subsection{4th-Level Psion/Wilder Powers}

\psionicList{Aura Sight}\textsuperscript{A}: Reveals creatures, objects, powers, or spells of selected alignment axis.

\psionicList{Correspond}: Hold mental conversation with another creature at any distance.

\psionicList{Death Urge}\textsuperscript{A}: Implant a self-destructive compulsion.

\psionicList{Detect Remote Viewing}: You know when others spy on you remotely.

\psionicList{Detonate}\textsuperscript{A}: Explode one object.

\psionicList{Dimension Door, Psionic}: Teleports you short distance.

\psionicList{Divination, Psionic}: Provides useful advice for specific proposed action.

\psionicList{Empathic Feedback}\textsuperscript{A}: When you are hit in melee, your attacker takes damage.

\psionicList{Energy Adaptation}\textsuperscript{A}: Your body converts energy to harmless light.

\psionicList{Freedom of Movement, Psionic}: You cannot be held or otherwise rendered immobile.

\psionicList{Intellect Fortress}\textsuperscript{A}: Those inside fortress take only half damage from all powers and psi-like abilities until your next action.

\psionicList{Magnetize}\textsuperscript{A}: Make metallic object magnetic.

\psionicList{Mindwipe}\textsuperscript{A}: Subject's recent experiences wiped away, bestowing negative levels.

\psionicList{Personality Parasite}: Subject's mind calves self-antagonistic splinter personality for 1 round/level.

\psionicList{Power Leech}: Drain 1d6 power points/round while you maintain concentration; you gain 1/round.

\psionicList{Psychic Reformation}\textsuperscript{X}: Subject can choose skills, feats, and powers anew for previous levels.

\psionicList{Repugnance}: Make a creature repugnant to others.

\psionicList{Shadow Jump}\textsuperscript{A}: Jump into shadow to travel rapidly.

\psionicList{Telekinetic Maneuver}\textsuperscript{A}: Telekinetically bull rush, disarm, grapple, or trip your target.

\psionicList{Trace Teleport}\textsuperscript{A}: Learn destination of subject's teleport.

\psionicList{Wall of Ectoplasm}: You create a protective barrier.




\subsection{5th-Level Psion/Wilder Powers}

\psionicList{Adapt Body}: Your body automatically adapts to hostile environments.

\psionicList{Catapsi}\textsuperscript{A}: Psychic static inhibits power manifestation.

\psionicList{Ectoplasmic Shambler}: Foglike predator deals 1 point of damage/two levels each round to an area.

\psionicList{Electroerosion}\textsuperscript{A}: Create a ray that erodes iron and alloys.

\psionicList{Incarnate}\textsuperscript{X}: Make some powers permanent.

\psionicList{Leech Field}\textsuperscript{A}: Leech power points each time you make a saving throw.

\psionicList{Major Creation, Psionic}: As \psionic{psionic minor creation}, plus stone and metal.

\psionicList{Plane Shift, Psionic}: Travel to other planes.

\psionicList{Power Resistance}: Grant PR equal to 12 + level.

\psionicList{Psychic Crush}\textsuperscript{A}: Brutally crush subject's mental essence, reducing subject to $-1$ hit points.

\psionicList{Shatter Mind Blank}: Cancels target's mind blank effect.

\psionicList{Tower of Iron Will}\textsuperscript{A}: Grant PR 19 against mind-affecting powers to all creatures within 3 m until your next turn.

\psionicList{True Seeing, Psionic}: See all things as they really are.




\subsection{6th-Level Psion/Wilder Powers}

\psionicList{Aura Alteration}\textsuperscript{A}: Repairs psyche or makes subject seem to be something it is not.

\psionicList{Breath of the Black Dragon}\textsuperscript{A}: Breathe acid for 11d6 damage.

\psionicList{Cloud Mind, Mass}: Erase knowledge of your presence from the minds of one creature/level.

\psionicList{Co-opt Concentration}: Take control of foe's concentration power.

\psionicList{Contingency, Psionic}\textsuperscript{X}: Sets trigger condition for another power.

\psionicList{Dimensional Screen}: Create a shimmering screen that diverts attacks.

\psionicList{Disintegrate, Psionic}\textsuperscript{A}: Turn one creature or object to dust.

\psionicList{Fuse Flesh}\textsuperscript{A}: Fuse subject's flesh, creating a helpless mass.

\psionicList{Overland Flight, Psionic}: You fly at a speed of 12 m and can hustle over long distances.

\psionicList{Remote View Trap}: Deal 8d6 points electricity damage to those who seek to view you at a distance.

\psionicList{Retrieve}\textsuperscript{A}: Teleport to your hand an item you can see.

\psionicList{Suspend Life}: Put yourself in a state akin to suspended animation.

\psionicList{Temporal Acceleration}\textsuperscript{A}: Your time frame accelerates for 1 round.




\subsection{7th-Level Psion/Wilder Powers}

\psionicList{Decerebrate}: Remove portion of subject's brain stem.

\psionicList{Divert Teleport}: Choose destination for another's teleport.

\psionicList{Energy Conversion}: Offensively channel energy you've absorbed.

\psionicList{Energy Wave}\textsuperscript{A}: Deal 13d4 damage of your chosen energy type in 36-m cone.

\psionicList{Evade Burst}\textsuperscript{A}: You take no damage from a burst on a successful Reflex save.

\psionicList{Incorporeality}\textsuperscript{A}: You become incorporeal for 1 round/level.

\psionicList{Insanity}\textsuperscript{A}: Subject is permanently confused.

\psionicList{Mind Blank, Personal}: You are immune to \spell{scrying} and mental effects.

\psionicList{Mindflame}: Kills, paralyzes, weakens, or dazes subjects.

\psionicList{Moment of Prescience, Psionic}: You gain insight bonus on single attack roll, check, or save.

\psionicList{Oak Body}\textsuperscript{A}: Your body becomes as hard as oak.

\psionicList{Phase Door, Psionic}: Invisible passage through wood or stone.

\psionicList{Sequester, Psionic}\textsuperscript{X}: Subject invisible to sight and \psionic{remote viewing}; renders subject comatose.

\psionicList{Ultrablast}\textsuperscript{A}: Deal 13d6 damage in 4.5-m radius.




\subsection{8th-Level Psion/Wilder Powers}

\psionicList{Bend Reality}\textsuperscript{X}: Alters reality within power limits.

\psionicList{Iron Body, Psionic}: Your body becomes living iron.

\psionicList{Matter Manipulation}\textsuperscript{X}: Increase or decrease an object's base hardness by 5.

\psionicList{Mind Blank, Psionic}: Subject immune to mental/emotional effects, \spell{scrying}, and \psionic{remote viewing}.

\psionicList{Recall Death}: Subject dies or takes 5d6 damage.

\psionicList{Shadow Body}: You become a living shadow (not the creature).

\psionicList{Teleport, Psionic Greater}: As \psionic{psionic teleport}, but no range limit and no off-target arrival.

\psionicList{True Metabolism}: You regenerate 10 hit points/round.




\subsection{9th-Level Psion/Wilder Powers}

\psionicList{Affinity Field}: Effects that affect you also affect others.

\psionicList{Apopsi}\textsuperscript{X}: You delete target's psionic powers.

\psionicList{Assimilate}: Incorporate creature into your own body.

\psionicList{Etherealness, Psionic}: Become ethereal for 1 min./level.

\psionicList{Microcosm}\textsuperscript{A}: Creature or creature lives forevermore in world of his own imagination.

\psionicList{Reality Revision}\textsuperscript{X}: As \psionic{bend reality}, but fewer limits.

\psionicList{Timeless Body}: Ignore all harmful, and helpful, effects for 1 round.